
\documentclass[twocolumn]{ctexart}
\usepackage{ctex}
\usepackage{graphicx}
\usepackage{titlesec}
\usepackage[hyphens]{url}
\usepackage[colorlinks=true, urlcolor=blue, linkcolor=black]{hyperref}
\usepackage{geometry}
\usepackage{cancel}
\setmainfont{Times New Roman}
\setCJKmainfont{SimSun}
\newCJKfontfamily\Kai{STKaiti}
\newCJKfontfamily\Hei{SimHei} 
\setcounter{secnumdepth}{0}
\setcounter{tocdepth}{1}
\titleformat*{\section}{\centering\Large\bfseries }
\titleformat*{\subsection}{\centering}
\titlespacing*{\subsection} {0pt}{0pt}{10ex}
\geometry{a4paper, scale=0.85}
\begin{document}
\pagestyle{plain}

\section{【空军】共军小道消息增补:Su-35}
\subsection{2014-08-15 01:45}


\section{1条问答}

\textit{\hfill\noindent\small 2015/11/22 00:00 提问; 回答}

\noindent[1.]{\Hei 答}:臺湾不衹不可能再有荣景,未来(如果没有统一)将会以慢于日本5-10年的间隔,尾随它进入不可避免的衰退与崩溃。

从个人的观点看,如果你还年轻,到大陆发展是选项之一。

这个问题比较适合《经济的最后支柱》或是《世界经济未来走向》。\\


\section{【美国】【戦略】美国的欧洲代理人板块重整}
\subsection{2014-09-19 22:14}


\section{3条问答}

\textit{\hfill\noindent\small 2020/04/17 10:59 提问;2020/04/18 05:35 回答}

\noindent[1.]{\Hei 答}:
這件事,我們在去年諾貝爾獎發佈的時候討論過了:他們的扶貧研究雖然粗淺,但是一方面這是諾貝爾獎協會對著正確的方向走出一小步;另一方面,和以往的經濟學獎得主相比,至少他們不是為財閥作倀,製造出十萬億元級別的損失和浪費。
\\

\textit{\hfill\noindent\small 2020/04/19 15:29 提问;2020/04/20 06:17 回答}

\noindent[2.]{\Hei 答}:解決帝國内部腐化,關鍵在於避免既得利益者不斷擴大自己的尋租能力,侵占底層民衆的基本人權和生活品質,但既得利益者當然也深度參與執政,所以這方面的改革在古今中外都是政治上的頭號難題。我說貧富不均是21世紀人類的最大挑戰,只不過是上述問題在經濟全球化之後的體現。
\\

\textit{\hfill\noindent\small 2022/02/27 20:56 提问;2022/02/27 23:28 回答}

\noindent[3.]{\Hei 答}:
幾年過去了,中國的實力更加强大、鬥爭意志更加堅實,而美國陣營越加虛弱、對第三世界影響力越來越低。Merkel的退休,當然是一大損失,但不足以扭轉最終結局。
我從沒説過“不能貿然加息”,我説的是加不加息一樣都會有危機。歐元區成立之後,就把美元的傳統壓力隔絕在外;2008年是德國中小銀行主動把資金投入美國資本市場,這次沒聽説他們敢重蹈覆轍。
\\


\section{【金融】【戰略】美元的金融霸權(一)}
\subsection{2014-10-02 21:29}


\section{11条问答}

\textit{\hfill\noindent\small 2015/11/19 00:00 提问; 回答}

\noindent[1.]{\Hei 答}:我觉得省略这些细节不妨碍理解歷史主轴。\\

\textit{\hfill\noindent\small 2015/12/18 00:00 提问; 回答}

\noindent[2.]{\Hei 答}:英文的Ounce含义极多,既是容量单位,也是重量单位,而且不论是容量还是重量,各又有好几种不同的定义。用在贵金属上,一般指的是Troy Ounce,相当于31.1035公克。\\

\textit{\hfill\noindent\small 2021/05/21 06:28 提问;2021/05/26 09:52 回答}

\noindent[3.]{\Hei 答}:
是的,的確是像玩Texas Holdem一樣,你的技術再好,如果對方的資金比你高出無限倍,每一手都可以All In,你要獲勝的牌運很快就隨指數而成爲無限小的機率。
好在美元在全球貨幣的市占率只有60\%,所以其他國家若能敵愾同仇,本錢上的對比是3比2,而不是∞。我所一再强調的策略,是提醒其他國家,提前避免最危險離譜的美元資產,不要重蹈2008年的覆轍,那麽美元收放循環的力量就無法完全轉嫁國外,當一個全力運行的熱機,功率輸出跳脫,那麽被釋放的能量自然會把熱機自身扯成碎片。
\\

\textit{\hfill\noindent\small 2021/10/21 15:33 提问;2021/10/21 23:35 回答}

\noindent[4.]{\Hei 答}:啊,既然你給了鏈接,我乾脆剪貼過來算了:

美聯儲的任務之一,是安撫市場情緒,所以撒謊或者答非所問是必要也常用的手段。例如文中被問到量化寬鬆,他卻大談國際溝通。被問到新的國際數字化貨幣,他就强調美元的優越性和美聯儲的必要性。

當然也有他不小心透露口風的時候,例如最後兩段明顯地是在討論過去三個月的Repo市場:Dodd-Frank這種限制金融業胡作非爲的法案,被他説成”使(提供穩定性和流動性)變得更加困難“,指的是Excess Reserve裏有4000-6000億美元其實是Semi-required,不完全能由銀行界自由花用,但其實流動性短缺的真正問題在於過去三年,美國金融界損失了15000億的現金給聯邦赤字和美聯儲,他卻不敢提,只在最後一段說以往的緊急程序,現在已經變成“常規程序”了。這真的是正面的消息嗎?
\\

\textit{\hfill\noindent\small 2021/12/08 02:59 提问;2021/12/08 12:17 回答}

\noindent[5.]{\Hei 答}:當時的先進工業國僅限於歐美日,而且對化石燃料有近乎絕對的依賴,所以美國簡單把美元和石油綁定,歐日只能被迫跟著印錢,把通脹全球化;這裏的重點在於整個過程中,美元的國際地位從未受到真正威脅,因爲根本沒有接近合格的替代選項。20世紀末,歐洲還有些具備戰略眼光的政治家,所以在事後(尤其是經過Plaza Accord的又一次打擊之後)推行了歐元,其用意就在於避免反復受美元的搜刮。

很不幸的,到了21世紀,歐洲管理階層的素質逐步退化,坐視歐系銀行、產業和歐元受美國的多方暗算而毫無所覺。現在美國又以無限QE來複製50年前的通脹搜刮,眼看著歐洲又要躺平了。中國的正確反應,在於聯合俄國、亞非拉和甚至中東,拒絕被動承受美國引發的通脹壓力,順勢挖美元霸權的墻脚。這裏的第一步,是要求(除了對美的)主要進出口產品立刻改用其他貨幣,尤其是人民幣;對這一個基本手段的任何猶豫,都是鼠目寸光、因小失大的非理性思維,也是中國社科學術智庫界,對國家人民的又一次辜負和危害。
\\

\textit{\hfill\noindent\small 2021/12/08 13:13 提问;2021/12/09 00:35 回答}

\noindent[6.]{\Hei 答}:志氣消磨、眼光短淺、思想奴役、自我催眠,和清華、台灣一樣,有什麽奇怪的?
\\

\textit{\hfill\noindent\small 2021/12/09 02:18 提问;2021/12/09 13:14 回答}

\noindent[7.]{\Hei 答}:作用是有的,但正因爲它是一個徹底的改變,又沒有緊急的外加動力,所以必然曠日費時,趕不上中美博弈的需要。這和電動車不一樣:後者有氣候變化引發的全球政治推力為後盾,因而進展一日千里,成爲新興工業國彎道超車的契機。
\\

\textit{\hfill\noindent\small 2021/12/12 03:15 提问;2021/12/13 05:49 回答}

\noindent[8.]{\Hei 答}:因爲金融是我的專業本行之一,從博客一開始寫了《美元的金融霸權》那一系列文章,早已解釋過針對美元來做反擊的重要性。不過我一直沒有對人民銀行做尖銳的批評,是因爲他們管理貨幣政策,並沒有犯過證監委在2015年股災期間那樣的明顯戰術錯誤,所以在戰略上我也給他們Benefit of doubt。

幾年下來,世界經歷了中美貿易戰、新冠疫情、美聯儲超發,然後通脹壓力浮現表面,再怎麽仁慈的旁觀者都沒有藉口繼續假設人民銀行有什麽隱性的正當理由不對美元下手,所以我才終於開口批評,而且談的只限貿易替代。貿易替代是最最基本的應有作爲,我在過去介紹俄方政策的時候已經反復論證過其壓倒性的正面效益,就不再贅述。不過這些利害考慮,有許多是超乎人民銀行日常職務視野的中美博弈戰略問題,原本就應該由負責戰略分析的智庫來做,所以我並不是把責任完全歸罪於金融管理部門。
\\

\textit{\hfill\noindent\small 2022/11/06 11:40 提问;2022/11/07 08:47 回答}

\noindent[9.]{\Hei 答}:
這篇很老的基礎文章以及最近這一年博客討論後美國時代新國際格局,都反復解釋過IMF在昂撒資本對外剝削過程中所占據的重要角色。匯率管控作爲資金管制的一部分,是金融弱勢國家對外的主要屏障,而强制拆除城墻自古(在公元前第三千禧就記載於Mesopotamia正史)以來是征服者的日常運作。
這裏順道澄清一下,真正在幕後主導昂撒體系的權貴財閥,並不須要事事躬親、直接參與執行細節的籌劃;對像是埃及這樣相對不重要的國家,更可能是接受自動駕駛的設定,放任少數主要高級主管帶領被百年來學術和媒體徹底洗腦的衆多學者和官僚執行既定的政策。
\\

\textit{\hfill\noindent\small 2023/03/31 16:23 提问;2023/04/01 03:48 回答}

\noindent[10.]{\Hei 答}:這件事博客多年來反復解釋過了:貨幣有多重意義和任務,美元作爲“國際儲備貨幣”的紅利,直接來自儲備(就在名字裏,稍微用心便能記得,畢竟沒有人說“國際定價貨幣”或“國際支付貨幣”,不是嗎?),定價只是間接鞏固儲備,貿易支付則是更間接地影響定價。當前美元和人民幣的儲備份額比是60\%:3\%,貿易支付的替換當然緩不濟急;不過以人民銀行主管的尿性,以上的道理他們不是不懂就是不在乎,反正在貿易方面改用人民幣簡單至極(博客自2014年起,就一再説過,這是最最基本的作爲,不知爲什麽一直沒有動手),特別適合用來搪塞悠悠之口,至於實際上爭取國家利益的作用是否足夠,從來不在他們的考慮之列。
\\

\textit{\hfill\noindent\small 2023/08/25 15:03 提问;2023/08/26 03:15 回答}

\noindent[11.]{\Hei 答}:
一年半前我寫正文的時候,也知道那些60後、70後的官員不可靠,可以簡單忽略聲量微弱的正確建議,但因爲此事有Nabiullina參與,總覺得以她的學識聲望和身份地位,這些人必須認真考慮最優方案,那麽就有較大的機率去采納。結果他們失去“沒想到、沒預見”的藉口之後,依舊選擇坑害國家和人類,我也極度失望,但除了繼續努力建言之外,還能做什麽呢?
\\


\section{【金融】【战略】美元的金融霸权(二)}
\subsection{2014-10-04 06:22}


\section{3条问答}

\textit{\hfill\noindent\small 2015/09/07 00:00 提问; 回答}

\noindent[1.]{\Hei 答}:你不必为了有新留言而留言;不须回答留言时,我比较有时间读书。

美元的失势会是一个缓慢而漫长的过程,期间会有几百个小步骤。\\

\textit{\hfill\noindent\small 2015/11/21 00:00 提问; 回答}

\noindent[2.]{\Hei 答}:套息交易的对衝姿态(Hedged Position)建立之后就成了摇钱树,每天都会有利息从天上掉下来;但是因为数额很大,进出过程都会有一些费用,尤其是做多小货币的那一半,所以最理想的是衹把做空美元(Short Dollar)的那一半转移到欧元去。美元和欧元都是国际货币,流通性极高,交易费用很低,几乎可以忽略。\\

\textit{\hfill\noindent\small 2020/09/06 23:34 提问;2020/09/07 04:24 回答}

\noindent[3.]{\Hei 答}:有關美元金融霸權對外收割的模式,博客已經反復説得很清楚,這五年多來網絡上讀了這些文章後自己重新做總結的,不可計數,連完全沒有金融背景的YST都來摻過一脚(中文網絡世界不列出處的習慣真不是好事)。你如果想印証自己的心得,搜索一下應該很容易找到很多其他人的結論。

像是古巴、伊朗和委内瑞拉,明明是被美國强力打壓而經濟萎靡,事後還要被美國學術宣傳體系拿來做爲自由經濟和直選體制優越性的負面證據,美國吃人透透的本領真是讓人嘆爲觀止。
\\


\section{【金融】【战略】美元的金融霸权(三)}
\subsection{2014-10-05 05:38}


\section{4条问答}

\textit{\hfill\noindent\small 2015/11/21 00:00 提问; 回答}

\noindent[1.]{\Hei 答}:我觉得还是用了太多美式经济学的思维模式,批判的力量不太够。\\

\textit{\hfill\noindent\small 2015/11/21 00:00 提问; 回答}

\noindent[2.]{\Hei 答}:有良知的大概有三四个,不是三四成。

页岩油都是中小企业,没有政治力量。\\

\textit{\hfill\noindent\small 2015/11/24 00:00 提问; 回答}

\noindent[3.]{\Hei 答}:但是它在政治影响力上仍然衹是个不入流的侏儒。收买政客不衹要花钱,也需要时间、精力和关系的。\\

\textit{\hfill\noindent\small 2017/09/28 00:00 提问; 回答}

\noindent[4.]{\Hei 答}:我没有说西方的经济学都是胡扯,只有美国因为Rockefeller有意创建芝加哥大学来为财阀洗白,所以才成为重灾区。写《21世纪资本论》的Piketty不就是法国人。马克思是德国人。Keynes是英国人。

即使在美国,也只有\&rdquo;淡水经济学\&ldquo;才是保证的胡扯,如果要申请经济系博士班,Columbia就是个不错的选择。

但是真正的人才,会有能力出污泥而不染,例如林毅夫就是芝加哥大学毕业的,回国后发现所学的和现实不一样,他选择现实。\\


\section{【金融】【战略】美元的金融霸权(四)}
\subsection{2014-10-05 18:37}


\section{1条问答}

\textit{\hfill\noindent\small 2015/10/07 00:00 提问; 回答}

\noindent[1.]{\Hei 答}:哪里。欢迎你加入我们的讨论。\\


\section{【金融】【战略】美元的金融霸权(五)}
\subsection{2014-10-05 20:03}


\section{23条问答}

\textit{\hfill\noindent\small 2015/12/01 00:00 提问; 回答}

\noindent[1.]{\Hei 答}:是的,只希望下一任韩国总统不要反復。

我的确不喜欢重复解説同一件事,而且大部分的文章讨论得都很完整,所以有新的新闻进展也不见得必须新写评论。希望大家能细读并且完全吸收。\\

\textit{\hfill\noindent\small 2015/12/17 00:00 提问; 回答}

\noindent[2.]{\Hei 答}:我也回去读了一遍,自己也很吃惊,居然没有什么好补充的,该写的都在那篇和后续的其他文章里谈过了。可能是我运气好,也可能是我这一年多来没什么长进。\\

\textit{\hfill\noindent\small 2015/12/19 00:00 提问; 回答}

\noindent[3.]{\Hei 答}:我也注意到了,在《市场经济地位》加了后注。

美国人被逼接受,是因为IMF的欧洲力量威胁着要绕过正常管道做特别增资,那么不但中国额分增加更多,美国额分也可能降低于15\%的否决权标准。

欧洲力量肯和美国决裂摊牌,虽然是被亚投行吓的,仍然是很好的前例。\\

\textit{\hfill\noindent\small 2017/05/12 00:00 提问; 回答}

\noindent[4.]{\Hei 答}:1.稀土和石油没有可比性。稀土的需要量很小,只用在少数高科技產品,而且很容易一次囤积好几年的消费量。石油不但是现代交通的动力,也是绝大多数化工產品的原材料,每个国家都有极大的需求,即使是欧美也只有三个月份的战略储备,中国只有一个月的。

2.人民币在2016年,基本上对世界整体是纹风不动(对美元贬值、对欧元升值),所以对进出口没有影响。在短期波动上,则是很灵活地杀伤炒手。总体而言,可圈可点。

3.我对香港事务不熟。不过英国应该不敢再在太岁爷头上动土了。\\

\textit{\hfill\noindent\small 2017/05/18 00:00 提问; 回答}

\noindent[5.]{\Hei 答}:一带一路是大战略,不是纯为赚钱的。一些建设公司会获利,但是被市场一炒作,只怕已经过头了。\\

\textit{\hfill\noindent\small 2017/05/25 00:00 提问; 回答}

\noindent[6.]{\Hei 答}:阿根廷手上的那一点儿人民币,还不够震撼全球货币市场。

原油期货是争取定价权的一步,但只是一小步,期望沙乌地一夕之间倒戈是不切实际的。\\

\textit{\hfill\noindent\small 2017/09/28 00:00 提问; 回答}

\noindent[7.]{\Hei 答}:很抱歉\&middot;,昨天电脑没有设定好,一直断音。

现在已经解决了。\\

\textit{\hfill\noindent\small 2017/09/28 00:00 提问; 回答}

\noindent[8.]{\Hei 答}:我在节目最后段讨论了这个主题。简单来说,中国要以人民币取代美元,只能先把各项准备工作做好,临门一击还是得靠美国自己犯蠢。

美国的GDP在1890年代左右就超过大英帝国,但是美元取代英镑还是又等了50年,期间英国严重犯蠢了不是一次(WWI),而是两次(WWII)才成功。\\

\textit{\hfill\noindent\small 2020/06/22 07:17 提问;2020/06/23 03:43 回答}

\noindent[9.]{\Hei 答}:中國對外貿易額佔全球的11\%,雖然不算是主導性的優勢,但在取代美元的過程中不無小補。我一直奇怪爲什麽中方沒有更積極地去推行人民幣結算,不過中國的金融智囊向來比做外交的高明,或許不是不想,而是實際執行上有困難。
\\

\textit{\hfill\noindent\small 2020/06/23 10:32 提问;2020/06/23 14:34 回答}

\noindent[10.]{\Hei 答}:1)目前中國對外的FDI有兩大類,第一是為技術,第二是為存錢/洗錢;但是其實FDI也可以用在購買一般的資源、產能、品牌和銷售管道,就近服務當地市場。如果有計劃、有策略地做投資,並不須要產生强大的去工業化壓力。

2)成爲國際儲備貨幣,其實會大幅減低匯率波動性,對實體貿易是有幫助的。真正的問題在於持續的升值趨勢,這的確會對國内的低階產業有殺傷力。

3)人民幣要成爲儲備貨幣,必須先自由兌換,這也的確是很危險的。國内的金融管理人員雖然聰明,但畢竟沒有與華爾街對戰的經驗,只怕無法或無力接下所有的陰招。中國國内自己的炒家,只怕也能繞著監管單位團團轉(Run circles around the regulators)。

4)使用慣性只代表推倒美元不容易,和人民幣適不適合國際化沒有因果關係。在大戰略上,中國沒有選擇,必須與美國做生死鬥爭,而在宣傳上打不過、全面戰爭代價太高,只能從美元著手。換句話説,不論表面上再怎麽穩固,美元依舊是美國霸權的命門和軟肋。

整體來説,擔心人民幣國際化的部分論點在短期内是切題合理的,這也是爲什麽我一直提倡先捧歐元、然後用上國際合成貨幣(例如SDR或金磚加密貨幣)的考慮。當前的戰略目標是打倒美元,以終止美國對外吸血,至於取代它的,並不須要是人民幣;等到中國在國際上全面領先,許多問題能得到緩解,再重新檢討人民幣的定位也不遲。
\\

\textit{\hfill\noindent\small 2020/06/23 12:52 提问;2020/06/23 16:32 回答}

\noindent[11.]{\Hei 答}:你是說個人嗎?美元在未來5-10年可能大幅貶值,必須慢慢把現金換為實體資產,可惜美國金融市場已經處在一個超大泡沫狀態,所以選項很有限。各國的中央銀行現在就已經面臨這個問題。
\\

\textit{\hfill\noindent\small 2020/06/24 09:26 提问;2020/06/24 09:46 回答}

\noindent[12.]{\Hei 答}:你沒有誤解,我的確是認爲美元在5-10年内會垮臺:如果中歐聯手,那麽不到5年,如果大家各自爲戰,則有可能拖過5年。

人民幣和臺幣的走勢應該是差不多的:短期内疲軟,美元崩潰時受益。
\\

\textit{\hfill\noindent\small 2021/03/08 23:49 提问;2021/03/09 23:24 回答}

\noindent[13.]{\Hei 答}:頂多如同當前的互聯網產業,中美各有一套體系;美國固然可以輕鬆搜刮歐、日、澳、英、台等傳統西方陣營國家,中國的體量足以保護自己。
\\

\textit{\hfill\noindent\small 2021/12/07 11:38 提问;2021/12/07 12:55 回答}

\noindent[14.]{\Hei 答}:零。
\\

\textit{\hfill\noindent\small 2021/12/16 10:48 提问;2021/12/17 04:22 回答}

\noindent[15.]{\Hei 答}:美元霸權除了靠金融系統本身的慣性之外,Nixon用石油來綁定歐日,然後再和歐日一起壟斷工業品的交易,從而間接控制第三世界。現在歐洲有自己的貨幣,早就主動進行過美元替代,第三世界的原料供應商和工業品消費者也都有了替代美元的意願。俄國的脫鈎準備主要針對歐盟:在於金融上脫離SWIFT(中方也已完成),以及在能源出口和工業進口上獲得替代;這裏中國剛好相反,需要的是能源和農產品的進口來源和工業品的出口市場,所以這些外貿關係的關鍵在於亞非拉,不但不是不作爲的藉口,反而是影響第三世界、對美元主動出擊的杠桿和底氣。至於什麽美方會翻臉云云,更純粹是瘋人院的夢囈:美國打擊中國還有矜持嗎?有什麽剩下的牌在手裏?別説俄國替換美元毫無外交後果,就看歐盟只是美國軍事外交上的附庸,都有足夠的勇氣和見識來用自己的貨幣做外貿替代,中國如果怕美方的報復,就是典型的拿自己的影子來嚇自己。
\\

\textit{\hfill\noindent\small 2021/12/16 23:06 提问;2021/12/17 03:14 回答}

\noindent[16.]{\Hei 答}:是被動地對被踢出美元體系做預先準備,我說的是主動出擊。
\\

\textit{\hfill\noindent\small 2021/12/17 21:30 提问;2021/12/18 03:23 回答}

\noindent[17.]{\Hei 答}:我以前提過,金融是所有學科中非常特別的一類,這個特殊性來自:1)它是高維度、極度複雜的博弈問題,必須實際操作才能掌握細節脈絡;2)一旦有了獨門的理解,可以轉換為大量真金白銀;3)但必須對上述知識嚴格保密,一旦外泄,就不再有價值(例如内綫消息)。所以學術人不可能真懂金融;真懂金融的人有對衝基金搶著要,不可能留在學術界或對大衆做營銷,也就不可能白紙黑字地寫下有用的心得;就算真有異類寫下這類心得,有99.9999\%的機率在你讀到之前就已經因爲公開而失效了。總結來説,你所要求的學習資料雖然表面上汗牛充棟,但基本都是無用的假話、廢話。

至於要看破美式經濟學的謬誤,其實對有嚴謹邏輯思路的人來説,容易得很。剛好我這兩年輔導了小孩的幾門經濟學課,他最不喜歡的就是我會自然而然地對著那些教材開罵(不是因爲他反對我的論證,而是對他準備考試有反作用)。我罵的是20世紀中期以後的大批經濟學理論,它們普遍基於與現實完全無關或甚至相反的邏輯前提。例如Coase的Property Rights Theory,號稱侵占社會公益的Externality問題來自產權不清,所以只要把全部社會公有財產私有化,然後讓資本在法庭互告就能解決。這個論述之離譜,是真正的罄竹難書:首先,社會公益先天就很難、甚至不可能以金錢衡量(例如正義、人命和文化思想,對應著法律、醫療和教育),所謂“產權”根本無從定義;其次,如果去强行談出一個價格,富者必然有不成比例的話語權;然後,實際執行和法庭紛爭都只在所有信息都是透明的前提下,才有可能,然而如果信息透明(事實上,信息不對稱本身就經常是Externality的來源之一),哪還須要用上訴訟這種效率極低、有許多攔路者來分一杯羹、並且可以用資源來做極限施壓(例如Trump就很習慣用專用律師來訛詐、欺壓商業對手)的手段?所以實際結果,就必然是資本利用Coase的歪論,反過來為侵占社會公益正名;這正是我一再解釋過的,美式經濟學人作爲資本的文字打手,奠定了腐化英美社會的理論基礎。
\\

\textit{\hfill\noindent\small 2021/12/20 04:52 提问;2021/12/29 04:32 回答}

\noindent[18.]{\Hei 答}:適合市場經濟的議題,一般都需要明晰的產權作爲前提。然而自由市場經濟先天就有局限性,這個局限性除了難以定價、或者公私必然對立的類別之外,還包括周期性長的產業;前者只能靠公權力來解決,後者則不完全排斥混合制產權(雖然必須小心國有資產流失的問題)。換句話説,明確私有產權的適用性比自由市場經濟要廣一些,但依舊不是100\%。
\\

\textit{\hfill\noindent\small 2021/12/22 00:32 提问;2021/12/29 04:40 回答}

\noindent[19.]{\Hei 答}:那是純粹胡扯:現在全球供給鏈處於供不應求狀態。
\\

\textit{\hfill\noindent\small 2021/12/30 05:20 提问;2021/12/30 06:17 回答}

\noindent[20.]{\Hei 答}:可以先從進口下手,尤其是大宗貨品。

每天重複解釋自己熟知的各種話題,一連幾個月下來是很膩的,腦子需要整休。從讀者的角度來看,天天求新成爲一種感官的刺激,也不是正確的心態,博客停頓更新一兩周,反而是大家復習舊文的大好時機,對學習思考有長遠的幫助。
\\

\textit{\hfill\noindent\small 2022/03/23 23:02 提问;2022/03/24 05:55 回答}

\noindent[21.]{\Hei 答}:其實這個做法非常理所當然,我前天上《八方論壇》的時候,還寫進自己的筆記裏;但當天談的亂,很多本來準備的議題沒有機會觸及,也包括這一點在内。一直沒有在博客討論,則是因爲Putin過去一個月已經連續幾次有脫離最優解的記錄,不能再和中國政府一樣被歸納於“接近絕對理性”的類別,所以也就不能再對他的選擇做“預測”,頂多只能當作“建議”來發表評論,然而博文和留言的建言一般都是從中方的角度來做的,如果沒人問,就沒有機會談起。另一個類似的手段是對油氣管道做“維修”;今天消息傳來,俄國國内一條主要管道在Novorossiysk附近因“風暴毀損”,必須“關停維修”一個月以上。大家可以拿來和盧布交易這件事一並考慮一下,是否俄方主動升級了反制。

盧布面臨美國的貶值打壓企圖,原本就抵抗得很成功:一開始掉了40\%,但很快就恢復了15\%,然後穩定下來。既然踢出SWIFT和扣押外匯等金融制裁的作用通道都是要逼迫盧布的持有者趕緊賤賣,俄方除了管制貨幣交易之外,要求歐盟國家以盧布來購買油氣,自然是釜底抽薪的合理策略;畢竟在當前西方濫用貨幣霸權、包括濫印鈔票的背景下,大宗貨品如能源才是真正的硬通貨,拿來和歐盟交換自己被禁用的美元和歐元實在是極度明顯的不合理。不過這是歐美貨幣霸權動搖的反映,不是對其推動促進的重要因素:歐美自行打擊既有國際金融體系的信譽,第三世界國家普遍想要改用替代性國際貨幣,這是一開始我就强調的客觀事實;能否把握機會趕緊建立一個大家都能接受的新貨幣選項,才是未來歷史發展方向的關鍵。
\\

\textit{\hfill\noindent\small 2022/07/21 13:11 提问;2022/07/22 01:07 回答}

\noindent[22.]{\Hei 答}:我自己也常常只記得説過,但找不到哪裏。

順便和大家解釋一下,新讀者往往以爲我喜歡自誇;但如果你去復習博客早年,就會看出我原本屬於内斂型的中國南方文化(參見前文《訪意大利有感》),連自己曾經領導瑞聯銀的創新團隊,發明世界第一個全自動程序交易系統的事,都是博客寫了幾年才提起的。近年來必須反復、高調、明白指出自己高明正確之處,其實是被逼的:整個華語圈就我一個能可靠地對時事做出成千上萬個原創性評論,自然吸引了一大堆盜用智慧財產的人(很不幸的,這就是當前的大陸網絡文化),有意而且習慣性地竊取我的創新論點、假裝成自己的。我曾專文討論過(參見前文《高能物理的牛屎文化》),丁肇中的諾獎研究被盜竊之後,性格轉了180°,從密不發聲變成又一個胡吹亂蓋的Bullshitter;我至少還繼續堅持事實。不過如果是自己的功勞,明知大陸有幾十、上百個網紅盯著想偷,不特別指明反而是不合理的。
\\

\textit{\hfill\noindent\small 2022/07/21 17:38 提问;2022/07/22 00:29 回答}

\noindent[23.]{\Hei 答}:那可能是我第一次直白評論這件事;問題在於時間太早,博客完全沒有名氣,也就完全沒有影響力。
\\


\section{【工业】中共的下一个產业技术攻关:晶片}
\subsection{2014-10-17 01:13}


\section{2条问答}

\textit{\hfill\noindent\small 2015/11/28 00:00 提问; 回答}

\noindent[1.]{\Hei 答}:工业化和城市化是经济发展过程中相辅相成的步骤。中国的工业化已初有小成,《中国制造2025》是继续提升工业能力的计划;而中国的城市化程度还很低,衹有不到40\%,8亿多乡村民众仍然等着踏进现代工业化之后的城市生活,所以中国的中长期经济展望极佳,那些唱衰中国的西方人士都衹是在发泄自己的私欲,而不是做理性的判断。

中国经济在短期(两年)内是有困难的。这主要是前几年以出口为导向的举债投资做得太过头了,现在全世界的需求跟不上来,原本就必须要做一些痛苦的调整。结果今年的股市又遇到严重的人谋不臧,所以眼前的挑战就更严峻了。

今天传出消息,中共常委会特别开了扶贫会议,我想这在全世界都是绝无仅有的。习近平至少大方向搞对了,现在只等着看执行团队是否给力。\\

\textit{\hfill\noindent\small 2018/03/14 03:19 提问;2018/03/14 03:33 回答}

\noindent[2.]{\Hei 答}:
其實有關半導體的Moore’s Law即將失效,我也提過好幾次了。我同意在2025年之後,半導體產業很可能會撞墻,成爲成熟的行業。這個過程必然會將計算機的重點轉移到軟件和應用上,你説的AI應該會紅上好一陣子。
我不詳談這些事,是因爲這個趨勢是大家都知道的,也就不需要我來插嘴。而且自由市場經濟,通常會一窩蜂、追捧過頭,即使大趨勢是對的。
中國正在全力投資追趕美、韓、台的半導體工業,這正是像你這樣人才囘國的大好機會,與其在美國公司做萬年的基層技術人員,回國15、20年,你可能當上廠長、副廠長,何樂而不爲?
\\


\section{【政治】西方对中国制度的短视}
\subsection{2014-10-24 02:12}


\section{12条问答}

\textit{\hfill\noindent\small 2017/05/28 00:00 提问; 回答}

\noindent[1.]{\Hei 答}:Piketty's book is a serious piece of academic work, not really meant for laypeople. You may find it easier to grasp his points by reading various summary and comments by others.\\

\textit{\hfill\noindent\small 2017/06/22 00:00 提问; 回答}

\noindent[2.]{\Hei 答}:从邓小平开始,中共认识到要进步必须先放任发展,等略为成熟之后才能开始限制规范。这是经济的先天规律,重点是在规范的过程中,必须是从国家社会的大局着想,而不应该是美国式的任由资本家们自行斗争分赃。\\

\textit{\hfill\noindent\small 2017/06/22 00:00 提问; 回答}

\noindent[3.]{\Hei 答}:你説的有关信用卡常被盗刷,没有错,但是不应该上纲到人身攻击。

美国的信用卡公司,对盗刷的态度就把它当作生意上的日常费用。这是因为去抓人所费不赀,最便宜的办法就是销账认赔。至于长期鼓励犯罪的社会成本,这些银行和信用卡公司是不在乎的。\\

\textit{\hfill\noindent\small 2017/06/22 00:00 提问; 回答}

\noindent[4.]{\Hei 答}:我想中共对这种事一向很敏感,应该很快就会有对策,例如可以一个官员一旦升到副部级,就要求所有记录全删。\\

\textit{\hfill\noindent\small 2017/06/22 00:00 提问; 回答}

\noindent[5.]{\Hei 答}:我觉得维持并发展一个社会的道德系统,是政府的责任。习近平的头五年,做了很多改革,但是在打假抓骗方面,还没有真正用力。希望在十九大之后,能有所作为。

这几年,有些地方官为了怕事,明明知道是跌倒的人在讹诈好心来扶的第三者,也不肯主持公道。长期下来,把一个原本见义勇为的社会搞成人人自危。这其实是所谓\&ldquo;不作为\&rdquo;中最恶劣的,可是纪委似乎还没有理解它的贻害之大,可嘆。\\

\textit{\hfill\noindent\small 2017/06/22 00:00 提问; 回答}

\noindent[6.]{\Hei 答}:真正可怕的是,连要捐钱建立一个医院,都差点被公家的官僚手续拖死了。这些繁文缛节,原本是西方凭着全球掠夺,所以可以无视经济效率的背景下的奢侈品。真不知臺湾凭什么,跟人家比奢侈?\\

\textit{\hfill\noindent\small 2017/06/22 00:00 提问; 回答}

\noindent[7.]{\Hei 答}:如果经济还在飞速增长转型,社会建设日新月异,很多法规上的细节必然跟不上。欧日这样的成熟经济,好处在于社会背景基本停滞,所以政治和法律理论上有时间精益求精;坏处是人民与政客往往都不思进取,只关心分赃,旧的法规不去反省,新的法规随意增添,反而作茧自缚。\\

\textit{\hfill\noindent\small 2017/09/12 00:00 提问; 回答}

\noindent[8.]{\Hei 答}:这种欣欣向荣、日新月异的气氛,每个工业化国家在早期都经歷过。问题在于如何避免制度和文化的僵化。这一代人有狼性、肯吃苦,从十岁起就玩手机的下一代呢?

中国还没有真正跨入先进国家的行列,要做的事还很多。\\

\textit{\hfill\noindent\small 2017/10/18 00:00 提问; 回答}

\noindent[9.]{\Hei 答}:我从部落格一开始,就强调贫富不均是21世纪人类的第一大问题,而中国是解决它的唯一希望。现在看到中共官方也有这样的人是,让人欣慰。\\

\textit{\hfill\noindent\small 2021/06/18 16:00 提问;2021/06/20 00:59 回答}

\noindent[10.]{\Hei 答}:
你的描述並非完全沒有道理,不過最後為地方官脫罪的説辭我不能苟同:這些項目都是有合同的,合同的核心是私營資本必須投入多少多少,“高風險投資失敗”是生意不賺錢,和廠房根本沒建成是兩回事。你事先不去檢驗資金是否存在,事後放著違約的法律責任不去查辦,這本身就是又一層次的犯法謀私。
\\

\textit{\hfill\noindent\small 2021/06/19 19:35 提问;2021/06/20 00:12 回答}

\noindent[11.]{\Hei 答}:所以國家紀委、監委早該介入;這些半導體詐騙所造成的損失,比之華融有過之而無不及,槍斃幾個人也不爲過。
\\

\textit{\hfill\noindent\small 2022/03/21 02:26 提问;2022/03/21 04:29 回答}

\noindent[12.]{\Hei 答}:我覺得是歐美社會全面愚化的結果,媒體、學術和智庫經過兩個世代的嚴格逆淘汰,只有最愚蠢、最偏執、最欠缺科學修養的人存活下來,而且還不斷被鼓勵强化自身的心理成見和宗教迷信,能維持理性思維的被徹底邊緣化(例如Stiglitz和Mearsheimer)。中國也有類似的現象,像是中醫教就已經腐爛了三四個世代,幾百萬從業者中,還想要像屠呦呦那樣做科學的不及1\%。
\\


\section{【太空】嫦娥5号T1顺利返回地球}
\subsection{2014-11-01 00:41}


\section{2条问答}

\textit{\hfill\noindent\small 2015/09/09 00:00 提问; 回答}

\noindent[1.]{\Hei 答}:但是它没有实际经济上的应用。

人类的科技并不是把同等难度的进步做齐平发展的,而是受到经济需求的强烈制约,有用的就突破到很难很细的地步,没用的就没人管。

深太空轨道的控制技术只能用在深太空,所以有没有用,基本取决于未来会有多少深太空探测计划,而至少50年内这些计划除了满足基础科学的好奇心之外,只有虚荣性的贡献,完全没有回馈经济成长的可能。\\

\textit{\hfill\noindent\small 2015/09/10 00:00 提问; 回答}

\noindent[2.]{\Hei 答}:I'm afraid not. Space travel is and will remain expensive, dangerous and largely a state affair.\\


\section{【迴响】与大陆网友的问答节录(一)}
\subsection{2014-11-02 18:20}


\section{1条问答}

\textit{\hfill\noindent\small 2018/03/18 12:16 提问;2018/03/19 04:54 回答}

\noindent[1.]{\Hei 答}:
資本先天是不可能完全聯合的;那1億的利潤該屬於誰,絕對不會有共識。
如果資本主義發展到極緻,就必然是回歸封建制度,各個資本集團和地方土豪成為世襲的諸侯,一個相對微弱的共主負責協調他們之間的矛盾。
我並沒有詳讀Marx的著作;印象中他對資本主義的批評是很到位的,但是並沒有提出真正可行的取代方案,後世的幾個早期實驗也失敗了。我們看看習近平是否能成功吧。
\\


\section{【政治】政府的第一要务 }
\subsection{2014-11-11 00:48}


\section{1条问答}

\textit{\hfill\noindent\small 2015/11/17 00:00 提问; 回答}

\noindent[1.]{\Hei 答}:这不衹是悲哀而已,有很大的真正经济损失。尤其美国财阀连自己国内的老百姓都不放过,臺湾的弱势群体被间接搜刮而吃的亏实在让我非常难过。\\


\section{【能源】【经济】谈油价}
\subsection{2014-11-24 22:28}


\section{1条问答}

\textit{\hfill\noindent\small 2016/01/30 00:00 提问; 回答}

\noindent[1.]{\Hei 答}:我低估了金融资本炒作的能力,所以没有想到油价会跌到\$30;不过的确是在2016年就开始回升,普丁也撑下去了。\\


\section{【能源】【經濟】2030年左右}
\subsection{2014-11-25 15:08}


\section{2条问答}

\textit{\hfill\noindent\small 2020/08/02 15:41 提问;2020/08/04 05:57 回答}

\noindent[1.]{\Hei 答}:這的確是人類在解決氣候變化這一大問題的當下,要提高電能來源中光伏和風能所占比率的主要難關;所以我一直在不斷更新這方面的消息。

不過先糾正你論述中的小技術錯誤:首先如果忽略傳輸和儲能上的費用,光伏和風能在很多地區已經比煤電便宜,尤其是前者;其次,煤電和核電不只是“能夠”24小時運轉,事實上是須要24小時運轉,否則每次停機都會有很大的效率損失。這也是爲什麽當前適合Load Following的天然氣電廠往往和光伏配對建設,以滿足傍晚的用電尖峰時段。

歐美因爲被他們白左政治正確的扭曲思維所誤導,連這種純粹是工業技術上的議題都走上歧途,去追求低效、困難、昂貴的氫經濟;目前只有中國一家對全釩電池的大型化、產業化在做持續投資。這當然又是西方主動送上門來的機遇,不過中國仍舊只把液流電池視爲衆多備份選項之一,沒有當作重點來攻關,在研發時間上不夠急迫,在明眼人看來有點可惜。
\\

\textit{\hfill\noindent\small 2021/04/08 16:59 提问;2021/04/08 19:22 回答}

\noindent[2.]{\Hei 答}:但是這並不代表中國沒有自私自利的利益集團來阻礙產業升級,各種騙補、為外企當馬甲的商人多得很,更別提只想著花公款搞“政績”的地方官員。反腐是絕對必要的,但光是反腐還不夠,對犧牲國家利益、只想著自己升官的幹部,也必須有正當的處置。
\\


\section{【经济】美式经济学是骗人把戏的又一表徵}
\subsection{2014-12-15 00:43}


\section{5条问答}

\textit{\hfill\noindent\small 2015/09/10 00:00 提问; 回答}

\noindent[1.]{\Hei 答}:他也就只有一个头衔,连基本的专业能力和态度都没有,不值一提。\\

\textit{\hfill\noindent\small 2016/05/29 00:00 提问; 回答}

\noindent[2.]{\Hei 答}:好,谢谢。\\

\textit{\hfill\noindent\small 2017/05/17 00:00 提问; 回答}

\noindent[3.]{\Hei 答}:我知道他,他也是学物理出身的,对忽悠有相当的免疫力。\\

\textit{\hfill\noindent\small 2020/04/06 01:56 提问;2020/04/06 02:23 回答}

\noindent[4.]{\Hei 答}:
《UDN》”新版“的bug很多,其中之一是如果你留言裏的網絡鏈接太長,就會搞壞網頁的版式。讀者不常用,不知不罪,但是這對我是個困擾,請大家盡可能避免。 
美國經濟系的“學人”轉化為財閥土豪的文字打手,其實源自Rockefeller設立芝加哥大學,1929年搞出大蕭條之後暫時噤聲,在二戰後好了瘡疤忘了疼,在Milton Friedman領導下學術賣淫業重新開張,比Murdoch進軍英美媒體界、游説業的興起、右翼智庫的建立和擴張都還早了20多年。早期(1960、1970年代)還有爲了名利而同流合污的,最近30年已經完全洗腦成功,對整個行業完成了逆淘汰,比超弦席捲高能物理還要徹底。 
台灣和香港是英美宣傳洗腦的重災區,社會的理性傳統一旦消失,就必須等所有人口自然死亡才能改變,這至少要兩代人的時間。我一再强調台灣最需要的是教改,就是希望挽救還在學校裏的那一代。
\\

\textit{\hfill\noindent\small 2020/11/02 14:24 提问;2021/01/20 06:38 回答}

\noindent[5.]{\Hei 答}:福山是聰明人,可惜年紀大了,依舊沒有發展出智慧。

新自由主義經濟理論,在美國是主流,以其他國家的角度來看,是極度離譜的胡扯;福山心中想用來替代它的修正,在美國是左翼思想,在客觀標準上,卻依然是非常偏袒資本的市場經濟理論。
\\


\section{【美国】富豪口袋里的国家}
\subsection{2014-12-16 05:15}


\section{1条问答}

\textit{\hfill\noindent\small 2015/11/26 00:00 提问; 回答}

\noindent[1.]{\Hei 答}:SWAP基本就是我同意把我的一门生意的进账和你的一门生意的收入交换。所以银行被禁止做什么生意,衹要另设一个不是银行的公司去做那门生意,再通过SWAP一样可以得到那个生意的回报。\\


\section{【美国】【金融】美式民主的真正主人}
\subsection{2015-01-06 12:25}


\section{3条问答}

\textit{\hfill\noindent\small 2015/11/05 00:00 提问; 回答}

\noindent[1.]{\Hei 答}:这个案件是芝加哥的联邦监察长办的,SEC和CFTC衹是当证人罢了。

这衹是打打苍蝇,被抓的这人是小猫一只,高盛的人还是高枕无忧的。衹要不敢抓高盛,就是做样子的。而到目前唯一一个敢动高盛的,就是我在《美国式的恐龙法官(三)》里提到的Preet Bharara,也衹动过一次,还衹是董事,不是高盛自己的生意。\\

\textit{\hfill\noindent\small 2021/08/12 16:52 提问;2021/08/13 02:35 回答}

\noindent[2.]{\Hei 答}:不止是程序交易,其實所有的無風險交易獲利都是這麽來的,例如“私有的管道”不但包括“内綫交易”或“信息不對稱”,也涵蓋Captive Order Flow獨占性交易流量。
\\

\textit{\hfill\noindent\small 2021/09/28 22:20 提问;2021/09/29 00:26 回答}

\noindent[3.]{\Hei 答}:Sachs始終是有良心、有理想的,只不過年少得志,成名時還沒有足夠的智慧,迷信課堂所學的那套美式經濟學,被幕後的美國財閥利用來矇騙蘇東政府。我覺得他其實也是受害者,事後他也明顯悔悟,30年下來試圖彌補過往的錯誤一直不遺餘力。

Stiglitz沒有這樣的污點,在過去30年被排擠得更加厲害。像是這類有良心、有見識的學術大佬,正是中國外宣的先天盟友;這裏的意思不是說必須私底下塞錢,那反而會弄巧成拙,而是政策細節不應該打這些人的臉。例如今年稍早有一個諾貝獎得主的聚會,達賴也獲得邀請,結果中國外交部居然公開譴責。這又不是官方訪問,難道達賴在紐約買麵包,你也要譴責那個麵包店嗎?結果Stiglitz被迫寫公開信反駁,這是典型的拿槍打自己的脚,真讓人懷疑幕後決策的外交部主管是否打著紅旗反紅旗、故意扯國家的後腿。
\\


\section{【美國】言論自由的假相}
\subsection{2015-01-18 13:40}


\section{2条问答}

\textit{\hfill\noindent\small 2015/09/13 00:00 提问; 回答}

\noindent[1.]{\Hei 答}:是的,详见《四场演讲》。\\

\textit{\hfill\noindent\small 2015/09/14 00:00 提问; 回答}

\noindent[2.]{\Hei 答}:犹太人专注的是以色列的利益。对国内外的经济掠夺则是所有财阀的共同利益。

犹太人的确是很傲慢的。在中国人大批到来之前,他们在学术上比一般白人强,于是就有犹太学者写论文说他们的基因天生就比较聪明。后来中国人比他们还会读书,智商测验也更高,他们就转为专心保障自己人在一流学校的入学名额了。目前哈佛每年的新生,犹太人是亚裔的两倍,但是如果公平竞争,后者应该是前者的三倍才对。柏克莱有法律禁止歧视,亜裔就占绝对优势。\\


\section{【台湾】【空军】乡愿,洋奴和冤大头}
\subsection{2015-02-08 02:33}


\section{1条问答}

\textit{\hfill\noindent\small 2015/03/05 00:00 提问; 回答}

\noindent[1.]{\Hei 答}:其实这种民眾的无力感在全球的主要民主体制下,如美国、欧洲和日本,都很明显。台湾政坛受统独争议的牵制,所以自然是更为不务正业。

未来十年,台湾还不会面临兵灾(除非民进党异常努力地自己找死),最大的潜在威胁是我在《中共的下一个產业技术攻关:晶片》里谈过的,在电子產品上被大陆超越取代。台湾的经济近年来只靠这个產业,一旦被打垮了,成长的动力能从哪里来?

台湾20年来已经有100万的精英移民国外。老实说,除此之外,我也想不出还有什么更好的个人选择。\\


\section{【金融】【战略】希腊与欧元 }
\subsection{2015-02-11 10:05}


\section{20条问答}

\textit{\hfill\noindent\small 2015/06/28 00:00 提问; 回答}

\noindent[1.]{\Hei 答}:自从我写了这篇文章后,四个月来基本上Syriza到处碰壁,而且希腊资金大幅流失,经济更为衰弱。如果Syriza还有点理性,早就投降了;偏偏他们是股市菜鸟的心态,赔得越惨就越要死抱错误政策到底,不肯认赔杀出。所以很可能再拖上一两个月,直到破產为止。

照理说,国债还不出来和退出欧元区是两回事,可是Syriza如此顽固,德国必须杀鸡儆猴,那么被踢出欧元区就成了理所当然的结果。其后Syriza只能重发Drachma,应该在一个月内就贬值70\%左右,基本把中產阶级的储蓄扫的干乾净净。这时希腊还是要再发行新国债的,但是除了中俄以外,有谁会拿钱往水里扔呢?中俄政府当然也不是蠢蛋,自然会要求希腊拿有真正价值的资產来交换,到时港口、铁路还是要卖的,而且价钱更低。

一旦希腊对中国完全开放,中共当然寧可希腊留在欧盟,成为中国企业、移民和贸易进入欧盟区的桥头堡。所以只要希腊政府配合,中方的援助可能极为可观。而长期来看,希腊成为中欧之间的快捷通道,对大家都有利,唯一吃大亏的是美国。对德国则有利有弊,有利是可以与\&ldquo;一带一路\&ldquo;进一步整合,有弊是欧洲国家有了另一个大哥可以抱大腿,德国对欧元区以外的欧盟国家影响力会下降。\\

\textit{\hfill\noindent\small 2015/06/28 00:00 提问; 回答}

\noindent[2.]{\Hei 答}:没水没电的时候大概也是发新闻稿要求地球尊重台湾的民意。整个社会一起犯蠢,还指望物理定律和国际现实也跟着配合,真是现代奇观。\\

\textit{\hfill\noindent\small 2015/06/30 00:00 提问; 回答}

\noindent[3.]{\Hei 答}:This has been the case since 1973, after the US broke the Brenton Woods System and forced OPEC to use dollar for pricing oil.\\

\textit{\hfill\noindent\small 2015/06/30 00:00 提问; 回答}

\noindent[4.]{\Hei 答}:其实Tsipras会想公投,就是已经见了棺材开始掉泪,准备用公投结果压服Syriza内部的激进派。

不过如果公投结果是\&ldquo;否\&rdquo;,那么希腊就必然被踢出欧元区,因为我已经分析很多次,德国不须也不能妥协。\\

\textit{\hfill\noindent\small 2015/07/02 00:00 提问; 回答}

\noindent[5.]{\Hei 答}:旁观者清,连深绿的传媒都知道希腊人是在撒泼,可是他们偏偏就不明白镜子里也有自己的影子。\\

\textit{\hfill\noindent\small 2015/07/02 00:00 提问; 回答}

\noindent[6.]{\Hei 答}:同意,最惨的是继续当美国的附庸,沉沦到底,人才企业丧失殆尽,财富还不断地向美国转移。\\

\textit{\hfill\noindent\small 2015/07/02 00:00 提问; 回答}

\noindent[7.]{\Hei 答}:若是希腊不赖账,Syriza下了台,德国人杀鸡儆猴的目的已经达到,就很可能忍痛提高援助金额,最终会同意注销一部分希腊国债,但是必须等几年才悄悄地做,以免鼓励其他欧猪国家里主张赖账的在野党。

至于\&ldquo;歷史地位\&rdquo;,人都死了,就不要看的太重。我前一阵子坚持讨论了毛的定位,是因为那事对中国未来的政策方向还有很大的影响,关系到几亿人未来的生活水准。我想台湾这些总统的歷史定位不会有那样的影响力。\\

\textit{\hfill\noindent\small 2015/07/02 00:00 提问; 回答}

\noindent[8.]{\Hei 答}:他把退出欧元和退出欧盟混为一谈,不靠谱。\\

\textit{\hfill\noindent\small 2015/07/04 00:00 提问; 回答}

\noindent[9.]{\Hei 答}:奇怪的是台湾媒体对希腊人的歪论看得很清楚,却不明白他们看的其实是镜子里自己民主制度的影像。\\

\textit{\hfill\noindent\small 2015/07/06 00:00 提问; 回答}

\noindent[10.]{\Hei 答}:希腊若是这样搞,欧元区就从债权人变被害人,民事问题变成刑事问题,希腊有可能连欧盟都留不下了。\\

\textit{\hfill\noindent\small 2015/07/09 00:00 提问; 回答}

\noindent[11.]{\Hei 答}:我觉得这只是希腊政府的拖延战术;即使协议的条件在表面上满足债权人的要求,实际上不可能会被执行。

德国早已学乖了,这次必然要求更强的监督权,也就是希腊必须放弃更多的主权。折衝的细节再过几天就会明朗些了。\\

\textit{\hfill\noindent\small 2015/07/10 00:00 提问; 回答}

\noindent[12.]{\Hei 答}:民主制度下,没有决断的自由,只要耍赖的人愿意拖下去,就几乎可以无限期的吵闹,这次不就已经从二月拖到七月了。

希腊人若是不投降,才有被踢出的危险,所以只要假装投降,又可以拖几个月。\\

\textit{\hfill\noindent\small 2015/07/11 00:00 提问; 回答}

\noindent[13.]{\Hei 答}:这要视希腊递交的计划细节而定。我们稍安毋躁,到周一自有分晓。\\

\textit{\hfill\noindent\small 2015/07/11 00:00 提问; 回答}

\noindent[14.]{\Hei 答}:很难说,我也觉得法国公开和德国唱反调有违常理。\\

\textit{\hfill\noindent\small 2015/07/12 00:00 提问; 回答}

\noindent[15.]{\Hei 答}:其实如果齐普拉斯真的缴交了欧元区能接受的提议,这是一个相当高招的两面讨好的手段:选民赢了公投,得了面子,欧元区得以强加原先坚持的条件,得了里子,而齐普拉斯更是走了唯一一条能让自己继续当首相的路。若是他公开对欧元区投降,则选民不会饶过他;若是他坚不投降,则希腊被踢出欧元区,选民还是不会饶过他,所以他唯一的出路就是对内譁眾取宠,对外则出卖自己的支持者。希腊选民如此脑残,也只有存心欺骗他们然后把它们包装出卖的政客才能当选并续任。

不过事情还不明朗,欧元区会不会买单很难说;就算这回合买了单,几个月内,齐普拉斯必然会再度反复,到时大家还是回到第一回合。债权人顶多再多损失一些钱,希腊百姓才是最大输家:原本已经稳定的经济再度崩溃,却毫无补偿,这叫自作自受。只有齐普拉斯得以呼风唤雨,算是赢家。

希腊不可能依附美国。美国连乌克兰都没钱支持,收买希腊哪有可能叫价叫得过中俄?

有些大陆读者的帐户被骇过了,来看文的时候会留下广告,我也只能一个一个的删除。我对中时建议过要把广告帐户禁了,但是还没听到回信。\\

\textit{\hfill\noindent\small 2015/07/12 00:00 提问; 回答}

\noindent[16.]{\Hei 答}:希腊人民自我割喉之后,连尸体都被政客卖了,还在乎\&ldquo;侮辱\&rdquo;?\\

\textit{\hfill\noindent\small 2015/07/12 00:00 提问; 回答}

\noindent[17.]{\Hei 答}:有可能,但是另一个可能是法、意、西等国民意同情希腊,所以执政党虽然私底下对Syriza幸灾乐祸,公开场合却必须顶撞德国。\\

\textit{\hfill\noindent\small 2015/07/13 00:00 提问; 回答}

\noindent[18.]{\Hei 答}:会的。我即将返美,一旦时差调过来,就会着手。\\

\textit{\hfill\noindent\small 2015/07/14 00:00 提问; 回答}

\noindent[19.]{\Hei 答}:德国的底线是不做让步,要给闹事的一个教训,所以一旦希腊投降了,自然还是可以留它在欧元区。

美国不愿希腊成为中俄的棋子,留它在欧元区也可以继续牵制德国。

齐普拉斯说给希腊民眾的话都是哄小孩的。\\

\textit{\hfill\noindent\small 2023/11/25 11:24 提问;2023/11/26 06:28 回答}

\noindent[20.]{\Hei 答}:應該就是正文所談的,三頭馬車要求希臘削減赤字的事。照理説,建碼頭和財政談判無關,但歐美社會的實際真相,原本就是汎政治化,什麽體育不涉政治、法律不涉政治、經貿不涉政治、文化不涉政治等等都是哄人的宣傳噱頭;以往只不過是藉著國力上的壓倒性優勢,有餘裕裝大尾巴狼罷了。
\\


\section{【美国】科学界的卖淫者}
\subsection{2015-02-27 19:28}


\section{4条问答}

\textit{\hfill\noindent\small 2015/03/24 00:00 提问; 回答}

\noindent[1.]{\Hei 答}:他已经比Paul Krugman晚了17年了,而他在金融经济上的功力是绝对比不上Krugman的。这种人在过去20几年,每个礼拜都有;反正就是又多给我们一个笑话可看。\\

\textit{\hfill\noindent\small 2015/09/20 00:00 提问; 回答}

\noindent[2.]{\Hei 答}:美国基础教育的问题是多重的。首先它一直都由地方自己办,所以贫富差距很大,这当然是富人有意维持的现象。其次是1960年代之后引进的自由化、独立化教育成了放任偷懒的藉口。

所以美国高等教育里的理工人才素来有一半以上必须靠进口。\\

\textit{\hfill\noindent\small 2016/08/19 00:00 提问; 回答}

\noindent[3.]{\Hei 答}:全球暖化的科学证据确凿,你要讲阴谋论可以,还是那三个标准:动机、能力和证据。除了能力之外,其实两项根本不存在。\\

\textit{\hfill\noindent\small 2016/09/26 00:00 提问; 回答}

\noindent[4.]{\Hei 答}:个别的证据可以被反驳,但是整体来看,正面的证据太多了,不可能全是假大空。这和超弦不一样:超弦从来就没有任何正面的证据。

等了一年,我总算针对这个话题写了专文,请参閲。\\


\section{【工业】【能源】永远的未来技术}
\subsection{2015-03-01 02:59}


\section{20条问答}

\textit{\hfill\noindent\small 2015/03/01 00:00 提问; 回答}

\noindent[1.]{\Hei 答}:那些技术极不成熟、增加费用和重量,而且对输送完全无解。

我的论点不是技术上不可能,而是经济上不可能。经济上的考量又包括效费比和风険。核融合和氢气动力都在这两方面输得一塌糊涂。\\

\textit{\hfill\noindent\small 2015/03/01 00:00 提问; 回答}

\noindent[2.]{\Hei 答}:希望如此吧。电池充电速度慢是电极的问题,现有的新设计都还很不成熟,还有很大的研发空间。\\

\textit{\hfill\noindent\small 2015/03/01 00:00 提问; 回答}

\noindent[3.]{\Hei 答}:不过开长途车还是一个问题。

我自己开混合动力车将近十年,非常喜欢它的节能成果。我觉得在可见的未来,混合动力车还会是减碳技术的主流。\\

\textit{\hfill\noindent\small 2015/03/03 00:00 提问; 回答}

\noindent[4.]{\Hei 答}:可是连潜艇的AIP(絶气推进系统),德国Siemens的燃料电池都不敢用氢。。。

在民用上,氢必须靠电能產生,全生命循环的效率很低,所以零售上真没有经济价值。电本身就是很容易传送的能量,只有储存是个麻烦。用氢来储能,以备尖峰用电时发电,或许是可行的。但是行内人忙着忽悠大眾,搞氢气车,在那方面反而没什么动静。\\

\textit{\hfill\noindent\small 2015/03/23 00:00 提问; 回答}

\noindent[5.]{\Hei 答}:这篇文章我看了,觉得这个技术构想没有氢动力汽车那么糟糕(主要是氢的储藏和传输限制在专业人员的手中),但是在经济上仍然是没有前途的。氢必须用电离来生產,然后在应用时再由燃料电池转换为电能;这两个步骤的效率都比内燃机高得多,但是仍然不是100\%,所以氢动力列车和电动列车相比,先天上就是画蛇添足,而这还没有考虑到氢在储藏和传输上的费用和风险。\\

\textit{\hfill\noindent\small 2015/04/09 00:00 提问; 回答}

\noindent[6.]{\Hei 答}:EAST是比ITER小半代的先期实验,他们这次吹嘘的\&ldquo;成就\&rdquo;还是在用磁场约束等离子体的技术细节上。

这些实验并不是完全没有价值的:高温等离子体本身就是一个物理研究的尖端项目,但是这是纯粹的基础科研,也就是为了新知识而研究新知识。它的实际应用或许包括氢弹的改进(我对那方面完全无知),但是绝对不包括发电。所以我的批评只针对核聚变电厂这类的不诚实吹嘘;如果投资方是为了基础科研或者改进氢弹而继续推进这方面的研究,我完全没有负面意见。\\

\textit{\hfill\noindent\small 2015/04/09 00:00 提问; 回答}

\noindent[7.]{\Hei 答}:钱投下去自然有进步,但是这些技术性的指标突破和最终要发电没有关系,因为他们对真正的难关还是无解。这些新闻稿只是用来骗经费的。

我的Motto是\&ldquo;Truth And Logic\&rdquo;,\&ldquo;事实与逻辑\&rdquo;,所以有时会写些文章来批判媒体所刊登的虚伪宣传。现代社会里的虚伪宣传当然是为各式各样的自私特权服务,而他们不怕被揭露辟谣的原因又依其靠山分为几类:有政治靠山的,有经济靠山的,和有专业知识门栏的。这些搞核聚变的属于最后一类。\\

\textit{\hfill\noindent\small 2015/06/05 00:00 提问; 回答}

\noindent[8.]{\Hei 答}:你不是学物理的,请不要拿一知半解的认知来哄人。我的时间有限,不能一个一个人地教。文章已经写了,补充材料得你自己去找。如果你的基础教育不够,不能理解,那就不该在这里下断言。美国人写文的步调和中国式的不一样,翻成中文很容易被误解。你只凭着一篇翻译过来的科普文,就要来否定有专业教育的人的意见,未免太自不量力。

磁场控制等离子体的困难,是可以靠技术和材料的进步来解决的。如果ITER还没有完全解决,下一代也必然会做到。

中子的处理问题是物理困难,这才是工程手段无法解决的。

最后说一句,为了自己面子而在留言栏死鸭子嘴硬纯抬杠的,我会直接删除。我的部落格是为了人口中前0.1\%的精英而写的;如果没那个水准,要旁听我不介意,要发言就必须以不打扰正题为前提。\\

\textit{\hfill\noindent\small 2015/10/20 00:00 提问; 回答}

\noindent[9.]{\Hei 答}:中子的穿透力很强,这是因为它是电中性的,所以所谓的中子屏蔽,衹能靠原子核藉强作用力来与之反应。但是强作用力的距离很短,反应截面很小;反应截面最大的是氢原子核,所以水是很好的中子减速剂。

理论上可以把水管装在磁綫圈内部,但是这就有了新的问题:1)磁场必须做得更大更强,但是人造磁场是有极限的;2)内壁或许可以用砖块,但是水管却是承力结构(里面是极高压超临界水)。

仔细想想,中子携带了核聚变反应后的大部分动能,电厂必须把它吸收到水里,所以你看到的那些砖块并不是用来屏蔽中子的,而是用来屏蔽等离子的。这是因为等离子体遵守波茨曼分布,不论磁场多强,总有一些离子的动能特别高,能够突破电磁障壁。

因此,真正的中子屏蔽其实正是那些水管,问题的核心也就在这些水管,它们不但是承力结构,还承受了全剂量的中子轰击,因此必须经常更换。人类还没有发明能在高放射性环境下,经常更换承力结构的技术。

这些水管是远远不足以屏蔽全部中子辐射的,所以连磁綫圈也会需要定期更换,衹不过是没有像水管那么频繁;不过总体来看,仍然是每隔几个月就必须把已受高辐射污染的整个反应器拆开重建。人类若是有这个技术,还是先把车诺比和福岛的反应炉清理一下吧。

正文里的那句话的确写得太简略了,我已更正。\\

\textit{\hfill\noindent\small 2015/12/11 00:00 提问; 回答}

\noindent[10.]{\Hei 答}:正是因为氦3聚变不產生中子,所以科幻小説里常常拿开采氦3作为殖民月球的经济动机。

实际上氦3聚变要求的温度比氘-氚聚变高十倍左右,远非目前的技术所能达到的。此外,虽然没有屏蔽中子的难题,但是怎样把聚变產生的热能有效地取出用以发电仍然是个未解的问题。\\

\textit{\hfill\noindent\small 2016/09/22 00:00 提问; 回答}

\noindent[11.]{\Hei 答}:这是一个安全的储存技术,但是密度很低,重量、容积都太大,还不如用电池。

此外,从生產点运输到消费点的问题还是没有解决。\\

\textit{\hfill\noindent\small 2021/09/28 21:47 提问;2021/09/29 00:58 回答}

\noindent[12.]{\Hei 答}:一般人往往不知道,核聚變比裂變還要早發現,根本不是什麽“下一代”的突破;剛好相反,裂變才是當年取代聚變、解決困難的神奇新科技。聚變之所以被無知群衆拿來當未來技術,恰恰是因爲它先天的基本缺陷無可跨越,導致90年下來還不如裂變發現後頭5年的進展。那些科幻敘事,把裂變到聚變說成“進步”,等同於說人類終究必須進化成猴子,或者汽車必須換成老鼠拉車,所以要求幾十年和幾千億美金來對人和鼠做基因改造;對外行人或許可以裝扮成高科技,但實際上是純粹的虛功,偏偏他們有辦法把這樣的計劃寫進十四五。

過去半個多世紀來,聚變研究人員的慣例是說還要“30年”(說“50年”的也有,是廉恥心沒有100\%被狗吃光的人,但不是多數)。不過最近5年,恰恰在NIF這類“大科學”計劃Crash \& Burn的背景下,誇口越來越離譜,對新的金主開始改口為10年或甚至5年,其基本原因是氣候變化的證據越來越明確,所以減碳政策越來越緊急,眼看其他真正有用的科技(太陽能和風電已經實用化,由裂變輔助的儲能電池是最後一步)即將完全成熟,10年後這套騙術將徹底失去市場,所以只好拼命撈最後一把了。
\\

\textit{\hfill\noindent\small 2021/09/30 11:19 提问;2021/10/02 04:09 回答}

\noindent[13.]{\Hei 答}:他說的基本沒錯,只是省略了一些新發展和細節;整體來説,不算離譜。

(1)目前批量投產的儲能電池,95\%以上是普通鋰電池,這的確有安全性問題,但是行業已經開始向磷酸鐵鋰和液流電池轉換,實際上沒有人認爲十年後還會繼續用傳統鋰電池。(2)氫能的危險性在於零售應用,用在電網儲能時的問題是經濟性,這一點沒有錯,不過我始終也是這麽解釋的啊。(3)水電如果存在,當然是最理想的,問題在於一方面它很有限,另一方面水資源是一個比電力更爲短缺的東西,發電/儲能往往不能是水庫的最優先任務。(4)同上,如果建成了,抽水蓄能的運作效率高、成本低,這也是實話,但前提是必須忽略修建的花費和選址的困難。

我以前已經解釋過,抽水儲能如果是在河川截流的水庫發電站做雙向運行,經濟性還可以説得過去,如果是另外專門挖蓄水池,那就是庸人自擾;以下針對後面這一點做詳細論證:

抽水儲能站必須滿足幾個條件:在一個豐富可靠的水源(中型以上的湖泊或河流)邊邊,剛好有一個懸崖,懸崖上有一個丘陵平臺,既平坦又寬廣,足夠建造人工湖,湖水冬天也不能結冰,而且這個山丘地質必須很穩定,能夠承受山頂新增的重量而沒有土崩危險。這裏我簡單估算一下:假設儲能容量為100GWh,這相當於3.6*10\^{}14J,再假設高度落差為200公尺,這對應著大約0.2個Gigatonne的水容量,相當於600座帝國大廈,對地基的要求非同小可。此外,假設人工湖平均深度為4米,則這個人工湖直徑大約為8公里,山頂平臺上若是有如此寬廣的平地,早就住滿人了。

這裏的基本問題在於重力位能的密度遠低於化學能,實際上選址只能找到直徑250米級別的地點,那麽容量只有前面計算的千分之一,亦即100MWh,用電池的話,相當於1200台Tesla 3的電池組,堆不滿一個小倉庫,建造價格更是差了兩三個數量級(這還沒有考慮地價和電力傳輸系統),所以相比之下,抽水儲能完全沒有競爭力,根本不可能普及。


有讀者反應,張談的應該是在同一個峽口建上下兩個水庫,既發電也可以儲能。但是他給的鏈接會導致UDN的留言欄發生格式錯誤,所以我必須刪掉,只好在這裏回復:

是的,上面我早先的回復版本可能有誤解/會引發誤解,我已經做出合適的增刪修訂。至於爲什麽我原本直接談人造池塘的儲水方案,這有幾個原因:(1)博客以前已經解釋過,水庫發電站逆向儲能的方案,在經濟性上一般是可行的,問題只在於適合建的河流不多,以及水資源的運用有超越發電或儲能的考慮;(2)水庫雙向發電儲能是很老的技術,不應該還有什麽爭議;(3)現在美國的Start-up,談抽水儲能時,指的就是新建人工池,這樣選址比較有彈性,至於浪費錢,那原本就是他們的用意。


爲了徹底澄清我的看法,在此再加一個總結:建雙重水庫的儲能站,原則上絕對是值得考慮的,執行上必須依個案的特點,做性價比的分析,但是在大局規劃上,水電儲能先天就嚴重受限於合適地點的Availability,不可能成爲低碳能源體系的主力。
\\

\textit{\hfill\noindent\small 2021/09/30 20:59 提问;2021/10/01 05:32 回答}

\noindent[14.]{\Hei 答}:謝謝更正。這來自我從小的壞習慣,懶得用筆算,即使在博士班,同學寫滿兩三頁的演算過程,我都先試圖用心算;年輕這樣做還有時可行,現在記性比以前差太多了,老是掉項,只好改成筆不離手。但是這個算式太簡單,我老毛病犯了,又想偷懶,原本考慮了重力加速度,結果算著算著就忘了。

所以年輕人千萬不要辜負光陰,現代世界裏有趣的知識太多了,適合學習的東西,就應該趕緊用心去學。等到了我這個年紀,基本就沒辦法再學完整的全新課題,頂多只能得其綱要輪廓。例如上個月我心血來潮,想深研Galois Group,結果20幾歲時可以幾天解決的題材,進度一拖再拖,學了東忘了西,只好淺嘗即止。
\\

\textit{\hfill\noindent\small 2021/10/19 12:30 提问;2021/10/20 05:47 回答}

\noindent[15.]{\Hei 答}:一般人只關注發電容量(Generating Capacity),其實越環保間歇性就越强,應該看的是實際年發電量;換句話說,不能拿不同類的電力來源以MW和MW相比,至少得用MWh。一個簡單的Rule of thumb是:核能持續發電能力為100\%,水電、風電2/5,光伏只有1/5。然後這樣的間歇性要求儲能上的配對投資,耗費一下子加倍,這還沒有考慮長程輸電所需的資金。

不過正如你的分析,光伏/風電+儲能的大規模全面應用,不但沒有任何基本的技術難題,連價格都已經接近可以和煤電直接競爭的門檻,只要繼續批量投產,經濟性完全達標指日可待。國家的責任,在於推動儲能和輸電的建設發展;這樣明確的邏輯結論,我在過去六年反復解釋傳播,結果歐美政府沒有理性倒也罷了,中國也被忽悠到核聚變和氫能源的歪路上,直接導致儲能和輸電建設落後需求,這是現在缺電限電的直接原因之一,經濟損失以百億計。然而若說肉食者鄙,就抓錯重點了,畢竟這是一個專業性高的議題;真正的問題在於學術界沒有人敢出來爲國説實話,縱容騙子滿足小圈子利益,明顯錯誤的認知被接受為“常識”。唉,楊先生這樣的國士終究還是單獨的異類。
\\

\textit{\hfill\noindent\small 2021/10/23 12:52 提问;2021/10/24 04:32 回答}

\noindent[16.]{\Hei 答}:電動車要從40、50\%的占有率,上升到80\%以上,最大的阻礙在於充電樁的普及,而且這不是產業自己能解決的。我很擔心中國政府沒有足夠的遠見,關注和投資不足,事到臨頭才試圖亡羊補牢,如同現在的缺電問題一樣,所以正在計劃專門寫一篇文章來詳細討論(當然,博客對能源問題已經提早幾年警告得極爲明白,但言者諄諄、聽者藐藐,也沒有用;這正是所謂的Cassandras curse)。

另一個附帶的觀察,是電池固然是電動車的頭號關鍵技術,但是Power MOSFET也是一個重要部件。當前的主流是SiC,五年後可能會進化為GaN,中國在這兩方面的商業應用都還沒有達到第一梯隊。要做追趕,在行業營收全面爆發之前會容易得多。這又是一個中國政府需要先見之明的角度。

至於電池的技術選擇,反而不成問題;私企達到規模之後,完全可以自行嘗試解決。鈉離子電池只是可能的路綫之一,更熱門的還有Solid-state Electrolyte(雖然我個人不看好;因爲電池最大的技術難關,向來都是電解液和電極之間的界面,把電解液改爲固態徒增其困難,感覺上是自找麻煩)等等,最終哪一個勝出取決於成本、安全性、甚至是推銷能力那些細節,政府不必操心。
\\

\textit{\hfill\noindent\small 2021/12/05 05:39 提问;2021/12/05 07:49 回答}

\noindent[17.]{\Hei 答}:很高興看到有等離子體物理出身的人願意説實話。最近英美媒體對核聚變的吹捧又上升了一個級別,本周MIT系的Commonwealth Fusion創下新記錄,騙到18億美元;然而來自美聯儲大水汎濫的容易錢,才是核聚變成爲熱門“未來科技”的真實原因。

除了p+B之外,我以前也討論過另一個Alternative,亦即用氦3,不過它的特性和缺點基本一樣:和D+T相比,所需的溫度高不止一個數量級,中子數只減少一個半數量級,而且氦3和氚一樣稀缺昂貴,所以更加不值深究。
\\

\textit{\hfill\noindent\small 2022/02/10 08:10 提问;2022/02/10 11:28 回答}

\noindent[18.]{\Hei 答}:這些新創的核聚變公司已經來晚了:今年美聯儲加息之後,股市的狂歡結束,他們大部分會無疾而終,只有我提過那幾個在去年底就大撈一票的,還有可能再熬上幾年。
\\

\textit{\hfill\noindent\small 2022/07/21 13:24 提问;2022/07/22 00:17 回答}

\noindent[19.]{\Hei 答}:是的;推崇電動車的人,以前就有,但最早明白指出這場產業革命對國際地緣政治有重大影響的,的確是這裏。事實上,特殊管道的聯絡人在看到博客那條留言評論之後,理解其重要性,特別指定要求寫入後來邀稿的正文(參見博文《2022年國際局勢的回顧與展望》)之中。

很不幸的,在大衆輿論場上,我早年的正確評論有些被忽略,以致謬誤的胡猜以訛傳訛、至今不衰。這裏我指的是世界主要工業國家之中,只有日本全力投入氫汽車路綫的決定:大陸有很多人腦補聯想為獨占專利的影響,其實2010年代早期豐田曾經是Tesla的最大股東,而且計劃很快推出自己的鋰電池汽車,因此他們在電動車方面的專利一樣領先。更早,半導體的專利也沒有阻止產業從美國轉移到日本、然後從日本轉移到韓國和台灣。這裏全世界選擇電動車而不是氫汽車的理由,和專利一點關係都沒有,純粹就是技術的優劣問題。
\\

\textit{\hfill\noindent\small 2023/03/11 15:45 提问;2023/03/12 03:35 回答}

\noindent[20.]{\Hei 答}:這種靠著專業術語來做的忽悠,不是行内人很難看破。博客固然是公共論壇上唯一的可靠實話來源,但即使執政者願意參考,最終還是必須有體制内的專家挺身出來確認結論,所以此事的真正關鍵在於這些人的良心;不過他們必然不會公開說得罪利益團體的話,所以也不用着急,我們可以持著審慎樂觀的態度繼續觀察。
\\


\section{【金融】【战略】再谈AIIB}
\subsection{2015-03-12 22:44}


\section{13条问答}

\textit{\hfill\noindent\small 2015/03/17 00:00 提问; 回答}

\noindent[1.]{\Hei 答}:正是。\\

\textit{\hfill\noindent\small 2015/03/18 00:00 提问; 回答}

\noindent[2.]{\Hei 答}:已经是金砖银行的大股东。 AIIB只留了25\%给亚洲外国家。而且最重要的是俄国与欧美交恶,加入了AIIB,反而会对AIIB的扩张有不利影响,不符合中共的利益。

现在大势已定,今天才传出俄国加入的消息。这时机掌握得有意思。\\

\textit{\hfill\noindent\small 2015/03/21 00:00 提问; 回答}

\noindent[3.]{\Hei 答}:大形势可以预判,转机发生的时间则有很多随机因素,很难预先确定。这次的转变的确是比任何乐观的估计还好。\\

\textit{\hfill\noindent\small 2015/03/24 00:00 提问; 回答}

\noindent[4.]{\Hei 答}:安倍和中共是不共戴天的,只是在企业家的压力下,必须保持一定程度的友好假相,以减低对自己的损失。肯对现实低头,才是像样的政治谋略。

台湾的新闻界和美国一样,是愚化民眾的工具。我只希望有自尊心的编辑和记者能偶尔来读读我的部落格。\\

\textit{\hfill\noindent\small 2015/03/24 00:00 提问; 回答}

\noindent[5.]{\Hei 答}:我写这些文章的首要目的是希望普及对国际事务的认知,所以请大家广为传播这些常识。我不是指替我的部落格做广告,而是在看到幼稚的社论时,请发言更正。台湾如果要增进集体智商,光是几千个人在部落格上聊天是不够的;必须有像样的报社主编和记者才行。\\

\textit{\hfill\noindent\small 2015/03/26 00:00 提问; 回答}

\noindent[6.]{\Hei 答}:中共在外交上走的是王道,和欧美传统的霸道不同。例如美国只有IMF16\%的股权,却拥有否决权;而中共有近50\%的AIIB股份,却自愿放弃否决权。只说中共在\&ldquo;争老大\&rdquo;,对事实真相是太简单化了。\\

\textit{\hfill\noindent\small 2015/03/26 00:00 提问; 回答}

\noindent[7.]{\Hei 答}:中共是当今世界中,唯一一个有足够纪律和意愿来对政治、社会、金融、经济做根本性整顿和改革的主要国家。很多问题是普世性的,美国的宣传却说它们是中共制度的结果;美国自己的制度和理念有很多疵瑕,他却说是有普世性的。我们看这些事,要从事实和逻辑出发,对宣传广告要有免疫性。\\

\textit{\hfill\noindent\small 2015/03/27 00:00 提问; 回答}

\noindent[8.]{\Hei 答}:美国进了也只会搅局。有国会作梗,欧巴马也提不出像样的交换条件,连调整IMF的股份都通不过。应该就把美日拒止在AIIB之外,不过不知有没有这魄力。

加拿大或者会先申请,那么AIIB的北美分部设在Toronto就行了。\\

\textit{\hfill\noindent\small 2015/03/28 00:00 提问; 回答}

\noindent[9.]{\Hei 答}:这两天美国传出新闻,证实欧巴马有考虑过加入,但因国会不可能通过而作罢。\\

\textit{\hfill\noindent\small 2015/04/01 00:00 提问; 回答}

\noindent[10.]{\Hei 答}:亜投行必然会以人民币为主要货币。

台湾的基础建设已经不错了(除了地铁之外),本身又没有能在国际上竞争的重工业,亜投行不是要点,自由贸易协定才是当务之急。不过这么明显简单的道理,愚民不想懂,你也没办法。\\

\textit{\hfill\noindent\small 2015/04/02 00:00 提问; 回答}

\noindent[11.]{\Hei 答}:加入了也没坏处就是了。TPP就不一样,加入后就成了美国的奴才。\\

\textit{\hfill\noindent\small 2015/04/02 00:00 提问; 回答}

\noindent[12.]{\Hei 答}:欧洲已经选边站了,中方就不在乎美国的态度。其实美日不加入,反而省了不少麻烦。

AIIB是替代世银的。IMF的SDR若是不改,中共大不了再建另一个替代性组织;欧洲已经学乖了,应该会马上加入,美国的损失只有越来越重,这真的像是九阳神功反震外敌。\\

\textit{\hfill\noindent\small 2015/08/13 00:00 提问; 回答}

\noindent[13.]{\Hei 答}:台湾的名嘴能有6岁智商的,百中无一,没有什么好讨论的。\\


\section{這個部落格}
\subsection{2015-04-12 20:52}


\section{5条问答}

\textit{\hfill\noindent\small 2015/10/25 00:00 提问; 回答}

\noindent[1.]{\Hei 答}:我知道现在的网页上,已经可以完全复制Word的所有功能,但是中时的程式员要打混,我也无法可施。

当影视明星或政客的,这种造势是理所当然的本行。但是经济学者是所有政府幕僚中最重要的,他们的建议和执行,可以造成几万亿元的差别,这样塑造名不符实的声势就是埋藏核子地雷了。我有空再详述吧。\\

\textit{\hfill\noindent\small 2015/10/26 00:00 提问; 回答}

\noindent[2.]{\Hei 答}:或许吧。我不在乎了。\\

\textit{\hfill\noindent\small 2015/10/26 00:00 提问; 回答}

\noindent[3.]{\Hei 答}:我老是奇怪,为什么没有其他志同道合的人一起出来揭穿美国财团过去40年的劣迹。唉,现在也衹能教育一个算一个。\\

\textit{\hfill\noindent\small 2015/10/26 00:00 提问; 回答}

\noindent[4.]{\Hei 答}:谢谢。我原本没有计划写那个系列,纯粹是和老同学网上聊天后,一时兴起写的,结果欲罢不能,写了五篇才写完。

后来另一个在美国工作过,但已回到臺湾的老同学反馈说这类揭穿美国内幕的讨论在臺湾特别难得一见,所以就引发了以【美国】为标题的后续文章。\\

\textit{\hfill\noindent\small 2023/01/24 13:06 提问;2023/01/25 01:19 回答}

\noindent[5.]{\Hei 答}:我對中國社科學術界不熟,不能確定你的描述是否精確,不過應用在1960年代後的美國社科界是完全貼切的。現代社科界量化分析的潮流始於經濟學,而其鼻祖是MIT的Paul Samuelson(亦即Larry Summers的伯父)。Samuelson本人雖然和Friedman同代,倒不像後者那樣是100\%財閥代言人;他引入數學模型,似乎並沒有惡意,而只是爲了追求新奇、方便發表論文。然而即使是高能物理這樣的自然科學,脫離現實的數學模型(超弦是典型)一樣成爲無意義的玄學空談;社科議題絕對不對應著簡單的數學模型,那麽要在論文中建立後者,就必然要脫離現實。換句話説,自然科學裏的數學模型還有好壞之分,社科界也搞數學模型,除了教學和發假大空論文方便之外,不可能有實際意義。所以美國社科界在過去半個世紀極度追求量化分析,有兩個基本推動力:除了學人内發的“Publish or Perish”考慮之外,也有外來的資本財閥要腐蝕學術界的意圖。畢竟深刻正確的邏輯辯證,自然會揭穿新自由主義的謊言,那麽鼓勵學術主流去搞量化玄學,就是幫助抹殺淹沒實話的必要方向。

數學在社科的正確運用,在於統計資料庫的建立和過濾;這是因爲未被虛僞模型假設所扭曲的統計結論,是做任何邏輯分析的事實基礎。博客討論過幾個常見的統計悖論,但還有更多沒有提及的細節必須小心應用,因此若要建立龐大完整的Database,在學術人力的質和量上都有很高的要求。不過我所知的社科人物,都遠遠沒有足夠的統計學常識,完全談不上過濾數據;再加上收集資料一點也不Sexy,無法反復發論文,所以自然無人問津。很不幸的,獨立自主的資料庫,正是國際政治經濟話語權的基礎;任何基於西方資料庫的研究,必然會自動包含他們有意無意埋下的扭曲和偏見。我反復提倡中方重視這方面的投入,但中國學術界似乎同時兼有著崇外和論文至上兩個迷思,如果沒有最高層的强力糾正,不可能有所改變。
\\


\section{【战略】【经济】帝国大反击}
\subsection{2015-04-16 21:11}


\section{6条问答}

\textit{\hfill\noindent\small 2015/04/16 00:00 提问; 回答}

\noindent[1.]{\Hei 答}:原本美国人条件开得很苛刻,我是不赞成台湾自愿成为美国的经济附庸的,必须戒急用忍。还好后来连韩国都得等,台湾更没有本钱当创始国。

到年中TPP签了,细节也公布了,如果像我所料,美国做了一些新的让步,那么台湾自然可以加入,只不过TPP对台湾的好处,絶对还是比不上和大陆的自贸协定。继续拒签服贸,大陆再不高兴,暂时也不能怎么样。台湾这几年对中共伸中指的次数还少了吗?等武统成了选项,台湾人见了棺材,应该就会掉泪了。\\

\textit{\hfill\noindent\small 2015/04/17 00:00 提问; 回答}

\noindent[2.]{\Hei 答}:1.我同意,正因为东亚已经形成跨国產业链,大陆才是台湾外贸的关键,TPP的好处是很有限的。
2.是啊,愚不可及。美国连乌克兰都不敢出手,怎么有可能为台湾卖命?台湾对美国来说,只是牵制中共的卒子,在大国博弈的棋局里,随时可以被牺牲。
3.从被优遇到被无视,台湾人割自己的喉咙已经成功了一半。只是这个割喉是慢动作的,还有十几年可以拖。
4.既然美国人急着要TPP,那么只要让点利自然可成。RCEP则比较难说;我若听到风声再写吧。\\

\textit{\hfill\noindent\small 2015/04/17 00:00 提问; 回答}

\noindent[3.]{\Hei 答}:像巴基斯坦、泰国、缅甸和斯里兰卡这些国家,它们的问题都在于政府不稳,政策随时可能转向。像高棉和寮国比较稳定些,投资风险小。\\

\textit{\hfill\noindent\small 2015/04/17 00:00 提问; 回答}

\noindent[4.]{\Hei 答}:访问巴基斯坦,安全风险很大,CIA说不定有什么特别计划等着他。\\

\textit{\hfill\noindent\small 2015/04/17 00:00 提问; 回答}

\noindent[5.]{\Hei 答}:亚投行或许无关紧要,TPP却是个自贸协定,正是台湾最需要的。不过台湾最大的贸易伙伴不在TPP之中。。。\\

\textit{\hfill\noindent\small 2015/04/17 00:00 提问; 回答}

\noindent[6.]{\Hei 答}:我想这还言之过早。现代国际社会的制度是西方制定的,秩序是西方维持的,中国式的行为准则,外国人往往不能理解,更谈不上崇拜追求。

澳洲总理前天说他对中国交涉是摇摆于Fear和Greed之间;Fear和Greed其实传统上是华尔街描述股市的用语。当前中国的外交,基本上还是以利害关系为动力;要说到软实力,还需要一段时间。\\


\section{【战略】【经济】絶地大反攻}
\subsection{2015-04-18 19:49}


\section{1条问答}

\textit{\hfill\noindent\small 2015/04/21 00:00 提问; 回答}

\noindent[1.]{\Hei 答}:我对CIPS不熟,不过印象中它是一个Back Office的货币结算通道,和SWIFT这种Front Office用的汇兑通讯系统是不一样。

CIPS在美元和欧元似乎没有应对的类似机构,它的存在来自人民币受管制却又必须内外流通的这个矛盾。之前,人民银行已经和一些国家签了SWAP,但是若要让人民币进IMF的SDR,它必须能在所有国家都进行结算才行。所以CIPS是为了SDR而办的。

AIIB对世银是一个严重打击,对IMF则是杀鸡儆猴,现在IMF自己急着要绕过美国,以避免把中国逼上梁山。在年底之前,人民币应该就能进SDR,所以人民银行必须配合它,赶快把CIPS办成。\\


\section{【台湾】2020年前的台海战役}
\subsection{2015-04-23 05:24}


\section{4条问答}

\textit{\hfill\noindent\small 2015/09/22 00:00 提问; 回答}

\noindent[1.]{\Hei 答}:我在留言栏简单回答过,大陆买美债是累积外匯的自然结果,而外匯在美元称王的世界里是防御美国金融打击的战力,必须有一定程度的累积,所以整体来说大陆过去20年的外匯政策并没有大错。\\

\textit{\hfill\noindent\small 2016/08/01 00:00 提问; 回答}

\noindent[2.]{\Hei 答}:日本的一个公家退休基金自安倍上臺,开始投资到股市,已经赔掉了GDP的1.2\%。再过十年,所有公私退休基金总共赔掉GDP的50\%都不奇怪。

现在全球利率都接近于0,对保险公司和退休基金来説是灾难性的大环境。其实我去年就开始怀疑臺湾有保险公司是Dead Man Walking,衹不过做假账是臺湾人的老本行,所以再拖个3年5年都是理所当然,也就一直没有写文章来讨论。奉劝住在臺湾的读者,千万不要贪心,衹要回报保证超过每年1.5\%的金融產品,还是不碰为妙。\\

\textit{\hfill\noindent\small 2016/08/01 00:00 提问; 回答}

\noindent[3.]{\Hei 答}:我一直觉得中国的GDP数据,很可能低估了生活水平和发展程度两倍以上。可惜这种事不可能有明确的证据,所以不能写在正文里面。\\

\textit{\hfill\noindent\small 2016/08/02 00:00 提问; 回答}

\noindent[4.]{\Hei 答}:衹是我个人对金融环境的猜测,不一定正确。\\


\section{【台湾】听其言而观其行}
\subsection{2015-04-26 08:08}


\section{1条问答}

\textit{\hfill\noindent\small 2016/04/06 00:00 提问; 回答}

\noindent[1.]{\Hei 答}:美国国内有好几个州也搞避税逃税的把戏。抓这些避税天堂,主要是肥水不落外人田,有好处由本土的律师享受。\\


\section{【工业】中共国营企业的改革}
\subsection{2015-04-29 02:06}


\section{4条问答}

\textit{\hfill\noindent\small 2015/04/29 00:00 提问; 回答}

\noindent[1.]{\Hei 答}:我想这是谣言。贪脏的人虽然看来是无中生有,其实是必须冒精神上的高度紧张的,所以他们一般不愿意再冒其他的険,现金交易,不赊不欠,这是人性的常情。玩股票总有点不可控性。\\

\textit{\hfill\noindent\small 2015/04/29 00:00 提问; 回答}

\noindent[2.]{\Hei 答}:金融业对美国的贡献不在税收,而在独断权。美元的利率和印行决定全球银根的宽松,但却是由美国一家决定的。各式金融交易,往往得靠美国银行来做中介。此外,跨国银行对开发中国家的掠夺成果,也有大部分送回美国。

在台湾一样是必须随波逐流,你没注意到我是退休了在美国吗?我家母亲那边就是典型的台南本省人;若是我住在台湾,光亲戚朋友的人情,就足够让我闭嘴。\\

\textit{\hfill\noindent\small 2015/04/29 00:00 提问; 回答}

\noindent[3.]{\Hei 答}:大陆现在和20世纪早期的美国很像,监管机关还没有建立完成,所以各式各样的假新闻很多。其实全世界对操弄股市的假新闻,执法都很松,只有美国偶尔打打苍蝇,老虎是没人打的。

我觉得这个假新闻是一个炒股的财主搞的,技巧很不错,因为他的预言方向是对的,只在程度和时机作假,而炒股赚钱最难的就在时机。\\

\textit{\hfill\noindent\small 2015/04/29 00:00 提问; 回答}

\noindent[4.]{\Hei 答}:是利润占美国所有企业的40\%;金融界逃税是本行,税交得很少。

美国不是讨厌意外,而是讨厌它不能控制的事。它打压中共,是以不损自身为前提;若是卒子做了什么蠢事把它扯进去了,那还得了。

美国现在的政略就是送日本、台湾、菲律宾和越南去死,然后可以名正言顺地\&ldquo;制裁\&rdquo;中共。中日若开仗,日本的财政必然崩溃,不过到时它要担心的事还多着呢。\\


\section{【政治】中共在欧洲的最好朋友}
\subsection{2015-05-02 05:04}


\section{1条问答}

\textit{\hfill\noindent\small 2015/05/06 00:00 提问; 回答}

\noindent[1.]{\Hei 答}:张教授对历史的分析有独到之处,但是对未来的预测我不赞同。

在英国正式放弃全球霸权的梦想(1956年苏伊士危机)后近60年来,伦敦的金融中心地位带给英国极大的利益,到现在金融业已是英国经济的支柱(没有之一)。但是它正面临法兰克福和巴黎这种欧元区的金融城市的挑战,如果英国退出欧盟,伦敦的金融地位将会遭受严重的打击。英国正在放弃当全球二流强权的梦想,军费在过去五年中被砍了将近1/3;再度崛起在军事和金融上都是不可能的,只要伦敦不被欧洲大陆城市取代就算好的了。

财政部部长楼继伟在清华给的演讲才是最近最重要的文章,不可不细读。其中他提到放松粮食自给自足的要求,加强与阿根廷的农业合作;我觉得这是很有自信的正确方向。在国内,应该同时加强科学教育和政府监管,以求审慎合理地推广转基因作物。
\\


\section{【金融】中国燕子}
\subsection{2015-05-03 16:37}


\section{36条问答}

\textit{\hfill\noindent\small 2015/05/03 00:00 提问; 回答}

\noindent[1.]{\Hei 答}:I think Bitcoin is just for criminals.

Central banks were invented because free currency did not work, not because the government wanted to steal money. That was invented later.\\

\textit{\hfill\noindent\small 2015/05/03 00:00 提问; 回答}

\noindent[2.]{\Hei 答}:中文的金融术语我不熟;\&ldquo;支付\&rdquo;到底是对应哪一个英文字,我那时不知道。术语和口语不同,不能凭一般翻译决定。人民银行文章里的英文只说是CIPS,我没有仔细想那是什么的缩写,所以就从文章内容来推断。别忘了,那篇文章有三年之久基本上就是有关CIPS唯一的公开资料。

Payment System送一个短讯,里面说我的某某客户要匯多少钱给你的某某客户。然后两个银行各自进行确认和准备之后,用Clearing System进行实际的匯兑。\\

\textit{\hfill\noindent\small 2015/05/03 00:00 提问; 回答}

\noindent[3.]{\Hei 答}:做金融的,一般的确没有上网聊天的习惯。\\

\textit{\hfill\noindent\small 2015/05/03 00:00 提问; 回答}

\noindent[4.]{\Hei 答}:我个人觉得像「支付宝」这样的非正规金融机构和美元霸权没有关系,反而是中国金融界的隐患。中国的融资过于依赖银行,必须发展替代性的融资手段,但是如何监管这些新的融资通道,将是一大难题。目前李克强还没有着手处理这事,但是10年内必成疾患。\\

\textit{\hfill\noindent\small 2015/05/04 00:00 提问; 回答}

\noindent[5.]{\Hei 答}:SDR是个合成货币,里面含着各种不同百分比的美元、欧元、日元等等。目前的应用局限在IMF的贷款上,并不很常见。它的重要性在于如果国际上要正式采用一个全球货币来替代美元,那么SDR是最佳选择。

目前人民币加入SDR,除了象征性的地位,也等于国际正式承认人民币是个自由流通的货币,美国要阻止人民币推广的主要藉口就不见了。\\

\textit{\hfill\noindent\small 2015/05/04 00:00 提问; 回答}

\noindent[6.]{\Hei 答}:如你所说,中国的国际金融地位是建筑在硬件上的,而美国的反击是建筑在软件上。TPP不只是美日澳的经贸同盟,它同时采纳了很多有关汇率管制、国营企业、智慧产权、环保和跨国企业的新限制,基本上美国和中国有什么不同的制度,它就说中式的是不合法的。中方现在已放弃日本,澳洲则不可能脱离中国的引力轨道,所以TPP做为一个围堵性的经贸同盟不会有大作用。但是在制定新国际贸易规则上,TPP反而才是个大隐患;我还没看到中方的反制措施。

美国最糟也就是成为另一个阿根廷,不过几率很小。美国菁英真正担心的是成为另一个英国。\\

\textit{\hfill\noindent\small 2015/05/04 00:00 提问; 回答}

\noindent[7.]{\Hei 答}:股市飞涨是政策的选择;既然房地产和债卷都没有像样的收益,自然要给百姓的存款一个出路。

第三代和第四代的核能都是值得做的,只是最近欧美在大工程上都无法控制成本,最终还是要看中国能不能搞出物美价廉的东西。\\

\textit{\hfill\noindent\small 2015/05/05 00:00 提问; 回答}

\noindent[8.]{\Hei 答}:OK,这种Commercial Banking的事,不是我的专业。我想这里Settlement的确比Clearing合适(股票方面,Settlement和Clearing一般是一起做的)。多承指教。

不过BIS不是负责Settlement的(原本在1930年代刚成立时是的,现在不是了)。现代的Settlement是银行与银行间一对一地进行,并没有统一的机构。\\

\textit{\hfill\noindent\small 2015/05/05 00:00 提问; 回答}

\noindent[9.]{\Hei 答}:股票是我的本行,我可以确定说是和SWIFT不一样的。所有的股市都是Double Auction,SWIFT是一对一的Messaging System。\\

\textit{\hfill\noindent\small 2015/05/06 00:00 提问; 回答}

\noindent[10.]{\Hei 答}:QE是美国近年来新发明的手段,究竟如何收尾还没有先例;不过我想要把几万亿收回来,要等美元大幅贬值以后才划算。

日本的国债欠的是国内的储蓄户(邮局和银行收集存款,然后买国债),可以用政治手段压着,所以就一直拖下去了,越养越大;将来爆发时也将更不可收拾。其实中国地方政府的债也是一样的,欠的都是国有的银行,所以不会马上爆发;不同的地方在于中共政府有能力和意愿来解决这个问题。\\

\textit{\hfill\noindent\small 2015/05/24 00:00 提问; 回答}

\noindent[11.]{\Hei 答}:我退休以后已经不太管金融的事,有关经济的,我主要看Economists(经济学人,有中文版吗?)和Google(我有十几个专门的搜索题材)。很抱歉,中文媒体我不熟。

自我写《谈油价》以后,新出来的消息是美国的页岩油生產成本降得比预期的快一些,所以產量缩减得慢一些,再加上过去几年投资的传统原油產能还在不断上线,而沙特是说什么也不减產,所以油价恢復了一阵后又下去了。未来这一年,油价应该还是在\$50-\$60的范围附近震盪,明年中期视需求和產量的新消息才有可能突出。\\

\textit{\hfill\noindent\small 2015/08/05 00:00 提问; 回答}

\noindent[12.]{\Hei 答}:他们会等人民银行自己也准备让人民币贬值的时候顺水推舟,刚好中国金融界有人在鼓噪着要贬值。

经过过去这几个月观察中国股市,我对中共的金融管理人员十分失望。他们基本上没有经验,太容易受国际资本的骗。我想这是因为很多官员是留英留美学经济的,却没有真正在华尔街打过滚。\\

\textit{\hfill\noindent\small 2015/08/05 00:00 提问; 回答}

\noindent[13.]{\Hei 答}:我的看法是中国又要流动性又要控制\&ldquo;合理\&rdquo;价位,根本是自相矛盾,屁股被狠踢是自然的。\\

\textit{\hfill\noindent\small 2015/08/06 00:00 提问; 回答}

\noindent[14.]{\Hei 答}:我没有意见。视李克强如何决定。\\

\textit{\hfill\noindent\small 2015/08/07 00:00 提问; 回答}

\noindent[15.]{\Hei 答}:是的。置常识不顾,却相信美国帮国际资本发展出来哄骗老百姓的那一套,看来大陆一流人才都进了自然科学,学经济的都没有天分,只会人云亦云。\\

\textit{\hfill\noindent\small 2015/08/11 00:00 提问; 回答}

\noindent[16.]{\Hei 答}:胡春华的任命还没有确定,中共的核心最好三思。\\

\textit{\hfill\noindent\small 2015/08/11 00:00 提问; 回答}

\noindent[17.]{\Hei 答}:台湾人对缺水的反应是耸耸肩,对断电的反应是殴打修理员,都是两岁以下智商的表现。\\

\textit{\hfill\noindent\small 2015/08/11 00:00 提问; 回答}

\noindent[18.]{\Hei 答}:Joe Francis不是我。

巴菲特买的那家公司不是一般的制造业,而是为航空、航天、石油工业提供零件的头号供应商,这些產业从来就没有离开美国过,在未来20年也不可能被消灭(尤其是美国的页岩油工业正方兴未艾),所以巴菲特的策略并不代表为\&ldquo;美国再工业化\&rdquo;背书。\\

\textit{\hfill\noindent\small 2015/08/11 00:00 提问; 回答}

\noindent[19.]{\Hei 答}:这只是刚开始,怎么收尾我不确定。7.8是有可能的,但是我觉得可能性还不到50\%。我们再等等看吧,或许我只是看得不够远。\\

\textit{\hfill\noindent\small 2015/08/11 00:00 提问; 回答}

\noindent[20.]{\Hei 答}:我对中共金融决策核心没有特别的了解,不过你的解释是合理的。\\

\textit{\hfill\noindent\small 2015/08/11 00:00 提问; 回答}

\noindent[21.]{\Hei 答}:怎么会笨成这个样子?难道李克强的经济幕僚都是芝加哥毕业的?\\

\textit{\hfill\noindent\small 2015/08/11 00:00 提问; 回答}

\noindent[22.]{\Hei 答}:越来越觉得李不够格做习的团队。\\

\textit{\hfill\noindent\small 2015/08/11 00:00 提问; 回答}

\noindent[23.]{\Hei 答}:贬值还不一定是错的,要看后续的控制如何。放弃控制却绝对是错的,是自废武功。\\

\textit{\hfill\noindent\small 2015/08/11 00:00 提问; 回答}

\noindent[24.]{\Hei 答}:李克强的执行力越来越值得怀疑。工业政策的细节我不知道,但是操弄股市以刺激经济本来就很明显是一招笨棋,因为它的受益者主要是大户,拿了钱就拍拍屁股走路,远不如直接为低级公务员加薪一举两得;怎么会有如此笨拙的幕僚群还能当到总理?我想共青团的制度太优惠,升迁太快而没有足够的执政经验,是胡和李的根本问题。\\

\textit{\hfill\noindent\small 2015/08/12 00:00 提问; 回答}

\noindent[25.]{\Hei 答}:希望这个所谓的浮动,只是口舌之惠。\\

\textit{\hfill\noindent\small 2015/08/12 00:00 提问; 回答}

\noindent[26.]{\Hei 答}:习近平已经有很多内部的敌人,打破规矩会给人很大的口实。不过经济第一,如果李克强的表现持续低迷,那也叫无可奈何。\\

\textit{\hfill\noindent\small 2015/08/12 00:00 提问; 回答}

\noindent[27.]{\Hei 答}:其实这件事全看掌权的人水准如何。以往我一般假设中共当局有足够的专业知识和态度来选择正确的策略,经过这次股市的问题才失去信心。\\

\textit{\hfill\noindent\small 2015/08/12 00:00 提问; 回答}

\noindent[28.]{\Hei 答}:那么就得撤换李克强,这可能有困难。\\

\textit{\hfill\noindent\small 2015/08/12 00:00 提问; 回答}

\noindent[29.]{\Hei 答}:很不幸地,现代复杂分工的社会的确如此。我以前的大学同学有做半导体的,我总觉得那才是真正从事生產;可是金融不搞好,不但资金不到位,连基本秩序可能都没有了。\\

\textit{\hfill\noindent\small 2015/08/12 00:00 提问; 回答}

\noindent[30.]{\Hei 答}:华尔街自己交易所根据的技术分析(Technical Analysis)用的是完全不同的一套东西;这些平均线之类的,是用来对小客户做广告的。\\

\textit{\hfill\noindent\small 2015/08/12 00:00 提问; 回答}

\noindent[31.]{\Hei 答}:好坏的差别完全在于是否对大鱷能做有效管控,这是国务院财经团队的政策细节,我没有比大家更深入的资讯或看点。

只希望李克强已经学了教训。\\

\textit{\hfill\noindent\small 2015/08/13 00:00 提问; 回答}

\noindent[32.]{\Hei 答}:不适合这个部落格。\\

\textit{\hfill\noindent\small 2015/10/09 00:00 提问; 回答}

\noindent[33.]{\Hei 答}:我还是喜欢最简单的解释:美元坚挺,人民币有强大的贬值压力,所以先退三步到合理的价位,再准备坚守阵地。\\

\textit{\hfill\noindent\small 2015/12/12 00:00 提问; 回答}

\noindent[34.]{\Hei 答}:低阶制造业必然还是得让出来的;其实这个过程已经开始了。

国家之间,优胜劣败,在所难免。中国主导的世界至少是公平竞争,不像美国财阀只想着不劳而获。既然中国主导下的国际关系基础是公平诚实的,那么就可以追求更大的整体效率,使世界经济规模成长得更大。这虽然不是人间天堂,至少比美国为一己之私,压榨全球财富,要好得多了。

Triffin Dilemma要求同时保持贸易顺差和国际收支(Balance of Payments)逆差,这其实很容易满足,中共衹要拿贸易赚的钱到海外搜购优质资產就行了。美国在50-60年代就是这么做的;到70年代之后,财阀主政,为了加大贫富差距,自我消灭了制造业,这才导致贸易逆差。\\

\textit{\hfill\noindent\small 2018/10/20 21:29 提问;2018/10/21 05:34 回答}

\noindent[35.]{\Hei 答}:這當然是一個好的開頭,但是美國必然會强力阻攔,一個有美軍駐扎的德國是否有那個膽量,還很難説。
\\

\textit{\hfill\noindent\small 2022/02/08 01:46 提问;2022/02/08 05:24 回答}

\noindent[36.]{\Hei 答}:俄國並沒有被踢出SWIFT;2014年那一輪制裁針對的是部分銀行和個人。
\\


\section{【台湾】一件小新闻}
\subsection{2015-05-08 21:49}


\section{4条问答}

\textit{\hfill\noindent\small 2015/12/03 00:00 提问; 回答}

\noindent[1.]{\Hei 答}:对中国的经济成长,我的确是比主流经济学界乐观些;不过经济学界对成长力的预测结果,向来和掷飞镖的猴子差不多。\\

\textit{\hfill\noindent\small 2015/12/04 00:00 提问; 回答}

\noindent[2.]{\Hei 答}:现代的世界事务,尤其是经济方面,越来越专业,很多\&ldquo;实际行动\&rdquo;的利弊非常复杂,再加上有心人在宣传上的扭曲,光做不说反而会吃闷亏。

媒体上的努力衹是治标,治本必须从培养训练国内国外的菁英做起;像是美国洗脑一个马英九的效应,远比几千个头条新闻还深远。\\

\textit{\hfill\noindent\small 2015/12/05 00:00 提问; 回答}

\noindent[3.]{\Hei 答}:不衹是自由主义经济学是如此,全世界做超弦的十几万人中,比较好的那一半说超弦无法以实验检验,但是物理界没有其他的理论(不是真的);另一半则说以往对科学的定义已经过时,新的科学就是像超弦这样不能被证伪、甚至不能被定义(超弦没有基本方程式,没有基本定义,一切计算都从一个天上掉下来的近似方程式开始)的东西。要知道一个教授属于哪一半,衹要问他相不相信Multiverse(多重宇宙)就行了;凡是答相信的,就是后一半。\\

\textit{\hfill\noindent\small 2015/12/10 00:00 提问; 回答}

\noindent[4.]{\Hei 答}:其实超弦界的态度是一模一样的。

他们集体疯狂之后,滥用新教授的雇用权和论文的审查权,于是劣币驱逐良币,最终是整个高能理论界被绑架一同跳崖。臺湾的政坛也是如此。\\


\section{【政治】【经济】民主政治与自由经济}
\subsection{2015-05-11 17:14}


\section{14条问答}

\textit{\hfill\noindent\small 2015/05/12 00:00 提问; 回答}

\noindent[1.]{\Hei 答}:不是每个人都熟悉统计,而数学还是用黑板解释比较容易,所以我在这里就只粗浅地提了CLT。

演讲在台北。\\

\textit{\hfill\noindent\small 2015/06/01 00:00 提问; 回答}

\noindent[2.]{\Hei 答}:我非常赞同他的资本税建议;《二十一世纪资本论》已经明确地表明资本成指数累积的先天性不稳定,长期来看,财阀必然是人类的最大最后公敌,只有强力的政府才能可能为民与财阀进行对抗。所以我只有两个实践上的问题:

1. 如何避免资本流失到国外?
2. 改革税政是必要的,但是把和会计细节有盘根错节关系的税法写到宪法上,有现实上的可能性和必要性吗?\\

\textit{\hfill\noindent\small 2015/06/01 00:00 提问; 回答}

\noindent[3.]{\Hei 答}:1. 我不认为资本有劣质和优质之差,大家都寻求最高的投资报酬率。
2. 很难。
3. 正是我所怀疑的。\\

\textit{\hfill\noindent\small 2015/06/03 00:00 提问; 回答}

\noindent[4.]{\Hei 答}:关门打狗是美国人的对策,但是不太成功。

我仍然觉得税政改革再加上削弱地方山头就够了,宪改太难也没有必要。\\

\textit{\hfill\noindent\small 2015/09/20 00:00 提问; 回答}

\noindent[5.]{\Hei 答}:税政真不是我的专业,请你直接去问虞先生吧。\\

\textit{\hfill\noindent\small 2015/09/21 00:00 提问; 回答}

\noindent[6.]{\Hei 答}:这些新政策是刘鹤的,他是习的智囊,所以政策是会推到底的。只希望配套的监管机构比证监会要清廉有效些。\\

\textit{\hfill\noindent\small 2015/09/21 00:00 提问; 回答}

\noindent[7.]{\Hei 答}:你贴吧。我没有更好的地方,连自己要贴个留言守则都没办法。\\

\textit{\hfill\noindent\small 2015/09/23 00:00 提问; 回答}

\noindent[8.]{\Hei 答}:我觉得在行销上,阿里巴巴是有真正的效率优势的,所以可以算是真正生產型的资本。

如果因为零售被整合到阿里巴巴,而能够统一收税,这是对国家财政的好事。\\

\textit{\hfill\noindent\small 2015/09/24 00:00 提问; 回答}

\noindent[9.]{\Hei 答}:这几篇文章是让大家参考而已,我自己不做评,也请大家不要再批评。\\

\textit{\hfill\noindent\small 2015/09/27 00:00 提问; 回答}

\noindent[10.]{\Hei 答}:我同意,不过我不是这方面的专家,还请大家到卢先生的网站讨论。\\

\textit{\hfill\noindent\small 2015/09/27 00:00 提问; 回答}

\noindent[11.]{\Hei 答}:除了国企方面之外,其他的税政也有改革的严重需要,不过李克强现在焦头烂额,就算有心也无力作为,何况原本就走错方向。\\

\textit{\hfill\noindent\small 2015/10/03 00:00 提问; 回答}

\noindent[12.]{\Hei 答}:这种\&ldquo;你的距离远,所以不懂\&rdquo;的歪论,我以前已经回答过了。你一辈子身处重力场,能写出爱因斯坦的重力方程式吗?生病的人懂得疾病,还是没病的医学专家?

真相只有一个。我有足够的经验和眼光,从事实出发,依逻辑做推论。如果你有异议,请提出不同的事实,或更正我的逻辑。避开这些不谈,找一些没有逻辑关系的论点,不但浪费大家的时间,而且徒然显示你在躲避真正的结论:你不但没有足够的能力发掘真相,连当真相被送到你眼前时都拒绝面对。\\

\textit{\hfill\noindent\small 2015/10/26 00:00 提问; 回答}

\noindent[13.]{\Hei 答}:的确是完全不切实际。要从根本打破贫富差距和资本垄断,衹有先解除美国霸权才有可能,所以光凭这点,中国就已经是全世界人民的未来希望所在,不过要小心不要自己也腐化了。\\

\textit{\hfill\noindent\small 2015/10/26 00:00 提问; 回答}

\noindent[14.]{\Hei 答}:美国政府已经是财团的马前卒,国际合作是不可能的。现在美国还在努力扩展财团的利益,TPP和TTIP就是实例。\\


\section{【战略】小国无外交}
\subsection{2015-05-17 08:26}


\section{1条问答}

\textit{\hfill\noindent\small 2015/05/17 00:00 提问; 回答}

\noindent[1.]{\Hei 答}:他的表现到现在还算中规中矩。胡温政权什么困难的决定都没做,十年下来积弊很深,要改革不是一两年的事。李克强至少大方向是对的,在执行上慢了些,大家应该有点耐心,毕竟他要处理的问题很多很困难。\\


\section{【工业】2025年的中国工业}
\subsection{2015-05-18 05:36}


\section{24条问答}

\textit{\hfill\noindent\small 2015/05/18 00:00 提问; 回答}

\noindent[1.]{\Hei 答}:我想印度要拿到一些低阶的制造业并不困难,把基础建设搞一搞也只是资金问题,印度的毛病都是慢性病,要到中级收入的程度才会完全体现。\\

\textit{\hfill\noindent\small 2015/05/18 00:00 提问; 回答}

\noindent[2.]{\Hei 答}:Contrary to American propaganda, American society has plenty of censorship. Please read my prior article 《言论自由的假相》. Also contrary to American propaganda, even the false sense of American freedom has very little to do with innovation. For example, prior to WWII, Germany was the authoritarian regime but also far more innovative, at least on a per-capita basis. For another example, in the 1980s, the Japanese were beating the Americans on multiple fronts of high-tech researches. Remember, Japan is as conformist as they come.

The reason is simple: the great majority of the population does not do innovation (Otherwise the US would have been in real trouble since its population is particularly stupid among industrialized nations); it is the top 0.01\% in IQ who is responsible for driving 99.99\% of the progress. It does not matter how much freedom those stupid majority has; only the smart few can really put their freedom to good use anyway. Their freedom is more often defined by being able to buy the right equipment at the right time. This is the effect of wealth. Once China can put enough investments on enough smart people, there is no reason why it cannot outdo the US in the innovation game.

Don't buy the American propaganda without thinking it through first. Use your own brain and follow the logic.\\

\textit{\hfill\noindent\small 2015/05/18 00:00 提问; 回答}

\noindent[3.]{\Hei 答}:可怜鸿海想挽救夏普,人家寧可独自等破產。\\

\textit{\hfill\noindent\small 2015/05/18 00:00 提问; 回答}

\noindent[4.]{\Hei 答}:我想大陆把这个叫做\&ldquo;傲娇\&rdquo;,如果只是他们自己受罪,我倒会是没有意见的,但是经济是一个联合的整体,连锁反应到最后总是穷人最承受不起。\\

\textit{\hfill\noindent\small 2015/05/18 00:00 提问; 回答}

\noindent[5.]{\Hei 答}:其实马英九第一任上有一个行政院长想找机床业老板座谈,他说他不去,因为他是绿营的。\\

\textit{\hfill\noindent\small 2015/05/18 00:00 提问; 回答}

\noindent[6.]{\Hei 答}:谢谢您的指教,主文已更正。\\

\textit{\hfill\noindent\small 2015/05/18 00:00 提问; 回答}

\noindent[7.]{\Hei 答}:政治挂帅,贫民饿死,愚不可及。\\

\textit{\hfill\noindent\small 2015/05/18 00:00 提问; 回答}

\noindent[8.]{\Hei 答}:其实已开发国家被新兴国家在技术台阶上被超赶时的标准反应是转入服务业,可是台湾人把服贸协定当洪水猛兽,这是典型的割喉自戕,可惜拿刀的手是绿卫兵,被割的喉咙却是无辜的贫民。\\

\textit{\hfill\noindent\small 2015/05/19 00:00 提问; 回答}

\noindent[9.]{\Hei 答}:常用就习惯了。\\

\textit{\hfill\noindent\small 2015/05/19 00:00 提问; 回答}

\noindent[10.]{\Hei 答}:日本人是世界上最排外的民族。他们没当成经济霸主是件好事。\\

\textit{\hfill\noindent\small 2015/05/19 00:00 提问; 回答}

\noindent[11.]{\Hei 答}:前面md51所说日本人因为旧有的优越感而心里不平衡,应该是这些日本名嘴们一起出洞抱怨的原因。

歷史上在二战后,美国已明显地成为新霸主,但还是到11年后的1956年,在金融上出手教训了英国后,后者才乖乖地退居二线。中国现在才刚超越日本几年,大于相当于一战后的英美局势,要日本软服还有很长的一条路。\\

\textit{\hfill\noindent\small 2015/05/19 00:00 提问; 回答}

\noindent[12.]{\Hei 答}:我想\&ldquo;中国制造2025\&rdquo;的重点不在总量而在高度。在机床方面,中方的企业可以做得出高精度產品,但是品质不稳定,没有市场竞争力;这是技术上的问题,必须靠钱+时间来解决。\\

\textit{\hfill\noindent\small 2015/07/27 00:00 提问; 回答}

\noindent[13.]{\Hei 答}:0到100km在4秒以下,就是超级跑车的级别。这些车价钱在美国居于\$25-110万之间,主要是Ferrari、Porsch还有一些更小的牌子,销量很少,Tesla抢他们的市场,没有太大的意义。

美国没有高铁,1000公里以内主要靠开车,名义续航力430公里还是远远不足以满足大眾需求。这个情势在3-5年不会有质的转变。\\

\textit{\hfill\noindent\small 2015/10/30 00:00 提问; 回答}

\noindent[14.]{\Hei 答}:我想这是给国臺办的方针,但是整体的对臺重视程度衹有越来越低。\\

\textit{\hfill\noindent\small 2015/10/30 00:00 提问; 回答}

\noindent[15.]{\Hei 答}:这次的衰退是全球性的,衹是中国不再能一枝独秀罢了。

东亚供应链整合程度很高;大陆承受逆风,臺湾这种胡搞的经济体就翻船了。\\

\textit{\hfill\noindent\small 2015/12/15 00:00 提问; 回答}

\noindent[16.]{\Hei 答}:连理工科本行都自愿政治挂帅,这岂不比文革还要糟糕?\\

\textit{\hfill\noindent\small 2017/08/01 00:00 提问; 回答}

\noindent[17.]{\Hei 答}:经营困难的真正原因,主管们往往不愿公开诚实解释,媒体的报导不能尽信,尤其是这种半页长的八卦式短聊。

自己犯的错,怪到政府头上,美国和臺湾也是所在多有的。\\

\textit{\hfill\noindent\small 2018/03/30 22:43 提问;2018/03/31 08:50 回答}

\noindent[18.]{\Hei 答}:
美國人想得很美,但是手段卻很低劣。
中國製造2025的主要目的是進口替代,美國人打關稅沒什麽大用。
\\

\textit{\hfill\noindent\small 2020/09/02 16:48 提问;2020/09/03 07:34 回答}

\noindent[19.]{\Hei 答}:我並不是說5G永遠沒用,而是它的用處沒有那麽急迫,偏偏早期的版本耗電太多,得不償失,所以最優的政策原本是慢慢部署,等待新技術解決耗電問題。但是美國把5G提升為鬥爭的焦點,全力打擊華爲,徹底改變了方程式,最優解成了順勢加速推動5G部署,以確保華爲的財務健康和技術優勢。
\\

\textit{\hfill\noindent\small 2021/04/12 17:57 提问;2021/04/13 04:32 回答}

\noindent[20.]{\Hei 答}:是啊,這我在2014、2015談一帶一路的時候,不是已經解釋過了嗎?中國的人口超過既有先進工業國家的總和,如果全球產能一下提升150\%,消費不可能跟得上,即使3、40年都遠遠不夠,因爲你還必須考慮科技進步、生產效率提升的效應。所以我們在改革開放到現在所經歷的,其實正是全球經濟的超大型“供給側”刺激。

美國經濟學界講供給側,是騙人的Voodoo理論,純粹只是爲了給減稅找口實,實際上嚴重加劇貧富不均。中國崛起,當然是比為美國富人減稅更正當得多的理由,但一樣加劇了歐美中產階級的衰退;即使在中國國内,如果不認真扶貧+打擊壟斷獨占,也會有嚴重的後果。
\\

\textit{\hfill\noindent\small 2022/10/11 10:38 提问;2022/10/11 10:54 回答}

\noindent[21.]{\Hei 答}:唉,對這些經貿制裁,中國是真正躺平了;要是對抗帝國主義霸權,有防治新冠1\%的精神,也不至於挨打無限升級至今。而且至少半導體方面還有找不到好回擊手段的藉口,人民銀行在匯率上的作爲才是難辤其咎。
\\

\textit{\hfill\noindent\small 2022/10/11 11:55 提问;2022/10/12 02:01 回答}

\noindent[22.]{\Hei 答}:那條新聞我也注意到了,不但至今沒有下文,即使本月就施行,依舊是Too little, too late. 專業人員應該有專業道德,亦即擇善固執,堅持對上位者提供精確、客觀、完整的專業意見,而不是依托過時的理論或政策為藉口,坐視國家整體利益被侵害。
\\

\textit{\hfill\noindent\small 2022/10/11 19:24 提问;2022/10/12 06:17 回答}

\noindent[23.]{\Hei 答}:是的,這些都是我在過去幾年説到爛的道理:就連半導體這種高科技貿易打擊,回擊也應該從美方的軟肋、亦即金融貨幣戰綫著手,結果在當前這個緊要關頭,第三世界全面揭竿而起,人民銀行居然選擇等同日本、英國、歐盟這些附庸國的躺平政策,幫助美元回血,真是讓人啼笑皆非。
\\

\textit{\hfill\noindent\small 2022/10/12 00:42 提问;2022/10/12 06:16 回答}

\noindent[24.]{\Hei 答}:官僚習性是普世現象。習近平整頓黨、政、軍是亟有成效的,繼續加强督促監管固然是正道,但我們必須理解這是逆水行舟,所以行政細節上永遠都會出現低效的不合理行爲。真正應該擔憂的,是思想學術界:事關正確大方向的選擇,對國運有著長期深遠的影響,而且剛好是過去40年腐爛最深的社會精英類別,習的十年任期又完全沒有做出任何針對性的改革,所以學術腐敗才是憂國憂民的知識分子應該專注的主要議題。
\\


\section{【工业】科技发展与美式自由无关}
\subsection{2015-05-19 01:59}


\section{11条问答}

\textit{\hfill\noindent\small 2015/05/19 00:00 提问; 回答}

\noindent[1.]{\Hei 答}:我想中国人(至少广东人、上海人和福建人)是天生的商人,有平移的机会自然不会错过,只要制度上允许就行了。

要发展基础科学,首重诚实。中国科学界目前真正的短板是学术道德,其他的只是时间和金钱的问题。\\

\textit{\hfill\noindent\small 2015/05/19 00:00 提问; 回答}

\noindent[2.]{\Hei 答}:组织力和执行力是今日中国的基本竞争优势,中共对此非常清楚。美国人却是说什么也不承认,所以才会一天到晚搞中国崩溃论:劳动人口老化了,必然会崩溃;劳动薪资提高了,低阶產业流失,必然会崩溃;房地產有泡沫了,必然会崩溃;股市涨得快了,必然会崩溃;等等。其实人口红利,廉价劳工,房產投资这些都是末节;中国经济的发展靠的是举世无双的组织力和执行力,只要这点没变,发展就会继续下去。\\

\textit{\hfill\noindent\small 2015/05/19 00:00 提问; 回答}

\noindent[3.]{\Hei 答}:我也是为了钱而待在美国的,所以很清楚。

近年来,美国签证越来越难拿,实在是美国国会拿枪打自己的脚。\\

\textit{\hfill\noindent\small 2015/05/19 00:00 提问; 回答}

\noindent[4.]{\Hei 答}:谢谢你供献的论点。我住美国,对这些\&ldquo;盟友\&rdquo;被洗脑的过程只有原则性的了解,细节就不清楚了。\\

\textit{\hfill\noindent\small 2015/05/19 00:00 提问; 回答}

\noindent[5.]{\Hei 答}:First of all, I don't bad-mouth; I just tell the truth.

Second, nobody is saying life in the US is no good. Quite the contrary, that is exactly the point: the people in the US have their good lives paid for by the other 95\% of the world. Just the fact that they forced their currency on the world, and then took away the promise of not printing money from thin air, is simply and indisputably the largest theft in human history. 

60\% of the CO2 in earth atmosphere comes from 5\% of its population. Do they bother to pay for it? In fact, they are pointing their fingers at China, which still has only 1/3 of emissions per capita .

I have been writing dozens of articles explaining that the US is strong not because of its "systems" but because of luck and shameless killing/robbing/stealing. If you want to dispute any of those, just lay out your arguments, but please don't pretend that they do not exist and just go on a rant using only baseless statements to the contrary.\\

\textit{\hfill\noindent\small 2015/05/19 00:00 提问; 回答}

\noindent[6.]{\Hei 答}:公平竞争一般会让人更努力,不过美国人是不相信公平竞争的,连美国驻北约司令都在上周说:\&ldquo;We don't believe in fair fight!\&rdquo;\\

\textit{\hfill\noindent\small 2015/05/20 00:00 提问; 回答}

\noindent[7.]{\Hei 答}:我先插几句话:
1. 薄熙来搞唱红打黑,打黑是由王立军办的,不但不遵守程序,基本上是我说谁是黑谁就是黑;不过比起文革还差好几个数量级。
2. 一般在街头示威,要求改变政体而不是政体内的反腐,就算是 "颠覆中共政权的活动"。在网上喊喊比较安全些,尤其是只针对个别官员的。
3. 习近平和王岐山的看法是既然党干有特权(其实在任何一个文明社会,都必须有人操作公权力,所以必然会有权力集中),他们就有义务在道德上达到比常人更高的标准,这些标准是由党纪委来执行的。\\

\textit{\hfill\noindent\small 2015/05/20 00:00 提问; 回答}

\noindent[8.]{\Hei 答}:台湾在1970和1980年代的快速发展,靠的也是有高度组织力和执行力的政府。不幸的是蒋经国死后,李登辉特意把它打烂了。现在台湾有的,就是零散的一些能干的技术人员,被其他组织力高的国家吸收去,是迟早的事。\\

\textit{\hfill\noindent\small 2015/05/20 00:00 提问; 回答}

\noindent[9.]{\Hei 答}:美国已经占尽便宜了,现在还要想办法搞TPP来\&ldquo;制定新规则\&rdquo;。这些规则当然不是公平的,而是只有美式的做法才合法,中式就是非法。所以我觉得TPP定是会定的,但是那些规则太离谱了,不会成为新一代的\&ldquo;普世价值\&rdquo;。\\

\textit{\hfill\noindent\small 2015/05/21 00:00 提问; 回答}

\noindent[10.]{\Hei 答}:搞高科技的一般以事业为导向,逐水草(研究资金)而居。中方资金充实而西方阮囊羞涩,自然就会有动力回去。\\

\textit{\hfill\noindent\small 2015/05/26 00:00 提问; 回答}

\noindent[11.]{\Hei 答}:我以前推荐过Ian Morris的《Why The West Rules-For Now》。我想这本书已经相当偏袒西方了(主要是主观而随意地把中国歷史上记载的经济数据往下修),但是它也认为你所说的这个\&ldquo;对外扩张的积极性\&rdquo;是地缘因素而不是文化因素。

我们观察的事实是他们扩张了,所以也可以说有了扩张性的文化。但是这个扩张是有明显的地缘利益因素(即必须绕过被新兴的土耳其截断的丝路而建立新的海运路线),那么再把文化列为扩张的原因,就是不科学的,因为你违反了Occam's Razor。更有可能的是文化是扩张的后果。\\


\section{【媒体】克拉运河的假新闻}
\subsection{2015-05-22 12:55}


\section{7条问答}

\textit{\hfill\noindent\small 2015/05/22 00:00 提问; 回答}

\noindent[1.]{\Hei 答}:基本是对的,不过结帐是分开的,亦即当联储会印钞票时,所以很间接,一般人看不出来。

这个留言适合《美元的金融霸权》那篇文章。你不用担心我看不到;所有的新留言都会被列表。\\

\textit{\hfill\noindent\small 2015/05/23 00:00 提问; 回答}

\noindent[2.]{\Hei 答}:他太不小心了,这种事用email等于是为检方送证据。

历史上新兴国家总是必须从外引进先进技术的,只是现在美国把智慧财产权搞得如火如荼,所以以前美国人对英国做的事,中国人做来就算犯法。不过这些华裔教授愿意做,就代表大陆的发展环境够好;只是他们必须下定决心,一回国就不再到美国来。\\

\textit{\hfill\noindent\small 2015/06/24 00:00 提问; 回答}

\noindent[3.]{\Hei 答}:1. 是的。
2. 是的,战略油库的建构标准高,所需时间长,前任政府没有未雨绸缪。

不一定,中共政府的预测只是一方面的看法,经济事务往往有出人意料的发展。\\

\textit{\hfill\noindent\small 2015/07/06 00:00 提问; 回答}

\noindent[4.]{\Hei 答}:目前可见的消息应该还没有严重到会把油价压得\$30的地步。若是再有大事发生,就难说了。\\

\textit{\hfill\noindent\small 2015/07/12 00:00 提问; 回答}

\noindent[5.]{\Hei 答}:对我来说不算,因为几个月前就知道了。

而且伊朗增產应该是有限的。\\

\textit{\hfill\noindent\small 2015/10/25 00:00 提问; 回答}

\noindent[6.]{\Hei 答}:我很快地看了看那篇文章,觉得是典型的\&ldquo;无限上纲\&rdquo;类歪论,亦即藉着挑肥拣瘦,衹给单方面的例子,将普世问题定调为制度问题,然后对制度大幅挞伐。

读这种文章的时候应该时时自问,这些问题在不同的制度下就不会发生吗?衹有\&ldquo;专制\&rdquo;才有权力集中吗?衹有\&ldquo;专制\&rdquo;才有腐败吗?衹有\&ldquo;专制\&rdquo;才会因私害公吗?衹有\&ldquo;专制\&rdquo;才有惨烈的政治斗争吗?其实读读希腊和罗马共和时期的歷史就知道答案何在。这种文章会有市场,往往是读者本身的学识不够,不能自行举出反面的例子,所以多读好书是有益于一个人对宣传的抵抗力的。\\

\textit{\hfill\noindent\small 2015/10/26 00:00 提问; 回答}

\noindent[7.]{\Hei 答}:这应该在《谈油价》里问。

答案视油田而定。像是沙特这种完美的油田,不到5美元一桶;像是加拿大的油砂,大约60美元。\\


\section{【经济】【美国】大停滞的真原因}
\subsection{2015-05-24 18:53}


\section{26条问答}

\textit{\hfill\noindent\small 2015/05/24 00:00 提问; 回答}

\noindent[1.]{\Hei 答}:1\%的收入也水涨船高,因为他们服务的对象是0.01\%。\\

\textit{\hfill\noindent\small 2015/05/24 00:00 提问; 回答}

\noindent[2.]{\Hei 答}:《21世纪资本论》是一本很重要的新作,用它的观点可以解释财阀是如何囊括新增财富的。不过后进国家并不一定会重蹈覆辙,一个强力政府仍然能制定合理的规则来奖励创业,抑制金主。\\

\textit{\hfill\noindent\small 2015/05/24 00:00 提问; 回答}

\noindent[3.]{\Hei 答}:计算一个国家的生產总值是个极困难的问题;目前的GDP算法是个不完美的折中方案。

生產的成果有两种:一种是可以跨国交易的,另一种则不行。前者的定价简单得多也精确得多,顺差国家如德国擅长生產这些东西。后者像是房地產和大部分的服务业,同样品质的產品在不同国家价钱会差很多,逆差国家如美国有较多的產值来自这类。一般的逆差国家货币会贬值,但是美元因为是国际储备货币,它的匯率比一般情况高很多,使得美国服务业的產值被人为地抬高了,这是美国GDP居高不下的主因。\\

\textit{\hfill\noindent\small 2015/05/25 00:00 提问; 回答}

\noindent[4.]{\Hei 答}:台湾的财富的确也是被财阀和政客捞光了,不过在细节上和美国不太一样:美国的财阀集中在大企业和金融业,台湾的主要是搞建筑业的土豪。美国的社会对立是种族、贫富和宗教上的,台湾的则是毫无意义的统独。

我每年夏天回台南乡下都很难过:不论是在街角蹲一个下午只卖一小篮自家水果的老婆婆,或是社区方圆100米内寂静无讯号的Wi-fi,都提醒我这个社会病得很重。中產阶级生活水准还不错,但是那是建立在极为便宜的人力工资上,弱势群体如贫民和年轻一代独自肩负20多年恶搞的苦果,只为了极少数的土豪和政客继续搜刮财富。\\

\textit{\hfill\noindent\small 2015/05/25 00:00 提问; 回答}

\noindent[5.]{\Hei 答}:这是需要一组专家一两年时间研究的问题,我没有答案。

我同意少用是最环保的;大肆消费以促进经济只有以往的美国那样不计社会成本和不受资源局限的情况下才合理。\\

\textit{\hfill\noindent\small 2015/05/25 00:00 提问; 回答}

\noindent[6.]{\Hei 答}:水退了,才知道谁在裸泳。台湾的水已经退了很远了,政府、社会和选民仍然光着屁股在吵统独,若不是我的老家,还真是个奇观。

《21世纪资本论》有很严谨的证明,为什么在和平时期资本主义社会必然是富者越富,贫者越贫。这个趋势是可以由政治力量扭转的,但是非常困难,政府不但必须是强力的,还必须是睿智的;一般民主政府两者都不是,所以后果可想而知。\\

\textit{\hfill\noindent\small 2015/05/25 00:00 提问; 回答}

\noindent[7.]{\Hei 答}:是没有错,但是最近20年,汽车的能源效率有明显的增长,所以今年的新车就比20年的旧车省油得多,污染控制也有效得多。废气排放有很难收费的社会成本,强制报废或许是太极端了,但是有其逻辑根据的。\\

\textit{\hfill\noindent\small 2015/05/25 00:00 提问; 回答}

\noindent[8.]{\Hei 答}:你说的都是真正实践上的困难,像台湾这样的小国大概是无解的,不过至少一个有点主见毅力的总统会比马英九好些。

美国在70-80年代是世界霸主,实在不须也不该为富人减税。

中国在中高速成长期,仍然对资金有吸引力,这时就该着手处理贫富不均的问题;等到进入已开发阶段就太晚了。

富人是很贪心的,连美国18\%的资本利得税都嫌高,所以才会有瑞士和加勒比海的私人银行业。美国税收入不敷出后,开始对它们追缴,瑞士的逃税生意在过去两年很受打击。中共应该未雨绸缪,和美国站在同一阵线;第一步可以是拔掉香港的逃税生意(香港是亜洲第一大逃税中心)。\\

\textit{\hfill\noindent\small 2015/05/25 00:00 提问; 回答}

\noindent[9.]{\Hei 答}:其实美国人喜欢想像大陆经济的统计数据是灌了水的,完全没有证据,全凭他们既有的偏见。

中共官方如果说有扭曲统计数据的话,也是低估而不是高估。每隔五年左右,GDP数字会做一次校正;印度的永远是向下(亦即把以前的数字下修,以便使未来的数字可以高一些,因为一般媒体只注意刚发布的本季估计值,后来修正了很少人知道),中国的永远是向上。而且光在房地產的等同租赁產值上,中国的会计方法就保证使总GDP表面上比美国的会计算法少将近10\%,在医药上也严重低估。\\

\textit{\hfill\noindent\small 2015/05/26 00:00 提问; 回答}

\noindent[10.]{\Hei 答}:至少有些台上台下的政客还肯想法面对问题,这比美国要好多了。\\

\textit{\hfill\noindent\small 2015/11/14 00:00 提问; 回答}

\noindent[11.]{\Hei 答}:日本的老本深厚,还要10-15年才会吃到差不多。

中国老龄化当然也会是个问题,尤其财富累积还远不如日本、欧洲。不过这是可以用政策解决的;习近平若是能成功地进行產业升级和党内反腐,下一任自然有余裕来处理这些慢性问题。\\

\textit{\hfill\noindent\small 2015/11/24 00:00 提问; 回答}

\noindent[12.]{\Hei 答}:这里的逻辑论证题目是生產财富是否必然会加大贫富不均,是你自己下的。我衹要一个反例,就已经证明它不成立。北欧未来发展如何,跟这个议题有何相干?现在你顶多是反过来给了两个正面例子(委内瑞拉和阿根廷),就要说有绝对普世性?你显然不懂逻辑,那么先回去重念高中和大学。不要在此丢人现眼,浪费大家时间。

就是这些逻辑、理智和修养完全不成熟的人,搞得我很烦。回了他纠缠不清,删了他也是纠缠不清。最后严重消耗了公共资源,亦即我的精力和时间。\\

\textit{\hfill\noindent\small 2015/11/24 00:00 提问; 回答}

\noindent[13.]{\Hei 答}:错了。一个积极有为的政府可以弥补资本主义的先天缺陷,创造均富的市场经济,北欧和罗斯福治下的美国就是前例,极左路綫不是极右之外的唯一选项。\\

\textit{\hfill\noindent\small 2015/11/25 00:00 提问; 回答}

\noindent[14.]{\Hei 答}:我引用的是北欧的过去,在这个议题上和北欧的未来没有关系。

至于贫富差距的合理性,那是一个不同的议题。适当的贫富差距如你所说是必要的,但是纯资本主义已经不再奖励用于生產的资本,赚大钱的多是非生產性的寻租掠夺,经常有很高的政商勾结成分;而真正的生產性事业天生风险高、报酬慢,在绝对自由主义经济制度下不但不能成长,反而会萎缩,所以不能说自由放任对整体国家社会有益。

至于公平性,过犹不及。大公司的CEO薪水曾经只高出小职员一个数量级,或许两个数量级算是合理的,但是现在的美国已经涨到四个数量级,还在向五个数量级迈进。这些CEO主要是靠集团内部政治斗争而掌权的,凭什么每年拿上亿美元?贫富差距奖励勤奋态度的效应衹有在差距是一两个数量级时有效,超过这个范围,你努力一辈子还不如富家子弟一个礼拜的零用钱,那么贫富差距奖励的就衹能是想法娶嫁入豪门,十九世纪的欧洲正是如此,你读读狄更斯的小説就知道了。\\

\textit{\hfill\noindent\small 2015/11/25 00:00 提问; 回答}

\noindent[15.]{\Hei 答}:我已经在几篇前文解释过美国的贫富不均是如何变迁的。你若是读过就知道Murray的理论完全脱离现实。

美国为财阀脱罪的歪论极多,这篇正文就描述了其中一个,你随便引用一个毫无意义。我已经说过多次,宣传假话汗牛充栋,请不要拿来浪费我的时间。\\

\textit{\hfill\noindent\small 2015/11/26 00:00 提问; 回答}

\noindent[16.]{\Hei 答}:你说的对,我相信医疗和教育这类基本的人权和机会应该是平等的。至于缩小劳动所得的差距,其实根本在于缩小资本利得与劳动所得的差距。大公司的CEO能付给自己这么多钱,主要在于他掌控了大笔的资金,绝对不是因为他的\&ldquo;劳动\&rdquo;有那种价值。\\

\textit{\hfill\noindent\small 2015/11/26 00:00 提问; 回答}

\noindent[17.]{\Hei 答}:谢谢体谅。\\

\textit{\hfill\noindent\small 2015/12/03 00:00 提问; 回答}

\noindent[18.]{\Hei 答}:巨头无耻抄袭小公司產品逼迫其退出,正是当年微软称霸PC业的手段,其后30年的软体业自然人人都照做,中外皆然。

自由市场在很多行业中和适当的管理规范下,还是有其存在的必要,竞争不一定是浪费。

不过作业系统有天然独占性,中国不能被动放任市场机制,应该主动扶植国内企业。这并不代表中共必须指定赢家,可以简单要求所有应用程式必须除视窗系统外,也能在UNIX或LINUX上执行,而且功能和稳定性不能有差异。\\

\textit{\hfill\noindent\small 2016/07/10 00:00 提问; 回答}

\noindent[19.]{\Hei 答}:如果Cowen的理论如你所説是正确的,那么停滞的应该是全球的整体成长率,毕竟科学技术是普世共通的。可是真正明显停滞的,不但是区域性的,而且衹是中位统计数字,而不是平均成长率。这当然是分配上的问题。

你所说的低端工作的外移,并没有错,但是你的逻辑假设它是Exogenous(外来给定的),这是错的。实际上你应该问问自己,为什么刚好到70、80年代之后,Outsourcing才成为潮流?正因为美国财团在1970年代,成功地打破了以往国家为先的社会共识,用一切以利润为先的思路取而代之,他们才得以自由地解雇美国工人。这样的策略,在1950、60年代和1970年代早期是不能想象的。以臺湾为例,与美国的工资比,在1950年代远低于1/10,到1980年代则超过了这个水平,那为什么是在1980年代才有真正的大规模的Outsourcing进入臺湾呢?

与其同时,美国公共基础教育的投资和水平也急转直下。如果你也把它当做Exogenous Event,那么它当然也帮忙财阀解释了Outsourcing的逻辑,可是实际上这同样是政策改变的结果,而不是原因。在1970年代之后,\&ldquo;自由启发式\&rdquo;教育取代了旧式的\&ldquo;填鸭式\&rdquo;学习;虽然这不是财阀主动发明的,但是你真的以为如果他们不是已经有Outsourcing这条路走,会容许工人教育水平的大幅下降吗?

总之,美国在1980、90年代,达到全球影响力的巅峰,即使有客观的因素,如Cowen和你所举的一些例子,会导致美国中產阶级的损失,如果美国的民主体制没有被财阀绑架,那么它应该以保卫大多数选民为己任,会采用很多政策工具来扭转局势。例如1985年的Plaza Accord,强迫欧日货币升值超过100\%,这样的流氓手段为了保护财阀的利益都可以搞得出来,那么你说中位收入停滞不前的问题,美国衹能束手就擒,岂不是自欺欺人?所以你的逻辑,假设了外来因素是给定的,美国政策不能反应(除了保护财阀之外),这才是颠倒因果。\\

\textit{\hfill\noindent\small 2016/07/11 00:00 提问; 回答}

\noindent[20.]{\Hei 答}:没错。\\

\textit{\hfill\noindent\small 2016/07/11 00:00 提问; 回答}

\noindent[21.]{\Hei 答}:Pollution control costs money. This is a problem of big money being greedy, not in manufacturing itself.\\

\textit{\hfill\noindent\small 2016/07/11 00:00 提问; 回答}

\noindent[22.]{\Hei 答}:China and much of East Asia are the incidental beneficiaries of the latest round of globalization, though. So at least some good comes out of it, although it was certainly not the original intention of the American hegemonists.\\

\textit{\hfill\noindent\small 2016/07/11 00:00 提问; 回答}

\noindent[23.]{\Hei 答}:说来说去,你就是相信自由化的潮流是天然而无可抗拒的。这是信仰,而不是事实,所有的证据都是相反的。例如到今年,美国已经衰态毕露,还是能推动TPP这种实际上是挖WTO墻脚的反自由化组织;在二三十年前,如果美国要反自由化,那真是弹指即来。别忘了,WTO就是当年靠美国大力推动才创立的。

全球化、自由化,当然可以找到科技的背景,但是最终还是全球霸主美国的权力核心同意了,才能发生。在当时政略上,美国自以为天下无敌,得意忘形,不再忌惮外国的竞争(冷战结束前,美国对苏联的围堵,就轻松地否决了后者参与\&ldquo;全球化\&rdquo;的过程,苏联集团可也是这个\&ldquo;全球\&rdquo;的一部分);在经济上,正是财阀的利益凌驾了中產阶级的影响力,才诞生了自由主义的理论、宣传和政策。

我的\&ldquo;这么重的社会主义思想\&rdquo;是观察了美国实际现象的结论,尤其是政策被决定前的幕后拉锯过程。你的自由主义信仰,在理论上是权力完全分散的极限结果,但是自由市场本身却有强烈的资本集中倾向。换句话说,自由主义必须假设平等的个体,但是却会自然导致极度的不平等,所以自由主义不衹是不符合事实,在逻辑上也是自我矛盾的。

我在前文《民主政治与自由经济》已经仔细讨论过这些问题。你若有新的证据或逻辑论点,可以提出来。光是喊口号,违反了这个部落格的规矩,也浪费大家的时间。\\

\textit{\hfill\noindent\small 2016/07/12 00:00 提问; 回答}

\noindent[24.]{\Hei 答}:It is true that only with wealth accumulation can a society deal with many problems and bring about better lives to its members. Industrialization being the key to wealth accumulation, we cannot possibly turn our backs to it.

That having been said, I don't think there is anything hypocritical about criticizing the rich and powerful, as long as the criticism is based on facts. Wealth is derived from the endeavor of the whole society, both active and passive. Even if some entrepreneurs do exceptional work, a rational observer should still have the ability as well as the right to acknowledge both their contributions and misdeeds, if there are any.\\

\textit{\hfill\noindent\small 2016/07/13 00:00 提问; 回答}

\noindent[25.]{\Hei 答}:Real innovation is hard. Much easier it is to make money by lying to and squeezing the weak. But let's stay grounded on facts and facts only. Avoid over-generalization.\\

\textit{\hfill\noindent\small 2017/08/21 00:00 提问; 回答}

\noindent[26.]{\Hei 答}:现代美国的上下层分离,源自雷根的政策转向,而其基础来自70年代富豪对詹森总统的\&ldquo;大社会\&rdquo;法案的反扑,这点我已经多次解释过了。

当时是苏联力量相对于美国最强的时代,所以\&ldquo;缺乏一个外在压力\&rdquo;无从谈起。\&ldquo;上层从底层吸收财富的速度变慢\&rdquo;,也不是因为\&ldquo;底层的衰弱\&rdquo;,而是詹森变法的结果,应该算是底层的强大才对。\\


\section{【战略】【美国】当代美国战略局势与策略}
\subsection{2015-05-27 18:46}


\section{1条问答}

\textit{\hfill\noindent\small 2015/05/27 00:00 提问; 回答}

\noindent[1.]{\Hei 答}:因为财阀不可能放弃特权,会越来越糟糕。但是美国的爱国宣传很到家,再怎么闹也不会动摇国本。\\


\section{【经济】谈GDP数字的局限性 }
\subsection{2015-05-31 01:56}


\section{21条问答}

\textit{\hfill\noindent\small 2015/05/31 00:00 提问; 回答}

\noindent[1.]{\Hei 答}:我想经济学家写的狗屎比吃的多。

自杀在实践上是不算的,逻辑上很难说。辅助自杀絶对可以算的。\\

\textit{\hfill\noindent\small 2015/05/31 00:00 提问; 回答}

\noindent[2.]{\Hei 答}:是,是,是,是,是。

美国几乎在每一种让GDP浮肿的手段都做了,所以人均GDP看来比德国高60\%还多,但是生活水准的差异其实即使有也是很小的。

台湾用PPP算的人均GDP比日本还高,所以中產阶级过得其实很舒服,但这是因为人工便宜,提供人工的下层百姓吃亏了。\\

\textit{\hfill\noindent\small 2015/05/31 00:00 提问; 回答}

\noindent[3.]{\Hei 答}:印美钞的极限是很大的。Bretton Woods系统下黄金定价\$35,自1973年美元可以自由发行后,现在黄金是\$1200,增为34倍,亦即美国多印了约34倍。

我的资料是M3在2008年达到13万亿美元(你的数字是不是以人民币算的?),此后联储会不好意思再公布,不过QE总共将近5万亿,所以现在应该是18万亿左右,比美国一年的GDP略高。

18万亿好像没有很多,可是里面有很多是1970、1980年代发行的,那时的美元比现在的购买力强3-5倍。

此外,美国人只有在自己经济低迷的时候印钞,不管别人是否通货膨胀,这是一种期权(Option),它的价值和印的钞票本身是同一个数量级的。\\

\textit{\hfill\noindent\small 2015/05/31 00:00 提问; 回答}

\noindent[4.]{\Hei 答}:我想李克强已经改用他自己的GPI了。

不过GPI里的社会成本那项先天就没有帐可以算,所以误差很大,这是为什么它没有被推广的原因。\\

\textit{\hfill\noindent\small 2015/05/31 00:00 提问; 回答}

\noindent[5.]{\Hei 答}:不算,但是发钱让贫民站在街头假装扫街是算的。我记得2008年时,扁政权就这样做过。\\

\textit{\hfill\noindent\small 2015/05/31 00:00 提问; 回答}

\noindent[6.]{\Hei 答}:好啦,玩笑话扯远了,我们就此打住了。\\

\textit{\hfill\noindent\small 2015/05/31 00:00 提问; 回答}

\noindent[7.]{\Hei 答}:一般国家会对\&ldquo;非正式经济\&rdquo;(\&ldquo;Informal Economy\&rdquo;)做估算,不过我不知道中国GDP在这方面的细节。\\

\textit{\hfill\noindent\small 2015/06/01 00:00 提问; 回答}

\noindent[8.]{\Hei 答}:世界银行去年就宣布中国的GDP用PPP计算已经超过了美国,但是中共官方当然马上就出来灭火。\\

\textit{\hfill\noindent\small 2015/06/01 00:00 提问; 回答}

\noindent[9.]{\Hei 答}:\&ldquo;中国用的是只计入固定资产虚拟折旧的成本估算法\&rdquo;没有错。你是学经济的吗?

你最后一段评论和我自己的猜想一样。我在这个博客讨论的一些实话,在美国大概只有白宫里面才听得到。\\

\textit{\hfill\noindent\small 2015/06/01 00:00 提问; 回答}

\noindent[10.]{\Hei 答}:我对黄仁宇或他的着作都不熟。

对近年来的大数据热潮,我是存疑的。在真正复杂的系统里,各因素的贡献不是线性的,而是有彼此加成或抵消的现象。但是大数据分析基本就是指纯粹从数据出发,而没有先对现象本身建立理论模型,所以这个分析基本上只能是线性的。线性的现象有是有,但是大部分算是低垂的果实,早已被摘下来了。

如果你是决策者,我会建议你不要迷信大数据,因为它基本上是丢给电脑的黑箱作业,到底线性假设对不对,很难看出来。所以除非是有理论根据,事先认为应该有线性反应,否则大数据分析结果的对错无法判断。

如果你是从业人员,只为混一口饭吃,大数据用来自欺欺人很有效,你可以轻松吃一辈子不怕被抓包。\\

\textit{\hfill\noindent\small 2015/06/01 00:00 提问; 回答}

\noindent[11.]{\Hei 答}:你问的很好啊。我早就想对大数据这个新潮流说两句话,可是有没有足够的题材写一整篇文章。很高兴这次有机会可以在留言栏做评论,希望解释得够清楚。

至于你提问的原意,抱歉我对两者都不熟,无从评论起。\\

\textit{\hfill\noindent\small 2015/06/02 00:00 提问; 回答}

\noindent[12.]{\Hei 答}:1. 那真的要看他书里的细节,我还是不要乱猜的好。
2. 我说的线性是数据分析用的Regression,在大数据应用上只能是Linear Regression。这在某些情形下是错误的假设。\\

\textit{\hfill\noindent\small 2015/06/02 00:00 提问; 回答}

\noindent[13.]{\Hei 答}:哪里。大数据分析最早是用在金融这种虚拟產业上,刚好是我的本行,所以我兴致来了,多说两句。\\

\textit{\hfill\noindent\small 2015/06/02 00:00 提问; 回答}

\noindent[14.]{\Hei 答}:你在《大停滞的真原因》的那张图上可以看出美国的中位家庭收入大约是五万美元出头,其实2013年的确切数字是\$51939,同年的人均GDP是\$52939,两者基本相同。2014年的美国人口是318900000,而家庭数是115227000,所以平均每个家庭有2.768人,那么人均GDP和中位人均收入的比值就是2.768\&times;52939/51939=2.821。当然在一个完全平等的国家,这个比值应该是1;如果是完全不平等的,比值会是无限大。

一般计算收入平等程度用的是Gini系数,它和上面的比值没有简单的函数关系。在一个完全平等的国家,Gini系数应该是0;如果是完全不平等的,Gini系数会是1。Gini系数又分税前和税后两种,根据Pew Research Center在2013年的统计,美国的两个系数分别是0.499和0.380。这两者中当然是税后的才有真正意义,美国的税后系数在31个已开发国家中排名第二高(亦即极为不平等),仅次于智利。

我找不到中国的税后数据。税前的数字和美国相当,但是开发中国家一般会有较高的Gini系数。\\

\textit{\hfill\noindent\small 2015/06/03 00:00 提问; 回答}

\noindent[15.]{\Hei 答}:Gini系数的计算很困难(主要是逃税的收入很难估计),误差很大,不同的研究对美国税前Gini系数的估计从0.41到0.5都有,不过我认为Pew的研究似乎是最详尽可靠的。

中国的税前Gini系数一般认为是在0.47左右,所以我说\&ldquo;和美国相当\&rdquo;。至于税后系数,大概连专业机构也做不出来,我们也就不用乱猜了。

一般来说,GDP成长率越高,Gini系数也会越高,这是市场经济的自然结果,中国并不算是很不平等的。\\

\textit{\hfill\noindent\small 2015/06/04 00:00 提问; 回答}

\noindent[16.]{\Hei 答}:TPP做为一个传统式的自贸协议,其实只是美国的次要目的。中国已经跟过半的TPP成员有了自贸协议(主要的例外是日本、美国和加拿大)。TPP成员中,并没有像中国这样的新兴工业国家(如南韩和台湾)。而且美国此前也已经和过半的成员签了自贸协议(主要的例外是澳洲和日本)。所以TPP在这方面,意义不大。

欧巴马急着签TPP的真正用意,是要藉此建立以往WTO没有的贸易规则。除了在版权和专利上,採用极为严苛和广泛的美式标准,最重要的是保护跨国公司,使未来各签约国不能自行决定会改变商业环境的国内法(例如禁烟)。澳洲和日本签了这个卖身契之后,基本上就对美国金主完全放弃主权,所以TPP有如葵花宝典一样,要先自宫主权才能入门。

中国刚要建立各类管理规章,而过去百年来任意以国内法规蹂躏外国公司的美国现在却不愿受他国法规约束,所以中国绝对不会笨到挥剑自宫。中方顶多藉口可能加入,来询问一下TPP的进度;是欧巴马故意扭曲事实,藉此编出中国要加入TPP的消息,主要还是要忽悠日澳两国乖乖引刀自宫。\\

\textit{\hfill\noindent\small 2015/07/04 00:00 提问; 回答}

\noindent[17.]{\Hei 答}:我相信大陆的很多经济活动为了逃税是躲在官方统计之外的,所以中国的经济总量可能超出美国不只20\%,本文中的估计是很保守的。\\

\textit{\hfill\noindent\small 2017/10/10 00:00 提问; 回答}

\noindent[18.]{\Hei 答}:没有错。

\$18.04T是2015年的美元,\$18.57T是2016年的美元。直观上总成长是3.0\%,其中含通货膨胀1.4\%,所以GDP成长率=3.0\%-1.4\%=1.6\%

中国的GDP还要受匯率影响。2016年,人民币对美元贬值,所以用美元计价的增长看来不大。

以美元匯率计价的中国GDP没有什么意义。如果真的要比较中美经济的体量,至少应该用PPP计价,最好是看工业產值。同样一块美元的GDP,是工业,还是诉讼或金融,差别很大。

目前中国的工业產值约当美国两倍左右,所以人均只有后者的一半。要达到人均工业產值的齐头,大概是2030-2035年间的事了。届时美国约等于华东,欧洲等于华南,日本相当于一个省,臺湾。。。\\

\textit{\hfill\noindent\small 2017/10/11 00:00 提问; 回答}

\noindent[19.]{\Hei 答}:对中国这种人口老化有压力,又从政策上会坚持实体经济的国家来説,机器人的进步是件好事。\\

\textit{\hfill\noindent\small 2019/12/07 18:52 提问;2019/12/08 01:05 回答}

\noindent[20.]{\Hei 答}:不是。中國的GDP數字,向來回溯修正是向上的多,和印度剛好相反。
\\

\textit{\hfill\noindent\small 2023/12/17 13:18 提问;2023/12/18 02:56 回答}

\noindent[21.]{\Hei 答}:
消費價格的差距,視產品和地域而異,五六倍並非不可能。例如康州最便宜的理髮店\$16一次,但非常不專業而草率;專業的平價理髮(Barbershop,不是Hair Salon)\$30一次。即便在食品上,和台灣相比,也是隨種類而有很大的差異:肉類價格基本一致,但青菜則倍數增加,華人愛吃的十字花科可以貴到十倍。
這類話題沒有深度,不適合在博客討論;僅此一次,下不爲例。
\\


\section{【经济】再谈TPP}
\subsection{2015-06-05 04:46}


\section{37条问答}

\textit{\hfill\noindent\small 2015/06/05 00:00 提问; 回答}

\noindent[1.]{\Hei 答}:正是蒋经国手下的国家资本主义,造就了1965-1990年的经济成长。李登辉上台后,保护手段还在,路却越走越偏。

这次TPP,美国眼看着要把日本生吞活剥了。台湾侥幸逃脱,若是蔡英文反过来求美国人剥削,那才是道道地地的卖国,也是莫大的讽刺。\\

\textit{\hfill\noindent\small 2015/06/05 00:00 提问; 回答}

\noindent[2.]{\Hei 答}:台湾还有一些有心的技术官僚,只是他们已经无权做事,而且太过崇美,容易上美国人的当。\\

\textit{\hfill\noindent\small 2015/06/05 00:00 提问; 回答}

\noindent[3.]{\Hei 答}:我想15年后,美元会只是许多重要货币之一。过去这七年的QE应该是最后一次这种规模的大吸血了。\\

\textit{\hfill\noindent\small 2015/06/05 00:00 提问; 回答}

\noindent[4.]{\Hei 答}:想了一段时间了,很高兴能把它解释清楚。\\

\textit{\hfill\noindent\small 2015/06/05 00:00 提问; 回答}

\noindent[5.]{\Hei 答}:未来15年是中国全面超越美国的过渡期。过了就没事了。\\

\textit{\hfill\noindent\small 2015/06/05 00:00 提问; 回答}

\noindent[6.]{\Hei 答}:我不知道日本人也有讨论。我的资讯主要来自澳洲。

国际法庭的确是不可靠,但是仲裁条款比这还糟糕,因为仲裁不是上国际法庭,而是独立的仲裁会。跨国公司不用一年就可以把仲裁会的官员里里外外全部收买。

TPP的主要忽悠对象,第一是日本,第二是澳洲。日本在1980年代后期,被逼着让日圆狂涨,以致经济衰退至今,没想到还是没学乖,又要被美国人再整一次。\\

\textit{\hfill\noindent\small 2015/06/06 00:00 提问; 回答}

\noindent[7.]{\Hei 答}:很难说。欧巴马和金主们比我两个月前想像的要贪心得多了,但日本和澳洲当奴才当惯了,说不定会真的签下卖身契。\\

\textit{\hfill\noindent\small 2015/06/07 00:00 提问; 回答}

\noindent[8.]{\Hei 答}:Welcome to the forum, and thanks for the link.\\

\textit{\hfill\noindent\small 2015/06/07 00:00 提问; 回答}

\noindent[9.]{\Hei 答}:所以现在TPP要把眾小国骗上车,开始有些困难。\\

\textit{\hfill\noindent\small 2015/08/15 00:00 提问; 回答}

\noindent[10.]{\Hei 答}:我写稿也是很累的,常常咖啡因过量,晚上就睡不好。

我说的\&ldquo;撕破脸\&rdquo;并不代表即将开战,而是会越来越公开的敌视,越来越不顾一般的外交礼仪。但是最终还是会受美国自身实力退缩的限制。

我想你转载的那段描述是\&ldquo;大崩溃\&rdquo;模式,并不是没有可能,但是更可能的是\&ldquo;慢失血\&rdquo;模式,就是内部虽然有胡闹发泄,但是对外仍是力不从心,利用美元吸血的规模也慢慢不能支持旧有的国际影响力和国内的消费力,于是只好逐步退缩,接受较低的军事威胁能力和人民生活水准,也就是西欧现在已经开始的过程。20年后的美国,可能就像今日的欧盟。\\

\textit{\hfill\noindent\small 2015/10/05 00:00 提问; 回答}

\noindent[11.]{\Hei 答}:It is very disappointing to me too, although I have been predicting this outcome all along.

Even after its approval, the details are still not forthcoming. This really is a very very bad sign.\\

\textit{\hfill\noindent\small 2015/10/05 00:00 提问; 回答}

\noindent[12.]{\Hei 答}:我早已说过,TPP是美国权力的真正核心、幕后金主的重点项目,再有什么困难也会解决。\\

\textit{\hfill\noindent\small 2015/10/06 00:00 提问; 回答}

\noindent[13.]{\Hei 答}:在美国被否决的机率是零,在其他国家稍大于零。

我也说过TTIP更重要。但是很多这些条款都是图利财团的,而欧洲也有很多跨国财团,所以中方要拉欧抗美有困难。还好中国现在是农產进口国,而农民在西方国家都有与人口不成比例的超大政治力量,应该从这点上着手;不知中国的贸易代表是否能在农產方面发力。\\

\textit{\hfill\noindent\small 2015/10/06 00:00 提问; 回答}

\noindent[14.]{\Hei 答}:和欧洲谈判的主要问题是他们的政客很多受美国控制,民眾则被仇中宣传洗脑过,所以会遇到一些莫名其妙的不友好态度。入手点除了像中国可以大量进口的农渔业以及我提过的金融业之外,必须专注在不受美国人操控的政治人物,如Osborne之流。Merkel自己虽然大概不是美国间谍,也在乎企业界的意见,她身边的幕僚却必然有私底下向CIA拿薪水的人物,要和德国做进一步突破只怕不容易。\\

\textit{\hfill\noindent\small 2015/10/06 00:00 提问; 回答}

\noindent[15.]{\Hei 答}:我已说过了,没有条文细节,大家都只能胡猜。那还不如只看晨枫的文章,至少他猜得还有些水准。\\

\textit{\hfill\noindent\small 2015/10/06 00:00 提问; 回答}

\noindent[16.]{\Hei 答}:正在考虑中。问题是条约虽然谈出来了,细节却仍然没有公布,这其实代表着美国人必然塞了很多地雷在里面。\\

\textit{\hfill\noindent\small 2015/10/07 00:00 提问; 回答}

\noindent[17.]{\Hei 答}:岂止是黑箱作业,是黑洞作业了。\\

\textit{\hfill\noindent\small 2015/10/07 00:00 提问; 回答}

\noindent[18.]{\Hei 答}:其实中国若是先跟挪威谈自贸协定,就会是很好的弹性政策。原本为了2010年的Nobel Peace Prize两边闹僵了,但是一旦挪威人急着卖鲑鱼,中国还是应该以务实的态度与之合作。

德国人赚欧元区的钱已经盆满钵满,企业界对Merkel所能施的压力有限。\\

\textit{\hfill\noindent\small 2015/10/07 00:00 提问; 回答}

\noindent[19.]{\Hei 答}:没有什么\&ldquo;也许\&rdquo;,台日最后一起沉沦是必然的,唯一的问题在于何时发生。\\

\textit{\hfill\noindent\small 2015/10/08 00:00 提问; 回答}

\noindent[20.]{\Hei 答}:South Korea is smart. They did not say they "want to join"; they said "they would consider joining if it was in their national interests". That is the correct attitude.\\

\textit{\hfill\noindent\small 2015/10/10 00:00 提问; 回答}

\noindent[21.]{\Hei 答}:看来的确是以抑制中国和图利财团为双重目标。要达成前者,就必须把TTIP也搞成,但是一旦TPP内容曝光,美国政府固然已是财阀专治,没有被拦下的危险,欧洲的社会力量要强得多,财团只怕无法瞒天过海。\\

\textit{\hfill\noindent\small 2015/10/10 00:00 提问; 回答}

\noindent[22.]{\Hei 答}:我觉得TPP的主轴还是知识產权以及贸易纠纷仲裁机制,这些利益交换只因为日本的农业保护太离谱,必须做做样子,美国给的汽车进口优惠更是微乎其微,完全只是口惠而不实。表面上看来关税砍得很凶,细节上却仍处处掣肘,例如实施年限就高得可笑。\\

\textit{\hfill\noindent\small 2015/10/11 00:00 提问; 回答}

\noindent[23.]{\Hei 答}:我没有太多的资讯,只注意到刘鹤曾出面高调鼓吹此事。他是习近平的经济智囊,所以应该是有习近平的幕后强力支持的。

我个人的猜想(没有真正的消息根据,纯粹是\&ldquo;如果我来做,会怎么办\&rdquo;的推理)是引入有分量但是不占主导性的私人基金,用意是让投资人来辅助中共党内的监督管理。其背后的逻辑在于这些企业有很强的专业性,党部高层不可能完全了解技术细节,须要外界专家来辅助。这个构想本身是好的,但是实践上是否成功,仍然要看详细的规则能否定得合理。\\

\textit{\hfill\noindent\small 2015/10/11 00:00 提问; 回答}

\noindent[24.]{\Hei 答}:没错,吃亏最大的是加入TPP的小国。但是习政府应该不会笨到飞蛾扑火的地步(台湾刚好相反),所以TPP真正能打击中国的路还是要经过TTIP。如果欧洲也接受这些条款,那么中国就不能再打太极拳,必须积极联络金砖国家制定另一套规则;如此一来世界贸易会分裂为南北对立,大家都吃亏,而且中国丧失欧美市场额份也会收到额外的负面影响。

我并不急着再写新文,因为事情的发展完全没有超出我几个月前的预料,纯粹只是把我当时的猜测印证为事实。所以很多细节,在正文里都有讨论,请大家仔细再看一次。\\

\textit{\hfill\noindent\small 2015/10/22 00:00 提问; 回答}

\noindent[25.]{\Hei 答}:Merkel身边都是\&ldquo;极度亲美\&rdquo;的幕僚,这种事不太可能会领头做。

英国做了示范之后,跟着来的大概会是法国和意大利。德国要到不得已的情形下才会上车。\\

\textit{\hfill\noindent\small 2015/10/23 00:00 提问; 回答}

\noindent[26.]{\Hei 答}:如果人民银行买进那么多英镑,会造成很大的升值压力,必须与英方协同才有可能。

中欧自贸区应该会谈,Merkel和Hollande都不会敢说不;但是这种条约一般旷日费时,衹怕不是几个月能有结果的。\\

\textit{\hfill\noindent\small 2015/12/05 00:00 提问; 回答}

\noindent[27.]{\Hei 答}:终于有专业人士写书来介绍这段歷史,很好,希望大众媒体也能报导一下。\\

\textit{\hfill\noindent\small 2015/12/18 00:00 提问; 回答}

\noindent[28.]{\Hei 答}:新加坡不是一个正常的国家,它是独立的港口城市,主权十分脆弱,必须依附远来的霸主来对抗附近的强邻。它的经济全靠贸易,所以TPP虽然对促进贸易衹有部分的效应,对新加坡还是有价值。至于代价,它的体制原本就没有贸易障碍,本土工业衹有少数几个,而且靠的是资本和地理位置,无需保护。TPP侵犯主权的部分,新加坡也不在乎,因为它的第一考虑是安全,第二考虑是经济,损失主权所失的利益很小,可以忽略不计。

正因为新加坡不可能成为军事强权,它依附霸主不但是理所当然,而且是没有不利后果的;若是霸权交替,新霸主知道衹要交代一声,它会自动乖乖易帜,完全没有动手的必要。\\

\textit{\hfill\noindent\small 2015/12/19 00:00 提问; 回答}

\noindent[29.]{\Hei 答}:我觉得应该是内政改革>经济发展>外交稳定>解决领土问题。\\

\textit{\hfill\noindent\small 2015/12/19 00:00 提问; 回答}

\noindent[30.]{\Hei 答}:石油进口国的石油价格一向是严格管制,像欧洲一般是明説汽油税200\%或300\%。

我想中国人没有承受重税的习惯,若是学欧洲衹怕骂声更响。\\

\textit{\hfill\noindent\small 2016/09/18 00:00 提问; 回答}

\noindent[31.]{\Hei 答}:我没有看过这本书,对中共金融界的细节也不熟。不过\&lsquo;资本逐利\&rsquo;至上论,原本就是新自由经济主义的歪论,在实践中已经被一再证明是非常昂贵的错误。\&lsquo;讲政治\&rsquo;就是金融受国家利益主导,这是好事啊。\\

\textit{\hfill\noindent\small 2016/09/20 00:00 提问; 回答}

\noindent[32.]{\Hei 答}:这事,我在《漫谈近来的经济态势》已经讨论过了。整个世界都不景气,中国要改革还不能让GDP成长率掉过6\%,还有什么办法呢?不过我也觉得这次又炒过头了。

话説回来,认赌服输。投资本来就是有风险的。如果买房赚了钱是自己的,赔了就没有资格抱怨。\\

\textit{\hfill\noindent\small 2016/09/20 00:00 提问; 回答}

\noindent[33.]{\Hei 答}:赌博怎么会没有庄家呢?庄家又怎么会不抽成呢?中国房產这个赌局的长期赢面远大于赔,哪一家赌场有这种好事?

房產的真正负面影响,不在于投资人的回报太低,而在于它是一种不劳而获、寄生性、非生產的投资,对实体工业有强烈的挤压作用,获利也很容易成为热钱,后续的麻烦更多。\\

\textit{\hfill\noindent\small 2016/09/20 00:00 提问; 回答}

\noindent[34.]{\Hei 答}:东北的沉疴,不是抄抄地皮可以解决的。\\

\textit{\hfill\noindent\small 2016/09/27 00:00 提问; 回答}

\noindent[35.]{\Hei 答}:世界经济低迷,十九大在即,不是深刻改革的好时机。

现在世界经济开始復苏,我们等十九大后2018年,看看会有什么手段。我个人认为应该从税率着手。\\

\textit{\hfill\noindent\small 2016/09/28 00:00 提问; 回答}

\noindent[36.]{\Hei 答}:是的。我们看2018年会有什么政策出臺吧。\\

\textit{\hfill\noindent\small 2017/04/16 00:00 提问; 回答}

\noindent[37.]{\Hei 答}:可能性不大。

中国应该加紧利用这个时机,把RCEP儘快搞定,亦即把印度这个专业搅屎棍先踢出去,然后集中压力让日本低头。\\


\section{【美国】美国式的贪腐总统}
\subsection{2015-06-13 07:05}


\section{17条问答}

\textit{\hfill\noindent\small 2015/06/13 00:00 提问; 回答}

\noindent[1.]{\Hei 答}:美国财阀的贪婪和短视是有目共睹的,我只是不明白这些\&ldquo;盟友\&rdquo;怎么会蠢到看不出来。\\

\textit{\hfill\noindent\small 2015/06/13 00:00 提问; 回答}

\noindent[2.]{\Hei 答}:是的,我想你在生物医疗界20几年,美国体制腐化的过程一定看的很多。

美国的医疗费用占GDP的18\%,比欧洲的6-9\%高一倍以上,还没有全民保险,这种结构性的浪费只有在美元霸权的支持下,靠着对外的搜刮才能持续,所以一但霸权受威胁,整个体制也受威胁。

其实医疗和诉讼一样,胜者全拿,所以没有所谓的供需平衡价格,也就没有所谓的市场机制。这些人只是用口号来哄骗群眾,养肥自己。\\

\textit{\hfill\noindent\small 2015/06/13 00:00 提问; 回答}

\noindent[3.]{\Hei 答}:我已经提到了最起码的处理标准,就是比照Savings \& Loan,把银行搞到实质破產的过程中只要有撒过谎的就进牢。

你的说法把要不要拯救银行系统和事后算不算帐两件不相关的事混淆在一起,是典型的共和党传媒的说法。你是Fox News的观眾吗?\\

\textit{\hfill\noindent\small 2015/06/13 00:00 提问; 回答}

\noindent[4.]{\Hei 答}:人类是群居动物,演化过程使\&ldquo;公平\&rdquo;(\&ldquo;Fairness\&rdquo;)这个概念成为天生的本能。这是货真价实的普世价值,是写在基因上的,所以百姓一旦得到温饱,就会开始在乎贪腐,甚至是公车私用这样的小事。

2008年的金融危机对美国经济造成几万亿美元的损害,不但没有事后算帐,而且始作俑者个个加官进爵,大富大贵,这一方面消灭了社会的公平假象,进一步威胁政权的合法性,但是更重要是连这样公开而且万眾瞩目的事都如此不公,那么真正有关实质利益的施政步骤(往往是一般人很难注意到的法律和行政细节)偏袒财阀的程度就可想而知了。

金融界必须为实体经济服务,所以不该有大利润也不能有大损失,越简单基本越好。要达到这个理想,我只知道国有制有可能。当然反过来说,国有制并不保证金融机构运行合理有效;内行而有力的监管是不可或缺的。\\

\textit{\hfill\noindent\small 2015/06/13 00:00 提问; 回答}

\noindent[5.]{\Hei 答}:The forces behind TPP are immensely powerful, and I won't be surprised if they succeed in bulldozing all oppositions.

The US has been exporting corruption to the Western world for over half a century. For example, the French banking industry used to be run like the Chinese one, i.e. focused on serving the real economy. Not any more, it was deregulated in 1960's after intense lobbying. Not even General de Gaulle could stop the deregulation.\\

\textit{\hfill\noindent\small 2015/06/14 00:00 提问; 回答}

\noindent[6.]{\Hei 答}:我对台湾金融界不熟。\\

\textit{\hfill\noindent\small 2015/06/15 00:00 提问; 回答}

\noindent[7.]{\Hei 答}:The reason why China becomes such a big threat to American hegemony is exactly that it refuses to be exploited.\\

\textit{\hfill\noindent\small 2015/06/15 00:00 提问; 回答}

\noindent[8.]{\Hei 答}:His tone indicates that he is immune to reason and logic, so I'm afraid you are wasting your time trying to reason with him. But I appreciate your efforts nonetheless.\\

\textit{\hfill\noindent\small 2015/06/15 00:00 提问; 回答}

\noindent[9.]{\Hei 答}:Seriously, Kiwis are not nearly as brain-dead as Americans, are they? I have always thought that NZ hasn't fallen as far as the other English speaking countries, but perhaps distance colored my view.\\

\textit{\hfill\noindent\small 2015/06/15 00:00 提问; 回答}

\noindent[10.]{\Hei 答}:This was my specialty, so I can tell you without reservation that the reality is a lot worse than any book can describe. One of the traders called it "cluster-fuck", which is basically what every Wall Street bank or fund tries to do to its clients.\\

\textit{\hfill\noindent\small 2015/06/17 00:00 提问; 回答}

\noindent[11.]{\Hei 答}:It's a big country with a lot of private money. We cannot expect everyone to use his/her wealth wisely.

The local stock market was allowed to run up exactly for the purpose of avoiding big unwise outflow.\\

\textit{\hfill\noindent\small 2015/06/25 00:00 提问; 回答}

\noindent[12.]{\Hei 答}:I think the Chinese government is facing two forces of drawing capital overseas: the first is the US Fed stopping QE, thus reversing the dollar flow; the second is the Chinese property market, whose bubble was recently deflated, releasing a tremendous amount of idle capital. The People's Bank wants to keep the majority of these idle capital in the country, so the stock market is allowed to absorb them.

It makes no sense to grow another bubble, but the market should not be allowed to crash either. The People's Bank will let it run within a very wide band. I do agree with you that the small foreign investors should keep clear.
\\

\textit{\hfill\noindent\small 2015/09/26 00:00 提问; 回答}

\noindent[13.]{\Hei 答}:我也认为这只是政策调整,不算是真正的改革。\\

\textit{\hfill\noindent\small 2015/09/26 00:00 提问; 回答}

\noindent[14.]{\Hei 答}:中国现在需要的改革很多很大,象徵性的工程并不重要。\\

\textit{\hfill\noindent\small 2015/09/26 00:00 提问; 回答}

\noindent[15.]{\Hei 答}:不能说是\&ldquo;大\&rdquo;改革。\\

\textit{\hfill\noindent\small 2015/09/26 00:00 提问; 回答}

\noindent[16.]{\Hei 答}:既然只是水到渠成、理所当然的政策选项,就不算是改革。

真正的改革是动到既得利益者蛋糕的。\\

\textit{\hfill\noindent\small 2016/03/14 00:00 提问; 回答}

\noindent[17.]{\Hei 答}:周小川的成绩远不止你提的这一点。\\


\section{【台湾】四场演讲}
\subsection{2015-06-26 04:28}


\section{11条问答}

\textit{\hfill\noindent\small 2015/06/26 00:00 提问; 回答}

\noindent[1.]{\Hei 答}:M3总量不可能是GDP的十倍,那篇文章大概不太可靠。

未来一年半,世界经济仍会处于低迷状态,但是2017年应该会大幅好转。我看不出有眼前的经济大灾难。

恕我孤陋寡闻,Jared Diamond没有读过,不过看网上的资讯,他似乎和我一样是学自然科学出身,然后半路出家去想政治和社会的问题。我有空会看看他的着作。

李登辉的确是搞了个绿色文革。奇怪的是没有上位者推动之后,台湾人搞这个文革却越来越起劲。\\

\textit{\hfill\noindent\small 2015/06/26 00:00 提问; 回答}

\noindent[2.]{\Hei 答}:两年前看的,现在不记得出处了。不过我的印象是70、80年代大陆的经济论文就一致地对\&ldquo;虚拟经济\&rdquo;和\&ldquo;实体经济\&rdquo;做分别,这其实是继承马克思在《资本论》里的分析。\\

\textit{\hfill\noindent\small 2015/06/27 00:00 提问; 回答}

\noindent[3.]{\Hei 答}:民主制度的局限性,我在头两场演讲讨论过了。你两场都到了吗?

现代工业產品所需的零部件动辄以十万计,连法国、西班牙、意大利都还有很多强项。如果对大陆產业链销售时,不做限制不敲竹杠,那么中共做替代性研发自然就不会针对它有高的优先顺序。只要新世代產品的开发快于中共非优先的学习,生意就可以永远持续下去。日本的问题是產品出口有限制,能出口的也大敲竹杠,那么自然中方的战略性投资都是针对它而来的。\\

\textit{\hfill\noindent\small 2015/06/27 00:00 提问; 回答}

\noindent[4.]{\Hei 答}:资本,尤其是大资本,先天上就不受国界限制。劳动力却是局限于地方的,所以一个国家衰败了,只有99\%的劳动阶级倒楣,0.1\%的富豪反而得以享受廉价的服务。这在美国是财阀有意而为,在日本和台湾则是政治阶层胡搞的结果,而这些政客会想胡搞而且能胡搞,却正是劳动阶级的选票做愚蠢选择的后果,所以我不觉得能怪到资方头上。

日本和台湾一样,拒绝加入中国的產业链,而德国和南韩则相反。后者已经在过去十年表现远优于前者,我猜未来十年差距会更大,你觉得呢?\\

\textit{\hfill\noindent\small 2015/06/27 00:00 提问; 回答}

\noindent[5.]{\Hei 答}:关于希腊和欧盟的谈判,我在前文《希腊与欧元》里已经分析过了。几个月下来,完全没有跳出我预言的范围,所以就没有必要再写。基本上德国花了五年把上次被扯到的小辫子收了回来,所以不必妥协;现在其他欧猪国家的极左极右派反对党也虎视眈眈,如果要建立欧盟的威信,就不能妥协。所以除非希腊无条件投降,否则就只有被踢出欧元区。这对中共是非常有利的。\\

\textit{\hfill\noindent\small 2015/06/28 00:00 提问; 回答}

\noindent[6.]{\Hei 答}:我个人以下层弱势群体的利益为优先,所以发展虽必须比公义先行,却不能离得太远。大陆目前的阶段已经必须两者并重了。\\

\textit{\hfill\noindent\small 2017/07/19 00:00 提问; 回答}

\noindent[7.]{\Hei 答}:市场原教旨主义者的问题,不在于固守成规,而是本身就是金主扶持出来的传声筒,以便他们加速垄断财富。\\

\textit{\hfill\noindent\small 2017/07/20 00:00 提问; 回答}

\noindent[8.]{\Hei 答}:理性的人,都早已看出市场原教旨主义的谬误。不过整理出一套完整的理论来取代它,还是很有价值的。\\

\textit{\hfill\noindent\small 2017/07/21 00:00 提问; 回答}

\noindent[9.]{\Hei 答}:苏武忠于旧主,林毅夫却是投靠新主。我想这不能相提并论。

\&ldquo;情怀\&ldquo;被推到那样的极限,显然就是完全非理性的。\\

\textit{\hfill\noindent\small 2017/07/21 00:00 提问; 回答}

\noindent[10.]{\Hei 答}:我说一国两制行不通,是专指\&rdquo;马照跑、舞照跳\&ldquo;的港式地方自治模式,并不是说要照搬大陆的政治体制。事实上,臺湾没有经过真正的土改,也没有足够的共產党员,根本不可能一夕之间成为另一个一般的行省。

香港基本是一个用来实验西式政治制度的试管。固然这个实验已经给出负面的结果,但是还有经济社会层面的实验,是未来的中国须要考虑的。例如现阶段的富商基本在垄断财富上没有上限,但是最终我们必须追求至少是北欧式的均富。即使看的近一点、细一点,也有很多税务上的改革,尤其是房地產税,以及退休制度,必须探索新的形式。在这些方面,臺湾都比大陆还要\&rdquo;成熟\&ldquo;,也就特别适合做实验。

要做这些实验,就只能军事管理,才能完全忽略既得利益者的抱怨。\\

\textit{\hfill\noindent\small 2017/07/27 00:00 提问; 回答}

\noindent[11.]{\Hei 答}:我以前説过,长久以后,臺湾之于中国,就像Florida之于美国,大概就是旅游/退休和热带水果两个工业。\\


\section{【经济】谈通货膨胀}
\subsection{2015-06-29 12:37}


\section{10条问答}

\textit{\hfill\noindent\small 2015/06/29 00:00 提问; 回答}

\noindent[1.]{\Hei 答}:我在《民主政治与自由经济》一文里建立的就是逻辑体系,有听眾说是把社会学公理化了。至于可以计算的定量分析,不确定性太大,会引起常人误解,还不如不做。\\

\textit{\hfill\noindent\small 2015/06/30 00:00 提问; 回答}

\noindent[2.]{\Hei 答}:这就是我的主旨。其实Bernanke自己也不想这样,在2008/2009年左右他还说过想要用直升机撒钱,以免QE的钱都给了银行和大户;可是体制如此,他最后也只能送钱给银行界。

中共还有指定贷款对象的能力,很好,千万不能再让银行业自由化了。\\

\textit{\hfill\noindent\small 2015/06/30 00:00 提问; 回答}

\noindent[3.]{\Hei 答}:现在的日元贬值是有意的,等到危机来了,就是无法阻挡的。不过日本国内的储蓄雄厚,到底什么时候用完,可能不是三五年的事。\\

\textit{\hfill\noindent\small 2015/06/30 00:00 提问; 回答}

\noindent[4.]{\Hei 答}:涨的都是不能进口的,如鸡蛋、牛奶等等。手机、电脑和电视价钱却是一泻千里,所以平均起来CPI就很低了。\\

\textit{\hfill\noindent\small 2015/07/02 00:00 提问; 回答}

\noindent[5.]{\Hei 答}:原本资本一旦累积起来,就只会越滚越大,远超过劳动阶级能挣钱的速度。只是20世纪的前半有两场世界大战把这个资本累积的过程暂时打断了。核弹的发明使大国之间的战争打不起来,已经让全球经济向资本一面倒,电子自动化更加速这个趋势。

这是21世纪人类的最大挑战,大于霸权归属和全球暖化;而美式体制对它是绝对无解的。只希望中共能做个良好的示范;不过任何限制资本的政策都会引起强力的反弹和外逃,实在是个棘手的问题。\\

\textit{\hfill\noindent\small 2015/07/02 00:00 提问; 回答}

\noindent[6.]{\Hei 答}:请提供链接。\\

\textit{\hfill\noindent\small 2015/07/02 00:00 提问; 回答}

\noindent[7.]{\Hei 答}:完全同意。

美日的问题是连央行制定利率,都是为了金融大鱷的方便,完全没有考虑到实体经济的需要。美式经济学不把通货膨胀讲清楚,就是为了要这样浑水摸鱼。\\

\textit{\hfill\noindent\small 2015/07/03 00:00 提问; 回答}

\noindent[8.]{\Hei 答}:谢谢。\\

\textit{\hfill\noindent\small 2021/02/28 05:48 提问;2021/03/01 06:30 回答}

\noindent[9.]{\Hei 答}:金融家怎麽可能把自己的錢放入基建?社保扶貧更加不可能。美聯儲當然寧可印出來的鈔票直接進入實體經濟,但民主黨和共和黨執政的差別,在於前者的法案是八二開:80\%進入消費者口袋、20\%給財閥,而後者相反罷了。
\\

\textit{\hfill\noindent\small 2021/03/04 13:21 提问;2021/03/15 01:50 回答}

\noindent[10.]{\Hei 答}:有關美式通貨膨脹計算錯誤地專注在一般消費品之上,多年我已經討論過了。周小川在這裏指出的是,即使依照美國經濟界常用的邏輯,房產也與其他金融資產不同,有很高的實用價值,是幾乎所有消費者的購買目標,應該列入通貨膨脹的考慮之中。

其實美國本身的GDP就包含大約9\%的住房價值,不但算進CPI(不過一直是有意低估,只算部分房租,不算房價;例如與Case-Shiller指數相比,少了一半多),也包含在GDP Deflater裏面,只不過會被衝淡稀釋罷了。中國的GDP數字故意嚴重低估住房,所以連Deflater都不怎麽包含它的效應。

當然退一步說,房屋市場極爲重要,是生活必需品中最大的一項,應該受到社會主義政策的特別關注。它的漲跌不但可以視爲通貨膨脹的重要成分,而且應該在稅務、貨幣、開發商管理上以强力手段確保價格平穩合理。

中國的問題在於過去20年放任自由開發炒作,不但養肥了許多有官方後臺的地產商,而且衆多中產階級在被搜刮之後,反而成爲市場的人質。這一來,要回歸正確的政策,就會扯動幾億屋主的切身利益,困難程度一下衝上天了。我批評前任主政者的不作爲、不改革、不獨立思考、盲目引進美式經濟治理,指的就是這類的案例。

但是房產的牛市不可能無限持續下去,如果漲得太高必然崩盤,届時全國經濟的損失將比擬美國2008年次貸危機。一般老百姓不懂經濟,眼光只看到自己財產的名義價格,那麽政府只能把房市先穩定下來,然後慢慢等待消費品的通貨膨脹趕上來,這將需要幾十年不説,而且對無屋民衆買不起的問題沒有立即的幫助,但是上兩届留下一個大爛攤子,早點開始收拾比繼續拖下去要好些。
\\


\section{【歷史】装甲将军}
\subsection{2015-06-30 14:03}


\section{3条问答}

\textit{\hfill\noindent\small 2015/07/01 00:00 提问; 回答}

\noindent[1.]{\Hei 答}:请参阅前文《谈GDP的局限性》。\\

\textit{\hfill\noindent\small 2015/07/06 00:00 提问; 回答}

\noindent[2.]{\Hei 答}:I just wrote an article on this.\\

\textit{\hfill\noindent\small 2015/07/06 00:00 提问; 回答}

\noindent[3.]{\Hei 答}:同意,但是国家仍然有责任不让大财主藉着股市搜刮小股民的财富。\\


\section{【金融】【经济】谈中国股市和其他问题}
\subsection{2015-07-06 03:23}


\section{35条问答}

\textit{\hfill\noindent\small 2015/07/06 00:00 提问; 回答}

\noindent[1.]{\Hei 答}:这还用问?当然是规定你若要在这里做生意,就得在这里上市。\\

\textit{\hfill\noindent\small 2015/07/06 00:00 提问; 回答}

\noindent[2.]{\Hei 答}:Insider trading is another big issue, but I don't have a simple solution, so I did not mention it in the main article.

On the other hand, the few suggestions I did make are all easy to implement and hard to weasel out. They should make big improvements on the capital markets. I hope they don't fall on deaf ears.\\

\textit{\hfill\noindent\small 2015/07/06 00:00 提问; 回答}

\noindent[3.]{\Hei 答}:是的,但是中共政权内的利益集团和资本财团并不完全重合,力分则弱。习近平既然敢挑战前者,自然也可能压制较弱的后者。\\

\textit{\hfill\noindent\small 2015/07/06 00:00 提问; 回答}

\noindent[4.]{\Hei 答}:规定他们不能在国外上市没有用?你懂不懂企业的资金需求?

我真厌恶这些只会乱喊自我猜测,不做逻辑推理的留言。你再写意识流,就是立刻删除。\\

\textit{\hfill\noindent\small 2015/07/06 00:00 提问; 回答}

\noindent[5.]{\Hei 答}:中共在1960年代就想要造卫星、核弹和核潜艇。我不觉得我建议的事项有那么难。\\

\textit{\hfill\noindent\small 2015/07/06 00:00 提问; 回答}

\noindent[6.]{\Hei 答}:在美国,每家银行都有几千个年薪50万美元以上的推销员,专职负责收买基金的职员。你真以为基金的雇员会以投资人的利益为第一要务吗?\\

\textit{\hfill\noindent\small 2015/07/07 00:00 提问; 回答}

\noindent[7.]{\Hei 答}:遏制囤积房產,压低房价,鼓励买房自住,再配合公建公寓。社会主义国家早有先例。

我的用意在避免大地主囤积房產的同时,容许中差阶级储蓄,所以一个房东的第一个房租免税或降税也可以,或者不设在50\%,而是跳造累进税率的高层。税法天生就需要很多微调,但是大方向对了,就该尽量去做,不能因为一句话说不清就放弃。\\

\textit{\hfill\noindent\small 2015/07/07 00:00 提问; 回答}

\noindent[8.]{\Hei 答}:讲阴谋论总是给人一种疯狂的感觉,但是这40年来美国人越做越离谱,阴谋论往往还没有事实那么邪恶。例如Snowden揭露NSA的内幕之前,有谁敢猜美国会试图对全世界的每一通电话做监听?又如十几年前,加州尝试把电力市场对私人企业半开放,结果马上断电频仍,州营的电力公司反而破產;一开始大家怪州政府,后来才知道是Enron收买发电厂职员,故意制造电力缺乏,谋取暴利。

我对中方公务员的退休津贴不熟,不过用退休俸来弥补以往薪水的不足,反而给反对派断章取义的藉口,台湾是前车之鑑。\\

\textit{\hfill\noindent\small 2015/07/08 00:00 提问; 回答}

\noindent[9.]{\Hei 答}:我觉得自由股市是大户宰小户的屠宰场,只有在炒作所得和交易频率上都下重税才能保护它做为长期投资的管道。\\

\textit{\hfill\noindent\small 2015/07/08 00:00 提问; 回答}

\noindent[10.]{\Hei 答}:大资本列管是对的,但是现在大陆也有一堆迷信美式经济学的\&ldquo;专家\&rdquo;,一天到晚鼓吹金融自由化;只怕连中共也被国际资本的宣传部门给\&ldquo;统战\&rdquo;掉了。\\

\textit{\hfill\noindent\small 2015/07/08 00:00 提问; 回答}

\noindent[11.]{\Hei 答}:我也是很失望。散户根本就不应该被允许用杠杆;那是美国财团搜刮人民财富的手段,中共照单全收是很蠢的。

李克强政府只怕执行力不足以满足习近平为中国所设的目标。\\

\textit{\hfill\noindent\small 2015/07/13 00:00 提问; 回答}

\noindent[12.]{\Hei 答}:只要是芝加哥大学出身的经济学博士都是国际财阀的文字打手,但愿全球的政界领导都能有此觉悟。\\

\textit{\hfill\noindent\small 2015/07/13 00:00 提问; 回答}

\noindent[13.]{\Hei 答}:My biggest surprise is that they even have a willing market in China! This latest round of stock market volatility is exactly the result of regulations which encourage it. Volatility brings profits only to the big money by hurting everyone else.\\

\textit{\hfill\noindent\small 2015/08/15 00:00 提问; 回答}

\noindent[14.]{\Hei 答}:我认为是前者。\\

\textit{\hfill\noindent\small 2015/08/20 00:00 提问; 回答}

\noindent[15.]{\Hei 答}:经济是一切进步的根基,军事科技什么的,都只是经济发展的结果,所以我一直说贫富不均才是人类在21世纪的头号问题。和全球暖化一样,中国都首当其衝,习近平的《中国制造2025》解决的只是中期的產业升级问题,资本集中、贫富不均这个长期问题其实更难解决,也更应该提早着手,只是李克强执行力不行,又有一堆被美国经济学洗脑过的愚蠢幕僚,只怕5年之内不会有任何动作。\\

\textit{\hfill\noindent\small 2015/08/20 00:00 提问; 回答}

\noindent[16.]{\Hei 答}:我觉得他太悲观了,他看到的那些问题是真的,但是它们是长期的慢性病,短期爆发的可能性反而不大。不幸的是,正因如此,李克强自以为这次还是可以蒙混过关,那么长期的改革就又拖下去,资本集中、贫富不均的问题也就越养越大,就算美国一直没有办法藉此占便宜,最后仍然会成为中国经济发展、社会进步的终极阻碍。\\

\textit{\hfill\noindent\small 2015/08/20 00:00 提问; 回答}

\noindent[17.]{\Hei 答}:他对财政和税务的了解,要比我深刻得多了,佩服佩服。只是李克强採纳他意见的机率基本是零。\\

\textit{\hfill\noindent\small 2015/08/21 00:00 提问; 回答}

\noindent[18.]{\Hei 答}:我不是税制方面的专家。不过主政的权力核心至少应该有基本概念,知道必须避免把问题弄糟;例如这次要鼓励消费、刺激经济,股市就是一个极糟糕的管道,纯粹帮助大户搜刮更多的资本,外逃也就更容易。\\

\textit{\hfill\noindent\small 2015/10/07 00:00 提问; 回答}

\noindent[19.]{\Hei 答}:人民银行的政策不是很灵活,前两年开始买黄金之后一直延续下来,并没有变更的迹象。\\

\textit{\hfill\noindent\small 2015/10/07 00:00 提问; 回答}

\noindent[20.]{\Hei 答}:人民银行在2000年到2010年间没有大量收购黄金,当时我就纳闷,现在可以确定是是极大的错误。\\

\textit{\hfill\noindent\small 2015/10/07 00:00 提问; 回答}

\noindent[21.]{\Hei 答}:我想他指的是人民银行在十月7日上午公布的官方数据。\\

\textit{\hfill\noindent\small 2015/10/07 00:00 提问; 回答}

\noindent[22.]{\Hei 答}:如果他一贯预测熊市,那么已经失去客观理性的态度,新的预言也就没有意义。\\

\textit{\hfill\noindent\small 2015/10/07 00:00 提问; 回答}

\noindent[23.]{\Hei 答}:我认为外匯上或许会关闸,但是应该不是崩溃的前兆,纯粹只是让Soros之流死心。\\

\textit{\hfill\noindent\small 2015/10/07 00:00 提问; 回答}

\noindent[24.]{\Hei 答}:我对香港不熟,只能说经济长期不看好,短期没有意见。

对中国的货币政策来说,我对周小川还是有些信心的,所以不认同卢先生的看法。美国人在有机可乘的情形下会手下留情更是匪夷所思。不过货币政策和我的专业(股市)还是有一点点距离的,或许他的预言才是对的吧?我们再等半年便知分晓。\\

\textit{\hfill\noindent\small 2015/10/13 00:00 提问; 回答}

\noindent[25.]{\Hei 答}:我也很佩服那篇历史税政的分析。\\

\textit{\hfill\noindent\small 2015/10/13 00:00 提问; 回答}

\noindent[26.]{\Hei 答}:执行细节或许如此,大目标却是有其一贯性的,毕竟发展经济的最终目的是为人群谋福利。当前美式经济和中国最大的不同,在于前者专为极富的一小部分人群服务,后者的理想却是贯注在全​​民身上的。\\

\textit{\hfill\noindent\small 2015/10/13 00:00 提问; 回答}

\noindent[27.]{\Hei 答}:我也觉得他的部分论点太过偏颇极端了,不过抑制投机和保护群眾的大方向还是对的。\\

\textit{\hfill\noindent\small 2015/10/13 00:00 提问; 回答}

\noindent[28.]{\Hei 答}:这一方面我已经在《谈通货膨胀》里做过阐述。货币是用来购买服务或物资的,所以它的膨胀或收缩都必须是相对于服务或物资而来谈。既然服务或物资有好几个大类,供需平衡状况往往不同,那么通货膨胀或收缩就不能一概而论。美式经济学不分辨这些细节应该是有意蒙混,以便图利财阀。\\

\textit{\hfill\noindent\small 2015/10/13 00:00 提问; 回答}

\noindent[29.]{\Hei 答}:美国经济学的问题是,除了三四个有良心的大师之外,其他不是甘心同流合污,就是为虎做伥而不自知。\\

\textit{\hfill\noindent\small 2015/11/29 00:00 提问; 回答}

\noindent[30.]{\Hei 答}:我也觉得习近平这一任不可能有时间管到这事。

德国的均富程度比美国好多了。他的房地產政策想来是比较值得参考的。\\

\textit{\hfill\noindent\small 2017/09/10 00:00 提问; 回答}

\noindent[31.]{\Hei 答}:我在视频访问中评论过了。

等搬家安定下来,会另写一篇专稿。\\

\textit{\hfill\noindent\small 2020/07/17 07:36 提问;2020/07/17 09:58 回答}

\noindent[32.]{\Hei 答}:這個議題我已經評論過了:我認爲現在完全開放金融的風險大於收穫,反正當前真正的戰略重點在於拉下美元,這和扶持人民幣成爲國際儲備貨幣是兩回事。
\\

\textit{\hfill\noindent\small 2020/07/18 10:17 提问;2020/07/19 11:07 回答}

\noindent[33.]{\Hei 答}:這裏的問題在於,英美真正懂金融的都是銀行或對衝基金的玩家,學術界純粹是為他們欺騙大衆、解脫監管的傳聲筒,論文和書籍實際上都是擦屁股用的。然而中國送一堆留學生拿了博士學位回國當教授,還自以爲高人一等,結果教的都是如何方便大戶割韭菜。基本理論就是歪的,怎麽能指望執行的政策細節考慮周全呢?

其實幾年前我已經把金融管理的頭號秘訣寫下來了,這裏再重複一次:金融必須越單調無聊越好,只要有人大賠或大賺,或者高管薪水高漲,監管單位就有失職。
\\

\textit{\hfill\noindent\small 2021/01/23 02:52 提问;2021/01/23 07:21 回答}

\noindent[34.]{\Hei 答}:這裏的問題在於很多中國中產階級在房產上吃了很大的甜頭,誤以爲所有金融投資都應該有類似的報酬率;其實剛好相反,中國房產在過去20年的驚人增值,是經濟成長大環境下的極端特例,基本不可能複製。如果大家都把股市當賭場,對整體經濟是個大災害。
\\

\textit{\hfill\noindent\small 2021/04/25 21:08 提问;2021/04/26 02:55 回答}

\noindent[35.]{\Hei 答}:是的,我以前已經做出同樣的判斷過。

這裏的難處,在於打擊陽奉陰違的地方官員,所以這次深圳的案件是非常正面的發展。
\\


\section{【陆军】共军的超级大炮}
\subsection{2015-07-06 10:14}


\section{1条问答}

\textit{\hfill\noindent\small 2015/07/10 00:00 提问; 回答}

\noindent[1.]{\Hei 答}:是如此。但是胡温任期内不做改革,美国却也自顾不暇,已经算是运气了。\\


\section{【台湾】如果我是总统候选人}
\subsection{2015-07-06 15:33}


\section{10条问答}

\textit{\hfill\noindent\small 2015/07/07 00:00 提问; 回答}

\noindent[1.]{\Hei 答}:很可惜的是,20多年的戒急用忍下来,台湾传统的制造產业已经被中国赶上了,却又大多没有插入中方的產业链的核心(其实即使像联发科这样彻底插入的,长期之后也没有用,因为台湾政坛以反中为正统,中共不会放心让台企掐着他產业链的脖子),所以中国做下一波產业升级的时候,台企首当其衝,必然会倒下一大片。

如果下一任台湾政府能力挽狂澜,重建与中国的良性互动,那么联发科或许可以反过来成为世界级企业,如同华为和中兴一样,不过其他的中型电子企业如日月光,和机床业还是死定了。

长期来看,台湾必须做大陆不想也不能全拿的东西,也就是服务业,尤其是生技医疗,大陆还很落伍,而台湾有相同的中医传统,这将是一个极大的市场,但是政治上不先建立条件,最终仍将是画饼一块。\\

\textit{\hfill\noindent\small 2015/07/08 00:00 提问; 回答}

\noindent[2.]{\Hei 答}:\&rdquo;还有什么比没钱没资源更被动吗?\&ldquo;正是我一再试图解释的。\\

\textit{\hfill\noindent\small 2015/07/08 00:00 提问; 回答}

\noindent[3.]{\Hei 答}:其实即使假设台湾在军事上有抗拒武统的能力(实际上连门儿都没有),在经济上也挡不住大陆的崛起,那么採取敌对态势的结果就是鼓励中方做替代式的竞争。连新加坡这种经济层次高台湾一级的国家,都不敢这样做(李光耀可也是从心眼底儿反共的),台湾过去这20多年真是努力割自己的喉咙。\\

\textit{\hfill\noindent\small 2015/07/08 00:00 提问; 回答}

\noindent[4.]{\Hei 答}:唉,对个人来说,去大陆就业是理性的抉择,但是对整体台湾社会,这是人才的流失。不过这责任是在创造就业环境的政府和人民身上,个人只能对自己的前途负责。\\

\textit{\hfill\noindent\small 2015/07/13 00:00 提问; 回答}

\noindent[5.]{\Hei 答}:是的,所以把那些美国支助的维权律师拿掉,是为全民着想的未雨绸缪。\\

\textit{\hfill\noindent\small 2015/07/14 00:00 提问; 回答}

\noindent[6.]{\Hei 答}:完全同意你对王金平的看法。我对国民党是很悲观的。

大陆的股市可能不是美国财团直接操控的;而是因为芝加哥系的美粉把制度搞成美国式的大户刮小户的乐园,以致中国自己的财团杀得不亦乐乎。\\

\textit{\hfill\noindent\small 2015/07/21 00:00 提问; 回答}

\noindent[7.]{\Hei 答}:台湾在国际上能拿出手的高科技工业只有两个:半导体和机床,但是这里面大部分还是以低价为核心竞争力的中型企业,我预计五年左右就会被全部打翻了。例外的有三家公司,就是台积电、联发科和鸿海。鸿海靠的其实不是技术,而是组织和经营能力,雇员又多不在台湾,所以可以忽略不计。联发科是唯一全面嵌入红色產业链的台湾高科技公司,几年之内应该还有不少成长余地,不过十年之后就要看中国是否要扶持自己的產业了。

台积电不\&ldquo;只是代工\&rdquo;,先进制程的智慧產权才是它的核心价值。目前大陆最先进的制程是台积电十年前就有的,但是中方和比利时签约要在2020年引进14奈米制程,这是台积电今年年底才能量產的东西,差距就只剩五年了。依这个速度,在2025年左右,大陆会赶上台积电,那也就是台湾制造业的末日,就算联发科还在,它的规模和產值都比台积电小太多,独木难撑大厦。\\

\textit{\hfill\noindent\small 2015/07/21 00:00 提问; 回答}

\noindent[8.]{\Hei 答}:他说的并非没有道理,中西亚的乱象的确代表着\&ldquo;一带一路\&rdquo;有其极限,一般媒体也的确是有很多浮夸的地方,例如欧亚高铁我也一直认为是不切实际的胡扯。

不过我不认为\&ldquo;一带一路\&rdquo;整体上是个忽悠的壳子;那些浮夸忽悠的言论并不来自中共,而是媒体自行扯出来的。欧亚高铁不切实际,但是欧亚货运铁路线却是有真正价值的新发展。\&ldquo;一带\&rdquo;是把俄国和中亚拉进中国经济引力圈的招牌;\&ldquo;一路\&rdquo;则包括东南亚、南亚、中东、南欧和东欧,都是中共外交、经济发展的重要方向。那么没有包括进去的东亚国家只有日本、台湾和韩国;前两者本来就是捣蛋鬼,后者则刚好相反,已经签了自贸协定。

中共自己也一直努力否认\&ldquo;一带一路\&rdquo;是马歇尔计划,不过媒体这样渲染,反而鼓励路上的各国热情参与、认真对中企开放,对中方的基建投资和战略路线开辟都有好处。实际上中共当局并不会做赔本生意,那位学者大概是多虑了。\\

\textit{\hfill\noindent\small 2015/07/21 00:00 提问; 回答}

\noindent[9.]{\Hei 答}:在一般的国际距离上,航空仍是最有效率的交通工具。因此我不认为欧亚高铁会有经济效益。美国虽然不在欧亚大陆,也因此而不会被孤立。\\

\textit{\hfill\noindent\small 2015/07/23 00:00 提问; 回答}

\noindent[10.]{\Hei 答}:因缘时会,只上了一次节目;刚好话题是我以前写过的,所以不觉之间就跳过论证的过程而直接讲到结论上了。我还是喜欢慢慢写,可以深思熟虑;开口就讲不是我的所长。\\


\section{【战略】再谈希腊与欧元}
\subsection{2015-07-20 13:29}


\section{20条问答}

\textit{\hfill\noindent\small 2015/07/20 00:00 提问; 回答}

\noindent[1.]{\Hei 答}:减债必须等几年,让德国选民忘记现在的激情,最好是在Merkel退休之后。\\

\textit{\hfill\noindent\small 2015/07/20 00:00 提问; 回答}

\noindent[2.]{\Hei 答}:希腊没有工业化就加入欧元区,等于是自愿成为德国人的市场和度假区,先天上经济就很脆弱。

服务业是枝叶,制造业才是根干,除了能无限印钞票的美国之外,没了制造业就必须成为大国的附庸。所以台湾除了和大陆整合之外,的确是没有其他的经济路线可走,除非台湾的工业能打败韩国和中国的夹击(和中彩卷的机率差不多)。\\

\textit{\hfill\noindent\small 2015/07/21 00:00 提问; 回答}

\noindent[3.]{\Hei 答}:我的看法是这样的:希腊和台湾有其不同,首先台湾人不像希腊人那么懒,其次希腊是自愿放弃制造业的,台湾只是丧失了竞争力。不过这些是经济的远因,在财政的近因上,两者非常相似:同样为了民粹,政府入不敷出,并且不顾產业升级。

你说的\&ldquo;大陆肯定不会袖手旁观\&rdquo;,其实那不可能是重建制造业,所以只能是支持台湾的农渔牧和服务业,刚刚好就是真正的希腊化。换句话说,希腊至少还保证有德国这个顾客;台湾要是再胡搞下去,连希腊都不如。\\

\textit{\hfill\noindent\small 2015/07/21 00:00 提问; 回答}

\noindent[4.]{\Hei 答}:很好,正是我这一年来一直解释的。希望所有读者都已经有了同样的心得。

蔡英文当然知道\&ldquo;摆脱对大陆的经济依赖\&rdquo;是痴人说梦,开会只是哄选民的手段。\\

\textit{\hfill\noindent\small 2015/07/23 00:00 提问; 回答}

\noindent[5.]{\Hei 答}:我想我在《民主政治和自由经济》和其后的一些文章里,已经试图解答这个问题。很不幸的是不但大部分新兴国家根本没有实行成功民主制度的条件,连以往似乎成功的国家也被逐步腐化。在实践方面,民主制度的最大毛病似乎是其没有很强的自清能力,政客可以很容易地不干正事,而还是让选民自我感觉良好。\\

\textit{\hfill\noindent\small 2015/07/23 00:00 提问; 回答}

\noindent[6.]{\Hei 答}:大规模的合作组织是人类进化的基础,它的作用靠的是详细分工而得来的更高效率,所以你父亲的态度是正确的。一个政治体制应该能让老百姓专心赚钱才对。\\

\textit{\hfill\noindent\small 2015/07/23 00:00 提问; 回答}

\noindent[7.]{\Hei 答}:你提出的这个差别是对的。台湾或许不会像希腊,但是有可能成为阿根廷。你说的那些改革,在台湾政坛根本没人在乎,所以也就没有可能通过。\\

\textit{\hfill\noindent\small 2015/07/26 00:00 提问; 回答}

\noindent[8.]{\Hei 答}:大陆正在以惊人的速度做软硬体的建设(\&ldquo;软体\&rdquo;包括人民素质),这是毋庸置疑的。我只是生性慵懒,素来不喜欢旅行,出门都是为了家人需要。\\

\textit{\hfill\noindent\small 2015/07/27 00:00 提问; 回答}

\noindent[9.]{\Hei 答}:原因之一。中共也不吃美国的那套宣传,有自己的思想理论体系,帮助很大。\\

\textit{\hfill\noindent\small 2015/07/27 00:00 提问; 回答}

\noindent[10.]{\Hei 答}:谢谢你的提醒。

胡温政绩的讨论到此为止。\\

\textit{\hfill\noindent\small 2015/07/27 00:00 提问; 回答}

\noindent[11.]{\Hei 答}:胡的个人品格似乎是不错的,但是我们讨论的领导阶级的责任。无论多么困难,他是总书记,改革停滞的责任终究在他身上。\\

\textit{\hfill\noindent\small 2015/07/27 00:00 提问; 回答}

\noindent[12.]{\Hei 答}:基层生活水准改善是基于几年前、甚至十几年年底政策改革。GPD成长和生活水平提高虽然发生在胡温任内,却不是他们的功劳;否则他们所作的改革应该是很容易列举的。\\

\textit{\hfill\noindent\small 2015/07/27 00:00 提问; 回答}

\noindent[13.]{\Hei 答}:正是。朱镕基是承接邓小平的改革大功臣。\\

\textit{\hfill\noindent\small 2015/07/27 00:00 提问; 回答}

\noindent[14.]{\Hei 答}:我已经说过了,GDP成长率可以全凭运气,不代表能力。

你还是不能举例,讲一些空泛的打分估计是小学生的论调,毫无意义。\\

\textit{\hfill\noindent\small 2015/07/27 00:00 提问; 回答}

\noindent[15.]{\Hei 答}:中国富起来还是最近的事;在经济高度成长的过程中,阶级间的流动自然会增大。在经济成长减缓之后,阶级间的流动性是否能保持还很难说,所以讨论其原因也就言之过早。

我个人觉得这会是很严重的问题,中共必须出台针对性的政策。当然习近平已经有很多更急迫的问题要处理,但是阶级间的流动性在长期来看,至少是同等重要的。\\

\textit{\hfill\noindent\small 2015/07/27 00:00 提问; 回答}

\noindent[16.]{\Hei 答}:福将就是没有能力,只有运气。运气就是随机,完全没有意义。\\

\textit{\hfill\noindent\small 2015/07/28 00:00 提问; 回答}

\noindent[17.]{\Hei 答}:美国人在2008年拿7000亿美元救市的时候,IMF在旁讚美有加。治国必须靠逻辑,而不是外国经济学人。\\

\textit{\hfill\noindent\small 2015/12/14 00:00 提问; 回答}

\noindent[18.]{\Hei 答}:难民这事有很多无辜的受害者,还是不要拿来説笑吧。

希腊财务的根本问题是公务员太多、福利又太好,尤其是退休金高的离谱。结果国家被搞得如此困难,弱势群体都活不下去了,唯一做不到的改革就是削减公务员退休金。\\

\textit{\hfill\noindent\small 2015/12/17 00:00 提问; 回答}

\noindent[19.]{\Hei 答}:Yes, I don't usually pay attention to celebrities. Their opinion is just that: their opinion.\\

\textit{\hfill\noindent\small 2017/05/30 00:00 提问; 回答}

\noindent[20.]{\Hei 答}:而且欧洲已经急到威胁要拿掉美国在IMF的否决权。\\


\section{【基礎科研】人類的起源}
\subsection{2015-07-23 15:50}


\section{10条问答}

\textit{\hfill\noindent\small 2015/07/23 00:00 提问; 回答}

\noindent[1.]{\Hei 答}:Their birth rate is twice as high as Taiwan. Besides, the bureaucracy there should be capable of making some changes, i.e. self-correct.\\

\textit{\hfill\noindent\small 2015/07/24 00:00 提问; 回答}

\noindent[2.]{\Hei 答}:其实台湾过去10年贫富差距严重恶化和经济发展减缓,不但有共同的政治原因,彼此间也有互相影响(当然前者受后者影响比较多)。\\

\textit{\hfill\noindent\small 2015/07/25 00:00 提问; 回答}

\noindent[3.]{\Hei 答}:欧洲的分裂状态或许鼓励国际竞争,从而间接导致了美洲的发现(但是前提还是大西洋不能像太平洋那么宽),然后才能有工业革命,但是21世纪已经没有新大陆,看来会是大者通吃的竞赛。我们还是专注在促进大规模的合作吧。\\

\textit{\hfill\noindent\small 2015/07/25 00:00 提问; 回答}

\noindent[4.]{\Hei 答}:是的。经济和技术发展是人类进步的真正标尺,而它靠的是大规模的合作组织能力。中国以前只专注在合作组织,却没有眼光或欲望把这个组织能力用在发展经济和技术上,所以输了一回合。现在卷土重来,自然不可同日而语。\\

\textit{\hfill\noindent\small 2015/07/26 00:00 提问; 回答}

\noindent[5.]{\Hei 答}:你说的没错。我原本以为这些都是基本的经济学常识,所以没有详细阐述过,只是常常把它们当作我论述的隐性前提。不过现在回想起来,不见得每位读者都对经济学熟悉,所以有人把它整理出来是件好事。

重点是在没有革命性的新资源或土地(例如发现美洲)的情形下,大家都执行合理策略时,规模领先者有极大优势。大陆在过去30年靠着新的人力资源(从农村解放出来加入国际產业链)而赶上了欧美,现在正在巩固这个新优势。周边的小国即使执行最合理的政策来维护自己的先进產业,都尚且不一定能成功(如芬兰),那么不断自行割喉的台湾下场如何可想而知。\\

\textit{\hfill\noindent\small 2015/07/26 00:00 提问; 回答}

\noindent[6.]{\Hei 答}:中国太大了,在可见的未来都一直会有低开发地区。中共的政策很明显地是除了最低阶的制造业以外,从中低到最高级的工业都要通吃。我们等15-20年看看他能不能成功。\\

\textit{\hfill\noindent\small 2015/07/27 00:00 提问; 回答}

\noindent[7.]{\Hei 答}:其实三大师(医师/律师/会计师)都只是技术人员,钱赚得多而已,在智慧上和对社会的贡献上都很平庸,律师尤其如此。\\

\textit{\hfill\noindent\small 2015/07/27 00:00 提问; 回答}

\noindent[8.]{\Hei 答}:一般的现象是富有之后,人民只追逐酒色财气,在文化和教育上也只追求赚钱,没有深刻的思想,所以其优越感也就更加可笑。\\

\textit{\hfill\noindent\small 2015/07/28 00:00 提问; 回答}

\noindent[9.]{\Hei 答}:问题是全民贫富差距越来越大;等到达到你所说的自然平衡时,已经有超过一半人口陷入赤贫(然后才会关心吃饭问题)。我认为那是不可忍受的。\\

\textit{\hfill\noindent\small 2015/07/29 00:00 提问; 回答}

\noindent[10.]{\Hei 答}:律师的GDP贡献其实应该乘上一个负号;不过我在前文《谈GDP数字的局限性》中只设它为零。\\


\section{【美国】美国式的恐龙法官(二)}
\subsection{2015-07-30 13:27}


\section{2条问答}

\textit{\hfill\noindent\small 2015/08/02 00:00 提问; 回答}

\noindent[1.]{\Hei 答}:你说的这个十次预测都中的骗术,据我所知是一百年前在美国被股票分析师发明的。现在的对衝基金公司往往一次开十几二十个基金,第一年输钱的就关掉,重开新的替换,所以他们的记录都很好看,其实是老骗术。

\\

\textit{\hfill\noindent\small 2015/11/30 00:00 提问; 回答}

\noindent[2.]{\Hei 答}:我想是欧美的人工太贵,即使没有本地的需求,生產的成本还是压不下来。

此外这些动物油应该是可以加工后做工业原料的,例如化妆品。\\


\section{【陆军】【海军】共军小道消息刷新(2015年第三季)}
\subsection{2015-08-02 20:23}


\section{3条问答}

\textit{\hfill\noindent\small 2015/08/03 00:00 提问; 回答}

\noindent[1.]{\Hei 答}:连暴民都不敢处理,经济大局哪能照顾?反正快饿死时,还可以学希腊办个公投,让大家空着肚子在街头庆祝。\\

\textit{\hfill\noindent\small 2015/08/05 00:00 提问; 回答}

\noindent[2.]{\Hei 答}:中国时报的网站没有移动留言到别的文章的选项,所以留言者必须自制。我已经要求大家到此为止,你置若罔闻甚是无礼。

看你写得辛苦,这次不删,下不为例。\\

\textit{\hfill\noindent\small 2015/08/05 00:00 提问; 回答}

\noindent[3.]{\Hei 答}:See the earlier article 《2030年左右》for China's commitment towards reducing carbon emission.

The US rules the world with an iron fist. If China does not build up a reasonable military, it is sure to be taken advantage of. You are simply falling for American propaganda that blames the victims for America's own aggression.\\


\section{【陆军】未来十年的中美武器对比(一)}
\subsection{2015-08-08 00:05}


\section{1条问答}

\textit{\hfill\noindent\small 2015/08/08 00:00 提问; 回答}

\noindent[1.]{\Hei 答}:美国维持高军费对我这样的纳税人来说是灾难性的;至于\&ldquo;砸烂旧世界\&rdquo;,欧巴马已经花了七年来做战略紧缩,下任总统应该也会继续。\\


\section{【空军】未来十年的中美武器对比(二)}
\subsection{2015-08-08 22:25}


\section{1条问答}

\textit{\hfill\noindent\small 2021/04/20 01:57 提问;2021/04/20 06:10 回答}

\noindent[1.]{\Hei 答}:美國權力精英公然無恥的自私自利,始於Milton Friedman的歪論,經由Reagan發揚光大,現在早已司空見慣、習以爲常。所謂的大家為私利奮鬥,自由市場自然會把公益最大化,當然只在霸權無敵的前提下,才可能維持私利、公益兼顧的假象;實際上是消耗祖產(小羅斯福遺留下來的全面霸權),再豐厚也有耗盡的一天。

我在1990年代剛轉金融,首席交易員就叫我去讀《Atlas Shrugged》,把我嚇了一跳,原來這麽明顯的胡扯,只要和自我利益重叠,聰明的人也會自願受洗腦。從那時起,我就常常引用Upton Sinclair的那句話:“It is difficult for a man to understand something, when his salary depends on his not understanding it.”
\\


\section{【海军】未来十年的中美武器对比(三)}
\subsection{2015-08-09 21:52}


\section{3条问答}

\textit{\hfill\noindent\small 2015/08/13 00:00 提问; 回答}

\noindent[1.]{\Hei 答}:这次天津爆炸,也是有很多人编谎话散布对自己无意义的谣言。人类是群居动物,撒谎是操弄其他社会成员的手段,在演化的过程中也就被留下来了;但是诚实对族群整体是较有利的,所以撒谎基因只占了总人口的一部分。

在现代社会里,大规模合作是效率和进步的来源,诚实和理性才是应该被鼓励的行为模式。\\

\textit{\hfill\noindent\small 2015/09/03 00:00 提问; 回答}

\noindent[2.]{\Hei 答}:人类工业化的过程只有两种成功的案例:国家主导,或是财阀主导,都是要求极度集权的。市场经济必然会產生超大型公司,而超大型公司不但完全违反民主原则,连市场经济的基本公理都无法满足,这个内建的矛盾是我以前已经讨论过的。\\

\textit{\hfill\noindent\small 2015/09/15 00:00 提问; 回答}

\noindent[3.]{\Hei 答}:核战若是打起来,没有什么谁占领谁的问题,两方都已被踢下世界文明舞台,航母这样的战术兵力真是无关轻重了。\\


\section{【台湾】愚民主政下的指鹿为马}
\subsection{2015-08-14 19:23}


\section{5条问答}

\textit{\hfill\noindent\small 2015/08/15 00:00 提问; 回答}

\noindent[1.]{\Hei 答}:Were it not for the big money waiting to take advantage of the situation, depreciation would have been a natural move. Now there is substantial risk.\\

\textit{\hfill\noindent\small 2015/08/15 00:00 提问; 回答}

\noindent[2.]{\Hei 答}:这是一个很大的契机,但是前提是中国菁英自己不被美国宣传忽悠去了。\\

\textit{\hfill\noindent\small 2015/08/21 00:00 提问; 回答}

\noindent[3.]{\Hei 答}:我在税制和财政上不是专业,只能从金融和经济的角度上评论。中国经济在2008年之后,就是全世界的火车头(没有之一),现在严重减速,世界经济自然会受到很不利的影响。真正的问题还是在中共自身,也就是李克强的团队是否能拟定并执行正确的政策。

原本这两年中国经济的减速是必然的,是为胡温十年因循无改革的欠债来买单。不过管理单位前一阵子操弄股市的笨拙表现,让我很失望,连带着对他们的匯率政策也不能再有绝对的信任。

我再想想吧。若是有具体的看法能在一篇短文解释清楚的再写出来。\\

\textit{\hfill\noindent\small 2015/08/28 00:00 提问; 回答}

\noindent[4.]{\Hei 答}:美国现在还没有到达欧洲一战前贫富不均的程度,却已经超过了自己在一战前的程度,所以美式制度其实已经破產,只靠美元霸权在硬撑表面。\\

\textit{\hfill\noindent\small 2015/08/30 00:00 提问; 回答}

\noindent[5.]{\Hei 答}:It is very easy for the privileged to convince themselves that they win on merits though, even if objective observers can see plainly that it was all luck and selfishness. Case in point, Donald Trump.\\


\section{【美国】谈一个河流污染事件}
\subsection{2015-08-19 04:39}


\section{4条问答}

\textit{\hfill\noindent\small 2015/08/19 00:00 提问; 回答}

\noindent[1.]{\Hei 答}:他们的文章不署名,是170多年来的传统,倒不是群体创作的关系。

最近两年他们的北京站似乎有一个记者愿意写持平的文章,其他人则继续仇中,所以我基本上可以猜出一篇文章是否是那个记者写的。\\

\textit{\hfill\noindent\small 2015/08/20 00:00 提问; 回答}

\noindent[2.]{\Hei 答}:台湾在80年代也开始吸收留学生回国创业,才有了台积电等国际性的高科技產业;希望《中国制造2025》能成功地复制那段经验。\\

\textit{\hfill\noindent\small 2015/08/20 00:00 提问; 回答}

\noindent[3.]{\Hei 答}:一般百姓的自信是国家强大的后果;菁英的自信却是排除美国宣传毒素、选择理性道路的先决条件。所以其实公信力欠缺的问题并不急迫,可以慢慢解决,这也是为什么我到现在才写了一篇文章的原因。真正紧急的是政策方向必须是理性前瞻的,不能让美国经济学的歪论左右;我以前已经多次着墨了。\\

\textit{\hfill\noindent\small 2015/08/31 00:00 提问; 回答}

\noindent[4.]{\Hei 答}:有能力做独立思考的人会当律师吗?英美法系一切论点都看案例,这是因循传统推到极限,所以他们的思维模式自然是制度如此,它就(至少表面上公开上)是对的。要听理性的声音,只有像福山和Stiglitz这样有自尊心的独立学者还有可能。\\


\section{【经济】世界经济未来走向}
\subsection{2015-08-22 22:54}


\section{58条问答}

\textit{\hfill\noindent\small 2015/08/23 00:00 提问; 回答}

\noindent[1.]{\Hei 答}:International capital has achieved total victory in the US and is now fanning out over the globe to invade other governments. It is sad to see no safe haven is left.\\

\textit{\hfill\noindent\small 2015/08/23 00:00 提问; 回答}

\noindent[2.]{\Hei 答}:You know that  NZ is one of the "resource exporting countries" I talked about, right? The economy is going to get tough really quickly, and I think big businesses are going to mobilize the media to blame it on activists. Good luck on fighting for a good cause.\\

\textit{\hfill\noindent\small 2015/08/23 00:00 提问; 回答}

\noindent[3.]{\Hei 答}:情绪反应和民族仇恨都是非理性的。

日本的外匯存底相当于GDP的35\%,两年的赤字就吃光了。

政府破產和民间财富没有关系。日本国债正是由邮政储蓄系统撑起来的,10-15年后还不出来,人民的储蓄会被\&ldquo;理髮\&rdquo;(\&ldquo;Haircut\&rdquo;)。或许日本人会静静承受吧;其他国家的人民是必然会起来暴动的。\\

\textit{\hfill\noindent\small 2015/08/23 00:00 提问; 回答}

\noindent[4.]{\Hei 答}:我并不是说股市的问题比房市严重,而是它比较明显地揭露主管人员的道德和职业缺陷。

股票是虚拟财產,管理起来比房地產容易得多;中共主管的表现实在令人心寒。\\

\textit{\hfill\noindent\small 2015/08/23 00:00 提问; 回答}

\noindent[5.]{\Hei 答}:日本自己问题这么大,却还到处放火,与唯一能挽救他的经济力量做对,也真是世界奇观,和台湾相互呼应;或许日台在蠢字上真的有特殊的联繫。\\

\textit{\hfill\noindent\small 2015/08/23 00:00 提问; 回答}

\noindent[6.]{\Hei 答}:美国的地產税是镇的税,而办学校是镇级的职能,所以也可以想成是学区税。例如我住的这个镇,每年的地產税是房地產价值的1/40,其中2/3是学区的预算。

以康州的标准来看,1/40是偏低的,附近的几个镇税率从1/30到1/20都有。

房地產并不是真正的生產,其价值主要来自地段,也就是周边环境,所以是一种寄生性的產业。价钱一高就成为资本家搜刮的手段,在这方面中国应该要比美国这种赤裸的财阀专政要进步些才对。\\

\textit{\hfill\noindent\small 2015/08/23 00:00 提问; 回答}

\noindent[7.]{\Hei 答}:是的,这正是我的看法。

最近很多人在预言中共的大崩溃,反过来也有些华裔的评论员认为美元的失势才会造成美国的大崩溃;其实我觉得中美有病也是慢性的,底子又厚,急性崩溃的机率不大。真正会摔下悬崖的恰是日本,而且基本上危机不可避免(因为国债必然会达到400\%的临界点),不确定的只是爆发的方式。\\

\textit{\hfill\noindent\small 2015/08/23 00:00 提问; 回答}

\noindent[8.]{\Hei 答}:我只看了第一个视频(第二个不能跳前)。美元将失势是对的,但是美国和日本不一样,2040年前国债利息到达税收50\%以上的预言必须假设税制不变,可是美国的税率现在正处于百年来的新低,这是因为财阀在过去40年控制政府之后最优先的事就是减税。如果美国愿意把税率调回60年代的水准,国债很快就还清了。

当然,财阀不会接受高税率,但是目前中產阶级的税率也不高,以避免人民不满。等国债问题真正严重,中產阶级就会被领出去刮毛了。

还有,他说的国债利息超过税收50\%以上国家就灭亡是明显错误的,这个比例在最近的日本已经超过55\%。他一定知道,却故意不提,不是诚实的态度。\\

\textit{\hfill\noindent\small 2015/08/23 00:00 提问; 回答}

\noindent[9.]{\Hei 答}:他是真不懂经济,只因为他的智囊都来自美国芝加哥学派,所以那些理论先天就是图利财阀的,倒不是他有意而为。

日本大概还可以撑10-15年,等国债涨到GDP的400\%以上(今年底会达到240\%)时,全年税收连付利息都不够,那时再看日本人能如何胡扯。\\

\textit{\hfill\noindent\small 2015/08/23 00:00 提问; 回答}

\noindent[10.]{\Hei 答}:台湾人只会短线操作;日本人有技术,台湾有低成本,短期上是天作之合。但是台湾不做大规模的研发,长期下来等日本人吃空了老本,台湾就连麵包屑都没得吃了。\\

\textit{\hfill\noindent\small 2015/08/23 00:00 提问; 回答}

\noindent[11.]{\Hei 答}:我想李克强政府已经着手在办,可惜是就地\&ldquo;市场化\&rdquo;,所以必然是有价值的被抢走,无价值的留给国家。反之,若是由央行直接买下,监督管理的步骤就无法进行,这是留下烂帐的官员最希望发生的。

正确的做法是设立一个全国性的专业机构来统筹消化这些债务,至于这个机构叫做某某局、基金还是银行都不重要。最重要的是必须在两三年内决定各笔债的真实价值,并且依此究责。\\

\textit{\hfill\noindent\small 2015/08/23 00:00 提问; 回答}

\noindent[12.]{\Hei 答}:放任股市上涨的理由是吸收游资(避免外流)并刺激消费,但是这先天上就不是最好的手段,因为真正赚钱的是内线的大户,而这些大资本恰是在涨高了后第一个外逃的,所以不但不能很好地刺激消费,最终反而会增进贫富不均,更造成大笔资金外流。

我以前就说过几次,正确的做法是为低级公务员加薪,不但立刻刺激消费,而且与反腐相辅相成。股市可以微涨,但不能大涨。结果放任大涨之后,果然成为大户杀小户的屠宰场,这时就应该亡羊补牢,马上出台我在前文《谈中国股市和其他问题》中所提到的资本利得税(当然必须有预案,最好一个周末就办完)。结果金融主管只是笨手笨脚地去一个一个查,这是事倍功零的蠢主意,果然到现在一个都抓不到。你只要先把所有股市账户里的资本利得冻结适当的比率来准备交税,然后慢慢地查,这些大资本就跑不掉。大部分的暴利还可以归公,刚好可以为公务员加薪。还有,资本利得税的比率可以视政策方向调整:根据买入的日期,政府想扶持股市时下降,想打压股市时提高。

总之,中共当局处理股市的过程不只是笨手笨脚,根本就是缚手缚脚,和美国政府如出一辙,显然是受既得利益者的控制。习近平应该不会如此腐败,所以大概是李克强的手下有问题。\\

\textit{\hfill\noindent\small 2015/08/23 00:00 提问; 回答}

\noindent[13.]{\Hei 答}:Good to hear from you again. How is the demonstration against TPP going?

Regarding your questions:

1. Without the QE, the US economy would have dived to great depth. Even if it eventually recovers, the GDP level will still be much lower than pre-2007.

2. GDP is not perfect, but it is convenient, so I use it for the sake of economy (pun intended).

3. Euro has been great for Germany and other export countries, not so good for the PIGS. Overall, though, Europe is better off with it than without.

4. Yes, tribal bias ends up hurting oneself. Kind of a poetic justice.\\

\textit{\hfill\noindent\small 2015/08/24 00:00 提问; 回答}

\noindent[14.]{\Hei 答}:但是国债一样的发。国债的问题在于很难选择性地赖帐,所以买家是谁最终不太重要。日本若是发明只对中央银行赖帐,等于是触发突然的大幅贬值,同样会造成一系列的危机。

其实日本的国债问题和安倍的政策是比较复杂的,我前面说的是简化版。安倍的确是想要藉通货膨胀再加货币贬值来将国债做实质的压缩;换句话说,用隐形的部分赖帐来延迟正式的一次性崩溃。不过这样一来,人民的储蓄提前被\&ldquo;理髮\&rdquo;,生活水准现在就开始下降,然而受人口结构和职场文化的限制,生產效率并没有提高,所以10年后的结果还是一样的,百姓只是白白地先受了10年的苦。\\

\textit{\hfill\noindent\small 2015/08/24 00:00 提问; 回答}

\noindent[15.]{\Hei 答}:印钞票是中央银行的职权,所以只有当中央银行买国债时才能说\&ldquo;印钞票来维持经济\&rdquo;。美国人的婉辞是量化宽松。

日本是由国内银行和邮政储蓄系统来买国债,所以只能说是\&ldquo;印债卷来维持经济\&rdquo;。这些资金的来源的确是人民的储蓄和保险金,一旦还不出来后果极为严重。若要避免金融机构连环倒闭,只有把所有储蓄和保险统一\&ldquo;理髮\&rdquo;一条路。理髮的比率应该会在50\%以上,再加上日币匯率会完全崩溃,日本人留在国内的资產一夕之间挥发近净。因为只有大企业和财阀才能到海外避险,结果是中產阶级会被一笔抹消,国家也从先进行列跌入\&ldquo;开发中\&rdquo;层次。二战后的歷史里,只有苏联解体才发生过类似的惨剧。俄国人素以能吃苦耐劳着称,不知现代的日本人还能不能忍受同样的经验。\\

\textit{\hfill\noindent\small 2015/08/24 00:00 提问; 回答}

\noindent[16.]{\Hei 答}:有可能会延迟加息,但是应该不会取消,除非GDP成长率连续两季掉到1\%以下。\\

\textit{\hfill\noindent\small 2015/08/24 00:00 提问; 回答}

\noindent[17.]{\Hei 答}:由俭入奢易,由奢入俭难。俄国人撑过来了,但是日本人会很难过。在某个程度上,台湾人和香港人的心理不平衡,也与这个现象有关。\\

\textit{\hfill\noindent\small 2015/08/24 00:00 提问; 回答}

\noindent[18.]{\Hei 答}:重点在于產业升级,也就是《中国制造2025》成功与否。

我最近读了一篇日本机床工业老人的回忆(有兴趣的读者可以去看nippon.com的原文:\href{http://www.nippon.com/en/currents/d00007/}{链接\footnote{\url{http://www.nippon.com/en/currents/d00007/}}}),原来日本在1970年代和中国现在的情形很相似,只占有低端市场。经过15年努力引进美国和德国的技术,才一跃而成为强国。中国的市场规模、组织能力和人员素质都只高不低,没有理由会失败。尤其是欧美日自顾不暇,若是有大规模的社会动盪,将会给予中国更好的机会。\\

\textit{\hfill\noindent\small 2015/08/24 00:00 提问; 回答}

\noindent[19.]{\Hei 答}:国债不是人民欠的债,而是政府欠的债;政府若是被革命推翻了,新政府就可以考虑注销这些债(当然要看与债权人谈判的结果,国际之间没有破產法,我已经在《美元的金融霸权》里解释过了),自然人的死亡和它无关。

日本的特色在于它的国债不是由外国人持有,而是邮政储蓄系统和银行奉政府之命拼命买的,所以利息支出到了税收50\%以上他们还是得乖乖地接受低利率继续买。到了2025年以后,这个比率必然会超过100\%,届时就算邮政储蓄系统和银行仍然奉命继续买,国债增加的速度也会大幅提升为指数函数,要硬撑也撑不了几年。

请务必读完所有的正文再发问,如果能读留言栏更好。絶大多数的问题已经被讨论过了。

还有,留言请勿使用俳句格式。\\

\textit{\hfill\noindent\small 2015/08/24 00:00 提问; 回答}

\noindent[20.]{\Hei 答}:经济归经济,股市归股市,两者只有名义上的牵连。

这一轮是因为预期联储会加息,全球股市又在极高点,所以国际资本从股市脱身。去年年底有人问我美国股市的前景,我不是说过6-9月后会向下突破震盪范围?请参阅《与大陆网友的问答节录》。

我写了一年多的稿子,只做过一次市场预测;希望大家能花点心思注意一下。\\

\textit{\hfill\noindent\small 2015/08/25 00:00 提问; 回答}

\noindent[21.]{\Hei 答}:这些都是相对悲观的预测,很多美国人也喜欢对中国做相对悲观的预测。

实际上人不是死的,只要向正确方向努力,总有或多或少的成功机会,大家自求多福吧。\\

\textit{\hfill\noindent\small 2015/08/25 00:00 提问; 回答}

\noindent[22.]{\Hei 答}:联储会\&ldquo;发行\&rdquo;的美金数量远超实际的钞票,大部分存在于银行间的虚拟帐目上。外匯储备也是这样的。

正是因活期存款利息极低,所以只好买债卷。\\

\textit{\hfill\noindent\small 2015/08/25 00:00 提问; 回答}

\noindent[23.]{\Hei 答}:我已经说过了,中国股市和经济的问题只是个引子,真正的卖压来自资本在新经济环境下,决定脱离股票;至于转移到哪里,一般是先以\&ldquo;现金\&rdquo;(金融术语Cash指活期存款或任何可以当天动用的资產,并不是真正钞票)放一阵再做决定。\\

\textit{\hfill\noindent\small 2015/08/25 00:00 提问; 回答}

\noindent[24.]{\Hei 答}:是的,但是至少成败由自己的努力决定,既不是必胜也不是必败。\\

\textit{\hfill\noindent\small 2015/08/25 00:00 提问; 回答}

\noindent[25.]{\Hei 答}:不会。首先,日本央行在股市中的资本不大(7兆日元真不算什么)。其次,如果日本央行亏了钱,那么日币就会承受匯率上的下行压力,不过安倍本来就希望日币贬值,以便把世界第一大的国债的实际价值一起贬下去。当然这是以牺牲人民的储蓄和生活水准为代价的,而且有触发不可控崩溃的危险,但是安倍的逻辑是反正十年后也一定会崩溃,不如赌赌看慢慢变穷是否有转机。目前看来,似乎只是让人民提前变穷而已。\\

\textit{\hfill\noindent\small 2015/08/26 00:00 提问; 回答}

\noindent[26.]{\Hei 答}:我何尝说过TPP是美国独赢?我一直都说美国是利用自己的进口市场来拐骗其他国家接受新的独利美国的贸易规则;换句话说,美国独赢的只是在新规则上(尤其是专利权、版权和诉讼权),在传统的制造业贸易上,美国必须让利来引诱参与国。越南可以吃下的中国出口额份最多,所以最奋不顾身。日本则是出自安倍奇异的战略考虑,他们国内哪里来的趋之若鹜?

TPP过与不过都不会影响我对6个集团中任何一个未来两年的GDP成长率预测,既然与本文主题无关,我提它干什么?我原本还以为你莫名其妙地提起TPP,在考虑因离题是否该删除呢!\\

\textit{\hfill\noindent\small 2015/08/26 00:00 提问; 回答}

\noindent[27.]{\Hei 答}:1. 俄国的经济前景很糟糕,\&ldquo;拥有核武的沙特\&rdquo;虽然过分,但是不远。

2. 我不认为20年内有任何顾虑。

3. 拥核国家只能被颠覆,不能被侵占。远东是俄国的一部分,不只是苏联的加盟国。中国当时既无实力,也无名分可以拿下它来。\\

\textit{\hfill\noindent\small 2015/08/28 00:00 提问; 回答}

\noindent[28.]{\Hei 答}:越南的人口比南韩还多,人少不是他的问题之一。

基本上新兴国家已经逐步认清正确的发展道路,所以大家拼的是执行和组织的能力。越南在这方面虽不是最糟糕,却没有长期高速发展的潜力。\\

\textit{\hfill\noindent\small 2015/08/28 00:00 提问; 回答}

\noindent[29.]{\Hei 答}:对不起,正文里写的不太清楚。我并不是指他们有同样的内部体制问题,而是他们同样地有内部体制问题。一般体制问题要是十年不处理就会浮现到表面,所以我才会那么写。\\

\textit{\hfill\noindent\small 2015/08/28 00:00 提问; 回答}

\noindent[30.]{\Hei 答}:我大致同意你的看法,不过未来几年越南经济会有较高速的成长,应该可以暂时缓衝内部矛盾。\\

\textit{\hfill\noindent\small 2015/08/29 00:00 提问; 回答}

\noindent[31.]{\Hei 答}:我想很多后进国家了解到,即使制度不能换,在可容许的范畴内,必须增大基础建设投资。这或许听来是很基本的一步,但是以往的华盛顿共识却强调维持民主,结果是穷国的一点资源都当做福利发掉了。

还有,请你写得精简一些。你最新留言的资讯密度实在太低,所以被删了。请参照我示范的写作风格:有好几篇正文都只有五六个段落,想想里面有多少资讯。浓缩文章要花很多精力,但是为了读者的时间,必须要做到。如果为了一己的懒惰,而把虚浮的段落随意丢出,是不负责任的污染公共资源;我做为博主,只能选择删除一途。\\

\textit{\hfill\noindent\small 2015/08/29 00:00 提问; 回答}

\noindent[32.]{\Hei 答}:我想他低估了竞争力的问题。例如在汽车工业上,中国已经花了20多年,实际上仍然不入流。如果没有政策上的改进,再20多年可能还是一样不入流。

中国的经济已经开始转型,资源的消耗不再像五年前看来那么急迫,反倒是与美、德、日在高科技工业上的竞争将越趋激烈。表面上只是市场的争夺,它的关键却是在效率和技术上的。这位作者没有强调这点,就不算说中了要害。\\

\textit{\hfill\noindent\small 2015/08/29 00:00 提问; 回答}

\noindent[33.]{\Hei 答}:我同意。\\

\textit{\hfill\noindent\small 2015/08/30 00:00 提问; 回答}

\noindent[34.]{\Hei 答}:我个人觉得中国与欧美争夺资源的野战阶段已经接近尾声,未来20年是在 许许多多高科技的细节进行巷战。

日本已经完蛋了,只是还在拒绝承认而已;但是欧洲和美国还有很多大大小小的技术要塞,必须一个一个地攻坚。中国自己也有人口老化的问题,时间并不是无限的。\\

\textit{\hfill\noindent\small 2015/08/30 00:00 提问; 回答}

\noindent[35.]{\Hei 答}:我对这两段话没有异议,只想指出争取大宗商品和货币的交易权是中共已经重点努力了十几年的事,所以我不怎么担心。\\

\textit{\hfill\noindent\small 2015/09/02 00:00 提问; 回答}

\noindent[36.]{\Hei 答}:1. 一个国家可以完全控制自己的货币,如果需要钱,多印就行了,后果只是通货膨胀。外匯就是政府手中的外币,那当然不能在国内流通,它是用来平衡匯率的。

2. 引进外商投资,要的不只是资金,更重要的是技术和管理。外商拿外币来设厂,必须先换成当地货币,才能购买材料和发工资,这其实是外匯的三大来源之一。国内储蓄若是愿意投资当然很好。我说过,促进生產是对国家最有利的资本应用方式,所以不论那个资本来自国外或是国内一般都是好事。

3. 外匯指的是(广义的)外币现金,它的好处是随时需要(例如被美国对冲基金攻击匯率时)随时可用。若是有多余的,可以投资在流通性低但是回报率高的固定资產,例如外国的土地和企业。所以外匯基本上就是在金融战场上防御美元霸权打击的国防力量,必须有足够的吓阻力,但是太多就影响国计民生。

4. 中国的外匯总额看来很大,其实占GDP的比率并不高,而且中共已经努力将部分转化为前面提到的高报酬率资產,所以最近开始下降(就像中共今天也宣布要裁军),未来应该也不会再显着增长。

顺便提一下,美国的名义GDP还比中国高一些,但是他基本没有外匯也没有必要,因为美元是国际储备货币,多印钞票造成的通货膨胀主要由其他国家承受,如果不拼命用白纸来换外国资產,那才真是蠢呢,所以美国自然就成了消费国。\\

\textit{\hfill\noindent\small 2015/09/14 00:00 提问; 回答}

\noindent[37.]{\Hei 答}:不太看得懂。外匯存量是货币政策的间接结果,不是人民银行能直接操控的,作者把它说反了。而且中国现有的外匯就算太多,也没有急着把它压低的理由。

美国不让人民币进SDR很明显地是在保护美元的霸权地位,扯上释放外匯实在很牵强。\\

\textit{\hfill\noindent\small 2015/09/15 00:00 提问; 回答}

\noindent[38.]{\Hei 答}:If the US dollar keeps depreciating for a few years, most people will switch to stronger international currencies.\\

\textit{\hfill\noindent\small 2015/09/15 00:00 提问; 回答}

\noindent[39.]{\Hei 答}:不玩不行,人民币成为国际储备货币之一有很大的利多。

前一阵子股市搞得乱七八糟,应该是李克强的经济团队的责任;管货币的周小川还没有捅出大漏子的记录,我们暂且再观察一阵吧。\\

\textit{\hfill\noindent\small 2015/09/21 00:00 提问; 回答}

\noindent[40.]{\Hei 答}:这个论调极其荒谬,完全是颠倒因果的胡扯,请不要再浪费大家的时间。

GNP早已被停用,所以近年的数据都是非正式的猜测,错误很大。\\

\textit{\hfill\noindent\small 2015/09/21 00:00 提问; 回答}

\noindent[41.]{\Hei 答}:GNI和GDP的差别在于前者把產值算给生產者的国籍,后者则直接算给生產地;除非是劳工输出大国,否则差异应该在1\%以内。

他所看到的那些论据连事实都明显是偽造的,立刻就给我过敏反应。\\

\textit{\hfill\noindent\small 2015/09/21 00:00 提问; 回答}

\noindent[42.]{\Hei 答}:对不起,美国看中国视频不方便。不过这类的谬误应该是很明显的。\\

\textit{\hfill\noindent\small 2015/09/21 00:00 提问; 回答}

\noindent[43.]{\Hei 答}:日本的衰退轨迹在上面的留言栏已经被详细讨论过,请先读完既有讨论再来发言。

如果你看不懂,请解释哪里不懂。我没有责任为你一人一再重复。

这里的论证都是以足够的事实为基础,经过完整的逻辑而达到的结论。有异议的人有责任自备与之衝突的事实与逻辑,并且以明确的文字表达清楚。一张空口再加上网络上节录来的他人意见不具任何意义,纯属浪费大家时间。在此严重警告,再犯将被直接删除。\\

\textit{\hfill\noindent\small 2015/09/21 00:00 提问; 回答}

\noindent[44.]{\Hei 答}:不是好脾气,只是看他们可怜,不忍心放任他们沉浸于自己的无知和不负责任中,总觉得至少该说几句话来提醒一下。

公共资源的主要问题,就是如你所说的,个人无须付出代价就可以无限占用,最后忽然质变崩溃,损失却由全体分担,非常不公平。这其实是资本主义自由经济的内建缺陷,因为像李嘉诚这样的超级富豪,不也是找了制度的空隙,无限占用原本应该属于社会大眾的隐形公共资源(亦即社区的发展),榨取不公平的利益。这必须以制度和公开明确的政策来反制,只是不知习政权能不能有这个智慧。我听他的头号经济幕僚刘鹤的发言,好像从未提到这类的话题。\\

\textit{\hfill\noindent\small 2015/09/24 00:00 提问; 回答}

\noindent[45.]{\Hei 答}:日本人守法的精神是不包括诚实交税的。\\

\textit{\hfill\noindent\small 2015/11/18 00:00 提问; 回答}

\noindent[46.]{\Hei 答}:加拿大地广人稀,主要是资源输出国,但是工业能力还不错,比澳洲强一些;全球暖化也会慢慢受益,不像澳洲会沙漠化。

未来5年受世界资源需求减缓影响,一些困难无可避免,不过它的底子硬,迷信绝对自由主义的旧总理又已下臺,如果新总理能有决断大破大立,长期的前途会胜于澳洲。\\

\textit{\hfill\noindent\small 2015/11/25 00:00 提问; 回答}

\noindent[47.]{\Hei 答}:你所说的大趋势并非没有道理,不过我觉得细节上比你说的复杂些。机器人是工业革命后,几百年来机械化、自动化的下一步。目前由非技术劳工所做的工作,有一部分会被取代,不过不是全部。这在长期来看,正如你説的,对工业化程度低于中国的地区有额外的压力;然而他们面对的最大问题还是中国本身。中国的体量太大、效率太高,挡在他们前面,他们永远衹能捡中国淘汰的產业。

日本原本可以藉此恢復一些竞争力,但因为受中国的挤压更严重,再加上安倍这种军国主义復辟份子的人谋不臧,长期的前景反而更为黯淡。美国和臺湾也是一样的。

中国的策略,是建立以自己为中心,但择优包容国外环节的全球生產链;那么10-20年后,能随中国一起同步升级的国家,就衹有全力合作融入其產业链而且能奋发自强、保卫自己专业的几个国家了,如韩国。

当然世界经济并不是零和的游戏,所以中国的长期战略主轴也包括提携低度开发国家进入至少是初步工业化的程度,这样才能扩大世界的总需求,否则所有高端產业的总產值连让中、美、欧分食都不够,会造成割喉竞争的残烈环境,对经济、外交和社会的稳定都很不利。所以像越南这样的国家虽然绝对生活水准还可能持续提升,相对于中国大概永远都处于低端的层次。反而是目前最原始的地区如非洲,如果有足够的政治智慧和机遇,可以有赶上相对高端国家(如越南)的空间。\\

\textit{\hfill\noindent\small 2015/11/25 00:00 提问; 回答}

\noindent[48.]{\Hei 答}:我对非洲比你乐观。当然不是每个非洲国家都会成功,但是应该会有几个相对稳定的政府,而中国不但做了很好的示范,更是资金和技术的来源。过去几百年欧美日从未真心扶持后进国家(冷战中美国支持自己的小弟是例外),现在中国愿意提供价格合理的帮助,有些东非国家可能会搭上顺风车。\\

\textit{\hfill\noindent\small 2015/11/25 00:00 提问; 回答}

\noindent[49.]{\Hei 答}:东非有好几国家还算上轨道的,他们都会从一带一路中受益。\\

\textit{\hfill\noindent\small 2017/09/26 00:00 提问; 回答}

\noindent[50.]{\Hei 答}:我和陈平在经济学的原则和理论上同意,在金融业的实践和执行上则有不同的看法。这可能是因为他是做理论经济学出身,而我是靠在金融界实战所获得的经验。

我觉得他太过乐观,远远高估了中国体制下打金融战的灵活性。灵活是西方自由经济体制的强处,中国不应该以弱击强。保守,并不只是修长城,而是在长城内也有纵深防御,层层布防。宋朝并不是亡在金兵入关,而在于没有第二綫防御能力。换句话说,陈平认为美国的金融界像是塞外的骑兵,灵活机动,所以我们也应该建立自己的骑兵部队,出塞远征。但是别忘了,汉武帝固然大破匈奴,也把国家搞破產了。而且同样的宋朝步兵,在岳飞这样的将军领导下,对抗金国的骑兵并无劣势。我的偏好是你打你的、我打我的,先蹲在家里开发火枪,一旦技术成熟,擅长骑射的游牧民族自然被歷史洪流淹没,从征服者变成被征服者,就像俄国拿下中亚和西伯利亚那样,不是靠针锋相对,而是依赖整体国力和体系的绝对优势。

中国的经济金融界,不但上层有迷信自由经济主义的带路党,中层的金融管理人员的品德和专业水准,也十分可疑。金融原本就是资產虚拟化的体系,自由、灵活的代价是复杂性和不透明度。整个行业先天就是靠信息不对称来占实业界的便宜,而越是复杂和不透明的金融体系,信息不对称性就越高。拿陈平所建议的自主的对衝基金来説,既然占自己人的便宜,必然更快更容易,那为什么要卖命和美国人打硬仗?如果有足够的监管和透明度来确定枪口对外,那么绝对就不会有什么灵活度可言。这是金融业的内建矛盾,就像热力学第二定律一样,注定了不可能有永动机。陈平提到明朝的长城,可是清兵入关并不是明朝败亡的原因,而是一个后果;明朝实际上是亡于内乱。如果采行陈平的建议,就有如唐朝一样,武力强大,但是藩镇割据,反而自食恶果;金融藩镇正是美国现今的写照,中国千万不可仿效。

我是讲究绝对理性的人,所以不会有不切实际的幻想。光衝着我没有中国籍这事,就注定我不可能在中国做主官。而且主官的特质在于能干、肯干,诚实和深思反而可能是负面的资產。

我所学甚杂,但是最强的几个方面,如金融、物理、战略,中国都有自已的专家,没有我的发声,也无关痛痒。反而是一些次要的能力,刚好是中国最弱的地方,如果我的意见能获得采纳,可能会有立竿见影的效果,例如科学行政和对外宣传。上次有关对撞机的讨论,就是我在科学行政方面间接做贡献的例子。\\

\textit{\hfill\noindent\small 2017/09/27 00:00 提问; 回答}

\noindent[51.]{\Hei 答}:是的,尤其以实业为主的经济,货币价值越稳定越好,为了金融炒作而让货币大幅涨跌,是英美去实业化的歪路,中国不走也罢。\\

\textit{\hfill\noindent\small 2017/10/06 00:00 提问; 回答}

\noindent[52.]{\Hei 答}:银行界的经济学专家,都是被雇来打屁、哄外行顾客的。

我没有说西方的经济体系更灵活,我说的是他们的金融企业更灵活。\\

\textit{\hfill\noindent\small 2020/04/12 16:24 提问;2020/04/13 04:50 回答}

\noindent[53.]{\Hei 答}:中國的產業升級,任重道遠,畢竟要在幾十年的時間内,趕上歐美幾百年纍積下來的技術根底,絕非容易的事;像是AI這種全新的技術領域,反而不是問題。

在必須從後超趕的工業範疇,原本因爲許多權力圈子内外的官員和“經濟學家”受美國自由主義理論的蠱惑,部分產業執迷於追求利潤的短綫操作,若干所謂民族企業純粹靠攔截竊取國家和人民的投入以自肥。在《中國製造2025》施行之後,整體趨勢有所改善。2018年開始的中美貿易戰,更是當頭棒喝,對中國國内產業發展路綫的議題上,有著極强的統一口徑、撥亂反正的效應。我認爲Trump的經貿打擊手段,將會適得其反,在短期的陣痛之後,反而加速中國的上升態勢。

至於中國學術界的腐敗,始終是成員找規則和執行上的漏洞來Game the System(玩弄體制?)的問題,修改規則只是促使他們去尋找新的漏洞。教育部仍然不願意直接出手打擊假大空,就好像有個輪胎漏氣,不去把洞找出來補上,而只是輪胎換位,那當然是一點正面效果都不會有的。
\\

\textit{\hfill\noindent\small 2020/07/23 10:30 提问;2020/07/24 02:31 回答}

\noindent[54.]{\Hei 答}:中美歐三大集團近年的發展局勢,我一直不斷在追蹤討論;你所説的是合理的總結。

印度的起點太低,即使只以很低的效率搞基建和工業化,也還有很大的向上發展空間。越南則對中共政策亦步亦趨,在現階段沒有走錯岔路的危險。既然這兩個國家不可能接受和平親善的橄欖枝,中方應該立刻停止資助他們產業升級,反過來扶持臨近的競爭對手,例如巴基斯坦、柬埔寨和泰國。不過不要對他們寄望太高;這些國家並不處於東亞儒家文化圈内,沒有適合現代工業社會的文化傳統和政治體制,發展的上限低於越南。

日本的安倍政權對中國同樣有深刻的非理性敵視,中日合作必須等到思路不同的新首相上臺才有可能。在等待期間,只要穩住日本,不讓它把自己全盤賤賣給美國即可,同時必須小心防範日本在區域整合過程中攪局的作用,對排除日方擋路的企圖做好預案。中長期來看,日本的衰退無可挽回,在世界舞臺上,只剩下一點工業技術方面的殘餘價值。
\\

\textit{\hfill\noindent\small 2020/07/23 16:00 提问;2020/07/24 02:14 回答}

\noindent[55.]{\Hei 答}:有關未來幾年的GDP成長率,的確是會受全球經濟環境的拖累,不過從長期大戰略的角度來看,重要的是中國相對於競爭對手的表現。對方承受100\%的下行壓力,如果中方只感覺到1/3,那麽固然在成長率的絕對值上有負面影響,但是相對來説,中國的崛起反而加速了2/3。

我認爲本届政府,對2009年那一波不定向内需刺激的負面作用已經有深刻的認識和反省,不會重新大幅舉債來吹大泡沫,而是改爲針對有長期效益的工業和基建方面做投資。中國經濟管理的重要考慮之一是維持穩定合理的就業率;剛好生育率降低所帶來的人口曲綫變化已經開始浮現,短期内反而對全球經濟減緩背景下的就業問題有幫助。

至於彎道超車,因爲既有的行業領袖在技術、人員、收支、銷售網絡等方面有很大的優勢,一般是在工業技術或商業模式有換代變革的背景下,後來者才有超趕的機會。例如目前汽車業正經歷從内燃機改爲電動的轉換,這是百年一次的技術過渡,中國必須抓穩這個難得良機。芯片則相反,5-10年内既有的技術路綫還不會撞墻,那麽直接挑戰臺積電的勝算就頗爲渺茫。大飛機比較特別一點:這裏是行業霸主波音的自我毀滅創造了新興廠家的生存空間;而且後新冠世界的長途旅行需求大減,使單通道機型的重要性進一步增加,剛好適合商飛的產品陣容,就連時機也恰到好處,C-919交機拖延反而避開了航空公司停飛的時段。

美國的國力衰退和歐洲的發展停滯,仍然會不斷提供中國升級、替代和超越的機會。如果未來五年歐美經濟是零成長,那麽中國接受5\%的年成長率並無困難。事實上中國發展模式的一大毛病,在於努力有餘、思考不足,沒有選好正確的方面,空耗了許多資源,有些地方政策甚至是花大錢砸國家的脚,例如貴州和高通的幾次合資。如果能放下對維持高速成長率的執拗,事先多做一些詳細的評估和整體的協作,未嘗不是件好事。
\\

\textit{\hfill\noindent\small 2021/05/20 14:36 提问;2021/05/21 03:29 回答}

\noindent[56.]{\Hei 答}:目前看來,完全沒有修正的必要。此外,2025-2030年不但是日本公共債務危機爆發的最可能時段,而且也是其在國際高科技供應鏈的額分再向下掉落一個臺階的預期時間值,届時日本會從二流强權進一步衰落為三流國家。
\\

\textit{\hfill\noindent\small 2021/10/22 00:10 提问;2021/10/22 02:38 回答}

\noindent[57.]{\Hei 答}:要解決這個問題,必須對房地產業和稅制同時做出深刻的改革。既然已經對恆大出手,就代表這些改革已經處於進行式;你想爲什麽最近我一直說稅制改革即將出臺?
\\

\textit{\hfill\noindent\small 2022/03/15 00:56 提问;2022/03/15 02:52 回答}

\noindent[58.]{\Hei 答}:沒有暴動,和社會結構、文化傳統和財富纍積都有關係,不是財團力量單一決定的。

日本在國債纍積到GDP266\%都還能夠繼續運作,基本原因正是這些國債所欠的對象是國内的儲蓄,而不是外債;事實上日本反過來有著極高的海外資產,所以我多年前就說過要等國債上升到400\%才比較可能會有財政上的質變。與此同時,如果重工業因電動車興起而進一步沒落,當然會有加速財政問題爆發的作用。
\\


\section{【基础科研】基因工程与分子生物学的新发展(二)}
\subsection{2015-08-26 15:06}


\section{1条问答}

\textit{\hfill\noindent\small 2015/09/23 00:00 提问; 回答}

\noindent[1.]{\Hei 答}:不一定。很多人才还是在美国,但是资金和自由却在亚洲。\\


\section{【台灣】【工業】經濟的最後支柱}
\subsection{2015-08-31 02:35}


\section{26条问答}

\textit{\hfill\noindent\small 2015/08/31 00:00 提问; 回答}

\noindent[1.]{\Hei 答}:南韩眼看着要真正跻身开发国家之列,我年轻的时候台湾可是领先他们不止十年的,现在却要走上阿根廷的老路,自然让人心有不甘,唉。\\

\textit{\hfill\noindent\small 2015/08/31 00:00 提问; 回答}

\noindent[2.]{\Hei 答}:Yes, the Florida model: tourism + fruits. But this will require a huge market, so it will actually make Taiwan even more dependent on China.\\

\textit{\hfill\noindent\small 2015/08/31 00:00 提问; 回答}

\noindent[3.]{\Hei 答}:这是他很老的一篇文章,已经被讨论过多次。

新读者请务必先读完整个部落格再发言,以避免重复。\\

\textit{\hfill\noindent\small 2015/08/31 00:00 提问; 回答}

\noindent[4.]{\Hei 答}:台积电16nm制程的经理四年前叛逃到三星,使三星得以后来居上。台积电在法院缠讼四年,本月才终于拿到一个不痛不痒的\&ldquo;禁令\&rdquo;。恐龙法官害国害民,这是一个直接的例证;若是在美国,那人一定会被判好几年徒刑。

或许联发科很快就要完蛋吧。我只觉得很可惜,他们和一般的台湾高科技企业不一样,是有自主技术的,只是门栏不够深。\\

\textit{\hfill\noindent\small 2015/08/31 00:00 提问; 回答}

\noindent[5.]{\Hei 答}:我回老家的时候也注意到,民眾不懒、努力创业,但是创的都是小吃。当一个镇上每家都卖小吃的时候,大家自然都不赚钱。\\

\textit{\hfill\noindent\small 2015/08/31 00:00 提问; 回答}

\noindent[6.]{\Hei 答}:其实民意若不改变,这些努力都没有实质的意义。短期内,我们还是专注在设法减低民生方面的损害吧。\\

\textit{\hfill\noindent\small 2015/08/31 00:00 提问; 回答}

\noindent[7.]{\Hei 答}:Florida靠着观光、水果和退休新家这三个\&ldquo;工业\&rdquo;,其实也过得不错,但是前提是与大陆完全对接开放。

台湾在生活水准倒退之后,是可能保留一些简单的工业的,例如螺丝。不过和南韩对比一下,总让人不胜唏嘘。\\

\textit{\hfill\noindent\small 2015/08/31 00:00 提问; 回答}

\noindent[8.]{\Hei 答}:台湾学的是日本哲学,寧为玉碎不做瓦全。\\

\textit{\hfill\noindent\small 2015/08/31 00:00 提问; 回答}

\noindent[9.]{\Hei 答}:是的,生活水准大幅下降是日本和台湾的共同前景,李登辉对台湾的日本化的确是相当成功的。\\

\textit{\hfill\noindent\small 2015/08/31 00:00 提问; 回答}

\noindent[10.]{\Hei 答}:就算还有新生代,十年后已经没有企业了,又有什么意义?

我想即使蔡英文上台,也不会看不出台积电的重要,但是中央已经是政令不出总统府,地方上稍作抗争,电厂就建不下去。台积电可能会等上几年,结果还是一场空,不得不出走。

连战的立场和20年前是一样的;立场变了的,其实是台湾的民意。\\

\textit{\hfill\noindent\small 2015/08/31 00:00 提问; 回答}

\noindent[11.]{\Hei 答}:看来真是要大举登陆了。如此一来,大陆企业要挖角就更容易。\\

\textit{\hfill\noindent\small 2015/08/31 00:00 提问; 回答}

\noindent[12.]{\Hei 答}:的确是非常沉重的消息:连台积电都待不下去,还谈什么新產业?

年轻人大不了到大陆去谋生路;贫苦民眾却是跑不了的,而且会越来越多。\\

\textit{\hfill\noindent\small 2015/08/31 00:00 提问; 回答}

\noindent[13.]{\Hei 答}:There is nothing wrong with being a tourist attraction. I pointed it out merely for the intrinsic contradiction: Taiwan lost its industrial base because of its boneheaded fascination on independence, and the result is exactly the total dependence on China economically. What an irony.\\

\textit{\hfill\noindent\small 2015/09/01 00:00 提问; 回答}

\noindent[14.]{\Hei 答}:High-tech manufacturing remains the key to high living standards. Taiwan is too big to qualify as an exception to the rule.\\

\textit{\hfill\noindent\small 2015/09/01 00:00 提问; 回答}

\noindent[15.]{\Hei 答}:No, Taiwan had about the best hand in the world 25 years ago and squandered it. Even now, it still have an ace in the hole; it simply refuses to play it.\\

\textit{\hfill\noindent\small 2015/09/01 00:00 提问; 回答}

\noindent[16.]{\Hei 答}:是的,台湾有很多幸运的因素,但是绝不是开发中国家里最幸运的,但是65-88年间的成长却是全球最成功的。经济是否能高速成长,其绝对性的因素是政府的组织执行能力。日本什么资源都没有,二战还挨了两枚原子弹,后来能把握冷战的机遇,还是靠了组织和执行的能力,否则日本和台湾的运气,菲律宾都有,为什么会成为女佣出口国呢?\\

\textit{\hfill\noindent\small 2015/09/01 00:00 提问; 回答}

\noindent[17.]{\Hei 答}:是的。

其实日本若真把中共惹毛了,光是禁止观光就是很大的打击。台湾也是一样的。\\

\textit{\hfill\noindent\small 2015/09/01 00:00 提问; 回答}

\noindent[18.]{\Hei 答}:太远了,不切实际。

当前的问题是如何减轻穷苦民眾所受的伤害。台积电一出走,上下游的工业必须跟着走,几十万台湾的高科技技术菁英跟着离开,他们常光顾的3C店、小吃店、理髮店、百货店、汽车代理店跟着倒闭,房地產价钱大跌,经济开始萎缩,打零工的底层民眾日子更为难过,可是政府的税收也大幅缩水,更加无法照顾他们。\\

\textit{\hfill\noindent\small 2015/09/01 00:00 提问; 回答}

\noindent[19.]{\Hei 答}:台湾政府引导的高速经济成长始自60年代初期,韩国到70年代才起飞。初始条件或许有些许不同,但是政策才是真正促成成长的因素。\\

\textit{\hfill\noindent\small 2015/09/01 00:00 提问; 回答}

\noindent[20.]{\Hei 答}:小国对霸权没有威胁,一般只要内部能整理好,外部条件都可以设法改进。\\

\textit{\hfill\noindent\small 2015/09/01 00:00 提问; 回答}

\noindent[21.]{\Hei 答}:是的,我指的是大陆。

我不觉得台湾有什么道德立场;如果要讲传统道德,几千万人多是汉奸国贼,这是中国传统文化里最大的罪过,还有什么脸跟人家吵?

不过侧重实际是一个好建议。\\

\textit{\hfill\noindent\small 2015/09/01 00:00 提问; 回答}

\noindent[22.]{\Hei 答}:那篇文章的作者voyager\_ho就在上面留过话。他的留言一向很广博深刻,想来是学经济歷史的。\\

\textit{\hfill\noindent\small 2015/09/02 00:00 提问; 回答}

\noindent[23.]{\Hei 答}:We know they are all hypocrites. In fact, they are also cowards, liars, thieves, traitors and mass murderers. But that does not change the fact that they control the island's majority opinions.

This is really way off the subject. I am putting a stop to it.\\

\textit{\hfill\noindent\small 2015/09/02 00:00 提问; 回答}

\noindent[24.]{\Hei 答}:台独怕的就是两岸交流,恨不得一刀切下,分住到不同的宇宙去。

若是他们能往实际利害想,台湾也就不会沦落至此。\\

\textit{\hfill\noindent\small 2021/11/14 19:14 提问;2021/11/16 03:10 回答}

\noindent[25.]{\Hei 答}:你說的和我的理解重合,適合其他讀者參考。

追根究底,這裏的問題在於懂的人不在乎、在乎的人不懂,又是專業知識形成的壁壘,這真的是中共理性治國哲學的軟肋。除了學術界、金融界之外,一般商業管理也是如此,即使簡單如回收外國垃圾一事,也是拖了好幾年,最終既懂又在乎的人依舊是極少數,只好以一刀切來解決。一刀切在垃圾問題上可行,對股市就不然,不過至少在學術方向,有解除學閥政治權力的這一步可做;那也是爲什麽我拿它做提議的考慮之一。
\\

\textit{\hfill\noindent\small 2021/12/05 13:28 提问;2021/12/05 16:48 回答}

\noindent[26.]{\Hei 答}:2015年我寫這篇文章的時候,的確是低估了英美台民粹的愚蠢,沒有算到居然會有脫歐、選Trump當總統和想要以武拒統的種種自殺性蠢事;不過我已經反復解釋過,我沒有魔法,只能以邏輯來試圖推算蠢蛋的作爲,原本就不可能是完全精確的。

我在早期博文中,也的確曾經極度高估了中方智庫、幕僚和官員的水準,後來中聯辦、國臺辦、教育部、科技部和各類學術機構的種種蠢事、爛事不一而足,強逼著我逐年下調對他們的評估。例如2017年貿易戰剛有苗頭,我就已經給出正確的總結和建議,但是一直到2019年,清華大學的金磚智庫居然還建議“站在人类生存角度...维持美国的世界大国地位是十分重要的”(參見\href{http://www.cbgg.org.cn/index/article/show/cid/20/id/72.html}{链接\footnote{\url{http://www.cbgg.org.cn/index/article/show/cid/20/id/72.html}}})。這一樣也是自殺性的行爲,一樣也是邏輯無法事先預估、或事後解釋的。事實上,我的文章被轉發到大陸,始終有人什麽事實邏輯基礎都拿不出來,也敢評論“大内自有高人”、或者“小不忍亂大謀”之類的空話傻話;所以如果你的論點是中方的學者、專家和一般知識分子之中也有許多蠢蛋,那麽事實根據很明確扎實,我無法反駁。

然而即使是菜鷄互啄,也有一方更菜;而且台灣連當菜鷄下場的資格都沒有,只不過是被爭奪的一隻小蟲罷了。正因爲台灣體量太微不足道,大國打個噴嚏,都可以讓它乘風飛起;你聽説過“風口上的飛豬”這個描述嗎?恆大風光了20年,平均年成長率達到33\%,一旦中國政府終於決定做早就該做的事,現在下場怎麽樣?台灣靠著美國打貿易戰而暫時獲利,能高速成長20年嗎?連恆大都不如,有資格自傲嗎?我的預測原本就是以三十年爲期、十年為一個階段,現在距離2025年還有四年,請大家稍安勿躁,短周期内的高頻噪音在所難免,但是歷史潮流的大方向,終究會呈現出來。

ps.我不想評論《Foreign Policy》的文章,因爲它並不是像《Foreign Affairs》那樣的學術期刊,而屬於“主流媒體”;我在討論假新聞的博文裏,該説的都已經説了。參見《讀者須知》。
\\


\section{【医疗】当前世界的公共卫生危机}
\subsection{2015-09-04 11:13}


\section{13条问答}

\textit{\hfill\noindent\small 2015/09/04 00:00 提问; 回答}

\noindent[1.]{\Hei 答}:滥用抗生素是全球性的问题。在开发中国家一般是医生图省事,对非细菌感染也乱开药。在美国主要是畜牧业工厂化之后,高密度的家禽家畜容易生传染病,所以会滥加抗生素到饲料里。

滥用旧有的抗生素和人体的抵抗力没有关系(但是会杀死很多有益的细菌),只是会培养出有抗药性的超级病菌。再加上过去三十几年基本没有人投资在开发新药上,所以问题就越来越糟糕。\\

\textit{\hfill\noindent\small 2015/09/04 00:00 提问; 回答}

\noindent[2.]{\Hei 答}:理性地讨论这些缺陷是健康而且必要的。真正的毛病在于有心人无限上纲,把普世问题硬说成体制问题。\\

\textit{\hfill\noindent\small 2015/09/04 00:00 提问; 回答}

\noindent[3.]{\Hei 答}:印度在这方面反而领先,就正是靠着仿制药创下的基础。其实中国发展机械工业也是走\&ldquo;引进-模仿-自制-创新\&rdquo;的路,不知为什么在药品工业就甘心做美国药厂的万年肥羊。\\

\textit{\hfill\noindent\small 2015/09/04 00:00 提问; 回答}

\noindent[4.]{\Hei 答}:美国的医保体系只保护保险公司的利润和有高级保险的病人。所以如果病人有身份地位,医疗品质绝对是世界第一的。前面有抱怨说中国的高级领导享受比一般民眾好得多的医疗服务,美国也是一样的,只是美国的领导不一定是来自党政,大多是企业界的头目。

美国医疗体系的真正问题是效率太低,因为有超过一半的钱被保险公司赚走了,所以医院只好先超收三倍以上,如此才能收支平衡。我以前已经提过,美国花费18\%的GDP在医疗上,这个比率超过任何一个其他先进国家的两倍,而还没有全民保险。不论如何,美国的制度绝对称不上榜样。\\

\textit{\hfill\noindent\small 2015/09/04 00:00 提问; 回答}

\noindent[5.]{\Hei 答}:谢谢你的补充。公司决定在哪方面研究,自然就只有那样的药出厂。这是自由市场与医疗业的天生不合。

美式制度已经被非生產性资本完全霸占,医药业只他们压榨全球人民的管道之一,而专利制度就是把利润最大化的工具;中国绝不可以照单全收。\\

\textit{\hfill\noindent\small 2015/09/04 00:00 提问; 回答}

\noindent[6.]{\Hei 答}:台湾在日据时代,最优秀的人才只能学医,留下一个传统至今,好处是医生的水准不错,坏处是人文社会思维水平很差。

我以前已经提过几次,台湾在生医业上有很大的优势,可以与大陆互补,可惜服贸协定通不过。很反讽的是反对的领袖很多是像柯文哲这样的医生,因为他们只懂医术,不懂经济。\\

\textit{\hfill\noindent\small 2015/09/04 00:00 提问; 回答}

\noindent[7.]{\Hei 答}:I don't think he was discriminating on the basis of skin color. My reading is that he was criticizing their culture, although the wording certainly would not pass any PC criteria.\\

\textit{\hfill\noindent\small 2015/09/04 00:00 提问; 回答}

\noindent[8.]{\Hei 答}:我知道Piketty的结论,但是他用的统计资料不是最新的。最近7年来,联储会的四万多亿量化宽松基本上都进了财阀的口袋,使美国的贫富不均成指数上升,这些资料大概还要5年左右才会反应到新的统计结果上,所以我猜测美国已经打破纪录了。

我自己对黑心食品的非正式估计是大陆5:台湾1,所以台湾奸商比大陆邪恶10倍吧。\\

\textit{\hfill\noindent\small 2015/09/04 00:00 提问; 回答}

\noindent[9.]{\Hei 答}:I don't think serving as guinea-pig is the major issue. The real scandal is how little the pharmaceutical industry spend on defeating those tropical diseases.\\

\textit{\hfill\noindent\small 2015/09/04 00:00 提问; 回答}

\noindent[10.]{\Hei 答}:Instead of charity, I think Chinese national governments will be able to do an even better job. It also has the side benefit of developing technical know-hows.\\

\textit{\hfill\noindent\small 2015/09/04 00:00 提问; 回答}

\noindent[11.]{\Hei 答}:If health care is considered a basic citizen's right, of course the government will do a better job than the dozens of private companies, whose profits depend on denying coverage and shuffling papers.\\

\textit{\hfill\noindent\small 2015/09/04 00:00 提问; 回答}

\noindent[12.]{\Hei 答}:其实药品专利完全图利厂商是一个举世皆有的常识,只因为美国霸权在过去30几年把药品专利法强加在其他国家上,才成了\&ldquo;国际惯例\&rdquo;和\&ldquo;标准\&rdquo;。连德国原本也对药品专利有很强的限制,后来被美国人逼着修法。

中国既然不是美国的跟班,就根本没有理由每年几百亿美元送给美国药厂,同时让几千万国内的病人吃苦受罪。这同样是留美或崇美的\&ldquo;经济专家\&rdquo;搞出来的祸国殃民的政策,早点抛弃迷信、长些常识,就早点为百姓造福。\\

\textit{\hfill\noindent\small 2015/09/04 00:00 提问; 回答}

\noindent[13.]{\Hei 答}:美国的医院正在连锁化、工厂化,也就是由会计师主管;这是一个不同的问题,但是浪费的资源更多。不过这个题目太大了,不是我能很好回答的。\\


\section{【经济】再谈股市和油价}
\subsection{2015-09-08 18:15}


\section{22条问答}

\textit{\hfill\noindent\small 2015/09/08 00:00 提问; 回答}

\noindent[1.]{\Hei 答}:你说的都没错,但是你认为外国的股市有本质上的不同吗?其实也只是程度上的不同。

要改变目前的恶劣风气,实际的步骤我在前文讨论过了,只看中共当局愿不愿意做。\\

\textit{\hfill\noindent\small 2015/09/08 00:00 提问; 回答}

\noindent[2.]{\Hei 答}:这其实是两个问题。首先,退休基金是否应该投入股市;其次,如果投入股市,如何杜绝弊端。

全世界的主要国家,包括美国在内,都不敢把退休基金投入股市。这是因为退休保险以保值为重,股市的风险太大,完全不合适。

如果非要投入股市,绝对不能雇用中介人来买卖,否则等于开空白支票给他。只能购买指数期货这种没有人为操作余地的金融工具。

台湾名义上是已开发国家,在民智和管理上其实是未开发国家的水准,所以当然又是反正道而行。等退休基金都被亏光了,大概只好找人用爱发功,看看是否能无中生有。\\

\textit{\hfill\noindent\small 2015/09/08 00:00 提问; 回答}

\noindent[3.]{\Hei 答}:人民币国际化目前有两条战线:阳的是争取IMF的SDR的股份,SDR是国际上最重要的合成货币,如果人民币被採用,那么各国都必须储备它;美国在这个问题上已经抵抗了好几年,现在又被拖到2016年去了。阴的就是正文里提到的原油期货。\\

\textit{\hfill\noindent\small 2015/09/08 00:00 提问; 回答}

\noindent[4.]{\Hei 答}:这两个问题的答案目前都还不明朗(目前抓的都只是小苍蝇),我们再等几个月看看吧。\\

\textit{\hfill\noindent\small 2015/09/09 00:00 提问; 回答}

\noindent[5.]{\Hei 答}:长短线资本利得税率不同是各国都有的,我只是认为税率应该更高些。

我不清楚中共组织内部的情形。\\

\textit{\hfill\noindent\small 2015/09/09 00:00 提问; 回答}

\noindent[6.]{\Hei 答}:刚好相反,是病急乱投医了。以往没有加重税,等到泡沫长大爆裂之后,为了救市,只好连轻税也免了。

这种作法是典型的美国式反应,可见金融管理阶层有很多被美国洗脑过的殭尸。\\

\textit{\hfill\noindent\small 2015/09/09 00:00 提问; 回答}

\noindent[7.]{\Hei 答}:你指的是那个新浪博客吗?如果不是,请提供链接。

人民银行在早年(亦即2010年之前)没有全力买黄金是错了;我不认为中国黄金储备有你想像的那么多。\\

\textit{\hfill\noindent\small 2015/09/09 00:00 提问; 回答}

\noindent[8.]{\Hei 答}:I think gambling, like alcoholism, has a large genetic element.\\

\textit{\hfill\noindent\small 2015/09/09 00:00 提问; 回答}

\noindent[9.]{\Hei 答}:Brent石油只占全球原油產量1\%左右,仍然是產业界的标的。

重点在于其他用户和卖家认为这个期货的定价是合理、公平、及时、类似的;当然美国霸权的政治考量对很多產油国的态度有影响,要夺取原油的定价权絶非一日之功。\\

\textit{\hfill\noindent\small 2015/09/09 00:00 提问; 回答}

\noindent[10.]{\Hei 答}:这里假设股票是在公开市场买的。不论是个人还是财团都必须经由交易所认证过的会员公司来交易,这些会员公司应该根据股票持有的长久来预扣税。如此一来,当局只须要管证卷交易公司,而不是每个有帐户的个人。

我并不认为所有的股票投资利得都必须征重税。若是持有股票经过很长的时间,税率可以低到30\%左右,这样可以鼓励储蓄;短线操作的利润则必须有很高的税率,以打击投机。\\

\textit{\hfill\noindent\small 2015/09/09 00:00 提问; 回答}

\noindent[11.]{\Hei 答}:The real problem is that the rules are usually meant to help the big insiders make even easier killing, not the other way around. But with expert help, it is possible to make stock market a little more little-guy friendly.\\

\textit{\hfill\noindent\small 2015/09/09 00:00 提问; 回答}

\noindent[12.]{\Hei 答}:中国人的确赌性特别坚强,美国的赌场很熟悉;大亨有专车至机场接送,唐人街打工的也有专门的巴士。\\

\textit{\hfill\noindent\small 2015/09/09 00:00 提问; 回答}

\noindent[13.]{\Hei 答}:好,谢谢你的补充。我对现有的中国金融规则细节不熟,只知道该有的法规还没有确立。\\

\textit{\hfill\noindent\small 2015/09/11 00:00 提问; 回答}

\noindent[14.]{\Hei 答}:在投资银行里,这些分析师的地位很低,完全只是养来哄人的。真正的生意主管如果有需要,当然会指示他们写宣传稿,但是一般时候他们只是自己随便乱写,反正读者也不懂,所以莫名其妙的东西很多。\\

\textit{\hfill\noindent\small 2015/09/11 00:00 提问; 回答}

\noindent[15.]{\Hei 答}:这些资本能出逃,正因为它们不事生產,是游资。

习政府收的摊子很烂,前三年必须先解决更重要的问题,未来应该会在这方面有作为,所以游资更急着要逃。\\

\textit{\hfill\noindent\small 2015/09/12 00:00 提问; 回答}

\noindent[16.]{\Hei 答}:1. 对外併购要使用外匯。这可以看成是把美元现金改成实际资產,所以人民银行其实是鼓励这样做的。
2. 没有正式的机制,只有交易所会自动停盘。中国的环境不一样,自动停盘没什么用。
3. 股市必须成为引导游资进入生產的管道,所以必须严格管理。我个人认为应该从税制开始。\\

\textit{\hfill\noindent\small 2015/12/21 00:00 提问; 回答}

\noindent[17.]{\Hei 答}:石油供过于求,现在生產国憋气,看谁气长。

航空业我不懂(零售业我一般不懂)。\\

\textit{\hfill\noindent\small 2016/09/19 00:00 提问; 回答}

\noindent[18.]{\Hei 答}:慢慢来吧,这种事不能一步到位。\\

\textit{\hfill\noindent\small 2017/06/13 00:00 提问; 回答}

\noindent[19.]{\Hei 答}:美国的页岩油是世界上最容易开采的。

中国的蕴藏,深得多(探测和开采都困难),处于地层破碎地带(现有技术无法有效开采),而且地表往往没有方便的河流(开采页岩油需要大量的水),所以虽然一再追加预算,產量仍然远远落后于原先的计划。\\

\textit{\hfill\noindent\small 2017/06/19 00:00 提问; 回答}

\noindent[20.]{\Hei 答}:几个石油公司都知道这是国家重点项目,也不敢怠慢,但就是连原计划的零头都做不到。

川西那些地方,连开条公路都是世界级的工程,开采页岩油的成本远超过100美元一桶(美国德州的开采成本已经降到16美元),根本不可能有投资效益。\\

\textit{\hfill\noindent\small 2017/06/25 00:00 提问; 回答}

\noindent[21.]{\Hei 答}:会是一大经济阻力。

不过也好,生质能源原本就不靠谱。叶绿素转化阳光的效率只有1\%左右,最新的光伏已经轻松超过20\%。\\

\textit{\hfill\noindent\small 2017/06/25 00:00 提问; 回答}

\noindent[22.]{\Hei 答}:最新的技术,用在最好的油田上,縂成本是16美元,不是边际成本。\\


\section{【海军】【空军】忽悠大眾的虚拟武器}
\subsection{2015-09-12 11:09}


\section{1条问答}

\textit{\hfill\noindent\small 2015/09/15 00:00 提问; 回答}

\noindent[1.]{\Hei 答}:老大和老三的全力压制第二号人物,并不在于后者是否韬光养晦(假设他不太过分),主要取决于前者自己有没有条件去做。美国在2001-2012年在中东陷入泥淖,自顾不暇,中国已经占了大便宜了。

在经济上中国已超越美国,只在外交和军事上还落后。如果中国继续装孙子,则美国可以轻易地用外交和军事力量封锁他的经济,例如IMF的股份和SDR都必然不会有中国的分。正因为习近平建立了亚投行和美国的世界银行对干,IMF才承受了压力,必须考虑中国的利益。如此一来,中国才有和平升起的余地,否则处处碰壁,最后反而只有军事手段可选。\\


\section{【金融】三谈股市}
\subsection{2015-09-17 14:41}


\section{55条问答}

\textit{\hfill\noindent\small 2015/09/17 00:00 提问; 回答}

\noindent[1.]{\Hei 答}:很简单,李嘉诚是典型的非生產性资本,主要靠地產之类的投资赚钱。香港的地產已经没有向上炒作的余地,那么当然只能另谋出路了。\\

\textit{\hfill\noindent\small 2015/09/17 00:00 提问; 回答}

\noindent[2.]{\Hei 答}:我说的不是自由主义,而是自由放任式经济。放任股市涨出泡沫,是典型的美国经济思维。\\

\textit{\hfill\noindent\small 2015/09/17 00:00 提问; 回答}

\noindent[3.]{\Hei 答}:我明知它可能是谣言还拿它出来讨论,是因为它所依靠的背景是真正存在的。这里的读者应该有足够的智商,在适当的警告下,做细微的分辨。\\

\textit{\hfill\noindent\small 2015/09/17 00:00 提问; 回答}

\noindent[4.]{\Hei 答}:抱歉,我不能举出其他的确切例子来,这是因为我是旁观者,没有详细的资料,只有在出了极大的毛病时,才能确定决策有问题。他过去三年的决定,都普遍偏向美国经济教条,我一直看得不是很舒服,但是不能直指为错误。\\

\textit{\hfill\noindent\small 2015/09/17 00:00 提问; 回答}

\noindent[5.]{\Hei 答}:制度上可以先行,但是產业却在衰退。

李嘉诚已经不再试图扩充财富,现在只想为子孙守成。\\

\textit{\hfill\noindent\small 2015/09/17 00:00 提问; 回答}

\noindent[6.]{\Hei 答}:主要是香港做为中国对外的门户,已经不再有意义了。它本身没有任何竞争上的优势,只能让内地的城市一个一个地超越。\\

\textit{\hfill\noindent\small 2015/09/17 00:00 提问; 回答}

\noindent[7.]{\Hei 答}:他对事件过度解读了。其实很简单,以前香港生產的财富被制度和地理环境集中到房地產上,他可以对其进行一次又一次的搜刮;现在香港开始没落,地產没有上涨余地,那么只好出走。\\

\textit{\hfill\noindent\small 2015/09/17 00:00 提问; 回答}

\noindent[8.]{\Hei 答}:主要是给他们合法发财的管道。

美国的官商界是轮流转的,要帮民企的朋友赚钱,有公开合法的贪腐方法,亦即先利益输送,过几年提早退休去当高薪顾问。

欧洲就必须靠政府自己出高薪养廉了。\\

\textit{\hfill\noindent\small 2015/09/17 00:00 提问; 回答}

\noindent[9.]{\Hei 答}:既然是学术界,就可以要求达到学术界的标准,也就是必须提出证据。提不出来是偽造学术成果,直接开除便是。

这种事是肯不肯做的问题,而不是能不能做。\\

\textit{\hfill\noindent\small 2015/09/18 00:00 提问; 回答}

\noindent[10.]{\Hei 答}:典型的仇中论调,全无事实或逻辑基础。

他们这么说已经有近20年了,但是中国的GDP数字向来都只有在事后上调的。上个月例外,把一个季度的数字下调了0.1\%,这些媒体就纷纷庆祝是中共作假的证据。这有点像一月有一天温度到了20度,这些人就宣布这是夏天到了的证据。\\

\textit{\hfill\noindent\small 2015/09/18 00:00 提问; 回答}

\noindent[11.]{\Hei 答}:谢谢指教。\\

\textit{\hfill\noindent\small 2015/09/18 00:00 提问; 回答}

\noindent[12.]{\Hei 答}:这是货币政策,我没有异议。不过这应该是周小川的职权,不是李克强的。\\

\textit{\hfill\noindent\small 2015/09/18 00:00 提问; 回答}

\noindent[13.]{\Hei 答}:听来不是好兆头。\\

\textit{\hfill\noindent\small 2015/09/18 00:00 提问; 回答}

\noindent[14.]{\Hei 答}:推测合理。\\

\textit{\hfill\noindent\small 2015/09/18 00:00 提问; 回答}

\noindent[15.]{\Hei 答}:地方官员不思生產,去与房地產商合作的,被问责是活该。\\

\textit{\hfill\noindent\small 2015/09/18 00:00 提问; 回答}

\noindent[16.]{\Hei 答}:南山卧虫已多次讨论过卢先生的文章,请先读完前文及留言。

税法非我所长,没有意见。\\

\textit{\hfill\noindent\small 2015/09/18 00:00 提问; 回答}

\noindent[17.]{\Hei 答}:李克强幕后的实情,现在还很难说。两年后自然分晓。\\

\textit{\hfill\noindent\small 2015/09/18 00:00 提问; 回答}

\noindent[18.]{\Hei 答}:不事生產的资本自然就是邪恶的资本,容许邪恶资本快速累积的制度将是21世纪里失败的制度。\\

\textit{\hfill\noindent\small 2015/09/18 00:00 提问; 回答}

\noindent[19.]{\Hei 答}:这个表有意思。左边的数据是中央政府欠的债,右边是其中欠外国人的。中国的负债主要来自地方政府,那些债就看不到。日本能撑着,是因为债券持有人是国内的银行和邮局。

谢谢你志愿帮忙《王孟源吧》。\\

\textit{\hfill\noindent\small 2015/09/18 00:00 提问; 回答}

\noindent[20.]{\Hei 答}:军事+人文,有趣的组合。\\

\textit{\hfill\noindent\small 2015/09/18 00:00 提问; 回答}

\noindent[21.]{\Hei 答}:习近平和王岐山在过去三年解决政治敏感问题时,都是从侧下方着手,先处理关键幕僚,最后才直面正主儿。我在旁观摩,印象深刻。\\

\textit{\hfill\noindent\small 2015/09/19 00:00 提问; 回答}

\noindent[22.]{\Hei 答}:Citadel我很熟,是以前的竞争对手。他们的策略不是Soros那样的单向做空。\\

\textit{\hfill\noindent\small 2015/09/19 00:00 提问; 回答}

\noindent[23.]{\Hei 答}:不太可能是剪羊毛,所以我才会把那个香港的谣言拿出来提。李克强真是搞砸了,两年后被明升暗降有相当的可能性。\\

\textit{\hfill\noindent\small 2015/09/19 00:00 提问; 回答}

\noindent[24.]{\Hei 答}:欧美\&ldquo;专家\&rdquo;主要是根据一年多前俄国的经验来期望中国的经济崩溃。当时欧洲开始讨论制裁,很快地几千亿美元的寡头财富就流出国境,致使卢布崩溃,严重打击了俄国经济。不过拿俄国的例子来想在中国复制,未免太小看人了。\\

\textit{\hfill\noindent\small 2015/09/19 00:00 提问; 回答}

\noindent[25.]{\Hei 答}:我并不同意这句话。地方债来自国有银行,最终是人民银行大开腰包,不一定和游资有关系。\\

\textit{\hfill\noindent\small 2015/09/19 00:00 提问; 回答}

\noindent[26.]{\Hei 答}:美国债券的利息是出售时就固定的,所以加息后新发的收益会增加,旧有国债的价值却会大幅缩水。\\

\textit{\hfill\noindent\small 2015/09/19 00:00 提问; 回答}

\noindent[27.]{\Hei 答}:外国资金要进出中国还是很不容易的,所以主要不是外国炒家来做空,而是自己国内的炒家。我猜大概有两类,一种是公司的大股东获利杀出,另一种是游资大户,可能和银行内线合作来炒作套利。\\

\textit{\hfill\noindent\small 2015/09/19 00:00 提问; 回答}

\noindent[28.]{\Hei 答}:一方面说中共没办法堵住外资,一方面说股灾在强力政府下不应该发生,这是自我矛盾,尤其是考虑到管制外匯要比管制股市容易得多。

都已经好几个月了,连证监会和中信的内奸都查出来了,怎么可能查不到外资?

除了强大后台的空头,管理人谋不臧不也是很明显的吗?\\

\textit{\hfill\noindent\small 2015/09/20 00:00 提问; 回答}

\noindent[29.]{\Hei 答}:你说的合情合理,如果真是这样,那么证监会对股市的管理不力就完全不是李克强的责任。但是危机处理只是问题的一半,另一半是当初让股市炒了上去,这有一年左右的时间,就算下面的人故意拖延,也该能施加些压力拿出些结果吧?

其实这些高层运作,外人都只是雾里看花,或许有很多其他的重要内幕我们完全错过了。不论这次的股灾是谁的责任,我想习近平不可能漠视不管,事后应该会有人事和制度上的改革;我们拭目以待。\\

\textit{\hfill\noindent\small 2015/09/20 00:00 提问; 回答}

\noindent[30.]{\Hei 答}:你若是有兴趣,可以找专门书籍来研究,这里我只简单举个例子。

目前最重要的美国国债是10年期的Notes(原本是30年期的Bonds,后来停发了三年,市场就把注意力转移到Notes去了)。假设你买面额\$100,000的Notes,它有固定的2\%的年利率(这些利息叫做Coupon),那么你每年可以收\$2000的现金利息,十年到期后再收一次\&ldquo;本金\&rdquo;\$100,000。但是你买的实际价钱并不是\$100,000,而是由一个公开大拍卖决定的。最近的一次拍卖得到的价钱是\$98,812.50,所以实际的报酬利率比2\%高一些,相当于2.13\%,这就是所谓的Yield。美国新闻对国债的报价都是直接採用Yield。

很明显的,市面上的利息涨了之后,国债的买家也会要求较高的利息,但是Coupon是不变的,所以结果就是拍卖价降低了,使Yield上升。如果你是上一季花高价买的,那对不起,你在一季之间已经损失了那个价差,这可以是很大的数目。例如你在市场利率是2\%时买,那么公平价就是面值\$100,000;等Fed加息后,Yield跳到2.13\%,那么同样的国债只值\$98,812.50了,你转眼就损失了\$1,187.50,亦即半年多的利息。\\

\textit{\hfill\noindent\small 2015/09/20 00:00 提问; 回答}

\noindent[31.]{\Hei 答}:这种资本的贪婪是普世问题,美国在2007-2008年出现的危机中被偷走的国民财富比这次大陆股灾大至少2倍(5兆人民币相比后者的上限2.3兆),事后没有一个人被关,而中共已经很认真地抓了几十个。制度孰优孰劣不是很明显吗?

处身事中的傻子、呆子、疯子会比隔岸冷眼理性的分析可靠?你从出生前就处在重力场中,能写得出广义相对论的重力方程式吗?

我吃过台湾米,所以想为台湾尽心说些实话。说瞎话害死贫苦民眾的人才是国贼,很不幸的,台湾的国贼已经占了多数。到了这个地步,你们还在担心说实话是\&ldquo;唱衰\&rdquo;?你们花了20年把台湾\&ldquo;做衰\&rdquo;了才是真正的问题吧?

不延续自己开始的讨论,只能顾左右而扯逻辑上无关的东西,而且还只是网上的意见,连事实都称不上。这个部落格仅限讨论事实与逻辑,两者皆无是浪费大家的时间,依规矩直接删除。\\

\textit{\hfill\noindent\small 2015/09/20 00:00 提问; 回答}

\noindent[32.]{\Hei 答}:要李嘉诚这种恶性资本家为国服务,只怕是不切实际的妄想。\\

\textit{\hfill\noindent\small 2015/09/21 00:00 提问; 回答}

\noindent[33.]{\Hei 答}:像证监会这种已经烂到根的组织,现在才刚开始抓人;我觉得反腐还有很长的路要走,不过习近平应该原本就计划要一办十年到底。\\

\textit{\hfill\noindent\small 2015/09/21 00:00 提问; 回答}

\noindent[34.]{\Hei 答}:很抱歉,我自己没有必要去读这些书,所以对它们不熟。

你或许可以从Michael Lewis的书着手,好处是他的文笔很好,很容易读,坏处是他不是理工出身,很容易被访问的对象忽悠,所以读者必须时时推想他的资料来源是谁,而这个来源会不会有动机撒谎。\\

\textit{\hfill\noindent\small 2015/09/21 00:00 提问; 回答}

\noindent[35.]{\Hei 答}:买卖国债其实风险不大,适合个人、保险业、退休基金、外匯储备等等保守的投资人。真要买空卖空那要靠金融衍生品,杠杆做到几百倍的都常见。当然,当初在2007年捅出大漏子的瑞联银交易员就是搞了很高的实际杠杆,却找了藉口把它严重低估了。\\

\textit{\hfill\noindent\small 2015/09/21 00:00 提问; 回答}

\noindent[36.]{\Hei 答}:It is a very basic and limited agreement. Let's hope that the US can hold on to the bargain.\\

\textit{\hfill\noindent\small 2015/09/21 00:00 提问; 回答}

\noindent[37.]{\Hei 答}:我以前已经写过两篇文章来讨论英国亲中的战略转向,也提过Osborne是背后的推手,伦敦金融界的代言人,所以他想以支持人民币进SDR来交换沪伦通是意料之中的事。中方大概会要求英国也买核电,这将是对中国核电的绝大广告,有很大的商业意义。

这次的习奥会应该和几个月前一样,异中求同,签署几个双方都有利的协定。中方的策略似乎是避实击虚,躲开美国政治仇中的力道,专注于鼓励大公司为中美间的商业利益而代言。在西雅图停留,主要应该是希望波音争取更宽松的技术输出,和警告微软可能被逐出中国市场如果美国政府继续炒作网络黑客的问题。我自己也还在等更多的细节。\\

\textit{\hfill\noindent\small 2015/09/22 00:00 提问; 回答}

\noindent[38.]{\Hei 答}:必须先把证监会整个清空重建,银监会也必须大批换人。如果一年内能做到这样,我同意是短空长多。\\

\textit{\hfill\noindent\small 2015/09/22 00:00 提问; 回答}

\noindent[39.]{\Hei 答}:这的确可能成为日本经济突然崩溃的导火线,不过机率不是太高,主要是在国际政治上一旦进了SDR就很难把一个货币踢出去。最可能的是日本在物价高涨和经济停滞双重压力下,以十年的时间逐步把中產阶级的生活水准压到开发中国家的层次。\\

\textit{\hfill\noindent\small 2015/09/22 00:00 提问; 回答}

\noindent[40.]{\Hei 答}:这个作者还算客气的,没有直指证监会坍方式腐败,其实我觉得那很有可能是背后的真相。\\

\textit{\hfill\noindent\small 2015/09/23 00:00 提问; 回答}

\noindent[41.]{\Hei 答}:两者无关。CIPS是用来取代SWIFT的。参见《中国燕子》。\\

\textit{\hfill\noindent\small 2015/09/23 00:00 提问; 回答}

\noindent[42.]{\Hei 答}:李克强的前途我也觉得可虑,不过还有可能留任;证监会那些人只怕个个都等着进牢。\\

\textit{\hfill\noindent\small 2015/10/15 00:00 提问; 回答}

\noindent[43.]{\Hei 答}:我觉得那个作者是被美式经济学洗脑过的,这样的人在证监会任职正是问题所在。\\

\textit{\hfill\noindent\small 2015/10/15 00:00 提问; 回答}

\noindent[44.]{\Hei 答}:我也觉得像\&ldquo;我妈是我妈\&rdquo;这个事件的公共讨论不够理性。别人的妈是谁,当然不是官僚能知道的;问题只在于是否有必要确认个人的母亲是谁。如果是意外伤害后的通知,那么就无关紧要;如果是有关社会福利的发放,那么可能就有必要看户口了。\\

\textit{\hfill\noindent\small 2015/10/15 00:00 提问; 回答}

\noindent[45.]{\Hei 答}:拿铅笔来解释自由市场能\&ldquo;自动\&rdquo;完成复杂的產业链管理,是美国经济学教科书常用的例子,可是它本身就是明显错误的。在这个例子中,美式经济学做了很多隐性的假设,例如没有坏人,也没有霸占市场的大户;实际上如果没有政府的积极管理,必然会有强盗,那么笔芯和木料就不能自由交换,也必然会有人组织起来独霸市场,那么价钱就会衝上天了。他们连这么基本的论述都得靠不切实际的假例子,可见其心虚的程度。\\

\textit{\hfill\noindent\small 2015/10/15 00:00 提问; 回答}

\noindent[46.]{\Hei 答}:物理人转金融是不得已的:物理界的教职越来越少,学生却越来越多,所以衹有运气好又肯坚持的人才能留下。转行金融的,也不见得就有独立的逻辑思考能力;大多数还是为了华尔街的财富而目眩神迷,以为有钱就是有道理。

这篇文章颇有问题,尤其是讲美国和韩国的段落,基本都是作者自己人云亦云,想当然耳的结果,经不起事实和逻辑的推敲。不过我很懒得吵这些事:军事问题还比较简单,经济上的来龙去脉不是三言两语能说清楚的。\\

\textit{\hfill\noindent\small 2015/10/16 00:00 提问; 回答}

\noindent[47.]{\Hei 答}:美国的霸权有很大部分靠宣传来建立并维持,很多庸人没有独立的逻辑思考能力,落入其陷阱而不自觉。一般百姓既没有能力也没有时间来思考这些问题,所以拨乱反正是头脑清楚的知识分子的责任。\\

\textit{\hfill\noindent\small 2015/10/16 00:00 提问; 回答}

\noindent[48.]{\Hei 答}:马克思对资本主义的批评是十分到位的。共產主义在20世纪的实行问题并不代表资本主义的绝对优越性,实际上必然还有更好的制度。如果我们没有事实证据、忽略了很强的逻辑论述,一头栽进\&ldquo;歷史的终结\&rdquo;这样的结论,那么就正中了既得利益者的下怀。\\

\textit{\hfill\noindent\small 2015/11/02 00:00 提问; 回答}

\noindent[49.]{\Hei 答}:他是典型的短綫炒手,所以我一直鼓吹对短綫交易收重税,釜底抽薪。\\

\textit{\hfill\noindent\small 2015/11/15 00:00 提问; 回答}

\noindent[50.]{\Hei 答}:我已写新的文章来讨论此事。\\

\textit{\hfill\noindent\small 2016/07/15 00:00 提问; 回答}

\noindent[51.]{\Hei 答}:不用负担后果的庸人,往往以为这是可以随意胡扯的机会,臺湾尤其如此。\\

\textit{\hfill\noindent\small 2017/09/16 00:00 提问; 回答}

\noindent[52.]{\Hei 答}:短期的经贸资料波动,主要是随机的,对大势来説没有什么意义。但是金融机构的分析师们却必须每个月发好几篇文章,所以自然是无中生有,拿随机波动来説事。\\

\textit{\hfill\noindent\small 2021/06/28 15:48 提问;2021/06/28 22:11 回答}

\noindent[53.]{\Hei 答}:1.因爲他們故意對外宣佈“軟銀做了投資”,導致市場的韭菜們把風險評估值下調,於是所有的金融資產價值自然因而上升,尤其内含杠桿越高的衍生品,價值升得越多,包括Convertible Bonds。當然,Convertible不是杠桿最高的金融工具,但因爲他們計劃立刻脫手兌現,必須考慮流通性,所以Convertible是最佳平衡。

2.這種轉讓,不像股市那樣一切都標準化,買家賣家必須簽訂契約,契約一般有標準版,但雙方(尤其是買家)可以要求添加條文。《Economist》的意思是這次的買家很謹慎,訂下了若干時間内不能破產的前提;這不一定準確,如果是真的,代表著軟銀的這批人也沒有意識到Wirecard的實際財務有多糟糕。換句話説,螳螂捕蟬、黃雀在後,他們被Wirecard的老闆作假帳騙了。

20年前,我和Deutsche Bank打交道的時候,就注意到那些美籍高管代表公司做交易,如果看到特別划算的生意,會先拖時間,然後再設法以個人身份和資金參與。很多人都有類似的經驗,所以我以前反復説過,DB的紀律差,名揚華爾街。
\\

\textit{\hfill\noindent\small 2021/11/26 17:04 提问;2021/11/27 04:31 回答}

\noindent[54.]{\Hei 答}:美式企業管理人想方設法為財務報表塗脂抹粉,當然是爲了給自己發錢正名。但這些花樣是無法永久持續的,所以原本就必須每隔一段時間,找一個藉口(經濟周期、政治風險、天災人禍等等),把隱藏的損失公開出來。像是中國現在正在嚴打無良資本,自然是非常合適壓縮利潤、一舉多得的理由。
\\

\textit{\hfill\noindent\small 2022/02/22 16:36 提问;2022/02/23 05:53 回答}

\noindent[55.]{\Hei 答}:我早説過,反腐知易行難,是極端艱巨的實踐性工作;我能做的貢獻,只在於澄清認知誤區,揭穿利用專業知識壁壘來做忽悠的騙術。
\\


\section{【国际】英国二三事}
\subsection{2015-09-23 00:04}


\section{6条问答}

\textit{\hfill\noindent\small 2015/09/23 00:00 提问; 回答}

\noindent[1.]{\Hei 答}:台湾的金融界再怎么贼,也胜不过老美,所以应该是会往大陆发展。\\

\textit{\hfill\noindent\small 2015/09/23 00:00 提问; 回答}

\noindent[2.]{\Hei 答}:他们行动很隐秘,我对英国的情况完全不了解。\\

\textit{\hfill\noindent\small 2015/09/23 00:00 提问; 回答}

\noindent[3.]{\Hei 答}:日本。\\

\textit{\hfill\noindent\small 2015/09/24 00:00 提问; 回答}

\noindent[4.]{\Hei 答}:我的层次没有高到和中央银行幕后的黑手打交道的地步,不能提供特别的见解。

我个人的看法是犹太资本并不会因为你是犹太人而给予特别待遇。这种互相扶持的效应主要是个人的偏爱,在影视界、学术界、媒体和律师间常见,但是大资本天生就是很自私的,所以基本上是家族的独霸,只是这些家族刚好是犹太人。\\

\textit{\hfill\noindent\small 2015/09/25 00:00 提问; 回答}

\noindent[5.]{\Hei 答}:犹太资本一般只有在维护以色列利益的时候才会惊鸿一瞥,稍现踪迹。他们是现有美国霸权体系的最大受益人,对中国的崛起必然会有所反应。\\

\textit{\hfill\noindent\small 2015/09/27 00:00 提问; 回答}

\noindent[6.]{\Hei 答}:Merkel只会讨好选民,搞内部的政治斗争,完全没有前任的眼光和魄力,德国衰矣。\\


\section{【空军】【海军】共军小道消息刷新(2015年九月特刊)}
\subsection{2015-09-29 21:23}


\section{5条问答}

\textit{\hfill\noindent\small 2015/10/02 00:00 提问; 回答}

\noindent[1.]{\Hei 答}:除了政府之外,没有力量可能强大到能制衡资本和其他社会上的利益集团,所以一个有强大执行力的独立政府是国家进步的必要条件。\\

\textit{\hfill\noindent\small 2015/10/06 00:00 提问; 回答}

\noindent[2.]{\Hei 答}:中国的稀土存量其实不是特别多,重稀土占世界总存量的比率高些,但也不是绝对性的优势。主要是中国採矿不管环境保护,所以特别便宜,结果十几年下来其他国家竞争不过就封场了。中国就是完全停產,对世界市场也只有一两年的震撼。可是稀土除了永久性磁铁之外,其他的用途用量都很小,所以大的用户都有好几年的库存。

此外,中国不像美国一样独霸国际媒体的话语权。美国人可以一年炸死几千个无辜百姓,还反过来只骂俄国;中国什么坏事都没干,就到处躺枪了。像是封锁稀土资源这种事还是不要尝试的好。\\

\textit{\hfill\noindent\small 2015/10/06 00:00 提问; 回答}

\noindent[3.]{\Hei 答}:我自己不喜欢在没有新细节的情况下做太多猜测。

你可以看看晨枫的文章;他的推测基本合理。\\

\textit{\hfill\noindent\small 2015/10/07 00:00 提问; 回答}

\noindent[4.]{\Hei 答}:大体同意,不过指望民进党改善经济似乎不切实际。\\

\textit{\hfill\noindent\small 2015/10/08 00:00 提问; 回答}

\noindent[5.]{\Hei 答}:不管哪一人上台都会急着签TPP,所以这和蓝绿、台独都没有关系。\\


\section{【德国】【美国】舍生取义的政治人物}
\subsection{2015-10-08 16:02}


\section{10条问答}

\textit{\hfill\noindent\small 2015/10/08 00:00 提问; 回答}

\noindent[1.]{\Hei 答}:清朝、日本和民国都或多或少地抑制了台湾人对歷史、文化、政治、社会和尤其是哲学的深刻了解,也难怪台湾的民主政治会有先天不足的问题;但是后天失调就是这代人自己的责任了。\\

\textit{\hfill\noindent\small 2015/10/08 00:00 提问; 回答}

\noindent[2.]{\Hei 答}:大陆在78年后的自由流通是个很有意思的论点。这不但有经济和文化的好处,在政治上也应该会培育出对国家整体的认同。\\

\textit{\hfill\noindent\small 2015/10/12 00:00 提问; 回答}

\noindent[3.]{\Hei 答}:我想他们并没有走得很左,只是从30年多前开始的极右路线往回踏了几步。美国霸权衰退之后,绝对自由经济不再被当做万灵药,真正的深刻反思还没有普及化,但是极右的反射性衝动已经不再流行了。

我说过很多次,本世纪的歷史主流是要解决三个问题:贫富不均、全球暖化和霸权转移;其中贫富不均是最重要也最困难的。\\

\textit{\hfill\noindent\small 2015/10/13 00:00 提问; 回答}

\noindent[4.]{\Hei 答}:连没有A钱考虑时,也是依民调治国;经济规律才不会管愚民高兴什么呢。\\

\textit{\hfill\noindent\small 2015/10/13 00:00 提问; 回答}

\noindent[5.]{\Hei 答}:迪顿的理论不就是我讲过很多次的组织执行能力决定经济前途吗?\\

\textit{\hfill\noindent\small 2015/10/14 00:00 提问; 回答}

\noindent[6.]{\Hei 答}:经济是李克强管的;我们再等两年看看习近平是否继续容忍下去。\\

\textit{\hfill\noindent\small 2015/10/17 00:00 提问; 回答}

\noindent[7.]{\Hei 答}:目前最合理的推测是国民党怕立委席次低于1/3,民进党将能剥夺其党產。朱立伦知道选不上总统,但是四年主席当下来,还是可以从党產上刮下不少油水,所以衹好临危受命,出头竞选。至于他是否真能提拉立委的选情,那是另当别论了。

臺湾的经济已经即将玩完了,\&ldquo;日子过得好好的\&rdquo;衹怕不再适用。\\

\textit{\hfill\noindent\small 2015/10/21 00:00 提问; 回答}

\noindent[8.]{\Hei 答}:因为习近平正在访问英国,所以有很多统计数字被刊出来。我很吃惊的是过去十年中国在欧洲最大的投资对象居然是英国和意大利,而不是德国。那么Merkel连在对华贸易的这个关键事项上,也是尸位素餐。Schroeder大概在边綫上,看得心痛了。\\

\textit{\hfill\noindent\small 2015/10/21 00:00 提问; 回答}

\noindent[9.]{\Hei 答}:我想你指的是九月初在留言栏的讨论,我说Merkel开放移民是玩小聪明,故做慷慨,实际上是要拖整个欧洲一起来买单。

现在病急乱投医,找上土耳其来当救生圈。这些问题,追根究底,就是Merkel没有远见,没有一贯的战略,一味短綫操作,衹想满足民意,敷衍了事。Syria的难民很难预见吗?那么北非的难民已经闹了好几年了,藉口在哪里?近五年来,大街小巷讨论欧洲前途的文章,没有一篇不提难民问题,结果Merkel衹想坐在欧洲霸主的宝座上,不但不能未雨绸缪,连紧急预案都没有准备。

Renzi在意大利搞真的改革,Osborne在英国做大战略的突破,Merkel却在虚耗,德国危矣。\\

\textit{\hfill\noindent\small 2015/10/21 00:00 提问; 回答}

\noindent[10.]{\Hei 答}:中国对欧洲的投资还远落后于美国,不过英国的示范作用极大,德法现在应该会忙不迭地跟进了。\\


\section{【美国】知识產权的财团独霸}
\subsection{2015-10-16 04:40}


\section{3条问答}

\textit{\hfill\noindent\small 2015/10/16 00:00 提问; 回答}

\noindent[1.]{\Hei 答}:不论是为了美国的私利或者中国的私利,危害全人类的福祉都不是件好事。我还是寧可中国在取代美国霸权的过程中,顺便把一些以强欺弱的不合理现象改掉了。\\

\textit{\hfill\noindent\small 2015/10/16 00:00 提问; 回答}

\noindent[2.]{\Hei 答}:There is a small chance that the new Canadian government may object to TPP. Let's hope it happens elsewhere too.\\

\textit{\hfill\noindent\small 2015/10/17 00:00 提问; 回答}

\noindent[3.]{\Hei 答}:这就是现代美国的霸权文化:明明是自私自利的手段,偏偏硬说成普世价值。拉美、亚洲和非洲的精英在受骗了三、四代人之后,终于有醒悟的迹象;臺湾却是説什么都不肯学乖,仍旧坚持要一头栽进美国人的陷阱,怎不令人摇头嘆息。\\


\section{【美国】【金融】美国式的恐龙法官(三)}
\subsection{2015-10-20 02:11}


\section{7条问答}

\textit{\hfill\noindent\small 2015/10/20 00:00 提问; 回答}

\noindent[1.]{\Hei 答}:这种事,魔鬼是在细节里(Devil is in the details)。我掌握的细节不够。\\

\textit{\hfill\noindent\small 2015/10/21 00:00 提问; 回答}

\noindent[2.]{\Hei 答}:王岐山自己也是金融业出身,这些人要在他眼前玩花样,真是关公面前耍大刀了。\\

\textit{\hfill\noindent\small 2015/10/21 00:00 提问; 回答}

\noindent[3.]{\Hei 答}:其实我这几个月看证监会出臺的新规则,就越看越摇头。明明有简单明瞭、一句话讲完的办法不做,拼命发表十几页长、极为繁琐的条文,很明显地就是阳奉阴违,故意把规则订成外行看不懂、内行钻漏洞。这是美国人的专长,我看了20年了,他们也来搞这套,唉。\\

\textit{\hfill\noindent\small 2015/10/21 00:00 提问; 回答}

\noindent[4.]{\Hei 答}:我在《三谈股市》里就説,最后一个阶段是中纪委出手处理善后。现在他们进驻证监会,很让人高兴。\\

\textit{\hfill\noindent\small 2015/10/21 00:00 提问; 回答}

\noindent[5.]{\Hei 答}:为底层人命谋福利的使命感,还得加上理性科学的执行态度才行;现代世界大概衹有中国还有希望能同时做到这两点。\\

\textit{\hfill\noindent\small 2015/10/23 00:00 提问; 回答}

\noindent[6.]{\Hei 答}:光是关人还不够,必须从制度上亡羊补牢。我一再强调,股市短綫操作,不是内綫交易,就是赌博,必须以重税节制;可惜至今还是没有明白事理的官员能下决心做正确的决策。\\

\textit{\hfill\noindent\small 2015/10/27 00:00 提问; 回答}

\noindent[7.]{\Hei 答}:官员的财產都是退休后在台面上公开领的顾问高薪,或者像克里顿演讲一小时25万美元的收入,公布了又能怎么样?

金融业、军工业和能源业腐败最严重,其他的行业不是不想收买官员,衹是还没做到金融业的水准。\\


\section{【工業】再談中國的核電發展}
\subsection{2015-10-23 19:17}


\section{4条问答}

\textit{\hfill\noindent\small 2015/11/05 00:00 提问; 回答}

\noindent[1.]{\Hei 答}:这种事还是由政治精英(前提是有理想,所以不能是财阀)决定,有效率的多。\\

\textit{\hfill\noindent\small 2015/11/05 00:00 提问; 回答}

\noindent[2.]{\Hei 答}:必然谈不出什么重要议题,纯粹是象徵性的姿态。

至于会后的政策怎么走,我个人认为应该完全放下政治考虑,专注在经济整合和社会救助上,毕竟臺湾社会被自己玩砸了,烂摊子最后还是得由中方来收拾。\\

\textit{\hfill\noindent\small 2015/11/06 00:00 提问; 回答}

\noindent[3.]{\Hei 答}:臺湾极度崇洋,其实可以借力打力:衹要有国际压力,就可以压下岛内的政治反对。例如紫光要兼并联发科,把WTO扯进来,威胁制裁,臺湾自然会乖乖就范。\\

\textit{\hfill\noindent\small 2021/11/18 03:28 提问;2021/11/18 03:49 回答}

\noindent[4.]{\Hei 答}:理論和技術上都毫無問題,關鍵在於經濟性:新科技的實際應用對性價比錙銖必較,20\%的差異就保證會被市場淘汰,像是核聚變那種8、9個數量級的距離,根本就是天方夜譚。SMR能否全面勝出,還在未定之天,但至少在海島和偏遠地區供電、供熱上有個基本Niche,中國這樣地理複雜的大國有其需要;你所設想的外銷市場,在煤電剛被否決的前提下,更加是重要的考慮,因此整體來看,對SMR的投資應該遠高於(亦即至少兩個數量級)核聚變和量子計算才對,結果十四五反其道而行,希望十五五能撥亂反正。
\\


\section{【政治】自断后路的狷介清官}
\subsection{2015-10-27 22:40}


\section{6条问答}

\textit{\hfill\noindent\small 2015/10/28 00:00 提问; 回答}

\noindent[1.]{\Hei 答}:绝对公平无法定义,衹有大略的公平。

此外精英治国是理所当然的,难道要选拔蠢蛋来治国吗?这里的要点是精英必须衹以品德和能力来定义,不是由财富和家世决定。\\

\textit{\hfill\noindent\small 2015/10/28 00:00 提问; 回答}

\noindent[2.]{\Hei 答}:你说的没错,但是他已经答应封口三个月,大家就放他一马吧。\\

\textit{\hfill\noindent\small 2015/11/06 00:00 提问; 回答}

\noindent[3.]{\Hei 答}:I still have hopes that countries like NZ or Canada will find the backbone to say no before it is too late.\\

\textit{\hfill\noindent\small 2015/11/06 00:00 提问; 回答}

\noindent[4.]{\Hei 答}:美国若是和欧洲国家一个一个谈,自然会有大困难。但是现在他的谈判对手是欧盟,这是一个超于国家的机关,不受选举和地方经济利益节制;而这些官僚都有亲美仇中的趋向。Merkel若是觉得兹事体大,可以施加影响力,但是我觉得她并不明白(或在乎)TTIP有多危险。

越南的确是为了5-10年的近利,而上了美国人的当;但是这5-10年中,中国的经济还是要转型,百姓也仍旧要吃饭。\\

\textit{\hfill\noindent\small 2015/11/13 00:00 提问; 回答}

\noindent[5.]{\Hei 答}:清理证监会是件大大的好事,但是股市本身的规则也该改变才行;可惜美式股市的根本毛病,若不是在华尔街混过很长时间,不太可能看得出来。\\

\textit{\hfill\noindent\small 2015/12/09 00:00 提问; 回答}

\noindent[6.]{\Hei 答}:我上周也注意到傅政华的新任命(我想是Reuters先报导了),同样也猜测是大举整顿财经界的前奏。

不过财经的专业度很高,光是要求干部清廉还不够,必须从根本的规则上抑制强豪才行。\\


\section{【空軍】即將出現的新裝備(一)}
\subsection{2015-10-31 06:39}


\section{2条问答}

\textit{\hfill\noindent\small 2015/11/01 00:00 提问; 回答}

\noindent[1.]{\Hei 答}:其实这很好。空间站和登月都是耗资不菲的形象工程,基本没有科学和实业上的用途;习近平一贯脚踏实地,在航天上也务实我是很乐见的。\\

\textit{\hfill\noindent\small 2021/08/01 03:20 提问;2021/08/01 04:47 回答}

\noindent[2.]{\Hei 答}:你讀這個博客也有一段時間了,憑空捏造的結論還看不出來?論證的基礎呢?邏輯架構何在?這種宗教教義式的口號沒有什麽好討論的,就是空話一句,或者用林肯的話“用料是鴿子影子的一碗湯”。
\\


\section{【海军】即将出现的新装备(三)}
\subsection{2015-11-04 20:40}


\section{1条问答}

\textit{\hfill\noindent\small 2015/11/28 00:00 提问; 回答}

\noindent[1.]{\Hei 答}:中国是產钨大国,但是其他一些重要的军工原料十分匮乏,例如铬。

美国也不是样样都盛產,但是因为二战和冷战的关系,有需要的都大量囤积了。

此外长程精确制导武器的发展普及,代表着二战那样的消耗战已一去不復返了。\\


\section{【經濟】談全球暖化}
\subsection{2015-11-11 05:16}


\section{14条问答}

\textit{\hfill\noindent\small 2015/11/11 00:00 提问; 回答}

\noindent[1.]{\Hei 答}:差别是世界大部分民众了解全球暖化的危险,臺湾民众既无知也不在乎。\\

\textit{\hfill\noindent\small 2015/11/11 00:00 提问; 回答}

\noindent[2.]{\Hei 答}:臺湾还会变得更乾更热;更乾所以需要水,更热所以需要电(冷气)。绿营不在乎缺水缺电,未来衹会更糟糕。\\

\textit{\hfill\noindent\small 2015/11/11 00:00 提问; 回答}

\noindent[3.]{\Hei 答}:是好是坏都是几代人之后的事,有胆当的政治家衹是对子孙负责任。\\

\textit{\hfill\noindent\small 2015/11/11 00:00 提问; 回答}

\noindent[4.]{\Hei 答}:环保型制造业是必然的趋势;不过这不代表你应该买他们的股票。

希望这次巴黎会议能谈出点有实际意义的约定。\\

\textit{\hfill\noindent\small 2015/11/11 00:00 提问; 回答}

\noindent[5.]{\Hei 答}:一百年后,黄河流域有如今日的长江,江南有如今日的广东,广东则有如今日的中南半岛。农產只增不减,但是生活品质会下降。\\

\textit{\hfill\noindent\small 2015/11/11 00:00 提问; 回答}

\noindent[6.]{\Hei 答}:bitcoin不是货币,而是投机工具;它真正的贡献在于软体上的一些创新。

离题了;请勿回復。\\

\textit{\hfill\noindent\small 2015/11/11 00:00 提问; 回答}

\noindent[7.]{\Hei 答}:我个人是很喜欢牛排的,所以把甲烷算作农业生產排放。\\

\textit{\hfill\noindent\small 2015/11/11 00:00 提问; 回答}

\noindent[8.]{\Hei 答}:问题在于中国一家减排没有用,印度人衹觉得可以趁机大肆污染;这在经济学上叫做Tragedy of the Common,也就是占整体社会的便宜。\\

\textit{\hfill\noindent\small 2015/11/12 00:00 提问; 回答}

\noindent[9.]{\Hei 答}:我也注意到了。

刚好印证了正文中有关世界问题根源的结论。\\

\textit{\hfill\noindent\small 2015/11/12 00:00 提问; 回答}

\noindent[10.]{\Hei 答}:我想主要的困难在于欧盟的官僚都被美国宣传彻底洗脑过,对中国自然抱有双重标准。\\

\textit{\hfill\noindent\small 2015/11/14 00:00 提问; 回答}

\noindent[11.]{\Hei 答}:我倒觉得美国最大的威胁是贫富不均。财阀们若是不能从国外输血,对内的压榨就衹能变本加厉,最终就是拉美化。\\

\textit{\hfill\noindent\small 2015/11/15 00:00 提问; 回答}

\noindent[12.]{\Hei 答}:主流媒体必须滥情,一方面争取眼球,一方面避免谈论贫富不均或社会上其他真正问题。明智的人无须随之起舞。\\

\textit{\hfill\noindent\small 2015/11/28 00:00 提问; 回答}

\noindent[13.]{\Hei 答}:I used to have high hope for NZ. Now we can only pin our hope on the new Canadian government.\\

\textit{\hfill\noindent\small 2021/08/07 06:41 提问;2021/08/11 03:07 回答}

\noindent[14.]{\Hei 答}:中國在執行效率上,的確有明顯的優勢;我擔心的是技術路綫選擇的問題。在這一方面,中國反而因爲剛剛後來居上而完全沒有經驗,再加上貪腐自私的學術風氣,科技圈子把大環境的需要當作自己發財的機會,視國家社會公益為無物,例如核聚變和氫能源,都沒有人挺身而出來說實話。我一個人在海外敲邊鼓,勢單力薄,實在很怕政府被忽悠進無用的投資,那樣的浪費將不只是中國的損失,也是全世界的“99\%”的損失。
\\


\section{【国际】市场经济地位}
\subsection{2015-11-16 00:08}


\section{45条问答}

\textit{\hfill\noindent\small 2015/11/16 00:00 提问; 回答}

\noindent[1.]{\Hei 答}:如果有政治意愿,总是能想出损人利己的办法。

我并不是说应该真做,而是必须让Merkel知道兹事体大。\\

\textit{\hfill\noindent\small 2015/11/16 00:00 提问; 回答}

\noindent[2.]{\Hei 答}:IMF在美国掌控之下,就永远是世界的祸根。中国必须先和欧洲完成合作,才能回过头来处理IMF。\\

\textit{\hfill\noindent\small 2015/11/16 00:00 提问; 回答}

\noindent[3.]{\Hei 答}:Yes, and that is why the US is ready to ignore WTO rules and continues to treat China as "non-market economy". Not only does it hurt China, but it also weakens the WTO.

Once China twists Merkel's arms into abiding by the WTO rules, the US will be further isolated internationally.\\

\textit{\hfill\noindent\small 2015/11/16 00:00 提问; 回答}

\noindent[4.]{\Hei 答}:我认为不被承认的后果很严重,基本代表欧洲站到美国那一边,和中国公开决裂了。

中国有很多牌可打。如果是我,现在就会先\&ldquo;整顿\&rdquo;一下大众汽车在中国的分公司,做出随时可以把它踢出中国市场的态势。不过中共当局的外交一向很软,有他们自己的考虑。\\

\textit{\hfill\noindent\small 2015/11/17 00:00 提问; 回答}

\noindent[5.]{\Hei 答}:被宣传洗脑之后,人云亦云,宣传成了连锁反应;这正是美式制度下宣传体系的特徵。\\

\textit{\hfill\noindent\small 2015/11/17 00:00 提问; 回答}

\noindent[6.]{\Hei 答}:你太高估我的影响力了。理性的人所见略同而已。\\

\textit{\hfill\noindent\small 2015/11/17 00:00 提问; 回答}

\noindent[7.]{\Hei 答}:我不是鼓吹真的闹翻,衹是德国的损失会远比中国严重,依西方的霸权功利思维,就是敲诈的良机;更何况中方并不是敲诈,而衹是要求欧洲遵守国际法。\\

\textit{\hfill\noindent\small 2015/11/17 00:00 提问; 回答}

\noindent[8.]{\Hei 答}:SDR是大半衹有象徵性意义的小事一件,中国不会出高价。\\

\textit{\hfill\noindent\small 2015/11/17 00:00 提问; 回答}

\noindent[9.]{\Hei 答}:欧洲决策者若是能理性思考,自然是如此。但是他们受美国传媒和自我种族偏见影响,先天就爱找亚洲强国的麻烦;与此同时,钢铁业和光伏业再在就业率问题上闹一闹,后果就有不确定性了。

我个人觉得中共应该未雨绸缪,早下伏笔,准备逼他们就范;不过习政府似乎还是决定来软的。\\

\textit{\hfill\noindent\small 2015/11/17 00:00 提问; 回答}

\noindent[10.]{\Hei 答}:08年时,高盛把有毒的资產都事先卖给无知的顾客,这些无知的顾客主要就是德国的中小银行。

中国的经济是实业型的,其產量永远会大于本国原料供应,也大于本国市场需求。欧洲不但是第一大市场,也是国际贸易制度的标杆,所以才会有一带一路的大战略,中共怎么会不在乎欧洲的态度?\\

\textit{\hfill\noindent\small 2015/11/17 00:00 提问; 回答}

\noindent[11.]{\Hei 答}:没有。瑞信是很大的公司。\\

\textit{\hfill\noindent\small 2015/11/17 00:00 提问; 回答}

\noindent[12.]{\Hei 答}:美国商界认定中国做生意完全靠关系,房地產尤是。\\

\textit{\hfill\noindent\small 2015/11/17 00:00 提问; 回答}

\noindent[13.]{\Hei 答}:其实习近平自己在昨天才说了2008年后,中国为世界而牺牲,大幅投资以刺激经济;我的解读是中共中央已有共识:当时是被美国忽悠,举债过度了。

如果过去三年是胡温继续当政,我同意会有很大的经济萧条的危险。习政权已经针对性地在处理问题,衹是必须小心从事、慢慢化解,自然不愿有人在戏院里大喊失火了。\\

\textit{\hfill\noindent\small 2015/11/18 00:00 提问; 回答}

\noindent[14.]{\Hei 答}:美日无可救药,欧洲才是关键。\\

\textit{\hfill\noindent\small 2015/11/18 00:00 提问; 回答}

\noindent[15.]{\Hei 答}:应该是没问题的,但是我觉得可以再积极一些。\\

\textit{\hfill\noindent\small 2015/11/18 00:00 提问; 回答}

\noindent[16.]{\Hei 答}:地方官员保护僵尸企业的问题,衹怕还是要等到2017年整体经济开始復苏之后,比较容易解决。

这个问题衹是2008年后过度举债刺激投资的后果之一,不太可能马上处理尽净。\\

\textit{\hfill\noindent\small 2015/11/18 00:00 提问; 回答}

\noindent[17.]{\Hei 答}:50年代是让移民进来工作,过去20年失业率高居不下,早已纯是为了实践人道主义信仰。

华人是黄种人,歧视还是比较严重的。\\

\textit{\hfill\noindent\small 2015/11/18 00:00 提问; 回答}

\noindent[18.]{\Hei 答}:原本是法律问题,他们却顾左右而言他,明显就是要耍赖。\\

\textit{\hfill\noindent\small 2015/11/18 00:00 提问; 回答}

\noindent[19.]{\Hei 答}:同意。\\

\textit{\hfill\noindent\small 2015/11/18 00:00 提问; 回答}

\noindent[20.]{\Hei 答}:服务业的就业容纳量比制造业大,所以现在还没有严重的失业问题。

举债和房地產泡沫都已不能持续,降速不是选择,而是必要的。\\

\textit{\hfill\noindent\small 2015/11/19 00:00 提问; 回答}

\noindent[21.]{\Hei 答}:唯一的出路是加入中国的產业链,可是连对方主动要求的都不肯合作。。。\\

\textit{\hfill\noindent\small 2015/11/24 00:00 提问; 回答}

\noindent[22.]{\Hei 答}:我觉得习近平是个剑及履及的明白人,处理这事衹是时间问题。\\

\textit{\hfill\noindent\small 2015/11/24 00:00 提问; 回答}

\noindent[23.]{\Hei 答}:还有一个很大的背景,你没有提,就是人口结构的变化。中国的工作适龄人口在现在刚好达到顶峰,所以从此就业问题的压力会越来越小。邓小平改革开放的时候,不敢也不能把大批国企员工丢到马路上让其自生自灭;现在已经没有这个顾虑,再不做就真是乡愿了。\\

\textit{\hfill\noindent\small 2015/11/24 00:00 提问; 回答}

\noindent[24.]{\Hei 答}:中东的乱局是英、法、美百年下来一连串政策的有意结果。他们除了你所说的解除文明威胁之外,也有保障长期石油供应的考虑。后来的动乱,虽然偶尔让油价高涨,宏观来説,其实避免了更紧更贵的石油供应。以往是欧美得利,现在则加惠中国,从美国人的观点就是中国不劳而获,但是他们又不能明説这个\&ldquo;劳\&rdquo;是指故意搞乱中东,所以衹能说中国没有驻军海外、保护航道。我想美国智库的那些人,在写这些文章的时候,必然真的是咬牙切齿的。

欧洲的确很可能会走下道德神坛。反正民主制度下,什么污烂事都可以由投票正名洗脱。未来几年的大选结果值得关注。\\

\textit{\hfill\noindent\small 2015/11/27 00:00 提问; 回答}

\noindent[25.]{\Hei 答}:再看吧。\\

\textit{\hfill\noindent\small 2015/11/28 00:00 提问; 回答}

\noindent[26.]{\Hei 答}:习近平把大方向认得清楚,大方针掌握得对,是很明显的。我怕的是在专业方面,例如金融,一些被西方宣传洗脑过的专家即使是认真办事,仍然没有足够的见识来采行正确的非西方式政策。我对股市法规的建议,从来没有任何其他人谈过,所以被采行的机率实在很小,那么中国的融资市场就永远是美国的那一套,也就是先天为财阀谋利的制度。\\

\textit{\hfill\noindent\small 2015/11/29 00:00 提问; 回答}

\noindent[27.]{\Hei 答}:谢谢你的挂念,我没有什么特别的事和他讨论。\\

\textit{\hfill\noindent\small 2015/11/30 00:00 提问; 回答}

\noindent[28.]{\Hei 答}:美国人的自私自利,是70年代财阀掌权后的副作用。\\

\textit{\hfill\noindent\small 2015/12/09 00:00 提问; 回答}

\noindent[29.]{\Hei 答}:是的,中国的体量相当于目前所有欧美和东亚先进国家的总和,中国向上爬,对这些先进国家和地区有强大的挤压效应,衹有少数与中国密切合作的(如德、韩)才能获益。\\

\textit{\hfill\noindent\small 2015/12/09 00:00 提问; 回答}

\noindent[30.]{\Hei 答}:大陆的平均薪资刚刚超过俄国。

过去三年,习近平不再专注在GDP成长上,而把重点放在就业率和薪资成长,所以薪资成长率已经稳定高于GDP成长率了。

再过十年,北上广的薪资水准会远高于臺北。\\

\textit{\hfill\noindent\small 2015/12/09 00:00 提问; 回答}

\noindent[31.]{\Hei 答}:购买力的差异是一个很重要的考虑。臺湾其实物价已经很低了,衹是大陆一般更低。

大陆的沿海一綫城市和内陆还有很大的差别,不能一概而论。我想对臺湾去的年轻人,以前者较为重要吧,那么物价不见得比臺湾低很多。当然臺湾本身也有城乡的差异。\\

\textit{\hfill\noindent\small 2015/12/10 00:00 提问; 回答}

\noindent[32.]{\Hei 答}:欧洲自二战后,决心放弃霸权模式,虽然偶然仍有重温帝国荣光的衝动,整体来説是不像美国那样堂而皇之地自认超越国际规则的。所以中国的崛起,可以依赖与欧洲的合作,来冲消美国的敌对行为,这也就是一带一路的根本构思。\\

\textit{\hfill\noindent\small 2015/12/11 00:00 提问; 回答}

\noindent[33.]{\Hei 答}:欧盟是德国説了算,中共专心和德国交涉就行了。\\

\textit{\hfill\noindent\small 2015/12/26 00:00 提问; 回答}

\noindent[34.]{\Hei 答}:我没有看到直接的关系,有可能是巧合。

Christine Lagarde靠拍Sarkozy马屁而拿到IMF的职位,她原本是Sarkozy的财政部长,当时就有这个案子在醖酿。我个人觉得案子本身的确是无中生有的政治报復,但是是法国内部狗斗的结果。

Lagarde的IMF政策衹是执行欧洲的共识,美国应该不会迁怒到她个人。我其实更担心Osborne;他还要五年才能继任首相,途中很可能被美国人暗算。\\

\textit{\hfill\noindent\small 2015/12/29 00:00 提问; 回答}

\noindent[35.]{\Hei 答}:正文里的话我若是绕着圈子说,衹怕大多数读者自己想不到全部的细节。

反正这里已经有足够的知名度,该来的读者最终还是会来的。\\

\textit{\hfill\noindent\small 2015/12/29 00:00 提问; 回答}

\noindent[36.]{\Hei 答}:这是推脱之词,因为我不是在辩论,而是做独白,当然是自説自话。他实际的意思是,这些实话我敢说,他们不敢印。

我早已注意到《中时》对我的部落格的自由放任,其实是很值得珍惜的。读者在被《中时》的软体缺陷困扰之际,请提醒自己那些是次要的考虑。\\

\textit{\hfill\noindent\small 2015/12/29 00:00 提问; 回答}

\noindent[37.]{\Hei 答}:美国人使坏,都是由宣传喉舌开第一炮,当初要侵略伊拉克,就是很好的例子。

这篇《金融时报》上的文章显然是要对Merkel施压之前,先声夺人、奠定舆论基础的手段。中方可能最后还是得拿出对大众汽车的杀手锏来。不过美国拖了这么久还不肯确定大众的罚款,可能也是想把两件事扯在一起。\\

\textit{\hfill\noindent\small 2016/01/28 00:00 提问; 回答}

\noindent[38.]{\Hei 答}:但是中国的新额分来自欧洲,美国仍然有IMF的否决权。\\

\textit{\hfill\noindent\small 2016/05/14 00:00 提问; 回答}

\noindent[39.]{\Hei 答}:这事说来话长。我过几天再写一篇文章吧。\\

\textit{\hfill\noindent\small 2016/09/18 00:00 提问; 回答}

\noindent[40.]{\Hei 答}:我倒觉得是预言Merkel在2016年会有地位不稳的问题,比较值得夸耀些。\\

\textit{\hfill\noindent\small 2016/12/28 00:00 提问; 回答}

\noindent[41.]{\Hei 答}:最后中方决定WHO的仲裁庭见胜负,反正胜算极大,就是要拖一段时间。\\

\textit{\hfill\noindent\small 2017/04/18 00:00 提问; 回答}

\noindent[42.]{\Hei 答}:正如我两个月前提过的,Trump的存在反而促进了欧盟的团结。

德法应该很快会开始对中亲善。\\

\textit{\hfill\noindent\small 2017/05/31 00:00 提问; 回答}

\noindent[43.]{\Hei 答}:不是,他们仍然在遵循对世界整体不升不降的原则。人民币对美元升值,是因为其他货币先升了。

当然,战术性地升多升快点,可以挤压空头。这点我已经提过多次。\\

\textit{\hfill\noindent\small 2017/06/01 00:00 提问; 回答}

\noindent[44.]{\Hei 答}:Merkel的声明很模糊,一副心不甘情不愿的样子;我觉得会如何发展还很难説。\\

\textit{\hfill\noindent\small 2020/06/21 15:39 提问;2020/06/23 03:17 回答}

\noindent[45.]{\Hei 答}:這件事原本就希望渺茫:不是中方理虧,而是因爲這與歐盟自身利益相關,即使不在乎美國的態度,歐洲人自己也想要作弊。

目前只能打落牙齒和血吞,徐圖另外與歐盟談自貿協定。當然歐盟會因此而姿態更高,那除了另起爐竈之外,也別無他法可想。然而中方連瑞典在新冠防疫中的惡劣行爲都不願拿出來做文章以壓制敵對方的氣勢,建立WTO的替代更難上千倍,所以非常不可能下定決心動手。
\\


\section{【医疗】现代医疗的大倒退}
\subsection{2015-11-19 23:02}


\section{26条问答}

\textit{\hfill\noindent\small 2015/11/20 00:00 提问; 回答}

\noindent[1.]{\Hei 答}:谢谢你的补充。\\

\textit{\hfill\noindent\small 2015/11/20 00:00 提问; 回答}

\noindent[2.]{\Hei 答}:私利诱惑之下,不可能人人都有公德;唯有强力的政府严格执法才能保护民众最大的利益。\\

\textit{\hfill\noindent\small 2015/11/21 00:00 提问; 回答}

\noindent[3.]{\Hei 答}:我所不懂的是为什么医闹会如此猖狂。若是在美国有到医院闹事的,警卫马上叫你出去,如果不听,就开枪了。\\

\textit{\hfill\noindent\small 2015/11/21 00:00 提问; 回答}

\noindent[4.]{\Hei 答}:他们掌控着电影、电视、新闻、学术(包括超弦早期的大佬)、律师、金融、钻石等工业。\\

\textit{\hfill\noindent\small 2015/11/21 00:00 提问; 回答}

\noindent[5.]{\Hei 答}:文化上是农业社会心态必须进化到工业社会,但是法治力量的欠缺是政治问题。\\

\textit{\hfill\noindent\small 2015/11/21 00:00 提问; 回答}

\noindent[6.]{\Hei 答}:慢慢来。我还有其他的话题想谈。\\

\textit{\hfill\noindent\small 2015/11/21 00:00 提问; 回答}

\noindent[7.]{\Hei 答}:就算不用枪,多叫几个警察来一样可以执行公权力。这正是我在正文里说的,执行团队没有纪律,所以就没有胆气执行法律,结果是刁民占便宜,老实人吃亏,全社会买单。\\

\textit{\hfill\noindent\small 2015/11/21 00:00 提问; 回答}

\noindent[8.]{\Hei 答}:或许中共正该等国民政府破產。\\

\textit{\hfill\noindent\small 2015/11/22 00:00 提问; 回答}

\noindent[9.]{\Hei 答}:I lost out on my main business 12 years ago exactly because Goldman Sachs got SEC to ban its competitors, so I know very well what you mean.

US is not the country it was, and certainly not the country it advertises to be.\\

\textit{\hfill\noindent\small 2015/11/24 00:00 提问; 回答}

\noindent[10.]{\Hei 答}:美国传统上医院虽然是私立的,但却算是公益慈善机构,受到大学或基金会的监督。近年来自由化思想也產生了所谓的盈利医院,问题一样多多。

我已经解释过市场经济不能为人命定价,所以还是以公营为宜。\\

\textit{\hfill\noindent\small 2015/11/24 00:00 提问; 回答}

\noindent[11.]{\Hei 答}:其实连一部分美国经济学家(不完全愚蠢或腐败的那一小部分)都承认,自由市场在诉讼和医疗两个方面走不通,因为公平和生命是无法由市场(亦即参与竞争的第三者)来定价的。

中国会在这两个方面也搞私有化,显然是中了美国宣传的毒。\\

\textit{\hfill\noindent\small 2015/11/25 00:00 提问; 回答}

\noindent[12.]{\Hei 答}:It is nothing more or less than pumping tax-payers' money into the pockets of drug companies, pure and simple.\\

\textit{\hfill\noindent\small 2017/08/14 00:00 提问; 回答}

\noindent[13.]{\Hei 答}:你是在城市里(臺北?)长大的吧?

在乡下,每个村都有几个望族,通常是大地主。要知道一个村里的望族是谁,一个办法是问问庙会的钱哪里来,另一办法是看谁是选举的\&ldquo;桩脚\&rdquo;,答案都是同一群人。

我不是说他们个人天生邪恶,而是他们的社会地位,先天就使他们与国家整体利益对立。

因为他们垄断了乡村的经济财富、政治权力和民意宣传,中国歷史上的中央政府,包括国民政府在内,都必须笼络他们。在隋朝以前,是乡举里选、九品中正等等;之后则是科举。清末废了科举,那么就只能搞地方自治。名义上是自治,其实是政治分赃,但是还不如18世纪法国大革命之前直接把收税的职能拍卖给出价最高者那样公开、诚实、透明。

这些世族的权力越大,贪腐的现象就越严重,全国的政治和经济规则也越向他们倾斜,损失的则是中央政府、城市里的中產阶级和乡村里的农民。后汉是直接因世族尾大不掉而灭亡的,其他的王朝也几乎都在中期之后因世族反对而无法改革,因而步向衰亡的命运,例如明朝的东林党,就是代表世族的既得利益集团。

未来的歷史学家,必然会把中华民国也加到这个名单上。

还有,中共的土改会搞到那么血腥,其实不是因为他们有远见,知道要为现代化的工业社会奠定基础。共產党没收土地,固然是他们的理念,但是把已经被剥夺财產的地主们斗争至死则是刘少奇在内战期间推动的。1947年三月,延安被国军占领后,他跑到河北,发现发动贫农开斗争会斗死地主之后,这些贫农都知道手上染了鲜血,如果国民政府获胜,自己会有大麻烦,因此自然成为坚定的解放战士。刘少奇在1947年一份给党中央的报告里面,解释了这个现象,宣称效果惊人。从此这被推广到全国。

一般人以为1947-1948年起,内战的天平开始向共產党倾斜,是因为苏联向林彪交付了大批军火。其实那只解决了枪的问题,人的动员是靠血腥土改解决的,而且这些新的兵员有狂热,战斗力很强。我个人认为影响比武器更大一些。\\

\textit{\hfill\noindent\small 2017/08/15 00:00 提问; 回答}

\noindent[14.]{\Hei 答}:歷朝歷代都是一样的。分久必合之后,有了稳定的中央政府,权力和财富都必然会持续累积集中,產生新的既得利益阶级。土改只能在建立政权的过程中执行,其后就不可能了。

中共不是魔术师,只怕也不能逆转升平社会下财富集中的经济定律,但是或许能继续导引资本离开寻租,而创造真的新财富。

只有武统才可能有深刻的改革。所以武统对中国、对臺湾长期来看都是好事。\\

\textit{\hfill\noindent\small 2017/08/15 00:00 提问; 回答}

\noindent[15.]{\Hei 答}:所以习近平的改革,其实还有很长的路走。

希望下一任,不会又是自由放任式的领导。\\

\textit{\hfill\noindent\small 2017/08/15 00:00 提问; 回答}

\noindent[16.]{\Hei 答}:绝对不能搞任何形式的地方自治。

最好能有若干程度的土改。特别顽劣的,送到新疆兵团去落户。\\

\textit{\hfill\noindent\small 2017/08/16 00:00 提问; 回答}

\noindent[17.]{\Hei 答}:是的。

而且土地的价值,不在于现在或过去的经济环境,而在于未来的持续使用所带来的财富。这是因为土地永远不能增加,也永远不会消损。如果中国继续以5-6\%的速度增长GDP20年以上,现在的土地价格就很合理。\\

\textit{\hfill\noindent\small 2017/08/16 00:00 提问; 回答}

\noindent[18.]{\Hei 答}:中共的土地政策,目前似乎还是一个\&ldquo;拖\&rdquo;字诀。

我个人认为必须有税制改革。两年前在留言栏已经讨论过。\\

\textit{\hfill\noindent\small 2017/09/01 00:00 提问; 回答}

\noindent[19.]{\Hei 答}:危机已经明显化了,美国政府不得不出面纠正财团的偏差行为,但是杯水车薪,只怕是来不及阻止未来大范围的超级感染。

中国政府从人道主义、医疗需要和发展科技的立场,都应该更积极地参与才对。\\

\textit{\hfill\noindent\small 2017/09/25 00:00 提问; 回答}

\noindent[20.]{\Hei 答}:不止是被动,而且美国内部利益输送所造成的不合理结果,中国也只能照单全收,最终是财政和病人一起受害。

就像用人民币替代美元霸权,必须由上至下,靠国家政策建立亚投行这样的机构,中国也应该有官方主导,建立起独立于欧美的医疗标准。如此则不但可避免遭美国医药财团搜刮,而且可以培养自己的高级生化人才。\\

\textit{\hfill\noindent\small 2017/09/27 00:00 提问; 回答}

\noindent[21.]{\Hei 答}:不会的。

公开排斥对方的行业标准,是美苏对立冷战的结果,背景条件是双方经济贸易没有交集。中国的经济是全球化的一部分,欧美的厂商要在中国卖器材、药品,就只好把中国标准当一回事。

这纯粹是做与不做的选择。\\

\textit{\hfill\noindent\small 2017/09/29 00:00 提问; 回答}

\noindent[22.]{\Hei 答}:美国有很多老人,虽然有联邦医疗保险,还是付不起20\%的自负费,于是移民到墨西哥或古巴去享受低价或完全免费的医疗服务。

美国是富人的天堂;一般老百姓的日子其实很没有保障,偏偏他们太蠢了,让财团掌控的媒体完全洗脑。\\

\textit{\hfill\noindent\small 2017/09/29 00:00 提问; 回答}

\noindent[23.]{\Hei 答}:我认识一个天津来的开餐馆的家庭,一个盲肠炎的小手术也拿到十二万的账单。

在美国,若是没有医疗保险,那是随时都可能破產的。所以Bernie Sanders的全民医疗保险,实在极有必要。\\

\textit{\hfill\noindent\small 2017/09/30 00:00 提问; 回答}

\noindent[24.]{\Hei 答}:医疗和法律一样,完全不适用于市场经济;如果像美国这样因为被财阀洗脑而硬上,结果就是人命价值与个人财富成正比。

美国不但没有全民保险,而且2017年的医疗花费已经高达GDP的18\%,是西欧国家(都有全民保险)的6-9\%的2-3倍。这主要来自三方面的浪费:1)迷信市场经济,硬是多出了私有保险公司这一层中间人的剥削,而且繁文缛节、效率低下;2)中上阶级的医疗保险,居然享受了税制上的大幅补贴,以致极尽豪华;3)美国的制药公司政治能量太大,追求的利润太高。Bernie Sanders的法案可以解决前两个问题,但是因为动了许多既得利益者的蛋糕,过关的可能性极低。

大陆的新健保,听来很好,但是必须注意长期的财政可持续性。臺湾的全民健保,算是过去30年少数几个方向正确的新政策,但是背后的财政管理乱七八糟、寅吃牟粮,只怕也是将来中共必须买单的项目之一。\\

\textit{\hfill\noindent\small 2017/09/30 00:00 提问; 回答}

\noindent[25.]{\Hei 答}:美国药品公司真正投入研发的资金,其实只是他们成本的一小部分,绝大部分给了经理、律师、会计师和股东去了。

中共对医药业的政策,的确是亟待改进,但是收购和模仿都不是最佳手段。必须以政、法、商、学组合成国家队,靠联合体系来与美国竞争。\\

\textit{\hfill\noindent\small 2020/04/06 22:07 提问;2020/04/07 05:15 回答}

\noindent[26.]{\Hei 答}:我對大陸的醫療體系一無所知,只能從大方向指出中醫教的嚴重長期危害。最終要如何解決,還是得靠國内有科學素養的知識精英詳細反思出好的方案,並且説服執政階層。
\\


\section{【美国】美国的开国神话}
\subsection{2015-11-29 08:14}


\section{7条问答}

\textit{\hfill\noindent\small 2015/12/01 00:00 提问; 回答}

\noindent[1.]{\Hei 答}:IMF是美国阵营的机构,自然不会容许苏联参与。世银和GATT也是如此。

黄奇帆在重庆做得真是不错,他被重用是件好事。\\

\textit{\hfill\noindent\small 2015/12/01 00:00 提问; 回答}

\noindent[2.]{\Hei 答}:前一篇文章才刚讨论过。

货币战略请参见《美元的金融霸权》。

我一向喜欢预言新闻而不是追述新闻,所以请先把旧文章仔细看完再发问。\\

\textit{\hfill\noindent\small 2015/12/04 00:00 提问; 回答}

\noindent[3.]{\Hei 答}:我也一直觉得美国在大学入学上,明目张胆地歧视亚洲人,中国实在可以拿来不断説事,任何中国人听了都能很快理解同情。\\

\textit{\hfill\noindent\small 2016/09/06 00:00 提问; 回答}

\noindent[4.]{\Hei 答}:真正有权的财阀,不是大家在新闻上常听到的名字。

我最近发现了一本好书,叫做\&ldquo;A Brief History of Neoliberalism\&rdquo;,完全印证了我以前自己体会到的1970年代开始的财阀反扑,而且提供了很多新的细节,你可以参考一下。\\

\textit{\hfill\noindent\small 2016/09/07 00:00 提问; 回答}

\noindent[5.]{\Hei 答}:因为美国对内对外的宣传洗脑,都是人类歷史上最先进、最彻底的。你看看那本书吧。\\

\textit{\hfill\noindent\small 2017/10/04 00:00 提问; 回答}

\noindent[6.]{\Hei 答}:嫌犯似乎还有几十万的资產(包括十万现金匯给女友,和一栋四十万左右的房子)。

我想最可能的是他生理、心理都有重病;偏偏在美国,要找几十个陪死的太容易了。\\

\textit{\hfill\noindent\small 2017/10/04 00:00 提问; 回答}

\noindent[7.]{\Hei 答}:\&rdquo;Massacre\&ldquo;还是有媒体用的,但是比较少,因为它有很强的贬义。现在嫌犯的动机不明,有些大众媒体怕不小心得罪顾客(例如如果这个人是因政党偏向而发疯,那么他偏爱的那个政党虽然不能为他辩护,但是同党党人就可能不愿意被骂得太狠),所以先用中性字眼。

欧美媒体素来是老板第一、读者第二;至于正义和真相,那是连用来擦屁股都不够格的。\\


\section{【空军】【海军】共军小道消息刷新(2015年第四季)}
\subsection{2015-12-01 06:15}


\section{5条问答}

\textit{\hfill\noindent\small 2015/12/07 00:00 提问; 回答}

\noindent[1.]{\Hei 答}:美国的社会福利比欧洲低很多,联邦每年的开支衹有2120亿美元,大约1.5\%的GDP。当然州一级还有花费,但是比不上联邦。

美国一般人不到不得已不会去领食物券;大部分的用户没有能力自力更生。虽然不理想,但是不发不行。

把扶贫说成维稳,本身就是个大扭曲。扶贫是政府的第一要务;把办正事的花费算成\&ldquo;极权\&rdquo;体制的浪费,听来的确是典型的美式抹黑。

习近平对扶贫似乎是很认真的。我们当然还要持续观察,但是他肯做这个努力,已经算是目前世界上独一无二的了。\\

\textit{\hfill\noindent\small 2015/12/07 00:00 提问; 回答}

\noindent[2.]{\Hei 答}:我个人曾经想对\&ldquo;维稳经费超过国防预算\&rdquo;这个説法挖根,但就是找不到确实的数据和根据。中国的国防预算衹有GDP的2\%,连美国的一半都不到,所以和\&ldquo;维稳经费\&rdquo;比起来当然吃亏些。不过我仍然不相信中国的政法系统预算超过美国的水准;美国光律师和法庭就消耗4\%的GDP,警察更是人工和装备都远贵于中国,而且叠床架屋,大机关(如学校)有自己的警察、镇有镇警、州有州警,我住的镇警察局就占镇预算的10\%左右,这还是高级住宅区,警察平常基本就是指挥交通;在低收入城区,警察的密度要高出好几倍。

我的猜测是中国的政法系统(包括武警)预算就算稍高于国防,比起美国来还算低的。

我不觉得维稳经费高是强力政府的结果;真正可以确定提高了维稳费用的因素是美国的宣传颠覆,全世界不肯当美国附庸的国家都必须付出这个代价。

\&ldquo;新孤立主义\&rdquo;被言过其实。美国衹是用兵花钱花得痛了;如果能够便宜地搞宣传颠覆,还是不会放过任何机会。\\

\textit{\hfill\noindent\small 2015/12/07 00:00 提问; 回答}

\noindent[3.]{\Hei 答}:我怀疑这个説法是故意对统计数据曲解而来的宣传伎俩,但是没有明证。

客观的背景似乎不支持那句话的宗旨(中国维稳费用远高于其他国家)。\\

\textit{\hfill\noindent\small 2015/12/09 00:00 提问; 回答}

\noindent[4.]{\Hei 答}:不衹是必须证实这句话的真实性,我想更重要的是国际性的比较,因为这句话的真正宗旨是\&ldquo;中国维稳费用特别高\&rdquo;。\\

\textit{\hfill\noindent\small 2017/08/14 00:00 提问; 回答}

\noindent[5.]{\Hei 答}:官员和稀泥,是牺牲长期国家社会的利益,会自己的仕途牟利。

救助跌倒,反被赖钱,长久下来,热心互助,被转化成冷漠自私。这是有严重社会成本的。\\


\section{【工业】浅谈光伏业}
\subsection{2015-12-06 05:22}


\section{9条问答}

\textit{\hfill\noindent\small 2015/12/07 00:00 提问; 回答}

\noindent[1.]{\Hei 答}:我在《2030年左右》已经解释过,即使在经济减速的背景下,未来15年的能源需求仍然是惊人的。

最近批准的高效煤电是为了替代旧有的煤电,不是用来提供增长。

大陆的风力水力资源一在西北一在西南,发电后的传输问题很大。核电和分布式太阳能可以就地满足需求。

至于外销,那是当然的。中国的工业化从来不是衹为满足内需;永远都会需要外来的原料和市场。\\

\textit{\hfill\noindent\small 2015/12/11 00:00 提问; 回答}

\noindent[2.]{\Hei 答}:铁路货运下降,完全受煤、铁等大宗散货影响,工业成品的运输反而有增加。

高速公路建得太好,会鼓励民众开长途车,这在美国是石油财阀有意造成的,在大陆我想还是以高铁为主,高速公路做辅助就好了。\\

\textit{\hfill\noindent\small 2015/12/11 00:00 提问; 回答}

\noindent[3.]{\Hei 答}:臺湾没有风电和太阳能的天然资源,不以科学态度去评估,硬是从政治出发,那么后果自然就衹能是浪费、贪瀆和產业出走。臺积电已经跨出一步了,再虚耗八年,臺湾还能剩下什么?\\

\textit{\hfill\noindent\small 2015/12/11 00:00 提问; 回答}

\noindent[4.]{\Hei 答}:\&ldquo;高速公路是开来给有钱人用的\&rdquo;是当年\&ldquo;党外\&rdquo;的诉求主题之一,不到几年这些鼓动民愤的无脑宣传不攻自破,那其实是年轻的我对绿营產生疑虑的滥觞。

中国如果在一年后拿到市场经济地位,席卷世界光伏市场指日可待。\\

\textit{\hfill\noindent\small 2015/12/11 00:00 提问; 回答}

\noindent[5.]{\Hei 答}:美国的经济学家,如果说了话事后有结算,那么你会发现他们个个都不如掷飞镖的猴子。

经济学本来就衹是美国财阀的遮羞布。\\

\textit{\hfill\noindent\small 2015/12/15 00:00 提问; 回答}

\noindent[6.]{\Hei 答}:中国的特高压电网是世界第一,但是目前仍然远远不足以将所有西北部的风电传送到东部的人口重心;西藏的水电就更别提了。

连接大陆的电网,在臺湾是政治上绝不可能的事。

留言请注意,以简洁为要,避免将琐碎的小事也打散多发。\\

\textit{\hfill\noindent\small 2015/12/15 00:00 提问; 回答}

\noindent[7.]{\Hei 答}:世界上决心和一带一路唱反调的,就是美、日、印三个大户和臺、越、菲三个小弟。中国或许不能和前三者翻脸;后三者却完全是不自量力。\\

\textit{\hfill\noindent\small 2015/12/18 00:00 提问; 回答}

\noindent[8.]{\Hei 答}:李远哲,陈健仁的专业知识也不在此,他们做这种鼓吹衹是政治挂帅罢了。\\

\textit{\hfill\noindent\small 2017/06/06 00:00 提问; 回答}

\noindent[9.]{\Hei 答}:我觉得没有什么深刻的考虑,纯粹就是中国越来越有自信的表徵。\\


\section{【宣布】前段时间很忙很烦,现在可以重新写作了}
\subsection{2016-01-06 13:53}


\section{1条问答}

\textit{\hfill\noindent\small 2016/06/14 00:00 提问; 回答}

\noindent[1.]{\Hei 答}:我想我的长处是对美式金融学术的不合理之处瞭解得多,例如美国标准的研究所金融教科书里,有一个章节是完全错的,我一直想要写一篇论文把它的谬误之处好好讨论一下。

我也不在乎赚钱存钱,衹要不是入不敷出就可以了。\\


\section{【经济】漫谈近来的经济态势}
\subsection{2016-06-09 00:12}


\section{45条问答}

\textit{\hfill\noindent\small 2016/06/09 00:00 提问; 回答}

\noindent[1.]{\Hei 答}:我想过去两年中美在金融货币上的斗争,称得上是无硝烟的战事;现在基本看谁的气长。中方的形势还是不错的。\\

\textit{\hfill\noindent\small 2016/06/09 00:00 提问; 回答}

\noindent[2.]{\Hei 答}:是的。\\

\textit{\hfill\noindent\small 2016/06/09 00:00 提问; 回答}

\noindent[3.]{\Hei 答}:应该还不至于吧。不过危险是现实存在,我就提了一提。\\

\textit{\hfill\noindent\small 2016/06/09 00:00 提问; 回答}

\noindent[4.]{\Hei 答}:我在臺湾出生长大,当然是咸党。奇怪的是,我前天到附近的越南铺子里买他们的粽子,居然是红豆花生的甜粽!

这几个月新吹出的泡沫正是房地產和期货。但愿这次他们不要失控。\\

\textit{\hfill\noindent\small 2016/06/09 00:00 提问; 回答}

\noindent[5.]{\Hei 答}:办出国是很不容易的。如果最终衹是在国内扩散,未尝不是件好事。\\

\textit{\hfill\noindent\small 2016/06/09 00:00 提问; 回答}

\noindent[6.]{\Hei 答}:我基本同意有关金融战争的描述,但是他凭一个数据就说美国已经要崩溃,衹怕言之过早。\\

\textit{\hfill\noindent\small 2016/06/09 00:00 提问; 回答}

\noindent[7.]{\Hei 答}:体制优势也在腐蚀之中;习近平想力挽狂澜,要做的还很多。

美国在南海闹,在臺海却静悄悄,就是距离的差别。在南沙真打起来,美军现在还是有胜算的;不过再过三四年,情势就要逆转了。\\

\textit{\hfill\noindent\small 2016/06/09 00:00 提问; 回答}

\noindent[8.]{\Hei 答}:It's open secrets to any real banker. There are hundreds of firms like that. They are part of the hidden mechanism which helps the rich become richer without doing anything.\\

\textit{\hfill\noindent\small 2016/06/09 00:00 提问; 回答}

\noindent[9.]{\Hei 答}:一二綫城市太贵了,就是三四綫的机会;衹是肯把握机会的市长好像不多。

澳洲和美国虽然不让买,欧洲的资產更好。我想这一波采购,最重要的还是Syngenta,再等几个月看看美国有没有能力拦。\\

\textit{\hfill\noindent\small 2016/06/10 00:00 提问; 回答}

\noindent[10.]{\Hei 答}:以购买力平衡的算法,在2014年就超过了。

不过大是大,產业层级还差个一两级,大概还要15-20年才能与美、德、日比肩。\\

\textit{\hfill\noindent\small 2016/06/10 00:00 提问; 回答}

\noindent[11.]{\Hei 答}:有趣的比喻。

欧美日的椅子更少,音乐也快玩不下去了。\\

\textit{\hfill\noindent\small 2016/06/11 00:00 提问; 回答}

\noindent[12.]{\Hei 答}:老人家是感情因素作用,这算是有无形的效益的,不一定要完全货币化计算。\\

\textit{\hfill\noindent\small 2016/06/12 00:00 提问; 回答}

\noindent[13.]{\Hei 答}:长期(三年以上)来看,你说的是对的,所以中国政府自己也在拼命囤积原油。

但是一般投机者衹看3-9个月,那么大宗商品就有泡沫的危险。而且大宗商品和工业有直接的关系,它的泡沫对实体工业的衝击更为强烈,所以迅速出手把它化解掉是件好事。\\

\textit{\hfill\noindent\small 2016/06/16 00:00 提问; 回答}

\noindent[14.]{\Hei 答}:负利率是弹尽援绝之后,拿起最后的几块石头来扔向敌人。安倍当然宣称他还在进攻,国际资本看到的,却是他将要被全歼了。那么日元匯率自然衹能往一个方向走。\\

\textit{\hfill\noindent\small 2016/06/16 00:00 提问; 回答}

\noindent[15.]{\Hei 答}:题目很大,我过一段时间再写专文吧。

低利率、零利率和负利率最主要的后果是资產泡沫,这是劫贫济富的机制,对社会有很坏的长期影响。\\

\textit{\hfill\noindent\small 2016/06/17 00:00 提问; 回答}

\noindent[16.]{\Hei 答}:我对那句话的理解是中国基础建设的投资累积还比不上先进国家,要继续花钱是有空间的。

至于\&ldquo;一带一路\&rdquo;,我比你乐观些。说不定连中俄自贸协议在五年内都能搞出来。\\

\textit{\hfill\noindent\small 2016/06/19 00:00 提问; 回答}

\noindent[17.]{\Hei 答}:其实日本已经在示范最终的结果,也就是拼命印钞票,把债贬值到无足轻重的地步。当然国民的国内储蓄也跟着化为轻烟,但是富人和企业还有海外资產,实际上是中產阶级买单。\\

\textit{\hfill\noindent\small 2016/06/19 00:00 提问; 回答}

\noindent[18.]{\Hei 答}:债是用美元算的,最后还是美国人説了算。

美国政坛有一句老话,选举的重点不在于票是怎么投的,而在于是怎么算的。贸易和国债也是一样的。\\

\textit{\hfill\noindent\small 2016/06/19 00:00 提问; 回答}

\noindent[19.]{\Hei 答}:我一直都关注着这事的发展,因为这是政府最重要的正事。很不幸的,全世界衹有习近平一个政权在认真、有效地干。\\

\textit{\hfill\noindent\small 2016/06/23 00:00 提问; 回答}

\noindent[20.]{\Hei 答}:可能是。

李克强还説要减轻建商的保证金。我开始觉得他又要做过火了。或许刘鹤就是因为早先已经听到消息,所以急忙预先警告。\\

\textit{\hfill\noindent\small 2016/06/24 00:00 提问; 回答}

\noindent[21.]{\Hei 答}:天下没有白吃的午餐,泡沫最后总是要破的。若是破得猛、破得快,更会火上加油。\\

\textit{\hfill\noindent\small 2016/06/24 00:00 提问; 回答}

\noindent[22.]{\Hei 答}:英国应该不敢弃中投美。两面讨好才是上策。\\

\textit{\hfill\noindent\small 2016/07/12 00:00 提问; 回答}

\noindent[23.]{\Hei 答}:这个问题我也想过。中国在经济学上,是有自己的套路的,所以才能不受西方歪曲理论的荼毒。去年底金刻羽拿着London的学位,戴着她爸爸的光环,回大陆用西方那套歪论给演讲,为自己造势,当场就被林毅夫逐条反驳,铩羽而归(pun intended)。

我觉得中国真正孱弱的思想理论和人才累积,在于金融、法律、新闻这三方面。例如去年的股市泡沫,对像我这样有金融经验的人,是非常低级的错误。法律上连契约都写不好,以致2001年签的WTO协定留下了自由经济地位这个问题,给欧美耍赖的余地。至于新闻和宣传,我在这个部落格已经多次批评,就不再赘述了。\\

\textit{\hfill\noindent\small 2016/07/23 00:00 提问; 回答}

\noindent[24.]{\Hei 答}:奇怪,很多地方已经把这篇文删了。

我一直很佩服黄奇帆,他不但执行力惊人,对经济发展的理论大纲和实践细节,都有极为深入和正确的见解,是经济管理的奇才。这篇演讲很值得大家读一读。\\

\textit{\hfill\noindent\small 2016/09/02 00:00 提问; 回答}

\noindent[25.]{\Hei 答}:那篇文章我读了。当年废除Glass-Steagall的时候,我大概就是整个巴黎国家银行交易厅里唯一一个没有欢呼的人。杨斌说Glass-Steagall的好,是正确的。

但是他以为Hillary会真心地改革,挡金融财阀的财路,可能是too young, too simple。\\

\textit{\hfill\noindent\small 2016/09/03 00:00 提问; 回答}

\noindent[26.]{\Hei 答}:一辈子孤单习惯了。\\

\textit{\hfill\noindent\small 2017/04/11 00:00 提问; 回答}

\noindent[27.]{\Hei 答}:金融这种专业性很强的行业,监管人特别必须才德兼备,的确是难找。

这次花了两年,抓了几个,主要也是靠一般的警察手段,没看到内行人吹哨,可惜。\\

\textit{\hfill\noindent\small 2017/04/16 00:00 提问; 回答}

\noindent[28.]{\Hei 答}:我上个月已经提过,世界经济有復苏的迹象。

改革是10年大计,短期的GDP震荡基本由大环境的外力决定。\\

\textit{\hfill\noindent\small 2017/08/08 00:00 提问; 回答}

\noindent[29.]{\Hei 答}:税制、财务和会计不是我的专业,但却是中共的本行,素来都会专家好好处理。\\

\textit{\hfill\noindent\small 2017/08/21 00:00 提问; 回答}

\noindent[30.]{\Hei 答}:那是Yeltsin上当了,让美国人和少数几个俄国人自由掠夺国家资產。也不是既有的富豪主导的。\\

\textit{\hfill\noindent\small 2017/08/21 00:00 提问; 回答}

\noindent[31.]{\Hei 答}:我对一季到一季的短期波动没有什么特别的见解,不过十九大之后,应该会有新一波振兴经济的政策。

搞垮俄国经济的短期导火綫是油价和制裁,两者都不是由富豪们主导的。
\\

\textit{\hfill\noindent\small 2017/08/22 00:00 提问; 回答}

\noindent[32.]{\Hei 答}:我不讨厌一个人,我讨厌的是他的恶劣作为。

我不讨厌富豪,我讨厌的是贪得无厌,扭曲政治和社会规则来寻租。

瑞士信贷有私人银行的分部,那就是转为富豪们藏钱逃税的。我工作的部门是投资银行。

投资银行的业务之一就是为各国政府以及大企业做融资。\\

\textit{\hfill\noindent\small 2017/08/28 00:00 提问; 回答}

\noindent[33.]{\Hei 答}:主要是美国财阀对这事很敏感,欧洲又没有利益前景(欧元国债的利率已经够低了),所以还在等适当的时机。

下月德国选举结束,然后Trump可能会对中、德发动贸易战争,如果失控,那么这才会成为反击美元的手段。\\

\textit{\hfill\noindent\small 2017/09/03 00:00 提问; 回答}

\noindent[34.]{\Hei 答}:有可能,但是在细节上扭曲GDP的计算,太专门了(介于经济和会计之间),不是我能在此断言的。\\

\textit{\hfill\noindent\small 2017/09/03 00:00 提问; 回答}

\noindent[35.]{\Hei 答}:臺湾企业,跨地经营的很多,更增加计算的困难和扭曲的容易。

统一之后一两年,就会真相大白。\\

\textit{\hfill\noindent\small 2017/09/13 00:00 提问; 回答}

\noindent[36.]{\Hei 答}:好吧,双方都提出论述了,但是我没有足够的专业知识来做裁断。

讨论到此为止。\\

\textit{\hfill\noindent\small 2017/09/14 00:00 提问; 回答}

\noindent[37.]{\Hei 答}:臺湾属于财不露白的低调文化,而且表面客气和善,实际上很难合作。不过最重要的原因,是土改只做了半吊子,土豪事后跑到城里买房改炒房地產去了。国民党的地方自治不愿意也不能得罪这些人,所以需要徵地的事能不做就不做,结果像是臺南,基本就是一个小镇放大20倍。\\

\textit{\hfill\noindent\small 2017/09/14 00:00 提问; 回答}

\noindent[38.]{\Hei 答}:我再解释一次:我没有足够专业知识做决断的话题,不容许大家在此打无限论战。

这个话题到此为止。\\

\textit{\hfill\noindent\small 2017/09/14 00:00 提问; 回答}

\noindent[39.]{\Hei 答}:这个道理,我以前也讲过。其实在美国的自由市场经济里,金钱不止代表着自由,它甚至是人身安全和基本人权(例如医疗保险、像样的基础教育和不会随便开枪的警察)的必要条件。\\

\textit{\hfill\noindent\small 2017/10/15 00:00 提问; 回答}

\noindent[40.]{\Hei 答}:你确定他用的数据是对的吗?在美国,对应M1的V是6左右,对应M2的V都有2以上。臺湾的V如果真的掉到1以下,那是非常不正常的。

我想他用GDP来简单估计P*T(也就是他説的P*Q)是不对的。例如组装一个手机,必须先买零件,零件商又有自己的上游。这些层层的交易,在P*T是简单的叠加,但是GDP必须消除重复计算,只考虑增加的附加价值。

臺湾的经济有虚胖的现象,但那主要是受目前全世界热钱充斥的影响,应该不是统计局有意作假。

\&ldquo;前瞻计划\&rdquo;的主要目的是为蔡英文和她背后的土豪们捞钱。当然,附带的作用是短期继续吹大泡沫,长期会使崩盘更加严重。

臺湾若要避免经济自杀,除了统一让中共来管理经营之外,就只能希望多数选民在很短时间内就长出他们一辈子都没有发展完全的大脑。这当然是不切实际的。\\

\textit{\hfill\noindent\small 2017/10/16 00:00 提问; 回答}

\noindent[41.]{\Hei 答}:全世界的普选政府都有为GDP注水的传统,但是臺湾具体做了多少、怎么做的,我不清楚。\\

\textit{\hfill\noindent\small 2017/10/16 00:00 提问; 回答}

\noindent[42.]{\Hei 答}:惊人的结论必须有惊人的证据。这位作者什么证据也没拿出来,连一个为什么和以往的结论不同的简单论述都没有,可以直接忽略。\\

\textit{\hfill\noindent\small 2021/04/29 02:00 提问;2021/04/29 02:28 回答}

\noindent[43.]{\Hei 答}:美元霸權對中國内部的經濟管理來説,只是有間接影響的大環境因素之一,並沒有邏輯因果上的必然。換句話説,當初放飛房地產市場,依舊是飲鴆止渴的短視行爲;未來處理這個問題,固然因爲美元霸權而更加困難,但即使人民幣成爲國際儲備貨幣,也不會因而迎刃而解。
\\

\textit{\hfill\noindent\small 2022/01/07 20:44 提问;2022/01/09 06:28 回答}

\noindent[44.]{\Hei 答}:中國的財富纍積還不夠,我認爲應該重點投入教育和醫療,而不是一般福利。
\\

\textit{\hfill\noindent\small 2022/03/16 18:34 提问;2022/03/16 23:43 回答}

\noindent[45.]{\Hei 答}:我已經反復説過了,扶貧的用意是對的,但直接發福利卻帶有很大的副作用,對經濟結構、社會風氣和生產效率都有强大的負面影響,而且一旦成爲習以爲常,基本無法扭轉;正確的投資方向是基礎教育和社區醫療,不但同樣達到扶貧的目的,而且對國家社會有長期、全面的改善作用
\\


\section{【国际】简评英国脱欧}
\subsection{2016-06-26 08:32}


\section{8条问答}

\textit{\hfill\noindent\small 2016/06/26 00:00 提问; 回答}

\noindent[1.]{\Hei 答}:2008年的欧元风暴是德国勉强将弱国纳入欧元区的后果,对欧洲统一的大局是一个打击。但是英国原本就不用欧元,如果没有移民和难民问题,不会闹到脱欧。\\

\textit{\hfill\noindent\small 2016/06/26 00:00 提问; 回答}

\noindent[2.]{\Hei 答}:德国人不擅长金融界所需的奸狡机诈;德银向来烂账很多,主要是被美国籍雇员为私利捅出的娄子。\\

\textit{\hfill\noindent\small 2016/06/27 00:00 提问; 回答}

\noindent[3.]{\Hei 答}:是,Bills+Notes+Bonds。

全世界流通的美元现金衹有一万两千亿,这是所谓的M0。大部分的美金储备是在各形各色的金融资產里,比现金大概高将近两个数量级。

有可能。最起码在美元霸权萎缩的过程中,美国会有所损失,但是不及当年印钞票换来的实际资產多。\\

\textit{\hfill\noindent\small 2016/06/27 00:00 提问; 回答}

\noindent[4.]{\Hei 答}:欧盟里面20多国,有20多种官方语言,经济文化更是参次不齐。英国用的英文是欧盟的第一语言,怎么称得上\&ldquo;书不同文\&rdquo;?此外英国的经济层次、规模和德法核心其实还算是很近似的。

法国的衰退比欧洲整体更为严重,现在欧盟基本就是德国的傀儡。如果有雄才大略的总理倒也罢了,Merkel依民调治国,讨好的可衹是德国的选民,其他国家能无限期地忍气吞声吗?我并不看好。\\

\textit{\hfill\noindent\small 2016/06/30 00:00 提问; 回答}

\noindent[5.]{\Hei 答}:自由移民在西欧诸国之间是可行的,问题在于东欧和境外来的经济移民衹有德国才想要。

英国若是从Commonwealth里挑经济发达的地区(如加拿大、澳洲和纽西兰)来组织自由贸易和自由移民的联盟,也是可行的。説不定未来两三年就会发生。\\

\textit{\hfill\noindent\small 2016/07/05 00:00 提问; 回答}

\noindent[6.]{\Hei 答}:德国不可能让德银倒闭,大不了像2008年瑞士政府那样由国家注入资金。\\

\textit{\hfill\noindent\small 2016/07/10 00:00 提问; 回答}

\noindent[7.]{\Hei 答}:工业竞争力流失、政治纷争人谋不臧,与法国相似,但是后者的农业和观光业却比日本强。

我衹是看坏日本的长期远景;崩溃这种Poisson Event是很有可能的,但应该不会在2020年之前,确实的日期也很难预言。\\

\textit{\hfill\noindent\small 2016/07/16 00:00 提问; 回答}

\noindent[8.]{\Hei 答}:这个作者似乎是引用两个月前《观察者》网上一篇文章的论点,当时也有人问我,我就说单凭一个月数据做论断言之过早。后来六月的数据出来了,果然新就业远高于预期。这其实是统计上常有的现象,有些案例没有算进某个月,就会在下个月弥补出来。

美国长远来看,国力的确是在被淘空,但是以他累积之多之深,要再拖上几年才有危机并非难事。所以我说2025年之前,中国还是面临着一连串的挑战。\\


\section{【战略】对美军在韩国部署THAAD之我见}
\subsection{2016-07-11 06:41}


\section{1条问答}

\textit{\hfill\noindent\small 2016/07/20 00:00 提问; 回答}

\noindent[1.]{\Hei 答}:在世界各国都是如此。中共中央当然不是万能的,但是至少比欧美政府要强势一些。\\


\section{【美国】【战略】Trump的施政方针}
\subsection{2017-03-29 20:57}


\section{3条问答}

\textit{\hfill\noindent\small 2017/04/06 00:00 提问; 回答}

\noindent[1.]{\Hei 答}:臺湾太小了,美国只是定期刮羊毛,不会存心打爆。

臺币原本是被央行压得极低,以维持贸易顺差,弥补行政体系对经济的损伤。最近的回升幅度不大,主要是央行主动Reload重新装弹,以备下一次刺激出口。与此同时,利率稍微降低,以缓衝对出口业的压力。

同样的药,用的适量可以治病,用的过多就是剧毒。1986年日币对美元上升了超过100\%,臺币现在的小波动不算什么。\\

\textit{\hfill\noindent\small 2017/08/07 00:00 提问; 回答}

\noindent[2.]{\Hei 答}:中国的每一个地方行政单位,其实都等同一个大企业集团,必须有能力很强的CEO来领导,黄奇帆就是其中的佼佼者。

中共经济在\&rdquo;去產能\&ldquo;方面已经做的差不多了,未来三年会集中精力\&rdquo;去杠杆\&ldquo;。杠杆就是举债,这在民选制度下是糖衣毒药,政客和选民都乐此不疲。但是它是以牺牲经济系统的储备和缓衝为代价的一次性刺激,美国有美元为后盾,还可以近乎无限地拖下去,像是臺湾这种地方,就纯粹是掠夺未来的财富了。

放眼世界,还有哪个国家敢说我们要去杠杆,而真有魄力去做的?\\

\textit{\hfill\noindent\small 2017/10/07 00:00 提问; 回答}

\noindent[3.]{\Hei 答}:Trump的\&ldquo;爱国主义\&rdquo;完全是形式和口号,他一辈子何时有为国服务来着?更别提牺牲了。

没有像廉价劳工、优质基建、高效组织这样的基础,贸易壁垒就只是哗众取宠、毫无实效的花样,徒然削弱自已仍有的实业企业。

美国的无证移民,工资比一般蓝领阶级低十倍,全都禁了,也没有美国公民愿意取代他们。

用补贴买来的新工厂,也只会是没有技术含量的组装厰,效费比比直接从国库付钱给那些工人还低两三倍。

我说的工业化,都是从教育、基建和效率着手,集中力量做產业升级,何时有提过像Trump这样锯箭疗伤的蠢事?你是在侮辱毁谤我吗?\\


\section{【基礎科研】大亞灣實驗的新結果}
\subsection{2017-05-07 02:34}


\section{4条问答}

\textit{\hfill\noindent\small 2017/05/07 00:00 提问; 回答}

\noindent[1.]{\Hei 答}:美国人可以流行读医、法、商,是因为有留学生补充理工科的人力。中国的确没有这个余裕,不能让顶尖人才浪费到金融那样对整体经济贡献是负值的领域去。

解决的办法,一方面必须让金融界不能赚大钱,另一方面必须像最近的\&ldquo;大国工匠\&rdquo;系列,奖励鼓吹有实用性质的技术研究和应用。我想习近平已经在做了,只是移风易俗,难度很大。\\

\textit{\hfill\noindent\small 2017/05/09 00:00 提问; 回答}

\noindent[2.]{\Hei 答}:让我解释清楚一下:金融业对经济的固定贡献(Fixed Contribution,就是只要有金融业就有的贡献)是很大的正值,但是它的边缘贡献(Marginal Contribution)是负值。所以最理想的金融业是吃不饱饿不死的那一种。\\

\textit{\hfill\noindent\small 2017/05/09 00:00 提问; 回答}

\noindent[3.]{\Hei 答}:I think the real difficulty in reforming finance is in the fact that there is simply too much money flowing too deeply under the cover of professional jargon. Those with expertise face tremendous temptation; those with discipline find it hard to acquire the expertise.\\

\textit{\hfill\noindent\small 2017/05/13 00:00 提问; 回答}

\noindent[4.]{\Hei 答}:我常説美式经济学的基本假设与现实无关,完全只是基于幻想以便为富豪的巧取豪夺来背书。其实这也不是巧合:美式淡水经济学的圣殿,芝加哥大学,当年就是美国歷史上的头号土豪J.D.Rockefeller特意为此而资助设立的,他后来还自夸说那是他最好的投资,没有之一。

我一直觉得中共对任何芝加哥大学经济系出身的人,都应该禁止入境。毕竟光算2008年一年,他们就摧毁了超过10,000,000,000,000美元的人类财富,破坏力之大,仅次于世界大战,远超人类歷史上恐怖分子的所有破坏的总和。这还只算了对财富总量的损害,他们真正的目标是搞贫富不均;我以前说过人类21世纪的最大挑战是贫富不均,就是拜他们之赐。

至于有一个章节在逻辑上是错误的,那指的是美国常用的金融教科书。这算是一个技术上的细节,我不想在此讨论。

这个留言,摆错地方了。\\


\section{【宣布】刚做手术,写作暂缓}
\subsection{2017-06-14 19:37}


\section{7条问答}

\textit{\hfill\noindent\small 2017/07/08 00:00 提问; 回答}

\noindent[1.]{\Hei 答}:从100多年前美国的崛起开始,一系列歷史发展暗示了统一的洲级国家,在现代的经济和技术环境下,有很大的优势。

至于当年欧洲的分裂,是否有利于近代工业革命的发生,并没有足够的证据来做论断。\\

\textit{\hfill\noindent\small 2017/07/20 00:00 提问; 回答}

\noindent[2.]{\Hei 答}:我注意到有几篇报导,似乎金融反腐仍在进行中。

我没有内綫消息,只能从原则上大概地猜一猜。证监会内部腐败,对金融系统的负面影响固然很大,但主要是长期慢性的容许内綫吸血致富;对2015年的股灾没有直接贡献。但是事后救市不利,就真的和监管人员阳奉阴违有很大的关系了。\\

\textit{\hfill\noindent\small 2017/07/27 00:00 提问; 回答}

\noindent[3.]{\Hei 答}:唉,其实金融管理的最大原则,是为实体经济服务,那么实际执行上的操作性目标,就是让人不能赚大钱。

歷史上的政权崩溃,往往就是由于金融危机,导致士绅的自我利益受损,才会有风起云涌的反对。例如辛亥革命,其实只是刚好遇上了上海金融泡沫爆破,再加上铁路私有化的资金被偷窃,不得不重新收归国营,而引起的全国性运动。\\

\textit{\hfill\noindent\small 2017/08/20 00:00 提问; 回答}

\noindent[4.]{\Hei 答}:很好的建议。

开交易所,钱还是由国际银行赚。直接做国与国的交易,可以使价格稳定,又可以架空美元,一举数得。\\

\textit{\hfill\noindent\small 2017/09/23 00:00 提问; 回答}

\noindent[5.]{\Hei 答}:资本是经济发展的必要因素,但是私有资本天然就会追求风险调适之后的最大回报(Maximum Risk-Adjusted Returns),而这类最大回报天然就会是寻租性的运作,例如官商勾结、房地產炒作或是资產垄断等等。所以绝对自由的市场经济必然是绝对邪恶而且低效的。\\

\textit{\hfill\noindent\small 2017/09/24 00:00 提问; 回答}

\noindent[6.]{\Hei 答}:严格来説,不一定是奴隶制,但必然是非常不平等的。

我以前已经説过,自由和平等两个普世价值,至少在经济方面有很大的矛盾。\\

\textit{\hfill\noindent\small 2017/09/25 00:00 提问; 回答}

\noindent[7.]{\Hei 答}:早年的美国,必须把印第安人不当人看(当害虫)看,才能算是绝对自由主义经济学的成功例子。同时期的英法在亚非殖民,如果把当地人不当人(当家畜)看,也可以算是绝对自由主义经济学的成功。所以事实上,即使在资源近乎无限的情况下,底层民众反而更惨,因为他们也会被当作\&ldquo;资源\&rdquo;的一部分。

我们的确已经对这个话题讨论过很多次。像山猫这样的新读者,应该先把既有的留言讨论都看过一遍,再考虑发问。\\


\section{【國際】印度二三事}
\subsection{2017-12-05 07:20}


\section{13条问答}

\textit{\hfill\noindent\small 2017/12/05 10:18 提问;2017/12/05 11:11 回答}

\noindent[1.]{\Hei 答}:
有可能。
但是金融危機的時間點根本上就不可能精確預測,否則就會被避免。
\\

\textit{\hfill\noindent\small 2017/12/05 16:11 提问;2017/12/05 21:47 回答}

\noindent[2.]{\Hei 答}:我同意。換了總統就必須加稅了。
\\

\textit{\hfill\noindent\small 2017/12/06 08:39 提问;2017/12/06 23:50 回答}

\noindent[3.]{\Hei 答}:
不是金融戰,只不過因爲美元還是全球儲備貨幣,所以美國財經政策再怎麽胡搞,仍然主要由其他國家買單。
Trump是在加速掏空美國。下任總統必然會反轉這些政策,但是不可能完全避免損失,例如美元的地位絕對會開始搖擺。
\\

\textit{\hfill\noindent\small 2017/12/09 01:59 提问;2017/12/09 08:17 回答}

\noindent[4.]{\Hei 答}:差別是,印度的這個辦法,不但合法,而且符合會計規律。
\\

\textit{\hfill\noindent\small 2017/12/09 05:57 提问;2017/12/09 08:15 回答}

\noindent[5.]{\Hei 答}:
這些CAR的法規,都假設銀行有呆賬就必須認賠。印度人對呆賬完全不管,繼續假裝它們是好賬,那麽自然只須要看名義上的資本有多少。
在實際執行上,因爲這些呆賬沒有被認出來,銀行可以繼續拿它們做抵押,向中央銀行借錢,再轉手借給自己的客戶。
\\

\textit{\hfill\noindent\small 2017/12/09 14:20 提问;2017/12/09 20:50 回答}

\noindent[6.]{\Hei 答}:沒關係,美國有很多投資銀行家、財務律師和會計師每天都在發明新的花樣,你只要出錢雇用他們就行了。
\\

\textit{\hfill\noindent\small 2017/12/10 06:12 提问;2017/12/10 07:19 回答}

\noindent[7.]{\Hei 答}:還是那句話:The market can stay irrational longer than you can stay solvent.
\\

\textit{\hfill\noindent\small 2017/12/10 08:49 提问;2017/12/10 14:50 回答}

\noindent[8.]{\Hei 答}:
印度現在好日子的來源,是國際游資的汎濫;英美媒體爲了仇中,替印度做了很多誇大的宣傳,所以歐美的資金不斷流入。
一旦國際銀根緊縮(這並不是資本短缺,而是經濟下行或風險升高,投資人不再追求報酬率,只想保值),印度的實際問題就會暴露出來。我想是Buffett説的:“Only when the tide goes out do you discover who's been swimming naked.”
\\

\textit{\hfill\noindent\small 2017/12/14 02:28 提问;2017/12/21 10:58 回答}

\noindent[9.]{\Hei 答}:指桑駡槐的確是我的壞習慣。。。
\\

\textit{\hfill\noindent\small 2017/12/14 09:09 提问;2017/12/21 10:57 回答}

\noindent[10.]{\Hei 答}:唉,現實裏面沒有天理,詐騙集團喫香喝辣,老實人喫虧受氣,最後倒的是行業或國家總體。高能物理是這樣,臺灣也是這樣。
\\

\textit{\hfill\noindent\small 2017/12/16 14:46 提问;2018/02/07 02:50 回答}

\noindent[11.]{\Hei 答}:是99\%,所以黑錢顯然都被洗白了。
\\

\textit{\hfill\noindent\small 2018/02/23 15:36 提问;2018/02/26 20:30 回答}

\noindent[12.]{\Hei 答}:我覺得Trump是真的想開打,但是中方已經有預案,不會由他爲所欲爲。
\\

\textit{\hfill\noindent\small 2019/10/24 00:44 提问;2019/10/24 00:57 回答}

\noindent[13.]{\Hei 答}:兩年前的問題,還只是一個定期炸彈;去年那個影子銀行倒閉,才真正確實揭開印度經濟衰退的序幕。
\\


\section{【國際】【戰略】回顧洞朗事件}
\subsection{2018-01-16 07:15}


\section{1条问答}

\textit{\hfill\noindent\small 2020/09/12 16:33 提问;2020/09/13 10:35 回答}

\noindent[1.]{\Hei 答}:這幾年下來,你的邏輯思辨反而有所倒退。有空應該自我反省一下,是否有效地運用日常閲讀和思索時間。尤其是博客這裏,你真的用心仔細讀懂了嗎?

討論中印關係,和“干涉別國内政”八竿子打不到一塊兒。這裏中美的真正差別,在於中方願意互利共贏,而美國一直是或暗搞、或明説的“America First”,所以對實力上升的大國必然出手打擊。我們面對的議題,不是中國也要學美國玩有我無人的霸權思維,而是對像印度這樣仇中態度深入骨髓的非理性國家,中國是否有必要主動犧牲國家利益、奉獻公共資產,損己利人。

你説亞投行大幅資助印度,是千金買馬骨;但馬骨只是欠缺正面價值,資助印度卻對應著負面價值,而且不只對中國是負面,對其他第三世界國家,亦即買馬骨的觀衆,也是負面的,因爲錢多給了印度,其他國家自然就少拿。那麽這個類比還恰當嗎?即使你硬拗它不是比喻不倫,那麽這個昂貴宣傳給予其他國家的教訓是什麽?中國是鄉愿,越是敵對、越有糖吃。這是適合“爭奪世界領導地位”的廣告嗎?“中印都開戰了,亞投行照樣給印度貸款”,那麽這些國家和中國親善有什麽意義?

亞投行必須獨立,原本就純粹是歐美防範中國增進國際影響力所下的絆子,沒有任何真正的邏輯性和合理性。資本主義一貫强調私有產權至高無上,怎麽中國人民的血汗錢就應該拿出來白送給仇敵?這和40多年前忽悠蘇聯自我閹割的那套宣傳如出一轍,結果Gorbachev把國家賣了,只拿到幾個獎牌;殷鑒不遠,居然還有中國人願意上當,真是愚不可及。
\\


\section{【戰略】【海軍】再談共軍的電磁炮}
\subsection{2018-02-11 07:30}


\section{1条问答}

\textit{\hfill\noindent\small 2018/02/12 10:13 提问;2018/02/12 10:29 回答}

\noindent[1.]{\Hei 答}:
私資公司都是短綫操作,頭幾年節衣縮食,想盡辦法搞好財務報表,一旦上市大賺一票,真正的毛病才紛紛出籠。這樣的態度,非常不適合大型火箭的長期發展。
SpaceX選擇專注的技術,都是短平快的,例如回收,如你所説,沒有真正的意義,所以別的國家從來沒有認真去搞,SpaceX為的就是可以拿來做“世界領先”的公關。現在Falcon Heavy的載荷能力也號稱是“世界領先”,但是我們還是等到它真的搭載了那樣的負荷,再相信吧。
最危險的是,低估出事的概率。這在金融界很容易,2008年的次貸危機,就是銀行選用特定時段(即前十幾年,沒有房貸問題的資料;1990年是上一次美國的房貸危機,絕對不能回顧那麽遠)的歷史資料來做統計分析,以得到低得離譜的風險估計,這樣一來,次貸的衍生品的價值就可以提高幾十倍。我怕SpaceX現在也是這樣搞的。
\\


\section{【戰略】談中共修憲}
\subsection{2018-02-26 22:48}


\section{6条问答}

\textit{\hfill\noindent\small 2018/02/27 00:11 提问;2018/02/27 01:48 回答}

\noindent[1.]{\Hei 答}:
臺灣房地產的頹勢已現,的確是會對金融和經濟有很大的拖累。那麽連帶著,引起金融危機(例如從保險業開始),確實是很有可能的。一旦發生,自然是統一的契機,但是仍然要看國際大環境和中方自己的利益得失而定。
金融危機這種事,是可以拖的,所以即使醖釀已久,也很難預測它何時發生,美國人也不例外。
如果臺灣拖到2025年之後才出大危機,那麽統一自然是天時地利人和兼具;如果在2020年左右就出事,反而是老百姓必須繼續多吃幾年苦。
\\

\textit{\hfill\noindent\small 2018/02/27 19:14 提问;2018/02/27 23:53 回答}

\noindent[2.]{\Hei 答}:
未來幾年,美國股市崩盤的機率很大,但是這不一定會造成嚴重的金融危機。
真正的金融危機,是指主要銀行和金融機構倒閉,並造成連鎖反應;如果只是投資人損失,還算不上。
\\

\textit{\hfill\noindent\small 2018/03/25 02:58 提问;2018/03/26 11:36 回答}

\noindent[3.]{\Hei 答}:
你沒有看懂我有關經濟學的文章:Thomas Piketty的研究已經證明貧富差距是自由市場經濟的必然結果,民主也是自由經濟的結果,不是原因,所以與貧富差距沒有直接的因果關係;北歐諸國是在時間和地域上有極大局限的局部例外。 
要和平地扭轉這個自然定律,唯一的希望在於不自私的精英(再提醒一下,不要把精英制度搞成一個大類),例如羅斯福和習近平。
\\

\textit{\hfill\noindent\small 2018/03/26 05:58 提问;2018/03/26 12:51 回答}

\noindent[4.]{\Hei 答}:
你既然熟悉Piketty的研究,那麽問題就出在你沒有仔細看我的評論,那裏的重點是民主制度和貧富均衡沒有直接的因果關係,所以即使你簡單得到的表面上(Nominal)Correlation是正值,仍然必須視因果樹的形狀來決定民主制是幫助還是阻礙貧富均衡。 
我已經在好幾篇文章裏解釋過了,實際上的因果樹是一戰和二戰創造了戰後貧富均衡的西方社會(這也是Piketty的結論之一),也同時創造了普選制的潮流。所以Nominal Correlation必然是正的。然而70多年來,普選制越走越極端,貧富差距卻越來越大。你的簡單Regression給出與事實相反的信號,就在於1)你只測量了Correlation;2)你只考慮不同國家這一個維度;3)你的主動變數只用了民主相對於所有其他制度,那麽很明顯地犯了我一再討論的籠統歸類的毛病。事實上,國家這個維度本身就是很糟糕的選擇,因爲大小、文化、歷史、資源和制度的差異極大。如果你考慮了至少時間這個維度,應該就會得到異號的正確答案。 
學術界只管發論文,所以可以做錯Regression,反正正負號搞顛倒了也沒人在乎。做金融的,絕對不能把Correlation當作Causality,否則就從賺錢變成賠錢。
\\

\textit{\hfill\noindent\small 2020/09/20 22:19 提问;2020/09/22 00:37 回答}

\noindent[5.]{\Hei 答}:這不是我已經反復談過,市場經濟的必然結果?在正常和平狀態下,投資報酬率必然高於經濟總成長率,所以富者越富。

英美建構的全球金融網絡,更方便資本隱匿逃竄,所以個別國家試圖平均貧富,必然要面對資金大幅外流的問題。今天才剛有報導,滙豐銀行多年來一直為詐騙犯洗錢;其實這是非常普遍的服務,只不過在美國一般是由大衆不知名的私人銀行來搞(除非監管機構是你家開的,例如“Government Sachs”),像是滙豐這樣的主流全科機構也來掙這點錢,實在是自找麻煩。然而歐系銀行做這種生意幾百年了,老習慣改不掉。
\\

\textit{\hfill\noindent\small 2022/10/23 23:58 提问;2022/10/24 08:20 回答}

\noindent[6.]{\Hei 答}:
我已經説過了,我對新出任高層職務者的思想、經驗、習慣毫無所知,因而對他們的政策偏好和眼界能力無從評論起;不過簡單邏輯指出你對這些人員的認識也必然沒有到足夠做出"他們不懂經濟"論斷的地步,所以可以警告你,不要搬運反中媒體的胡扯抹黑來污染博客。
此外,那些退休的“經濟專家”,大多是美式經濟學的信徒,在發展消費性產業的時代搞搞招商是可以的,現在美國要扼殺中國早已圖窮匕見,世界經貿版圖開始重整,繼續迷信自由市場和昂撒體系是自殺性行爲,這些人退下去,有何可惜或擔憂?
\\


\section{【美國】【經濟】再談Trump的政策}
\subsection{2018-04-08 05:32}


\section{1条问答}

\textit{\hfill\noindent\small 2018/04/15 03:39 提问;2018/04/16 02:43 回答}

\noindent[1.]{\Hei 答}:
地方過度舉債,是胡溫的遺產,這是衆所周知的事。
不過要根本解決地方藏債的問題,應該得靠紀委;美國的金融業只會鑽法規漏洞,如何對當前的局勢有助益,我看不出來。
\\


\section{【戰略】【經濟】再談中美貿易戰}
\subsection{2018-08-22 05:31}


\section{1条问答}

\textit{\hfill\noindent\small 2018/08/24 00:17 提问;2018/08/24 03:46 回答}

\noindent[1.]{\Hei 答}:
是的。在戰術上,應該以胡蘿蔔與大棒並用,最有成效。在戰略上,美國到2021年,必然會重返軟實力方案,其最大的受惠者就是財團;届時對他們再怎麽討好,也比不上中國被强迫“改革”所代表的紅利。
這兩點其實很簡單,但是中方的智庫至今還不懂,仍然死抱著舊式統戰的一些教條,想著要爭取次要敵人。其實再過兩年多,他們才是主要敵人的事實就會明顯化了!
\\


\section{【經濟】放任經濟學的邏輯謬誤}
\subsection{2018-09-02 02:18}


\section{11条问答}

\textit{\hfill\noindent\small 2018/09/02 14:38 提问;2018/09/02 15:16 回答}

\noindent[1.]{\Hei 答}:
你這是文科式的聯想,邏輯跨度非我所能及。
不過話說回來,你的故事至少抓到一個重點,就是個人逐利最大化和社會全局最優化,在經濟學上往往是有矛盾的。Braess' Paradox的基礎在此,放任經濟學的大毛病也在此,一般經濟學家談Game Theory時用的Prisoners Dilemma也基於此。不過Braess' Paradox比Prisoners Dilemma更進一步,不只是明示了前述的矛盾,而且重點在於新的流通路綫(在經濟學裏就是新的工業和商業手段)也可以壞事。
\\

\textit{\hfill\noindent\small 2018/09/03 18:37 提问;2018/09/04 01:53 回答}

\noindent[2.]{\Hei 答}:
你說的這點,的確是放任經濟學另一個有問題的隱性假設。我們以前討論過了,我不再多做評論。
不過你對Braess Paradox有所低估。它的邏輯核心,在於世界上的實際問題,幾乎都是非綫性的,常常會有在遠離全局最優解的地方,有多個局部最優點,而參與者自發的平衡和套利,必然會被局限在附近的局部最優解上。
即使在正文裏這個極度簡單的例子中,A與B之間建再多的公路也無濟於事。如果是要從Start到End直接開通12綫高速公路,這已經不是經濟學,而是幻想了,畢竟人類的科技和資源都是有限的,即使放任經濟學在微觀的問題上也承認這點。
\\

\textit{\hfill\noindent\small 2018/09/04 16:55 提问;2018/09/04 17:34 回答}

\noindent[3.]{\Hei 答}:
放任經濟學的確常常把資源無限做為前提,但都是隱性的假設,像是你所舉的例子。若是和他們坐下來談微觀問題,例如正文中的例子,他們不會提出要直接建新的捷徑,因爲那會是顯性的資源無限。
總而言之,放任經濟學的不合理假設很多;我在這裏提出的Braess' Paradox,是在那些假設(包括資源無限)之外的另一個新問題。
\\

\textit{\hfill\noindent\small 2018/09/04 23:09 提问;2018/09/05 06:19 回答}

\noindent[4.]{\Hei 答}:
在19世紀或許如此,不過當時的經濟學還很原始,也沒有多少影響力。
到了20世紀,放任經濟學的核心有兩個:最極端的是Rockefeller創辦的芝加哥大學,另一個比較溫和的是Vienna學派。後者主要是反對先是蘇聯、然後意大利和德國的國家社會主義。
\\

\textit{\hfill\noindent\small 2018/09/05 10:22 提问;2018/09/05 12:34 回答}

\noindent[5.]{\Hei 答}:
所以强力而理性的公權力,是經濟高速發展的不二法門。
臺灣的悲劇,就在於李登輝之後反其道而行。
\\

\textit{\hfill\noindent\small 2018/09/07 03:31 提问;2018/09/08 02:56 回答}

\noindent[6.]{\Hei 答}:放任經濟學的荒謬假設,我已經一再提起,像是你這樣的讀者也貢獻了很多内容。我想已經足夠了。
\\

\textit{\hfill\noindent\small 2018/09/07 19:40 提问;2018/09/08 03:04 回答}

\noindent[7.]{\Hei 答}:
你說的私營只為資本,公營才可能是為社會全局著想,正是本文的主旨之一。
不過公營必然是獨占性的,如果腐化了,無法自然淘汰,這也就是放任經濟學者的口實。中國的解決方案,當然是要靠賢能的中央官僚來做監督。如果那些放任經濟學者真的相信自由市場、放任競爭,就應該對中西兩個方案的競賽樂觀其成,而不是用“非市場”的手段(也就是他們的宣傳批評),來扼殺制度競賽的一方。
\\

\textit{\hfill\noindent\small 2018/09/30 01:03 提问;2018/10/01 00:06 回答}

\noindent[8.]{\Hei 答}:
沒有裁判的球賽,當然對惡霸最有利。
黑幫也是放任市場經濟下的自然產物,如果沒有政府的打壓,必然會凌駕其他勢力之上,包括財主在内;所以美國財閥資助的放任經濟學,並不容許它,反而鼓勵半獨立、能濫殺無辜的警察系統,其目的就是維持基本“秩序”,確保資本家的資產不受黑道威脅。
\\

\textit{\hfill\noindent\small 2019/01/31 18:50 提问;2019/01/31 21:16 回答}

\noindent[9.]{\Hei 答}:Hayek如同大多數西方經濟學者一樣,並沒有科學的態度,不是由事實與邏輯出發來推導結論,而是先預設結論再找例證,所以姑妄聽之就好了,不要太當真。
\\

\textit{\hfill\noindent\small 2019/02/10 14:21 提问;2019/02/11 04:29 回答}

\noindent[10.]{\Hei 答}:
國有化的前提是政治紀律,這才是中國與其他左派經濟的最大差別。
林毅夫在國有和私有經濟這個問題上,已經寫過許多文章了。請你自行搜索閲讀。
\\

\textit{\hfill\noindent\small 2019/11/10 11:24 提问;2019/11/10 11:58 回答}

\noindent[11.]{\Hei 答}:
Milton Friedman根本不是搞錯真相,而是從頭到尾都在故意撒謊,爲企業和財閥的獻金奮鬥,也就是我以前説過的”學術娼妓“。他所宣傳的歪理,還剛好可以遮掩解釋自己的醜行。在這個例子上,純粹是為富人牟利,可以上私立學校,還不用付教育稅。 
我敢這麽說,當然不是胡亂揣測,而是來自一個美國老教授Richard Wolff透露的秘辛。Wolff年輕時在Stanford念研究所,當時是1960年代,他和好朋友們都是加州的典型左派青年。有一天,其中一人在畢業前,忽然宣佈將要到芝加哥大學加入Friedman的團隊。Wolff私下問他怎麽回事,他說Friedman開出比別家學校高出好幾倍的薪水,條件是必須寫違心之論。這人後來成爲知名經濟學家,自由主義的幹將,最近退休了,但是Wolff說他私下的意見和公開的寫作是完全相反的。 
既然Friedman主動去收買聰明的學人來編造自己都不相信的謊話,那麽他本人當然大機率也是幹同樣的勾當。
\\


\section{【美國】美國政壇的系統性腐化}
\subsection{2018-09-26 16:17}


\section{2条问答}

\textit{\hfill\noindent\small 2018/10/03 23:59 提问;2018/10/04 00:48 回答}

\noindent[1.]{\Hei 答}:
你聽起來像是政策研究方面的專業人士;我很高興能得到這樣的反饋。
兩三年前,這個博客上有一系列關於房地產稅制的討論;不過最近我已經注意到中共在這個方向的準備動作,所以這次就沒有把它列爲一個問題。
教育改革,真正的關鍵不在細節,而在於它對階級固化的影響,這只有長期和間接的效果,中共傳統的地方試點不是合適的解決方案。這個議題的高度,到達了共產黨和社會主義存在的根本意義,同時又是歷史上大帝國維持社會、經濟和政治活力的樞紐,只有最高領導階層從原則上定調,才有做出正確選擇的可能。
至於在貿易戰中打擊美商的利益,這不能只是低調、獨立地做,必須是公開、系統性的對等反應;這是因爲中方的最終目的不在於“打贏”一場必然兩敗俱傷的戰爭,而在於以戰止戰,對歐盟和其他國家做出示範和嚇阻。
\\

\textit{\hfill\noindent\small 2019/12/01 02:10 提问;2019/12/03 08:36 回答}

\noindent[2.]{\Hei 答}:20幾歲的博士生,能寫出這樣的文章,真是相當不錯了。我自己在同樣年紀,都還沒有開始思考那些議題呢。
\\


\section{【金融】【國際】印度的影子銀行}
\subsection{2018-10-19 08:56}


\section{14条问答}

\textit{\hfill\noindent\small 2018/10/19 16:33 提问;2018/10/20 01:12 回答}

\noindent[1.]{\Hei 答}:
你是做國際關係的嗎?我注意到中國的國際關係專業,都有特別樂觀的趨勢。
比利時的那個管道,我也曾在公開媒體上看到過。美聯儲或許自己不會注意(美國經濟學者對統計數據有字面上的迷信),但是CIA必然是能追根究底的。
我對中國國内的政策不熟,只看到幾個整頓P2P的例子。如果真如你所説,是系統性的處理,那麽是件極大的好事。
收緊地產是有點晚了,但是當然比不做要好。
美國在金融和貿易上的好牌,說來説去,根本還是美元的國際儲備貨幣地位。但是Trump越是欺凌弱小,中俄要把美元拉下神壇就越容易。
歐盟和英國的問題都是慢性病,即使有了全球性的經濟衰退,要再拖下去並不難。Merkel和May固然有大幾率會在2019年底前下臺,但是她們原本就是和稀泥的專家,換個人還是一樣的。Kosovo這樣的小打小鬧很常見,要搞成真正戰爭的機會並不大。
\\

\textit{\hfill\noindent\small 2018/10/19 22:18 提问;2018/10/20 01:15 回答}

\noindent[2.]{\Hei 答}:是因爲這些資產會被賤賣,不但價錢只有原本的幾分之一,而且不會有政治上的反對,對方還得跪求你的資金。
\\

\textit{\hfill\noindent\small 2018/10/19 22:40 提问;2018/10/20 01:16 回答}

\noindent[3.]{\Hei 答}:是剛好相反:Trump在消耗美元的信譽和地位,換取短期的政治宣傳話題。
\\

\textit{\hfill\noindent\small 2018/10/20 00:00 提问;2018/10/20 01:44 回答}

\noindent[4.]{\Hei 答}:
美元既是美國力量的泉源,也是它最應該保護的資產。但是Trump的種種倒行逆施,基本上是消耗美元的國際信譽,來換取自己的政治利益。
然而更換國際儲備貨幣,如同要取代Windows一樣,是非常困難的。中國最好是有歐洲的合作,才會有些勝算。在美元被替代之前,美國處於不敗地位:不論出了什麽經濟或金融問題,沒有多印鈔票不能解決的。
東亞的國家,在1997年被剪過羊毛之後,普遍纍積巨大的美元外匯存底,這是最基本的自衛手段。永不學乖的,主要是東歐、西亞、和尤其是南美;其主要原因是它們的經濟理論和政策,都是由美國訓練的人來主導。
\\

\textit{\hfill\noindent\small 2018/10/20 10:59 提问;2018/10/20 12:55 回答}

\noindent[5.]{\Hei 答}:從國家層面,憑空印出來的白紙,可以買別人辛苦做出的產品。從企業層面,外包生產,可以大幅提高獲利率。從個人層面,階級固化,沒有提升的指望,只能今朝有酒今朝醉,盡情消費。我以前提過,美國的消費品便宜,而且娛樂業極其發達,都是這個背景下的必然結果。
\\

\textit{\hfill\noindent\small 2018/10/20 13:49 提问;2018/10/21 05:55 回答}

\noindent[6.]{\Hei 答}:
早就應該如此。
中國是英文裏的“Nanny State”,也就是“保姆國家”,國民像是幼兒一樣,政府對他們的一切事務和福祉都有責任。
老鼠會騙了錢,不只國庫要負擔善後,國家的信譽也會遭受打擊;而重建國家在國民心中的信譽,似乎是習近平的主要執政目標之一。
\\

\textit{\hfill\noindent\small 2018/10/20 14:37 提问;2018/10/21 05:49 回答}

\noindent[7.]{\Hei 答}:
是的,短期内要替代美元,歐元仍然遠遠是最佳候選。
中國的態度和策略應該是只要替代美元就是好事,鼓勵多元化,包括對歐元做出合理的支持。
\\

\textit{\hfill\noindent\small 2018/10/20 16:31 提问;2018/10/21 05:47 回答}

\noindent[8.]{\Hei 答}:
看起來邏輯自洽的理論,還是要和現實對照之後,才知道是否靠譜。
這裏最大的問題是,以中國經濟體量之大,這樣的效應是否達到能有主導意義的貢獻。我的猜測是,它遠遠不夠。
\\

\textit{\hfill\noindent\small 2018/10/21 19:03 提问;2018/10/22 03:15 回答}

\noindent[9.]{\Hei 答}:
一般老百姓對此是很遲鈍的,所以通貨膨脹有它獨立的“預期”心理,一旦建立、成爲社會生活的日常之後,就會與事實脫節,中央銀行必須極度緊縮銀根,才有可能扭轉這個預期。參見1980年代初聯儲會主席Volcker的經歷。
美國現在不須擔心通貨膨脹,是因爲開發中國家的體量上來了,石油美元由全世界吸收,一來一往,還能推動不勞而獲的利潤永動機。
\\

\textit{\hfill\noindent\small 2018/11/10 22:09 提问;2018/11/10 23:56 回答}

\noindent[10.]{\Hei 答}:
美國金融界十幾萬個極度高薪的博士/碩士員工,能搞出來的花樣千變萬化,就算我都知道,這裏也寫不下。
不過萬變不離其宗,對民族企業的搜刮,其原理基本上仍然是賤買貴賣,其中最大筆的,還是靠美元一發一收的循環。
至於美國人老是逼著其他國家開放金融,除了上述的這種周期大約10年的循環,平常也有很多較小、較快的炒作手段。它們的共同點是價位都必須大漲大跌,最終被壓榨的財富來自消費者、小投資人、企業員工和整體社會。這些手段,中國自己國内的炒家也會,只不過美國人的資金雄厚、經驗豐富、又可以全世界地來搞。
\\

\textit{\hfill\noindent\small 2020/05/10 11:19 提问;2020/05/11 07:56 回答}

\noindent[11.]{\Hei 答}:
首先,PMI是一個既非客觀也非綫性的指數,它是企業界對經營環境持樂觀態度的百分比,除了環境好壞之外,接受調查的企業願不願意說實話,其實是更重要的變數。這次新冠疫情明顯是無可究責的外加因素,所以印度商界可以很高興地承認生意受到負面影響;其他國家的企業或許仍然有必須說場面話的壓力。此外,5.4雖低,但是說跌幅多少並無實際意義,因為PMI與經濟環境沒有綫性的關係。 
當然,印度經濟的確面臨很大的問題:不但浮腫很像美國,而且金融界還纍積了大幅不良貸款,再加上他們高度依賴來自美國的資金和業務,現在後者已經面臨2-3個季度10\%的GDP衰減,印度像是狗尾巴,搖動的幅度會比狗屁股再增加幾倍。 
至於Modi爲了轉移民衆注意力而發動戰爭的機率,反而因爲新冠而極度減小,這是因爲上面已經提過的,疫情本身就是絕佳的藉口,雖然人謀不臧才是主因,但他已經可以簡單卸責,也就無須在國際上冒險。
\\

\textit{\hfill\noindent\small 2021/02/16 18:11 提问;2021/02/17 08:20 回答}

\noindent[12.]{\Hei 答}:只説“不久”毫無意義,就像預測退下去的潮水會再漲上來一樣,純屬廢話。美聯儲必須等美國經濟能撐得起加息才會動手,所以真正的議題是,什麽時候美國國内經濟,尤其是底層就業能復蘇。是今年下半,還是明年上半?或者會有Double Dip?這通通屬於“不久”,所以光看那報導的選詞用字,就知道内涵=0,不論是野鷄評論與否,都不該在此復述討論。
\\

\textit{\hfill\noindent\small 2021/02/17 10:19 提问;2021/02/22 09:50 回答}

\noindent[13.]{\Hei 答}:美國聯邦政府今年的三萬億赤字,若沒有美聯儲兜底,利息要冲上天了。會蠢到這個地步的,比一般的野鷄評論還低級,難不成是香港或台灣媒體?Yuck!

違反《讀者須知》規則,禁言一個月。


我原以爲美元是博客的重點話題之一,應該已經解釋得很清楚了,但一連串有人徹底誤解,所以這裏點出誤區何在,對宏觀經濟學不熟的讀者,請自行找閲讀資料補充基本知識。

美元在Nixon任期與黃金脫鈎之後,美聯儲得以針對國内的經濟熱度和就業情況來自由調整銀根,一開始是控制短期收放利率,到本世紀開始量化寬鬆。這個循環的回收階段,一般始於美國GDP加速成長,通貨膨脹壓力浮現,長期國債利率上升,然後美聯儲做出政策因應,美元匯率隨即升值,再然後金融抄手才有機會搜刮世界。但2008年的金融危機,不是普通的周期性經濟衰退,而是美國經濟長期不可逆衰頹的重要步驟,美聯儲一連放水6年,到2014年才敢收手,其後又努力了幾年,也只收回1/3,到了2019年又必須重回放水階段,而且變本加厲,一年就放出接近上次6年的總量。所以2015年中國外匯流失的壓力,其實並不大,後來才能在有内賊的情況下還簡單渡過。

這次的新冠危機,與Trump的倒行逆施互相叠加,應該是足以威脅美元霸權的歷史性關鍵事件。雖然雪崩的確實時間點無法預先確定,但美國金融界自己已經開始擔心,所以在經濟依舊走低的前提下,長期國債利率卻面臨上行壓力,然而美聯儲不可能不為今年的赤字買單,所以利率和匯率的聯係被打斷,不能套用1997年的脚本。
\\

\textit{\hfill\noindent\small 2021/02/26 22:44 提问;2021/02/28 03:09 回答}

\noindent[14.]{\Hei 答}:其實美聯儲自己早已把下一階段的決策標準說明白,而且還怕金融市場沒聽懂,公開高調地反復了幾十次:就是要等美國國内的通貨膨脹率漲到2\%以上,而且明顯地即將維持在2\%以上,才會開始緊縮銀根。

拿2021年的美國來和1990年的日本相比,有三個重大的差異:1)美國社會有極端的不平等,底層民衆佔多數,其消費基本是剛需;2)美國經濟以消費爲主,日本則偏重製造和外銷;3)美國有美元金融霸權。

所以我認爲美國的零利率政策不會陷入日本式的陷阱,一段時間之後,消費還是會復蘇,而且不見得要很久。這並不是說美國經濟不會像日本那樣摔下懸崖,只不過方式會有所不同,而且必須等美元的國際地位動搖之後才可能發生。這正是爲什麽我已經强調許多年,要對美國軍事外交霸權釜底抽薪、一勞永逸,必須認識到美元既是那個霸權的基礎,也是它的軟肋。
\\


\section{【政治】談損人不利己}
\subsection{2018-11-11 03:38}


\section{3条问答}

\textit{\hfill\noindent\small 2018/11/12 02:59 提问;2018/11/12 04:43 回答}

\noindent[1.]{\Hei 答}:
你説的沒錯,損人利己的事遠遠更多更普遍,畢竟每一個經濟上的交易,買方和賣方在價錢上都是零和游戲。但是損人利己還是理性的,所以至少理論上可以靠談判妥協來解決。而這種談判妥協,是所有正常政治決策過程必經的細節。你對周邊的不滿,反應的是美國政治僵化所帶來的無力感,它在邏輯上是政治僵化的後果,而政治僵化的原因卻是體制加上人類的非理性因素。 
非理性的損人不利己通常是無解的,就算等到老一批非理性的人死光了,也不見得下一代會有更多的理性。實際能改的,是體制,但是美國人把自己的體制神化了,所以這也成爲不可能。那麽結局就只有在緩慢或者迅速衰落中選擇一項。
\\

\textit{\hfill\noindent\small 2018/12/15 23:20 提问;2018/12/16 03:22 回答}

\noindent[2.]{\Hei 答}:
至少在歐、美、韓這些國家,4G早於中國。
5G原本就是個空殼技術,實際上的使用體驗比起4G來,遠遠沒有2G到3G或再到4G的進步那麽大。美國政府的反中戰略者不懂技術細節,全凴過去20年的經驗來推斷。不過電信業爲了廣告價值,還是必須花錢更新,而5G的專利稅比4G高很多,幾乎達到3G的水平,這對高通非常重要。
\\

\textit{\hfill\noindent\small 2018/12/16 10:48 提问;2018/12/16 12:07 回答}

\noindent[3.]{\Hei 答}:不是。我的意思是5G沒有太大的社會利益,主要就只是廠商的商業利益;那麽與其讓歐美企業賺,不如讓技術更好的華爲來賺這筆錢。
\\


\section{【工業】談未來的能源技術}
\subsection{2018-11-13 13:20}


\section{3条问答}

\textit{\hfill\noindent\small 2018/11/21 14:41 提问;2018/11/21 16:18 回答}

\noindent[1.]{\Hei 答}:
因爲私人資本在完全無風險的情況下,每年還指望5-8\%的報酬率。有風險的時候,自然要再加上溢價。儲能技術,從國家觀點來看,十幾年内做出來是十拿九穩,又有極大的好處,投資是應該的。私人資本,對十幾年後才可能有回報的計劃,卻根本就不可能有興趣,因爲兩三年就可以翻牌的投資項目太多了,每兩年賺30\%,17年是130\%的8.5次方=930\%,這種長期工業技術開發根本不可能有這麽高的回報(因爲大部分的利益,被整個國家社會分享了),而且8.5個不同的投資項目,風險自然抵消一部分(中心極限定理),多餘的風險溢價讓這個短期投資策略更加有利。
臺灣的問題,根本在於體制,偏偏大家還是每隔兩年就指望一個新的救世主。就算韓國瑜真是個聖人,又能怎麽樣?Renzi當到總理,還不是鎩羽而歸,何況區區一個市長?
\\

\textit{\hfill\noindent\small 2018/12/05 10:37 提问;2018/12/06 01:32 回答}

\noindent[2.]{\Hei 答}:
從實驗室到實用是最大的關頭,必須要是耐用、符合用戶規格、品質一致可靠、價格合理有競爭性。至於國家或私人,這是投資報酬率的問題,私人資本一般不在乎社會收益,也不願意做太長遠的投資。
慈善家投資到長期工業項目,在美國已經有前例,Gates就是TerraPower背後的主要金融推手。這個技術叫做Traveling Wave Reactor,是快滋生反應器的一種變型,中核已經同意在福建建造一個原型機。
在核聚變方面,也有不少小公司拿了富豪的半慈善資助,不過那些人沒有Gates的經驗,不知道它們有多不靠譜。
\\

\textit{\hfill\noindent\small 2019/01/16 20:11 提问;2019/01/17 18:41 回答}

\noindent[3.]{\Hei 答}:題目的確是太大了,你有空自己看看林毅夫的著作好了。我知道這些經濟學的文章比起我博客的科普要長得多、也難讀得多,但是我寫簡短科普的能力是有極限的,並不是所有複雜的議題我都能在一頁的空間裏討論完畢。
\\


\section{【語言】咬文嚼字}
\subsection{2018-12-04 00:04}


\section{1条问答}

\textit{\hfill\noindent\small 2018/12/04 09:55 提问;2018/12/04 10:21 回答}

\noindent[1.]{\Hei 答}:
我也看到了,覺得他們很可憐。
在美國,有一種隱藏式的高利貸,叫做Payday Checking Service,有效年利率可以高到1000\%以上。它專門針對的,就是有一點點收入的窮人,因爲他們沒有利率的概念,奸商可以輕鬆地剝削,所以窮者越窮。
這些人也是一樣的;正因爲他們在知識和教育上落後,所以反而更加不懂得如何選擇好的文章來閲讀,因而陷入一個惡性循環。
\\


\section{【美國】再談美國的腐化}
\subsection{2018-12-06 04:30}


\section{2条问答}

\textit{\hfill\noindent\small 2018/12/06 12:45 提问;2018/12/06 13:15 回答}

\noindent[1.]{\Hei 答}:Thomas Piketty在《21世紀資本論》裏已經論證過,除非有全國性、根本性的大災難,資本纍積的速度高於GDP的成長,所以資本只能隨時間而越發集中;換句話説,和平狀況下,自由市場經濟必然會加劇貧富不均。
\\

\textit{\hfill\noindent\small 2018/12/08 12:50 提问;2018/12/08 22:57 回答}

\noindent[2.]{\Hei 答}:
無限升級對雙方都沒有好處,對中國尤其如此,然而坐以待斃又會鼓勵Trump得寸進尺,所以處理中美衝突,必須依照低調、對等、及時等原則,等待美國的損失產生内發的阻力。
我以前已經解釋過,因爲美元是國際儲備貨幣,可以無限發行,中方抛售美債的效果有限。我擧這個例子,主要是爲了指出10年前那個決定所根據的邏輯有問題,已經明顯不再適應當前的現實。
中方能對美國產生最大傷害的痛點,還是美元的地位;剛好Trump對此毫無知覺,主動濫用SWIFT進行制裁。如果是我來做決定,會利用美國制裁伊朗的這個關頭,聯合歐洲設立一個取代SWIFT的新國際機構,並且順便改爲以歐元和人民幣爲主要貿易貨幣。
\\


\section{【美國】Trump的權力萎縮}
\subsection{2018-12-26 07:50}


\section{5条问答}

\textit{\hfill\noindent\small 2018/12/26 09:48 提问;2018/12/26 11:50 回答}

\noindent[1.]{\Hei 答}:
陰謀論很容易編造,但若是沒有證據,就只不過是一個猜想;如果連情理都說不通,那麽就純粹是胡扯,你所提的這個“大牛”就是一個例子。買空賣空搞的都是小股票,一般市值幾千萬美元;而美國的高科技股,光是Apple和Amazon就各是萬億美元級別的公司,說要賣空,那是完全不懂金融的人説的胡話。如果華爾街真要整Trump,花個幾億給他的對手做競選資金就可以保證他的落選,哪兒用得著擔上幾萬億的風險? 
美國的經濟成長週期即將告一段落,2019年再升息已經晚了。 Powell怎麽決定都無關緊要,即使他因爲慣性而在年初加息,反正到年底就反過來必須考慮降息,所以隨便路邊的擦鞋童亂猜也會有接近100\%的勝率,因爲不管往哪邊猜應該都能自稱是對的。 
至於世界其他區域的經濟,由於油價反轉,忽然下跌,所以像是印度這樣的大消費國自然受益。中美的互鬥也會讓許多國家漁翁得利,尤其是越南、印尼等等低端製造業的新出口國,可能會因之而完全避免衰退。所以受害者必然是有的,但是發展成97年或08年那樣的大災難,機率很小。
\\

\textit{\hfill\noindent\small 2018/12/26 10:23 提问;2018/12/26 11:36 回答}

\noindent[2.]{\Hei 答}:
長期來看,中國的氣當然更長,但是短期來説,中國的經濟損失(不是我以前討論過的利潤,而是就業和投資)大約是美國的兩倍,再加上中方的經濟衰退剛好比美國早兩季,所以在2019年前半的任何談判,都是美方的氣勢佔上風。 
不論如何,Trump已經接近狗急跳墻的階段,要預測他會怎麽胡搞,也就更加困難。
\\

\textit{\hfill\noindent\small 2018/12/26 10:23 提问; 回答}

\noindent[3.]{\Hei 答}:長期來看,中國的氣當然更長,但是短期來説,中國的經濟損失(不是我以前討論過的利潤,而是就業和投資)大約是美國的兩倍,再加上中方的經濟衰退剛好比美國早兩季,所以在2019年前半的任何談判,都是美方的氣勢佔上風。 不論如何,Trump已經接近狗急跳墻的階段,要預測他會怎麽胡搞,也就更加困難。\\

\textit{\hfill\noindent\small 2018/12/26 11:52 提问;2018/12/26 12:04 回答}

\noindent[4.]{\Hei 答}:中國官方的經濟指標,連總理自己都不相信,所以我倒不是特意要預測GDP成長率會降到哪裏,而只是說景氣週期循環已經開始進入每五年裏最糟糕的那一年。不論官方的數字是多少,它的實際GDP成長可能是4-5\%左右,一般商業感受的,就是營銷額下行。如果你有親友做生意的,不用等到春節,現在去問問,他們應該已經感受到寒意了。
\\

\textit{\hfill\noindent\small 2018/12/27 13:41 提问;2018/12/28 05:34 回答}

\noindent[5.]{\Hei 答}:
歐洲的經濟,如同中國一樣,已經開始減速了;法國的動亂就是在這樣的經濟周期背景下才發生的。
至於收割歐洲,當初德國加入歐元,就是為了抱團增重,不再讓美元這條狗當成尾巴一樣來搖。這個效應還是存在的;再加上“美國第一”的政策使得美方的投資環境惡化,所以特別大的資金外逃機率並不大。
\\


\section{【基礎科研】高能物理界的新動態}
\subsection{2019-01-02 10:25}


\section{2条问答}

\textit{\hfill\noindent\small 2019/01/02 22:39 提问;2019/01/03 05:59 回答}

\noindent[1.]{\Hei 答}:
上周南山臥蟲問我有關中石化在期權上大虧的事,我說他們必須有風險管制系統,而我可以勝任,我心裏假想的是開一天的培訓班,就能保證學員有華爾街銀行的專業程度。
年輕的時候,不是特別喜歡教書,結果年紀大了,反而習慣性地想要教導別人。
\\

\textit{\hfill\noindent\small 2019/01/03 09:44 提问;2019/01/03 09:55 回答}

\noindent[2.]{\Hei 答}:監督期權風險的事,1/3是技術,2/3是態度,光寫下技術是沒有用的。
\\


\section{【心理】爲什麽事實與邏輯對群衆無效?}
\subsection{2019-01-17 20:31}


\section{1条问答}

\textit{\hfill\noindent\small 2019/08/15 18:15 提问;2019/08/16 05:04 回答}

\noindent[1.]{\Hei 答}:我可不是在開玩笑,香港的確是亞洲的頭號金融中心,比東京還要重要。它的吸引力就在於方便做中國的生意,卻又是中國管不着的。
\\


\section{【美國】域外管轄權}
\subsection{2019-01-27 12:27}


\section{2条问答}

\textit{\hfill\noindent\small 2019/01/29 19:31 提问;2019/01/30 05:08 回答}

\noindent[1.]{\Hei 答}:
資本主義和市場經濟的妙處,就在於參與者不須要直接合謀,每個人只想著自我利益最大化,結果自然就達成複雜的分工合作。
這裏GE不必直接參與,只要擺出它手中的國會議員和法院人脈,在幕後主導起訴Alstom的那批司法部官員、律師和法官,自然就要看它的面子、把好處送給它。
\\

\textit{\hfill\noindent\small 2019/02/10 14:04 提问;2019/02/11 04:23 回答}

\noindent[2.]{\Hei 答}:這裏,中美有根本性的不同:中國是貿易型的經濟,美國卻可以靠强勢美元,用印鈔機印出來的白紙交換其他國家的勞動成果,是掠奪型經濟。域外管轄權對美國來説,只不過是換一種掠奪方式;由中國來用,卻會嚇走貿易對象。
\\


\section{【基礎科研】大對撞機不是好的基礎科研項目}
\subsection{2019-04-08 04:37}


\section{1条问答}

\textit{\hfill\noindent\small 2022/03/18 02:28 提问;2022/03/18 06:35 回答}

\noindent[1.]{\Hei 答}:傳統土木工程的問題不在於技術風險和專業忽悠,而是經濟效益的評估,這只能由當地當事團隊仔細分析,我一個外人無從置喙。這類議題應該留給直接瞭解真相,而且願意實名作證的人來討論;除非你有第一手核實的論據指出官方評估結果有錯,並且看過/能反駁支持建造工程者的意見,否則就應該閉嘴,以免傳播噪音污染公共論壇。尤其在這個博客,如果你引用的論述來自匿名的網絡評論,直接違反《讀者須知》第六條,嚴重警告一次,禁言兩個月,再犯拉黑。


在此特別提醒大家,不尊重我的時間精力,把這個博客的留言欄,當成其他網絡論壇一樣隨便發言的人,都有被立即拉黑的危險。這裏的一切討論,必須字字以求真為目標;網絡上到處都是網紅“學者”,專門滿足求爽的意願,一般人用不著來這兒騷擾真正想做學問、求真相的絕對少數。
\\


\section{【美國】【戰略】三談中美貿易戰}
\subsection{2019-05-15 05:42}


\section{6条问答}

\textit{\hfill\noindent\small 2019/05/15 22:38 提问;2019/05/16 12:43 回答}

\noindent[1.]{\Hei 答}:
不只是Napoleon III,他之前的Louis Phillippe I也是吹泡沫的專家。只是Trump之不學無術,比他們有過之而無不及。
不過美國經濟體量很大,真要努力吹泡沫,虛胖可以持續多年。Trump逼迫聯儲會停止加息後,經濟就又升溫。他能否把經濟衰退推到明年大選之後,還有待觀察。
\\

\textit{\hfill\noindent\small 2019/05/23 19:18 提问;2019/05/24 04:34 回答}

\noindent[2.]{\Hei 答}:
根據經濟周期來炒股的難處,不在於看出正確的大方向,而是如何確定轉折爆發的時機。華爾街有一句老話:The market can stay irrational longer than you can stay solvent.
美國經濟絕對是嚴重虛胖的,但是要真正確定這兩年就會泡沫爆裂,我想是超出人類所能。
\\

\textit{\hfill\noindent\small 2019/08/16 03:19 提问;2019/08/16 05:07 回答}

\noindent[3.]{\Hei 答}:這正是過去10年,中國央行的政策方向之一,但是國際金融市場的慣性很強,中國又不能完全自由化,所以先讓歐元興起或許是比較實際的目標。
\\

\textit{\hfill\noindent\small 2019/08/24 20:12 提问;2019/08/24 20:29 回答}

\noindent[4.]{\Hei 答}:
美國的經濟已經到衰退的邊緣了;我連月份都預測正確,自己也很滿意。
Trump現在真是黔驢技窮、色厲内荏;連一般消費品也加關稅,今年聖誕節購物季一定會讓老百姓怨聲載道。所以他想撐也絕對撐不過12月。我覺得是遠在那之前他就必須找臺階爬下來了。
\\

\textit{\hfill\noindent\small 2020/01/15 22:40 提问;2020/01/16 01:49 回答}

\noindent[5.]{\Hei 答}:
知之爲知之,不知爲不知;這件事屬於後者。我一向對主流經濟學沒有好感,所以也不會去深研其細節。 
還有,再提醒你一次,我不喜歡一句一段落的格式。
\\

\textit{\hfill\noindent\small 2023/08/26 23:13 提问;2023/08/27 01:20 回答}

\noindent[6.]{\Hei 答}:
戰術選擇,由前綫將領自行裁決,後方不宜指手畫脚。
這裏的問題在於,投降背叛在初期也可以假裝為戰術選擇,例如我對易綱就給了三四年觀察期,未做批評,等到局勢無可挽回才能確定真相。
\\


\section{【國際】【政治】21世紀之民粹}
\subsection{2019-06-04 11:20}


\section{9条问答}

\textit{\hfill\noindent\small 2019/06/04 20:02 提问;2019/06/05 00:51 回答}

\noindent[1.]{\Hei 答}:
Piketty的結論,是大資本家每年的稅前資本利得,在和平時期必然高於GDP成長率。以往歐美獨霸世界的工業產值,那麽因爲GDP成長率夠高,大資本家還勉强同意通過纍進稅率、工人高薪和福利政策,來使稅後資本利得接近GDP成長率,結果是貧富極端化還不明顯。 
在1960-70年代,因爲德、日的復蘇,瓜分了工業產值,然後又有石油危機,英美的GDP成長率忽然掉下去了,這時大資本家就不可能坐視每年只拿2\%或3\%的回報,他們決定盡全力對社會主義福利政策做反撲是必然的。所以英美資本家對中產階級的殺鷄取卵,其實早在中國現代化之前就開始了,只不過後果因爲1980年代的負債消費和1990年代的冷戰勝利紅利而暫時沒有浮現而已。 
當然,中國的體量比德日加起來還大得多,對工業產值的瓜分作用也遠遠更强,即使考慮到世界經濟不是零和游戲,總體和長期來説,先進工業國家在21世紀還是必須面對一個更大的負面衝擊。
\\

\textit{\hfill\noindent\small 2019/06/05 04:19 提问;2019/06/05 06:53 回答}

\noindent[2.]{\Hei 答}:這些理論細節並不影響他結論的正確性,因爲有以千計的事實證據支持它,所以實在沒有必要去吹毛求疵。
\\

\textit{\hfill\noindent\small 2019/06/09 13:30 提问;2019/06/10 01:53 回答}

\noindent[3.]{\Hei 答}:
我再說一次:我對寫論文或者叠床架屋的理論建構沒有興趣,我的態度永遠是基於所有可驗證的事實,用最少層次的邏輯,追求達成最大程度自洽的認知。我相信這是要做出準確預測的最佳方案。
Piketty幾百頁的論述,絕對有錯誤或不足的細節,但是他的主旨(亦即資本報酬率和GDP成長率之間的關係)是被事實反復驗證,所以沒有疑義的。
\\

\textit{\hfill\noindent\small 2019/07/02 10:16 提问;2019/07/03 21:07 回答}

\noindent[4.]{\Hei 答}:
我覺得這種“微貸款”方案,構想很美,執行卻很難。非洲和南亞的幾個嘗試,在得了一堆國際獎項之後,無一例外在三四年内就破產,包括上個月剛停業的一個孟加拉機構。
其實原因不難理解:要分辨一個貸款戶是否能夠還錢,需要懂經濟金融的人才,花不少工夫才能研究完成;這個交易成本是固定的,不隨貸款金額大小而變動。所以銀行提供貸款,先天就是金額越大、效率越高。一旦爲了社會政治理念而强迫反其道而行,那麽自然無法負擔這個交易成本,結果必定是有過高的違約率,從而入不敷出。
中國在這方面,一直是依賴國營銀行的獨占性,由央行做政策性要求,强制增加對中小企業的貸款。那麽因爲貸款數額仍然不是太小,而且國營銀行的人工成本比較低,所以負面效應還可以忍受。如果真的照抄外國的微貸款,那麽後果反而會是金融危機。
\\

\textit{\hfill\noindent\small 2019/07/04 08:38 提问;2019/07/04 20:34 回答}

\noindent[5.]{\Hei 答}:Micro-Lending這門生意屬於商業銀行,我的專業卻是在投資銀行的交易所設計上,所以只有原則性的了解。它在執行上遇到的困難,過去十年有零零星星的新聞報導,我注意到了,也深思過其中的緣故,才答得出你的問題。至於要怎麼改進,這是世界級的難題,我不是最有資格解決它的人。
\\

\textit{\hfill\noindent\small 2019/07/14 11:23 提问;2019/07/19 22:01 回答}

\noindent[6.]{\Hei 答}:
美國在二戰後的經濟周期,原本是比較有規律的5-7年,但是冷戰結束後有一波全球化過程,美國内部的通脹壓力得以藉外力紓解,周期也就增長到10年左右。 
西方民選制度下的政客,一般是用減稅、減息、印鈔和貨幣貶值這些手段來人爲吹大泡沫,希望把破滅的時間延遲到他們退休之後。當然如此一來,其結果是經濟危機更爲嚴重。 
我沒有看《流浪地球》,但是兒子看了,他説比漫威的電影還幼稚,勸我不要浪費時間。我讀《三體》斷斷續續三年了,現在總算看到第二本,覺得還不是太糟糕,就是對物理和數學的描述,只有大學工科一年級生的程度;邏輯的破洞太多,讓人無法融入;人物的心理和個性也很單極,不是故事裏頂尖戰略人才應有的複雜多面。總之,劉慈欣的長處在於想象力豐富,對一般讀者來説,很有娛樂性。但我不是一般讀者;一本書裏每隔三四頁就看出一個學術、技術或人性上的嚴重錯誤,自然無法融入故事之中。
\\

\textit{\hfill\noindent\small 2020/07/01 05:57 提问;2020/07/02 04:35 回答}

\noindent[7.]{\Hei 答}:我什麽時候説過高經濟成長率導致高利潤率?美國在二戰後,工業一枝獨秀,以致成長率和利潤率雙高;我在正文只提了前者,並不代表你可以自己樹靶自己打。

你老是要堅持資本論裏面利潤率隨時間降低的論調,其實那是馬克思在一些隱形前提下的結論。他的假設主要是把歐美工業國家當做一個封閉系統,其他國家只能做爲原料供應地和成品傾銷的市場,而不考慮科技和產能向全世界擴散的過程和效應,也不考慮形成跨國托拉斯的可能。實際上實體產業的利潤率當然是逐步降低,尤其是外包給後進國家之後,那些真正從事生產的工廠利潤特別微薄,但是做零售、服務和尤其金融,完全可以維持舊有的高利潤,例如美國的商用房地產在過去30年的年報酬率在7\%以上,一直到今年在Amazon和新冠的雙重打擊下才有停止的勢頭。Amazon自己的利潤率很低,但這是爲了在行業興起早期奪取市場額分,目的是一旦沒有競爭對手,就可以盡情收割顧客和供應商,届時利潤率反而會隨時間增高;這一個道理廣爲人知,所以它的股價才能不斷高漲。

科技創新靠的不是自由民主,而是資源和組織,這一點我寫過博文《科技發展和美式自由無關》專門討論過了。

中國現在還是有邪教的,其中市場教和民主教已經被反復批判,還在被鼓勵的是中醫教;這一點我也已經仔細討論過了。
\\

\textit{\hfill\noindent\small 2020/07/02 06:42 提问;2020/07/02 08:26 回答}

\noindent[8.]{\Hei 答}:But this long run is a misleading guide to current affairs. In the long run we are all dead.---John Maynard Keynes

我對幾百年後人類社會的平衡態沒有興趣,因爲科技發展和國際政治完全無法推論到那麽遠。在可見的未來,維持高資本利得絕對是部分大資本能夠輕易做到的,只不過不是一般人所知的企業罷了,所以討論利潤率的長期下降趨勢毫無意義。
\\

\textit{\hfill\noindent\small 2020/07/06 12:11 提问;2020/07/07 08:56 回答}

\noindent[9.]{\Hei 答}:
此前還有讀者說中方在印度沒有基建生意。“不知道有”和“沒有”是兩回事;這是非常簡單基本的邏輯道理,但卻是當代公衆討論中的一大誤區,台灣和美國尤其嚴重。
那個河内輕軌建設期間還出過意外,至今越南輿論仍然以此爲例,斷言中國製造等於垃圾,後來中方就很難再拿到工程了。
Nigeria和Zambia這種名列前茅的無賴國家,更加不應該跪求他們的生意。在工程和商業界,一個重要Rule of Thumb是追求最後10\%的性能/額份,往往必須付出10倍的代價。在外交上,試圖討好最無良的10\%,也一樣是自討苦吃。
\\


\section{【台灣】返臺隨筆}
\subsection{2019-06-19 18:31}


\section{2条问答}

\textit{\hfill\noindent\small 2020/09/05 10:40 提问;2020/09/06 01:48 回答}

\noindent[1.]{\Hei 答}:當年日本和四小龍,都屬於冷戰中的美方陣營,所以若干程度的經濟發展是被容許、甚至鼓勵的。一旦工業能力接近美國,日本就被嚴厲打擊,只有四小龍能在對美方財團開放的前提下繼續發展。這主要是因爲後者的體量不足以威脅美國對外金融掠奪的霸權。

中國的體量十倍於日本,美國已經明定中國為頭號威脅,中方當然沒有新加坡式的餘裕來對賣國分子搞寬容和氣。出手必須快狠准,務求釜底抽薪、一勞永逸。
\\

\textit{\hfill\noindent\small 2021/08/24 00:42 提问;2021/08/25 04:54 回答}

\noindent[2.]{\Hei 答}:我在博客反復提過,貧富不均是21世紀人類面臨的三大問題之首,其背後的原因是市場經濟自然使資本不斷集中和纍積,所以要靠政治手段逆水行舟,先天就極爲困難。更糟糕的是,資本完全掌控了既有的國際霸權,所有現行的國際規則、制度、機構和理論,都是英美在過去200年精心設計來保護自家財閥的,所以只有中國的興起,才能挽救人類社會。

習近平在過去幾個月出手打擊國内的壟斷性財閥,然後進一步提出“共同富裕”的政策方針,已經是比我預期的更積極、强力的措施,值得全世界的良心人稱贊和支持。然而我們必須認識到,這些依舊只是非常初級的步驟,而且必然會面臨國内國外財閥的反撲。下一階段的鬥爭,將會是全方面的,而其中最讓我擔心的有兩項:

1)資本無國界,國内財閥早已有準備做資產轉移,國外的巨鰐則始終想要吃下中國的金融體系,以往的壁壘或多或少減低了金融方面的危險;現在中國正式向資本宣戰,卻同時大開國門,在金融上“與國際接軌”,這顯然是戰略冒進,承擔了不必要的風險。

2)一旦試圖把以往資方獨霸的權力和利潤,拿來與國内的勞方和消費者分享,那麽無可避免地會影響財團在國際上的競爭力。如果中國有獨步全球的研發環境和團隊,倒也罷了,偏偏中國的科研學術界沉厄甚深,造假、抓權、騙錢的風氣極盛,甚至形成多種畸形產業(例如代爲假造論文、假期刊等等),幾乎每個我深入研究過的領域,都由自私自利、竊位自肥的學閥主導。這在追趕階段,因爲只需複製歐美的成功前例,還不成問題,一旦中國進入第一梯隊,那麽美國在過去6、70年研發路綫選擇的低效現象,在中國會遠遠更加惡劣,很難想象不靠996和泡沫投資來維持競爭力,技術水平要如何進一步提升,整體經濟要如何進一步發展。

總結來説,我一直認爲解決貧富不均非常困難,即使不看整個人類世界,先專注在中國本身,而且只考慮初級步驟,也有三大難關:政治意願、金融策略、和研發效率。目前習近平只解決了第一項,後面的金融管理和學術風氣,是我多年來反復討論的議題,但是中方並沒有全面采納那些建議的跡象,所以我對這個“共同富裕”政策能否持久深入,並不樂觀。
\\


\section{【台灣】再談統一}
\subsection{2019-07-09 16:00}


\section{11条问答}

\textit{\hfill\noindent\small 2019/07/09 20:22 提问;2019/07/09 21:05 回答}

\noindent[1.]{\Hei 答}:如果沒有武統,台灣大機率會拖到有嚴重的金融危機,然後徹底希臘化,亦即有頭腦的人通通移民出國,弱勢群體的自殺率再創世界新高。届時中共如何來收這個爛攤子,現在還不容易預測細節。
\\

\textit{\hfill\noindent\small 2019/07/10 06:17 提问;2019/07/11 12:03 回答}

\noindent[2.]{\Hei 答}:
普選制的領袖都是短視近利,這是體制問題。像是德國的Schroeder為國為民做了那麼大的犧牲貢獻,奠下了經濟復興的基礎,結果反而是Merkel長期執政。 
台灣有自己的貨幣發行權,所以出了金融危機之後,只要大幅貶值,倒不須要有外來的救援,只不過居民的儲蓄會被一掃而光,生活水準更會斷崖式下降而已。
\\

\textit{\hfill\noindent\small 2019/07/11 13:36 提问;2019/07/11 16:40 回答}

\noindent[3.]{\Hei 答}:其實這樣的結局是最完美的:事後能夠做武統式的改革,但過程中抵抗特別弱,所以死傷也就特別少。我擔心的是,他們繼續做假賬,一路拖下去。
\\

\textit{\hfill\noindent\small 2019/07/11 18:06 提问;2019/07/11 21:27 回答}

\noindent[4.]{\Hei 答}:這個未來是必然的,問題只在要多久。
\\

\textit{\hfill\noindent\small 2019/07/11 18:23 提问;2019/07/11 21:50 回答}

\noindent[5.]{\Hei 答}:
至少還要六七年,多則20年,急慌慌地搞不划算,不過如果台幣升到和美金30比1,可以把閒錢換成外幣。
這個機會不一定會出現,因爲拖延金融危機的最簡單辦法,就是讓貨幣貶值。
\\

\textit{\hfill\noindent\small 2019/07/11 20:33 提问;2019/07/11 21:31 回答}

\noindent[6.]{\Hei 答}:問題是金融危機可以一直拖;希臘的難處在於它沒有主權,要受歐盟管轄。
\\

\textit{\hfill\noindent\small 2019/07/15 06:14 提问;2019/07/15 10:45 回答}

\noindent[7.]{\Hei 答}:
事實上,因爲現代資本一般可以輕鬆跨越國境,即使中共找到解決貧富不均的方案,一執行起來必然會使大資本奪門而出。所以解決貧富不均問題,還必須是世界協作,這的確是近乎不可能的。
不過至少在教育上,留給貧家子弟一些向上發展的機會,就能延遲貧富不均這個炸彈爆發的時間點。這是應該做,也做得到的。
\\

\textit{\hfill\noindent\small 2019/07/16 09:23 提问;2019/07/16 23:43 回答}

\noindent[8.]{\Hei 答}:
我個人覺得保險公司素來就是做賬的專業,所以出了問題,表面是看不出來的。我之所以懷疑他們,是因爲他們對保戶付出4\%的年收益,而轉過來在投資上要能保障5\%以上的收益,必須是一流的金融抄手,台灣的這些保險公司只怕不見得個個都有一流的金融人才。
Voyager他講得很詳細,可能太詳細了。我也頗爲好奇。
\\

\textit{\hfill\noindent\small 2019/07/18 08:49 提问;2019/07/18 10:22 回答}

\noindent[9.]{\Hei 答}:
大資本(亦即百億美元以上)可以僱傭第一流的管理和金融人員,一般追求的是在風險不大(單年度有風險,但是以十年來看基本無風險)的前提下,5-8 \%的年回報率,例如軟銀和Black Rock其實有時還能超過這個目標。 
隨著西方GDP成長率的下降,要維持投資報酬率就必須到開發中國家去,這就是所謂的國際熱錢。你所舉的報酬率降低現象,主要是因爲世界經濟的高成長轉移到中國,而國際資本基本無法到中國炒作。
\\

\textit{\hfill\noindent\small 2019/07/29 13:12 提问;2019/07/30 00:44 回答}

\noindent[10.]{\Hei 答}:可惜臺灣走的是土豪主導的資本主義;如果是國際財團主導的,衰退可能還慢一些。
\\

\textit{\hfill\noindent\small 2019/09/13 02:00 提问;2019/09/13 05:15 回答}

\noindent[11.]{\Hei 答}:
有關宇宙起源的問題,人類目前所知還遠遠不足以做猜測,更別提定論。在事實和邏輯根據都完全匱乏的狀態下,任何討論都是無意義的。
至於關於熱熵的比喻,我們取的是它必須隨時間單調增加的性質。實際上,熵的增加是同質化,亦即高溫的熱量會流向低溫,而經濟上貧富不均這個隨時間單調增加的性質,卻是異質化,也就是富者更富、貧者更貧。兩者之間並不完全等同。
\\


\section{【英國】談Brexit}
\subsection{2019-09-10 08:57}


\section{4条问答}

\textit{\hfill\noindent\small 2019/09/10 09:37 提问;2019/09/11 04:45 回答}

\noindent[1.]{\Hei 答}:
他們都是政治鬥爭的專家,但對金融、經濟和社會學並無瞭解。 
現代專業分得很細,跨界的知識非常罕見。政府的智囊團裏,能真正得到信任的,頂多就是幾個,結果自然是大家都是搞政治出身,偶爾有一個學經濟的,已經很了不起了。但是絕大多數的經濟學家,也是不懂金融的,社會學就更別提了。很不幸的,經過二戰後70多年的資本纍積,金融資本已經遠遠超過各國的經濟規模,所以他們往往是幕後最強大的政治力量。不懂金融,就不懂政治。 
此外,這些道理在我講清楚之前,你已經想到了嗎?真理的特徵之一,是事先很難想通,事後理所當然。
\\

\textit{\hfill\noindent\small 2019/09/12 10:17 提问;2019/09/13 08:02 回答}

\noindent[2.]{\Hei 答}:
你低估了英國國際財閥,尤其是倫敦金融界的力量。英國媒體雖然常常和美國一鼻孔出氣,卻並沒有真正的從屬關係。
\\

\textit{\hfill\noindent\small 2019/10/06 00:32 提问;2019/10/06 06:25 回答}

\noindent[3.]{\Hei 答}:
經濟學界被資本收買,不是“恐怕”,而早就是即成事實。我以前已經一再給出例子了。這個資本控制美國經濟學的系統化、常規化和體制化,才是一般人難以注意到的。 
至於金融在政策分析上的重要性,的確是遠超出一般學者的瞭解,也是爲什麽我能做出他人無法比擬的正確預測的原因之一。其道理,我以前在留言欄也提過了:二戰後70多年長期和平,資本得以成指數纍積,所以已經遠大於世界的GDP,然後在70年代起,又建立了一系列宣傳、收買和控制的組織和機制,使得英美政客成爲他們的傀儡。 
與其同時,金融銀行界和他們結盟,利用不斷演化的金融創新來剝削國民,並且為資本極度加大杠桿,使金融力量達到GDP的上千倍,那麽連專業客觀的財政措施也必然是以資本金融的利益爲先,所謂“Too big to fail”,就是這個原理的體現。 
總之,要瞭解英美的政治社會走向,就必須先分析出資本和金融的利害所在。美國訓練出來的經濟學人,如果是芝加哥系的,反而本能地要掩飾資本和金融的操作和影響,結果連經濟上的道理都會說反;東西兩岸學校出身的,並不懂資本和金融的重要,只能盲人摸象,講一些片面的細節;學金融的,只管在既有法規下鉆漏洞賺錢,沒有大局觀;做媒體的,原本應該暴露資本的惡行,但是媒體企業已經被嚴格掌控了,所以只能去搞政治正確類的白左或右翼議題。最終結果是,像我的博客這樣,把資本、金融、經濟、政治、宣傳之間的互相作用講清楚的,在英文世界並不存在。
\\

\textit{\hfill\noindent\small 2020/08/05 21:19 提问;2020/08/12 14:30 回答}

\noindent[4.]{\Hei 答}:學術能力建立在工業和經濟實力之上,失去根基之後,人才四散紛飛只是時間問題,1990年代的蘇東是前例。
\\


\section{【台灣】【工業】台灣能源供應的未來}
\subsection{2019-09-28 12:25}


\section{1条问答}

\textit{\hfill\noindent\small 2019/11/26 12:56 提问;2019/11/27 15:47 回答}

\noindent[1.]{\Hei 答}:
這個問題的癥結,其實在於GDP計算方法,你可以參考我以前寫的相關文章。 
中國的經濟,仍然主要是實體工業,所以GDP和工業用電有近乎正比的關係。美國的製造業在總量上看已經回天乏術,少數幾個還有優勢的區域,如石化工業並不是用電大戶,高科技則獨木難撐。舉個具體的例子,工業上用電最凶的,可能是煉鋁;在1988年,世界產量第一是美國,中國只有它的1/5,到了2018年,第一是中國,達到美國的35倍。 
至於美國GDP的持續增長,我以前也反復討論過了,虛胖的現象很嚴重;不過這和印度不同,不是統計單位刻意作假,而是金融、醫療、法律等等與生產沒有關係的服務業掠奪太大的利潤。例如美國的醫療佔GDP的18\%,而西歐有全民健保的國家,平均也只有9\%;這其中藥品公司雖然以天價著稱,其實只拿了醫療開支的14\%,真正刮錢的是保險公司和醫院,光是後者就佔30\%(不包括醫生和小診所,這兩者另佔了20\%),也就是藥品的兩倍多,隨便一個盲腸炎手術就要十多萬美元,GDP不膨脹才怪。
\\


\section{【美國】【經濟】從回購利率暴漲談美國經濟周期}
\subsection{2019-10-01 14:51}


\section{21条问答}

\textit{\hfill\noindent\small 2019/10/01 19:03 提问;2019/10/02 02:30 回答}

\noindent[1.]{\Hei 答}:
是的,這些道理雖然一直有人隱約瞭解,自從我四年前的《美元的金融霸權》之後,中文讀者開始有了比較深刻、具體而且系統性的認識,後來有許多其他作者做了更多的演繹引申,是我教育新一代中國人的一個成功案例。
\\

\textit{\hfill\noindent\small 2019/10/02 01:58 提问;2019/10/02 02:29 回答}

\noindent[2.]{\Hei 答}:
對外借貸,只是誘惑,或許可以用政策遏制;但是貨幣價值堅挺,對實體貿易不利,就很難避免。
一個解決方法,是創造一個世界貨幣,例如中國一直鼓吹的IMF SDR(Special Drawing Rights)。當年德國加入歐元,就是同樣的思路。
\\

\textit{\hfill\noindent\small 2019/10/03 21:20 提问;2019/10/04 01:00 回答}

\noindent[3.]{\Hei 答}:
1. 對Trump這人實在很難做出預測;照理説,他早應該和中方妥協了,但實際上,他是現在美國右翼民粹總操弄者,又正在全力發動他們抵制罷免,所以整體來看,還是只能指望回歸到今年六月G20後的短暫休兵狀態。 
2. 這其實正是正文裏,那句“與其去研究2008年的金融危機...”的用意,亦即消費者固然不像2008年那樣負債纍纍,企業卻是被淹沒在垃圾債券之下。 
3. 同樣的,這次回購利率暴漲,危險並不是2008年金融危機式的連鎖反應,而是銀行手頭現金不足,必須削減貸款量,這不但會使經濟衰退無法避免,事實上本身就可以是衰退的導火綫。 
一年前,我忽然去討論印度的一個影子銀行破產的事,就是覺得它對印度經濟會有同樣的作用。現代社會,每天的新聞以百萬計,但是我每隔幾天才會寫一篇文章,其實已經是從無數沙礫中去挖出鑽石了,只不過偶爾沒有把方方面面的細節全部直白討論。你如果有時間再多讀幾次,我想還是可能會有繼續的收穫。
\\

\textit{\hfill\noindent\small 2019/10/04 20:25 提问;2019/10/05 00:35 回答}

\noindent[4.]{\Hei 答}:
這個過程需時比打破美元+金本位的Bretton Woods還要久,而且也不是持續逐步改變,是壓力慢慢增加到軍事、外交、宣傳霸權攔不住了,一夕之間水壩崩塌。
\\

\textit{\hfill\noindent\small 2019/10/06 13:22 提问;2019/10/06 21:14 回答}

\noindent[5.]{\Hei 答}:
我看到的房貸統計,遠遠沒有到2008年的水平。
不過的確,財政和貨幣供應都沒有多大刺激的空間,這是我在正文裏所説的“慢性症狀”。
\\

\textit{\hfill\noindent\small 2019/10/12 01:19 提问;2019/10/12 03:56 回答}

\noindent[6.]{\Hei 答}:基本原理是如此,但要談周期長度,就必須討論“失衡”的過程細節,也就是正文裏的那兩點。
\\

\textit{\hfill\noindent\small 2020/01/03 13:23 提问;2020/01/04 05:44 回答}

\noindent[7.]{\Hei 答}:
中央銀行當然可以印鈔票來刺激經濟,短期的代價是通貨膨脹、貨幣貶值和外匯流出;不過這對美國來説並不適用,因爲美元是國際儲備貨幣。 
中期的危險是把泡沐繼續吹大;這是爲什麽美聯儲原本不太願意降息的原因。但是未來總是有隨機因素的,説不定泡沫自己會消失(雖然機率很小很小);而現在不放水則有100\%的機率會馬上有金融危機,那麽也就只好賭一把了。 
長期的負面作用是美元的信用下降,可能失去國際儲備貨幣的地位。但是美國現在的政經體制,已經無力考慮長期的利害關係了。
\\

\textit{\hfill\noindent\small 2020/02/23 02:00 提问;2020/02/24 02:27 回答}

\noindent[8.]{\Hei 答}:這些説法和我過去這一年所做判斷的大方向是一致的,或許那位教授直接或間接看到這些意見。我寫作的目的,一向是要影響意見領袖,才可能扭轉輿論,所以如果真是如此,是件好事。

不過我要修正這裏的若干細節。現在美國貸款利率其實又接近了歷史性的低點;不但長期利率因爲大銀行看衰經濟所以低迷不振,短期利率也只有百分之一點多,還有美聯儲拿出几千億美元來針對性地放水。企業就算不走借貸這條路,股市的狂歡也打開了其他融資方法的大門(例如Preferred Shares,Convertible Bonds等等)。

我不知道今年七月企業債到期有多麽集中,不過如果真的造成融資瓶頸,美聯儲必然會再一波放水,輕鬆解決。金融的特性是表面上容易看到的一級效應,自然會吸引資本套利或央行補償,真正會引發麻煩的,是慢性問題由二級或三級效應引爆。這是需要實戰經驗才能領會到,所以也往往是學術界的盲點。
\\

\textit{\hfill\noindent\small 2020/03/10 12:27 提问;2020/03/11 02:42 回答}

\noindent[9.]{\Hei 答}:美國經濟在過去這年,靠的是三個支柱:私人消費、股市泡沫和美聯儲放水。疫情會直接打垮消費,股市泡沫已經開始爆破,美聯儲花了12年都沒有辦法收回上一輪的量化寬鬆,這一次繼續印鈔票必然會面臨收益遞減的現象。所以一個至少等同2000年的經濟衰退是跑不掉了。

我已經解釋過,2008年先倒閉的是金融機構,這次大銀行早早就置身事外,所以問題會集中在非金融企業,尤其是資本密集的產業,例如頁岩油氣。Putin和沙特決定不減產,最終的考慮就是要落井下石,一舉消滅美國的中小型頁岩油氣公司。

如同2008年後,美國一直到2014年才喘過氣來,這次美國至少也會經歷三四年無力外顧的階段;雖然輿論上已經完成仇中的全面動員,有心無力也是無法可施。例如法國在二戰前,不是看不出德國國力復蘇,但是國家客觀力量不足,政治主觀内鬥不止,也就不可能主動出擊,只能選擇戰略收縮和防禦。中方恢復元氣,也就是一年半載,届時可以安享幾年的冷和平,持續改變雙方的力量對比,直到美方知難而退爲止。
\\

\textit{\hfill\noindent\small 2020/03/13 10:24 提问;2020/03/14 03:31 回答}

\noindent[10.]{\Hei 答}:這次的經濟危機,早已遠超Trump和美聯儲能堵上的程度;事實上大前年的減稅、去年的放水和三年來預算的濫用,都徒然吹大泡沫,讓他們自作自受。

我已經一再説過,和2008年相比,這次的差別在於大銀行沒有參與狂歡,所以不會有金融界的連鎖反應。爆炸的核心是企業債,中小銀行會因此而倒下一片,但是知名的國際銀行不會需要2008年級別的聯邦救援。但是我原本拿2000年的衰退來相比,這在新冠這個黑天鵝出現,又被Trump政權胡搞因人禍而擴大之後,必須稍作修正:震央仍然是股市和非金融企業,所以性質類別相似,但是程度會更嚴重許多。這是因爲美國的經濟和國力比20年前衰弱不少,而新冠不但會打擊經濟需求(Demand),而且會大幅遏制供給(Supply),甚至會影響社會穩定。
\\

\textit{\hfill\noindent\small 2020/03/24 11:34 提问;2020/03/25 11:39 回答}

\noindent[11.]{\Hei 答}:在這類大銀行裏,往往總部在擔心經濟風險的時候,若干交易員仍然在偷偷增加杠桿,以謀求更大的賬面利潤。

不過3萬億的CDS實在太多,這些銀行不太可能管理鬆懈到這樣的地步,所以這個報導可能有誤。
\\

\textit{\hfill\noindent\small 2020/03/30 18:12 提问;2020/03/31 10:04 回答}

\noindent[12.]{\Hei 答}:這次美國的經濟衰退,一個合適的歷史前例,是1990年前後日本經濟泡沫的爆破,兩者同樣是浮腫的繁華假象被徹底戳穿。

當時日本股市崩盤,一般人也以爲已經過度强勢的日元(在1985年的Plaza Accord,美國强迫日元升值,從250:1一年就升到120:1,正是日本經濟吹起泡沫的主因)會開始貶值,但是真正内行的人應該看出事實會剛好相反:正因爲日本的公司現金流開始出問題,他們必須賤賣海外資產,把錢匯回日本,所以後來日元不降反升,從1990年到1994年,匯率從130:1升到80:1,其後才慢慢貶值,上下振蕩到近年的120:1。

現在也是一樣的:一切其他資產都比美元現金的風險更大,再加上美元的地位比當年的日元還更強得多,不只是美國人,連其他國家的資本也會想要換成美元來“避險”,這樣一來,美元反而有很大的升值壓力。當然最終美國經濟空洞化、貨幣過度發行,這些利空的長期基礎因素會顯現出來,但是那必須等到一個替代美元的新國際儲備貨幣有能力接收百萬億級別的資本流動才會發生。
\\

\textit{\hfill\noindent\small 2020/04/01 01:06 提问;2020/04/01 06:06 回答}

\noindent[13.]{\Hei 答}:歐洲本身經濟和國力也不樂觀,所以歐元沒有能力單獨取代美元。我個人還是比較看好像是金磚加密貨幣這樣的籃子貨幣,和歐元聯手,瓜分美元的額分,先進入一個三國鼎立的階段,在二三十年後,才有新的貨幣霸主脫穎而出。
\\

\textit{\hfill\noindent\small 2020/04/01 05:37 提问;2020/04/01 08:30 回答}

\noindent[14.]{\Hei 答}:本質上沒有差異,細節上增加了可以用作抵押的金融產品種類,數額也大幅提高了。

最近幾天,這些Repo窗口的訂購率遠低於100\%,這其實並不奇怪:這次的危機,核心不在金融產業,大銀行基本沒有問題,影子銀行的總規模有限,美聯儲放水的流量增加太多太快,自然超出影子銀行的總需求。我已經多次解釋過,最大的毛病出在企業債和股市,所以美聯儲在未來幾周開始直接采購企業債券(目前只在一周前開始買一天期的周轉性企業債)和股票的機率是相當高的。

請避免使用一句一行的格式。
\\

\textit{\hfill\noindent\small 2020/04/01 07:34 提问;2020/04/01 08:24 回答}

\noindent[15.]{\Hei 答}:這裏的差別在於歐美沒有國營企業和銀行能有效直接地把資金注入到實體基層經濟裏去,不論是中央銀行放水還是政府擴大支出,能受益的都只有大財團,他們沒有責任立刻花錢(事實上市場局面明顯地是等待越久價格越低),等到下滲到老百姓手裏,已經所剩無幾,所以這一次直接把錢交給消費者,其實是一個進步。不過你看美國的法案裏,這種直接刺激只占總額的15\%左右,大部分的錢還是交給富豪,就顯然不如歐洲。
\\

\textit{\hfill\noindent\small 2020/04/28 04:34 提问;2020/04/28 07:12 回答}

\noindent[16.]{\Hei 答}:是的。
\\

\textit{\hfill\noindent\small 2021/03/07 06:19 提问;2021/03/07 08:10 回答}

\noindent[17.]{\Hei 答}:兩年前全世界還不知道會有新冠的時候,我已經明確指出美國經濟泡沫會在2020年爆破,其後美聯儲必須放棄只完成1/3的銀根回收過程,直接返回大水漫灌政策,而且這會是美元周期循環收割全球的絕唱。後來疫情大幅加劇了經濟衰退的程度,美聯儲的量化寬鬆至今比上一輪增快了4倍,而上次2014、2015年的回收階段力量已經很微弱了,所以這一輪吸力近乎消失是很自然的事。

這個現象體現出來,除了美聯儲從主動落為被動之外,另一個特徵是利率和匯率的脫鈎。如果你仔細去看這兩年至今4萬億的新美債是誰在認購,就會發現傳統的外國買家(亦即以中日爲首的中央銀行)基本沒有增購,人民銀行反而在慢慢減持,專業金融機構也頗爲小心,結果除了美聯儲之外,真正購買的大頭就只是零售和半零售(指沒有什麽技術和知識含量的金融載體,例如Mutual Fund和Pension Fund)投資人。如果美聯儲被迫在財政部還在狂抛債券的階段就提升利率,那麽不但匯率的反應會很有限,而且中產階級投資人要吃大虧,一旦債券利率飆升失控,聯邦政府的利息支出將以倍數成長,反而要敲響美元霸權的喪鐘。中國應該預做安排,和歐盟、德、法、歐元銀行等相關機構事先準備好協同預案,以便迅速出手保護中歐的共同利益。

至於美國國内加稅,你放心,有共和黨在,這麽重要的正事絕對幹不成。尤其是財政崩潰眼看著要發生在Biden任内,剛好是Trump把責任賴在民主黨頭上的天賜良機;早先我說過,Trump面臨幾十件州級和民事官司,只怕不再能夠自身出任公職,但如果有了世紀級別的財政危機,自然另當別論。
\\

\textit{\hfill\noindent\small 2021/06/20 05:26 提问;2021/06/21 12:10 回答}

\noindent[18.]{\Hei 答}:
長期通貨膨脹的主要動力,正在於人群的預期心理,所以它不但是非綫性的,而且是Path Dependent(路徑依賴,亦即不是一個State Function態函數),還可能是Chaotic。很不幸的,這也代表著它難以確實預測,只能約略估計危險程度高低。
中方沒有理由再為美方壓低進口物價。反過來看,進口物價也不是美聯儲的頭號麻煩,當前真正最讓他們憂心的導火綫,其實是美國服務業的大重整:新冠不但導致暫時的歇業,也永遠地改變了消費習慣和上班常態,網上購物和遠程工作並不會因爲疫苗普及和店面重開而消失。這大幅延長了勞動力重置(Worker reallocation)的過程,以致出現了就業率低迷、而同時企業卻雇不到人的矛盾現象。這裏的危險在於後者會導致薪資上漲,而前者卻迫使美聯儲繼續放鬆銀根。
我個人認爲,美聯儲騎虎難下,早已把自己逼到墻角(Paint self into the corner),現在連國債市場原有的通脹估價職能都被洪水般的美元流量(Liquidity)所淹沒,卡在同一個極低利率的讀數上。這篇正文討論的2019年回購利率暴漲問題,現在剛好反過來,成爲銀行爭先恐後地搶購“反回購”(“Reverse Repo”,亦即用現金向美聯儲交換債券,最新的數字高達7000億美元),反映的正是金融系統裏現金充斥,而國債回報率又過低的窘態(至於爲什麽Reverse Repo的回報率居然會高於直接買國債,讀者別忘了,前者是美國國内金融機構的特權,後者卻是美國對外吸血的管道)。
美聯儲的難處是一方面已經印鈔過多,實體和虛擬經濟都無力繼續吸收,另一方面低就業率和Biden的巨幅赤字又預先排除了積極緊收銀根的選項,只好硬著頭皮持續放水,靠喊話來安撫群衆,同時指望爆雷後果由外部經濟體來承受。拿破侖曾説過(這裏采用常見的英文版,我確認過這合理地反映了他所説法文的原意):“Never interrupt your enemy when he is making a mistake.” 可惜2008年,中國不但打斷了美國的錯誤,而且進一步獻身擋子彈;這次至少應該設法置身事外,以自保為優先。當然更好的做法,是聯絡其他利益相關國家,將美元的破壞力盡可能局限到自己境外,這需要多國中央銀行的協作,可以從俄國做起。
\\

\textit{\hfill\noindent\small 2022/01/08 12:25 提问;2022/01/09 06:27 回答}

\noindent[19.]{\Hei 答}:被逼的,但不可能做到。

是的。
\\

\textit{\hfill\noindent\small 2022/05/25 01:48 提问;2022/05/26 13:41 回答}

\noindent[20.]{\Hei 答}:首先大家都不看好資本市場,例如股市下跌,很多基金趕緊縮表。原本現金的短期儲存主要靠買短期國債,但目前Treasury bill的yield是0.51\%,而Excess Reserves剛剛從0.4\%上調到0.9\%,Reverse Repo則是0.8\%,顯然是較優的選擇。

至於爲什麽Reverse Repo的利率稍低,反而更受歡迎,這是因爲Excess Reserves是大銀行才有的特權,而現在這些現金主要來自影子銀行界,Reverse Repo是他們的最優解。

Excess Reserves的利率在五月4日之前是0.4\%,遠低於其他管道,所以總量下降是正常的。Excess Reserves的利率調整,必須由FOMC開會通過;Reverse Repo則是NY Fed一家説了算,所以後者要靈活很多。
\\

\textit{\hfill\noindent\small 2023/03/11 11:43 提问;2023/03/13 04:09 回答}

\noindent[21.]{\Hei 答}:SVB是地區小銀行,主要服務矽谷當地的初創企業;這次出事原因在於主管對美聯儲QT和升息毫無預期和準備,資金大部放在國債和MBS(Mortgage-Backed Securities;這兩者正是美聯儲QE/QT的主要管道,表面上很保守,實際上對貨幣政策非常敏感),升息後價值下跌,資不抵債的危險最近被公開,因爲初創企業的存款遠超FDIC保險限額,立刻就引發擠兌。總之是很特殊的個例,並不會導致系統崩潰。


剛剛有朋友私下通信來做進一步討論,把我的意見在此也和大家分享:

1)對若干初創企業存款戶會造成困難,但嚴重程度要視財政部如何善後,而且對美國經濟體系整體來説,初創企業原本就是可以隨時倒閉的高風險行業。換句話説,SVB破產的真正影響,在於風投業,而不是金融體系。

2)至於富國銀行(Wells Fargo)也出現擠兌現象,以及美國銀行(BOA)股票大跌,過去幾天紐約證券商的確正在風聲鶴唳,一些基金急著脫身,另外有一些想做空趁機牟利。不過美式金融體系原本就是高度不穩定的,謠言橫飛的實際影響,主要看財政部和美聯儲出手的早晚和强弱。目前的大銀行(含富國銀行)財務狀況,純粹只是Liquidity、而不是Solvency 的問題 ,Yellen和Powell無需總統和國會授權,就可以簡單解決,他們必須是異常的蠢才會放手不顧、任其惡化,所以我預期富國的最糟糕脚本是美聯儲注入資金,不會到破產的地步。

3)對金融的真正負面效應,在於逼迫美聯儲提早結束QT;不過考慮到民間美元現金的存量仍在高位,地區性銀行倒閉的作用有限,而且引發的通脹惡化問題是慢性的,必須等待其他更嚴重的黑天鵝和灰犀牛事件(例如金磚貨幣)做出打擊,才會有顛覆性的連鎖反應危險。
\\


\section{【宣佈】讀者須知}
\subsection{2019-10-06 05:50}


\section{1条问答}

\textit{\hfill\noindent\small 2023/03/22 01:43 提问;2023/03/22 03:31 回答}

\noindent[1.]{\Hei 答}:低端產業外包是美國40年來的國策,再多一條銷售管道有何奇怪?商業巨頭如Walmart和Amazon都是倚靠著中國製造而崛起的,除非Temu從他們手中奪取足夠利潤,才會有動力找藉口來做打擊;不過我覺得中短期内不會發生,因爲Temu競爭替代的主要是Ali Express.
\\


\section{【藝術】【基礎科研】藝術與科學的衰敗}
\subsection{2019-10-11 08:34}


\section{1条问答}

\textit{\hfill\noindent\small 2022/07/06 14:03 提问;2022/07/08 07:27 回答}

\noindent[1.]{\Hei 答}:WireCard的確是極爲惡劣的案例;尤其事先官方曾經想要處罰、起訴《FT》,事後主犯在逃至今,都是匪夷所思的腐敗症狀。我曾討論過柏林Brandenburg機場和Stuttgart 21的貪腐案件,最後也是沒人認責,不了了之。德國如同日本一樣,在文化上也被昂撒有意滲透腐化,在衰敗的路上走在前列,成爲隨時可以視需要拿來宰殺充飢的走狗。
\\


\section{【海軍】有關航母的一些新消息}
\subsection{2019-10-18 09:01}


\section{1条问答}

\textit{\hfill\noindent\small 2019/10/20 03:50 提问;2019/10/20 04:21 回答}

\noindent[1.]{\Hei 答}:
這類問題,隨著科技和社會結構的進化演變,是必然會出現的;一個朝氣蓬勃的國家,必須與時俱進,時時刻刻檢討反省新的現象是否對整體公共利益有助益。你如果去看美國40-60年代的輿論,他們大體上可以做到這一點;從70、80年代開始,就反過來了。這個反轉最早的始作俑者,是芝加哥大學的Milton Friedman;他開始宣傳“Greed is good;greed is the source of all human progress.”我看過他在電視上受采訪,他舉的頭一個例子居然是Einstein!但是不論他的歪論如何離譜,芝加哥大學硬是能出雙倍的薪水給年輕經濟學教授,所以在1980年之後,成爲美國經濟學的正宗。
擔心國家社會整體利益就是“善”,堅持事實與邏輯是“真”,排除商業性低級娛樂是“美”。能强調追求“真善美”的國家民族,理當興起;反其道而行者,該當沒落。
\\


\section{【美國】簡評民主黨彈劾Trump}
\subsection{2019-11-08 01:43}


\section{2条问答}

\textit{\hfill\noindent\small 2019/11/12 13:08 提问;2019/11/14 03:46 回答}

\noindent[1.]{\Hei 答}:
爭取時間的前提是做不能馬上見效的深刻改革;美國在過去40多年,只見財閥努力扭轉羅斯福和詹森遺留下來的健康社會和政治體制,基本是大步倒退,時間拖得越久則退得越遠,談何改革?
\\

\textit{\hfill\noindent\small 2019/11/15 08:16 提问;2019/11/16 14:02 回答}

\noindent[2.]{\Hei 答}:
我並不迷信任何一個個人的寫作,因爲我有自己的雙眼和大腦,可以自行依事實和邏輯得到正確的結論。 
馬克思對資本主義欠缺公平合理性的批評,是到位的,但是像“資本主義的趨勢是長期利潤率下降”這樣的論述就是明顯錯誤的。我看到的21世紀被動資本投資報酬率(也就是不參與主動經營),依舊是5-8\%,這和18、19世紀並沒有本質上的差別。實際上是,任何特定產業遲早都會飽和,然後利潤率下降,但是整體工業社會有不斷的汰舊換新,新的產業出現的頻率甚至還是隨時間而增高的。做真空管的,利潤早就消失了,但是半導體工業起而代之,所產生的總利潤遠超既往。 
階級鬥爭是一個更大的誤區:不論出身如何,一旦有了政權,就自然成爲新的最上層階級,無產階級也不例外。不管經濟路綫走的是市場還是計劃,不管掌權者來自投票、指定、還是父死子繼,他都會面對無數誘惑要侵害國家的利益以自肥,這是不可能從制度上完全防堵的,最終還是要依靠個人的理想、能力和道德。政治經濟體制的設計和改革,不在於追求一勞永逸,而是盡可能提拔適任的人來居高位,然後鼓勵他們謀求公益的最大化,所以也必須時刻檢討組織自我的缺失,因應世界環境的改變而與時俱進。 
我已經在《讀者須知》裏面説過,拿既有的迂腐理論來説教,是浪費大家的時間。我這次是法外施仁,讓你知道爲什麽你的那一套説法沒有在這裏討論的價值;以後再犯,我還是格殺勿論。 
你還是執迷不悟,我只好把你拉黑了;我的時間有限,必須兼顧幾萬名讀者,一個人想要獨占大百分比的公共資源,是非常自私惡劣的事。
\\


\section{【金融】【戰略】BRICS高峰會議的重要議程}
\subsection{2019-11-16 13:11}


\section{12条问答}

\textit{\hfill\noindent\small 2019/11/17 08:47 提问;2019/11/17 09:23 回答}

\noindent[1.]{\Hei 答}:
BRICS抱團,美國很難用上威脅、利誘、侵略、顛覆的故技,再加上大選年已至,還有彈劾案占據媒體版面,然後經濟財政正在出問題,只怕是自顧不暇了。 
不過英美媒體對此事一致自我封口,有點詭異;目前刊登這個消息的美國網站,無一例外都是做加密貨幣的,從俄國傳媒轉載而來。不過這也可能是因爲大家的注意力都投注在内鬥上,要過一段時間才會有智庫出報告。
\\

\textit{\hfill\noindent\small 2019/11/17 14:59 提问;2019/11/18 03:24 回答}

\noindent[2.]{\Hei 答}:
中共在四年前,也面對類似的難關,也就是世界銀行和IMF兩個美國控制的傀儡組織。後來決定取代前者,威懾後者(因爲至少IMF的老闆必須是歐洲人)。效果可以說是讓人滿意的。 
SWIFT的威脅更大。中方不是不想早日解決,而是金融界的慣性和其他國家的無感,使得替代性的系統無法得到足夠的支持。現在幸虧有Trump三年來不懈的努力,讓世界各國(包括SWIFT系統的合夥人:歐盟)都瞭解事態的緊急和嚴重性。我想這次中俄兩國謀定而後動,成功指日可待。 
至於SWIFT本身,美國不可能會放棄對其的完全控制,所以長期下去,應該會淪爲美國和幾個緊密小弟(如加拿大)所專用的區域系統。
\\

\textit{\hfill\noindent\small 2019/11/18 03:41 提问;2019/11/18 10:44 回答}

\noindent[3.]{\Hei 答}:
你問這個問題,不但太小看了Putin,也置正文的前三段於不顧。我寫作已經是極度精簡,半句廢話都沒有,就算是一連重複讀上三四遍,也用不了多久。請不要因個人的懶惰,浪費大家的時間。 
在現代互聯網普及之前,日本和美國(AOL)都有舊式的數位網絡;互聯網的優勢在於應用廣、適用靈活,所以後來居上。區塊鏈和以往的交付系統相比,也有類似的優勢。
\\

\textit{\hfill\noindent\small 2019/11/18 21:15 提问;2019/11/19 00:58 回答}

\noindent[4.]{\Hei 答}:我當然不會假設美國政壇和背後的財閥都是傻白甜;不過取代SWIFT這件事名正言順,美國沒辦法用民主自由那套顛覆性宣傳來忽悠其他國家,只能玩對亞洲投資銀行或者華爲那一種威脅利誘的套路,但是你看效果如何呢?SWIFT雖然是美元這個主目標之前的一個主要外圍屏障,但是世界銀行也很重要啊,美國當時還是由對財閥言聽計從的Obama主政呢,所以中共原本自己也不樂觀,采行的是姑且一試的心態。我想這次在Trump主政三年之後中俄合作,中方内部的評估應該是有相當自信的。
\\

\textit{\hfill\noindent\small 2019/11/20 00:46 提问;2019/11/20 02:45 回答}

\noindent[5.]{\Hei 答}:
Saudi太過依賴美軍,不能指望他倒戈;伊朗和委内瑞拉則早已表明立場。
現在的變數是,美國剛在本季成爲油氣净輸出國,頁岩油的開采成本還在不斷下降,而且品相是最高級的Light Sweet。未來十年,美國必將成爲各產油國的競爭對手;這有兩個可見的後果:
1)美國會更樂意在產油國製造軍事和政治動亂;
2)能源消費國的話語權會上升。
長期來説,是危險也是機遇,是福是禍要看多方短兵相交的戰果來決定。
\\

\textit{\hfill\noindent\small 2019/11/20 07:52 提问;2019/11/20 08:29 回答}

\noindent[6.]{\Hei 答}:美國人向來只優先照顧自己,別人搭順風車或因而倒霉,都不是主要的考慮。
\\

\textit{\hfill\noindent\small 2020/01/01 21:24 提问;2020/01/02 07:20 回答}

\noindent[7.]{\Hei 答}:
我在三個月前《八方論壇》講Repo的那一集節目裏,不是已經預測過全世界的中央銀行,包括人民銀行在内,都會開始降息?這是因爲在全球化的背景下,主要經濟集團的興衰都緊密相連,尤其是當前中、美、歐都在低檔掙扎,降息是正確而且必然的。
中國的經濟在2020年,不會有大起大落,基本和2019年類似,比較可能還會稍稍再減緩些。這裏的長期因素還是2009年的過度刺激,中央、地方和企業都債務纍纍,實在沒有大作爲的空間。
\\

\textit{\hfill\noindent\small 2020/04/03 00:47 提问;2020/04/03 05:20 回答}

\noindent[8.]{\Hei 答}:美國頁岩油的生產成本,從每桶\$20一直到\$100+都有,但是廉價的很少,絕大多數在\$40以上。例如在過去兩年油價\$60期間(代表著沒人會去開采成本超過\$60的油井),中位成本已經高於\$50,大部分開發商根本賺不到什麽錢,只能債上滾債,表面上越做越大,其實是越欠越多。現在企業債市已經崩潰,俄國和沙特再聯手把油價壓到一桶\$20,我估計90\%的頁岩油企業活不過兩年。聯邦政府若是出手,當然可以挽救一批,但是目前要花錢救助的對象太多,頁岩油商排隊輪著等並不是一件樂觀的事。
\\

\textit{\hfill\noindent\small 2020/04/09 23:25 提问;2020/04/10 01:51 回答}

\noindent[9.]{\Hei 答}:美元至少腰斬,財富大幅縮水,不能再肆意進口消費。但是美國本身的體量足夠、農業發達,工業的衰落也不至於到無法生產進口替代的地步,比起現在的俄國,還要強一些。
\\

\textit{\hfill\noindent\small 2020/07/10 00:43 提问;2020/07/10 05:16 回答}

\noindent[10.]{\Hei 答}:可以由俄國出面帶頭。
\\

\textit{\hfill\noindent\small 2020/08/31 12:13 提问;2020/09/03 08:05 回答}

\noindent[11.]{\Hei 答}:美元必須要被取代;這是主戰役,SWIFT只是前哨戰,否則收割和長臂管轄都會繼續下去,資助美國軍事和宣傳力量對外進行侵略和顛覆。

歐元適合作爲過渡性答案,是因爲歐盟沒有有意收割和長臂管轄的習慣,也不太可能廣汎對外軍事侵略。最終應該是要建立Keynes設想的全球貨幣,不過那無法一步到位。
\\

\textit{\hfill\noindent\small 2020/10/12 09:50 提问;2021/01/20 07:04 回答}

\noindent[12.]{\Hei 答}:世界在二戰後,長期和平,流動資本得以不斷纍積,現在早已達到百萬億美元以上的級別;這其中只有一小部分是所謂的“Smart Money”,他們在任何重大財經事件中都能先行一步,其他的未必。
\\


\section{【美國】【工業】波音衰敗之源}
\subsection{2019-12-06 05:59}


\section{2条问答}

\textit{\hfill\noindent\small 2019/12/08 14:55 提问;2019/12/09 01:53 回答}

\noindent[1.]{\Hei 答}:
是股市,但不只是股市。 
我在《八方論壇》的節目中已經評論過,這次又是一個典型的金融資產泡沫,但是比較類似於2000年那次,而不是2008年的大災難,主要的差別在於銀行業(至少是大銀行)學乖了,不但沒有加入狂歡盛宴,而且一見泡沫吹得太大,就提早收回現金入手,準備過冬。大銀行不願意繼續借錢給影子銀行,正是最近Repo Rate不斷上升的近因,遠因則是政府發債和美聯儲削減QE,使美國金融業的現金流量捉襟見肘。 
美國經濟周期内的短期榮衰,有三個決定性的支柱:金融業負責放貸,企業界負責投資,消費者則必須願意花錢;這次的問題主要出在企業上。因爲美聯儲放水太多太久,利率長期低迷,美國的企業普遍發債借貸(美國的債市發達,借錢並不一定要向銀行借),或者用來購并競爭對手,或者直接回購自己的股票,這兩個行爲同樣都會人爲地把股價推高。至於企業主管爲什麽會想要人爲地推高股價,剛好就是這篇正文要論證的。 
觀察現在的美國企業界,最重要的兩個指標是負債對收入的比例(Debt to Income Ratio,DTI Ratio),以及股價對收益的比例(Price to Earnings Ratio,P/E Ratio;Earnings就是Net Income,如果只説Income,通常指的是Gross Income,也就是稅前的總收入),它們都在2016年之後,突破合理的區間,直綫上升到泡沫級別。這就好像是山坡上一直下雪,雪崩是必然的,但美聯儲拼命在避免任何震動,我們也就無法預期什麽時候雪崩會真正發生。 
本周的失業率數字被Trump吹噓成50年來最佳,其實美國中產階級的工作早已停滯甚至萎縮,現在增加的就業都是因應消費的低級職位。但是這仍然代表著消費堅挺,而在美國經濟中,消費向來是主導;雖然長期來看,消費者的負債率也會向泡沫級別邁進,在短期卻給了美聯儲一個喘息的空間,可以針對金融界的流動性(Liquidity)問題直接對影子銀行放水,利率反而不一定要調降了。 
順便提一個題外話,“泡沫”/“Bubble”這個字是1720年南海泡沫事件(South Sea Bubble)期間被發明的,但是原本不是指整體經濟,而是針對個別公司;換句話說,它的用法原本是“Bubble Company”(“泡沫公司”)而不是“Bubble Economy”(“泡沫經濟”)。
\\

\textit{\hfill\noindent\small 2019/12/09 09:45 提问;2019/12/09 23:14 回答}

\noindent[2.]{\Hei 答}:美國的債市要分成國債和公司債來分別討論,我想你説的“美債”指的是前者,這是由美元的國際霸權來背書的,所以不是一兩個經濟周期能扭轉的。
\\


\section{【美國】【政治】美國崛起時代的治理哲學}
\subsection{2019-12-15 02:41}


\section{2条问答}

\textit{\hfill\noindent\small 2020/05/12 09:26 提问;2020/05/13 03:26 回答}

\noindent[1.]{\Hei 答}:首先,美聯儲如果需要錢,印鈔是沒有上限的,光是今年就應該會印5萬億美元,去搶中國的那1萬億並沒有什麽實質上的財政意義。

其次,國債是不記名的;中方向來也廣汎使用歐洲的中介公司,尤其是比利時和盧森堡。要搶錢有實際執行上的困難。

1971年的去金本位,違反的是美國和小弟們之間的國際條約,美國面臨的純粹是外交壓力,這對霸主來説只要臉皮夠厚就可以簡單挺過去。美國國債運行的基礎卻是契約法(例如我以前討論過Magna Carta就是特定貴族和個別國王之間的契約,可見在Anglo-Saxon法系裏,至少在理論上契約法的地位高於國家元首和政權),這不只是國内法,而且是繼承自英國的Common Law,來自千年纍積的幾百萬個判例,遠早於國會的成立。若是硬要以新的明文法條去推翻它,那麽整個市場經濟的法律根基都會受威脅。

此外,過去30年金融衍生產品大行其道,其中包含了許多Credit Default Derivatives(主要是Credit Default Swap,CDS,信用違約交換)。這些契約要求在債務人違約之後,第三方要向第四方付出巨金。如果美國向中方賴債,光是CDS市場就會天翻地覆,金融界絕不會喜歡這種必然延續多年的極大不確定性。

總之,賴國債的實際意義是對選民作秀,凸顯自身的强硬,這有很多其他不會對自身體制傷筋動骨的選項,何況連建墻都無法逼迫墨西哥直接出錢,美國的統治階級嘴皮子動完,最終還是得優先考慮金融財閥和商業資本的偏好。
\\

\textit{\hfill\noindent\small 2021/04/08 12:42 提问;2021/04/08 14:25 回答}

\noindent[2.]{\Hei 答}:上個月我不是才討論了大資本爲何願意讓律師系統來分一杯羹?法不禁止即可的説法,只不過從另一方面加强了這個互利的聯盟。

中方對社會學科的研究,真是很薄弱,除了宏觀經濟管理有些主見之外,其他英美財團用來自肥的宣傳洗腦花樣,怎麽總是照單全收呢?
\\


\section{【公共健康】【財政】一個嚴重的公共健康問題}
\subsection{2019-12-27 00:59}


\section{1条问答}

\textit{\hfill\noindent\small 2019/12/27 23:56 提问;2019/12/28 02:09 回答}

\noindent[1.]{\Hei 答}:
玉米糖漿的問題之一是它太便宜了,只有蔗糖的一半不到。加徵100\%的稅之後,它的價格與蔗糖接近,就會減少加工食品商加甜的動力。別忘了,加工食品到底用什麽成分,消費者完全無法控制。 
至於把美國玉米糖漿的市場額分轉移到巴西蔗糖上,當然不見得對代謝疾病有幫助,這只是國際市場經濟供需的考慮,所以我在正文裏沒有討論。 
冷飲的問題則在於它的攝取太容易了,尤其是在快餐店,不知不覺已經喝了300卡路里和30克的果糖;更糟糕的是果糖並不產生飽腹感,可樂裏的咖啡因反而讓你越喝越渴、越喝越餓。這是爲什麽美國人在過去40年,平均每頓飯攝取熱量不斷上升的主因;這一點我在正文裏暗示了,或許沒有說得特別清楚。
\\


\section{【台灣】台灣大選後可行的内政改革}
\subsection{2020-01-22 00:47}


\section{2条问答}

\textit{\hfill\noindent\small 2020/01/22 10:16 提问;2020/01/23 07:11 回答}

\noindent[1.]{\Hei 答}:我以前説過,小鳥的程度比起一般博主算是好的,但是他的經歷還是不夠,光凴空想總會誤入歧途。

貨幣和治理電子化之後,銀行的重要性會更大幅提升;這是因爲處理瑣碎的現金出入並不是它們的核心任務,反而是雜務。銀行的核心任務是選擇優質的企業來放貸,然後持續監督這些企業的經營。如果有明智的財政主管,就會把銀行進一步融入政府的管理回路,成爲經濟治理的基層單位。
\\

\textit{\hfill\noindent\small 2020/01/22 15:55 提问;2020/01/23 04:56 回答}

\noindent[2.]{\Hei 答}:所謂的“機會”,局部性的就是“資源”(亦即“道天地將法”中的“地”,全局性的機會是“天”),而資源和能力(“將”)的對立,正是許多重要社會議題的根本。

我們討論貧富差距、階級固化、濫用特權等等問題,其實基本上就是要排除獨霸資源以尋租的現象,容許有能力的人辦實事以推進整體福利。資本主義市場經濟有天然的擴大貧富差距的趨勢,正來自資源的稀缺性和重要性。BNP的那位首席交易員,並不是不知道我所貢獻的算法是公司盈利的關鍵,但是他認爲把我開除之後,他就獨占所有既有的資源,那麽我再怎麽有能力,要複製同樣的系統所需的大量資源也很難從陌生人獲得,他勝利的機會是99.9\%。雖然在這個例子裏,我撞上了0.1\%的好運,但是人生不是電影,運氣在現實裏是可一不可再的事,下一次我的境遇就回歸機率分佈的平均了。

資源的關鍵性,是客觀的事實。控制資源本身並不邪惡,反而是創造社會新福利的必要元素;邪惡的是獨霸資源來尋租,因爲它打擊新生的、更高效的競爭對手,以損害社會整體為代價來謀求小集團利益的最大化。我們討論不同的政治體制,正是要找尋能鼓勵把資源分配盡量做到有利於全社會的制度。

美國商學院的全套學説,其實濃縮到核心,就是教人怎麽爭奪然後獨占資源來尋租。這是他們對美國文明腐化的貢獻。

美國在雷根之後的政治、外交、經濟、宣傳體系由富豪掌控,基本上就是把商學院那一套應用到全球治理。商業文化不應該被推廣到文明社會的所有層次和方向,是這裏的基本教訓。

中共的人事體制雖然絕非完美,但是相對於美國和台灣來説,有一個很大的優越性,就是它並不主要依據資源獨占的程度來提拔人才,因爲每個地方官都有同一級別的資源,成就上的差別就能反映出個人的能力。

中國商界的成功人士,如同美國的一樣(如Musk),絕大部分也是撞上或騙到關鍵性的資源;其實真正辦事的,是大衆所不知的幹部(如Musk手下的火箭專家)。企業家對公司和經濟的真正貢獻,是識別並扶持有能力的幹才,把稀缺的資源做最高效的再分佈,這是市場經濟的正常運作,沒有什麽不好。學術界也有類似的現象;在找到更好的資源分配體系(例如E-Government)之前,沒有必要過度抨擊。
\\


\section{【科研】【媒體】新冠病毒和媒體亂象}
\subsection{2020-02-02 06:53}


\section{1条问答}

\textit{\hfill\noindent\small 2020/03/24 11:11 提问;2020/03/25 11:39 回答}

\noindent[1.]{\Hei 答}:我在半年前預測會有經濟衰退時已經説過,這一輪QE會是美國最後一次利用美元來做全球搜刮。換句話説,在下一次經濟衰退(大約2030年)之前,美元的國際儲備貨幣地位會被推翻。最近的發展,完全是依照我所説的劇本來進行的。
\\


\section{【工業】【能源】再談氫經濟}
\subsection{2020-02-14 01:07}


\section{1条问答}

\textit{\hfill\noindent\small 2021/08/22 04:01 提问;2021/08/25 04:27 回答}

\noindent[1.]{\Hei 答}:有關AI用在互聯網對整體經濟無益的討論,的確沒有出現在正文,但在留言欄我一直支持做這種論斷的讀者。

美國的經濟“奇跡”,其實300\%是印鈔票買的。這裏有一個巨觀經濟學的簡單概念,叫做“Velocity of money”,亦即多注入一美元現金,能產生的額外GDP其實是好幾倍(以往至少是3倍,但最近兩年實在印的太多,超出實體和虛擬經濟的容納量,所以我估計大約已經略低於2倍);例如金融巨鰐從美聯儲拿到錢去買游艇,游艇商賺了錢又可以去買Ferrari,汽車經銷商賺了錢又去買別墅,等等。美聯儲在過去兩年每年印3萬億,相當於GDP的15\%,但考慮Velocity of money大約是2,灌水效應其實應該是讓GDP浮腫30\%,而今年的實際成長率遠遠不到10\%,所以我說這些GDP成長,300\%是印鈔票買的。
\\


\section{【科研】流行病的起源(上)}
\subsection{2020-04-02 16:53}


\section{1条问答}

\textit{\hfill\noindent\small 2020/04/18 23:17 提问;2020/04/19 06:56 回答}

\noindent[1.]{\Hei 答}:理論上,股市應該反應6-9個月後的經濟狀態,但是實際上當然可以有人爲的操弄,尤其美聯儲有近乎無限的鑄幣權力,目前大家預期今年會有大約5萬億美元(相當於25\%GDP)的注入,這都必須流入金融市場,而股市是其中流動性最高、相對於債券資產更實在的去處,所以吸收了大部分的新錢很自然。

未來三季,關於實體經濟的壞消息會不斷確認,財政和貨幣刺激的效應卻會遞減。請你稍安勿躁,等到年底再來復盤。而且這還只是第一階段;1990年日經指數從歷史高峰開始崩潰,經歷了三個斷層式下降才真正觸底:第一階段花了一年,下一個谷底在1992年,最低點則到2003年才完成。我認爲這一波美股崩盤,至少也要三年才會利空出盡。
\\


\section{【醫療】我的新冠經驗}
\subsection{2020-04-27 08:38}


\section{5条问答}

\textit{\hfill\noindent\small 2020/05/04 06:54 提问;2020/05/04 10:05 回答}

\noindent[1.]{\Hei 答}:你一次問了兩個很大的問題。

首先,在重要的產業升級方向,國外既有的領先企業占據了長時間、大投資所獲得的技術纍積、人力資源和市場額分,可以很簡單地承受短期的殺價競爭,遏制新對手於襁褓之中,所以任何以利潤為導向的新來者,基本不可能成功;這其實正是爲什麽英美的資本普選教對外也要强調自由主義經濟理論的主因。所以在高科技工業上放任地方政府向國營銀行借錢搞自負盈虧,注定會人、財、技三失。要成功,只能以長期立足市場為唯一目的,集中授權並投資給有理想、有能力、有誠信(尤其如果“自主可控”都是哄人的,那麽事先就很明顯沒有什麽誠信可言)的主管專心致志地去建設成長,至於國營或是民營反而不是決定性的因素,例如高鐵、核能是國營,京東方是半國營,華爲是民營,他們的共同點是都以公司的技術能力和長期成長為第一優先考慮。相反來看,公營的和私營的一樣都可以短視近利,以幫助主管們大撈一票為核心目標,前者如貴州和高通搞的幾個合資企業,後者如聯想,除了成就出一批黃皮膚的國際富豪之外,完全沒有其他的意義。

至於美聯儲大印鈔票的事,你要記得經濟危機越嚴重、全世界對國際儲備貨幣的需求量也越高,所以一年印5萬億(我的預估)雖然在幾個月前還是駭人聽聞,美元體系並不會因此而自行崩潰。要推翻美元,最終還是得靠被剝削的其他國家聯合起來反抗虐政,共同發行有替代性的新國際貨幣。這件事我們也是討論了很多年,我個人覺得人民銀行可以更積極些,對内對外都應該加速推行加密貨幣,尤其是金磚貨幣。不過中國的貨幣管理人才,素質還是要遠高於宣傳公關,或許他們内部有合理的疑慮和爭議。我們再等兩年看看是否會有所行動。
\\

\textit{\hfill\noindent\small 2020/05/21 00:00 提问;2020/05/21 03:49 回答}

\noindent[2.]{\Hei 答}:美國的財富基本是私有,而私有財富絕對集中在富豪手裏。但是基尼係數因爲是歷史上用來測量貧富不均的主要指數,所以美國的學術界和智庫早就上下其手,對它徹底按摩過了,就等著國内外天真爛漫的象牙塔居民來引用。你只要拿它和近年一些獨立做出的新統計(例如MIT的0.01\%富豪財富占比)做比較,就能馬上看出它不對頭。

印度的GDP成長率還連年高於中國呢,歐美的公共衛生防疫能力更在紙面上高出中國、越南和南非幾個數量級。盡信官方數字的不是在做研究,而是做撒謊者的傳聲筒。
\\

\textit{\hfill\noindent\small 2020/05/21 01:24 提问;2020/05/21 04:07 回答}

\noindent[3.]{\Hei 答}:自從Milton Friedman重啓美國經濟學界的學術賣身(Academic Prostitution)業務之後,這些統計結果越來越只爲政治服務,自相矛盾非常嚴重。個別數字大家姑妄看看就好,要評估真相還是必須先綜合多個來源做出嚴謹分析。
\\

\textit{\hfill\noindent\small 2021/02/11 09:34 提问;2021/02/16 09:35 回答}

\noindent[4.]{\Hei 答}:我在博客已經反復解釋過了(參見《當前世界的公共衛生危機》、《現代醫療的大倒退》、《有關環保和全球暖化的幾點想法》、以及對新冠疫情的一系列討論),資本主義體制追求的是企業利潤最大化,而不是社會公益的最大化,所以絕對需要政府的嚴格監管和扶正。

這裏問題最明顯的行業有兩大類:1)把私人開支轉嫁為社會成本的,經濟學上叫做Exogeny,例如污染排放、全球暖化等等。這一類,美國經濟學界很願意討論,並且建議用市場化來解決,像是碳排放交易,名義上是把社會成本計入賬目,實際上是找藉口為金融巨頭發明新的游戲平臺。這是因爲這些問題的真正關鍵在於準確監察記賬,市場化根本與其無關,也就是整個課題上的實際工作依舊留給政府,收錢發合格書的權利卻分割出來交給財閥,是典型的芝加哥學派把戲。

2)行業先天就有比利潤遠遠更重要的服務目標,例如軍事是爲了保護國家、醫療是爲了救治人命、法律是爲了維護公義、教育是爲了下一代的心智。這類矛盾基本無法解決,所以芝加哥學派只能避而不談,堅持市場化只能靠宗教式迷信,在英美之外的忽悠就沒有像碳排放交易那麽成功,把軍隊、醫院、警察和學校全面私有化仍然是很極端的政見。不過正因爲有著基本而且絕對的矛盾,即使只是部分私有化危害也極大,必須靠知識分子群起而攻之。我已經把道理講得很清楚,傳播出去是你們的責任了。
\\

\textit{\hfill\noindent\small 2022/01/15 21:35 提问;2022/01/16 14:59 回答}

\noindent[5.]{\Hei 答}:典型的猶太家族企業,錢賺飽了,留下公司金蟬脫殼,然後和聯邦檢察官達成默契,交出2\%左右的贓款換取全部高管免責,和波音的故事如出一轍,唯一的差別在於忘了收買法官,結果後者在去年底宣稱和解條件太離譜,拒絕批准。但是不要指望真能把他們繩之以法,頂多是再多交點罰款,更可能直接換掉法官。

你們以爲我寫《美國式的恐龍法官》是挑選特例嗎?博客這裏從來只有懶得重複,不會試圖誤導讀者:我舉出的每一個案例背後,都有成千上百個類似的現象。


有關和解條件的新聞報導,因爲是典型的美式公關,一般讀者很可能會被其中所含的烟幕所矇騙,所以我想解釋一下,爲什麽我說“2\%”。

這裏的分母有三種選擇:第一是國家經濟所受的損害,這難以精確定義,所以我沒有采用,不過美國任何一個試圖估算的經濟學人,得到的結論都超過10000億美元。其次是Purdue Pharma賣了幾十年OxyContin的總收入,這大約是370億美元。第三個,也是最保守的算法,是Sackler家族從Purdue獲得的財富,至今還剩下140億美元。至於分子,所有的美國媒體都說罰款是45億美元,這是假新聞!因爲這其中只有2.35億來自Sackler家族,其它是Purdue Pharma的責任,而後者早已被掏空,甚至都已經申請破產,根本不可能交出40多億美元。所以罰款比率即使最誇大也只有2.35/140=1.7\%。
\\


\section{【外交】【經貿】後新冠世界(一)}
\subsection{2020-05-20 09:36}


\section{10条问答}

\textit{\hfill\noindent\small 2020/05/20 12:29 提问;2020/05/21 02:51 回答}

\noindent[1.]{\Hei 答}:Trump的中美經貿脫鈎其實有兩個方向:首先希望外包出去的中低端製造業回流,這純屬癡人説夢,頂多就只能强逼組裝廠轉移到其他國家;其次是利用美國占據的產業鏈最上游來打擊新興的中國高科技公司,這是自損八百來傷敵一千的做法,中方又被地方政府以及商界和學術界的帶路黨坑矇拐騙了幾十年,準備得很不充分,所以會是較大的麻煩。可以考慮整合全國同行的所有企業,對美方限賣產品中有可替代的,統一禁用,例如高通的CPU。
\\

\textit{\hfill\noindent\small 2020/05/20 13:12 提问;2020/05/21 03:12 回答}

\noindent[2.]{\Hei 答}:英國的製造業,因爲起源太早、獨霸太久,其實效率一直很差,小作坊非常普遍,極其依賴廉價勞工,所以德國一工業化,馬上在生產效率上超趕英國,在1870年代來自德國的進口工業品已經汎濫成災。一開始還可以假裝德國產品品質低,所以搞了個法案要求進口商品標明生產國(“Made in Germany”),但是很快連這個藉口都沒有了,於是只能玩政治,在原材料和市場上封鎖德國。

所謂的自由市場經濟,從來在實踐上就沒有真正成功過。效率最高的經濟,如果不是由國家來組織生產力(如德國、日本、中國),就必須靠大財團來執行這個統籌規劃的職能(如美國和南韓)。我在批判美式經濟學的時候,重點之一就是指出大企業對他們的經濟運作至關重要,然而企業的内部卻是絕對極權的反市場、反民主機制。
\\

\textit{\hfill\noindent\small 2020/05/20 13:15 提问;2020/05/21 05:52 回答}

\noindent[3.]{\Hei 答}:在1929年後的大蕭條期間,全球表現最佳的主要經濟體就是蘇聯;1933年,小羅斯福和希特勒分別在美國和德國掌權,基本模仿蘇聯運用國家力量來重整經濟,差別只在美德兩國沒有整肅資本家,而是强迫他們為國家政策服務。這三個國家成爲二戰的主角,經濟力量在戰間期後段能迅速復蘇是主因之一。

然而國家統籌並不代表官僚獨大,至少在和平時期來自市場的反饋仍然必須被理性客觀地考慮和回應(說得明白點:不能任由市場以追求最大利潤爲唯一目的來制定決策,但是市場做爲反應經濟客觀事實的信息通道是極爲重要的)。Stalin沒有理解到這一點,把經濟管理鎖定在戰爭體系,長久下來種下蘇聯解體的禍根。
\\

\textit{\hfill\noindent\small 2020/05/21 00:51 提问;2020/05/21 04:00 回答}

\noindent[4.]{\Hei 答}:不只是美國人懊悔不已,歐洲人也有同感。

中國的體量大、制度效率高,不但有最强的能力來挑戰先進工業國,挑戰成功之後造成的衝擊也最大;這是中國和歐美的基本矛盾。還好這個矛盾是過去式,只有在能時空穿越囘到2、30年前的前提下才有意義。中方在和歐洲,尤其是法國溝通的時候,必須解釋中國的崛起是既成事實,其他第三世界國家的工業化也不可逆轉。要制定未來政策,必須以今日的現實為基礎。事實上要强求中國倒退成爲落後國家,不但不可能成功,而且歐洲也必須付出很大的代價。接受過去、立足現在、展望未來,才是成熟明智的選擇。
\\

\textit{\hfill\noindent\small 2020/05/23 11:53 提问;2020/05/24 03:30 回答}

\noindent[5.]{\Hei 答}:美元被美國政府多方濫用,在近年已經走向極端,如果用戶有自由來選擇,早就換用其他貨幣。但是這裏不但有店大欺客(一個供應商對應幾十億用戶)的問題,還有規模、體系、習慣、方便等等因素。簡單來説,美元作爲國際儲備貨幣,是人類史上最大的托拉斯。偏偏這個世界沒有全球政府,國際法是由霸主説了算,所以這個托拉斯不但不受法律規範,反過來還有著絕對的政治、外交和軍事上的鞏固支持。

要打破這個托拉斯,必須正面和側面多方向進攻。我在考慮是否寫一篇博文來專門討論。
\\

\textit{\hfill\noindent\small 2020/05/24 09:49 提问;2020/05/24 11:06 回答}

\noindent[6.]{\Hei 答}:法國人喜歡反權威,源自兩百多年前的法國大革命事後被神化。其實它是東施效顰:美國大地主搞革命是爲了擺脫英國控制,獨占整個大陸的資源和財富;巴黎的一些窮人跟著亂起鬨,最終還不是便宜了軍閥?

法國人的教育體系與衆不同,特別强調抽象哲學和標新立異,不過至少階級特權問題沒有英美嚴重。

澳洲是中國崛起過程中,獲益最大的國家。這是因爲它在上世紀末去工業化之後,經濟產出集中到農礦,剛好和中國互補。問題是澳洲的右翼民粹比英美還厲害,中國以爲能夠在商言商、單純經貿互利,卻沒有預料到澳洲紅脖子會得意忘形,對内搞種族歧視,對外是帝國主義。繼續扶持巴西、阿根廷這類替代供應商是絕對有必要的。
\\

\textit{\hfill\noindent\small 2020/05/31 05:01 提问;2020/05/31 14:38 回答}

\noindent[7.]{\Hei 答}:希望如此吧。

歐洲人心思變越來越明顯,我已經寫了多篇文章和評論來提倡中歐聯盟;但是如果我是習近平的幕僚,其實會進一步建議簽署中方對歐洲完全開放服務業以交換歐盟對中國開放工業品進口的全面自貿和合作協定。中國從美國的服貿進口接近600億美元,來自香港的超過1000億美元,加起來剛好彌補歐盟對中國的貿易逆差。

不過不論中歐能馬上談出多少協議,我想中芯的光刻機在習近平訪歐之後會有轉機,畢竟歐盟和美國切割、向中國靠攏,Trump限制半導體生產設備外銷,等同是自我割喉,AMAT、KLAC的股價要完了。
\\

\textit{\hfill\noindent\small 2020/07/14 02:41 提问;2020/07/14 05:06 回答}

\noindent[8.]{\Hei 答}:你不用客氣;Dunning-Kruger微笑曲綫的最低點,其實一般就在博士班程度;換句話說,在拿到博士學位之前,學得越多,對自己的無知認識越清楚,要當上教授之後,才能慢慢重建自信。

我覺得美國體制先天效率很低,但是穩定性很高,出現社會秩序完全崩潰的機率很小。疫情如果繼續惡化,真正須要擔心的還是經濟。美聯儲能做的都已經做了,Trump也不吝於花費公帑來刺激消費,但是一方面新冠對經濟的打擊是全球性的,另一方面美國的GDP太依賴零售,所以五月和六月的復蘇趨勢基本不可能長期持續。我認爲有儲蓄的中產階級所面臨的最大危險還是就業。

在公共衛生方面,Trump的確是在向瑞典式的群體免疫策略靠攏。還好目前維生素D對免疫的重要性已經被一系列雙盲實驗反復確認,甚至在新冠的季節性和黑人易感性上都很可能有重要影響,我建議你開始每天服用,輔以抗氧化劑(如NAC)和抗凝血劑(如Aspirin),並且保持睡眠充足,應該可以把家人的生命危險降到最低。

至於高等教育的因應措施,我其實一直在很密切地注意著。知名大學最擔心的,是如果全部改爲網課,許多新生會申請延一年入學,那明年基本不用招生了。所以很普遍的一個方案是只讓一年級新生進駐校園,高年級生才以網課爲主。Trump政權原本要求留學生的網課不超過3學分,幾天後就退讓為有3學分不是網課;我想這已經很容易繞過去,大部分的留學生不會因此而停學,只是教書的要辛苦些。
\\

\textit{\hfill\noindent\small 2020/07/20 00:09 提问;2020/07/20 00:46 回答}

\noindent[9.]{\Hei 答}:光是一方出手沒有用,因爲1)中國仍然必須有足夠的美元外匯儲備來防範金融打擊;2)只要美元還是霸主,它就會比歐元强勢。從數學的觀點,美元霸權是一個局部最優解(Local Optimum),要過渡到全局最優(Global Optimum)必須先通過一個很大的障壁(Barrier),只有大家約好,一同發力,才能成功。
\\

\textit{\hfill\noindent\small 2020/08/13 10:24 提问;2020/08/14 01:21 回答}

\noindent[10.]{\Hei 答}:哪裏有不小的政策空間?一個新興工業國家怎麽能引用夕陽西下的金融帝國的負債標準?

如果其他工業國今年(指未來的那5個月的同比數字)的成長率是-4\%,那麽中國完全可以接受5\%。如果中國能進一步選擇自己成長4\%而他國是-6\%,絕對也是划算的。
\\


\section{【金融】【戰略】後新冠世界(二)}
\subsection{2020-05-27 05:36}


\section{24条问答}

\textit{\hfill\noindent\small 2020/05/27 22:39 提问;2020/05/28 14:33 回答}

\noindent[1.]{\Hei 答}:右翼民粹太蠢,根本不知道珍惜美元的金融霸權,反而主動削弱它的根基,所以我才會建議現在就出擊。否則以美元既有的絕對優勢,小心維護之下,要撐到2030年之後簡單之至。
\\

\textit{\hfill\noindent\small 2020/05/28 22:42 提问;2020/05/29 11:47 回答}

\noindent[2.]{\Hei 答}:我想Powell自己都不能確定撒錢的功效和報應有多强,但的確無可選擇,即使是毒藥也得先解渴再説。

人類歷史其實從來沒有過一個只占全球貿易額10\%,卻擁有70+\%的貨幣霸權的國家如此大印鈔票,不過可以簡單看出任何惡果都會被冲淡至少7倍,所以要看自作自受還是先把美元打下神壇最保險。
\\

\textit{\hfill\noindent\small 2020/05/28 23:01 提问;2020/05/29 11:50 回答}

\noindent[3.]{\Hei 答}:這個比喻沒有考慮到貨幣使用上的網絡效應。

其實國際儲備貨幣的自我加强作用比Windows OS還要強:當經濟危機到來時,儲備貨幣最能保值,作業系統還沒有這個功能。
\\

\textit{\hfill\noindent\small 2020/05/29 12:43 提问;2020/05/29 13:17 回答}

\noindent[4.]{\Hei 答}:財閥花了半世紀,在美國建立了他們的人間天堂,現在這種全球搜刮的優勢被中國威脅了,中方能管住自己的資本家就很了不起了,哪可能指望對方棄暗投明?
\\

\textit{\hfill\noindent\small 2020/05/29 18:45 提问;2020/05/30 03:34 回答}

\noindent[5.]{\Hei 答}:資本的特徵之一就是短視近利,指望他們有大局觀非常不切實際。這正是因爲自由市場是自我主義的實現,一遇到囚徒困境,立刻把集體利益抛諸腦後。
\\

\textit{\hfill\noindent\small 2020/05/30 01:46 提问;2020/05/30 03:52 回答}

\noindent[6.]{\Hei 答}:這是你在這個博客留下的評論最好的一個,原因是你自已做了獨立的邏輯分析,而不是拿書本上的理論來硬套。

最近有不少人討論新冠會帶來通縮還是通脹,這當然是連前提都沒搞清楚的無意義爭議。我在四五年前已經特別寫了博文解釋過,討論通脹/通縮必須指明資產的類別和性質,工資和原材料價格沒有必要走同一個方向,工業成品和金融資產更完全是兩回事。

一個嚴重的經濟危機自然會壓低大部分實業的上下游價格,然而量化寬鬆卻會捧高金融資產,所以必然會加劇貧富不均。2009年中國政府的四萬億財政刺激,最大的毛病就出在沒有考慮這一點,容許這些國家給的資金去追逐金融資產,尤其是房地產。希望他們這次能換上有點主見的經濟管理人員,而不只是照搬美式經濟學理論;畢竟美國人作死胡搞,有美元托底,中國可沒有這個餘裕。
\\

\textit{\hfill\noindent\small 2020/05/30 15:19 提问;2020/05/31 01:01 回答}

\noindent[7.]{\Hei 答}:我在美國金融界任職期間,經歷了從T+3改爲T+1的過程,覺得它對割韭菜沒有太大的影響,只是稍微方便了中頻炒作(買進/賣出發生在同一天叫高頻,這裏指買進/賣出隔了幾天的交易模式)。

真正必要的監管手段,是一旦發現有假造數據,就把公司高管和大董事全部終身禁足於公共持股的企業。這和整頓學術界是一樣的道理,做過一次賊、一輩子都會是賊,只要有一篇論文造假,就應該踢出學術界。
\\

\textit{\hfill\noindent\small 2020/05/31 15:43 提问;2020/06/01 01:18 回答}

\noindent[8.]{\Hei 答}:當年願意幫助Milton Friedman為財閥洗白的“經濟學家”,光是薪水就多出兩三倍,然後還有許多企業界的顧問職可以輕鬆地把收入再乘上整數倍,這是事先就做的抉擇,哪有什麽後悔可言?

你拿Einstein和一群自願賣身的學術娼妓做比較,實在不合適。
\\

\textit{\hfill\noindent\small 2020/06/02 13:45 提问;2020/06/02 14:20 回答}

\noindent[9.]{\Hei 答}:美元霸權還在,所以一有全球性經濟危機就有大筆資金轉換為美元避險,然後各國央行都面臨擠兌的問題,外匯存底不一定夠用,只有美國的親密盟友才能和美聯儲做互換。換句話説,美國通過美元來壓榨其他國家,如果稍擡貴手、少刮一點,你還得跪下謝恩,這就是國際儲備貨幣的特權。
\\

\textit{\hfill\noindent\small 2020/06/15 10:19 提问;2020/06/15 12:10 回答}

\noindent[10.]{\Hei 答}:過去三個月,國債增加了2.5萬億美元,企業債增加了7000億美元,股市反而反彈了,七月到期的企業債能有多少?只要美元霸權還在,沒有什麽問題是美聯儲不能靠印鈔解決的。
\\

\textit{\hfill\noindent\small 2020/07/08 15:24 提问;2020/07/08 23:34 回答}

\noindent[11.]{\Hei 答}:
飲鴆止渴,旁觀者看來非常不理性,但是當事人就比較難以剋制自己的衝動。 
中方主管單位一直明白匯率以平穩爲上,但是管股市和管貨幣從來不是同一個班底,前者並不瞭解平穩原則放諸金融市場皆准,現在有了刺激消費的需要,就急著先解渴再説,重蹈2015年的覆轍。這是美式經濟學在中國污染了30年人心的後果,欠下的債總是要還的。
\\

\textit{\hfill\noindent\small 2020/08/28 04:43 提问;2020/09/03 08:12 回答}

\noindent[12.]{\Hei 答}:股票市場試圖預測的,是公司未來一兩年的盈利,這已經和經濟的長期健康狀態不完全吻合;波音是一個很好的例子。

上述理論的隱性前提是,投入股市的資金總量被控制在理性平衡值,這基本上從來不可能實現。現在的情勢,只是美聯儲史無前例地防水,把那個前提完全推翻,所以股市自然和經濟徹底脫節。
\\

\textit{\hfill\noindent\small 2020/09/28 20:36 提问;2020/09/29 02:29 回答}

\noindent[13.]{\Hei 答}:這位“大師”的説法和我的經驗認知南轅北轍。當然我只在20、21世紀的美國住了30多年,既沒有時空穿越和無中生有的能力,也從未在修道院呆過,如果他討論的是14世紀的歐洲我無法置評。

國内外雙循環是應對美國經貿脫鈎的政策方案之一,另一個反應是對美國高科技產品做針對性的替代;這是我多年來反復預警的事,Better late than never。
\\

\textit{\hfill\noindent\small 2020/11/16 04:13 提问;2021/01/20 06:21 回答}

\noindent[14.]{\Hei 答}:小程度上有幫助,但不足以解決問題。自由市場先天是贏者全拿,像是馬雲這樣憑著運氣和不完全合法的手段,迅速纍積了極大的財富,不但普遍,而且處理起來是很棘手的。
\\

\textit{\hfill\noindent\small 2021/01/23 03:15 提问;2021/01/23 07:17 回答}

\noindent[15.]{\Hei 答}:我素來不喜歡討論投資事宜,因爲金融市場其實是高度非理性的,價格回歸實際價值可能需要幾十年,例如美元在50年前就放棄金本位,開始無限印鈔,到現在還是國際儲備貨幣;又如Bitcoin,内含價值是零,但在美聯儲極限放水的背景下,很自然地不斷上漲。所以不要從大戰略考慮來做投資決定;至於是否回國,則看你個人身邊的小環境。
\\

\textit{\hfill\noindent\small 2022/01/10 06:07 提问;2022/01/10 06:17 回答}

\noindent[16.]{\Hei 答}:現代貨幣理論的鼻祖是Milton Friedman---enough said.
\\

\textit{\hfill\noindent\small 2022/01/10 08:48 提问;2022/01/10 10:32 回答}

\noindent[17.]{\Hei 答}:要提升研發效率,整頓學術風氣是第一,投資義務教育是第二,發福利給年輕人,連前一萬都排不上。
\\

\textit{\hfill\noindent\small 2022/09/06 01:38 提问;2022/09/06 06:33 回答}

\noindent[18.]{\Hei 答}:他説的大致正確;不過政府兩年前就明白其道理,並且出手整治。博客的任務在於引領思潮,主流輿論理解了的意見,就沒有必要再精細論證;官方已經采納的政策,更是連提都不用提。
\\

\textit{\hfill\noindent\small 2023/02/18 00:48 提问;2023/02/21 03:03 回答}

\noindent[19.]{\Hei 答}:中國的經濟學界,在所有社科行業之中,顯然是水平最高的。即使在外交戰略議題上有同樣的天真趨勢,但最起碼在1990年代就拒絕全盤接受華盛頓共識,為人類留下一個能證明昂撒謊言的希望種子,是非常驚人的先見之明。00年代社會思想混亂,經濟系難免也出現雜音,但並沒有整體屈服。10年代國家對美國發動新冷戰後知後覺,是大政略上的認知錯誤,並不是經濟專業的特有責任。

30多年前我還只是個年輕學生的時候,曾經迷過《銀河英雄傳説》。看到楊威利攻占Iserlohn要塞(宇宙曆七九六年的第七次Iserlohn戰役)過程中,帝國艦隊指揮官當斷不斷,任由楊威利自由行動,等到大勢已去才無謂送命(換句話説,這位蠢蛋完美地展現了“戰略定力”和“傳統智慧”),覺得實在蠢得太過,小説家描述失真。後來年紀大了,見識廣了,才明白這是人性之常:正道説明白之後,固然會讓人覺得極爲簡單自然,但若沒人指點,絕大多數人一輩子都無法自己想通。過去十幾年的中美關係,尤其經濟貿易戰綫上的指導原則,也是如此;讀完博客覺得理所當然的邏輯,其實遠超一般學人思維所能及。
\\

\textit{\hfill\noindent\small 2023/11/15 11:32 提问;2023/11/15 14:38 回答}

\noindent[20.]{\Hei 答}:我對台灣的數據沒什麽研究,純粹從身邊日用品來觀察:最近一年的通膨大約與今年年初的美國相當,當時美國的官方通脹率在造假之後還有5\%上下。所以官方數字說今年的通膨是1.9\%,顯然是相當離譜的。
\\

\textit{\hfill\noindent\small 2023/11/15 19:25 提问;2023/11/19 21:48 回答}

\noindent[21.]{\Hei 答}:
是的,台灣政經體系的最主要剝削,就是直接針對一般消費者;這與美國體系金融剝削的間接性有所不同。
\\

\textit{\hfill\noindent\small 2023/11/25 01:37 提问;2023/11/25 01:37 回答}

\noindent[22.]{\Hei 答}:
中國的經濟學界,在所有社科行業之中,顯然是水平最高的。即使在外交戰略議題上有同樣的天真趨勢,但最起碼在1990年代就拒絕全盤接受華盛頓共識,為人類留下一個能證明昂撒謊言的希望種子,是非常驚人的先見之明。00年代社會思想混亂,經濟系難免也出現雜音,但並沒有整體屈服。10年代國家對美國發動新冷戰後知後覺,是大政略上的認知錯誤,並不是經濟專業的特有責任。 
30多年前我還只是個年輕學生的時候,曾經迷過《銀河英雄傳説》。看到楊威利攻占Iserlohn要塞(宇宙曆七九六年的第七次Iserlohn戰役)過程中,帝國艦隊指揮官當斷不斷,任由楊威利自由行動,等到大勢已去才無謂送命(換句話説,這位蠢蛋完美地展現了“戰略定力”和“傳統智慧”),覺得實在蠢得太過,小説家描述失真。後來年紀大了,見識廣了,才明白這是人性之常:正道説明白之後,固然會讓人覺得極爲簡單自然,但若沒人指點,絕大多數人一輩子都無法自己想通。過去十幾年的中美關係,尤其經濟貿易戰綫上的指導原則,也是如此;讀完博客覺得理所當然的邏輯,其實遠超一般學人思維所能及。 
 (2023/02/21 03:03)
\\

\textit{\hfill\noindent\small 2023/11/25 01:38 提问;2023/11/25 01:39 回答}

\noindent[23.]{\Hei 答}:
我對台灣的數據沒什麽研究,純粹從身邊日用品來觀察:最近一年的通膨大約與今年年初的美國相當,當時美國的官方通脹率在造假之後還有5\%上下。所以官方數字說今年的通膨是1.9\%,顯然是相當離譜的。 
 (2023/11/15 14:38)
\\

\textit{\hfill\noindent\small 2023/11/25 01:40 提问;2023/11/25 01:40 回答}

\noindent[24.]{\Hei 答}:
是的,台灣政經體系的最主要剝削,就是直接針對一般消費者;這與美國體系金融剝削的間接性有所不同。 
 (2023/11/19 21:48)
\\


\section{【邏輯】一個重要的統計悖論}
\subsection{2020-07-09 11:49}


\section{1条问答}

\textit{\hfill\noindent\small 2020/07/12 00:22 提问;2020/07/12 03:50 回答}

\noindent[1.]{\Hei 答}:Simpson's Paradox比你想的更複雜、深刻,它的根源在於統計只告訴你相關性(Correlation),我們真正有興趣的因果關係(Causation)卻必須依靠其他的考慮來確定。一般被研究的社會現象有著許許多多的維度,到底哪一個是因、哪一個是果,不能簡單論斷;即使把所有的變數依照Correlation的强弱一一列出,也只是減小搞錯因果關係的機率,事實上完全可能有某一個與最終結果無直接因果關係的變數,通過與幾個真正因子的聯動而獲得最大的Correlation。不過這已經考慮得太深入了;近年一般社會科學的論文基本上都是只搜集關於自己有興趣的單一變數的資料,然後做個簡單的綫性回歸(Linear Regression)分析,這樣天真的研究幾乎必然是錯的。一個典型的例子是三個月前《經濟學人》拿新冠死亡率和“民主程度”來做統計,發現有正向的相關性,然後斷言民主自由有利於防疫,就是忽略了富裕程度這個主變數,所以得到是非顛倒的答案。可笑的是如果他們本月再重做一次,連富裕程度都無須考慮,美國、巴西、印度等“民主國家”一樣名列前茅;所以他們必然不會重做,做了也不會登出來。

科學不是題材,而是態度;詳細來説,是人類總結用來求真最有效的一系列方法和原則。現在的中醫學院拿做生意廣告的態度來搞中醫,那自然是成爲宗教而不是科學。
\\


\section{【歷史】【戰略】希特勒的戰略選擇}
\subsection{2020-07-14 11:29}


\section{2条问答}

\textit{\hfill\noindent\small 2020/07/14 23:59 提问;2020/07/15 04:01 回答}

\noindent[1.]{\Hei 答}:德國財政混亂的主要後果,還是在於欠缺外匯;對内納粹早就全面管制經濟,工資和商品價格都是公定,就像中共在改革開放前一樣,所以赤字不是緊急的問題。

“一國的戰略綱領豈能只靠大規模的搶劫來解決?”這是中國文化的看法,西方剛好相反,從大航海到殖民時代到20世紀三次全球戰爭,强權的興起基本就是靠大規模的搶劫,内部的改革建設只不過是用來支持對外搶劫罷了。我在正文解釋了,希特勒其實是英國式殖民主義的忠實信徒,所以他準備依靠搶劫來解決經濟財政問題是很自然的。現在的美國不也是在對日本和蘇東吸血之後食髓知味,想要對中國和歐盟搞同樣的掠奪肢解嗎?

你如果回去看二戰期間的第一手資料,就會發現在德軍決策階層做討論的過程中,希特勒居然還是唯一在乎經濟、資源和後勤問題的人。那些名將只會打仗,在經濟上純屬文盲。例如在戰後,Halder和Guderian還寫書說Operation Barbarossa沒有集中所有力量在中路進攻莫斯科,是整個計劃失敗的主因,這真的是顛倒是非。希特勒發動對蘇戰爭的戰略目標就是南路的烏克蘭和高加索地區,打下莫斯科一點用都沒有,拿破侖的前車之鑒還歷歷在目。

實際上,Halder也做了手脚,把希特勒原本計劃集中在Army Group South的兵力偷偷轉移了一個Panzer Army給中路,所以戰事前期南路在名將Rundstedt的指揮下依舊進展最慢(另一個原因是他面對的Southwestern Front西南方面軍司令Kirponos是所有蘇軍前綫指揮官之中最能幹的)。還好史達林很理性地預期德軍會以南路爲主攻重點,所以一開始在中路準備不周,必須緊急把戰略預備隊全部投入,然後Zhukov跑到Kiev去監督Kirponos,强迫後者從機動防禦轉爲全面反攻(有可能是揣摩史達林的旨意),結果正如Kirponos事先警告的,德軍得以中央突破,把西南方面軍撕成兩半,這時希特勒終於注意到Halder的花樣,緊盯著他(希特勒在此前對手下授權很放任)把裝甲兵團送囘南路,剛好形成鉗形包圍,70萬蘇軍中60多萬被俘虜,Kirponos戰死,德軍歪打正着,反而比原本的計劃還要有效,這才把戰事拖到1942年決戰Stalingrad。

但是蘇聯在1939年到1941年兩年期間是德國的原油主供應商,他們當然知道後者的窘境。Timoshenko元帥在一開戰(當時是Stavka長官,也就是參謀總長)就强調,蘇軍的總戰略在於掐死德國的石油供應。後來Stalingrad之所以成爲兩軍會戰的焦點,也在於石油;這是因爲蘇聯的高加索原油必須經由Volga River運輸往後方,而Stalingrad就位於Volga River最靠德軍方向的突出部,因此德國要切斷蘇聯的原油供應,必須拿下Stalingrad。有英美的“歷史學家”宣稱希特勒爲了那個城市的名字而落入陷阱,這是胡扯;希特勒的戰略定力有問題,但是整體層次還是遠超這些“專家”所能想象的。
\\

\textit{\hfill\noindent\small 2020/08/07 06:57 提问;2020/08/12 15:32 回答}

\noindent[2.]{\Hei 答}:大哉問。長期來看,新通訊科技、新產業鏈模式和新增財富的確是推進國際間政經整合的强大動力,不過...

首先這並不保證會是全球化,基於主權意識的地緣政略考慮,使得大型區域板塊更加可行。其次,整合不必是完全或絕對的。以汽車產業爲例,20世紀末有幾個并購的嘗試,並不太成功;本世紀比較流行的是中等程度的聯盟,例如Renault-Nissan-Mitsubishi。航空公司之間也自然形成了幾個大聯盟,聯係更鬆散些。

目前歐盟把握英國脫歐的時機,把重點放在内部整合,這會形成最緊密但也是最弱小的主要集團;美國企圖組建反中聯盟,這天然必須是中等緊密的主僕關係;中國則是尋求相對單純的經貿合作,所以可以在第三世界廣汎搜羅對象,然後以鄉村包圍城市。

未來20年國際關係的主軸是霸權交替,也就是親中集團逐步分解、吸收、壓倒親美集團的過程,歐盟基本保持中立。中國確立領導地位之後,我希望它能以理性、互助、公平的原則,持續推動全球的進一步整合,但中國必須先解決自身内部的許多問題(例如我多次討論的學術腐敗和教育公平問題),所以目前言之過早。
\\


\section{【美國】【工業】熱帶風暴之後}
\subsection{2020-08-12 13:39}


\section{4条问答}

\textit{\hfill\noindent\small 2020/08/14 10:45 提问;2020/08/15 06:14 回答}

\noindent[1.]{\Hei 答}:我對中國商界的現況不熟,不過美國會全面MBA化,的確是壟斷優勢的後果。

這裏的壟斷有兩個層面:一方面美國獨占世界科技產業鏈、儲備貨幣和經貿規則的一哥地位,所以能夠在國際上壟斷遠高於GDP和貿易額佔比的利潤額分;另一方面,雷根的新自由主義政策,使得在美國國内也是由財閥階級壟斷經濟生產價值。其結果,是美國的企業不須要擔心幹實事(亦即認真開發新技術),依舊可以賺得盆滿缽盈(例如高通靠收專利費賺大錢),那麽真正需要投入精力和資源的,就是内部分贓的競爭。

這些競爭也有幾個不同的層面:首先在國際上,靠著霸權開道,可以用金融手段控制他國的高科技企業(例如三星),無須真正參與生產;其次是國内的階級和專業之間的利益分配,打壓員工、榨取利潤是重要的熱點,所以脫穎而出的是Neutron Jack或Chainsaw Al,他們都以大規模裁員而得名;最後,既然沒有實事可幹,各級企業幹部的選拔自然成爲相互忽悠的人緣競賽。你如果去仔細想想這三層競爭,就會發現它們都極度偏好自吹自擂、不事生產的MBA;40年下來,連總統都換成這種人(Trump只從Wharton商學院拿到本科學位,但他的言行舉止依然是典型的MBA),那麽全面腐化當然是勢不可擋。
\\

\textit{\hfill\noindent\small 2020/08/14 10:52 提问;2020/08/15 06:36 回答}

\noindent[2.]{\Hei 答}:你所討論的這套辦法,是硅谷私募基金在過去30年的心得,後來Softbank把它在全球推到極限(可能已經超出極限了,所以在WeWork上弄巧成拙),和是否尊重法規沒有直接的關係。

我覺得私人資本要冒險,是他們自己的決定,政府的責任,在於確保社會和經濟不受他們的連帶拖累。以共享單車爲例,這些金主要浪費錢就讓他們浪費錢,但是國家必須强制保護消費者和環境,所以應該事先要求把押金由第三方保管,破損單車的回收處理,也必須提早規劃,要求企業預先付款。我的印象是中國的主管單位並沒有盡責;這種尸位素餐的表現是印度之類國家的日常,中國不應該如此放任懶政的官僚。
\\

\textit{\hfill\noindent\small 2020/08/16 23:13 提问;2020/08/17 05:55 回答}

\noindent[3.]{\Hei 答}:因爲迷信私有制,所以電力公司PG\&E必須自負盈虧,也無法和消防單位協作。

像這樣一方面强制自負盈虧,另一方面又要求繼續負擔公益責任,很自然地只能在以下的脚本裏二選一:1)基本放棄公益責任,社會整體利益大幅受損,例如PG\&E;2)因爲公益責任而入不敷出,反而成爲新自由主義用來洗腦宣傳、鼓吹私有制優越性的例子,像是鐵路公司Amtrak,常年作爲媒體和國會的出氣筒(Punching Bag)。

郵局USPS更加可憐,不但背著公益責任的包袱,還必須面臨私有企業的逐利競爭。近年有客觀的學術研究指出,如果USPS不是必須單獨擔負對偏遠住民服務的責任,它的經營盈利效率已經高過UPS和FedEx。但是財閥作爲私有制的既得利益方,手下的學術買辦根本不跟你講理,多年來消滅USPS一直是他們的目標之一,背後支持他們的農民剛好是最需要郵政服務的族群。這又是一個典型的火鷄投票過聖誕節的例子。

在新自由主義的指導下,不論公益事業選擇哪一條因應對策,也不論它們是私有還是公用,最終的結果都是服務水準下降、收費水平上升,這兩者都壓低經濟整體效率,而且是由底層民衆買單。像是停電頻仍,能負擔得起的住戶只好裝發電機,這是非常低效的設置,遠不如在電力公司的層級就做出足夠的投資來保證不會經常斷電;更不用提低收入民衆根本沒有這個選項,被熱死、冷死也只成爲一個被大衆媒體有意忽視的統計數字。我曾提過光在英國每年就有上萬人被凍死,美國的數字藏得更深,但絕對是更大更驚人的。中國民衆驚訝於美國人對新冠死亡以十萬計而無動於衷,其實是他們幾十年來對這類窮人無辜喪命的案例早已習以爲常。
\\

\textit{\hfill\noindent\small 2021/10/29 02:29 提问;2021/10/29 03:03 回答}

\noindent[4.]{\Hei 答}:過去40年,美國將20世紀中期的社會主義政策扭曲反轉,中底層勞動力被徹底踢出經濟利益的分配桌,到現在已經基本回歸歐洲16世紀的Indentured Servitude,這是當前工資上漲而勞工仍然在不斷主動離職的背景因素。

我在一年前就把新冠的長期影響總結為增大貧富不均,而且是全球性的、長期性的影響。專注在美國,這反映為經濟供給鏈斷裂,社會勞資對立,然後政治進一步分裂,金融危機隨之而來,都是我已經反復預言幾百次的事,不再贅述。當然,Cassandras Curse在歐美比中國更爲嚴重,但我原本就不指望對英語世界做建言,反正無視歷史巨輪走向的族群,終究會被碾壓。這裏的問題在於美國的軍力和財富纍積極爲豐厚,人類世界如何管理它的衰敗過程、避免瘋狂的同歸於盡,才是我們必須關心的事項,這也是博客建言的重要方向之一。
\\


\section{【學術管理】從假大空談新時代的學術管理}
\subsection{2021-01-12 07:57}


\section{7条问答}

\textit{\hfill\noindent\small 2021/01/21 11:18 提问;2021/01/21 13:12 回答}

\noindent[1.]{\Hei 答}:這一波的AI成果,發生在2012年之後,我已經離開金融界了,所以對内幕和工程細節沒有掌握,但是相信在高頻程序交易必然有作用。
\\

\textit{\hfill\noindent\small 2021/01/27 17:34 提问;2021/01/29 08:47 回答}

\noindent[2.]{\Hei 答}:我從學術界轉到金融的時候,一開始不太適應,因爲他們勾心鬥角、浮上臺面;後來慢慢地習慣了,反而看出一個很大的相對優點,就是當時超弦席捲高能物理界,把行業變成說空話、寫科幻的選美大賽;但是我加入的程序交易,卻是完全真刀真槍地拿銀行自己的資本來做交易賺錢,而且頻率極高,一個季度下來策略已經被重複檢驗幾百萬次,一絲浮誇運氣的成分都容不下,這很對我的胃口。

我以前提過,當時工作的實質就是大數據分析。因爲還沒有AI,我必須自己設計全新的非綫性方法,這時的危險不但在於找不到足夠信號,也有相反的找到太多虛假的信號。十幾年下來,我對大數據分析和非綫性統計的特性和極限,自然有了比較深刻的認識。
\\

\textit{\hfill\noindent\small 2021/01/29 20:02 提问;2021/01/29 23:21 回答}

\noindent[3.]{\Hei 答}:我自己的金融生涯就是這樣被毀掉的,對英美權貴階級肆無忌憚、目無法紀的真相,自然是有深刻的體會;我不是一直說英美的所謂法治,最終也是人治嗎?我想老讀者都知道我從來不會空口白話,這類論斷都是有深刻廣汎的證據才會説出來,但是這裏的真相太過離譜,所以我以前很不想談自己的經歷,談了也沒人相信;現在現成證據確鑿,才順便一提。

其實只要睜開眼睛,例子到處都是,只不過他們掌控媒體宣傳管道、又極端不要臉,所以幾百年撒謊洗腦、顛倒黑白,在沒有能力獨立檢驗事實與邏輯的一般群衆心中,種下了一個完全虛幻的假象。

所以讀完我博客,就是“take the red pill—you stay in Wonderland and I show you how deep the rabbit-hole goes.”英美的宣傳,正是現實版的Matrix,the world that has been pulled over your eyes to blind you from the truth.
\\

\textit{\hfill\noindent\small 2021/01/30 06:28 提问;2021/01/30 09:35 回答}

\noindent[4.]{\Hei 答}:這次事件和我在2003年的遭遇,雖然同樣是政治能量大的輸家改變規則來打擊競爭對手,細節上還是有很大的不同:這裏的“華爾街”,是二三流的對衝基金,並沒有左右SEC決策的實力,而散戶本身又有很强的政治意義和公關價值,所以SEC站隊完全相反。

正確的金融監管原則,我以前已經反復提過了,就是絕對禁止任何大賺大賠的可能;最近對螞蟻的出手,是一個很好的實踐。不過商業上總是要和國際接軌,資本主義系統的流毒難以完全避免,這是我同樣反復討論的,處理貧富不均的難處,因爲管制金融是其中的重點之一。
\\

\textit{\hfill\noindent\small 2021/02/08 21:24 提问;2021/02/09 06:30 回答}

\noindent[5.]{\Hei 答}:我自己也覺得很奇怪:這許多失敗的案例,不但在金錢上是巨大的浪費,在時間上也拖了國家戰略需要的後腿,在實際發展上更打壓了真正有潛力的研發團隊,等於是花大錢請人來割自己的肉,而且是事前就可以輕鬆預期,實際上也有足夠的警告聲浪,但偏偏就是一而再、再而三的發生。全國的公款吃喝加起來,還不到這類損失的零頭,怎麽體制内就沒人想要提醒高層注意?

其實從經濟的基本理論原則上,就可以簡單看出,民資在乎的只是報酬。這種戰略性產業,所需資本極大、周期極長、風險極高,還要面對外國敵對政權的打壓,任何頭腦正常的資本家都立刻可以看出是最糟糕的投資項目。單是有補貼,並不改變其難度和需時,那麽民資的理性選擇自然是騙補。所以唯一可行的道路,是技術上集合既有團隊,但國家必須牢牢掌握財務和經營權;這可以是間接的,但必須確定管理人不能自肥、不能跑路。主管單位似乎是有思考盲區,對國企和半國企有反射式的排斥,但自由市場和私有企業其實只適合消費性產業,戰略性行業是沒有火藥的軍事戰場,其目的不是利潤、而是勝利;在這裏也迷信自由主義經濟學,難道解放軍也要轉為民營、自負盈虧?放任只在乎創造就業和GDP的地方政府搞,則是另一條保證失敗的路綫,就像把解放軍完全拆解為互不統屬的省、縣單位一樣。

至於事後不追責,更加匪夷所思。以弘芯爲例,民資拿了經營權,卻違約沒有注入資金,這麽明顯的竊盜國庫,怎麽能不追究?我是局外人,看的一頭霧水;有明白内幕的讀者,請解惑。
\\

\textit{\hfill\noindent\small 2021/02/10 14:53 提问;2021/02/16 10:25 回答}

\noindent[6.]{\Hei 答}:你這個問題問得太大了,讓我猶豫了幾天,還是想不出能在合理篇幅之内做出完整解釋的辦法,所以只挑主要重點來談。

這裏最基本的毛病,來自經濟的成員是Homo sapiens,一種由演化自然產生、高度非理性的個體,所以經濟產出的價值先天就是極端主觀、隨機、易變的。以食物爲例,理性的要求是沒有毒性、營養豐富均衡,但市場上的食材珍貴與否,顯然和那兩個條件沒有什麽關聯。這還是第一產業,如果談到高度抽象、虛擬的第三產業,誰能用理性解釋爲什麽郭敬明是名導演?

所以雖然經濟產值的實際内含品質差距很大,最終還是只能用會計手段事後簡單計算成交的總金額,很難排除虛胖。當然這留給奸商很大的運作空間,但理性人士頂多也只能靠發表真相來聊盡心意;例如減肥和中醫,我都解釋清楚了,但即使是已經相對很高階的博客讀者群,也沒有完全接受,那麽自然更無法指望普羅大衆在乎並且瞭解真相。在國家層面上,既然GDP的品質差異很大,那麽當然不應該只拿總量來當作執政的主要指標;中國政府不是不明白這個道理,而是實在很難找到好的替代罷了。
\\

\textit{\hfill\noindent\small 2021/07/24 13:32 提问;2021/07/24 17:46 回答}

\noindent[7.]{\Hei 答}:高速鐵路運輸的電力消耗,主要來自氣動阻力,不論是磁浮還是軌道,都與速度平方成正比,那麽600公里時速的電費就是350的三倍。或許京滬綫能用得起吧?不過我聽説他們算過賬,連380都不是特別划算。
\\


\section{【美國】【國際】新年的回顧與展望(一)}
\subsection{2021-01-18 07:25}


\section{2条问答}

\textit{\hfill\noindent\small 2021/01/24 12:03 提问;2021/01/24 23:02 回答}

\noindent[1.]{\Hei 答}:市場經濟的優點,在於靈活定價、激烈競爭,其好處是如果不計社會成本,可以提供較高的財富創造速率,壞處則是不但這些新生的財富集中在少數贏家的手裏,而且一旦這些贏家完成寡頭獨占,他們繼續纍積的財富不再是新創,而來自對整體社會的搜刮。所以政府對市場必須做至少兩個主要方面的調控(次要的問題,例如信息不對稱,我們就先忽略了):一方面人爲地把隱性的社會成本轉爲明顯的市場定價,另一方面必須抑制寡頭、確保社會公益,這在無需大量研發和資本投入的消費性行業尤其重要,因爲這裏受損的純粹是一般消費者,而受益的純粹是寡頭。至於名義上投資高科技產業,實際上做金融炒作的,則進一步危害國家的長期戰略利益,主政者如果迷信芝加哥學派的鬼話、尸位素餐,那麽就是國家民族的罪人。

這次馬雲撞上鐵板,正是因爲貪婪過度,把手伸向影子銀行業。我以前早已反復解釋過,影子銀行是特別危險的金融搜刮手段,暴雷之後,國家的損失可以是寡頭前期所得利潤的十倍甚至百倍;這是爲什麽現代銀行業有層層監管的原因。中國的金融監管單位,在2015年失守(參見當時博客的留言討論)之後,有所整頓和强化,所以這次出手並不是完全沒有徵兆的。
\\

\textit{\hfill\noindent\small 2021/04/10 23:35 提问;2021/04/11 05:11 回答}

\noindent[2.]{\Hei 答}:他們連怎麽向哈佛捐款來保證子女入學,都摸得一清二楚,Rockefeller當年怎麽玩壟斷、怎麽玩學術洗腦,當然也有人傳授秘籍。

所謂的自由市場、公開競爭,其實是大家自由競爭來成爲最後的獨占企業;芝加哥學派只講前半、不提結局,伎倆幼稚得很,有教授頭銜還信他的,非蠢即壞。
\\


\section{【美國】【國際】新年的回顧與展望(二)}
\subsection{2021-02-01 12:34}


\section{3条问答}

\textit{\hfill\noindent\small 2021/02/02 09:43 提问;2021/02/02 11:01 回答}

\noindent[1.]{\Hei 答}:美國金融主管人員,在幾十年逆水行舟之後,已然放棄政客們能平衡赤字的希望,集體隨波逐流,雖然明知無限印鈔是在消耗美元的霸權地位,但面臨世紀級的天災人禍,火燒眉毛,也只能且顧眼下。
\\

\textit{\hfill\noindent\small 2021/02/09 19:34 提问;2021/02/10 07:05 回答}

\noindent[2.]{\Hei 答}:資本的特性是必須有意願犧牲一切(尤其是良心和公益)來追求利潤報酬,才有大機率的可能繼續纍積財富,所以任正非是偶然的例外,在國際資本主導的絕對自由市場背景下,他的財富纍積速度必然落後於聯想之類公司背後的大亨。這是極爲嚴格的逆淘汰,長期下來,適者生存的最終勝利者只能是Trump之流無恥、無知、無良的吸血鬼。

我以前已經解釋過,文革的確是世紀級的大災難,光從當時各種虛報誇大造假成爲每個人生存下去的必要手段,就可以看出是絕對的錯誤。我説過,我一生追求的只是一個“真”字,其原因在於“真”是“善”和“美”的前提,要作惡必須先造假,虛假的“美”則内含絕對的醜陋。

然而我也同意,中國在改革開放40多年之後,近年的問題早已不是太左,而是遠遠太右了。一般民衆雖然不完全明白其幕後的複雜效應,卻可以感覺到資本和權力處處結合起來,扭曲經濟和社會上的公平競爭。但正因如此,我們必須堅持以事實真相和理性邏輯來引導他們,不能容許社會思潮走上虛僞的捷徑(例如美化文革的歷史),否則就徒然給別有用心者挾民意自重的機會,薄熙來、MAGA和白左都是前車之鑒。
\\

\textit{\hfill\noindent\small 2021/02/16 23:36 提问;2021/02/17 18:53 回答}

\noindent[3.]{\Hei 答}:其實説明白了,並不難理解:996是資方犧牲勞工的生活品質,來換取較高的研發生產效率。這種勞資對立的議題,應該由國家(而不是選票)來做仲裁,這時整體利益是決定因素。因爲人類社會目前處在一個高度分裂、相互競爭的國際環境下,中國的首要考慮必須是維持國際競爭力,也就是高人一等的研發生產效率。只有在學術界和工商業界廉潔高效的前提下,才不需要勞工加工加時來彌補,也才有餘裕强力出手立法遏止强迫加班。所以縱容學閥腐敗,不只是他們侵占的那些公款損失了,而且是整個產業升級的長期努力都受其掣肘,連帶地全民都必須在收入和生活品質上做出原本不必要的犧牲。這裏每一步次優後果都把損失放大了千百倍。科技部和稀泥危害之大,遠超它整個預算好幾個數量級,既然公款吃喝只占預算的一小部分,放任前者而嚴查後者是典型的Penny wise, pound foolish。

至於忍受美元搜刮,正是因爲美國有全方位的霸權,中國必須在整體國力上超越美國,才可能以人民幣取代美元。國力競爭的關鍵,正又是產業升級,而產業升級的必要條件,是高效的研發,追根究底,又回歸到健全的學術研究環境。我在2014、2015年就提過,習近平的反腐,是創造内部條件,一帶一路則是創造外部條件,真正的長期戰略目標是產業升級。既然國家的最高戰略就在於此,怎麽能容忍一個貪腐低效的科技部和學術界?這麽簡單基本的邏輯矛盾,被置之不理,真令人百思不解。

你沒看懂這些我幾天前的解釋,是不是沒有細心去讀舊文?我的文章和評論不是寫來給讀者消遣用的,内涵比多數教科書還要廣汎深刻,文字卻極度簡潔。反復閲讀直到熟練爲止,是讀者的責任。我鼓勵大家有空回頭復習舊文,尤其是理論性較高、較爲抽象的文章,對照這幾年的新時事做為例證,應該會有更深刻全面的理解。
\\


\section{【國際】新年的回顧與展望(三)}
\subsection{2021-02-01 12:37}


\section{5条问答}

\textit{\hfill\noindent\small 2021/02/05 02:14 提问;2021/02/05 05:30 回答}

\noindent[1.]{\Hei 答}:
我在六年前就解釋過,SWIFT是歐洲公司,只不過因爲主要處理美元,被美國人捉住小辮子,强迫與之合作。
如果人民銀行覺得必須和SWIFT合夥開公司,有大機率是因爲中國燕子CIPS的擴張遇到障礙(包括行業的惰性),需要一些外來的助力。
\\

\textit{\hfill\noindent\small 2021/03/04 10:22 提问;2021/03/04 12:37 回答}

\noindent[2.]{\Hei 答}:首先,你可能有誤解:在歐洲和東亞都面臨生育率驟降的問題後很久,美國一直保持一枝獨秀,到2008年還維持2.055的Fertility Rate(生育率,亦即成年女人一生中所生產嬰兒數目的期望值),在金融危機之後才慢慢減低,過去四年穩定在1.78上下。相對的,中國的生育率在1995年就跌到1.75,過去20年穩定在1.6幾的範圍,開放二胎的效應要等到2022年才會重新衝上1.70的關卡;所以中國的老齡化問題始終是比美國嚴重的。

不過客觀來看,1.7其實不是特別糟糕,例如德國是1.60,瑞士1.55,日本1.37,新加坡1.23,台灣1.20,南韓1.09。女性能普遍接受高等教育、進入職場的社會,自然會有低於2.1(考慮到小孩夭折和男嬰多於女嬰等等因素,一般接受2.1是長期穩定的級別)的生育率,但是這樣的現代工業國家,如果能確保健康的經濟成長、合理的社會福利以及公平的教育機會,並不難把生育率維持在接近2.0的程度,例如美國在1970年代遭遇能源危機,也曾經有過一波生育率萎縮到1.77的經驗,一旦經濟復蘇,生育率也跟著回升了。

中國處理老齡化問題的關鍵,在於一方面必須推進產業升級以保障經濟成長,另一方面必須立刻扭轉醫療和教育私有化過程,反過來大幅增加在這些領域的公共投資,改善全民的基本生活品質。這兩者最終都依靠科技研發效率必須高於國際上的競爭對手,所以我才一再强調學術界的貪腐是當前中國内政上最大的威脅。如果有了誠實健康高效的科技研發,自然就會有足夠的新創財富來解決其他的問題。
\\

\textit{\hfill\noindent\small 2021/04/02 06:31 提问;2021/04/03 05:22 回答}

\noindent[3.]{\Hei 答}:金融巨鰐掌控的美國學術界和傳媒,花了半個多世紀吹噓自由市場,細節之一是故意忽略匯率大幅浮動所隱含的代價(其實大部分的資產價格波動也有同樣效應),原因是漲跌之間有的賺的是金融資本,賠錢損失的卻是實業。中國的經濟以實體製造業爲主,如果匯率自由追逐每個月、甚至每天的平衡點,就會與生產運輸周期大約是3-6個月的進出口貿易脫節,所以做決定必須前瞻幾個月。在當前這個時節,特別不能放任人民幣大幅上漲,這是因爲美國通脹已經冒出苗頭,長期債券的利率開始上升,依往例這是美元大水回收的前奏;雖然這次美聯儲受聯邦赤字拘束,不能自由提高利率、收緊銀根,但匯率出現多空交戰、上下波動是必然的,中國雖然不像印度、土耳其和巴西那類高債務國家那麽敏感,但仍然應該小心謹慎、避開風頭,2014、2015年資金外逃的風波是前車之鑒。
\\

\textit{\hfill\noindent\small 2021/05/20 20:14 提问;2021/05/21 03:33 回答}

\noindent[4.]{\Hei 答}:我的解讀是,上峰要求加速美元的替代,但是金融業行内人不願意把資源用在支持歐元上,所以除了繼續搞緩不濟急、無關國際大宗交易的數字人民幣之外,現在又準備放棄對實體產業貿易極爲有利的匯率穩定機制,以便對外推銷改以人民幣定價。至於這是敝帚自珍,還是胸有成竹,我們等下一輪美國泡沫爆破再下斷言吧。我已經解釋過,這應該會在2-5年内發生,届時是否有方便合適的美元替代,將是決定其影響是周期性還是歷史性的關鍵因素。
\\

\textit{\hfill\noindent\small 2021/05/21 11:56 提问;2021/05/26 10:04 回答}

\noindent[5.]{\Hei 答}:
人民幣在五年内成爲國際儲備貨幣的可能性是零,所以這裏的選擇是,要把貨幣霸權留給美國獨享,還是讓美歐爭奪對抗。
不到六歲的小孩,可以不顧現實,反復打鬧要Unicorn。成年人必須尊重現實,選擇最不壞的選項。
\\


\section{【宣佈】幾則博客事務和感言}
\subsection{2021-02-22 01:56}


\section{2条问答}

\textit{\hfill\noindent\small 2021/03/10 09:15 提问;2021/03/14 13:28 回答}

\noindent[1.]{\Hei 答}:是的,如同房地產市場吸收民衆儲蓄一樣,一旦縱容它發生,無數中產階級就成爲既得利益的人質;還有哪一種既得利益集團比大批中產民衆更危險的?清朝的覆滅,真實原因是搞國内自籌鐵路開發,資金被貪污掉了,於是有保路同志會,鄂軍入川,才有武昌起義,而且各地投資拿不回來的土豪紛紛響應。

這類改革的凶險,幾乎和反腐在同一個級別,我也不知習近平有沒有那個魄力出手。
\\

\textit{\hfill\noindent\small 2021/04/01 08:26 提问;2021/04/01 13:10 回答}

\noindent[2.]{\Hei 答}:這個問題的答案,在Piketty的《21世紀資本論》中就有了:5-6\%的複利只發生在沒有世界級天災人禍的前提下。那麽既然那樣的複利不可能永遠持續,似乎資本主義只能一步一步邁向毀滅全世界(不只是自我毀滅)的宿命。不過自從馬克思開始,無數理想主義者希望通過社會主義改革,來解決這個問題,雖然至今還沒有真正成功的案例,但考慮到不作爲的後果是如此地嚴重,我們只能繼續努力。
\\


\section{【國際】再談Biden任期内的中美博弈等議題}
\subsection{2021-04-08 05:41}


\section{10条问答}

\textit{\hfill\noindent\small 2021/04/08 07:44 提问;2021/04/08 08:14 回答}

\noindent[1.]{\Hei 答}:我很不希望這個博客成爲炒股論壇;請大家以國際戰略的眼光來看這些事情。

金融泡沫一般沒有時限性,如果拖得夠久,甚至可以成爲行業内的常規,例如比特幣就有可能無限存活;像是SPAC這樣時限確定的定時炸彈,其實很少見。
\\

\textit{\hfill\noindent\small 2021/04/08 23:26 提问;2021/04/09 00:01 回答}

\noindent[2.]{\Hei 答}:因爲現代美國由資本統治,政客只是他們手下的打工人。
\\

\textit{\hfill\noindent\small 2021/04/09 12:24 提问;2021/04/09 12:39 回答}

\noindent[3.]{\Hei 答}:這件事不是針對中國,而是民主黨爲了稍微平衡赤字而向資本做出的微弱挑戰;它的妙處是對小國施壓,不用管國會的意見,可以由行政單位單方決定。

民主黨向來是願意維持較高稅率的;而從美國的根本利益來看,增加稅收、減低赤字,也是推緩美元崩潰的不二法門。不過在過去40年,資本對美國政府的掌控,快速增强到近乎絕對的程度,Biden想要扭轉Trump2017年的減稅額度的一半,都面臨極大的反對聲浪,這個國際稅率標準也只能是杯水車薪。

從中國的觀點,這是防堵無良資本外逃的一小步,應該樂觀其成。
\\

\textit{\hfill\noindent\small 2021/04/10 23:13 提问;2021/04/12 12:53 回答}

\noindent[4.]{\Hei 答}:三萬多億的美元外匯儲備,購買力當然會縮水,但是和外交、經貿、軍事上的實利相比,微不足道。

疏不間親,而且宣傳是美國的强項,要離間美國資本,事倍功半,頂多只在小事上有成。

我以前解釋過,金融市場往往90\%的正確還不如100\%的錯誤,所以要做精確的預測是不可能。不過從國際大戰略的眼光,綜合考慮演化趨勢和不確定性,還是可以做一些處方的。我最近一直想要上史東的節目談這個話題,不過身體不好,一再延期。大家稍安勿躁,再等等吧。
\\

\textit{\hfill\noindent\small 2021/04/11 11:19 提问;2021/04/12 12:40 回答}

\noindent[5.]{\Hei 答}:自從秦始皇建立中央集權的“現代式”政府,中國人先天就假設公共事務運作必須有内部紀律的一致性,然而英美完全不同,他們的政治體系其實只是各方勢力的談判桌,只有在很重要的國安和霸權議題上,拜兩次大戰和冷戰之賜,有系統性的組織和策略。

2006和2007年的時候,有沒有金融界内部人士看出危機將至呢?當然有的,而且還不少(區區在下也是其中之一),但這並不代表有任何機制可以敦促公權力出手挽救局面,連美聯儲都迫於既得利益集團的壓力而只想矇混過關,所以預見危機的人唯一能做的就是設法從中取利,例如高盛和好幾個對衝基金都大幅做空市場。正是因爲有這樣的例子,我才能斷言他們的體制是低劣的。你如果憑空想象出陰謀論,就太擡舉他們了;當然高盛他們佔的是德國中小銀行的便宜,但那不是出於戰略考慮,而是因爲後者天真幼稚,方便使然。猶太人要騙錢搶錢,連自己人都不忌諱的;不止在納粹時期有很多例子,近年的Madoff案也是一樣的。
\\

\textit{\hfill\noindent\small 2021/04/25 23:27 提问;2021/04/26 03:26 回答}

\noindent[6.]{\Hei 答}:正文中已經强調,美元霸權是美歐之間最大的矛盾,更高於中歐之間爭奪高科技工業主導權的競爭態勢。中方一直沒有拿來利用,是非常不智的浪費。

至於如何利用美元來離間美歐的手段,我也已經明確列舉出來,亦即當前美國股市的SPAC狂熱。其實如果這篇博文討論的重點是金融,那麽雖然美國經濟有明顯的過熱現象,但正因爲美聯儲可以無限印鈔,所以泡沫爆破的具體導火綫和時間點都有很大的不確定性,SPAC只是其中的可能之一;畢竟三萬億雖多,但仍然在美聯儲可以遮掩的範圍之内。我特別把SPAC挑出來在正文中討論,是因爲它最明顯、最簡單、最易懂,非常適合中方拿來對德國人説事。在金融、貨幣的管理上,德國是全世界主要經濟體中最保守的,沒有之一,所以是這類離間法的天然受衆。
\\

\textit{\hfill\noindent\small 2021/04/26 17:29 提问;2021/04/27 02:07 回答}

\noindent[7.]{\Hei 答}:你沒有讀懂我這幾年有關美元霸權的討論。

美國復蘇快於歐盟,正因爲前者印鈔票是後者速度的五倍;所以歐元獲得若干美元的額分,不但是歐盟經濟亟需的助益,對中國的大戰略,也是建立多極世界的重要步驟。

至於歐元對美元的競爭力,光看經濟表現當然不足,否則前者也不會在過去13年步步倒退。然而美國的泡沫越吹越大,已經進入指數成長的階段,就算美聯儲能填補SPAC這個坑,其他類似的浮誇資產不勝枚舉,我估計2-5年之間會終於壓不住。下一個金融危機應該就是美元的末日,但歐元區還是必須預做準備,尤其是避免投資美元資產,靜等作爲避風港,則自然會受益。

中國必須爭取歐洲,是國際形勢下的必要,不是因爲方便而順水推舟。在貿易全球化的背景下,國際話語權是有很大的實質影響的;中俄聯盟依舊是弱勢,只有歐洲中立,才可能對抗英美外交宣傳體系。
\\

\textit{\hfill\noindent\small 2021/05/06 12:20 提问;2021/05/07 11:37 回答}

\noindent[8.]{\Hei 答}:她説的是常識啊。Yellen是學術界出身,改不掉清談的習慣;其實做了主管,就沒有這樣説話的自由,因爲聽衆搞不清楚那是客觀評論、還是主觀政策。

因爲美元是國際儲備貨幣,美國的經濟規則和管理,都是獨一無二的。全世界只有美國才能自己大印鈔票、通貨膨脹卻主要出現在其他國家。所有中央銀行中也只有美聯儲,才能無視任何國外因素,只考慮自己的經濟和貨幣,來決定收放的方向和程度。

美國經濟管理的基本問題,在於自20世紀中期開始,其宏觀經濟學受芝加哥學派影響,有意地忽略了外包風潮所帶來產業空心化的後果,實業被挖空的問題當然不能靠貨幣政策來解決,然而手裏的工具只剩下這一個(亦即美聯儲放水來資助超額的聯邦赤字)。它一方面極爲强力,另一方面卻完全無助於根治經濟的痼疾;換句話説,美聯儲的策略在短期内完全可以剋服所有其他因素,自由決定美國經濟的走向,然而長期來看,目前越是死撐、泡沫爆破的後果就越嚴重(除非又能轉嫁給歐盟和第三世界)。這個自主和宿命兼具的二元現象,使得觀察者很難做出準確的預言;我一直想要討論這個議題,但不確定性太大,遠超我平常寫作的標準,所以一直拖著。
\\

\textit{\hfill\noindent\small 2021/09/12 08:37 提问;2021/09/12 22:50 回答}

\noindent[9.]{\Hei 答}:還不到時機。國際儲備貨幣發生危機的時候,全世界都會知道的,不用急。

Biden的電話,用意純粹就是要避免動武,其他一切照舊,參見我稍早的另一個回復。這裏補充一下,Biden並不是看到共軍準備動手,而是這類危機從挑釁到真正動手有好幾個臺階,因爲事關重大,所以美方必須在第一個臺階(也就是Lithuania的挑釁)就趕緊消除升級的危險;在這個階段,美方從經驗判斷,認爲只要一再重複那句“我們堅持一中原則”,就足以安撫中方,所以那也是Biden實際上説的話。

雙方都公佈了談話内容,我兩邊都看了;外交詞匯先天就很晦澀,但以上是唯一合理的解讀。
\\

\textit{\hfill\noindent\small 2021/09/22 19:15 提问;2021/09/23 01:34 回答}

\noindent[10.]{\Hei 答}:
表面上類似,實質上相反:Lehman Brothers是整個體系在極度狂歡、而執政者卻繼續煽風點火前提下頭一個爆破的倒霉鬼;恆大是整個體系在低度狂歡、因而政府決定主動刺破泡沫後頭一個爆掉的倒霉鬼。所以兩者雖然有相似之處,亦即2008年金融危機是美聯儲多年寬鬆政策和對行業作爲不聞不問的結果,2021年恆大破產則來自多年來(尤其起自2009年過度財政刺激所導致的)錯誤房地產管理政策,但後者的泡沫還沒有吹大到不能倒(Too big to fail)的地步,因而和美國相反,中方決策單位這次是主動出擊,為經濟的全面轉型做準備。
恆大的杠桿雖然極高,但它的投機性資產(主要是造車)佔少數,即使中國整體房地產價格下跌,只要能維持既有的融資速度(亦即不必像老鼠會那樣指數成長),它就不會倒閉。換句話説,恆大的危機,是個Liquidity Crisis而不是Solvency Crisis。依照美式經濟學理論,政府完全應該出手救助;相對的,Lehman則是典型的Solvency Crisis,所以美國財政部才會任其倒閉。恆大危機最驚人之處,在於政府不但不救,而且原本就是有意造成它破產的。這裏的證據很明確,中央在一年前出臺的“三條紅綫”,立刻打斷了恆大所有的融資來源,是這次危機的起因。如果政府不想讓恆大破產,早就可以和許家印交涉,以融資交換產權,並要求他逐步消化杠桿;結果中央選擇堅決打擊,連許在地方政府的關係戶也被嚴格禁止出手幫忙,恆大拖了一年,終於撐不下去。
既然在戰術上選擇犧牲,那麽中央必然是在戰略上有更重要的目標;這個戰略意圖也不難推測:是要為房地產市場消火,將經濟建設的資源,尤其是資金,轉投到實體產業(如半導體和電動車)之上;而且這些投資,必須是華爲式長期性的全神投入,而不是恆大造車那樣的短期投機;換句話說,這是“雙循環”裏建設内循環的那一部分。其實回顧過去這一年的諸多新政策,這個轉向是明確而且全面的,例如打擊互聯網公司,就是完全一樣的思路。在考慮過去四年的中美博弈之後,我們可以進一步推論,這是2019-2020年之間,高層汲取教訓,統籌規劃出來的一盤大棋,雖然和《中國製造2025》目的相同,但邏輯思路遠遠更深更基本。
所以恆大的破產,早在一年前就是定局。破產後的資產分配和企業重整,對中國經濟的資金配置,更有開創健康先例的意義。至於媒體的所謂什麽崩潰啊、什麽爆盤啊,大家就當成娛樂看吧,現代社會裏的新聞報導原本就是笑話佔絕大多數。在恆大事件中,中國政府從一開始就是幕後的編劇、導演兼製作,對未來的進程必然會繼續輕鬆主導,維持秩序更是不在話下。
\\


\section{【國際】【宣傳】如何破解當前歐美的宣傳攻勢}
\subsection{2021-04-26 02:40}


\section{8条问答}

\textit{\hfill\noindent\small 2021/04/28 02:18 提问;2021/04/28 05:16 回答}

\noindent[1.]{\Hei 答}:國内如果沒有愛國企業,那麽至少東南亞有些吧?

現在看來,内奸太多(高能所的人,你們不覺得羞愧嗎?),的確只能讓美國贏這一回合,靜待下一代年輕人成長起來,健全内部。不過正是如此,所以教育改革和學術反腐不是更爲重要嗎?儘快打倒中醫教這樣的非理性思想毒瘤,不是更急迫嗎?你看這次《寫真地理》的孵蛋醜聞,明明是學術界買賣論文版面的生意,卻被大事化小,成了編輯監督不嚴的個別問題。顯然學術腐化根深蒂固,不能指望隨機事件引發改革,只有明智之士群策群力、正面批判,才有撥亂反正的可能。
\\

\textit{\hfill\noindent\small 2021/04/28 04:06 提问;2021/04/28 15:51 回答}

\noindent[2.]{\Hei 答}:不要假設博客解釋的簡單道理對其他人也是明顯的。這是真理的特性:有人説穿了才理所當然。

這篇正文討論的是戰術,前一篇文章討論戰略;在那裏我選擇用美元霸權和金融危機做結尾,不正是因爲那是聯歐制美的關鍵所在?我從《美元的金融霸權》一文開始,多年來反反復復地解釋這個道理,至今中國政府卻依舊在溫吞吞地建設人民幣的國際地位,根本就緩不濟急。

錯過2020年代中期的這個泡沫爆破,代表著必須多等十幾年才有中美力量反轉,那麽台海問題也就不會在2030年之前有乾净利落的解決方案。這些事項都是連帶的,大家不要覺得我性急:我只不過是看得遠、看得清罷了。
\\

\textit{\hfill\noindent\small 2021/05/08 15:44 提问;2021/05/09 10:00 回答}

\noindent[3.]{\Hei 答}:很困難、需要很多步驟、時間跨度太大、國際環境背景要求很高,和博客這裏要求高度確定性的原則有矛盾,沒辦法一次寫出完整而嚴謹的策略建議。反過來看,因爲它們都是正確的施政方針,我在留言欄裏常常會討論其中的各個成分。換句話說,其實都提過了,只是沒有整理出來、放在一起、加上“解決貧富不均指南”的標簽。

簡單來説,首先必須全面清除資本在國際間逃稅和躲避監管的隱藏處所(例如香港),而這必須有中美歐三方共識才可能發生,那麽在中美霸權博弈階段就不可能有大的進展,但是像最近美國提議制定全球企業稅最低標準,起碼面朝正確的方向,所以中方應該予以支持。

然後在國内經濟監管上嚴打獨占性企業(別忘了,獨占是自由市場經濟的必然結果);全面整頓稅務,從收入稅和交易稅轉向財產稅(包括房地稅;可以有較高的個人免稅額,避免傷害中產階級,不過中國民衆沒有誠實繳稅的文化傳統,所以依舊會有很高的政治代價)。

在宏觀貨幣管理上,換檔到較高的通貨膨脹平衡值(用意在於減低資本纍積的速度,但這只是一小部分,更重要的是消除大戶和散戶之間投資報酬率的差距,亦即必須解決割韭菜現象,所以就有接下來的金融改革項目);在金融上,消滅所有不為實體經濟服務的不勞而獲方法。

在科技研發上,維持高效和競爭力,這要求建立良好的學術界風氣,消弭造假、誇大,打破學閥山頭勢力。這是博客近年的重點話題之一,讀者應該很熟悉了。

在社會階級的管理上,必須提供優質、廉價的全民公共教育,以及入學和參政的絕對公平門檻,以保證垂直流動性;在醫療之類的基本社會服務上,加大投資,盡可能消除城鄉差距。

前面提的每一項,都是巨大的改革工程,而且面臨既得利益集團的强大阻力。

我光是談學術風氣問題,就用了五年時間、十幾篇文章才解釋清楚;中國政府采納實行,則一點苗頭都還看不見。解決貧富不均的難度和複雜度,又高出幾十倍,目前我只能指出大致的方向、以及可以立刻實現的部分政策。這個龐大艱巨的任務,在我的壽命期限内是不可能完成的,那只好等待年輕世代持續的努力。
\\

\textit{\hfill\noindent\small 2021/05/09 13:05 提问;2021/05/09 22:12 回答}

\noindent[4.]{\Hei 答}:不可能消滅中國以外的投資機會,尤其是亞非低開發國家,永遠都會對資本有相當大的需求,所以也就必須付出必要的回報。不過作爲一個真正王道的國際領袖,可以制定合理規則,盡可能將資金導向實體經濟發展。
\\

\textit{\hfill\noindent\small 2021/06/09 23:15 提问;2021/06/10 11:26 回答}

\noindent[5.]{\Hei 答}:
這個博客的終極目標是對人類社會做出最好、最大的可能影響,所以文章寫作向來是以政策建言為基本設定。我最近解釋過幾次,博客建言的重點在於診斷和預後,至於處方,執行者應該有自行決定細節的空間,所以我一般只簡單舉例示範。
美元霸權是個典型的案例:中方顯然連應該出手的認知都沒有,所以解釋其必要性遠比詳列執行方法重要得多。其實Trump早已不分青紅皂白、不顧中方的妥協和退讓,得寸進尺,把美方的牌都打光了,中國打擊美元根本不會有更壞的後果。這種死豬不怕開水燙的邏輯,俄國人早已實踐示範,公然高調地去美元化,結果美國根本就沒有出手、也無從出手。這裏有部分原因是美國經濟學界有共識,知道美元霸權的副作用之一是貿易逆差,既然減低逆差、重建本土工業在Trump之後成爲全美政壇無分左右的政治正確,美元霸權也連帶被視爲雙面刃,在政權對外折衝的過程中,優先順序反而還不如美國互聯網企業在歐盟的逃稅自由來得重要。至於如何把美元推下神壇,俄國也同樣做過示範;他們那個中央銀行行長是個厲害角色,中方只要願意合作,她自然能想出各式各樣的花招,用不著我來越俎代庖。
外宣實在讓人看得着急,我才被迫反復地談策略細節,其實道理簡單得很:堅持理性的陣地,把中美對抗重新定格為科學專業與政治宣傳之間的搏鬥,然後中方自然獲得一群極爲强大的盟友,亦即全世界所有沒有預設政治種族偏見的科學從業者。一旦明白這個原則,如何執行就順理成章;我真沒有預期這麽清晰淺顯的道理會如此難以理解,唉。
我知道台灣的有識之士,早已被愚民噤聲;兩代之後,回顧“綠色文革”期間的瘋狂,或許歷史學者會引用博客這裏的討論,來證明台灣的理智火苗並沒有被徹底毀滅。
\\

\textit{\hfill\noindent\small 2021/06/12 01:24 提问;2021/06/12 03:28 回答}

\noindent[6.]{\Hei 答}:
其實總體策略原則一旦被接受,就很難改變。這裏鄧小平留下的外交上韜光養晦和經濟上走市場路綫,在當時都是絕對正確的不二法門,畢竟40年前中國還很羸弱貧窮,不宜在國際上貿然出頭,在工業發展上,也必須從消費品輕工業做起,自由市場顯然是最高效的。但時過境遷,方程式的參數變了,最優解自然不再是同一個。中國在2010年左右,就應該在外交和經濟管理上準備開始轉型,可惜一直沒有出現類似當年王滬寧在政體設計上獨排衆議、又能上達天聼的前瞻性學者。
\\

\textit{\hfill\noindent\small 2021/06/17 07:18 提问;2021/06/17 09:04 回答}

\noindent[7.]{\Hei 答}:
目前美元依舊是絕對强勢,只能蠶食;若是美國跌入嚴重的經濟危機,則可以鯨吞。這兩者的實際戰術,有所差異,尤其是下一個經濟危機的細節還屬未知,所以我只討論前者。
貨幣的國際地位體現在許多方向上,包括貿易、定價、儲備、兌換、銀行等等,彼此互相加成,所以打擊美元也必須全面出擊。
在貿易上,中國顯然還沒有用心用力地去美元化;照理這是最基本的手段,俄國已行之多年,所以從這裏就可以簡單看出中共高層並沒有下定戰略決心。其實正如我以前討論過,不只是貿易,像是援助和貸款都不應該繼續使用美元。
我說“定價”,指的是大宗貨品,尤其是期貨。中國近年建立了内部自有的期貨市場,固然有些許斬獲,但金融市場有很强的慣性和網絡效應,如果只靠自己對抗全球,注定會頂上很低的天花板。“兌換”也是同樣的情況:我以前提過,建立外匯期貨市場很容易,吸引交易量卻極難,連人民幣換歐元都還是以美元為中轉最方便划算,那麽人民幣直接兌換其他貨幣的期貨當然也是擺設而已。正因爲金融市場的强大網絡效應,交易越集中於少數貨幣、效率越高,所以連兩種國際貨幣都不容易做到,更別提“多元”;而偏偏中美博弈已進入高峰期,打擊美元的關鍵時段是未來15年内,所以中方實在沒有餘裕慢慢提升人民幣的地位;短期内唯一有實用性的可能替代,只能是歐元。
在貨幣儲備上,各國中央銀行原本就趨於保守,IMF又被美國實際把持,中方能做的,主要是在美元熱機循環到了收割階段可以出手截胡。中國的私營財團受體制規範,先天只想著洗劫國内經濟,然後伺機轉移資產到國外,在美國財閥眼中是韭菜而不構成競爭力量,所以只能由國營金融財團出手;這需要人才、組織、資源和策略的預先準備,我怕的是如同最近的外宣一樣,臨時抱佛脚會弄巧成拙。
在銀行方面,美國削減歐元威脅的手段之一,是用各式各樣官方和非官方的手段來打擊歐系銀行,在過去十幾年非常成功。這其實是中國對歐外交的另一個嵌入點,中歐投資協定是很好的開始,但中方似乎是出於外交戰術考慮而做出的讓步,並沒有理解到金融戰略上的意義:當前全世界勉强算是能挑戰美系投資銀行的,只有歐系,這是又一個必須藉助歐元來打擊美元霸權的原因。中系銀行和影子銀行可以出手的方向,是到亞非第三世界去資助、控制新生的互聯網支付系統,這顯然對去美元化的努力會有反哺作用。
以上談的是中方單獨執行的可行方向;國際上的聯盟協作可以提供新的可能。兩年前曾有金磚貨幣的相關討論,這因爲印度公開全面投入美國陣營而自然作廢。但俄國是去美元化的急先鋒,中國與其合作是理所當然的事,至今遲疑不決是很嚴重的失誤;南非和土耳其等等區域性强權也可以爭取,畢竟他們正是美元對外收割的主要對象。
\\

\textit{\hfill\noindent\small 2021/06/23 15:57 提问;2021/06/24 03:47 回答}

\noindent[8.]{\Hei 答}:
是離題了。原本我會刪這條留言,不過剛好本期的《Economist》有一篇相關文章(參見Hard truths about SoftBank | The Economist),值得推薦。它談的是孫正義和軟銀,原本一年多前已經瀕臨崩潰,但受益於美聯儲的無限放水,又滿血復活。這裏最有意思的細節,在於孫正義在過去幾年高調起用一批來自Deutsche Bank的交易主管,而這批人的專長就在於佔投資人和雇主的便宜;這正是我20年前和Deutsche Bank打交道的經驗。而且孫所雇的,居然是一個印度人團隊,他們比美國人還要更無恥、自私、糟糕。如果美國真如我所預期的,在未來5年内出現經濟金融危機,那麽孫正義和這批印度人或許可以抱著掠奪的贓款舒服地退休,但是軟銀這個機構除了由中方監管維持的阿里巴巴之外,只怕會面臨連續而全面的爆雷。
我不太想要繼續這條討論鏈;如果讀者真有深刻的心得,也請移步到金融類文章下留言。
\\


\section{【宣佈】Whoops,不小心刪了一些值得保留的討論}
\subsection{2021-06-27 04:14}


\section{1条问答}

\textit{\hfill\noindent\small 2021/08/01 22:54 提问;2021/08/02 05:24 回答}

\noindent[1.]{\Hei 答}:我已經解釋過了,這是自由市場體制下大小資本家玩弄規則,對外搞尋租、建立托拉斯,對内欺下瞞上、爭功諉過的典型現象。消費性行業原本就以群衆的主觀滿足為標準,難以定義客觀的是非好壞,容許若干程度的市場自由還説得過去;中國放任學術圈的專業精英也走上私利最大化的邪路,才是奇怪之處。

英美對學術界的管理原則是依托個人主義傳統,指望良心人施加同儕壓力,遇到離譜行爲主動吹哨揭發,雖然長久下來,原先健康的文化也會被逐利心態慢慢腐蝕,但至少比起中方要高明多了。
\\


\section{【歷史】冷戰、學運和五四運動}
\subsection{2021-07-08 08:45}


\section{1条问答}

\textit{\hfill\noindent\small 2021/07/08 09:07 提问;2021/07/08 09:18 回答}

\noindent[1.]{\Hei 答}:在當時,鄧小平沒有立刻反腐那個選項:中國經濟落後太遠,非有2、30年急速瘋狂的成長,不足以纍積足夠的資本,進一步挑戰第一世界。雖然他不可能預見習近平的掌權和改革,但我相信事後歷史的實際進程,正是他所期望的最佳脚本。
\\


\section{【金融】我對引入美國投行的一些看法}
\subsection{2021-08-09 03:25}


\section{22条问答}

\textit{\hfill\noindent\small 2021/08/09 06:07 提问;2021/08/09 09:50 回答}

\noindent[1.]{\Hei 答}:其實我從幾年前就在正文和留言裏一再提過:1)必須支持歐元;2)要支持歐元,就必須支持歐系銀行;3)容許歐系銀行進入中國市場,剛好是改善中歐關係的極佳誘餌。去年年底出現了中歐投資協定的消息,我還以爲中方的金融戰略終於上了正軌,結果反而是美國銀行捷足先登。這無論如何都不可能符合中方的戰略利益(因爲即使要對外開放,也應該讓歐洲優先);換句話說,你這個有關内奸的論斷雖然突然,但我剛好有可以達到同樣結論的邏輯論證。唉。
\\

\textit{\hfill\noindent\small 2021/08/09 11:06 提问;2021/08/10 13:18 回答}

\noindent[2.]{\Hei 答}:有任何報導證實這是第一階段協議的一部分嗎?請提供鏈接。

2015年股災的時候,規則就不是中國自己訂的?我早説過,只要是有大賺大賠,金融管理就還不到位。中國的管理到位了嗎?就在這篇正文裏,我還特別强調“國内奸商”汎濫,然後又詳細解釋了美國投行會教導他們提高尋租、鉆漏洞的效率。如果監管單位稱職,會有這麽多奸商?正文最後一句的“精神美國人”,真的在中國決策和執行單位裏一個都沒有?

美國投行裏,人人都是哈佛、耶魯的頂尖學生出身,繼承了百年來無數伎倆的知識纍積,爲了每年幾百萬美元的年終分成而日以繼夜地殫精竭慮,不斷繼續發明新的花樣。中國的金融監管單位也都是清華北大的畢業生嗎?有實際金融游戲的經驗嗎?獎勵機制呢?這些金融花樣,不説穿,連其他業内專家也想不到(如果他想得到,就應該自己應用來賺大錢),監管人員怎麽可能?如果沒經驗的人也能想得到,美國投行怎麽會笨到要每年花費幾千億美元獎金?反之如果監管人員看不懂,那又談何制定規則?
\\

\textit{\hfill\noindent\small 2021/08/09 11:24 提问;2021/08/10 11:46 回答}

\noindent[3.]{\Hei 答}:第一階段協議包含引入美國投行的消息,我自己沒有看到,請列出你們的資訊來源。
\\

\textit{\hfill\noindent\small 2021/08/09 22:08 提问;2021/08/10 11:37 回答}

\noindent[4.]{\Hei 答}:是的。我也沒有很好的解釋,所以顯然有外人不知的内幕。正因如此,我的這篇正文其實已經寫得很含蓄了。
\\

\textit{\hfill\noindent\small 2021/08/09 23:08 提问;2021/08/10 11:56 回答}

\noindent[5.]{\Hei 答}:這件事處處透著詭異(因爲實在太離譜),我懷疑手上缺了幾塊關鍵的零片,拼圖(Jigsaw Puzzle)凑不起來,所以正文並沒有把話説死,一方面要針對最壞可能做出批判和建議,另一方面如果幕後另有曲折,我也不想無事聳人聽聞。
\\

\textit{\hfill\noindent\small 2021/08/10 02:08 提问;2021/08/10 11:38 回答}

\noindent[6.]{\Hei 答}:不可能只是券商(Retail Brokerage),主體必然是投行(Investment Banking)。
\\

\textit{\hfill\noindent\small 2021/08/10 02:08 提问;2021/08/10 13:04 回答}

\noindent[7.]{\Hei 答}:正文有一整個段落,專門論證金融行業的任何效率提升(除了最基本的業務人事管理品質之外)都會自動投入尋租,而不是反饋經濟。你的整個論述的基本假設卻正是美國投行的高效是在資本配置上。我知道這是美國經濟學課本裏面的基本教義之一,但我在部落格批判美式經濟學沒有一萬也有上千次了,讀者沒有藉口。而且美國這些投行搞出全球性經濟危機遠遠不止一次,任何客觀的人都可以看出他們的盈利高效純粹是杠桿作用,大賺10年,還不夠危機期間一年賠的,最終全靠美元霸權來兜底。

迷信課本、重複口號的人,選擇忽略我已經寫過的幾千萬字,參與留言討論有什麽用呢?難道你指望我專門爲你寫下幾億字的回復?我追求的只是真相;既然美國經濟學教科書在你心中高於真相,那麽就請你繼續和那些“學者”討論,這個博客與你無緣。反正你已經違反了《讀者須知》的第八條,我們依照規則處理。


要提高資本配置的效率,我以前也解釋過了,必須先對奸人利用信息不對稱來詐財做出打擊。中國資本配置效率低,不是因爲沒有美國投行這類玩金融游戲的高手,剛好相反,是監管不足,例如美國都沒有的刷單產業,卻在中國欣欣向榮。錢被騙走了,談何效率?要解決像漢芯那樣的負效率案例,美國投行能有什麽貢獻?

犯罪學早就有一個共識:定罪率若是低於30\%,對一般罪犯的嚇阻力就近乎0。中國對商界的詐騙行爲比美國還要放任,定罪率別説30\%,連1\%都不到,才導致資本配置效率低下,你們這批被美國洗腦的帶路黨居然還在鼓吹要進一步去監管,罪莫大焉。
\\

\textit{\hfill\noindent\small 2021/08/10 15:45 提问;2021/08/10 17:53 回答}

\noindent[8.]{\Hei 答}:原來如此。那麽的確可能是19年談判時太心急,做出過度的退讓。當時很可能也有想要藉機開放金融的官員;在最近的金融監管改革浪潮下,希望那個危險已經消退了。

最早的投行生意,原本只是幫一般企業上市或重組,後來因爲往往需要自行處理一些IPO股票,開始參與證券交易,越做越大。到20世紀後期,這些大宗證券交易的部門早已喧賓奪主,成為利潤的重心。然後從80年代起,有人開始獨立出去,以基金的名義搞同樣的交易策略,成爲影子投行。過去40多年的幾個主要金融危機都有真假投行在搞大宗交易的過程中,為追求利潤把杠桿調得太高的“貢獻”。

“證券、基金管理和期貨服務領域”固然有面向小投資人的零售銀行業務,但即使在那裏,交易單(Order Flow)匯集起來之後,幕後也要有人集中處理,這就屬於投行的大宗交易了。20年前我在UBS創建程序交易的時候,當然也準備要順便把這方面原本外包出去給像是Madoff那類公司人工作業的小額多頻交易(就是因此和Madoff的弟弟、兒子打過交道),藉著自動化收回自理。那個商業計劃,後來被整個華爾街廣汎參考複製,很快就完全普及。JP Morgan的團隊據説做得不錯,利潤率很高(處理客戶交易單的利潤,當然最終還是來自客戶,差別只在於吃相難不難看)。此外,不知他們這次會不會連消費者端的門市生意也一起做。我以前提過,在2000年我曾建議UBS買下一家既有的美國證券公司來一次性獲取更高的交易流量(後來高層選擇兼并Paine Webber),但是中方可能不會(也不應該)容許美國人買下大型的零售證券商,除非這也是外交承諾的一部分。
\\

\textit{\hfill\noindent\small 2021/08/10 22:11 提问;2021/08/11 02:04 回答}

\noindent[9.]{\Hei 答}:這些考慮,都是博客這裏反復談過無數遍的。我們討論的,都是極度複雜的現象和議題;我寫新文章,不可能把所有的相關論點都再詳細重述一次。最近又有一批新讀者,帶著壞習慣來做海鷗式的留言評論,幾乎必然會立刻露餡,提出早已被博客證僞的論點。我已經是很客氣了,只有明顯忽略最新正文内容的才拉黑。所以再有想説“中央自有能人;主場監管沒有什麽可怕的”,我在此特別警告,這是已經被明確論證的議題,《讀者須知》第八條的鍘刀完全適用。
\\

\textit{\hfill\noindent\small 2021/08/10 23:44 提问;2021/08/11 01:10 回答}

\noindent[10.]{\Hei 答}:這要看引入過程中的確實條文如何。我對時事做分析,最缺的就是内幕消息,所以不是適合回答你這個問題的人。
\\

\textit{\hfill\noindent\small 2021/08/11 07:42 提问;2021/08/13 02:43 回答}

\noindent[11.]{\Hei 答}:像是雇用官二代來“疏通管道”這類低技術性的伎倆,外國投行固然喜歡采行,中國國内的金融玩家還真不需要有觀摩對象。我真正擔心的是,技術性高的間接搜刮,往往連這些投行的交易員自己都只知其然而不知其所以然,更別提監管人員了。引進這類的做法和手段,腐化剝削遠遠更爲隱蔽,還方便學術娼妓編造浮面理論,加貼“增進資本配置效率”的標簽,不但負面影響更大、更深、更長久,而且反腐糾正也極度困難。
\\

\textit{\hfill\noindent\small 2021/08/15 07:04 提问;2021/08/16 08:02 回答}

\noindent[12.]{\Hei 答}:《經濟學人》剛剛有專文(China’s future economic potential hinges on its productivity | The Economist)介紹中國在2020年3月出臺的32條提升全要素生產率(Total factor productivity)的政策,很奇怪我用中文搜索不到;不過從《經濟學人》的評論來看,還是自動化和城市化那些老套的宏觀措施,完全沒有考慮到地方亂投資和不追究騙補所帶來的資本浪費。其實投資效率,的確是生產率的決定性因素,只不過這靠的不是美式經濟學所鼓吹的金融“創新”和“自由”,反而是監管和紀律,而中共中央管經濟的人,似乎還沒有理解到這一點。


Risk-adjusted return和Risk premium都是廣爲人知的金融概念,理論上當然也應該納入行政效率評估和人事績效考核;但在實際執行上難度很高。這是因爲風險比回報(Return)的估算還要難得多,就連計算GDP成長率(屬於Return)都有一大堆貓膩(參見《談GDP數字的局限性》),風險的計算公式更加容易被扭曲。現代投行裏都有龐大的風險管控部門,但金融危機一樣反復發生,就是這個道理。與其强行數量化和公式化,不如把行政的其他考慮因素直接列為指標,例如環保成果、地方政府的隱形負債等等。
\\

\textit{\hfill\noindent\small 2021/08/17 11:51 提问;2021/08/18 01:39 回答}

\noindent[13.]{\Hei 答}:豈止是“不具體”,根本就是信了美式經濟學那一套。正文所批判的金融“開放”、“創新”,似乎是真實的危險。

至於十四五計劃裏,居然依舊包含核聚變,量子信息更是列爲重點,看來科研界的自私忽悠在中央決策層還是很有市場的。
\\

\textit{\hfill\noindent\small 2021/09/02 03:40 提问;2021/09/02 11:32 回答}

\noindent[14.]{\Hei 答}:專業抄手追捧牛市,幾年下來分紅就足以退休,最終爆倉了帳也是銀行的,自然有美聯儲兜底。私人抄家靠借貸來炒股,爆倉了可沒人接手。現在美聯儲還在大印鈔票,大大小小的豬都飛得很漂亮,但是大風總有停息的時候,若是任其自生自滅,能安全著陸的必然是極少數;美聯儲既然已經開始計劃Tapering,自然會擔心政治後果,所以預先做個姿態,至少留下記錄,證明自己曾經試圖遏制最離譜的亂象。
\\

\textit{\hfill\noindent\small 2021/09/03 02:17 提问;2021/09/04 05:09 回答}

\noindent[15.]{\Hei 答}:類似兩個月前由劉鶴出掌全國半導體產業統籌規劃,這是中央對金融交易監管的收權措施,方便就近管理。

博客多年來反復强調專業知識含量高、内建周期長(金融危機大約每十年一次)的尖端產業,不能放任地方和市場胡搞,半導體是如此,金融也是如此。
\\

\textit{\hfill\noindent\small 2021/09/08 23:53 提问;2021/09/09 00:23 回答}

\noindent[16.]{\Hei 答}:直接影響是美元投資人,不必在乎;間接影響是當地的通貨膨脹和逆淘汰擠占效應,一旦放進來,就不可能消除,只能等著擦屁股。
\\

\textit{\hfill\noindent\small 2021/10/05 19:26 提问;2021/10/06 05:38 回答}

\noindent[17.]{\Hei 答}:你説的大部分是正文的意思;不過有關李克强的評論,中共決策過程一直很隱秘,外人難以一窺究竟,這件事的内幕和責任其實很難說。
\\

\textit{\hfill\noindent\small 2022/08/27 09:16 提问;2022/08/28 11:08 回答}

\noindent[18.]{\Hei 答}:雖然博客一直只用一兩句話給出簡單結論(若不對基本定理一筆帶過,金融經濟是如此的複雜,光是這篇正文討論的内容就至少需要十倍於1+1=2的證明篇幅,而Bertrand Russell的那本書已經有一千多頁),如何測量國際貨幣份額,其實是個很深的實用性議題,完全足夠寫出多篇學術論文,然而至今我沒聼過任何經濟系的教授,含那群諾獎得主,能正確解答(這當然可能是昂撒經濟學腐朽的間接後果之一);一個數學系出身的也理解得八九不離十,難爲你了。

是的,正因爲貨幣原本就有多種功能和特性,所以不同國際貨幣的相對重要性,也反映在遠遠不止是國際貿易一個層面之上,還包括了儲蓄、融資、大宗商品定價等等。而各國中央銀行的外匯貨幣選擇,剛好就對應著該國經濟與國際體系對接的總體需要,所以外匯儲備是國際貨幣額分的最佳標杆。這裏的誤差主要來自對各國做平均的權重,理想上應該大致與GDP成正比,最好能與戰略性天然資源挂鈎,實際上偏重的是進出口順差;這個相關性Correlation是正的,但不是100\%。不過這是高階修正,在初級近似上,外匯儲備很明顯是最佳指標。
\\

\textit{\hfill\noindent\small 2022/08/28 03:16 提问;2022/08/28 06:32 回答}

\noindent[19.]{\Hei 答}:對的。事實上這樣的“份額”不論是短期熱錢還是長期財產,都是金融性而不是工業性或貿易性,所以純屬掠奪民族資產,是絕對負面的。
\\

\textit{\hfill\noindent\small 2022/10/03 09:33 提问;2022/10/04 05:54 回答}

\noindent[20.]{\Hei 答}:不止是Credit Suisse,其他歐洲銀行也是一個比一個衰弱,未來兩年内應該會倒下一大片。不過2008年Lehman被容許直接倒閉,是因爲它第一個出問題,美國財政部一開始還天真地以爲可以執行“不干預市場”的教條;這次我預期歐洲國家政府會出手挽救,所以可能只有次要的銀行倒閉,CS這類一級金融機構不是收歸國有,就是和其他銀行(如UBS)合并。

歐洲的經濟前途已經完蛋了,政治上又有極高的不確定性,不具有關鍵技術的公司不值得冒險收購,尤其對非實業資產應該躲得遠遠的。
\\

\textit{\hfill\noindent\small 2023/03/19 12:11 提问;2023/03/20 01:54 回答}

\noindent[21.]{\Hei 答}:歐洲人民的反抗,目前還在初始階段,未來一兩年社會秩序必然會持續惡化;然而深層政府對政壇和媒體的掌握,已經强到史無前例的地步,光是政權轉手(Regime Change,亦即歐美對第三世界搞的顔色革命)很可能依舊沒有實際作用,必須是政體崩潰才能有根本的改變,但這個門檻太高,現在還不能斷言結果。事實上,我認爲歐盟深層政府剛剛針對此事又預做了準備,參見《對俄烏戰爭的新觀察》一文的【後註二十六】。
\\

\textit{\hfill\noindent\small 2023/11/20 11:22 提问;2023/11/20 22:22 回答}

\noindent[22.]{\Hei 答}:政策的建議、策劃和執行其實往往出自中上級官員(也就是部級和廳級)之手;這些人基本都是60嵗那個世代。
\\


\section{【學術管理】中國的學術管理問題來自基本的邏輯謬誤}
\subsection{2021-09-08 02:34}


\section{2条问答}

\textit{\hfill\noindent\small 2021/11/18 01:05 提问;2021/11/18 11:39 回答}

\noindent[1.]{\Hei 答}:我並沒有什麽特殊管道,所以這類有關腐敗、壟斷的内幕消息一般無法置評,不過本周看到《經濟學人》上的一篇文章(參見《Attack on the Tycoons》,\href{https://www.economist.com/finance-and-economics/china-attempts-to-clean-up-its-sleaziest-regional-banks/21806193}{链接\footnote{\url{https://www.economist.com/finance-and-economics/china-attempts-to-clean-up-its-sleaziest-regional-banks/21806193}}})有所感觸:中國改開40年,全面引進市場經濟卻沒有配套的監管系統的結果,是重蹈19世紀英美的覆轍,朝向Gilded Age演進。美國能夠慢慢吞吞、用20世紀的前2/3來對自私的資本做出若干限制,是因爲擁有霸權,沒有太大的外來危險。中國在崛起階段,現任霸主已經在全力打壓,然而學術、金融、商業管理上都是一塌糊塗,欠債極深,不改不行。習近平的改革速度已經是超過我以往認知的人力極限了;上面那篇文章討論的是全國幾千家地方金融機構被私有資本侵占成爲Piggy Bank,這似乎是現在整頓的重點方向,我完全同意應該是最優先,所以大家稍安勿躁,有什麽不合理的問題提出來公開討論、以便提醒執政階層是應該的,但政府只能一步一步來也是我們必須體諒的現實。
\\

\textit{\hfill\noindent\small 2022/05/02 02:40 提问;2022/05/02 06:49 回答}

\noindent[2.]{\Hei 答}:有關乘用車換電池的不可行,我以前論證過了。不過你懷疑得對,重卡車有不同的條件:這裏不但所需的電池容量大好幾倍,充電等待時間成倍增加,更換電池有其價值;而且重卡對空間和重量要求不嚴,電池倉不必為不同型號各自優化;更重要的是,現在電動重卡市場還沒有發展到百花齊放的程度,由政府及早制定共通標準的話,真有可能被行業接受。
\\


\section{【外交】【戰略】美國制華歷程分析及對中國外交政策調整的建議}
\subsection{2021-09-23 00:12}


\section{16条问答}

\textit{\hfill\noindent\small 2021/09/29 16:59 提问;2021/09/30 07:55 回答}

\noindent[1.]{\Hei 答}:澄清一下,Mercedes在2030年轉爲100\%EV的計劃,專指歐盟而言,對其他地區還在觀望之中。


先説德國。德國政壇當前群蟲無主,即使幾個月後塵埃落定,新執政集團必然也偏向弱勢。但是Merkel向來就是無爲而治的典型,所以基本不會有什麽實質上的大改變,頂多就是嘴皮上玩賤。中方體諒他們也是美宣的受害者,包涵一下也就罷了,中德經貿關係應該不會有嚴重的波動,畢竟不讓德國在中美博弈中全盤倒向美方是當前的重要外交政策方向之一。多年來我反復强調,中國外交戰略的重點在歐盟,而爭取歐盟的重點在法國;最近的新發展不是完全印證這個預測?

至於日本,其屬性是昂撒集團最重要的外緣附庸。我在《再談Biden任期内的中美博弈等議題》一文中,已經列舉出中方應該采取的正確認知和姿態,這些預測是長期性、宏觀性的,所以當然不須要每半年更動一次,請自行復習。以下只補充前文沒有詳談的細節,來和新的大選結果發展做簡單的綜合對照討論。

我在這篇《美國制華歷程分析及對中國外交政策調整的建議》正文裏,解釋了昂撒集團的核心(亦即美英兩國)在未來兩三年都面臨斷崖式崩潰的危險(尤其是英國),所以中方只要以治待亂,以靜待嘩(參見《孫子兵法軍爭篇》)就行了。其實日本也類似,只不過需時較長、程度較緩,而且有其獨特的背景,以往我一直沒有機會提及,在此詳細解釋一下。

日本經濟是外貿主導的,而其外貿的主力產業,原本在於消費性電子產品支持半導體、以及汽車支持精密機械和重工兩個大方向。然而前者在過去30年被美國有意打擊瓦解,已經大量流失到韓、台和大陸,現在日本的國力根基只剩後者。然而汽車工業正面臨創建以來最大的換代轉向,從内燃機向電池動力過渡。我以前曾多次談起這事,雖然只論證這是中國奪取造車市場額分的百年一遇良機,不過只要稍微再往前推論一步,考慮汽車是很老的產業,全球總需求量不可能有像電子產品那樣的急劇波動,那麽很簡單可以推論既有的產業龍頭不但必須讓出市場額分,而且會面臨營收和利潤的急速緊縮。當前國際上汽車外銷的龍頭是誰呢?德日再加上美韓。

電動汽車的興起速度,如同當年的光伏一樣,也因爲中國的全力投入而遠超產業早先的預期。在中、德、日、美、韓五個主要玩家之中,只有前兩者屬於第一梯隊,他們EV佔所有汽車的生產額分,在2022年前半就會超過20\%。不過雖然現在兩者並駕齊驅,卻只有中方掌握上游動力電池的關鍵技術和產能,所以德國不可能長期維持並列第一的局勢,參見《我對引入美國投行的一些看法》一文中的討論。

第二梯隊是美韓,目前的估計是到2025年他們生產的EV可以從内燃機搶占20\%的額分。這兩國不是不想加速,而是受制於韓國電池生產商點錯鋰電池科技樹的影響,一連串EV自燃事件導致生產銷售基本停頓,必須從頭研發、擴產安全性較高的磷酸鐵鋰技術路綫(所以BYD的商業前景十分光明,不過請不要在博客討論炒股的議題);這其實是科技學術管理重要性的又一個例證:選擇三元鋰電池尚且有這麽嚴重的惡劣後果,把大量資源浪費到完全無用的假未來科技上,真正是自殺性的行爲,參見下文日本的例子。

日本點錯的科技樹(亦即氫氣燃料電池)遠遠更加離譜,從一開始就可以簡單預見今日的窘境,我也的確白紙黑字地寫下來(參見前文《永遠的未來技術》)。然而你如果去看本田和尤其是豐田企業主管最近的談話(例如Akio Toyoda去年在日本國會作證時的論點,參見\href{https://www.autobodynews.com/index.php/industry-news/item/24053-toyota-ceo-going-all-ev-could-cost-japan-millions-of-jobs.html}{链接\footnote{\url{https://www.autobodynews.com/index.php/industry-news/item/24053-toyota-ceo-going-all-ev-could-cost-japan-millions-of-jobs.html}}}),那真是典型的商業恐龍,他們居然預期到2030年才能做到20\%的EV額分,届時中國、歐盟和美國市場的EV必然已經達到或接近80\%的占有率(例如Mercedes已經明確計劃在2030年之前,停產所有的内燃機),日本的汽車外銷也因此必然會大幅萎縮。Toyoda(他不但是豐田的老闆,也是日本汽車產業協會的主席)自己的估算是上下游生產綫會面臨550萬的減員,這還不包含對整體經濟的間接影響。届時日本的國力若還能維持現在意大利的水平,就算是不錯的了。

新當選的首相岸田文雄是安倍派的人,所以外交政策不會有什麽大變動,頂多是權力不穩,必須多搞些民粹花樣而已。其實中國已經在絞殺日本,只不過手段是非常間接的外貿競爭,需要5-10年才會有明顯的成果。中國政府不必耗費一指之力,就可以靜待日本的全面崩潰。希望以上的討論,回答了你的問題。
\\

\textit{\hfill\noindent\small 2021/10/10 06:36 提问;2021/10/10 08:44 回答}

\noindent[2.]{\Hei 答}:For people interested in the US debt ceiling, it is beneficial to understand its origin. The US started out with a barebone federal government. It was not until the Civil War did it begin to consolidate political powers familiar to the modern society. In fact, federal taxes were levied only on a contingency basis during major wars, and regular taxation became constitutional as late as 1916 with the passing of the 16th Amendment. This is an important historical background unknown to most.

In 1917, the US entered World War I and once again began to issue war bonds. Because the main job of a parliament/congress had always been considered to be the authorizer of taxation ever since the days of 《Magna Carta》, the US congress was naturally concerned that recent changes in the constitution and external circumstances would combine to lead to government's unchecked taxing power, by first creating huge debt loads as fait accompli. Out of consideration for practical feasibility, they simply set a ceiling and left the details to the treasury secretary.

Over the subsequent century, the original intention has long been totally forgotten, and the debt ceiling first became a formality and later a partisan tool. Unfortunately, as a partisan tool, it is not just counterproductive but also unidirectional, thus making its repeal effectively impossible. In fact, it got progressively worse, when in 1995, the congress instead got rid of the Gephardt Rule, which said that as long as the spending had been properly authorized, borrowing money to fund the spending was automatically legal.

The partisan utility of the debt ceiling is unidirectional because the Democrats never have a problem raising it, while the Republicans selectively obstruct the process whenever the White House is not under their control. Therefore, although the former have all the reasons in the world to repeal the ceiling completely, the latter are adamant in its preservation, and they have the senate filibuster rule on their side. The whole long-running mess is simply a reflection of the overall dysfunctionality of the US political system, and nobody expects it to fixed anytime soon.
\\

\textit{\hfill\noindent\small 2021/10/18 10:35 提问;2021/10/19 03:00 回答}

\noindent[3.]{\Hei 答}:4.9\%是和2019年相比後換算的年增長率,很合理啊。畢竟疫情還沒有結束,全球供給鏈問題不斷,原材料價格飛漲,能源供應不足,而且中國還全力啓動了減碳和均富兩個極大的改革。

中國這一波的改革,其深度和幅度為40年來僅見,目前還處於起始階段,所以未來兩三年GDP成長率在5\%上下並沒有什麽大不了的(如果有全球性經濟危機,當然會更低);連下一波高速成長的時段和動力(電動汽車)我都預測了,還能有疑問的讀者顯然是沒有用心。
\\

\textit{\hfill\noindent\small 2021/10/22 13:36 提问;2021/10/22 23:19 回答}

\noindent[4.]{\Hei 答}:Merkel的經濟管理其實還不算太差的,尤其是和其他歐洲主要國家相比,她至少算是謹慎(連諸葛亮都被如此稱許,這絕對不算壞事),願意聼工業家的意見,所以當白左思潮威脅到貿易實利,她有時能夠選擇後者,例如對華政策。真正的問題在於一般人沒有注意的地方:首先,她沒有足夠的專業眼界和魄力(參見廢核;這很可能是幕僚素質的問題,不過這也是西式民主的通病);其次是她沒有關心新增財富的合理分配,尤其是幾千家Mittelstand的家族企業,在她任内許多忽然成爲暴發戶,而老員工卻什麽都沒拿到。

至於給華語界印象很差的柏林機場貪腐事件,其實怪不到總理頭上;這是因爲我以前也提過,德國的中央和地方分權特別徹底,地方政府完全有辦法單獨搞出特大的婁子。事實上,它代表的是德國社會文化和國民紀律的普遍腐朽,比它更離譜的還有一個Stuttgart 21工程:一個簡單的火車站從1980年代開始計劃,1994年公佈(你如果以爲那個“21”指的是2021,就太高估他們了),2009年開工,到現在才建了一半,預算已經膨脹到100多億美元,學術界都把它寫成案例來研究(參見\href{https://www.emerald.com/insight/content/doi/10.1108/JPIF-11-2019-0144/full/html}{链接\footnote{\url{https://www.emerald.com/insight/content/doi/10.1108/JPIF-11-2019-0144/full/html}}}),讀者可以參考。在這樣的歷史背景下當十幾年的總理,不試圖撥亂反正固然是遺憾,但維繫整體秩序不墜也不是每個人都能做到的事。

Merkel真正的貢獻,在於把歐盟帶大到當前可以積極準備進一步整合的地步;這個過程可不是風平浪靜,有好幾次停滯甚至崩解的可能,參見《歐盟内部的無色革命》。如果歐盟如願成爲聯邦,Merkel絕對有資格列為第一級的Forefathers(Foreparents?國父/國母?)。
\\

\textit{\hfill\noindent\small 2021/10/28 22:47 提问;2021/10/29 05:11 回答}

\noindent[5.]{\Hei 答}:你説的大致不錯,但國債真不在Biden政府當前的考慮之中:中方所持的比率低於10\%的級別,過去幾年的購買量在1\%的級別,這些外匯存底主要是用來作爲預防美方金融戰的盾牌,除此之外你買多少美國人在沒有金融體系崩塌的威脅下毫不在乎。

中方當然不想主動過早武統,但問題在於事態已經升級到對等反擊會有擦槍走火的危險,那麽你冒還是不冒呢?不冒的話,美方必然又再升級,届時問題一樣沒有解決,反而更糟糕。這正是我反復預警過、可以事先避免的窘態。現在唯一合適的方案,在於打擊美元,但若是在這裏也猶豫不決,等到美國出現經濟危機,美方會主動把貨幣政策放在雙邊關係的核心,那麽他們固然有更大的理性動力來退讓,非理性的升級誘惑也同時增加。別忘了,如果美國人的理性思維能力能靠得住的話,我們根本就不會有當前的困難。
\\

\textit{\hfill\noindent\small 2021/10/31 01:39 提问;2021/10/31 03:54 回答}

\noindent[6.]{\Hei 答}:最近有一系列内幕消息,顯示Biden對外政策上的自殺性愚昧錯誤,全都由Blinken帶領中生代幕僚主導推動,連戴琪和他相比,都算是溫和派:據説戴還只不過是想延續Trump的策略,拿關稅做爲討價還價的工具,而Blinken卻毫無妥協的餘地,剛好相反,他是一個Closet Neocon,民主黨版的Rumsfeld+Wolfowitz,認定中俄是不共戴天的仇敵,一味堅持要全面升級衝突,包括越俎代庖、在貿易策略上也要求加徵更多的新關稅。

如果這個報導真實可靠,那麽基本不必指望這個政權能學乖了。這是因爲Blinken是Biden的長期心腹,外交和國安的總管,不論捅出多麽離譜的婁子,依Biden的個性都不可能在短短兩三年内下定決心來換人;例如AUKUS,他就只以一句“Clumsy”結案。換句話説,從中方的觀點來看,不但退讓會鼓勵對方得寸進尺,連對等談判都是浪費時間、給對方繼續升級的餘裕。後悔早先沒有及時對等反擊固然爲時已晚、無濟於事,不在對方軟肋(亦即美元)上另開戰綫,卻真讓人費解。這裏不但應該立刻加緊用歐元和人民幣替換美元,更必須和俄國中央銀行預先協調、定下預案,在下一輪經濟危機中把握時機,合力落井下石,但我沒有看到任何人民銀行做這類準備的跡象。
\\

\textit{\hfill\noindent\small 2021/11/04 01:16 提问;2021/11/04 05:34 回答}

\noindent[7.]{\Hei 答}:Sovereign Wealth Fund完全去美元化一點問題都沒有,因爲這基本就是一個國有的儲蓄基金,原本就應該隨意追求高回報、低風險的金融資產。不過這和中國沒有什麽相關,因爲中方的美元資產主要在於外匯儲備,和Wealth Fund是兩回事。

俄國的外匯儲備目前大約是歐元和黃金各25\%、美元20\%、其他佔30\%;這裏的美元比率依舊是遠低於中國的。不過討論去美元化的正確角度,不是要求人民銀行立刻大幅調整外匯儲備的貨幣比例(因爲外匯儲備是抵禦1997年式金融搜刮的盾牌,參見前文《美元的金融霸權》),剛好相反,俄國央行能在外匯上做到這一步,是多年來全面與美元脫鈎後的成果,例如其金融系統,包括融資,已經基本不用美元。我對中方的建議,在於先從貿易、外援、和尤其是金融等方向模仿俄國的經驗,盡全力替代、排除美元,然後能夠安心把美元在外匯占比壓到20\%以下,才算是有小成。

如同兩年前我警告必須儲備天然氣來彌補新能源的不穩定性,結果被當作馬耳東風一樣,留給中方的時間其實不多了,這是因爲美國有大約50\%的機率會在未來兩年内陷入滯漲性經濟金融危機,届時再手忙脚亂地搞貨幣脫鈎,賬面上的損失固然驚人,大戰略上錯失的自我保護和反擊作用則更爲嚴重。
\\

\textit{\hfill\noindent\small 2021/11/10 00:51 提问;2021/11/10 06:15 回答}

\noindent[8.]{\Hei 答}:美國官員的基本修養,就是對著記者只説空話。這裏最起碼的要求,是内部決策即使要公開宣佈,也必須由專人專稿在特定場合發表。所以像是Sullivan這樣現場即興回答問題所提的,絕對不是確實的官方決策,頂多只是和幕僚聊天時曾談過的哲學論點,是對既有現實(亦即美宣早已沒有顛覆中國的影響力;這是幾年前博客讀者群就瞭解的事,參見前文《美國宣傳戰的新困境》)的認知,他自己都覺得是學術性的廢話,才能順口而出(如果有人因爲老年癡呆,而説了好像不是廢話的發言,事後官方還得趕快出面辯解“澄清”呢!),我不知中方政論單就浮面字義來斷章取義、穿鑿附會能得到什麽洞見。尤其Sullivan談話的主軸其實是中美文明體系的對抗,那麽閑談“不能指望中方改變”不正反過來暗示“只好直面對抗打擊”的心態嗎?能把這種發言做正面解讀的人,須要反省自己的邏輯能力。

至於中國的金融管理,雖然我作爲一個外人,很難管窺其内幕,但從2015年股災開始,就可以判斷其問題很深很廣(參見當時的留言欄討論)。這其實和學術管理的難處類似,都是行内人憑藉專業權威來忽悠決策者,而且因爲不像後者那樣有大批公開的論文可以檢視,以私害公的行爲更加隱蔽得多,如我這樣的良心人所能做的建議也就只限於大方向的原則性敘事。
\\

\textit{\hfill\noindent\small 2021/11/12 10:12 提问;2021/11/13 04:01 回答}

\noindent[9.]{\Hei 答}:先依據你的説法來評論第一段:

1.首先,美聯儲的忠誠不針對政黨或總統,而在於金融行業。美聯儲主席的決策,當然會配合政府開支,但這不是出於政治考慮的協作,而是被動地擦屁股;換句話説,優化的標準依舊是金融業的利益,而不是政黨的偏好。

2.其次,美聯儲對金融業的影響力,有50\%左右是裝腔作勢搞出來的“Confidence”信心(Greenspan時期曾經是90\%),如果換新人這方面立刻受損。所以如果我是Biden,就會讓Brainard留在副主席層級;但他是白左教的忠實信徒,爲了性別而讓Brainard硬上並非絕無可能。

3.美國濫發國債從來就不是直接違約的問題,因爲總有美聯儲來兜底;它的危險在於間接地推高通膨,然後導致貨幣政策徹底失控;這一點我已經解釋過幾百次了。

4.我很早就説過,一般人到35嵗之後,三觀就絕對硬化了,必須是有科學素養的聰明才智之士,用心壓制自己的反射性直覺,才能接受新的事實證據,扭轉成見,選擇理性最優解。你覺得Biden、Blinken、Sullivan這些人,算得上“有科學素養的聰明才智之士”嗎?

5.民選制的反智效應,隨時間而做指數增强。晚清沒有那個問題都撐了70年還死不悔改,英國打了兩次大戰、被掏空兩個世代才承認霸權衰落,美國獨霸資源豐富的美洲,要在沒有世紀級災難的前提下做出深刻反省,必須有更高一級的智商,這絕非目前就可以確定的事。

至於第二段:

1.我不是已經反復建議,徹底放棄中美夫妻論,主動挑選有利己方的角度來脫鈎?這除了替代美元之外,也應該考慮美國的經濟周期和金融需要,反其道而行。

2.美國的下一個經濟危機,是否會在兩三年内發生,尚且不能確定,所以更加不可能事先確認其主要導火綫來自何處。我的論點是,不同於2000年問題重心在股市,而2008年是房地產,這輪泡沫被美聯儲全力拖延,已經擴散到金融和經濟的每一個角落,SPAC只是股市方面最離譜的現象;災難固然可能引發自SPAC,其它的候選多得很。正因爲如此,我個人覺得能拖到2024年大選而不爆發經濟危機的機率,是小於一半的。

3.Mid-Term選舉之後,總統跛脚兼無能,只能靠行政命令搞白左花樣,要挽救經濟你想得太多了。

4.最近真忙得不可開交,只好優先學習,保證持續長進;連内定的下一篇博文都已經拖了一個月,所以要等有空才能再上《八方論壇》。
\\

\textit{\hfill\noindent\small 2021/11/14 12:55 提问;2021/11/14 15:25 回答}

\noindent[10.]{\Hei 答}:答案都是博客一再解釋過的:所謂的“合作通道/夾縫”,其實是白左外交理論中打擊/停火的單向自由選擇權。美國全民,而不只是共和黨人,一致同意中國是反人類的邪惡軸心,“良性互動”從何談起?至於“範式”更加不適用於美國政治,他們言而無信、出爾反爾,你以爲是Trump一人的問題嗎?Biden即使連任無望,也必須設法減少民主黨在國會席次上的損失,怎麽可能冒全民之大不韙和中方真正和解?而且他越是在對中打擊上猶豫,越會引誘共和黨對手在競選過程凸顯這個話題來做攻擊,那麽下一任總統也必然更加極端。

在美國民意和政治體制的現實條件下,中方頂多能爭取減低關稅,所以和戴琪談判我不反對。然而其餘的諂媚動作,徒然鼓勵美方得寸進尺,毫無任何實際收益的可能性:高峰視頻會議固然談不出真正成果,氣候問題中國自己做自己的事就好了,扯上美國幹什麽?難道你真指望他們可能捨己爲人、資助落後國家嗎?這還沒有考慮2024年黨派輪替,必然又有反復。當然,未來三年中美外交官聚會、吃喝、談笑的頻率可以大幅提高,有些人會想要把它與成功外交劃上等號,但是中聯辦正是以類似的標準執行了17年,後果是什麽?

然而鄉愿心態的最大危險,還是在於未來幾年美國經濟有摔落斷崖的可能,届時中方是應該處於“合作”模式,還是趁機解除美元霸權,一次性解決美國仇中的問題?中美夫妻論者喜歡談英美霸權和平交接,但是美國正是在一戰、二戰和1956年運河危機,一連三次趁人之危,對英國在金融財政上落井下石,才成功逼迫後者和平退讓。要讓美國人心甘情願地坐視中國崛起,只有兩條道路:一是像美國對日本和德國那樣在軍事上徹底打服,二是像美國取代英國那樣從金融財政方面釜底抽薪。我覺得在核子時代,第二條路遠遠更合適,那麽現在全心去搞和解,不正是事先放棄天賜良機,反而提高最終兵戎相見的機率嗎?
\\

\textit{\hfill\noindent\small 2021/11/15 23:18 提问;2021/11/16 04:36 回答}

\noindent[11.]{\Hei 答}:做這類分析必須考慮美國行政效率的指數下降現象,典型的案例是加州高鐵的建造時程和價格,所以要靠印錢完成“下一次工業革命”,即使假設無限印錢毫無代價,可以指數性地提高投資額,時間上的拖延卻是無解的。而且不只是執行效率有問題,在資金配置上,現代美式體制内公關謊言充斥,拜30年前的遺產,沒有落入大對撞機的陷阱,並且留下一個强大的半導體產業,但2000年之後就因爲無限印錢而使市場紀律和政治文化進一步腐朽,當前幾十個核聚變公司幾億幾億地從市場吸錢,比起SPAC的危害居然還微不足道。換句話説,即使資金能無限成長,投資效率也會因應地下降,長期來看,很可能是絕對負面的事;這裏的典型案例是16世紀的西班牙,從美洲獲得比原本GDP高出一個數量級的金銀礦,結果卻是迅速腐敗,成爲新進强權的獵物。

我自己在大二開始對歷史有真正的興趣;在研一遇上六四事件,開始思考政治體制的問題;到研四明白高能物理是死路一條,爲了留下後路,開始閲讀經濟性文章。30多年的自我訓練,至今也不過是掌握了小部分最重要的議題罷了。例如現在經濟學裏的周期理論,還是依靠外加(Exogenous)變數直接驅動周期性(這竟然是近年諾貝爾獎的得獎研究),這顯然是錯的:實際機制必然是内生(Endogenous)自發的波動;如果我的專業是經濟學術界,當然會花幾年時間去把正確的理論研究出來,但我是無正式職位的散人,時間的最佳應用在於當中國的Walter Lippmann。一旦決定了人生的大方向,就應該一往無前、專心致志地努力,放棄遠離核心任務的課題,把可用的時間精力專注在傳播和教學上。

作爲數學家,要思考社科問題,的確是很極端的轉變;物理原本就是等效理論,連定義都是操作型的,如果對所學有正確深刻的理解(然而在論文至上的體制原則下,能做到這點的是極少數),並不難應用到社科議題上。我對你的建議是,先習慣操作型定義的思路,同時從歷史著手,吸收大數量的案例,然後慢慢地歸納出其中的脈絡,一步一步提升自己思維的Metalevel。這一般需要至少幾十年的全神投入,不過我的博客可以幫助讀者加速這個過程。
\\

\textit{\hfill\noindent\small 2021/11/16 09:05 提问;2021/11/17 05:58 回答}

\noindent[12.]{\Hei 答}:因爲我在評論一個議題之前,必然要先想清楚全面的考慮,所以細節和用詞都是很精確的。一般人囫圇吞棗,然後凴反射直覺來回應(亦即我説過的“海鷗式”留言),已經不適合在博客發言;一個數學系出身的人,也犯這個毛病,就更加不應該。考慮到你一輩子在象牙塔裏成長,剛剛面臨陌生的現實世界,我暫且不禁你的發言;不過請自重,在發問之前先把這裏的論證仔細閲讀、徹底吸收。社科類的邏輯,不像數學是單維單綫的,而必須是多角度、多維度的考慮,才能剋服複雜度所帶來的先天不確定性。

美國對英國的財政釜底抽薪,也經過了40年、三輪的過程;我何嘗説過一次就能讓它從國際社會消失,剛好相反,我也討論過美國在霸權交接塵埃落定之後,應該會反過來和中國親善,届時中方也只好假裝已經忘卻他們的惡行。我和其他政評的差別,在於我認爲這種2、30年後的長期必然結果,是和當前政策考慮無關的廢話;建言的目的,應該是要在未來幾個月到幾年的進程中,保證並加速歷史的和平演進,尤其要從細節上選擇最優解,而其所引發的利害差別,從世代的觀點似乎不重要,但其實出入往往以萬億計,這才是評論國安外交策略的用意,不是學術性、沒有實際影響的清談。

回到你的問題,這裏的關鍵在於,財政金融上的打擊效果,原本就有不確定性,但其他人完全忽視它才是極端,我的意見是即使不致命,也會讓美國病在床上十年,而這正是一般空談的中國和平崛起所需要的餘裕。你看在一戰後,英鎊的國際額分只降到50\%(現在的美元是60\%),英國一樣很快主動推動了《Washington Naval Treaty》,接受美國可以有100\%同級的海軍,這相當於美國自發邀請中國簽1:1平等的限制核武協定,保證了老霸主放棄以武力打壓新挑戰者的選項,不正是所有人一致同意的戰略目標嗎?我只不過指出什麽都不做、坐等老天眷顧,不是最優方案罷了。
\\

\textit{\hfill\noindent\small 2021/11/20 19:34 提问;2021/11/21 06:06 回答}

\noindent[13.]{\Hei 答}:戰略儲備是爲了戰爭(例如臺海)緊急需要而準備的,對中國這樣的主要消費國尤其重要,拿來平穩油價是本末倒置。所以中方的頭號選項是根本不要理美國的要求(除非美方提供重要交換籌碼,例如承諾要遏制Lithuania,但現實裏這是不可能的);如果非要答應美國空口白話的使喚不可(請注意,這個可能性正對應著我最近討論過的戰術和解的壞處),也應該只象徵性地少量釋放。與此同時,繼續積極建設油氣儲存設備不可鬆懈。

Biden的用意,主要在於平穩物價,以減低通膨。最新的統計結果依舊讓經濟學界的鷹派和鴿派各執一詞,而後者主要指出通膨集中在少數大幅漲價的產品類別。然而我認爲整體數字之所以還不太嚇人,正是因爲有些產業的明面價格反而下降了,真正的重點在於這些似乎通縮的產品(例如旅游業),才是受疫情影響而暫時失準的部分,實際的通膨壓力是高於賬面數字的。Biden一意孤行,不斷增加赤字,美國所面臨的經濟崩潰危險越來越大。

當前能源短缺有三個原因:首先是去年疫情開始之後,有全球性普遍的大幅減產(不是封存,而是停止探勘和開采),然後歐美的財政刺激反過來創造了消費的新歷史高峰,短期内供給無法跟上需求;其次,中東產油國和美國頁岩油氣企業的財務都有嚴重困難,現在一方面終於可以享受一下較高的價格,另一方面也還沒有意願追加投資;最後,是美國頁岩油田經過連串的瘋狂擴張,低價高產的優質油井已經所剩無幾,平均開采成本在隨技術進步而逐步下降了十幾年之後,反而開始緩慢回升,更加減低增產的速度。至於美國在中東的影響力,倒不是主要因素,例如你去看Saudi的產能利用率,原本就很接近100\%,並沒有短期内可以迅速增產的可能。
\\

\textit{\hfill\noindent\small 2021/11/21 18:10 提问;2021/11/22 03:09 回答}

\noindent[14.]{\Hei 答}:中方引入華爾街文化,鼓勵資本做公關炒作、在股市大割韭菜、並且向地方政府騙補,對發展半導體這類的長周期產業危害很大,是過去20年進展有限的主因;追根究底,又是把關聯性錯當因果,只因爲英美先進國家搞自由市場,就以爲自由詐騙是他們科技先進的原因,事實是剛好相反的:是因爲他們科技先進,又擁有金融霸權,可以承受浪費,所以公關詐騙沒有嚴重後果。

Foundation裏的科學教,是拿科學的皮毛表象來做反智忽悠,合適的類比在於白左教。
\\

\textit{\hfill\noindent\small 2021/11/21 19:53 提问;2021/11/22 03:37 回答}

\noindent[15.]{\Hei 答}:好的,那麽《Reuters》談釋放原油儲備這件事,就和《FT》報導中美核武談判一樣,是美方有意造謠,一方面供内宣消費,另一方面對中國間接施壓。這種施壓方式當然是很幼稚的,但美國掌權的精英在高中和大學裏都是Bullies,習慣搞這些花樣;英國的Johnson也是如此。

未來幾個月,美方可能因爲通膨問題纏身,不得不在關稅上做出退讓,但這和中方是否接受政治外交上的和解假象毫不相干;事實上,如果中國也根據己方利益來對個別方向選擇脫鈎(外交)或談判(經貿),反而可以得到更好的條件。我預見沒有邏輯因果概念的社科“專家”會把關稅和解拿來吹噓他們的綏靖政策,所以提早在此澄清真相。

汽車的芯片需求,可以保障供給鏈的生產底綫,但要獨力支撐先進的半導體產業是遠遠不夠的。
\\

\textit{\hfill\noindent\small 2021/12/04 00:45 提问;2021/12/04 03:49 回答}

\noindent[16.]{\Hei 答}:你想多了。我不是反復解釋過,關稅是中美之間唯一可能達成的實質進展?中方只是在明年美國中期選舉之前,加緊推動此事罷了。

我同樣反復解釋過,美聯儲面臨的是資助財政赤字和避免通貨膨脹的絕對矛盾,而且前者的優先順序是高於後者的,所以至今依舊在進行QE,只不過規模有所收縮。因爲這已經極度偏向寬鬆,要反過來擴大QE是不可能的。
\\


\section{【美國】【國際】2022年國際局勢的回顧與展望}
\subsection{2022-01-25 11:39}


\section{24条问答}

\textit{\hfill\noindent\small 2022/01/27 13:33 提问; 回答}

\noindent[1.]{\Hei 答}:不用急,我夢到這個建議已經正式被接受爲官方政策,開始多綫執行;但是和產油國談當然必須低調,而且需要一點時間。我說“多綫”,指的是跨越多個部會的全面協作。至於是誰托夢,不便奉告。總之,經過七年多的努力,將來博客的政論文章終於不必再對某些相關議題反復囉嗦。\\

\textit{\hfill\noindent\small 2022/01/31 22:15 提问;2022/02/01 02:23 回答}

\noindent[2.]{\Hei 答}:1.NGO無關緊要;學術管理的不作爲,有其深刻複雜的歷史背景和理論原因,我已經反復論證過了。

2.把去美元化和推動人民幣,混爲一談,的確很可能是引發不作爲的基本誤區。我一直懷疑(當客觀分析指出反復的低級錯誤,所以非蠢即壞的時候,我傾向於假設非專業人員是前者、專業人員是後者)是金融方面的主管人員利用專業暴政(Tyranny of Expertise)來欺瞞最高層,亦即當後者要求有動作時,他們故意把資源和時間浪費在提升人民幣額分的無用功(“無用”指在相關時段内)之上,例如數字化人民幣,以逃避采納直接打擊美元的有效手段。然而所有的金融工具,都有極强的網絡效應,亦即既有市場額分越高,交易摩擦(Friction of trades)就越低,所以試圖拿佔2\%市場的人民幣,單挑佔60\%的美元,已經是不自量力,數字化貨幣更加是全新的概念,指望全球金融系統迅速采納,毫無成功可能。去美元化是未來幾個月國安上的頭號緊急要事(我從七年多前就開始公開高調地倡議,現在已經被那些人拖到最後關頭,無可再拖),中方反而重點投入需要十年左右時間的長期發展,不但緩不濟急,而且已經構成資敵的賣國行爲。我雖然預期這些謬誤,在正文中再次指明正道,結果卻馬上聽説有部門又重施故技,把上峰的意志斗轉星移,轉投到不可能在一年内有任何結果的方向上去。這其實和假未來科技異曲同工,偏偏我是外人,即使事先把道理解釋清楚,内賊事後要在細節上做扭曲,我也沒有管道置喙。

3.這事對中國崛起來説,還算是相對的小事,該説的我也早都説過了,台灣内部鉗制實話的新政策,更加不方便我繼續評論,請不要在此發泄情緒。
\\

\textit{\hfill\noindent\small 2022/02/01 22:44 提问;2022/02/04 07:51 回答}

\noindent[3.]{\Hei 答}:還有幾個月時間,中央也已經理解前因後果,所以我仍然持審慎樂觀態度。


根據《觀察者網》報導(\href{https://www.guancha.cn/internation/2022\_02\_03\_624801.shtml}{链接\footnote{\url{https://www.guancha.cn/internation/2022\_02\_03\_624801.shtml}}}):“普京在华出席冬奥期间,中俄两国将就加强天然气和金融合作议题展开讨论”;讀者可以與正文對照。
\\

\textit{\hfill\noindent\small 2022/02/02 03:34 提问;2022/02/02 12:19 回答}

\noindent[4.]{\Hei 答}:保護自己和打擊美元有很高的重叠;主要差異在於眼光是否長遠,是否能看到幾步之後的棋局。

我在正文中列舉建議,已經考慮過時間緊迫,所以都是速成的步驟,部分還需要美國經濟危機作爲幫襯。至於長期性的措施,早年已經錯失時機,未來則是做完緊迫手段理所當然的後續動作,均屬當前無需再提的類別。
\\

\textit{\hfill\noindent\small 2022/02/15 17:10 提问;2022/02/16 05:21 回答}

\noindent[5.]{\Hei 答}:在冷戰期間,美元並不是霸權的支柱,反而反過來,必須靠其他方面的美國霸權力量來强迫歐日持續接受美元的搜刮。美國在1990年前後,擊敗蘇聯、日本兩個挑戰者,都是靠文化、宣傳、經濟、軍事的全面壓力,誘導並脅迫對手采納自殺性政策而成功地從内部瓦解他們;美元匯率的操弄,是整個過程的通道,而不是壓力的原始來源。冷戰後初期,美國的政治軍事霸權更進一步踏上頂峰,美元得以超越以往受害的其他工業國,轉而搜刮新興國家,才有了1997年的國際金融危機。其後全世界爲了避免重蹈覆轍,普遍開始纍積超量的美元外匯儲備。

其實在Nixon打破Bretton Woods體系後的1970、80年代,歐日就已經爲了試圖減輕美元通漲的搜刮力道,而努力改用自己的貨幣,到1990年,美元的國際額分已經降到40\%。1990年日本的經濟崩潰和1997年的亞洲金融危機,徹底反轉了這個趨勢,到2000年,美元額分漲回到72\%,然後歐元的出現和2008年美國金融危機又使其轉爲下降。然而2010年起始的歐元危機,卻大幅減緩了這個進程,所以經過20多年,美元額分也只降到59\%。

多年前我在《美元的金融霸權》系列,向華語世界解釋清楚美元一放一收的周期性搜刮機制,這個波動周期對應著美國國内經濟的起伏,大約是5-10年。然而正如AM無綫電的高頻載波承載低頻信號,在美元搜刮過程中,也有一個更長周期的波動,對應著美元的國際額分,這個低頻周期大約是30-40年。當美元額分處於高點,搜刮的重點在於“放”,亦即印鈔通漲;等其他國家反應過來,開始改用其他貨幣,就必須把重點放在“收”,力求以金融結合政治軍事手段,製造各式各樣的國際危機,反過來逼迫世界建立高額外匯儲備,保障美元的地位。

所以單從貨幣角度來看,美國目前的處境,類似1979年美元超發再加上供給鏈問題(當時是第二次石油危機,現在是COVID)所引發的嚴重通漲;我認爲不論美聯儲主席由誰來當,拿著當年Volcker留下來的教科書案例,都會選擇專注在收緊銀根,所以未來兩年經濟蕭條的機率很大。新興國家雖然總體的外匯儲備有了顯著提升,但因爲40多年來工業經濟的普及和發展,可供壓榨的對象也大幅增加,其中金融體質較弱的,如土耳其、南非、巴西和墨西哥,依舊足以讓美國大量的吸血。

中國的應對之道,在於一方面聯合金融體質較强的所有非昂撒工業國,以高於當年歐日的速度進行去美元化(這是最近幾年博客評論,包括這篇正文,的核心建議),另一方面在美國成功引發弱國經濟崩潰之後,出手截胡,阻斷美國財團賤價兜底收買優質民族資產的企圖。這裏最理想的手段,是有自己的國際金融穩定機構,例如亞投行;問題在於正確的運作,是對國際金融危機的既有仲裁機構International Monetary Fund做替代或至少競爭,而當前的管理人卻把亞投行搞成專責扶貧的World Bank的低級代用品,拿國家的錢在金融戰綫的和平時期就當凱子亂花掉了。我在多年前已經反復批評過這個錯誤,現在事到臨頭,來不及再改,所以正文中也就沒有提及。緊急機制和其他管道也還是有,但可以等這一輪經濟危機態勢明朗化之後,再做針對性的建言。
\\

\textit{\hfill\noindent\small 2022/02/16 07:55 提问;2022/02/16 13:36 回答}

\noindent[6.]{\Hei 答}:美國利率在過去40年波動過程中,一波比一波低的現象,其實主要反映的是通膨壓力的消失;而通膨之所以消失,是工業外包、導致中國廉價高效的勞動力取代美國工人的結果;參見博客既往的評論。Trump針對中國所設立的關稅等等貿易壁壘,固然遏止了中低產業向中方轉移的進程,但It's too little, too late,而且反而帶有加劇通膨的副作用(對這一點我也早就預期、並且反復强烈批判過Biden的不關注態度)。所以未來兩三年,美聯儲加息必然會超過2018年的尖峰,甚至2007年、1999年的記錄都有可能被打破,但並不保證美國經濟100\%會摔下斷崖:這裏的不確定因素,除了美國自己之外,中、俄、歐的貨幣和貿易政策都會有影響;這也正是我寫這篇正文的主旨。
\\

\textit{\hfill\noindent\small 2022/02/16 11:28 提问;2022/02/16 14:35 回答}

\noindent[7.]{\Hei 答}:經濟和金融由許許多多利益對立的玩家組成,一般是高維度的博弈論,不能只做初級的綫性擾動分析。你所談的美元利率上漲超過實體產業投資報酬率,就只是初級效應,一旦廠商無利可圖,首先淘汰弱者,接著必然導致報價上升,反饋成爲新的通膨壓力。

利率和通膨有很強的交互作用;美聯儲利率會提升多高,取決於控制通膨的需要;但是美國已經不再生產許多民生消費品,提升利率固然會壓縮大筆開支,但基本消費沒有什麽彈性,一旦因上一段落所討論的機制而持續漲價,美國國内的底層員工生活水平就會兩頭受壓,逼迫聯邦政府增加緊急福利支出,從而擴大赤字,讓美聯儲在利率決定上陷於兩難。這其實是我以前反復討論過的美國經濟、金融和財政胡搞40多年的惡果,説得稍微詳細一點罷了;你的論述只是這個邏輯鏈的一個小環節,對博客的主軸議題來説沒有什麽重要意義,爲了賦予把你留言保存在博客的價值,我已經很耐心地一連詳細回應了三次,請你暫時收斂,不要占用太多時間。

至於美國企業界的垃圾債券,這倒是值得討論的議題,不過答案也很簡單:總額不到2萬億美元,其中一半是過去兩年内發行的,所以還不會很快到期。雖然SEC打壓SPAC為股市泡沫提前泄了一點氣,使得債市成為當前的焦點之一,但既然短期内會到期的只是1萬億美元,對美國經濟體量的占比實在不大,只靠它一個泡沫並不足以引發連鎖反應,所以即使金融危機表面上從此開始,也不代表它是危機的真正源頭;美國的財政金融困境,是40多年來全方位腐敗衰弱的結果。
\\

\textit{\hfill\noindent\small 2022/03/06 22:34 提问;2022/03/07 04:45 回答}

\noindent[8.]{\Hei 答}:因爲新博文可能還要十天才能刊出,我在這裏補充一點細節,解釋一帶一路爲什麽和如何過時。

一帶一路的基本思路,是由中方提供技術和初始資金,幫助落後國家進行開發,尤其是基建。那麽一方面這些國家得以加速發展經濟,成爲中國出口的潛在客戶;另一方面中國則短期能利用過剩產能,長期可以間接獲得更大的國際影響力;尤其中歐之間的交通綫,更是一帶一路建設的重點,兼具整合歐亞大陸的潛能。當年建立亞投行,遵循的是世界銀行的扶貧模式,就正因爲它被構思為一帶一路的一個環節。

這個策略的問題,在於它築基於一個隱性假設,亦即全球外交經貿規則的制定和執行,在時間軸上相對穩定,在空間維度基本完整,在原則上大致公平互利,以及中歐經貿關係不可能完全破裂。然而實際上,當前國際政經治理體系是歐美先進國家的俱樂部,對新興國家的參與只做了非常表面的敷衍工作。在冷戰結束後的全球化階段,美國在面子上有矜持,歐盟也有務實的領導,所以一帶一路還可以發生作用。一旦Trump撕毀昂撒集團的假面具,隨意出手對外做出經貿壓榨,立刻就暴露了國際體系的Impotence無能,WTO和國際法庭的癱瘓是最突出的案例。現在歐盟為英美站隊,則是更進一步將既有國際行爲規則從“無力制約昂撒”演變為“昂撒對外打擊的工具”,如果繼續埋頭做生意,是保證會被先搶劫後謀殺的被動反應。在這個新時代、新環境之下,中國外交戰略的重點必須改爲對歐美掌控下國際規則體系的所有主要環節做出替代,例如SWIFT。進一步考慮,我們可以拿IMF來和世界銀行對比:前者其實是不挂名的國際破產法庭兼Lender of the last resort,享有後者完全欠缺的規則制定權和執行權。當年亞投行專職搞扶貧的時候,客觀威脅還沒有明顯化,我只表達了潛在的憂心,說那是金融和外交資源的次優使用;幾年發展下來,已經可以認定的確是完全錯誤的選擇。


答案其實都在新博文裏,這裏我先簡單回復:

1.是,一帶一路已經過時(其實從2017年,Trump撕破“Rule-based international order”的假面具,針對性地直接出手打擊挑戰者時,就已經過時了;歐盟為昂撒霸權站隊,只不過是徹底消除了中方以拖待變、矇混過關的戰略選項);不過新博文不能直説,所以選擇“必須更進一大步”這樣的委婉用語。

2.哈哈,現任霸主以心狠手辣、下流無恥著稱于世,又剛剛得到所有老工業國的背書,你覺得中國有避免戰略投入的餘裕?

3.不但不必考慮避免接盤,而且反過來不得不另找替代;中國的實質盟友不嫌多,而是怕不夠。俄國若孤立無援,尚且會有經濟全面崩潰的危險;以中國對外貿依賴之深,哪可能承受得起被踢出國際經貿體系的打擊?美國有歐盟站隊,會找不到全面制裁中國的藉口嗎?中方應該賦予美國對不聽話國家各個擊破的閑暇,還是提前主動聯合所有潛在受害者,預做防範?這裏和昂撒霸權前例的根本差別,在於經貿同盟只要擴張有序、管理得法、不揠苗助長,可以有益無害,只有軍事同盟才必然會有反噬自身的危險。
\\

\textit{\hfill\noindent\small 2022/03/09 09:32 提问;2022/03/09 10:52 回答}

\noindent[9.]{\Hei 答}:Stiglitz近年加入了支持無限印鈔的陣營,原因不是他看不出會有通脹,而是他認爲通脹能幫助最底層的美國民衆削減欠債。作爲同樣關心低收入群衆的社會主義者,我因此也曾考慮過通脹對貧富不均的影響,結論卻是與Stiglitz相反的:這是因爲富豪階級自然有辦法把財富轉移到能抵抗通脹的資產,例如Bill Gates從去年開始就大筆搶購農場地產;真正底層的民衆連貸款都沒有資格,通脹的主要效應是提高他們的基本生活費;普通領薪水的中產階級,只有很有限的儲蓄通道,更是最大的輸家。所以我在新博文中,明確地指出中國必須盡力將美聯儲印鈔引發的這一場全球性通脹排除在國境之外,人民幣有序升值是必要的手段之一。
\\

\textit{\hfill\noindent\small 2022/03/09 14:19 提问;2022/03/10 02:10 回答}

\noindent[10.]{\Hei 答}:通脹,尤其滯脹,必然大幅壓低各式各樣金融資產的價格,這是Stiglitz鼓吹印鈔的考慮之一。問題在於有辦法找到能抵抗通脹的實體或國際資產的還是財閥,我已經給了Gates的例子。

在2008年金融危機之後,美國國會通過了Dodd-Frank法案,包含Volcker Rule,禁止投行自己下場炒作資產。後來雖然被各大投行聯合起來修法,挖出一個大漏洞(參見前文《富豪口袋裏的國家》),但已經來不及挽回炒作金融資產的統治性市場額分,被新興的Private Equity取而代之;後者的體量在過去十幾年中成長五倍,超過了10萬億美元,例如總量1.8萬億的Junk Bond中Private Equity就佔了過半。換句話說,2022年Private Equity的市場角色,相當於2007年的投行;所以我懷疑在這一輪將至的新危機中,扮演Lehman的會是一個Private Equity Firm。届時聯邦政府和美聯儲必然又會在事後拿國家資源來補貼大財閥,這才是他們竊國的模式,不是像你所想的那麽有遠見。
\\

\textit{\hfill\noindent\small 2022/03/09 18:03 提问;2022/03/10 02:14 回答}

\noindent[11.]{\Hei 答}:誰來承受美國人無限印鈔造的孽,取決於美元的國際地位是否被推翻;而美元的地位是否在這一輪全球危機中動搖,取決於中方能否采納我在新博文中的建議。
\\

\textit{\hfill\noindent\small 2022/05/07 10:31 提问;2022/05/10 01:32 回答}

\noindent[12.]{\Hei 答}:我説過很多次,預期經濟衰退,就如同預測雪崩一樣:只能斷定它必然會在一個大致時段内發生,但實際發生的確切時間和方式,卻不可能準確説定,因爲這類崩潰過程是混沌現象(Chaos Phenomenon)。

美國第一季的GDP負成長,是美聯儲去年底終於明白自己放水太多之後,緊急Taper的後果;一旦發現經濟承受不住,Powell又反過來只加息50bps。客觀評估只能確定他們無法走鋼索到底,至於是從哪個方向墜落深淵,要視美聯儲未來的決策而定。既然這些未來決策還沒有發生,旁觀者當然不可能準確預言。不過這裏是二選一,所以不明就裏的傻子反而可以很高興地隨意猜測,也有50\%猜中的機率;相對的,一個理性的邏輯分析者,面對絕對隨機的混沌現象,應該指明這個事實,然後拒絕參與這個胡猜游戲才對。

我的確擔心中方繼續配合美元霸權的吸血機制,主動幫助美國渡過今明兩年的金融財政難關。博客不是一直說,蠢往往比壞還要糟糕嗎?

我以前也早解釋過,除了軍事和金融之外,想不出世界要如何擺脫昂撒霸權的桎梏。軍事衝突的風險太高,但願是貨幣革命解救人類。
\\

\textit{\hfill\noindent\small 2022/07/21 16:06 提问;2022/07/21 23:30 回答}

\noindent[13.]{\Hei 答}:這是因爲政治是權力精英的游戲,而且必須拉夥結黨,玩家數目並不多,例如在最新後注的討論中,英國政壇真正的勢力就是五個(昂撒金融、非金融、猶太金融、愚民直覺反應、和少數殘存的良心人,而且後者在Corbyn下臺之後已經無力參與權力的游戲);而經濟裏做獨立動作的參與者,數目遠遠高得多,這不但立刻高度複雜化,而且容易產生囚徒困境等等悖論。

另一個差別是,玩政治的對己方的行動目標認知很清楚,如果能觀察上一段時間,通常可以確定他們的Modus Operandi;經濟則由於題材複雜、深奧,再加上主流學術理論被長期扭曲,個別玩家做出不可預期的隨機非理性動作是家常便飯。

所以總結來說,我對前者的預測有時可以精確到個人和月份,對後者則只能談必然的大趨勢,而且有+-一兩年的不確定性。
\\

\textit{\hfill\noindent\small 2022/07/23 03:39 提问;2022/07/23 04:08 回答}

\noindent[14.]{\Hei 答}:有關囚徒困境,你所舉出的是因素之一;更基本而廣汎的考慮在於參與經濟的普通個體(含大公司的主管),連綫性優化問題都沒法理性解答,更別提非綫性題目。尤其囚徒困境的最優解,依游戲的回合數而改變,但估計回合數往往要求對長遠未來的大環境做出正確估算,這本身就是一大難題(參考幾個月前臺積電設立政治風險評估部門,基本必然是浪費時間和金錢,問題只在於怎麽浪費、和浪費多少)。此外,有效市場假説所依賴的套利機制,在一般經濟活動中是不存在的。

至於你對相變的評論,很有意思,但我想提醒你,這種類比聯想,切忌推得太廣太遠。博客談類比,基本都是爲了教學方便;真正嚴密的邏輯論證,是必須講因果的。
\\

\textit{\hfill\noindent\small 2022/08/12 22:39 提问;2022/08/13 04:44 回答}

\noindent[15.]{\Hei 答}:Aramco的事我知道,習近平訪問Saudi我原本還沒注意到,謝謝提醒。這的確是跨時代的重要新聞,我過去8年(《美元的金融霸權》其實是2014年十月寫的)倡議的大戰略終於被實現,美元霸權的落幕正式開始。可惜因爲歐盟的愚蠢自殺政策,我們必須多等幾年,讓歐元區先倒退為第三世界(參考100年前的拉美),不過整個西方體系的衰敗過程,當然依舊遠遠用不著一個世紀。
\\

\textit{\hfill\noindent\small 2022/10/04 07:18 提问;2022/10/04 09:37 回答}

\noindent[16.]{\Hei 答}:沒有。

土豪有足夠的政治動能,可以拿國運來冒險賭博,試圖推動再一次減稅矇混過關;但一旦證明國家承受不起,其他利益集團就會聯合起來反對,土豪自己也不想在全面恐慌中被雪崩式的經濟危機所掩埋。
\\

\textit{\hfill\noindent\small 2022/10/10 16:28 提问;2022/10/11 05:52 回答}

\noindent[17.]{\Hei 答}:“自由”“民主”的牛皮吹太久,連自家的百姓也都信了,右翼藍領階級覺得應該身體力行,保障自己的“人權”,結果和不想得新冠的幕後財閥統治階級發生局部矛盾。背景裏也有試圖利用群衆對體制的不滿來為自己爭權奪利的民粹政客和自媒體人。
\\

\textit{\hfill\noindent\small 2022/10/19 23:07 提问;2022/10/21 08:57 回答}

\noindent[18.]{\Hei 答}:我不覺得這超出了博客的討論預測範圍。

我在2019年就詳細並精確地預言,這次危機的基本核心在於通漲,而這個通脹來自需求面和供給面的雙重壓力。博客討論過的美國對外吸血資金、企業和家庭儲蓄豐厚、大銀行預備金充足等等正面因素,都只對2000年和2008年那類純粹泡沫爆破有幫助,對解決通脹反而有一點負面效果。只要通脹不緩解,美聯儲就必須繼續加息和量化緊縮,債市也就只能跌跌不休,那麽投資人自然寧可抱著現金不下場,即便現金的購買力也在不斷縮水。所以國債市場的虛弱,是美聯儲出手太晚的必然後果,是問題的表徵而不是根源。
\\

\textit{\hfill\noindent\small 2022/10/23 17:19 提问;2022/10/23 21:54 回答}

\noindent[19.]{\Hei 答}:知之爲知之,不知為不知;中共内部高層人事屬於我所不知的範疇。連帶地,除非讀者有客觀事實證據要分享,請不要討論主觀臆測、分析或傳言,因爲我無力仲裁。

我想順便指出,這次人事變動,一個很值得注意的重點觀察,是人民銀行系出身的金融貨幣主管全部退休替換。
\\

\textit{\hfill\noindent\small 2022/10/24 10:19 提问;2022/10/24 12:59 回答}

\noindent[20.]{\Hei 答}:即使是全面而且徹底的政策轉向,也不會將多級管理人員100\%替換。一次換掉三個最高主管,已經是非常驚人的人事變動;像這類個別中層經理的留任或晉升,並不足以作爲任何邏輯推論的根據。
\\

\textit{\hfill\noindent\small 2022/10/26 04:54 提问;2022/10/26 09:13 回答}

\noindent[21.]{\Hei 答}:
這有兩個可能的解釋:首先,瑞士的經濟規模相對很小,對金融的依賴又深,在美國成爲最“安穩”的先進經濟體的背景下,資金流失在所難免,所以中央銀行未雨綢繆,先拿些額外美元作爲儲備,並非不合理。另一個可能性則是文章中提到的,瑞士法郎的Overnight利率明顯高於美元,所以大銀行會設法借美元、存法郎來賺取利率差。我個人認爲110億美元數目不大,還不足以因而下任何斷言。
\\

\textit{\hfill\noindent\small 2022/11/05 12:38 提问;2022/11/07 09:56 回答}

\noindent[22.]{\Hei 答}:
CAI作爲錨定中歐全面經貿合作、確保歐盟在霸權轉移過程中保持中立地位的地緣戰略意義,早已一去不復返。就算它再度被提上日程來討論,國際社會物是人非,其對中國的重要性也必然大幅下降,所以沒有多做臆測的必要。
\\

\textit{\hfill\noindent\small 2022/11/07 11:12 提问;2022/11/08 14:09 回答}

\noindent[23.]{\Hei 答}:我相信幕後的談判折衝一直都在緊鑼密鼓地進行,雖然很可能因爲中方專業官僚的消極抵制而有所拖延,最終必然還是會有所成。不過這裏的關鍵還不止是中俄,Saudi的加入才是美元霸權的喪鐘;如果習近平真能在年底前做出訪問,將是極爲正面的發展。
\\

\textit{\hfill\noindent\small 2022/12/05 20:02 提问;2022/12/05 23:06 回答}

\noindent[24.]{\Hei 答}:我已經解釋過了,能負擔得起(例如50嵗以上有足夠儲蓄)的人提早退休,總勞動力來源減少個5-10\%,就不是捉年輕男性能填補的,尤其這些所謂的藍領工作往往還有一些技術含量。
\\


\section{【科技】量子通信和計算是中國學術管理的頭號誤區}
\subsection{2022-02-09 05:30}


\section{1条问答}

\textit{\hfill\noindent\small 2022/07/29 13:41 提问;2022/07/29 23:48 回答}

\noindent[1.]{\Hei 答}:2015年股災我判斷監管單位的處置不合理,結果果然是貪腐。過去幾年抱怨半導體產業政策胡搞,現在證實又是貪腐。狹義來看,一次抓了四五個,是清理門戶的好事。然而拖了這麽多年,騙補都已經形成產業鏈了,依舊讓人難過,更別提對國家的嚴重危害。中國縱容思想腐化太久,笑貧不笑娼的風氣深入各個領域;我批評學術界造假誇大,並不是迂腐,而是簡單預見了這類思想腐朽的後果。光是事後抓人,猶如竹籃提水,空費力氣;必須事事嚴抓,以建立健康文化,讓業内人自相監督,才有長遠的效率可言。

昨天上唐湘龍節目,不知道直播已經開始,以爲還在私下聊天,談到此事,其實並非我的本願。這是因爲這類突發新聞,我處於信息鏈的下游,在能夠多方搜證、補齊視野之前,不應該妄自評論。節目内討論半導體的部分,我就有意回避中方的管理問題。

此外,一般人可能沒有注意到,我上《龍行天下》,視頻訊號通過網絡繞行地球,來回有近半秒的延遲,尤其錄下來的是台灣視角的版本,所以雖然容易出現我和主播搶著發話的情況,並不是有什麽“不禮貌”或“欠缺默契”的現象。
\\


\section{【戰略】【國際】從SWIFT制裁俄國,看中國的對應之道}
\subsection{2022-03-11 15:08}


\section{78条问答}

\textit{\hfill\noindent\small 2022/03/13 02:35 提问;2022/03/13 05:20 回答}

\noindent[1.]{\Hei 答}:新興工業國擔心的是進出口,不是GDP,所以名義GDP超出美國只有正面的宣傳效應,不過依舊只是宣傳效應,不值得過度重視。
\\

\textit{\hfill\noindent\small 2022/03/16 01:34 提问;2022/03/16 10:33 回答}

\noindent[2.]{\Hei 答}:昨天已有朋友私下問過此事。我認爲的確是沙特表態,願意跳槽,現在就看中方如何把握這個機會。

更早已經有消息證實,EAEU(前蘇聯成員國之間的經濟合作組織)在三月十一日和中方達成協議,將共同建立“全新的國際貨幣和金融體系”,計劃的核心是一個合成貨幣。再早一天,王毅和歐盟國家外交部長通電話,討論協同調解歐俄爭端,選擇的對象是法國和意大利,但不包含德國。中國政府需要的只是好建議,有了正確方向之後執行效率真是很高,所以我們又一次看到問題出在智庫和學術界。
\\

\textit{\hfill\noindent\small 2022/03/22 09:40 提问;2022/03/23 04:46 回答}

\noindent[3.]{\Hei 答}:當年陳水扁爲了炒弄選舉而公開搞臺獨的時候,美國還有意願踩刹車;現在的美國執政團隊,以及他們所操弄的第二綫炮灰如德、日、韓,都已經把刹車系統拆下來扔進垃圾堆,如果蔡英文不趕快長出一點戰略智慧,在面對戰爭懸崖時依舊只聽命於主子的口令,那麽軍事升級就成爲無可避免的選項。然而當前英美集團本身内部問題極度嚴重(參見過去幾年的【國際】和【戰略】文章,包括這篇正文和尤其是【後注二】;這些因素可能導致理性的自顧不暇,也可能導致非理性的狗急跳墻),俄國在烏克蘭也不尋求速戰速決,未來幾個月的歷史發展方向有很大的不確定性;不過在俄烏衝突期間另外向中國挑起嚴重事端,違反英美行爲模式,發生機率並不高(亦即即使栽贓中國軍援俄方,也只是爲了施加宣傳和外交上的壓力,並不是鬥爭的主軸),真正要造假來强迫升級對抗,可能還是會如【後注一】所討論的那樣針對俄軍以“化學武器”為藉口。
\\

\textit{\hfill\noindent\small 2022/04/04 12:36 提问;2022/04/04 17:25 回答}

\noindent[4.]{\Hei 答}:這似乎是延續過去十年,要“與國際金融體系接軌”的一系列政策,或許有什麽玄妙的配套手段還未公佈,但當前我根據公開事實做分析無法認同,邏輯論證細節請大家參考《我對引入美國投行的一些看法》。我想等二十大之後,金融戰略決策人員汰舊換新,再做進一步的評論,現在只能祈禱這批一心想要和華爾街挂鈎的人,至少不要否決創立亞元的建議。
\\

\textit{\hfill\noindent\small 2022/04/04 20:22 提问;2022/04/05 07:01 回答}

\noindent[5.]{\Hei 答}:我昨天已經解釋過了:中共政府新舊交接,是掃除接受昂撒宣傳洗腦舊官僚的一個契機;問題只在於危機已經提早到臨,出手反擊的機遇稍縱即逝,我盡力把核心的思路和方案反映出來,其中又以建立亞元最爲關鍵,希望不要因肉食者的固化思想而被擱置,畢竟天予弗取,反受其咎。
\\

\textit{\hfill\noindent\small 2022/04/07 12:37 提问;2022/04/07 13:41 回答}

\noindent[6.]{\Hei 答}:亞元如何設計,牽涉到的專業考慮太過艱澀,所以我在正文裏只草草帶過,反正人民銀行的主管一定會被咨詢,他們懂得其中的奧妙,自然應該同意是最優解,這裏只粗淺解釋一下。像是歐元這樣沒有財政聯盟(亦即各成員國的稅收和支出相互獨立)的合成貨幣,最大的問題在於兩點:首先彼此之間匯率固定,無法視各經濟體的經濟健康狀態和貿易順逆差來做調整;其次各國的財政順逆差也有不同,嚴重逆差必然導致貨幣增發,最後爆發希臘式的危機。歐盟至少還有統一的市場監管標準,而且有長遠、持續的整合意願;中國建立新國際貨幣的主要用意,卻是在於取代美元和歐元,做爲鬆散國際社會的公平通貨,所以這個新貨幣聯盟也必須很鬆散,尊重各國的主權和自由。

用多個國家貨幣組成的籃子來作爲新合成貨幣的本位,自動滿足上面的所有要求:各國仍然保有自己的貨幣在國内使用,匯率可以浮動,增發完全自由,亞元銀行只要預先訂立規則抵抗要求救援資助(執行上是增發亞元來換該國貨幣,等於是人爲地維持其成分占比;所以股份不能固定或只依GDP計算,而必須考慮貨幣濫發的處罰)的政治壓力,就不會有金融上的大損失和危險。參股的國家甚至可以隨時退出,也可以選擇投入的深度,而且成立快、彈性大,是遠遠最優的方案。資源國家現在看來是最重要的成員,其實是世界經濟周期進入整體通脹階段所引起的暫時現象,十幾二十年之後局面可能完全不同,不過難關在當下,所以爭取他們確實是首要優先。中俄的軍事力量也的確會是必要的保障,雖然不足以全面對抗美軍的全球優勢,但在亞洲保護不被顔色革命輕易推翻,應該還是做得到的。至於不作爲,那等同於把幾十年纍積的國際資產送給美國當人質,是明顯的下下策。
\\

\textit{\hfill\noindent\small 2022/04/08 04:39 提问;2022/04/09 00:22 回答}

\noindent[7.]{\Hei 答}:我對亞元的構想,的確是中國主導,爭取主要資源國合作,容許友好並富裕的中型國家有限參與,小國跟著用就是了,至少亞元銀行不會像歐美那樣去沒收外匯儲備。這裏要小心的是,如何處罰濫發自己貨幣的成員國。一個非常重要而簡單的原則,是中國必須勇敢站出來占據舞臺中心的聚光燈焦點,不要畏畏縮縮地搞多邊平等,否則必然是邀請自私國家來白占便宜。這其實是比不建立亞元更大的危險,因爲要看出建立亞元是正確方向並不困難,The devil is in the details。從這届中國政府搞盲目金融開放、無限擴張上合、一帶一路跪求參與、亞投行自願當凱子等等前例來看,這樣子把正確戰略決定在執行細節上搞砸的機率很高,所以我也故意在文章裏留下一個綫索:直白地批評了邀請印度入股的思想,但放在最後那個章節,以避免若干讀者看到一半就起反感。不過如果幾個月後國務院公佈,亞元銀行創始成員包括印度,我們就可以簡單判斷國家和人類整體利益又被某些思想扭曲的傻子懦夫打包白送了。

Biden這次聽從Hillary系的NeoCon,斷送美元霸權,金融系的不滿已經明白顯示出來,若是Obama系決定藉機落井下石,就可以簡單推出己方的Harris上位,這是我最近剛剛詳細討論過的事,讀者應該仔細閲讀,不要重複發問。


Harris依附Obama,然後一起對Biden捅刀的可能性,共和黨媒體也注意到了,大家可以參考以下這個視頻:\href{https://www.youtube.com/watch?v=j58Bx0Ebppw}{链接\footnote{\url{https://www.youtube.com/watch?v=j58Bx0Ebppw}}}
\\

\textit{\hfill\noindent\small 2022/04/19 11:49 提问;2022/04/20 05:24 回答}

\noindent[8.]{\Hei 答}:中國政府在過去一年多,一直默默地在清理地方財政和金融,尤其是地區小銀行被土豪當提款機的現象,排查起來非常吃力,又不討好。這其實是普世性的難題,其他國家只能等爆炸之後來收拾殘局(例如1990年美國的Savings \& Loan危機,以及2017年以來印度的影子銀行問題),中國勉强算是未雨綢繆,雖必須付出代價,但至少和外國對照下是較優的。招行當然不是小銀行,那麽或許清理金融界已經進入下一個階段。

Gonzalo失蹤之後,有些人說風涼話,認爲他太過招搖、自取其咎;其中大部分是原本就沒有什麽良心可言的,所以自然難以體會為傳播事實真相而犧牲奮鬥的思路,不值得我們理會討論。不過另外還有極少數是同路人,例如《Moon of Alabama》,多年來小心隱藏身份,似乎有資格做批評,但我想在此為Gonzalo辯解一下:和原本就專注在揭發政治黑暗面的MoA不同,Gonzalo並沒有事先計劃要做像是拆穿烏克蘭納粹政權謊言的工作,他的本行是經濟/金融方面的記者,多年來談的主要是歐美資本主義體系的剝削性,不但沒有必要匿名,而且必須實名報導才有公信力。他做公開評論已經很多年了,忽然戰爭在身邊發生、宣傳戰謊言滿天飛,他人剛好處在關鍵地點,有許多獨家的信息;不談的話,對不起良心,匿名談的話,沒人理,所以繼續用既有的管道是唯一的選項。對這樣的選擇,我只能說很佩服他的勇氣和執著;譏嘲他不夠謹慎的,連事後諸葛亮都算不上。

我自己在八年前開始寫博客的時候,也只想要糾正台灣公共論壇的一些錯誤認知,用實名絕對是有助於建立公信力的做法。後來博客討論的越來越廣汎深入,開始碰觸尖銳敏感的話題,要退縮或匿名重來,已經太晚了;這時只能把局面看作考驗自己良心和勇氣的挑戰,在不影響傳播真相的前提下,盡可能避免招惹事端,例如我至今仍然沒有從大陸的任何機構(不論是公家或私營)收過一分錢。這樣的慎重當然不足以保證絕對安全,但如果沒有願意為理想和公益犧牲冒險奮鬥的知識精英,一個國家民族還有前途可言嗎?
\\

\textit{\hfill\noindent\small 2022/04/28 14:44 提问;2022/04/29 07:45 回答}

\noindent[9.]{\Hei 答}:三月初正文發表之後,我收到體制内金融方面人士的私下來信,說人民銀行的主管因循苟且、目光短淺,不會自願主動出擊,只有在最高層强力要求之下,才可能有動作;然而20大在即,許多核心執政官僚心有旁騖,所以正文中所討論的建議再怎麽合理,只怕短期内不會有結果。

我對中方政壇的内幕毫無所知,無法置評,只能希望那位專業人士的意見太過悲觀;不過從過去幾周人民幣匯率變動來看,或許是我自己太天真樂觀了。
\\

\textit{\hfill\noindent\small 2022/05/06 11:08 提问;2022/05/06 12:45 回答}

\noindent[10.]{\Hei 答}:這種無硝烟的全面金融戰爭,商業銀行哪有能力“防範”或“準備”?唉,反擊之道我在第一時間就解釋清楚,他們非要掩著耳朵、閉著眼睛去摸爬滾打,帶來的無謂風險和損失最後又是國民和人類一起買單。
\\

\textit{\hfill\noindent\small 2022/05/06 13:54 提问;2022/05/10 00:41 回答}

\noindent[11.]{\Hei 答}:先向所有讀者致歉:這個周末生病,在床上躺了兩天,所以沒有照顧博客。

一個龐大的官僚體系,内部人員素質參差不齊是難免的;事實上,有才華的必然是極少數。我以前反復論證過,中國政府已經算是選拔人才效率最高的。當然這是相對來看,所以有很大的比爛成分。美國固然社科界極度愚昧瘋狂下流,中方人員再怎麽尸位素餐也略勝一籌,真正讓人擔心的,卻是歐美的自然科學界還沒有衰敗到那個程度,而中國吃香喝辣的都是職業騙子。
\\

\textit{\hfill\noindent\small 2022/05/16 10:59 提问;2022/05/18 03:01 回答}

\noindent[12.]{\Hei 答}:和俄國央行的聯動,必然是最高機密,民間智庫學者不可能知道。

國内新冠疫情引發經濟下行壓力,稍作寬鬆有其道理。我構想的匯率操作,不是日常的利率調整,而是通過多國協作、政策宣言和外匯買賣來改變市場預期。


留言欄還在接受初級班,但前提是必須虛心發問,而不是趁機發表自己的長篇大論,也不能把博客當成網絡上聊天扯淡的論壇。你的另一個新留言違反《讀者須知》第二和第三條,刪了。嚴重警告一次,再犯拉黑。
\\

\textit{\hfill\noindent\small 2022/05/16 22:48 提问;2022/05/17 11:40 回答}

\noindent[13.]{\Hei 答}:“亞元”和美國制裁是兩回事:前者是從戰略上去刨美元的根;後者則是戰術上規避美國的打擊,可以簡單建立專責銀行和企業。
\\

\textit{\hfill\noindent\small 2022/05/19 05:15 提问;2022/05/19 07:21 回答}

\noindent[14.]{\Hei 答}:沒有證據或理由可以那樣解讀;畢竟他只是個技術官僚。
\\

\textit{\hfill\noindent\small 2022/05/20 15:40 提问;2022/05/21 02:02 回答}

\noindent[15.]{\Hei 答}:借此處宣佈:剛剛上了唐湘龍的節目,\href{https://www.youtube.com/watch?v=yD7HbsVBxYo\&t=5s}{链接\footnote{\url{https://www.youtube.com/watch?v=yD7HbsVBxYo\&t=5s}}},其中提到烏軍疲態畢露,未來幾周軍事局勢的演變,可能會對全球外交態勢有良性的影響;我下周末會上史東的節目,做進一步的討論。

此外,許多大陸聽衆對“過去20多年中國外交戰略上基本毫無作爲,所幸運氣極佳”那條論述非常不滿。博客讀者應該知道,這指的是2001年Rumsfield原計劃在2004年挑起臺海戰事、被911事件打斷;2008年金融危機後,美國虛弱不堪,以致雖然立刻開始宣傳動員,對中國做全面妖魔化,Obama/Hillary政權卻被迫反復推遲戰略圍堵,一直到卸任前才出臺TPP和TTIP;然後2016年Trump又意外當選,打斷了建制派的長期戰略部署,並且把對華敵意公諸於世;2022年,又是俄國主動觸雷,代替中方承受西方的新一波全面打擊。這裏的一系列正確戰略認知和反擊方案,都是我提前好幾年預先帶頭提倡,運氣好的,才有中國官方姍姍來遲,運氣不好的,例如當前的貨幣政策,當局還在拖泥帶水、猶豫不決之中。這其實是我一直不熱衷於上大衆媒體節目的主要原因之一:我做的是學術性、教育性的邏輯論證,那些節目聽衆習慣的卻是娛樂性、政治性的主觀反射動作,聽到新觀念,想的不是去原始來源追查細節、補充知識,而是坐地反噴。不過反過來看,明知普羅大衆是無可救藥的愚蠢,也要為國家和人類的前途,努力教育其中少數有理性思考能力的人,原本就是博客的使命;既然還有許多改革沒有被采納,我也只能强迫自己、捏著鼻子和非理性者打交道。


戰術被扭曲,是股市這類國内利益分配的體系會有的問題;貨幣政策則是大國博弈、人類歷史轉折的關鍵,即便有突破性的戰略建議,也必然需要體制内的專家先認可,才有被采納的可能。這裏的問題在於如果他們日常關心的是蠅營狗苟之事,哪可能有餘裕去考慮相關的宏大背景和複雜取捨?反腐固然是必要的糾正,但也會打破組織内部的理性氣氛,阻礙專業意見的客觀交流取捨,讓有心人藉機無限上綱,抹黑出事者的所有立場。例如孫國峰剛下臺,同一天《觀察者網》就有文章宣稱因爲孫反對Modern Monetary Theory,這代表著官方對MMT的肯定;還好幾個小時之後文章就下架了,應該是《觀網》編輯被提醒其作者別有用心,不應爲他站臺。
\\

\textit{\hfill\noindent\small 2022/05/21 07:06 提问;2022/05/21 07:28 回答}

\noindent[16.]{\Hei 答}:這裏的“標準經濟學理論”,指的是例如Krugman, Obstfeld \& Melitz所寫的《International Economics》這類的“國際金融”或“國際經濟”的教科書。其所用的模型極度簡化,完全脫離現實,應用起來絕對不能照本宣科,必須深入瞭解所有的隱性假設,否則政策分析的結論反而適得其反。
\\

\textit{\hfill\noindent\small 2022/05/21 13:51 提问;2022/05/22 12:27 回答}

\noindent[17.]{\Hei 答}:你猜對了。

中國學術界推銷MMT的論述,一般是强調商業銀行在收取存款、轉頭放貸的過程中,已經實質增加了通貨(M2),所以應該根據MMT把這個發行貨幣的權力完全歸公。這其實是徹徹底底的謊言:MMT並不真的鼓吹禁止商業銀行提升資產和(實質)發行通貨;而且要限制它們無序增長,早就另有一套機制(準備金)。MMT的核心論點,在於主權貨幣可以無限超發,而不會有任何不良後果;這種無中生有、憑空創造財富的好事,當然是美元這類享有極高杠桿率的國際儲備貨幣才做得到,而且實際上是在出賣祖傳的無形資產,亦即國際儲備貨幣的地位。
\\

\textit{\hfill\noindent\small 2022/06/10 12:34 提问;2022/06/12 04:19 回答}

\noindent[18.]{\Hei 答}:英國媒體比較嘴硬,美國政府、軍方、情報和媒體在最近幾周已經紛紛開始為戰敗卸責預做準備了。最新的是這篇《紐時》的文章:\href{https://www.nytimes.com/2022/06/08/us/politics/ukraine-war-us-intelligence.html}{链接\footnote{\url{https://www.nytimes.com/2022/06/08/us/politics/ukraine-war-us-intelligence.html}}},居然宣稱Zelensky把美國蒙在鼓裏;這很顯然是情報部門怕被Biden/Blinken/Sullivan等人拿來當背鍋俠,事先準備擊鼓傳花、把鍋丟給烏克蘭;當然這是Biden自己批准並采用的策略(參見\href{https://apnews.com/article/russia-ukraine-donetsk-education-business-85a2489b4fe1042cd03b0ae6f4421a04}{链接\footnote{\url{https://apnews.com/article/russia-ukraine-donetsk-education-business-85a2489b4fe1042cd03b0ae6f4421a04}}}),也是上個月我在唐湘龍節目裏就預言過的。
\\

\textit{\hfill\noindent\small 2022/06/12 14:48 提问;2022/06/13 03:11 回答}

\noindent[19.]{\Hei 答}:20世紀末的學術界建立了共識,確認地質變動、氣候變化和生物演化等等自然歷史的現象都有一個共同的特徵,也就是絕大多數時段處於穩定態,當這個長期穩定態終於崩潰的時候,過程是迅速而突然、而且細節上是隨機、混沌的。抽象來説,這是大Scale之下Phase Transition的自然結果:只要系統的規模夠大,就會有這個趨勢。其實人類社會也遵循類似的規律,所以後世的歷史學人才得以將過去劃分出不同的時代。

我們正在經歷冷戰後建立的全球化國際秩序的崩潰過程。如前所述,這個過程在巨觀上是必然、但微觀上是混沌的;例如美俄之間的金融戰和俄烏之間的軍事衝突,雖然有客觀的戰略利益脈絡主導,但執行細節依舊取決於極少數人主觀的一念之間,而執行細節的取捨卻是勝敗的關鍵;這些勝敗得失稍有出入,就影響世界歷史的未來走向。中國如果選擇不參與金融戰,那麽作爲一個被動的旁觀者,更加只能靜待主動玩家之間的鬥爭塵埃落定。所以要談戰略環境是改善或惡化,端視俄方是否勝出、勝出多少、如何勝出。已知事實指向審慎樂觀。

上面只能“審慎”樂觀的主要考慮在於,雖然過去三年我一直估算美方這次的滯漲衰退應該足以導致霸權失落,但這個邏輯推演是建立在美國之外的主要玩家都能避免重要、明顯戰略錯誤的假設之上。現在歐盟如此積極、主動、反復地采納自殺性政策,把原本是美元最佳替代品的歐元搞成先死一步的墊背,美國熬過這場劫難忽然不再是極小機率的事件;而中國在貨幣金融上的不作爲,如果持續下去,更可能會把那個機率翻轉過來,容許昂撒霸權苟延殘喘到再下一場金融危機。

武統時段是邏輯鏈的遠遠更後端,連美國這波經濟衰退的程度都還在未定之天,要精確估算其概率所需的假設層面太多、不確定性太高,毫無實際意義。我覺得和幾年前相比,可能Expectation Value沒變,只是Standard Error反而大幅增加了。(學過統計學的讀者可能覺得有點奇怪,武統是個“Dummy”或者“Boolean”變數,談何SE?這裏我考慮的理論Framework是所謂的“Limited Dependent Variable”,内含一個“Utility”或者“Latent”連續變數,對應著所有主要玩家對引發武統的綜合貢獻;因爲它是連續變數,所以可以談Standard Error。有興趣進一步瞭解細節的讀者可以搜索“Probit”函數,那是Limited Dependent Variable的一個簡化特例。)
\\

\textit{\hfill\noindent\small 2022/06/13 07:21 提问;2022/06/14 02:55 回答}

\noindent[20.]{\Hei 答}:其實我前兩年已經都詳細討論過:這次美國經濟衰退,正規銀行界早有預期,留下非常足夠的資金儲備,所以完全不會有類似2007、2008年Lehman Brothers崩潰的事件。中產階級受益於股市和房市,儲蓄也相當充足。真正的軟肋在於影子銀行界;他們利用上次金融危機後正規銀行的監管被收緊,在過去15年大幅擴張,一般估計光是Private Equities一類,净資產就超過十萬億美元。不過他們的杠桿借貸周期平均是兩年,所以不會馬上出現周轉危機,必須等美聯儲持續加息、收緊銀根,先收回目前閑置的兩萬億Reverse-Repo現金,然後在一兩年後才會開始吃緊(這個遲滯效應也發生在恆大身上),引發體系崩塌。

所以替代美元,是這個經濟周期就打破美國霸權的充分條件,而不是必要條件。然而如果美元繼續堅挺,那麽美聯儲會有一年多時間來收拾爛攤子,期間若是因爲從歐盟吸血、或出現其他利好事件,説不定美國國内通脹提早溫和化,那麽就可能又可以矇混過關。畢竟烏克蘭的人力物力資源有限,雖然目前俄軍進展緩慢,但烏方要撐過年底而不全面崩潰,機率基本爲零;其後能源和農產價格轉為疲軟,國際通脹壓力會有所疏解。
\\

\textit{\hfill\noindent\small 2022/06/22 19:16 提问;2022/06/23 01:53 回答}

\noindent[21.]{\Hei 答}:啊,原來俄方學術界也有人曾經想到這一點(我原本不知道,只是估算既然籃子貨幣是最優解,中方提議後,Nabiulina必然會支持),那就難怪今天出現消息,中俄已經準備要通過金磚組織搞正文中建議的合成貨幣,參見\href{https://www.guancha.cn/internation/2022\_06\_22\_645913.shtml}{链接\footnote{\url{https://www.guancha.cn/internation/2022\_06\_22\_645913.shtml}}}。當然這並不是完美的結局:如果依照我的建議,在三月俄方還打得很辛苦的時候,主動先正式提上外交談判桌,中方的姿態就會高得多。現在俄國已經大獲全勝,國際秩序的重組必須由俄方主導,並且必然讓印度坐上董事會,與中俄平起平坐。這是躲在後面讓俄國打前陣的缺點:風險低,但沒有功勞,在30年一次的國際架構重組過程中,平白損失了應有的話語權,徒然因爲俄方平衡中國的需要,讓印度白佔便宜,而且日後會有長期不斷的頭疼。

順便提一下,中方在這場第三世界反抗昂撒霸權的起義中,並非100\%以靜態的存在來“出力”:除了和印度一起購買俄國油氣、資助後者的外匯收入之外,其實還更進一步減低了自己成品油的外銷,所以對提高歐美通脹有一點額外的推動作用。不過這並不是出於外交大戰略的考慮,而只是一個純粹的巧合:亦即過去這年基於環保政策,持續減低煉油廠產能利用率,以致外銷量降低了超過一半;在當前的金融戰大環境之下,主要的影響是抵消了印度向歐洲轉出口產自俄國原油的成品油供應,維持全球煉油產能瓶頸,使得美國石油財團得以隨意加價,趁亂將煉油的毛利率從不到20\%一步提升至62\%,進一步增强了歐美的通脹壓力。

當然這種間接、附帶的貢獻,遠不足以拿上談判桌作爲決定未來國際架構話語權的籌碼。此外,這個作爲不但不是出於打擊霸權的考慮,而且反而是爲了取悅歐美白左,難免給予有識之士啼笑皆非之感,唉。
\\

\textit{\hfill\noindent\small 2022/06/23 17:43 提问;2022/06/24 02:32 回答}

\noindent[22.]{\Hei 答}:首先,你違反《讀者須知》第四條規則,警告一次。

Nabiulina身爲央行行長,她的首要任務是避免2015年資本外逃所帶來的匯率崩潰,和其後對經濟和政治的一連串連鎖打擊。現在這個危險已經過去,她自然會逐步下調利率。

在美國霸權虎視眈眈之下,如果不能打倒美元,最優解就只能是管制外匯;這是正文裏已經詳細解釋的道理。這次俄烏衝突的最主要意義,就在於終於有人領頭起義,打破被壓迫者所面臨的囚徒困境,讓第三世界能安全地合作起來,推翻歐美獨霸的既有國際體系。
\\

\textit{\hfill\noindent\small 2022/06/23 18:21 提问; 回答}

\noindent[23.]{\Hei 答}:你是經濟系的學生嗎?美國經濟學教科書依舊被普遍翻譯采納,用來教學,這是一個很大的潛在問題。我並不是說Keynes(除了他的學説之外,我對他的個性也很認同,畢竟他不但也是理科出身,喜歡解奧數類的數學題,而且常常不給情面、直接罵人笨;他的一位同僚曾經抱怨,Keynes自封爲經濟學界的Einstein不是問題,他把其他學者當白癡來斥責才是問題所在;我想指出,Einstein有整個猶太媒體界自動自發當免費公關,Keynes可沒有那個運氣)之後,昂撒學術界就沒有具備洞察力的經濟學者,而是整個行業從Milton Friedman開始,被資本徹底滲透掌控,研究方向預先被過濾,結論則被有意扭曲或以偏概全,以服從昂撒體系的資本利益(尤其是跨國資本)為最高原則。這裏就是很好的一個例子:三元悖論的研究主要來自歐洲各國之間從50-90年代的貨幣交易史,當時那裏最大的經濟體是西德/德國,而德國馬克遠遠達不到壟斷獨霸,地位甚至還不如英鎊。實際上正確的結論應該是,一個沒有全面全球霸權(包括貨幣、宣傳、外交和軍事;參見1985年廣場協議)的國家,不可能同時完成那三項金融政策目標。你針對俄國這次金融戰戰術所做的評論,細節基本沒錯,但必須放在前述的理論認知下來考慮。\\

\textit{\hfill\noindent\small 2022/06/23 18:21 提问;2022/06/24 06:13 回答}

\noindent[24.]{\Hei 答}:你是經濟系的學生嗎?美國經濟學教科書依舊被普遍翻譯采納,用來教學,這是一個很大的潛在問題。我並不是說Keynes(除了他的學説之外,我對他的個性也很認同,畢竟他不但也是理科出身,喜歡解奧數類的數學題,而且常常不給情面、直接罵人笨;他的一位同僚曾經抱怨,Keynes自封爲經濟學界的Einstein不是問題,他把其他學者當白癡來斥責才是問題所在;我想指出,Einstein有整個猶太媒體界自動自發當免費公關,Keynes可沒有那個運氣)之後,昂撒學術界就沒有具備洞察力的經濟學者,而是整個行業從Milton Friedman開始,被資本徹底滲透掌控,研究方向預先被過濾,結論則被有意扭曲或以偏概全,以服從昂撒體系的資本利益(尤其是跨國資本)為最高原則。這裏就是很好的一個例子:三元悖論的研究主要來自歐洲各國之間從50-90年代的貨幣交易史,當時那裏最大的經濟體是西德/德國,而德國馬克遠遠達不到壟斷獨霸,地位甚至還不如英鎊。實際上正確的結論應該是,一個沒有全面全球霸權(包括貨幣、宣傳、外交和軍事;參見1985年廣場協議)的國家,不可能同時完成那三項金融政策目標。

你針對俄國這次金融戰戰術所做的評論,細節基本沒錯,但必須放在前述的理論認知下來考慮。
\\

\textit{\hfill\noindent\small 2022/06/25 11:36 提问;2022/06/26 02:20 回答}

\noindent[25.]{\Hei 答}:你的理解正確。我喜歡偷懶,所以只提了一句廣場協議,行内人自然應該懂。

我的原創意見(主要是指出表面上沒有關聯,但實際上有因果關係的命題),在博客俯拾皆是,很多是推翻或修正學術界正統共識的,就等著年輕學者拿去寫論文。

發表論文時,請別忘了給博客一個Citation。
\\

\textit{\hfill\noindent\small 2022/06/26 08:23 提问;2022/06/26 23:53 回答}

\noindent[26.]{\Hei 答}:有關日本的必然衰退,我在2015年討論過金融和貨幣層面,去年談了因電動車興起而完全喪失工業競爭力的問題,這兩者都依舊有效,請自行復習。當前日本中央銀行的政策,只不過是不再將無可避免的崩潰向未來無限拖延罷了。

我對岸田文雄的樂觀評估,原本就聲明是小機率但無妨一試的類別。
\\

\textit{\hfill\noindent\small 2022/07/22 19:00 提问;2022/07/22 23:47 回答}

\noindent[27.]{\Hei 答}:視頻訪談,談不上“嚴謹”,例如Chrystia Freeland已經從外相升了副首相,而且二戰期間當烏克蘭納粹宣傳編輯的,不是她父親,而是外公。不過Freeland的確曾經公然撒謊,說她外公是二戰前就移民加拿大。此外,和阿富汗接壤的是蘇聯,而不是俄國。當然如果美軍留在阿富汗,還是必須擔心被俄國買人頭的。

金磚貨幣的前途如何,目前的不確定性太高,達不到我在《讀者須知》裏要求的70\%信心底綫,所以一直沒有在博客提。換句話說,20-50\%的估算,應該是Unbiased,但是Error很大。

有關歐盟是否會解體,我在節目中已經回答過了:應該會,但可能不是很快。以歐洲財富的底蘊,苟延殘喘一段時間並不太難。

替代IMF,不但在節目中已經明確回答,而且博客已經討論多年:這是建立新國際政治經貿架構的必需品,中方沒有提前佈局,把AIIB浪費掉,實屬極不明智的作爲,亟須加緊彌補。

道義性來自合理性,也就是價格不要太離譜。


囘了你的問題,才注意到你的賬戶太新;難怪問的都是我解答過的事。只好拉黑,罰你再等六個月。
\\

\textit{\hfill\noindent\small 2022/07/30 10:43 提问;2022/08/01 12:45 回答}

\noindent[28.]{\Hei 答}:正文中已經正面、而且多層次地未雨綢繆,例如第一層防護是不讓印度加入。然而現在主導新國際秩序的是俄方,那麽中國就只能在規則制定上下功夫。正文中已經討論了一些基本的原則,但中方似乎對防範會員國濫發貨幣的考慮,既不用心、也不積極,我們也就沒有必要進一步空談理想架構;就像過去20年半導體產業發展的案例那樣,内部到處都是蛀蟲,甚至當年搞出漢芯的騙子依舊還在國内開公司騙錢,那麽外人爭論技術細節的優先順序是毫無意義的。
\\

\textit{\hfill\noindent\small 2022/08/09 23:51 提问;2022/08/10 01:24 回答}

\noindent[29.]{\Hei 答}:1970年代美國的通脹越演越烈,其背景是歐洲和日本在Bretton Woods被撕毀之後,不約而同的開始進行去美元化,以致美元佔國際儲備貨幣的份額出現明顯持續的下降,造成對其國内通脹的正反饋。

我自2014年博客一開始就關注這一輪美元濫發問題,强烈建議中方必須積極去美元化,正是借鏡了當年的經驗,並考慮到中國在當前國際經貿體系的地位,所以最優解是和歐元聯合起來,將通脹局限在美國自身。現在德國新政府自願割肉喂鷹,固然有幫助美國苟延殘喘的效果,但若是第三世界能聯合起來,最終結果還是不變的,只不過必須多等幾年,讓歐盟、英國和日本的崩潰先明顯化。
\\

\textit{\hfill\noindent\small 2022/09/10 08:44 提问;2022/09/11 06:23 回答}

\noindent[30.]{\Hei 答}:自從四五年前我開始上視頻訪問之後,博客就面對一個新問題,亦即原本爲了追求精簡,非必要不重複細節,然而我在聊天過程中隨口談過的論證和結論,並沒有寫下來,是否要在博客再復述一次?此外,我的邏輯始終是環環相扣的,沒有寫下來的論證部分,要引用並不方便。

你討論的歐洲工業如何流失的議題,其實我在上個月唐湘龍的節目上已經解釋過了:德國和歐盟面臨的不是沒有天然氣和石油,而是它們的價格大幅上漲、甚至以數量級計的程度。換句話説,這不是兩次大戰期間的物資短缺,而是金融性的超級通脹,而且是史無前例只有歐洲獨自承受的超級通脹(請注意,三年前我在《八方論壇》節目中,已經詳細解釋過下一場經濟危機會是某霸權集團獨自承受的超級通脹;今年發生的新轉折,只在於歐盟志願為美國替死),所以受打擊最早、最嚴重的是高能耗工業、尤其是可以用進口替代的高能耗工業,也就是化工(BASF和SKW Piesteritz)和那篇報導所談的冶金。

大衆媒體熱議的冬天將至,其實影響的是民生用能源,所以嚴重性在於政治層面;不過正因如此,歐洲的這群愚蠢無良政客必然優先保障政治性需要,真正事關經濟命脈的工業用能源供應,早已絕望。歐洲的自我救贖,必須先徹底清除幾乎所有主流的政治、媒體和學術精英;就算他們能做到,也不是三天兩頭的事,何況綠黨的支持率還在上升之中,昂撒系傳媒已經開始為Habeck和Baerbock繼任總理宣傳造勢,甚至可能出現兩者相爭的局面,若不是因爲要有幾萬條人命陪葬,這可以算是人類政治歷史上的最大笑話之一!
\\

\textit{\hfill\noindent\small 2022/09/17 10:40 提问;2022/09/17 11:16 回答}

\noindent[31.]{\Hei 答}:不是,我指的是Private Equity。你那篇文章談的是高利貸,規模相對很小,倒霉的也只是窮人。
\\

\textit{\hfill\noindent\small 2022/09/23 02:46 提问;2022/09/23 07:57 回答}

\noindent[32.]{\Hei 答}:間接的合成貨幣不會是問題,因爲人民銀行增發以滿足國際需求的時候,亞元的其他成員國也必須對等增發。換句話説,升值的壓力由大家分擔了;這是解決美元堅挺、不利出口的最佳方案,參考德國在建立歐元後,輕鬆保持匯率穩定的經驗。

這個考慮,我在介紹新亞元概念的時候解釋過的;照理讀者不應該重複發問,而是自行溫習既有敘述,不過因爲這個問題有相當的專業性,一般讀者可能過目即忘,破例一次。
\\

\textit{\hfill\noindent\small 2022/09/25 09:37 提问;2022/09/26 01:00 回答}

\noindent[33.]{\Hei 答}:正如Putin所發現的,Scholz和Macron不管當面說什麽好話,一轉頭又是乖乖做昂撒集團的小狗。所以和他們的外交,沒有談戰略議題的意義,不過講講經貿還是可以的;尤其現在正是新興的中方電動車企業要出國攻城略地的關頭,確保這個進程的順利發展還是有相當價值。而Scholz和Macron在國内經濟受到嚴重打擊的前提下,在貿易政策上也必然有求於中方,做些交換固然是互利,中方的收穫應該可以更大。

俄國固然打贏了第一波金融戰,但這裏的關鍵在於能源價格的上漲。一旦第三世界和歐洲的經濟衰退明顯化,油氣價格有隨之崩潰的可能。所以我一直覺得Putin不速戰速決是個戰略錯誤;夜長夢多,就算他不怕北約對烏克蘭的加碼支持,經濟形勢的危險也是難以控制的。如果我來決策,既然已經動員,就會儘快解決軍事議題,以便專心應對即將來臨的這一波全球經濟問題。

美歐不可能同意Zelensky認輸,只有在軍事上兵臨城下,迫使烏方出現政權更迭,才可能迅速達成妥協。同樣的,歐洲現在的這群領導人也不可能承認錯誤,只能靠替換來變革;然而這在宣傳媒體一面倒的背景下,並不容易,我覺得有可能拖過冬天。

中方的策略,應該是盡可能和歐洲維持表面上的客氣,繼續深化經貿上的交流,一方面搶占新能源市場,另一方面吸引歐洲的先進工業轉移來華。至於政治和外交層面的問題,做旁觀者就行了。
\\

\textit{\hfill\noindent\small 2022/11/10 23:51 提问;2022/11/11 09:08 回答}

\noindent[34.]{\Hei 答}:
這都是小打小鬧,無關宏旨。真正要緊的是有Saudi參加的金磚貨幣;我在半年前估計會是在今年底或明年,現在看來像是後者了。
\\

\textit{\hfill\noindent\small 2022/11/13 22:51 提问;2022/11/14 09:51 回答}

\noindent[35.]{\Hei 答}:
不是。你應該復習四五個月前我在《龍行天下》所作的評論,當時我解釋了2022年的前兩季的確不算明確的衰退,這是因爲金融和高科技兩大高收入職業還沒有開始裁員,仍然足以支持整體消費水平不做明顯的下降;換句話説,美國這輪經濟危機,要等高收入者也開始緊縮支出才算正式開始。不過經濟衰退和危機,是周期性現象,在冷戰後平均十年一次;對霸權交替有決定性影響的經濟崩潰,則必須達到百年一見的程度,亦即超過2008年金融危機,至少與1929年大蕭條同級。這在英國和歐盟都有大機率會發生,在美國則還沒有徵兆。
\\

\textit{\hfill\noindent\small 2022/12/06 11:37 提问;2022/12/07 02:34 回答}

\noindent[36.]{\Hei 答}:不會,但變革的時間點很難説。
\\

\textit{\hfill\noindent\small 2022/12/08 17:20 提问;2022/12/09 00:40 回答}

\noindent[37.]{\Hei 答}:我對歐洲政客的智商並不樂觀。此事有可能反過來成為歐盟也采納保護主義的藉口,對中國貿易產生進一步的障礙,尤其是過去兩年我反復擔憂的新能源車出口,可能首當其衝;Stellantis中國分公司最近破產,更加給予他們破罐子破摔的餘裕。
\\

\textit{\hfill\noindent\small 2022/12/20 10:01 提问;2022/12/21 02:52 回答}

\noindent[38.]{\Hei 答}:我自己不但是金融專業,而且是金融專業中極端少數願意考慮體系整體得失而不只是如何占體系便宜來賺錢的人。此事已反復專門著文討論過,其邏輯解釋得很直白了,有疑問的讀者請務必先自行復習,我在下一樓一並總結。
\\

\textit{\hfill\noindent\small 2022/12/20 12:45 提问;2022/12/21 02:22 回答}

\noindent[39.]{\Hei 答}:美元的國際地位,對美國的實體產業有著絕對的腐蝕作用。對外的金融掠奪,直接圖利華爾街,再怎麽分發福利,也會加劇貧富不均(因爲政府能直接掌握的獲利,必然只占少數)。經濟比政治,還有更多反直覺的現象,但只要小心,並不難看出要點,例如這裏MMT的基本假設是印鈔就可以無中生有,憑空創造出新財富;事實上只要金融體系壯大起來,它自然會繼續繁衍、追求進一步的利潤,而最簡單、而且基本無法防治(因爲金融產品先天就是100\%人造的,可以隨意修改、複雜化)的利潤,就來自人爲地製造信息不對稱;信息不對稱必然導致資金配置的低效,從而引發比賬面利潤高出幾個數量級的隱形社會(含國境外的人類社會)損失,只不過一般金融系或經濟系的教授們看不出來罷了。

這還沒有談到對世界做金融剝削,完全與中國的國際大政略以及傳統文化背道而馳。
\\

\textit{\hfill\noindent\small 2022/12/21 13:25 提问;2022/12/22 04:39 回答}

\noindent[40.]{\Hei 答}:美元和歐元外匯用來保障人民幣升值以將通脹反施彼身的過程安全,不好嗎?美國在這一波升息吸血階段,中國見機抄底不需要外匯嗎?正當的用途多得很,只不過人民銀行假裝看不見罷了,用不着慌不擇路、劍走偏鋒。
\\

\textit{\hfill\noindent\small 2022/12/21 14:21 提问;2022/12/22 03:39 回答}

\noindent[41.]{\Hei 答}:你把我的回復先搶答了。
\\

\textit{\hfill\noindent\small 2023/01/11 13:43 提问;2023/01/12 02:17 回答}

\noindent[42.]{\Hei 答}:有關Powell的貨幣政策,其實是在複製1971年Nixon打破Bretton Woods開始拼命印鈔到1973年第一次能源危機,通脹壓力持續纍積後Arthur Burns的和稀泥方案,亦即只做出足夠避免通脹率持續惡化的最低努力,所以在70年代中後期,美國年通脹率穩定在5-7\%的範圍内,歐洲和日本雖然開始替換美元,短期内倒還不是特別難受。問題在於Burns的運氣不好,他賭的是他所能控制的需求面被調整為中性之後,供給面的天然波動能自行解決難題,結果拖了幾年沒有好轉,工會的薪資談判反而將通脹固化了;他1978年下臺的時候名聲還不太臭,但1979年第二次能源危機徹底打破均衡,證明他賭輸了(有點像Alan Greenspan2006年退休時還是“Maestro”,兩年後出了世紀級的金融危機,名聲一下爛了大街),Carter只好緊急再換上鷹派的Volcker取代Miller,接下來是家喻戶曉的20\%短期利率。

Powell有前例可以借鏡,應該自認是條件更好的:不但這次能源危機的規模較小,而且美國的工會早已名存實亡,更重要的是歐盟、英國、日本不但沒有抛棄美元,反而捨身喂鷹,連中國人民銀行都在最關鍵時段全力配合美聯儲壓制美國國内第一波的通脹壓力。當然也有負面因素,亦即美國的產業虛擬化、金融化,而且過去20年的印鈔幅度比1970年代初期短短幾年要大得多;不過這些都是慢性問題,説不定能自然解決(例如中方金融主管繼續為美國犧牲自己國家的利益,開放金融搜刮),所以我完全理解他的邏輯。
\\

\textit{\hfill\noindent\small 2023/01/16 10:08 提问;2023/01/16 11:15 回答}

\noindent[43.]{\Hei 答}:Saudi應該最晚在八月的高峰會議期間加入金磚,那麽乾脆等一等,反正石油定價才是新貨幣的最重要關鍵。
\\

\textit{\hfill\noindent\small 2023/01/17 05:54 提问;2023/01/18 01:13 回答}

\noindent[44.]{\Hei 答}:這裏的答案有兩個層面:最優解當然是中方積極主動地領導新國際體系的設計和建立,但實際上他們既缺乏眼光能力(主要在於學術界欠缺能提供引導性原則和設計細節的人才)也沒有意願一步站到臺前聚光燈下,因而最可能的現實發展是半被動地參與由俄方主導的多邊談判。不過你所描述的也是Putin心中的願景,所以結果應該是大差不差。

至於過程進展速度,美國已經熬過了通脹危機的急性期,短期内沒有崩潰的危險,未來幾年會延續過去20年的漸進衰敗。當然,也出現了SEC和FTC的改革企圖,不過拖延續命可以,要說力挽狂瀾、扭轉頹勢還是遠遠不夠的。
\\

\textit{\hfill\noindent\small 2023/01/18 22:12 提问;2023/01/19 05:40 回答}

\noindent[45.]{\Hei 答}:其實2021和2022年,美國面臨全面供給鏈問題,成本以倍數成長,匯率漲跌個2、30\%根本不影響他們下單進口。結果人民幣匯率依舊遵照2019年之前的舊有公式運作,隨日幣和歐元起伏,基本是自願承受同比例的割肉飼鷹。最近美國通脹緩解,容許附庸國貨幣部分回升,人民幣也才有所上漲。我的質疑在於,如果金融單位的最高主管,完全無視經濟、金融、貿易和戰略現實,盲目遵循大幅傷害國家利益的簡單綫性公式,這是隨便一個小電子表格都可以做到的,那麽國家養他們來幹什麽?

美國金融界的資金汎濫極爲嚴重,美聯儲的升息只做了避免通脹持續惡化的最低程度,一年來量化緊縮所回收的美元佔總量的九牛一毛,所以資產市場定價仍然偏高。然而全面崩盤的危險已經渡過,危機轉爲慢性,美聯儲得以繼續走鋼絲;要讓他摔下來,必須等下一個黑天鵝事件。金磚貨幣當然會是一個非常嚴重的打擊,但目前我還不能確定其時間點和緊迫性;換句話説,其對美國經濟金融的負面效應必然是很大的,但如果平攤為多年的逐步影響,那麽就可能不足以獨自引發非綫性的連鎖反應,就像承力結構對static load的彈力極限明顯高於dynamic load;這也是爲什麽四年前我就提早警告要好好把握這輪通脹危機,一年前則特別着急的原因,現在已經追悔莫及了。
\\

\textit{\hfill\noindent\small 2023/01/19 13:00 提问;2023/01/20 20:38 回答}

\noindent[46.]{\Hei 答}:經濟活動的總量其實有多種算法,其中兩種比較常用,你説的是入門經濟學拿來教GDP的那一種。但不論怎麽算,差別都很小,這是因爲它們必然都基於呈報和問卷資料,最終依舊是只能反映經濟交易浮面數額的會計算法,而GDP在作爲治理參考上真正的主要缺失有二,都屬於不可能用會計手段來解決的偏差(bias,相對於誤差error而言):1)忽略隱性社會成本和收益;2)忽略服務和產品的品質。後者在市場競爭下依然經常有大幅偏差,是資本為追求壟斷暴利,日常進行公關炒作,創造信息不對稱的結果,長期下來更可以做出制度上的扭曲,例如美國的健保醫療業和律師訴訟業是典型案例,全球浪費在量子計算、核聚變和氫能上的錢,也是有意詐騙的後果。
\\

\textit{\hfill\noindent\small 2023/01/20 13:12 提问;2023/01/20 21:52 回答}

\noindent[47.]{\Hei 答}:是的,因爲歐洲國家選擇割肉喂鷹,可能會是首先引爆英國、然後歐盟的急性危機,然而這是非常間接的金融效應,不會有什麽外交或宣傳上的後果,畢竟美元才是國際儲備貨幣,直接壓力還是來自與美元的兌換匯率。就算美國的金融經濟能撐得過,北約也會陷入全面混亂,徹底解除美國在東亞出手的餘裕;這又有什麽不好呢?歐盟官員以美國利益為優先考慮,受盡明眼人的譏嘲,難道你認爲中國主政者應該優先保護歐盟的利益(間接保護美國)嗎?Putin根本不理會歐盟的得失,純粹優先建設國内民心士氣,這是我特別寫博文强調的事,你若不同意,可以提供反面論證,空口白話、顛倒黑白是博客不容許的。
\\

\textit{\hfill\noindent\small 2023/01/26 15:01 提问;2023/01/26 22:24 回答}

\noindent[48.]{\Hei 答}:令我感到不安:既然金磚貨幣正在籌備之中,巴西並沒有理由另起爐竈。這裏有兩個可能:1)金磚貨幣談不攏,只能做到貿易上改用各國既有貨幣;2)正文中所討論的,必須提防雙逆差國家的考慮被重視,所以至少排除了巴西和南非;如果能順便剔除印度,那更是理想。不過中方金融決策實在太讓人失望,我不敢再自動假設他們能遵從簡單合理的思慮;我們再等等吧。
\\

\textit{\hfill\noindent\small 2023/01/27 00:36 提问;2023/01/27 01:07 回答}

\noindent[49.]{\Hei 答}:本月初我也估計是八月收Saudi之後,一並宣佈合成貨幣;正文中也詳細論證了巴西、南非和印度不適合作爲創始國。雖然中方金融主管不入流,俄方卻是懂事的,所以我並不是悲觀,而只是去年被人民銀行嚇怕了,不敢簡單樂觀。
\\

\textit{\hfill\noindent\small 2023/01/28 23:09 提问;2023/01/29 02:20 回答}

\noindent[50.]{\Hei 答}:你説的是Static Load Threshold高於Dynamic Load嗎?那並不是數學,而是物理和工程學上的常識;當然,對一般社科學人來説,同樣是絕對陌生的概念。另一個他們完全不懂的案例,是經濟周期波動的起因:這裏的真正機制在於非綫性自發不穩定現象,類似Karman Vortex Street這類渦流,但學經濟的當然又是摸不着邊,結果兩次諾獎給的都是高中物理級別的理論;這一點我以前也提過了。

是的,人口結構是典型的慢性長期挑戰,只要能1)預見問題;2)找出解答;3)有效執行,完全可以剋服。現在人口問題本身已經是全民共識,需要理解並整頓的只是學術和教育方面的腐敗現象。
\\

\textit{\hfill\noindent\small 2023/01/29 10:50 提问;2023/01/30 01:36 回答}

\noindent[51.]{\Hei 答}:因爲現在美國的外交大戰略由NeoCon主導,他們的特質之一就是沒有智慧、不尊重專業、不在乎現實,只看到打垮瓦解俄國的獲利,完全不想想己身的軟肋。第三世界既然連俄烏衝突都不給美國面子,替換美元的事是西方沒收俄國外匯資產倒逼出來的,除了内部賣國賊作祟,哪有理由不全力推行呢?

至於人口結構問題,就算沒有能力看出解決方案,至少也應該有點自知之明,瞭解這種長期慢性阻力,別人會有剋服的辦法,最最起碼不是可以先來博客確認一下嗎?凡是蹲在家裏誇誇而談、拍腦袋說中國必敗的,都明顯是在遵循“我沒聽過,所以不存在”的邏輯,也就是Dunning-Kruger曲綫裏笨蛋峰的典型居民。我説過很多次,請大家不要引用不入流的他人言論,參見《讀者須知》第六條。
\\

\textit{\hfill\noindent\small 2023/01/31 01:33 提问;2023/02/01 23:57 回答}

\noindent[52.]{\Hei 答}:你既然有興趣討論,我就談得再深一點。其實經濟景氣程度,當然也受外來(Exogenous)意外變動所左右,而且這些大大小小的外來因素非常多,反而形成研究經濟周期動力機制的真正困難所在。這是因爲它們無所不在,又有著確實的影響,所以每次經濟衰退,必然都有一堆負面的因素,可以簡單拿來解釋,特別方便低級學人偷懶、批量發表無意義的類比聯想論文。但從長遠的角度來看,這類外加因素顯然應該遵循隨機的Poisson Process,如果假設它們完全左右經濟榮枯,那麽經濟成長就會是發散性的Random Walk,顯然不符事實。所以至少必須假設還有内發性的回歸平均機制(換句話説,實際上這個内發動力機制應該兼有失穩和回歸性,而當前昂撒經濟學界選擇性地只看片面),兩者叠加之後,才可能有目測上的“波動”現象;但這樣的僞周期機制,其規律性和可預測性都近乎零,依然不符合實際的現代經濟歷史記錄。事實上放任政策導致經濟泡沫,是有金融常識的人可以在爆破前幾年就簡單斷言的事,博客這裏已經反復示範過。所以現代經濟周期的實際主導機制,在於内發的不穩定性,外來的隨機變動,一般應該被視爲加成的導火綫,而且隨著人類科技能力和組織程度的提高,内發因素越加蓋過外來影響。例如古代天災的確對國運有立即和決定性的打擊,但同等程度的災害在21世紀,越來越像是日常政務處理;即使出現新冠這樣百年一見的Pandemic,固然導致嚴重的經濟下行,但其作用明顯獨立於真正的“經濟周期”;後者的内發過熱問題依舊需要處理,中方一樣花了兩年多來壓制房地產炒作,而把外加和内發因素搞混的歐美國家,不就必須面對50年來最大的通脹危機嗎?

地震和渦流,同樣都是内發的周期性現象,然而其確實的動力機制是完全不同的(至少當前的理論,遠遠達不到有統一性的地步)。這沒有什麽太大的問題,因爲經濟周期本身,也有其獨特的内發不穩定性機制;這裏的確只有定性的共通,不過在定量上,我覺得經濟周期和渦流更接近些。
\\

\textit{\hfill\noindent\small 2023/02/19 05:41 提问;2023/02/20 03:20 回答}

\noindent[53.]{\Hei 答}:我自《美元的金融霸權》以來反復解釋過,只要美元依舊是國際儲備貨幣,美國在財政和貨幣管理上就可以爲所欲爲。這句話翻過來說,就是赤字和QE的負面作用,只在於威脅美元的國際份額;然而因爲全球經濟體系的慣性以及美國在國際治理(含政治、軍事、經貿、文化、學術、宣傳等等)上的統治性地位,這個負面作用效應十分間接而且微弱,這也爲什麽多年來我一直强調中方必須積極主動地針對美元這個關鍵做反擊的原因。

當前的情況也不例外:聯邦政府的債務危機,在零級近似層級(亦即依效應大小從近而遠做展開之後的領頭項,Leading Term)純屬政治作秀,沒有經濟金融上的實際影響。在下一級近似上,對美元名譽影響是負值,但很小,基本可以忽略不計。再下一級,對當前的通脹壓力,也沒有直接的效應,必須考慮對美聯儲QT政策的間接影響,才有值得討論之處。

去年年初美聯儲從QE反轉為QT的時候,總資產(亦即印鈔的總量)大約為\$8.8T。一年下來縮減約5\%,到了\$8.3T。然而從美聯儲各種貨幣窗口的交易量來看,顯然銀行業並沒有因爲流失了\$500B的現金而感到資金短缺;這正是幾個月前我談過的,美國經濟的各個層面,含家庭、企業到金融機構,都因爲2019年以來的急劇QE(總額\$5T)而纍積了大量剩餘資金,需要時間來逐漸消耗。上次我說家庭和企業大概要到2023年年底或2024年才會用完這些剩餘儲蓄,金融方面其實也差不多。

講了這麽多細節(沒辦法,你問的是高階修正而且是間接效應,要討論就必須先談領先和直接的展開項),聯邦債務爭議的真正作用,只在於先暫時(六個月?)停止發行國債,緩和現金被美聯儲QT抽走的影響,然後在國會作秀完畢、將國債上限再次上調之後(今年夏天?秋天?),會有\$1T這個數量級的聯邦債券一次性洗劫金融體系,從而對美聯儲QT產生競爭性的加成作用,其後美國金融系統才有可能會重新面臨2019年的現金短缺危險,進一步迫使QT提早結束,參見我在《從回購利率暴漲談美國經濟周期》一文的討論。
\\

\textit{\hfill\noindent\small 2023/02/20 07:32 提问;2023/02/23 00:16 回答}

\noindent[54.]{\Hei 答}:你能看懂我的用心,很好。美國國債本身無關緊要,是博客已經評論過幾百次的議題,所以我假設你問的是高階修正項,順便示範如何將多方面的因果分析,針對一個特定問題整合成爲精確的估算,所以必須根據效應大小和作用遠近(亦即從直接到間接)來做二重展開。我以往反復批評過的類比聯想,即使僥幸撞上真正邏輯相關的議題,也必然會是盲人摸象、斷章取義;連既有事實都不全面考慮,怎麽可能對未來做準確的預測呢?沒有準確的預測,則談何政策選擇最優化?

熟悉高能物理的讀者,應該知道行業的基本理論工具是量子場論,而場論的計算太複雜,根本不可能有確解,所有的高能物理理論計算,都是序列展開的近似,這叫做Perturbation Theory,所以把這個思維方式應用在社科議題上,對我來説當然是件駕輕就熟的事。
\\

\textit{\hfill\noindent\small 2023/03/11 00:01 提问;2023/03/11 02:40 回答}

\noindent[55.]{\Hei 答}:一年前俄烏戰事初起,Saudi拒絕支持西方陣營,就已經代表著MBS準備和伊朗和解,因爲這是加入上合以及金磚的先決條件,現在只是原則性談判終於完成,可以正式公開。實際上早期的斡旋是Putin推動的,但因爲當前的歐美制裁氣氛,由中方接手調節工作,並在北京宣佈可以避免旁生枝節;這也是過去一年中俄分工合作(亦即和戰立場上的白臉和黑臉)的示範案例,參見正文的建議。

未來半年國際上還會有一連串的重大轉折(連墨西哥都已表態希望加入金磚,更極端的可能還有以色列,這是沙伊和解後以方的最佳對應方案,不過不知道中俄有沒有胃口去策反),金磚貨幣只是其中的核心項目。
\\

\textit{\hfill\noindent\small 2023/03/12 10:54 提问;2023/03/13 02:09 回答}

\noindent[56.]{\Hei 答}:感到意外,然而這並不影響博客既有的核心論述,亦即過去幾年有高官否決出手打擊美元,是一個極大的錯誤,只不過現在的新推論必須更改爲,幕後的責任人是比易綱層次更高的人物;這是因爲很難想象未來這一年人民銀行還有比與俄國協作、建立金磚貨幣更重要的任務,行長必須是層峰信任能完成任務的人。
\\

\textit{\hfill\noindent\small 2023/03/13 12:12 提问;2023/03/14 04:23 回答}

\noindent[57.]{\Hei 答}:是嗎?我沒看到正式宣佈,至於墨西哥總統談話說有意願,則已經反復發生幾次了。

墨西哥和以色列,都和美國綁定很深;現任的Obrador是有反美傾向的左翼,但下一任很可能會反復。如果由我做決策,會先讓他們成爲觀察員,要求他們盡責任(尤其是采用金磚貨幣),但暫時不給投票權。等美國進一步式微,邊緣成員反復的可能性和危害都減小之後,再正式接納升格。
\\

\textit{\hfill\noindent\small 2023/03/14 20:36 提问;2023/03/15 06:51 回答}

\noindent[58.]{\Hei 答}:1)會計既不在乎實質真相,也不在乎邏輯自洽,它只是固有規則和慣例的反復運用。然而提升一個meta level來看,這些規則和(尤其是不成文的)慣例是怎麽定的?當然是多元的資本權力山頭彼此妥協的結果,所以會計事務所和信用評價公司事先沒有預期SVB出事,根本就沒什麽好奇怪的。

2)昂撒的信評當然優待自家的主權信用,參見上文。

3)我自2019年不知説了多少次,美聯儲的貨幣政策是在走鋼索,寬鬆則通貨膨脹、緊縮則泡沫爆破;我們現在看到的這些地區性銀行出問題,只不過反映了這條鋼索越來越窄的事實,然而摔下去的時間點還是難以事先確定,我依舊看好八、九月的金磚貨幣發行+國債上限上調這個double whammy(雙響炮?)。至於以調高外銷價格來進一步推動通脹,時機已過,歐美的貨品消費從火熱轉爲蕭條,國際消費品貿易進入買家市場,硬要去搞會得不償失。
\\

\textit{\hfill\noindent\small 2023/03/23 16:41 提问;2023/03/24 05:21 回答}

\noindent[59.]{\Hei 答}:如果有讀者想從頭復習博客内容,我想點明貫穿多年的一條論證脈絡,亦即中方在21世紀面對美國霸權帝國的凶殘打擊,應該如何應對,是博客自一開始就想要解釋的重要議題,但因爲正道與當時的主流意見完全相反,我不得不從委婉地留下伏筆做起,先介紹如美元國際地位、美國戰略歷史、IMF和SWIFT的殖民主義作用等等基本知識,一直到Trump上臺,美方終於撕下假面具,直接發動貿易戰,我才敢把正確的戰略回應明説出來。然而即便長年保持如此的耐心,在當時還是連老讀者都有許多無法接受“不能挨打不還手”的簡單邏輯,更別提“美元是美國霸權的軟肋”這個結論。還好過去五年是華語世界在國際政略方面民智爆發的轉折,以博客讀者群(含爲了剽竊而來潛水的網紅;讀者私下向我反映,看到照抄或重寫的文章,不計其數:列舉來源出處真有這麽難嗎?公共論壇由一群竊賊主導,真的是健康的現象嗎?)為核心,正確認知迅速地散佈出去;很不幸的,中國學術界卻依舊掉隊。現在翟教授這樣的頂尖人物終於完全接受並公開宣揚真理,希望學術圈能儘快做出正向淘汰,不再拖國家的後腿。

另一個拖國家後腿的,是受昂撒教育洗腦的官僚。博客自2019年開始就反復解釋了美國經濟危機的必然演化路綫,而且連時間點(三年)都明確指出來。人民銀行如果還有一絲理性,就應該知道不能繼續買美國長期國債,外匯投資於債券必須只限於2022年之前到期;不這麽做的結果,就是可能高達20\%的虧損。我在本月稍早的《龍行天下》節目做了這個定量計算,不過引申應用到人民銀行的外匯管理,只説了一句“一言難盡”,觀衆可能無法全部領會,在此簡單澄清。

既然已經回顧《龍行天下》中一個匆忙提過、沒有解釋清楚的細節,順便在此也對另一個論點做補充説明;稍有離題,請勿見怪。SVB固然是美國金融貨幣政策左右爲難下的病徵,但因爲大銀行財務健康,收拾善後並不危及國本;相對的,歐洲和日本的經濟、財政和金融都要更虛弱得多,美聯儲可以接受的5\%利率,歐日明顯承受不起,卻又不得不因應跟隨,前景遠遠更爲陰暗。瑞士政府在整合瑞銀和瑞信的過程中,强迫變更契約,將若干Contingent Convertibles的條文徑行作廢,侵害了歐洲債權人的利益以及歐盟國債市場的信心,讓ECB焦頭爛額,是這個深層問題的最新表徵。關心國際態勢發展的讀者,應該特別注意未來一年有關ECB和日本中央銀行的新聞。
\\

\textit{\hfill\noindent\small 2023/03/23 23:08 提问;2023/03/24 01:41 回答}

\noindent[60.]{\Hei 答}:現實世界裏的政治經濟治理是個比爛的游戲,中方可以追求完美,但達不到也無須焦慮,只要强過其他國家就算贏了。因而,歐美經濟的自我爆破在短期内當然對全球都有負面影響,但中國只須要做到受影響最低;這並不難達成,而且長期來看,反而有助於轉向發展西部内地和與第三世界的經貿關係,理順與美國脫鈎的過程,在大政略上是件好事。
\\

\textit{\hfill\noindent\small 2023/03/25 09:22 提问;2023/03/26 01:20 回答}

\noindent[61.]{\Hei 答}:瑞信和德銀現在的窘境,過去15年美國的幾百億罰金功不可沒。瑞士出手拯救瑞信的過程中,必須撕毀一些金融衍生品的契約,美國司法體系正在研究如何用長臂管轄再撈一票。要美國去幫忙,你想多了。

這些金融資產内含許多未爆炸彈,再加上銀行業原本就對政治和監管非常敏感,連沙特都學乖了,中方完全沒有理由去碰;這是過去幾年我反復説過的事。
\\

\textit{\hfill\noindent\small 2023/03/30 18:51 提问;2023/03/31 02:06 回答}

\noindent[62.]{\Hei 答}:和SVB的管理階層一樣,把美國經濟系和金融系的教材奉爲聖經,資本霸權體制機構當成無上神殿,只要是信評AAA的債券就認定是寶貝,成天在裏面鑽研挑選利率高0.5\%的,即便本金剛剛承受20+\%的虧損也沒有任何反思,反正人民銀行不像SVB那樣會出現擠兌和破產的問題,國王新衣絕對可以穿到退休。
\\

\textit{\hfill\noindent\small 2023/03/31 15:07 提问;2023/04/01 01:08 回答}

\noindent[63.]{\Hei 答}:你說的會計區分叫做Available for Sale(AFS)和Held to Maturity(HTM):前者是應對提款的準備金,必須Mark to market(亦即依實際市場價格起伏計算其價值),後者是長期投資,在賬目上依據購入價計算。SVB的資產有大約10\%是AFS,主要放在國債;這部分的虧損早就公開,沒人在乎。上月出問題,是因爲存戶在2022年不斷提款,資產總額下降太多(約17\%),被迫將原本HTM的Mortgage Backed Security(MBS)也認賠變賣,以致Equity Level(净資本)跌破法規下限,在周一被報導之後,立刻引發大規模擠兌,無可挽回。人民銀行雖然不怕存戶擠兌,但美國的敵對態度不斷升級,若是臺海形勢有需要、或者層峰下決心出手反擊,這些美元資產必然要在一夜之間換成現金,届時就連名義上的會計賬目都無法遮掩。

不過中國政府的金融專業管理階層已經腐爛至極,“無能”和“賣國”仍舊是Understatement;昨天不是剛剛又傳出劉連舸被捕的消息嗎?
\\

\textit{\hfill\noindent\small 2023/04/14 16:47 提问;2023/04/29 03:06 回答}

\noindent[64.]{\Hei 答}:易綱被昂撒宣傳和美式經濟學洗腦太深,真正把人民銀行搞成美聯儲北京分行(參見\href{https://www.economist.com/finance-and-economics/2023/04/20/is-china-better-at-monetary-policy-than-america}{链接\footnote{\url{https://www.economist.com/finance-and-economics/2023/04/20/is-china-better-at-monetary-policy-than-america}}})。幾年下來,光是會計上有跡可循的外匯浮虧就是幾千億美元,戰略上打擊美元的錯失時機,更是價值幾萬億的機會損失,過去十年所整肅的幾萬貪官,去掉半導體大基金那些人之後,危害加起來都沒他一個人大,現在居然還不讓下臺,真是天下第一奇事。
\\

\textit{\hfill\noindent\small 2023/05/06 23:52 提问;2023/05/11 07:51 回答}

\noindent[65.]{\Hei 答}:中俄互信合作還在持續升級,是正確的敘述,但對俄國想和印度改用人民幣只有間接的影響,直接的因素是Nabiullina並不信任一個貿易財政雙重逆差的半工業國貨幣。
\\

\textit{\hfill\noindent\small 2023/05/17 14:17 提问;2023/05/24 04:05 回答}

\noindent[66.]{\Hei 答}:從2014年一開始,博客就委婉但明確地反復解釋金融貨幣策略的重要性和其後應有的戰略考慮;對中國的專業主政者,卻一直賦予Benefit of Doubt(“寧縱勿枉”?),八年不予指責。直到一年多前,本波危機最後的直接出手機會也被無視,這些人繼續以SVB主管的心態來管理人民銀行,我才終於確定他們是名副其實的賣國賊,並做出針對性的批評。如今宵小依舊在位,持續既有的錯誤政策是自然的;我當然希望金磚貨幣能有驚喜,但也必須做好最壞的心理準備。
\\

\textit{\hfill\noindent\small 2023/05/18 12:27 提问;2023/05/24 04:08 回答}

\noindent[67.]{\Hei 答}:這依然沒有跳出美國中小銀行主管的心態窠臼。我們再等三個月吧,如果金磚貨幣沒有下文,那麽事情就真正大條了。
\\

\textit{\hfill\noindent\small 2023/07/01 16:22 提问;2023/07/02 03:47 回答}

\noindent[68.]{\Hei 答}:我一個外人只能指出正確方向何在,實際的人事秘辛無從得知,也難以立刻作評,必須等到新任官員有了政策取向才可判斷。

不過,至少郭功勝不是旅美香蕉人,而且出身自唯一像樣的社文科學府:人民大學。
\\

\textit{\hfill\noindent\small 2023/07/01 17:03 提问;2023/07/02 03:50 回答}

\noindent[69.]{\Hei 答}:兩年前是美方通脹危機急性化的關鍵時刻,一旦錯過,匯率就不再是絕對關鍵。美國接下來的下一波危機在於財政赤字,這可能要到2025年前後才會急性化;我計劃在本月的《龍行天下》中做討論。

當然正如博客9年來反復强調,替代美元國際儲備體系是長遠金融戰略上的最大關鍵,任重而道遠。
\\

\textit{\hfill\noindent\small 2023/08/03 01:29 提问;2023/08/05 05:09 回答}

\noindent[70.]{\Hei 答}:我從一開始就預見到這些問題,所以才會做出中俄沙搞籃子貨幣的建議;也不知是中方決策者不懂(考慮到他們的專業建議來自易綱,實在難以讓人有信心),還是其他國家不容許。等幾周,看實際宣佈的協議吧。
\\

\textit{\hfill\noindent\small 2023/08/05 02:41 提问;2023/08/05 04:48 回答}

\noindent[71.]{\Hei 答}:易綱犧牲中國利益以拯救美國的行爲,絕對沒有習近平正確理解後的許可,因爲後者的所有對外政策都與你的論述相矛盾。

但其他的中高層人員思想扭曲,以致被英美學術歪論所迷惑(蠢)和情報單位所滲透(壞),則是博客反復批判過的問題(但前者遠遠更普遍更嚴重)。習近平留任,可能就有在60後和70後之中找不到合適繼任者的因素在。
\\

\textit{\hfill\noindent\small 2023/08/16 17:46 提问;2023/08/17 05:20 回答}

\noindent[72.]{\Hei 答}:當前的訊息很混亂,既然會議很快就開始,大家稍安勿躁,再等幾天自然會知道多國折衝的實際妥協結果。

至於策略上的考慮,我從2014年開始鋪陳理論基礎,俄烏戰事初起就斷言是實踐的時機,而且不但立刻給出最佳選擇,連其他選擇會有什麽問題都反復解釋過了;中方如果人謀不臧,堅持要在錯誤路綫的墻上撞得頭破血流,我也沒辦法。這裏我想指出幾個已經被驗證了的細節預測:(1)印度是建立新國際儲備貨幣的最大障礙;(2)中印的確因爲金磚是否大幅擴員而發生激烈衝突(參見\href{https://www.rt.com/news/581310-china-india-brics-currency/}{链接\footnote{\url{https://www.rt.com/news/581310-china-india-brics-currency/}}});(3)籃子貨幣是最可行的方案,Nabiullina的背書可以簡單證明。
\\

\textit{\hfill\noindent\small 2023/08/24 21:46 提问;2023/08/26 01:52 回答}

\noindent[73.]{\Hei 答}:
當前世界有足夠財政紀律和工業體系的國家寥寥無幾,像你這樣挑挑揀揀完全不切實際。反正金磚有了印度之後,根本就不可能作爲國際治理的決策核心來有效運作,只能當作一個鬆散聯盟用來整合外圍國家。我不是從一開始就不建議金磚貨幣嗎?其廣義的原因正是取代美國霸權所建立的新國際秩序,必須有一個核心,而有資格進入核心的,除了理所當然的中國之外,頂多只有俄國和改革如果成功的Saudi。換句話説,正如昂撒把Golden Billion劃歸一個四層的洋葱,尊重第三世界的新國際體系也還是必須有從内外到外的幾個層次;這並不是像美國那樣爲了方便將搜刮來的利益向自己集中,純粹是爲了有效治理而別無選擇,否則就會倒退100年回到毫無作用的League of Nations。
至於易綱,兩年前的匯率操弄已經足夠他入祀忠犬祠;這次更離譜,直接抹煞Nabiullina的專業意見,護主如此奮不顧身,他的牌位必須排在von der Leyen之上了。
\\

\textit{\hfill\noindent\small 2023/08/25 09:19 提问;2023/08/26 06:40 回答}

\noindent[74.]{\Hei 答}:
以下是博客在SVB危機期間所作的既有分析:美國經濟沒有立即危險,必須等到家庭、公司和金融機構的流動現金大致耗盡,才會受到真正的擠壓,這還需要大約12-24個月,届時美聯儲可以簡單地將QT反轉為QE來“解決問題”,而QE對美國經濟的附帶惡果必須靠美元作爲國際儲備貨幣來稀釋。
所以你的問題的答案是“是的”,是上述邏輯論證的自然結論,除非未來24個月内替代美元有顯著新進展。
\\

\textit{\hfill\noindent\small 2023/08/25 13:38 提问;2023/08/26 06:26 回答}

\noindent[75.]{\Hei 答}:
我重申金磚組織不適合作爲貨幣核心,只適合當推廣機制。
\\

\textit{\hfill\noindent\small 2023/11/20 09:34 提问;2023/11/20 23:03 回答}

\noindent[76.]{\Hei 答}:
是個真實的問題。
這也是不應該試圖用人民幣替代美元儲備功能的另一個原因(參見新博文):美國當年是先取得軍事霸權,接著是宣傳霸權,最終拿下貨幣霸權之後,才敢搞全球化,根源就在於沒有軍事護航的全球化經貿根本不可能保障規則,連資本自由進出都有讓國内的金融秩序被外人攪亂的危險。這裏新博文沒有提到的一個典型案例是1929-1933年的德國:正因爲一戰戰敗,被迫將經貿金融對戰勝國開放,所以1929年華爾街引發金融危機之後,在歐洲工業國中德國受害最深,最終導致Hitler上臺。
中國不是戰敗國,也沒有外軍占領,居然對美國的金融掠奪也予取予求,實在是天下奇觀。
\\

\textit{\hfill\noindent\small 2023/12/17 15:53 提问;2023/12/18 03:51 回答}

\noindent[77.]{\Hei 答}:
“聼其言而觀其行”,在充滿各種虛僞的政壇尤其重要;專業上知曉其原理,但卻因各種主觀因素反其道而行,實際上比起懞然無知更可惡、更難辦。不是有句老話:“裝睡的人永遠叫不醒”嗎?學術詐騙集團如此,金融貨幣主管也是如此。
博客當然不是世界上第一個思考凱因斯方案的,連凱因斯自己也不是;但我應該是華語世界第一個能夠明確解釋所有脈絡、並敢於强力倡議正確措施的學人。畢竟學識必須匹配品格,才能對國家社會做出正面貢獻。
\\

\textit{\hfill\noindent\small 2023/12/17 15:55 提问;2023/12/18 03:38 回答}

\noindent[78.]{\Hei 答}:
唉,銀行界的壓力測試,我是真正動手做過最前沿、最高端的開創性研究的,算是業界的權威之一。
實際上這些測試裏最重要的,正是各種黑天鵝事件;然而它們先天就是統計曲綫上的尾巴,不可能從歷史案例簡單推論。偏偏政策監管必須要求能明文寫下細節的規範,以防止作弊。正因爲以上這個基本矛盾,壓力測試不可能完全成功;最好的銀行防爆政策,是增高安全儲備底綫,例如現在正在美國國會吵翻天的Basel III Endgame;但這又直接壓縮銀行的獲利率,所以JP Morgan帶頭强力反對。
SVB正是因爲歷史案例只用了不到50年,所以沒想到通脹復發會造成長期國債在短期内價值就大幅縮水。
\\


\section{【政治】社會主義國家應該如何管理資本}
\subsection{2022-07-17 00:21}


\section{21条问答}

\textit{\hfill\noindent\small 2022/07/17 04:24 提问;2022/07/17 23:14 回答}

\noindent[1.]{\Hei 答}:是的。

我做原創性見解向來沒有困難,所以也就沒有必要重複既有論述。這篇正文雖然内含一些新論點,但整體來説,主要還是總結過去8年寫過的許多博文。這當然是要有外來的重要管道邀稿,才值得我花時間精力去做。然而正規管道會先修飾改寫,比較尖銳的批評必然被刪,所以這裏的原稿還是有其價值的,希望有體制内人士來博客看到。
\\

\textit{\hfill\noindent\small 2022/07/17 16:26 提问;2022/07/17 23:31 回答}

\noindent[2.]{\Hei 答}:瞭解正確的大原則並不困難,難處在於應用到實際執行時,如何落實在細節上。例如馬克思的唯物辯證,是幾十萬個大陸學人朗朗上口的口號,但能説出怎麽用來分析21世紀的經濟和國安政策的,卻是鳳毛麟角。其實馬克思自己都有意回避如何運用國家機器來推行社會主義這個大題目,所以才會輪到列寧來頭一個吃螃蟹,然後20世紀有許多其他嘗試,其中鄧小平主導的改開已經算是最成功的了。然而中國顯然到了必須跨出下一個大步的時候,習近平也有心改革,所以用確實的案例、指出正確的方針,是重要而且急迫的工作。
\\

\textit{\hfill\noindent\small 2022/07/18 00:58 提问;2022/07/18 10:04 回答}

\noindent[3.]{\Hei 答}:你猜對了,正是“大一統理論”因連續違反規則而被拉黑之後,所選用的第6個、還是第7個新號。既然他依舊不在乎規則,那也只好先一見即刪,直到“蔚藍海洋”將滿6個月再拉黑。

這些誇誇而談的空想家有兩個共通點,就是1)連三人小組都沒有領導過;2)一輩子沒生產過任何實際用品。馬克思至少有自知之明,知道胡猜必然導致出醜;這些人連馬克思的著名先例都不懂得模仿,根本就是不值得評論的不入流人物。
\\

\textit{\hfill\noindent\small 2022/07/18 01:02 提问;2022/07/18 03:23 回答}

\noindent[4.]{\Hei 答}:現在都已經2022年了,如果不是被這些“馬克思主義者”搞成宗教性空談,即使在資本打壓之下,社會主義這樣理所當然的思想也早就應該普及世界。我不是説過,現實社會裏的好人,必須比壞人更精明能幹好幾倍。連續劇編劇自己的智商,就是劇中人物的上限,如果寫不出像樣的故事來,知識分子還是少去看的好。
\\

\textit{\hfill\noindent\small 2022/07/18 10:22 提问;2022/07/18 23:42 回答}

\noindent[5.]{\Hei 答}:可以看出這幾年你很用心閲讀博文,已經開始能把零零星星的討論在自己腦中組合起來;這正是我對讀者的期許。新讀者可以參考前文《科技發展與美式自由無關》。
\\

\textit{\hfill\noindent\small 2022/07/22 01:12 提问;2022/07/22 03:21 回答}

\noindent[6.]{\Hei 答}:我對Greenspan印象不好,在2015年的舊文裏已經詳細解釋過其根據緣由,請自行復習。對Outsourcing和中國崛起的作用毫無感覺,絕對不是美聯儲主席應有的表現;當然,幾年後有Yellen和Powell對照襯托,可能相對好看些。

美國的印錢始自1971年Nixon打破Bretton Woods,要説50、60年代興起的Monetarist Theory純屬巧合,就太小看宣傳洗腦的效果了。90年代的低息政策,和2008年後的QE都只是同一套思路自然演化的結果。

中文的“市場萬能論”並沒有與之對應的英文名稱;這也不是巧合:我說Friedman是傳銷天才,而在名稱上玩花樣,避免被抓小辮子,是傳銷學的基本功。
\\

\textit{\hfill\noindent\small 2022/07/27 05:42 提问;2022/07/27 11:31 回答}

\noindent[7.]{\Hei 答}:謝謝你的評論;畢竟我不在行業内,也沒什麽熟人可問,全憑公開信息來估算,心中總有忐忑。

我在2015年對半導體產業所作的預測,其實偏差不大,稍顯樂觀,但主要還是Trump撕破臉是當時無法預見的。不過從大局來看,這類產業升級和競爭只在全球化潮流下才是終極努力的目標:俄烏戰爭後的世界裏,外交、軍事和貨幣才是最重要的。
\\

\textit{\hfill\noindent\small 2022/07/30 11:07 提问;2022/08/01 12:48 回答}

\noindent[8.]{\Hei 答}:其實成品又分自產和進口兩類,其價格波動完全可以是互相獨立的。例如當前美國自中國進口的東西,價格依舊穩定,甚至因爲過去一年過度訂貨而疲軟,但是他們國内自產的貨品就因原物料和勞工的短缺,成爲這一波通脹的主要動力。一旦基層百姓的生活費用不斷升高,自然會有要求工資隨之上漲的動力,然後反饋到日用物品和服務的定價上,這就是我所談自我加强的循環效應。這個效應會使通脹固化:原本在初始階段,可以用2-4\%的真實利率打斷的循環,徹底固化之後就非得10\%才有效果(參見1980年的經驗)。然而現在的美聯儲貼現率仍舊只加到2.5\%,在通脹率已達8-9\%的背景下,對應著大約負6\%的真實利率,這正是我不相信美國光從歐日吸血就能解決通脹問題的主要考慮。

中國智庫的主要任務之一,是建立自己的統計資料庫,因爲統計數字是最重要的學術話語權;這是我以前反復談過的事。我知道上了年紀記性自然衰退,或許你該設法抽空復習博客的舊文。
\\

\textit{\hfill\noindent\small 2022/08/09 22:35 提问;2022/08/10 01:26 回答}

\noindent[9.]{\Hei 答}:無可抵賴,而且視地點、頻率、高度、時間(例如民進黨競選集會)有多重嚴重性級別可選擇,預先容許後續升級的空間。

有用;參考台灣人對日本統治的反應。

《博客須知》第九條規則要求注冊滿6個月才能發言,正是針對他而訂的;連帶影響其他心智正常的新讀者參與留言討論,是不得已的犧牲。如同人類社會必須將資源浪費在無直接經濟產出的法律和警察系統一樣,這也是一個“公地悲劇”"Tragedy of the Commons"的案例,並沒有完美的解決辦法。
\\

\textit{\hfill\noindent\small 2022/08/10 16:12 提问;2022/08/11 03:27 回答}

\noindent[10.]{\Hei 答}:是的,你説的都是客觀理性分析的必然邏輯結論,只要個人記性超過48小時,就不得不同意。這場事件與其說是國際博弈的過程,不如看成愚民自我洗腦的案例,對群衆心理學的示範意義遠大於外交戰略理論。
\\

\textit{\hfill\noindent\small 2022/08/11 23:14 提问;2022/08/15 00:23 回答}

\noindent[11.]{\Hei 答}:你説的是中共既有的標準政治經濟學。内容並沒有錯,但屬於基礎理論,距離實用有十萬八千里。就像做半導體的當然要先學量子力學,但繼續研究Schrodinger方程式並不會對開發新製程有幫助;這篇正文的主旨在於起到類似張汝京和梁孟松的作用。
\\

\textit{\hfill\noindent\small 2023/01/28 23:33 提问;2023/01/29 01:08 回答}

\noindent[12.]{\Hei 答}:金融先天就是Extractive加Extortionary,這是正文的要點之一。昂撒經濟系和金融系最喜歡說,發達的金融業有助於資金配置,事實是剛好相反,因爲真正賺大錢的“高等”金融,其核心本質都在於憑空創造信息不對稱,從而榨取暴利。這一點很難理解嗎?不是,但昂撒經濟金融學界原本就是資本利益集團的公關分支,不但不可能説實話,而且要千方百計地掩蓋、扭曲、反轉真相。
\\

\textit{\hfill\noindent\small 2023/04/23 18:12 提问;2023/04/29 01:56 回答}

\noindent[13.]{\Hei 答}:這裏要推動的不是電子政務,而是數字貨幣。
\\

\textit{\hfill\noindent\small 2023/05/17 11:20 提问;2023/05/24 03:12 回答}

\noindent[14.]{\Hei 答}:有關這波美國滯漲和50年前的對比,上月我已經在博客和《龍行天下》都提過,由於整體政治和學術文化的衰敗,當前完全看不出任何可行的解決方案。

連美國普通高中畢業生(例如泄密案的Jack Teixeira)都知道自身體制病入膏肓,還在迷信的台灣群衆,只能用上“弱智”兩字;其中居然還有大批所謂的精英,更加匪夷所思。我對馬英九和龍應臺的痛恨和鄙視,便源於此。
\\

\textit{\hfill\noindent\small 2023/06/08 00:08 提问;2023/06/10 00:25 回答}

\noindent[15.]{\Hei 答}:我覺得是這篇正文中有關金融的建議被采納了;不過我還在等待針對商學院的整頓。
\\

\textit{\hfill\noindent\small 2023/07/19 23:26 提问;2023/07/20 06:29 回答}

\noindent[16.]{\Hei 答}:經濟學雖然是社科中資源最豐富、研究方法模仿自然科學最激進的一支,然而因爲原本的用意就在於模仿皮毛表面來為財閥竊國做公關背書(參見博客多年來的相關討論,尤其兩個月前有關量化分析的評論),實際的水準非常糟糕,就連諾貝爾獎得主這個層次也有絕大多數完全脫離現實,對未來所作預測的準確度還不如根據現有最新資料做簡單綫性外推(亦即考慮統計噪音之後的linear extrapolation;換句話説,他們的理論分析貢獻是負值,參考過去四年這些人對歐美通脹壓力的評估)。但正因如此,少數脚踏實地的經濟研究人員往往可以比學術大佬、官方領導和主流媒體都更早領悟真正的發展趨勢,這裏剛好是一個案例:與我合作的深圳智庫半年前就提早私下聯絡,認爲中國經濟走勢其實弱於當時的官方和媒體預期,問我有何對策。我說這個落差來自歐美進口需求的急劇收縮,其“因”是西方世界的長期衰落,其“果”則主要打擊代工出口導向的經濟體,如台灣、韓國、越南等等,中國只是因爲全球化供應鏈而受到次要波及。從經濟管理的角度,我們原本就處在一個比爛的世界,只要相對强勢就是勝利;在戰略競爭的層面,對手受傷而導致一點附帶損害(collateral damage),更明顯是件好事,所以完全不必對短期波動做激烈反應,只要繼續穩健管理即可;真正的關鍵在於消弭影響經濟長期發展的潛在阻力和浪費,所以既有的改革(例如房地產)不該撤銷,像是教育部大幅擴編本科招生這種對國家社會的深遠自殘才必須儘快反轉。

與我合作的智庫是經濟學專業,因而對韓策略這種外交軍事方面的議題不太説得上話。
\\

\textit{\hfill\noindent\small 2023/07/20 21:36 提问;2023/07/25 07:15 回答}

\noindent[17.]{\Hei 答}:幾個月我解釋過,美國這一波通脹危機的遠因是印錢過多,發作的導火綫則是新冠和俄烏戰爭,但跨行業傳播的媒介卻是底層工資(尤其服務業)補償40年來的拖欠,以及財團壟斷所導致的商家定價權(Pricing power)。以長遠的眼光來看,底層服務業工資終於上漲,有利於縮減貧富差距,促進社會收入分配的合理化,但這是一個很次要的作用,遠遠不足以解決美國經濟的真正沉疴,也就是去工業化,或者說虛擬化。

中國的零通脹也是一樣的:物價低當然有利於消費者,尤其是底層消費者;但長期經濟管理的關鍵還是在於實體工業的持續發展和升級。只要股市和GDP成長率不影響重點方向的投資數額和效率,就無關宏旨。博客向來專注批評的重點,一個是科技報導的假大空,另一個是高等教育脫離實業需求,前者決定投資的效率,後者關乎經濟發展的方向(亦即實體經濟所佔人力資源的比重),才是中國未來30年的前途所繫。
\\

\textit{\hfill\noindent\small 2023/08/15 00:17 提问;2023/08/15 02:07 回答}

\noindent[18.]{\Hei 答}:不只是“近10年”,而是房地產業(及其相關的金融運作)從私有化開始就被多方反復操弄,中央政府拿來保障GDP成長率,地方政府則當成搖錢樹;2008年更是一次性透支了20年的經濟合理性和發展潛力。歸根究底,主管經濟和尤其金融的官員及幕僚,以模仿複製美國體制和政策爲榮,對其中資本搜刮效應損害實體經濟和腐蝕社會根基,都完全無知並無視。

不過這種飲鴆止渴、因小失大的政策,還不算最糟糕、最離譜的:高等教育低級化,四年擴張大學招生40+\%,完全沒有任何正面價值,連立即結果都是非常負面的青年失業,更別提長期的挖空國力是在比財務更重要的人力資源方面,這才是無比邪惡愚蠢的。
\\

\textit{\hfill\noindent\small 2023/08/19 00:30 提问;2023/08/19 04:03 回答}

\noindent[19.]{\Hei 答}:房地產是金融資產的一個大類,也是中產階級最方便觸及的那一類。玩弄房地產泡沫必然面對的,既有虛擬和實體經濟之間的傾軋,也有階級利益的矛盾,非常不適合用來填補政府的經濟或財務數據。

請注意《讀者須知》第三條“不能自説自話”和第四條“用言精簡”,嚴重警告一次,禁言一個月。
\\

\textit{\hfill\noindent\small 2023/08/31 15:33 提问;2023/09/08 07:11 回答}

\noindent[20.]{\Hei 答}:這些統計數據並不包括新興的巨富理財工具,例如對衝基金和私人股權投資;事實上,這些特殊金融管道興起所代表的剝削,很可能正是主流投資報酬率下降的主因。
\\

\textit{\hfill\noindent\small 2023/09/19 10:07 提问;2023/09/20 09:25 回答}

\noindent[21.]{\Hei 答}:是的,房價由未來的使用權決定,而這個使用權的價值,絕大部分來自周邊的外在因素,例如所在地公共設施的方便性和地籍所賦予的政治經濟特權。但房產公有制雖然聽來理想,歷史上沒有真正成功的案例;我認爲錯誤並不在私有制,而在於社會主義國家本應該盡可能壓低“周邊外來因素”中所含的不平等因素,例如城鄉戶籍差異所代表的教育和醫療服務上的品質差距。當然,中共實際上是把房地產當成經濟調控工具和搖錢樹,那當然會有無數長遠而難以解決的後遺症。
\\


\section{【美國】海湖莊園抄家事件幕後的美國政治鬥爭}
\subsection{2022-09-16 10:55}


\section{1条问答}

\textit{\hfill\noindent\small 2022/09/23 23:56 提问;2022/09/25 13:19 回答}

\noindent[1.]{\Hei 答}:一.是的,這是Deep State“勸阻”Trump參選的手法之一,正文裏已經簡略提過。以家人為人質,的確很下流。不過美國的司法體系對白領犯罪非常寬鬆(是資本多年來有意造成的;參見《美國式的恐龍法官》系列),Trump一家要脫罪的可能性並不小,所以算不上是什麽殺手鐧,説不定反而鼓勵他積極參選,以便用特赦來對聯邦刑事案一勞永逸,總統權力對州級的民事案也有間接影響。

二.這其實是一周前為上唐湘龍節目準備過的第三個話題,時間不夠沒有談;這裏我簡單列舉大綱。美元匯率和利率調整,影響的主要是國際貿易和資金流動(對國内需求面的影響,很容易就被通脹上升所引發的消費潮所淹沒;參見兩周前我在另一篇文章後注中討論的企業和家庭現金儲蓄);問題在於美國經濟對進出口的依賴相對來説並不高,而其國内供給面面對的卻是40年的舊債必須一次還清。原本美聯儲和聯邦政府如果在2021年政權輪替之後,立刻以控制通脹為第一優先,還有機會暫時壓下、維持既有經濟格局;現在已經太晚了,所有的國内通脹因素一起爆發,最近的經濟資料都指出通脹已經普及到整體經濟的每個層面(指薪資和成品,過去40年一枝獨秀的資產通脹必須做出若干囘吐,大宗貨物則有起有落,見下文),並且正在繼續深化的過程中。短期内美國通脹指數的稍微緩解,除了美元强勢的作用之外,最主要是能源價格回落(而這裏的最主要原因是中國經濟不景氣,能源消費大幅降低)。雖然在對未來一兩年的中期預測上,歐、日都將面臨世紀級的經濟衰退,進一步減少全球能源消費,然而農產品因氣候因素而短缺的問題還在,美國國内的通脹因素也足夠壓倒美聯儲當前這種一次75基點的升息,而且還有最具決定性的去全球化帶來對美元的替代,所以Yellen所説的明年徹底解決,是癡人説夢。

至於昂撒體系通過IMF收割第三世界,博客已經談了八年,討論亞投行的時候,說得尤其直白。當時中國知識思想界和金融管理階級都不聽,現在臨時抱佛脚怎麽可能來得及?只能眼看著他們在金融上被搜刮,財富被用來填補美國的國力空缺;不過政治上倒不是問題,因爲這些國家有了全新的教訓,對建立替代性的國際體系應該會更積極。
\\


\section{【戰略】【國際】對俄烏戰爭的新觀察}
\subsection{2023-10-31 14:28}


\section{2条问答}

\textit{\hfill\noindent\small 2022/12/21 19:03 提问;2022/12/22 03:40 回答}

\noindent[1.]{\Hei 答}:如同德國和歐盟的擴軍計劃一樣,不必理會。這些努力要有實際結果,少則10年、多則20年,届時霸權歸屬早已塵埃落定,何況他們的經濟即將摔下斷崖,政治必然因而極度不穩定,能維持政策到有成果的機率甚低。
\\

\textit{\hfill\noindent\small 2023/02/06 14:56 提问;2023/02/08 04:35 回答}

\noindent[2.]{\Hei 答}:你拿俄烏戰爭和越戰做對比,是個很好的觀察;這裏的差別,在於前者不但是謀定而後動,而且已經退守到己方邊境勢力範圍之内,道義和後勤上都與遠程侵略相反。此外,我還要補充一點:越戰也是美國貨幣超發、迫使Nixon打破Bretton Woods的主因,從而引發了全球通脹、美元石油霸權和後來美國經濟金融化(爲了扭轉滯漲,除了美聯儲嚴控需求面之外,也必須從供給面壓制通脹,對工資循環上漲做釜底抽薪,是80年代開始產業外包的重要背景因素,否則當年不少政經精英對去工業化有所質疑;參考Volcker自己的演講詞:“A controlled disintegration in the world economy is a legitimate object for the 1980s.”)。

至於實際執行,俄軍動員了至少50萬機械化部隊,現在下場的從去年九月的1/5提升至1/3,就已經扭轉戰場態勢,讓烏克蘭窮於招架,後續純粹是貓玩老鼠,最終結果沒有任何疑問;你沒注意到上周我上節目,完全懶得討論戰況細節嗎?
\\


\section{【戰略】【國際】簡評當前的G7對華態勢}
\subsection{2023-05-07 03:26}


\section{9条问答}

\textit{\hfill\noindent\small 2023/05/07 05:53 提问;2023/05/11 08:01 回答}

\noindent[1.]{\Hei 答}:你引用錯了。我説的是,美國經濟在短期内沒有系統性崩潰的危險,美元反而是其中最弱的一環。

美國面臨新一波Stagflation,卻和70年代有差異,傳播通脹的主要機制不是工會、而是企業的寡頭獨占。當年靠著Outsourcing解決通脹、打贏冷戰解決對美元地位的威脅,這次看不出體制可能容許的解決辦法。
\\

\textit{\hfill\noindent\small 2023/06/05 00:47 提问;2023/06/06 06:10 回答}

\noindent[2.]{\Hei 答}:數字是否正確,我無暇確認;倘若是真,那麽應該正是教育與產業不相匹配的徵象,只有大幅削減文科名額能很好地解決。


數據出處我找到了,是2023年4月的24嵗及以下失業率超過20\%,創下歷史新高。國務院的因應措施我也看到了,居然是要求地方政府創造臨時崗位,這別説治本,連治標都談不上,純屬亂作虛功、敷衍上級;官員及智庫的水準真是一言難盡。
\\

\textit{\hfill\noindent\small 2023/06/07 11:28 提问;2023/06/10 00:06 回答}

\noindent[3.]{\Hei 答}:哎,日本這30多年,始終是在消耗以往藏富於民的底蘊,圖利國際資本。常用的經濟指數流於浮面,特別方便瞎子摸象、自説自話。

以Warren Buffett爲例,他清倉BYD之後,國際“投資”的重點就是轉入日本,但這和日本經濟健康一點關係都沒有,而是因爲日本的實際通脹率超過4\%,中央銀行利率卻低於1\%,而且還獨步天下地公然壓制整個yield curve(QE其實本質也是壓制yield curve,但算是偷偷摸摸地幹),所以他很簡單地去買日用品和Utilities這些recession/inflation-proof的股票,連美元都不用出,直接在日本借日元;這是無本萬利還沒有風險(含匯率風險)的行當,但只限有本事一次借上几萬億日元的人才夠資格真正受益。日本在1990年代引進美式金融體系,所以這樣“無中生有”的利潤來源被層層遮蔽,然而實際上買單的當然是當地的中產階級。
\\

\textit{\hfill\noindent\small 2023/06/07 16:51 提问;2023/06/09 23:34 回答}

\noindent[4.]{\Hei 答}:我看到國務院因應失業率飆升的處置,也是囘想起陳水扁,所以才會對官員的水準做出直面批評。

我建議大幅削減文科生名額,的確不是因爲鄙視文科,剛好相反,我對真正追求“美”的文藝和追求“真”的社科都極度尊重;正因為這個尊重,所以不能容忍庸人拿它們當招牌來混飯吃,更不能坐視他們成爲昂撒殖民帝國的走狗。
\\

\textit{\hfill\noindent\small 2023/06/07 21:29 提问;2023/06/10 00:13 回答}

\noindent[5.]{\Hei 答}:長期和短期因素的確是獨立的,但失業率並不是純粹的短期效應,而是長短期的叠加。像是近年來旱澇加劇,每一個災情都有短期天候因素,但整體來看,頻率和嚴厲程度的增加顯然是全球暖化的結果。

這次高中和大學畢業生失業率突破20\%,其重點不在於特定的數據門檻,而在於它在過去十幾年級級上升,屢次創下歷史新記錄,因而明顯是長期的結構性問題。再考慮同時段生育率大幅下降,那些高喊“人口紅利消失”的人,居然能轉過頭來又說“經濟不足以拉動就業”,如此明顯的邏輯自相矛盾,實在可笑至極。我再説一次,這就是教育與產業錯配的結果:很多高中生應該念高職(台灣用語,指等同高中級別的職業學校),很多大學生應該念工專(台灣用語,指等同本科級別的職業學校),尤其文科教育90\%是垃圾,正是問題的核心所在。一旦沒有垃圾學科可供混文憑之用,中學生自然會認真考慮念職業學校。
\\

\textit{\hfill\noindent\small 2023/06/08 03:52 提问;2023/06/10 00:22 回答}

\noindent[6.]{\Hei 答}:美國的確也有嚴重的垃圾文科問題(事實上幾年前我已經討論過了,只不過當時著重於它對白左興起的貢獻),之所以沒有大幅反映在年輕人的失業率上,其實正是其產業空心化、虛擬化、金融化的效應之一,製造了大批垃圾服務業剛好可以消化垃圾文科生。過去三年那麽方便在家辦公,現在企業企圖扭轉這個趨勢如此困難,也都是同一效應的附帶結果。
\\

\textit{\hfill\noindent\small 2023/06/09 14:39 提问;2023/06/10 00:24 回答}

\noindent[7.]{\Hei 答}:深究細節當然有其意義,但我覺得此事光是從歷史趨勢和環境因素來看,就足以做出定性結論。
\\

\textit{\hfill\noindent\small 2023/11/06 20:45 提问;2023/11/07 12:17 回答}

\noindent[8.]{\Hei 答}:當前美國治下的“遵從規則的國際秩序”,是一個金字塔型的大一統全球殖民帝國。細分坐在頂端的美國本身,最上層的真正統治階級是幾十個隱藏著的巨富家族,其中有相當一部分是猶太人。地球的幕後統治者要搞種族滅絕,那麽我們除了試圖推翻既有秩序之外,還有什麽有實際意義的反抗作爲呢?你覺得我爲什麽一直處心積慮地要推翻美元霸權?因爲那是這一切邪惡的基礎啊。所以犧牲國家利益也要維護美元地位的周小川和易綱,不但是賣國賊,也是人類最大邪惡勢力的重要幫凶。你想這些崇美官員爲什麽看到成千上萬的兒童被屠殺還在支持以色列?如果承認巴勒斯坦的苦難,就等同招認自己的邪惡和罪過呀。這在心理學上叫做Cognitive Dissonance,有空自己查查Wikipedia。

政治經濟社會的真相原本就是極度晦澀、艱難、複雜的,然後再經過大衆媒體的扭曲遮掩,群衆當然沒有能力自己看穿真相,所以我才投入這麽多年來做出系統性的解釋,然後任何人只要肯花點工夫來閲讀博客,自然就能跨越財閥所依賴的先天和後天信息障壁,不再有藉口當鴕鳥了。現在我們面臨的挑戰是,如何讓中方的金融主管們追趕上群衆的認知。
\\

\textit{\hfill\noindent\small 2023/11/13 01:44 提问;2023/11/13 09:15 回答}

\noindent[9.]{\Hei 答}:以下是我在兩周前寫下但還沒有公開的總結;重點句包括“基本不需要中方主動求變"和“對美方不排斥交流、不幻想和解、不接受詆毀、不相信承諾":

(因爲人民銀行錯過了調整匯率的時機,容許美國穩住經濟,短期内難以撼動;而且)在美國成功渡過急性通脹危機之後,其不擇手段維護殖民帝國體系的剝削性本質卻也圖窮匕見,新一波反殖民運動引領世界潮流,包含俄烏戰爭、Niger革命、金磚擴員、以巴戰事、以及還將不斷發生的新響應,短期内國際情勢演化方向已成定數,基本不需要中方主動求變,只要維持既有的外交站位和戰略方針,不隨内外噪音起舞,繼續為第三世界反殖民鬥爭的第一綫提供後勤和道義支援即可。換句話説,對美方不排斥交流、不幻想和解、不接受詆毀、不相信承諾,遇有真金白銀的衝突,則固守理想和原則,堅決爭取合理合法前提下的最大利益。
\\


\section{【國際】【戰略】國際政治與個人事業規劃}
\subsection{2023-09-17 07:45}


\section{1条问答}

\textit{\hfill\noindent\small 2023/09/18 09:51 提问;2023/09/18 12:42 回答}

\noindent[1.]{\Hei 答}:真正統治在金字塔頂端的,是幾十個或者一兩百個Billionaire家族,他們早就針對社會和政壇的理性力量做好了多層的縱深防禦:最前方的警戒陣地是票選政客,第一道防綫是民衆白癡化,第二道是食利的司法體系,第三道是資本傳媒,第四道是文化腐蝕,第五道是學術扭曲。現在整個歐盟,除了匈牙利,都在Soros和WEF所培訓的歐奸掌握之下;換句話說,理性思維連警戒陣地都無法突破,更別提後面的真正防綫了。
\\


\section{【金融】【戰略】國際金融未來趨勢}
\subsection{2023-09-17 08:25}


\section{6条问答}

\textit{\hfill\noindent\small 2023/09/21 17:57 提问;2023/09/22 09:08 回答}

\noindent[1.]{\Hei 答}:哪有什麽相似?定價權是商品價格使用某貨幣計算,不是貨幣價值錨定在商品價格上(更別提我明明剛剛解釋過,定價權直接拿人民幣爭取就行了,不需要合成貨幣)。商品價格先天就會根據意外消息而浮動,而貨幣不管是國際還是國内,都必須保值,否則不是貨幣而是投機工具。尤其像這樣子名義上錨定在商品價格,卻不囤積後者以保證兌換,集金本位和信用本位之缺點於一身,連Crypto都不如,基本等同NFT。拿這麽明顯低劣的詐騙工具和猶太/昂撒集團做鬥爭,是嫌死得不夠快嗎?

此外,正文中特別指明人民幣不適合直接當國際儲備貨幣的原因之一,就在於鼓勵發行國債、赤字消費,更別提以前已經有人拿翟的這個説法來問過了。最可惡的是,明明他説的,都是我反復駁斥過的錯誤論點,上周才强調是禍國殃民的歪論,你居然敢説“與您非常類似”;難道你以爲爲了情面我就會容許顛倒黑白、指鹿爲馬嗎?像你這種人不拉黑,還有什麽人值得拉黑?
\\

\textit{\hfill\noindent\small 2023/09/21 21:26 提问;2023/09/22 08:21 回答}

\noindent[2.]{\Hei 答}:這些錯誤説法,以前博客反復駁斥過了,不值得再提。

經濟學最有趣、最精微的地方,就在於一些反直覺的現象,它們通常是Game Theory的結果。金融學作爲經濟學中專門研究虛擬資產的分支,反直覺的案例特別多。我並不是說只有科班出身的人才能參與討論,但是否融會貫通應該自己心裏有譜,如果沒譜,就不應該以噪音污染公共論壇,尤其不能在金融議題上一拍腦袋只凴直覺説話。翟東升相對於大陸99\%的政治學者來説,明顯是優秀的,所以我一直想要客氣,但這並不代表我必須在金融方面也假裝他説得有道理;畢竟求真才是博客的基本原則。
\\

\textit{\hfill\noindent\small 2023/09/22 09:00 提问;2023/09/22 09:21 回答}

\noindent[3.]{\Hei 答}:這些論點博客解釋過幾十遍,但偏偏始終有讀者硬要以名氣取人,前仆後繼地想方設法要利用我的客氣來為歪論背書。今天殺鷄儆猴之後,更加不會手軟;還想把自己的半吊子偶像強植在博客這個净土上的,我保證來一個殺一個,來兩個殺一雙。若有必要,會考慮將發言冷靜期延長為一年,以嚇阻自私自利者濫用公器的衝動。
\\

\textit{\hfill\noindent\small 2023/10/10 21:10 提问;2023/11/06 17:04 回答}

\noindent[4.]{\Hei 答}:
最近有一連串處理債務的相關措施,包括今天出臺的《財政部通報8起地方政府隱形債務問責典型案例》;但是也有幾天前新發一萬億國債來“挽救經濟”的決定。
我認爲新國務院想要解決多年沈厄,這是好事;但他們並未理解發債撒錢的代價正在大幅提高之中,這是隱憂。參見幾周後會發表的新博文《金融史觀下的政策改革建議》。
\\

\textit{\hfill\noindent\small 2023/12/14 10:44 提问;2023/12/15 05:02 回答}

\noindent[5.]{\Hei 答}:
美國金融殖民體系植根於潮汐式定期收割,被放牧的國家除了在外交、軍事和宣傳上被綁定之外,經濟方面也會對美方所飼喂的市場、技術和投資上癮,然而這些飼料是含有高量經貿瘦肉精的。換句話説,中國自改開以來,尤其是加入WTO之後,幾十年的粗放快生快長,的確是美國有意縱容配合才可能發生的捷徑。現在一方面美國從外收回飼料,另一方面帶路黨和既得利益者在國内哭餓(臺語的“靠夭”韻味更足),這種兩難的夾擊困局,正是當年全心加入全球惡霸所主導國際秩序的必然後果。目前中國的選項,實際上只有兩個:要嘛跟從德國,任由美國搜刮;要嘛學習俄國,忍一時之痛,換取真正的獨立自主。
這是因爲那些傻蛋所幻想的發債刺激、重啓經濟,其實是飲鴆止渴的自殺性行爲。人民幣並沒有美元的國際霸權,中國也無法像美國那樣定期從外部吸血幾十萬億美元;任何輕率短視的金融財政政策,不但不能以鄰爲壑,而且在美國虎視眈眈之下,惡果會成倍爆發。也就是說,無限發債狂歡的結果,是兩三年的6\%成長之後,GDP大幅萎縮的全面危機會越來越無可避免,届時即使國家勉强維持完整,政權核心也不可能再有對美國説不的底氣。歐盟至少還是白人,又兼黃金十億的相對内層,尚且經歷最近兩年的宰殺;中國是美國霸權的終極威脅,三五年後美國財政窟窿又比上次更大幾倍,到時的肢解吃人,連骨頭都不會吐出來。若干政學商精英固然不介意成爲中國版的von der Leyen,跟著起鬨的網民卻必須分擔自己愚蠢的代價;考慮到全球70億的無辜受害群衆,他們不值得任何同情。
出自《三體》的那句“弱小和無知不是生存的障礙,傲慢才是",其實似是而非,畢竟傲慢同是愚蠢無知的必然後果。正確的論斷是:“這個世界上最大的危險,莫過於真誠的無知和認真的愚蠢。”
\\

\textit{\hfill\noindent\small 2023/12/26 16:59 提问;2023/12/27 07:55 回答}

\noindent[6.]{\Hei 答}:金融投機就投機,何必美其名為投資?投資是非被套牢也計劃30年期退休用的,散戶的收益能彌補通脹就不錯了。我自己私產投資尚且知道拼不過對衝基金的資訊和管道,不能跟著炒作投機,所以平均持有期在5年以上,何況非專業人員?

全世界的投機者都只想著賺容易快錢,與實體經濟發展原則背道而馳;他們不爽,才代表監管單位可能做對了。Powell可以爲了取悅華爾街而提早降息,靠的是美元金融霸權,中國可沒有那樣的背景可倚靠。所以我常説中方的監管不是太嚴,而是不夠嚴格,以致被金融巨鰐完成搜刮之後才亡羊補牢,反而讓股民更加不爽。
\\


\section{【金融】【歷史】金融史觀(一)歷史由來}
\subsection{2023-11-20 23:55}


\section{2条问答}

\textit{\hfill\noindent\small 2023/11/25 08:31 提问;2023/11/26 06:29 回答}

\noindent[1.]{\Hei 答}:是的,這兩年是潮汐周期中的宰殺收割階段,平常被放牧養得越肥,死得就越慘,例如南韓和越南。

中國造船業的確在技術上還稍微落後韓方,同樣功能的船隻往往多出百分之幾的死重,但在成本上一直有些優勢,足以彌補前述的不足。韓國靠的則是來自歐美人爲的技術、資本和市場支援;現在國際資本價格開始向天然合理的利率回歸,就保不住市場份額了。
\\

\textit{\hfill\noindent\small 2023/12/23 05:09 提问;2023/12/24 05:40 回答}

\noindent[2.]{\Hei 答}:
是的。你所理解的,正是我在《社會主義國家應該如何管理資本》一文中所强調過:金融越簡單越直接越好,任何不是絕對必要的自由度,都必然會成爲金融巨鰐掠奪國民財富的空間和手段;國内如此,國際亦如此。不替換美元,國際金融體系只會不斷為美國搜刮方便而無限複雜化,正解只能是砍掉重建,參考Gordian Knot的故事。
\\


\section{【金融】【戰略】金融史觀(二)當前局勢}
\subsection{2023-11-21 00:02}


\section{10条问答}

\textit{\hfill\noindent\small 2023/11/22 01:21 提问;2023/11/22 06:26 回答}

\noindent[1.]{\Hei 答}:減排的關鍵在於碳核算。所謂的碳補償和碳交易都故意規避這件事實,完全無助於解決問題;唯一的功用是作爲金融財閥榨取社會財富的新手段。所以美元錨定於碳排放,也只不過是又一種金融掠奪的新發明罷了,不會對替代性儲備貨幣造成阻礙。
\\

\textit{\hfill\noindent\small 2023/11/23 10:22 提问;2023/11/24 00:02 回答}

\noindent[2.]{\Hei 答}:戰鬥在第一綫是必須做出額外犧牲的;既然俄國和Hamas志願,而美國的通脹弱點已被暫時彌補,中國當然沒有必要再衝出來殺敵八百、自損一千。這一點我在《從SWIFT制裁俄國,看中國的對應之道》早已解釋過。

碳交易、碳回收和碳補償能讓中國跟著買單固然是白賺,但財團的第一優先是利潤最大化,而大部分的錢往往還是由富國出。例如2022年美國能源業因俄烏戰爭而提高營收5000億美元,其中有3000多億來自北美,剩下大半來自歐洲,只有小部分來自第三世界。美國醫藥界的暴利,也是主要來自對國内消費者的搜刮。
\\

\textit{\hfill\noindent\small 2023/11/29 08:19 提问;2023/11/30 02:57 回答}

\noindent[3.]{\Hei 答}:美國人民是全球殖民體系中利益向上集中傳送的管道,所以同時是吸血的受益人和被害人。傳送過程中他們是净吸血還是净出血,視大環境和財閥需要而定:在1980年之前明顯是前者,後來越來越向後者傾斜。

不過只因爲他們也吃虧就指望他們覺醒,那也太天真了。聽説過囚徒困境嗎?即使他們是全知而且理性,也無法擺脫自私的局限,更何況美國這些年開發出來的快樂教育和奶頭樂媒體,正是為了抹殺理性思維。台灣人在殖民利益輸送管道的位置還更低得多,尚且對現狀十分滿足,美國老百姓那樣的智商是不可能集體醒悟的。
\\

\textit{\hfill\noindent\small 2023/11/30 13:52 提问;2023/11/30 23:45 回答}

\noindent[4.]{\Hei 答}:是的,一旦認知架構有了正確的提綱挈領,改用金融史觀來觀察現代國際社會的治理體系和經貿結構,就會豁然開朗,到處都發現美國不勞而獲、剝削全球的細節。

這裏真正驚人的有三點:其一是美國剝削所得規模之豐厚,遠超一般人所能想象,參見正文【後注一】。其二是美國人指鹿爲馬、顛倒黑白之無恥,同樣遠超常人所能及;別説對中國的多般無端指責,就連人類史上最大的種族滅絕和資源掠奪,而且還發生在工業化已開始的近代,居然也被描述成文明蓋過野蠻的“建立人間天堂”的過程。其三是受害國家裏,有如許多的崇美者,心甘情願地作爲出賣自己同胞的買辦,而且還能普遍脫穎而出,成爲文化甚至政壇的“精英”。

我覺得以上這三點,一個比一個離譜;百年之後的歷史學家回顧這段時期,必然會瞠目結舌,驚嘆不已。
\\

\textit{\hfill\noindent\small 2023/12/06 06:41 提问;2023/12/07 03:11 回答}

\noindent[5.]{\Hei 答}:
如果堅持持有至到期,那麽損失體現在所得利息遠遠低於通脹,所以實際價值大幅縮水,幅度至少等同正文中所提的50-70\%。
2019年我已經精確預測通脹即將來臨,完全有足夠時間將資產配重轉移到抵抗通脹的類別,包括農田(例如Bill Gates)、礦產和Utilities(“Alternative Assets”,過去幾年有許多Private Equity和Hedge Funds投入數以萬億美元計的資金)、黃金、或者揩美國殖民地的油(例如Buffett玩的發日圓債炒當地股的游戲)等等,最不濟也可以在利率曲綫即將倒挂之前改用短期債券,總之就是不應該在長期國債上被套牢。
\\

\textit{\hfill\noindent\small 2023/12/07 01:30 提问;2023/12/07 03:40 回答}

\noindent[6.]{\Hei 答}:
是的,利用美元超發來做剝削的模式A,50多年百試不爽,除非國際社會集體替換美元,否則沒有徹底解決方案。所以我一再解釋,易綱真正的罪過不在於那幾千億美元的Unrealized Loss,甚至不在於沒有對美國通脹危機落井下石、使其無暇對中俄搞事,而在於對新國際儲備貨幣的暗中破壞,從基本層面維護了美國搜刮全球、維護殖民帝國體系的能力,使中國的外交、内政、經濟等等都必須面對額外十年的强大逆風。
\\

\textit{\hfill\noindent\small 2023/12/07 04:04 提问;2023/12/07 04:54 回答}

\noindent[7.]{\Hei 答}:債券和股票還算是最基本的“Underlying Assets”,一旦開始玩Derivatives,那就連發行方都無法完全瞭解其風險,更不用提買家了。
\\

\textit{\hfill\noindent\small 2023/12/07 12:54 提问;2023/12/08 06:12 回答}

\noindent[8.]{\Hei 答}:
我不認爲Putin會在那種級別的會議中,討論長期美債損失這種金融專業細節;這更可能是體制内理解美國金融操作的幕僚所提起的,畢竟國務院剛剛換新。
\\

\textit{\hfill\noindent\small 2023/12/08 13:42 提问;2023/12/09 01:50 回答}

\noindent[9.]{\Hei 答}:
金融獲利手段如果被寫入教科書,一般是錯誤或已經失效才會發生。就算原本還沒有失效,一旦公開,大家一起爭取這個價差,立刻就會將“獲利機會”抹除並扭轉。這有點像把“我曾在某地撿到一張大額鈔票”寫進教科書,然後讓大家一起到那個地方等鈔票出現。能長期獲利的例外手段,背後必定有一個永不退縮的“凱子”,這若不是中央銀行,就是公衆。所以不管是炒股達人或金融教科書,你在閲讀他們的獲利指南的時候,第一個問題必須是“凱子是誰?”如果答不出來,那麽你自己就是。
至於你所引用的這個特例,只要看到“做空”兩個字,就應該知道這是在唬零售客戶,和中央銀行的政策選擇無關。原因有二:1)幾千億的資金運作,而且是官方的錢,根本不可能搞大規模做空。2)利率變動對債券剩餘價值的影響,與其期長(duration或者term)成指數關係;換句話說,對短期債券的價值影響極小。所以做空短期債券,交易成本極高,而獲利極小,先天就不在中央銀行的考慮範圍之内。
我之所以說應該買短期債券,是因爲這次的利率倒挂並不是普通刺破泡沐的Generic Event,而是源自幾十年的通脹壓力一次爆發,所以可以預見倒挂的過程中,是短期利率和長期利率一起上升(而不是長期利率下降;一般刺破泡沫之後會有經濟衰退,引發通縮,從而壓低長期利率),只不過前者先行一步而已。我說在即將倒挂就動手,也考慮了幾千億的資金運作需要很長時間(一年以上),必須預留足夠時間;剛好短期債券的損失在利率上升過程中無關痛癢,而上升之後卻能提供合適的回報,所以是值得考慮的手段。
這裏還有幾個細節值得解釋:第一,判斷通脹時代來臨,比判斷即將倒挂難多了,更別提要提前兩年;畢竟前者是50年一次的例外,後者是經濟周期中的常態;所以遵循我建議的利潤也遠高於一般的利率曲綫買賣運作。這是因爲競爭者少,不但價差大,而且量足。不過這裏靠的,正是超越整個行業的獨特眼光,所以也就不可能在行業内尋得。第二,長期美債流通性並不高,買了之後要賣出很不容易,特別容易套牢;短期債券沒有這個問題,反正賣不出去也可以等它到期。第三,流通性不高的問題,對中央銀行這種動輒几千億的投資客戶尤其重要;一般金融系“學者”以爲反正可以做空或甚至更複雜的Derivatives,其實流通性低還硬要去做交易,就代表著極高的交易成本,使用Derivatives不但不能解決這個問題,事實上Derivatives的發行方會在這些交易成本之上,再加上幾倍的利潤,買家吃虧更大,只不過經過Derivatives的包裝隱藏,會計報表上看不出來罷了。
\\

\textit{\hfill\noindent\small 2023/12/17 15:49 提问;2023/12/18 03:43 回答}

\noindent[10.]{\Hei 答}:
全球外匯儲備都這麽高,基本原因是作爲抵抗1997年式(模式C)搜刮的壁壘;這一點在正文裏特別强調過了,請用心理解。所以至少一小部分必須是相對流通性較高的資產,短期國債尤其合適。
當然不能指望高層領導如人民銀行行長能夠自己懂金融達到提早兩年預測50年一次大轉折的地步,但體制内外的專業討論總該注意一下,這也是幕僚、參謀和智庫存在的意義;例如我這十年來寫的文章,並不只在公共論壇上流傳。
那句“一夜之間換成現金”,指的是若臺海開戰,怕被歐美扣押。其實幾萬億美元的央行資產當然必須多元化:有些追求長期保值,有些必須靈活備用;前者可以設法躲開官方,後者沒辦法(除非替代美元),只能盡可能壓低規模。1950、60年代開始,英美金融精英開啓了專藏黑錢的新管道來和老牌瑞士銀行競爭,成長至今的規模完全可以輕鬆容納央行對“Alternative Assets”的需求。如果堅持要直接投資實體資產,挪威、沙特、新加坡等國的國家財富基金也早就開路過了,沒有藉口。
\\


\section{【經濟】【學術管理】金融史觀(三)政策建議}
\subsection{2023-11-21 00:06}


\section{9条问答}

\textit{\hfill\noindent\small 2023/11/21 23:02 提问;2023/11/22 01:07 回答}

\noindent[1.]{\Hei 答}:一)在科技炒作這件事上,我不想對股市監管單位過度苛求,畢竟科研管理單位已經批准立項的東西,你怎麽能要求金融專業人員去評審否決?所以正文中只建議緊縮立項的評審過程;金融方的責任,在於必須禁止立項論證以外的吹噓被用來炒作股票。

二)揭穿真相是唯一的解決辦法,但如何上達天聼是最大的難處。大家幫忙傳播事實吧。
\\

\textit{\hfill\noindent\small 2023/11/22 19:21 提问;2023/11/22 23:43 回答}

\noindent[2.]{\Hei 答}:最近那個一萬億的開銷,固然可以解讀為延續以往的放水浪費,卻也可能是清理地方債務的正面措施。這是因爲絕大部分的地方債是隱性的,由名義上獨立的法律實體所背負,解決的第一步就是做出統計,而要讓地方政府敢於承認,必須拿出錢來支持。

問題的根本在於中央和地方的財政資源分配不合理,地方的稅收不足以支持開支,不得不另闢蹊徑、靠房地產搞錢。所以改革起來,傷筋動骨、工程浩大。這篇正文的用意,不在於對這個宏大的改革劃出藍圖,純粹只是從國際金融經貿環境的角度,指出時不我予、不能再拖了。
\\

\textit{\hfill\noindent\small 2023/12/01 09:11 提问;2023/12/01 23:29 回答}

\noindent[3.]{\Hei 答}:我的意見,潘行長應該是沒看到的。好在他的上面似乎有人在乎,所以我們看就要看黨中央的核心政策改動。

剛剛有人私下解讀了最新的中央金融會議報告;既然你在問這方面的事,就把它放在正文後注欄讓大家參考好了。
\\

\textit{\hfill\noindent\small 2023/12/04 22:58 提问;2023/12/05 03:06 回答}

\noindent[4.]{\Hei 答}:
我以前早就説過,體制外所做的論證止於議題輕重的判斷和政策方向的選擇,執行上如何剋服既得利益者的阻力,本來就是那幾萬高官存在的意義和任務,無需外人置喙。
現任最高層可不是省油的燈。你説的固然都是真實的問題,但他對這類鬥爭經驗極爲豐富、意志極爲堅決,一旦理解癥結所在,不徹底解決絕不放鬆(前天才創下親自視察人民銀行的新例);這也是我們支持並看好他的原因,不是嗎?
\\

\textit{\hfill\noindent\small 2023/12/06 09:52 提问;2023/12/07 04:52 回答}

\noindent[5.]{\Hei 答}:
是的,我覺得博客多年來在金融方面的建言已經傳達到最高層,未來的政策方向已經徹底修正。這裏並不是說我的文章一言定調,而是其中的論證為原本體制内就存在的、願意接受正確認知的幕僚和學者,提供了完整嚴謹的邏輯架構以及詳盡切實的證據,幫助他們做出更明確有力的論述,從而在決策爭議上占據上風;原始文章上的署名可能被忽視,所以新的建言依舊必須先得到這些人的支持。
學術和科技管理的問題卻很難解決,這是因爲部分學術人在金融議題上還願意當我的意見盟友,一旦提到縮編大學和嚴厲打假,卻全都立刻跳出來極力反對並過濾,連我志願獨自領銜得罪人都不可得;這其實是精英階級形成利益集團之後的典型後果,如同魏晉時代曹操曾努力削弱世族、明朝皇帝則千方百計要壓抑士人階級,基本只能知其不可爲而爲之,因爲不作爲的後果明顯是國家整體的腐化和衰亂。
\\

\textit{\hfill\noindent\small 2023/12/12 14:38 提问;2023/12/12 23:33 回答}

\noindent[6.]{\Hei 答}:
正文裏有關學術和科技管理的建議,是我反復思考多年的結晶,不但確定它們都是可行的,而且是絕對必要的,事實上也已經只剩最低程度的要求了。可惜這部分一進體制内依舊被刪得最厲害;當權者如果認爲内參已經足夠作爲開通言路的管道,那就太低估學術利益集團的能量了。
\\

\textit{\hfill\noindent\small 2023/12/12 22:57 提问;2023/12/12 23:34 回答}

\noindent[7.]{\Hei 答}:我對金融改革是樂觀的,對學術改革卻是悲觀的,原因見前一個回復。
\\

\textit{\hfill\noindent\small 2023/12/16 13:20 提问;2023/12/18 02:45 回答}

\noindent[8.]{\Hei 答}:
台灣在國際經貿上的定位,其實和韓國、越南相比,並沒有本質上的不同;同樣都是被美國金融殖民體系所放牧的中型經濟體。之所以今年經濟態勢明顯優於後兩者,原因有二:一是臺積電,來自美國緊縮半導體技術管控的收益已到,而吃虧卻還在未來;其次就是中方不但不針對科技臺階上最近一環的台灣出手,反而專注打擊韓國,而且還在貿易上繼續全方位大幅讓利。
Powell屈服於資本市場重啓泡沫的要求,在貨幣政策的平衡鋼絲上選擇繼續既有的潮汐式金融榨取,不正是今年幾篇博文裏已經反復談過的和稀泥態度嗎?
\\

\textit{\hfill\noindent\small 2023/12/23 00:46 提问;2023/12/24 05:41 回答}

\noindent[9.]{\Hei 答}:
行不通;痛苦指數不是開玩笑的。
在現代自由市場背景下,越是巨富欠債越多;用通脹來稀釋負債,更加鼓勵他們鉆利率差漏洞,形成超巨額的資產轉移管道,財富從中產階級向巨富(尤其是金融財團)快速集中,不可能由政府手段來彌補。這裏的典型案例是日本:薪水階級慘不堪言,股市卻欣欣向榮,美國金主樂不可支。
\\


 \end{document}