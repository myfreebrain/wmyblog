
\documentclass[twocolumn]{ctexart}
\usepackage{xeCJK}
\usepackage{titlesec}
\usepackage[colorlinks=true, urlcolor=blue, linkcolor=red]{hyperref}
\usepackage{geometry}
\setCJKmainfont{SimSun}
\titleformat*{\section}{\Large\bfseries }
\titleformat*{\subsection}{\centering}
\geometry{a4paper, scale=0.85}
\begin{document}

\title{问答1000}
\author{王孟源}
\date{2022-03-11}

\twocolumn[
\begin{@twocolumnfalse}
\maketitle % need full-width title
\end{@twocolumnfalse}
]
\section*{【戰略】【國際】從SWIFT制裁俄國,看中國的對應之道}
\subsection*{2023-04-10 09:50}

就算貨幣談崩了,還是會有其他協議和組織擴張的重要宣佈,大家願意和Putin握手言歡,更有外交上的意義,所以他有很好的理由出席。
至於Macron,霸權集團對他的掌控,可能不是很穩,否則沒有必要讓Von der Leyen緊急厚著臉皮硬是跟著;中方似乎也心裏有數,特別安排了廣州之行,以便習本人和Macron再好好談一次(此行的蹊蹺不在於Macron的行程,而在於習近平也去了),談的内容重點則應該正是博客讀者熟知多年的美國各番對外剝削掠奪伎倆,以及去年以來第三世界團結起來的逐步反抗。不過這類努力,中俄兩方都已經嘗試過好幾次,結果始終是Macron道理聼進去了,卻沒有能力膽量違逆國際白左霸權集團,所以我依舊不敢樂觀。
\subsection*{2023-03-31 15:07}

你說的會計區分叫做Available for Sale(AFS)和Held to Maturity(HTM):前者是應對提款的準備金,必須Mark to market(亦即依實際市場價格起伏計算其價值),後者是長期投資,在賬目上依據購入價計算。SVB的資產有大約10 \% 是AFS,主要放在國債;這部分的虧損早就公開,沒人在乎。上月出問題,是因爲存戶在2022年不斷提款,資產總額下降太多(約17 \% ),被迫將原本HTM的Mortgage Backed Security(MBS)也認賠變賣,以致Equity Level(净資本)跌破法規下限,在周一被報導之後,立刻引發大規模擠兌,無可挽回。人民銀行雖然不怕存戶擠兌,但美國的敵對態度不斷升級,若是臺海形勢有需要、或者層峰下決心出手反擊,這些美元資產必然要在一夜之間換成現金,届時就連名義上的會計賬目都無法遮掩。
不過中國政府的金融專業管理階層已經腐爛至極,“無能”和“賣國”仍舊是Understatement;昨天不是剛剛又傳出劉連舸被捕的消息嗎?
\subsection*{2023-03-24 14:57}

博客是華語世界獨一無二的存在,任何用心回溯過舊博文的讀者,都知道我在大議題上,始終遠遠早於其他人做出正確論斷。這個現象反復發生了千百次,即便我出於環境必要和天生性格,向來選擇網紅的相反,亦即自抑低調,確實已知的成就和影響也不一定明言,但新讀者看到名氣大的作者開始複製博客結論,興高采烈地以爲這可以拿來對照,依舊讓我啞然失笑:他們顯然違反了《讀者須知》第一條規則,但禁言是不得已才用的嚴厲紀律工具,我很不想依據間接邏輯做定論。
這裏還有一些客觀的考慮:首先,我既沒有、也不想追求官方或學術上的身份地位,所以雖然借鏡者不列舉意見來源有恃無恐(相對的,他們若是要引用林毅夫的意見,大概率不會不提來源;博客並不認可這種因人而異的待遇,參考兩天前Kim DotCom的Twitter鏈接被一視同仁地標明列舉),我權衡厲害得失之後,卻也必須忍辱負重、以傳播正道爲先,即便是非常特定的細節被抄襲借用,因而可以100 \% 確定博客為直接來源,一般也忍氣吞聲,畢竟肯支持博客論述的,已經是學術界中出類拔萃的人物,比起遍及社科界的公知要強上萬倍。另一方面更嚴重而普遍的是,中國網絡論壇剽竊成風,尤其博客位居海外,專注在陽春白雪,不求競爭流量,更加方便非學術的網紅將創見據爲己有。這裏的惡劣影響,除了直接攔截潛在讀者之外,也間接地將博客分析的原創性抹殺了,後續的引用者可以簡單假稱是受公共輿論啓發後的獨立發明。
我並不是在怨天尤人。前面已經説過,放任這些現象持續發生,是我權衡過自身處境、客觀條件、人生目標和實際可行性之後的有意選擇;管理微博賬戶的“世界對白”,早年就曾經埋怨過無數次我不積極“維權”的決定(這裏有一個額外的考慮,簡單解釋一下:“維權”是法律議題,不但可以用重寫文章來簡單規避,而且舉證責任在於原告;“剽竊”則是道德議題,適用範圍遠遠更廣,而且“原告”證明時間上的先後順序之後,自證無辜是被告的責任義務;尤其如果犯行反復發生,基本無可辯駁)。不過博客一直有教育性的一面,也就是實質上的網絡書院,所以希望至少書院成員理解相關真相是合理的。至於對真相和洞見純粹只是看熱鬧或想要藉機剽竊的人,請得了便宜之後不要賣乖,畢竟我從來不要求閑人捧場,只有反過來驅趕不入流者的習慣。
\subsection*{2023-03-23 16:41}

如果有讀者想從頭復習博客内容,我想點明貫穿多年的一條論證脈絡,亦即中方在21世紀面對美國霸權帝國的凶殘打擊,應該如何應對,是博客自一開始就想要解釋的重要議題,但因爲正道與當時的主流意見完全相反,我不得不從委婉地留下伏筆做起,先介紹如美元國際地位、美國戰略歷史、IMF和SWIFT的殖民主義作用等等基本知識,一直到Trump上臺,美方終於撕下假面具,直接發動貿易戰,我才敢把正確的戰略回應明説出來。然而即便長年保持如此的耐心,在當時還是連老讀者都有許多無法接受“不能挨打不還手”的簡單邏輯,更別提“美元是美國霸權的軟肋”這個結論。還好過去五年是華語世界在國際政略方面民智爆發的轉折,以博客讀者群(含爲了剽竊而來潛水的網紅;讀者私下向我反映,看到照抄或重寫的文章,不計其數:列舉來源出處真有這麽難嗎?公共論壇由一群竊賊主導,真的是健康的現象嗎?)為核心,正確認知迅速地散佈出去;很不幸的,中國學術界卻依舊掉隊。現在翟教授這樣的頂尖人物終於完全接受並公開宣揚真理,希望學術圈能儘快做出正向淘汰,不再拖國家的後腿。
另一個拖國家後腿的,是受昂撒教育洗腦的官僚。博客自2019年開始就反復解釋了美國經濟危機的必然演化路綫,而且連時間點(三年)都明確指出來。人民銀行如果還有一絲理性,就應該知道不能繼續買美國長期國債,外匯投資於債券必須只限於2022年之前到期;不這麽做的結果,就是可能高達20 \% 的虧損。我在本月稍早的《龍行天下》節目做了這個定量計算,不過引申應用到人民銀行的外匯管理,只説了一句“一言難盡”,觀衆可能無法全部領會,在此簡單澄清。
既然已經回顧《龍行天下》中一個匆忙提過、沒有解釋清楚的細節,順便在此也對另一個論點做補充説明;稍有離題,請勿見怪。SVB固然是美國金融貨幣政策左右爲難下的病徵,但因爲大銀行財務健康,收拾善後並不危及國本;相對的,歐洲和日本的經濟、財政和金融都要更虛弱得多,美聯儲可以接受的5 \% 利率,歐日明顯承受不起,卻又不得不因應跟隨,前景遠遠更爲陰暗。瑞士政府在整合瑞銀和瑞信的過程中,强迫變更契約,將若干Contingent Convertibles的條文徑行作廢,侵害了歐洲債權人的利益以及歐盟國債市場的信心,讓ECB焦頭爛額,是這個深層問題的最新表徵。關心國際態勢發展的讀者,應該特別注意未來一年有關ECB和日本中央銀行的新聞。
\subsection*{2023-03-14 20:36}

1)會計既不在乎實質真相,也不在乎邏輯自洽,它只是固有規則和慣例的反復運用。然而提升一個meta level來看,這些規則和(尤其是不成文的)慣例是怎麽定的?當然是多元的資本權力山頭彼此妥協的結果,所以會計事務所和信用評價公司事先沒有預期SVB出事,根本就沒什麽好奇怪的。
2)昂撒的信評當然優待自家的主權信用,參見上文。
3)我自2019年不知説了多少次,美聯儲的貨幣政策是在走鋼索,寬鬆則通貨膨脹、緊縮則泡沫爆破;我們現在看到的這些地區性銀行出問題,只不過反映了這條鋼索越來越窄的事實,然而摔下去的時間點還是難以事先確定,我依舊看好八、九月的金磚貨幣發行+國債上限上調這個double whammy(雙響炮?)。至於以調高外銷價格來進一步推動通脹,時機已過,歐美的貨品消費從火熱轉爲蕭條,國際消費品貿易進入買家市場,硬要去搞會得不償失。
\subsection*{2023-02-20 07:32}

你能看懂我的用心,很好。美國國債本身無關緊要,是博客已經評論過幾百次的議題,所以我假設你問的是高階修正項,順便示範如何將多方面的因果分析,針對一個特定問題整合成爲精確的估算,所以必須根據效應大小和作用遠近(亦即從直接到間接)來做二重展開。我以往反復批評過的類比聯想,即使僥幸撞上真正邏輯相關的議題,也必然會是盲人摸象、斷章取義;連既有事實都不全面考慮,怎麽可能對未來做準確的預測呢?沒有準確的預測,則談何政策選擇最優化?
熟悉高能物理的讀者,應該知道行業的基本理論工具是量子場論,而場論的計算太複雜,根本不可能有確解,所有的高能物理理論計算,都是序列展開的近似,這叫做Perturbation Theory,所以把這個思維方式應用在社科議題上,對我來説當然是件駕輕就熟的事。
\subsection*{2023-02-19 05:41}

我自《美元的金融霸權》以來反復解釋過,只要美元依舊是國際儲備貨幣,美國在財政和貨幣管理上就可以爲所欲爲。這句話翻過來說,就是赤字和QE的負面作用,只在於威脅美元的國際份額;然而因爲全球經濟體系的慣性以及美國在國際治理(含政治、軍事、經貿、文化、學術、宣傳等等)上的統治性地位,這個負面作用效應十分間接而且微弱,這也爲什麽多年來我一直强調中方必須積極主動地針對美元這個關鍵做反擊的原因。
當前的情況也不例外:聯邦政府的債務危機,在零級近似層級(亦即依效應大小從近而遠做展開之後的領頭項,Leading Term)純屬政治作秀,沒有經濟金融上的實際影響。在下一級近似上,對美元名譽影響是負值,但很小,基本可以忽略不計。再下一級,對當前的通脹壓力,也沒有直接的效應,必須考慮對美聯儲QT政策的間接影響,才有值得討論之處。
去年年初美聯儲從QE反轉為QT的時候,總資產(亦即印鈔的總量)大約為 \$ 8.8T。一年下來縮減約5 \% ,到了 \$ 8.3T。然而從美聯儲各種貨幣窗口的交易量來看,顯然銀行業並沒有因爲流失了 \$ 500B的現金而感到資金短缺;這正是幾個月前我談過的,美國經濟的各個層面,含家庭、企業到金融機構,都因爲2019年以來的急劇QE(總額 \$ 5T)而纍積了大量剩餘資金,需要時間來逐漸消耗。上次我說家庭和企業大概要到2023年年底或2024年才會用完這些剩餘儲蓄,金融方面其實也差不多。
講了這麽多細節(沒辦法,你問的是高階修正而且是間接效應,要討論就必須先談領先和直接的展開項),聯邦債務爭議的真正作用,只在於先暫時(六個月?)停止發行國債,緩和現金被美聯儲QT抽走的影響,然後在國會作秀完畢、將國債上限再次上調之後(今年夏天?秋天?),會有 \$ 1T這個數量級的聯邦債券一次性洗劫金融體系,從而對美聯儲QT產生競爭性的加成作用,其後美國金融系統才有可能會重新面臨2019年的現金短缺危險,進一步迫使QT提早結束,參見我在《從回購利率暴漲談美國經濟周期》一文的討論。
\subsection*{2023-02-02 12:57}

法國人素來喜歡上街(這點英國人是相反:他們的護士比法國人受虐嚴重多了,到最近才開始罷工,居然還有著普遍的猶豫自責心態,可見昂撒對内部民衆洗腦之成功)或甚至革命,但當前的反對潮流顯然還沒有到1968年5月的程度,歐美Deep State又已合流,對權力的掌控更加全面。既然55年前都沒有導致政權更迭,現在談Macron下臺,當然是言之過早。其實就算幾個月之後,情勢持續激化,Macron被迫辭職,也只是換一個資本的傀儡罷了;更嚴重一級的第六共和,機率相當低,而且依舊不能保證帶來真正的改革。總體來看,歐洲未來的命運,更可能是日本式的失落N年,而不是洗心革面的重新出發;我想等待的轉折點也不只是政權更迭(已經在幾個歐洲的次要國家發生),而是政體崩潰,這裏英國的危險最大。
\subsection*{2023-01-31 01:33}

你既然有興趣討論,我就談得再深一點。其實經濟景氣程度,當然也受外來(Exogenous)意外變動所左右,而且這些大大小小的外來因素非常多,反而形成研究經濟周期動力機制的真正困難所在。這是因爲它們無所不在,又有著確實的影響,所以每次經濟衰退,必然都有一堆負面的因素,可以簡單拿來解釋,特別方便低級學人偷懶、批量發表無意義的類比聯想論文。但從長遠的角度來看,這類外加因素顯然應該遵循隨機的Poisson Process,如果假設它們完全左右經濟榮枯,那麽經濟成長就會是發散性的Random Walk,顯然不符事實。所以至少必須假設還有内發性的回歸平均機制(換句話説,實際上這個内發動力機制應該兼有失穩和回歸性,而當前昂撒經濟學界選擇性地只看片面),兩者叠加之後,才可能有目測上的“波動”現象;但這樣的僞周期機制,其規律性和可預測性都近乎零,依然不符合實際的現代經濟歷史記錄。事實上放任政策導致經濟泡沫,是有金融常識的人可以在爆破前幾年就簡單斷言的事,博客這裏已經反復示範過。所以現代經濟周期的實際主導機制,在於内發的不穩定性,外來的隨機變動,一般應該被視爲加成的導火綫,而且隨著人類科技能力和組織程度的提高,内發因素越加蓋過外來影響。例如古代天災的確對國運有立即和決定性的打擊,但同等程度的災害在21世紀,越來越像是日常政務處理;即使出現新冠這樣百年一見的Pandemic,固然導致嚴重的經濟下行,但其作用明顯獨立於真正的“經濟周期”;後者的内發過熱問題依舊需要處理,中方一樣花了兩年多來壓制房地產炒作,而把外加和内發因素搞混的歐美國家,不就必須面對50年來最大的通脹危機嗎?
地震和渦流,同樣都是内發的周期性現象,然而其確實的動力機制是完全不同的(至少當前的理論,遠遠達不到有統一性的地步)。這沒有什麽太大的問題,因爲經濟周期本身,也有其獨特的内發不穩定性機制;這裏的確只有定性的共通,不過在定量上,我覺得經濟周期和渦流更接近些。
\subsection*{2023-01-29 10:50}

因爲現在美國的外交大戰略由NeoCon主導,他們的特質之一就是沒有智慧、不尊重專業、不在乎現實,只看到打垮瓦解俄國的獲利,完全不想想己身的軟肋。第三世界既然連俄烏衝突都不給美國面子,替換美元的事是西方沒收俄國外匯資產倒逼出來的,除了内部賣國賊作祟,哪有理由不全力推行呢?
至於人口結構問題,就算沒有能力看出解決方案,至少也應該有點自知之明,瞭解這種長期慢性阻力,別人會有剋服的辦法,最最起碼不是可以先來博客確認一下嗎?凡是蹲在家裏誇誇而談、拍腦袋說中國必敗的,都明顯是在遵循“我沒聽過,所以不存在”的邏輯,也就是Dunning-Kruger曲綫裏笨蛋峰的典型居民。我説過很多次,請大家不要引用不入流的他人言論,參見《讀者須知》第六條。
\subsection*{2023-01-19 13:00}

經濟活動的總量其實有多種算法,其中兩種比較常用,你説的是入門經濟學拿來教GDP的那一種。但不論怎麽算,差別都很小,這是因爲它們必然都基於呈報和問卷資料,最終依舊是只能反映經濟交易浮面數額的會計算法,而GDP在作爲治理參考上真正的主要缺失有二,都屬於不可能用會計手段來解決的偏差(bias,相對於誤差error而言):1)忽略隱性社會成本和收益;2)忽略服務和產品的品質。後者在市場競爭下依然經常有大幅偏差,是資本為追求壟斷暴利,日常進行公關炒作,創造信息不對稱的結果,長期下來更可以做出制度上的扭曲,例如美國的健保醫療業和律師訴訟業是典型案例,全球浪費在量子計算、核聚變和氫能上的錢,也是有意詐騙的後果。
\subsection*{2023-01-18 22:12}

其實2021和2022年,美國面臨全面供給鏈問題,成本以倍數成長,匯率漲跌個2、30 \% 根本不影響他們下單進口。結果人民幣匯率依舊遵照2019年之前的舊有公式運作,隨日幣和歐元起伏,基本是自願承受同比例的割肉飼鷹。最近美國通脹緩解,容許附庸國貨幣部分回升,人民幣也才有所上漲。我的質疑在於,如果金融單位的最高主管,完全無視經濟、金融、貿易和戰略現實,盲目遵循大幅傷害國家利益的簡單綫性公式,這是隨便一個小電子表格都可以做到的,那麽國家養他們來幹什麽?
美國金融界的資金汎濫極爲嚴重,美聯儲的升息只做了避免通脹持續惡化的最低程度,一年來量化緊縮所回收的美元佔總量的九牛一毛,所以資產市場定價仍然偏高。然而全面崩盤的危險已經渡過,危機轉爲慢性,美聯儲得以繼續走鋼絲;要讓他摔下來,必須等下一個黑天鵝事件。金磚貨幣當然會是一個非常嚴重的打擊,但目前我還不能確定其時間點和緊迫性;換句話説,其對美國經濟金融的負面效應必然是很大的,但如果平攤為多年的逐步影響,那麽就可能不足以獨自引發非綫性的連鎖反應,就像承力結構對static load的彈力極限明顯高於dynamic load;這也是爲什麽四年前我就提早警告要好好把握這輪通脹危機,一年前則特別着急的原因,現在已經追悔莫及了。
\subsection*{2023-01-11 13:43}

有關Powell的貨幣政策,其實是在複製1971年Nixon打破Bretton Woods開始拼命印鈔到1973年第一次能源危機,通脹壓力持續纍積後Arthur Burns的和稀泥方案,亦即只做出足夠避免通脹率持續惡化的最低努力,所以在70年代中後期,美國年通脹率穩定在5-7 \% 的範圍内,歐洲和日本雖然開始替換美元,短期内倒還不是特別難受。問題在於Burns的運氣不好,他賭的是他所能控制的需求面被調整為中性之後,供給面的天然波動能自行解決難題,結果拖了幾年沒有好轉,工會的薪資談判反而將通脹固化了;他1978年下臺的時候名聲還不太臭,但1979年第二次能源危機徹底打破均衡,證明他賭輸了(有點像Alan Greenspan2006年退休時還是“Maestro”,兩年後出了世紀級的金融危機,名聲一下爛了大街),Carter只好緊急再換上鷹派的Volcker取代Miller,接下來是家喻戶曉的20 \% 短期利率。
Powell有前例可以借鏡,應該自認是條件更好的:不但這次能源危機的規模較小,而且美國的工會早已名存實亡,更重要的是歐盟、英國、日本不但沒有抛棄美元,反而捨身喂鷹,連中國人民銀行都在最關鍵時段全力配合美聯儲壓制美國國内第一波的通脹壓力。當然也有負面因素,亦即美國的產業虛擬化、金融化,而且過去20年的印鈔幅度比1970年代初期短短幾年要大得多;不過這些都是慢性問題,説不定能自然解決(例如中方金融主管繼續為美國犧牲自己國家的利益,開放金融搜刮),所以我完全理解他的邏輯。
\subsection*{2022-12-20 12:45}

美元的國際地位,對美國的實體產業有著絕對的腐蝕作用。對外的金融掠奪,直接圖利華爾街,再怎麽分發福利,也會加劇貧富不均(因爲政府能直接掌握的獲利,必然只占少數)。經濟比政治,還有更多反直覺的現象,但只要小心,並不難看出要點,例如這裏MMT的基本假設是印鈔就可以無中生有,憑空創造出新財富;事實上只要金融體系壯大起來,它自然會繼續繁衍、追求進一步的利潤,而最簡單、而且基本無法防治(因爲金融產品先天就是100 \% 人造的,可以隨意修改、複雜化)的利潤,就來自人爲地製造信息不對稱;信息不對稱必然導致資金配置的低效,從而引發比賬面利潤高出幾個數量級的隱形社會(含國境外的人類社會)損失,只不過一般金融系或經濟系的教授們看不出來罷了。
這還沒有談到對世界做金融剝削,完全與中國的國際大政略以及傳統文化背道而馳。
\subsection*{2022-12-19 14:08}

印度在未來10-15年,可以繼續依靠基建和消費性產業來維持相對高的經濟成長,但是一旦前進到先進工業的門檻,它的弱點就會暴露無遺。這裏我指的是教育的普及性;所有在20世紀之後(亦即沒有依靠殖民利益和先發優勢)才邁進先進工業國行列的地區,都在東亞儒家文化圈内,都是在經濟起飛之前,就已經極度重視教育。印度不但能受像樣教育的人口比率極低,產出的精英即使是理工科,也整天想著拿MBA,在人力素質這個極度重要的因素上,他們還不如越南;這基本保證印度無法產生像TSMC或華爲這樣的企業。此外,在昂撒主導全球、所以印度可以靠英文優勢為他們打工的階段,吸引外資不是問題;但當前的Golden Billion和第三世界開始撕裂,十幾年後塵埃落定,印度就無法左右逢源了。
我批評學術腐敗,其實專注在理工科研。它的確是目前可以看出的最嚴重長期問題,不過當然不是唯一的長期問題,例如習近平退休之後的交接,如何保證困難的改革和反腐能持續下去,幾年前我也討論過。
\subsection*{2022-12-18 19:01}

昂撒是近代以來最暴力的民族,然而其宣傳洗腦是把其他國家的合法合理暴力完全抹黑,對自己的非法而且極度殘忍的暴力(包括人類有史以來最大最慘的種族滅絕)卻反過來美化成爲挽救世界的努力,偏偏愚蠢的民衆就吃這一套顛倒黑白的胡扯。所以博客多年來反復指出暴力不但不必然是壞事(以私害公的暴力才是邪惡的,例如黑道和昂撒;但問題的基本在於“以私害公”,而不在暴力,後者只是放大前者的嚴重性罷了),而且是公權力的必要基礎。這裏既然你引發這個話題,我就説得再清楚一點:孫中山所說的“政治是管理衆人之事”,是幼稚園程度的認知:家庭、學校、公司、工會、社區、教會等等,哪個不是“管理衆人之事”?政治是政府的運作,而之所以有必要成立政府,純粹是爲了執行必須强制的公共事務(收稅、監管、司法、治安、戰爭等等),而要能有效地“强制”就必須先獨占暴力;所以正確的定義,是“政治是獨占並運作暴力以保障公衆利益的努力”,這一點毛澤東有著徹底的瞭解,只不過他不懂經濟,所以對“公衆利益”的認知有偏差罷了。
\subsection*{2022-09-30 10:44}

擴軍這事,不是三天兩月就能見到成效的,現在才開始已經來不及了。所幸習近平一上任就著手,軍改和山東艦都是積極備戰的案例,真正沒有及時擴充的,只有核武一項;不過這個錯誤也已經在一兩年前就被理解並改正了,我們無須再談。
至於俄烏戰爭的核子升級,Putin或許有點迂腐(過去八個月有不少非最優解的選擇,我覺得最好的解釋是Putin學法出身,所以過於拘泥於程序),卻絕對不是傻子,不會輕易上當。烏克蘭引發核戰,我現在只能想到兩個可能:1)Zelensky用髒彈攻擊俄方(有報導宣稱他自己公開承認有此計劃);2)美國用我談過的栽贓手段先把Kiev炸了。在這些非常嚴重的前提下,中國並沒有什麽事先“做好準備”的餘地,只能事後譴責美國,規勸俄方不要衝動,並自求多福。如果北溪被炸還不足以喚醒Scholz(很可能如此),那麽中國外交官説什麽都不會有用。
\subsection*{2022-09-29 14:21}

我對昂撒的評語,向來都是“心狠手辣、下流無恥”;那可不是説著玩的。你只要看看James Bond系列電影,能把全世界頭號恐怖組織MI6描述成英雄、神人,就知道必須把他們的宣傳反轉180°才能找到真相。
歐洲冬天冷,不方便搞示威抗議;縱觀歷史,革命一般不發生在嚴冬。例如1848年民族之春,雖然始於當年1月,但一開始局限在不下雪的Sicily島,到4月才在歐陸普遍爆發。又如1789年法國大革命,從5月起始,7月到達高峰,Bastille Day就在7月14日。當然這不是一個真正的鐵律,可以拿1917年的2月革命做爲反例;然而其實當時也先只有零星的罷工,大幅示威抗議到3月才發生,只不過在俄國特有的東正教曆法上還算“February”。
\subsection*{2022-09-25 09:37}

正如Putin所發現的,Scholz和Macron不管當面說什麽好話,一轉頭又是乖乖做昂撒集團的小狗。所以和他們的外交,沒有談戰略議題的意義,不過講講經貿還是可以的;尤其現在正是新興的中方電動車企業要出國攻城略地的關頭,確保這個進程的順利發展還是有相當價值。而Scholz和Macron在國内經濟受到嚴重打擊的前提下,在貿易政策上也必然有求於中方,做些交換固然是互利,中方的收穫應該可以更大。
俄國固然打贏了第一波金融戰,但這裏的關鍵在於能源價格的上漲。一旦第三世界和歐洲的經濟衰退明顯化,油氣價格有隨之崩潰的可能。所以我一直覺得Putin不速戰速決是個戰略錯誤;夜長夢多,就算他不怕北約對烏克蘭的加碼支持,經濟形勢的危險也是難以控制的。如果我來決策,既然已經動員,就會儘快解決軍事議題,以便專心應對即將來臨的這一波全球經濟問題。
美歐不可能同意Zelensky認輸,只有在軍事上兵臨城下,迫使烏方出現政權更迭,才可能迅速達成妥協。同樣的,歐洲現在的這群領導人也不可能承認錯誤,只能靠替換來變革;然而這在宣傳媒體一面倒的背景下,並不容易,我覺得有可能拖過冬天。
中方的策略,應該是盡可能和歐洲維持表面上的客氣,繼續深化經貿上的交流,一方面搶占新能源市場,另一方面吸引歐洲的先進工業轉移來華。至於政治和外交層面的問題,做旁觀者就行了。
\subsection*{2022-09-15 01:50}

那要美國願意同甘共苦才行得通;現在美國對歐洲可是吃乾抹净,對日韓台也會是一樣的。歐洲眼看著要成爲21世紀的拉美,這樣的集團能有什麽力量?
昨天我談外交部該如何回應美方的《台灣政策法》時,可能沒有説清楚:這裏的目標聽衆,根本就不是美國,而是日、韓。中國應該把握每個機會,指出歐洲追隨美國對俄制裁的惡劣後果,重點强調美國人躲在後方,趁機榨取盟友脂膏的事實。那篇Rand的機密報告,尤其是值得大作文章的把柄;日韓的媒體不想討論,就該由中國官方設法提醒,讓他們不得不報導。不過實際上除了博客的讀者群之外,根本不會有人注意;我把“博客的長期讀者”和“不蠢”劃上等號(請注意,我並沒有談因果關係的方向,所以由此就斷定我自大的人,其實是在招認自己沒有邏輯能力,甚至沒有欣賞邏輯的能力),固然是玩笑話,但反映的不止是事實,也是多年來的心酸和無奈。

有關那個“博客長期讀者”的玩笑話,既然已經點明了,我想就解釋清楚,以避免誤解。
我的意思是,會成爲長期讀者的人,不一定有很强的邏輯能力,但必須事先就有欣賞邏輯的能力,然後在公共議題上只要點出脈絡和重點,自然會理解並同意正確結論。相對的,連邏輯擺在面前都看不見的人,不但不可能接受博客的論述,我反過來還急著把他們趕走,以免妨礙大家討論正事,那麽自然不會成爲“長期讀者”。
有很多人因爲機緣不巧,沒有及早發現博客這個理性的桃花源,這當然並不代表他們的天資不好。如果興趣不在公共事務,或者工作家庭事情忙,無法關注相關話題,自然也不會因爲不看博客而成爲蠢人。就連我昨天説的,落入美國人反復使用的套路陷阱是件蠢事,也和人蠢不蠢沒有關係;聰明人絕對也做蠢事,蠢人也可能誤打誤撞。所以這裏唯一的標準,在於是否能夠並願意尊重事實和邏輯(很不幸的,在現代社會中,這和學位高低似乎沒有太大的正關聯)。我向來批評是蠢人的,都不含嘗試做邏輯分析卻不幸得到錯誤結論的案例,只限於不知事實邏輯為何物,卻硬要在公共論壇高調品評公共事務的人,例如方方和龍應臺,以及網絡上的許多噴子。
\subsection*{2022-09-14 02:12}

美國的典型套路,是由立法機構先天外飛來一個異常離譜的點子(參考Pelosi訪臺事件),經由昂撒媒體炒作,在全球公共認知層面把非常極端的手段正常化,然後一面否認其代表“官方政策”,一面視對手和國際的反應來決定是否做出“退讓”。這個“台灣政策法”也不例外;其中有個很明顯用來忽悠中國的籌碼,是要把台灣正式列爲“非北約盟友”。試想,這若是通過立法,豈不是强制美方參與臺海戰爭?美國人最不願意的,就是在面對强敵的時候,自己先上第一綫。我以前已經强調過,縱觀歷史,自從1812年第二次獨立戰爭結束,200多年來美國對外用兵400多次,卻從來沒有一次是主動參與事先沒有壓倒性優勢的戰局。何況現在的中國海空軍,是100年來對美軍最嚴重的威脅;而NeoCon和權貴政要心裏想的是生意和利潤,不是自殺。所以中方的正確對策,是抓住美國人言不由衷的辮子,由官方高調“鼓勵”、譏嘲這個法案,尤其是那個盟友地位;若是一面公開抗議、一面私下交涉,試圖做出妥協,就正中其圈套了。不過這雖然很蠢,卻是外交部最可能做的選項,因爲他們不是博客的長期讀者。
\subsection*{2022-09-13 20:13}

博客八年來始終解釋得很清楚,讓“盟友”去當炮灰是一舉兩得的妙招:除了本身不付代價就能打擊對手之外,盟友衰弱或崩潰也是吸血吃肉的大好良機。然而美國人一直在公開訊息上小心謹慎,從來不談後者,那麽外界的觀察者也就無法完全排除“走狗烹”是“狡兔死”之後才考慮的附帶、備用、次要手段。
【後註二十八】所談的Rand報告,如果屬實,是第一次證明“盟友”其實是和對手同一級別的打擊對象。即便不屬實,經過這次俄烏衝突大獲“成功”,NeoCon和他們背後的權貴政要也絕對會總結經驗,學習到在中俄實力堅强、又有第三世界普遍支持的前提背景下,專注在打擊然後消化炮灰才是效費比最高的選項。所以即便原本理想中應該在圍剿中國的過程出大力的歐盟現在自顧不暇,美國在東亞挑釁反而很可能會變本加厲,其主要目的不再是要推翻、瓦解、或圍堵中國,而是要打爛日本、韓國和台灣,以接收這些地區的資本、技術和人才,在2024年來自歐洲的養分開始枯竭的時候,能有另一場新的盛宴。
\subsection*{2022-09-11 11:52}

一般估計俄軍在烏總兵力最高為19萬,當前約15-17萬,其中約半數不是正規重裝部隊。烏克蘭在動員之後,號稱100萬,實際約50-60萬,其中大多數為只能守戰壕的民兵炮灰,但受過3-4個月基本訓練的“正規軍”可能依舊略多於俄方,有8-10萬人。當然不論是正規和正規比,或者民兵和民兵比,俄方都在單兵和幹部素質上占優,但是在至關重要的C4ISR上,俄軍只在戰術層級有優勢,在戰略層級面對的是整個北約的體系,所以才會吃情報失準的虧。
昨天我已經討論過俄軍在戰術方面的失誤,但戰爭出現失誤是難免的,國家領導人的責任之一是提供足夠的冗餘,以減低失誤的頻率和危害。然而俄方的這次特別軍事行動,戰綫長達1500公里,還要在主攻方向集中力量,又要在占領區維護治安、避免襲擾破壞(例如核電站、機場、彈藥庫等等),十幾萬部隊怎麽用都不夠。這筆賬只能算到Putin頭上,因爲即使考慮到要維持戰略警戒,防範北約突襲,俄國再多投10萬部隊進入烏克蘭的能力還是有的。至於你談1917年革命,其實説反了,因爲一戰受沙俄百姓反對是因爲傷亡過多、戰事膠著不利;現在多派10萬人反而能減低傷亡,而且防止戰事反復,恰恰是避免民意反轉的正道。

至於Putin爲什麽會犯兵力投入不足的錯誤,《Moon of Alabama》剛發表了一篇新評論,認爲他可能有意讓戰情出現反復,包括犧牲Izyum的居民接受納粹暴行,以便激起國内民眾對升級投入的共識。我覺得這純屬臆測,沒有任何證據支持;依循博客既有的科學方法和原則,在沒有足夠確實資訊的情況下,我們應該遵守Occam's Razor,繼續接受最簡單、直接、常見的解釋,亦即Putin是類似Merkel的優秀官僚,所以也具有優秀官僚的務實、保守、謹慎性格。現在Putin的確可能必須將戰事升級,但這是被動的反應,而不是主動設計出來的計謀。
\subsection*{2022-09-10 08:44}

自從四五年前我開始上視頻訪問之後,博客就面對一個新問題,亦即原本爲了追求精簡,非必要不重複細節,然而我在聊天過程中隨口談過的論證和結論,並沒有寫下來,是否要在博客再復述一次?此外,我的邏輯始終是環環相扣的,沒有寫下來的論證部分,要引用並不方便。
你討論的歐洲工業如何流失的議題,其實我在上個月唐湘龍的節目上已經解釋過了:德國和歐盟面臨的不是沒有天然氣和石油,而是它們的價格大幅上漲、甚至以數量級計的程度。換句話説,這不是兩次大戰期間的物資短缺,而是金融性的超級通脹,而且是史無前例只有歐洲獨自承受的超級通脹(請注意,三年前我在《八方論壇》節目中,已經詳細解釋過下一場經濟危機會是某霸權集團獨自承受的超級通脹;今年發生的新轉折,只在於歐盟志願為美國替死),所以受打擊最早、最嚴重的是高能耗工業、尤其是可以用進口替代的高能耗工業,也就是化工(BASF和SKW Piesteritz)和那篇報導所談的冶金。
大衆媒體熱議的冬天將至,其實影響的是民生用能源,所以嚴重性在於政治層面;不過正因如此,歐洲的這群愚蠢無良政客必然優先保障政治性需要,真正事關經濟命脈的工業用能源供應,早已絕望。歐洲的自我救贖,必須先徹底清除幾乎所有主流的政治、媒體和學術精英;就算他們能做到,也不是三天兩頭的事,何況綠黨的支持率還在上升之中,昂撒系傳媒已經開始為Habeck和Baerbock繼任總理宣傳造勢,甚至可能出現兩者相爭的局面,若不是因爲要有幾萬條人命陪葬,這可以算是人類政治歷史上的最大笑話之一!
\subsection*{2022-09-10 07:53}

烏軍在過去半年,靠著英國和波蘭重新訓練了一批正規軍,這次將5個旅投入Kherson戰綫踢了鐵板,3個旅打Izyum反而長驅直入30公里。
這有三個原因:首先Kherson以北是典型的Steppe地形,草原一望無際,人口密度又低,村落既少又小,在絕對懸殊的空優和炮兵劣勢下,靠裝甲兵和步兵來發動攻擊純屬自殺;相對的,Izyum在主要河流之畔,屬於丘陵地帶,又有大片的茂密森林,限制了空軍和炮兵火力的打擊,有利於多數兵力的發揮。
其次,俄國動用的正規軍真不多,半年多來一直捉襟見肘,必須靠民兵和武警(叫做“National Guard”,不過美國的National Guard有等同正規軍的編制和重武器,俄國的就只是純粹輕裝的警衛部隊,類似中國的武警)來填補;即便如此,和全民動員後的烏克蘭兵力對比,依舊是反過來的1:3,這是史無前例以少打多的攻勢持久戰。在Kherson戰綫正規軍比例較高,而且是素質過硬的空降部隊,所以戰力優於烏克蘭正規軍;而在Izyum西側的突入口,卻是俄軍守兵最弱之處,只有很少數的武警,他們的戰力比東烏民兵還弱得多,基本只夠維持秩序和警戒邊境,遇到重點打擊,掉頭逃跑是唯一選項。
第三,Kharkov和Izyum戰綫由俄軍西部戰區來負責,一直是表現最差的。3月底開始從Izyum以重兵向南切入,五個多月下來不但寸土未進,而且戰損比很難看,我聽説(但未證實)西部戰區司令已經因此被撤換過了。其實以俄方的空軍和炮兵優勢,只要有進攻方1/3的兵力防守,就沒有潰敗的道理。戰區司令再怎麽魯鈍、情報再怎麽失準(其實兩周前就有公開消息指出烏軍準備在東北方向另開第二戰綫),總該有至少一個旅的預備隊,剛好足夠擋住三個旅的進攻。我們再觀察幾天吧,如果烏軍能夠繼續深入,那麽才證明俄軍出了異常嚴重的問題。
從戰略觀點來看,現在Putin和他的高級將領都在遠東,戰事忽然出現反轉,難免給人過於托大的印象。不過俄國國内的輿論譁然,反而敦促他糾正兵力投入不足的錯誤。我從戰事一開始,就指出這絕對是錯的;即便Putin在戰略上準備好打持久戰,有多餘的兵力至少也有減低傷亡和風險的作用。
其次是俄國管理下區域的百姓,突然明白有可能被納粹重新占領並清洗,其士氣和態度必然要受影響。然而這也可能激勵他們積極組織民兵自衛,幫助緩解俄軍兵力不足的問題。
但是影響最大最深遠的,是重振北約的士氣,堅定了Von der leyen和Stoltenberg戰鬥至最後一個烏克蘭人和最後一立方米天然氣的決心。然而在Putin眼中這不見得是件壞事,對中國和第三世界來説則絕對是件好事。
\subsection*{2022-09-09 19:33}

Schroeder的改革除了他個人貢獻的人和(例如與俄國和解,建設北溪一號,從而大幅降低能源成本)之外,也受益於許多天時地利因素,包括中國的崛起和歐元區的建立(這原本是法國引領主導的)。真正讓人嘆息的,是Scholtz的愚蠢和無能:原本德國的天時地利依舊在極高水平,Merkel留下了北溪二號、英國脫歐自爆、Trump自絕於國際體系,很自然地為歐盟開拓了一條康莊大道,可以在軍事上置身事外、經濟上奪取英國的服務業、外交上占據國際規則制定權、金融上取代美元,完全足以使歐盟晉身新多極世界中的領頭地位,奠定未來30年富裕生活的基礎。歷史上一手好牌轉眼間打成褲子都輸光的例子,固然是有的,但都來自軍事失利,像他這樣上任不到半年就通過和平過程斷送全球一級强權的國運,別説事實經驗上史無前例,就是在純理論上也匪夷所思。
\subsection*{2022-08-14 06:38}

美國現在只要是共和黨籍的政法系統官員,全都是Federalist Society的會員了。他們代表的,的確是傳統建制派,和各大智庫、主流媒體、以及其他Deep State組織的聯係合作,都有脈絡可循,只是不會被廣汎報導,所以一般民衆難以注意罷了。
共和黨建制派對Deep State(狹義的Deep State指的是事務性常任官僚和他們的政務官盟友所組成的利益山頭,廣義的則是不論政黨如何輪替,實質掌握國家權力的資本和權貴集團;這裏我指的是前者)打擊Trump的企圖,至少是樂觀其成,這也是爲什麽現在聲援Trump的國會議員零零落落的原因。然而我在一年前所評論的,Trump只控制黨内初選,在大選中反而是毒藥,那是在Biden把建制派的名聲徹底搞砸之前的現象;現在拜與對手比爛所賜,Trump又有大批“堵爛票”(Protest Vote)可以依靠。而且Deep State越是拿國家機器來打擊他(還有Alex Jones、Roger Stone、Steve Bannon等人以及一月六日抗議者;都是媒體和法庭聯合起來鷄蛋裏挑骨頭,誇大構陷、雙標執法),選民的怒氣越是上升。如果總統大選是今年進行,那麽基本可以確定Trump會重新當選了。實際上還有兩年多,美國政壇各種狗皮倒竈的事還難以預料,但期中選舉的結局已經很明顯,會是共和黨大獲全勝。然而這裏真正獲利的,依舊是建制派,所以只不過再一次示範英美民選制的傳統真諦,亦即民衆的“選擇”毫無意義,兩邊同樣是資本的代言人。
Trump目前還是民粹反抗力量的核心,但那並不是因爲他有理想或領袖特質,純粹是覺醒民衆別無選擇、無可奈何的妥協。然而他執政四年,雖然和建制派衝突不斷,對竊國的大盜卻並沒有任何實質處理成果,右翼反建制派(左翼被白左洗腦太深,完全沒有作用,更不願意和右翼合作)中教育程度較高的那批人已經開始另外組織起來,這是我正在關注、並且希望寫入下一篇博文的題材。
\subsection*{2022-07-22 19:00}

視頻訪談,談不上“嚴謹”,例如Chrystia Freeland已經從外相升了副首相,而且二戰期間當烏克蘭納粹宣傳編輯的,不是她父親,而是外公。不過Freeland的確曾經公然撒謊,說她外公是二戰前就移民加拿大。此外,和阿富汗接壤的是蘇聯,而不是俄國。當然如果美軍留在阿富汗,還是必須擔心被俄國買人頭的。
金磚貨幣的前途如何,目前的不確定性太高,達不到我在《讀者須知》裏要求的70 \% 信心底綫,所以一直沒有在博客提。換句話說,20-50 \% 的估算,應該是Unbiased,但是Error很大。
有關歐盟是否會解體,我在節目中已經回答過了:應該會,但可能不是很快。以歐洲財富的底蘊,苟延殘喘一段時間並不太難。
替代IMF,不但在節目中已經明確回答,而且博客已經討論多年:這是建立新國際政治經貿架構的必需品,中方沒有提前佈局,把AIIB浪費掉,實屬極不明智的作爲,亟須加緊彌補。
道義性來自合理性,也就是價格不要太離譜。

囘了你的問題,才注意到你的賬戶太新;難怪問的都是我解答過的事。只好拉黑,罰你再等六個月。
\subsection*{2022-07-15 01:47}

小國擔心被吸收吃掉是正常反應,Lukashenko也曾努力對俄保持距離,但Tokayev最近閙得太過,有以下幾個可能的背景因素:1)擔心Putin偏袒Nazarbayev;2)Tokayev本身是白左教的信徒;3)一月政變並沒有英美的直接參與(亦即是我當時解釋的,由外逃大亨主導,在MI6能插手之前就結束了);4)因爲Kazakhstan的地理位置,扼住俄國向東的交通咽喉;5)BRICS形成正式的國際政經板塊,反而制衡俄國對中亞的控制,放大小國的話語權,正如歐盟内部那樣。
這是當前大局面之下,一個無意義的Distraction,Putin應該會盡力息事寧人;然而如果(1)或(2)的情節嚴重,Tokayev一意孤行,就會害人害己,但機率實在不大。
\subsection*{2022-07-09 16:11}

我想特別澄清:昂撒的“分贓”雖然聽來是貶詞,但和古希臘的“民主”相比,其實是遠為優越的:這是因爲前者專注在利益分配之上,所以先天就是理性的討論;而後者的問題出在一般群衆的非理性心態,因而損人害己的決定是日常。
讀者應該復習我對工運和學運所作的對比,以及有關“損人不利己”的一系列文章,尤其注意這些議題的一個共通脈絡,亦即沒有理性思維能力的普羅大衆,其對公衆事務的參與應該被局限於反映與自己切身利益相關的問題;第三者做評論,即使是作協主席或文化部長,如果不能尊重事實真相、遵循理性邏輯,就純屬情緒發泄、挑動愚民,不應該被廣爲傳播。
至於牽扯到隱性社會成本的議題,因爲並沒有明顯的直接受害人,那麽反而必須更加小心謹慎地權衡利害得失,切忌將決策過程下放給Activist、媒體或市場,否則必然會被有心人利用、扭曲,不但問題不能解決而越演越烈,而且連帶地腐化整個國家社會,拖累其他應作的改革(細節機制參見前文《談全球暖化》、《有關環保和全球暖化的幾點想法》和即將刊出的新博文《社會主義國家應該如何管理資本》)。
\subsection*{2022-07-09 10:26}

前天我已經解釋過了:NeoCon固然有瘋狂的核心和愚蠢的外圍,但它在英美體制内終究只是工具、而不是真正的統治階級,連心腹代表都算不上(所以才必須有Pompeo和Blinken這類“闢仗使”監軍)。國務院、MI6、以及Lithuania這種工具的工具,當然會想要無限升級,但統治階級的其他工具組織,例如國防部,更別提財閥自身,面對必敗的局面,還是有克制能力的。
從Pompeo演講所能得到的正確結論,其實正是我在兩個月前唐湘龍節目裏所説的,中國在外交戰略上運氣極佳,美國最凌厲的打擊手段,一而再、再而三地自行夭折,俄烏戰爭是最新案例。過去一年我反復解釋,昂撒集團挑起俄烏衝突,真實目的在於提前整合歐盟,然後自然可以簡單地支持臺獨,引發臺海戰爭;Pompeo只不過是高興得太早,説漏嘴罷了。

説的再清楚一點:三月初,俄軍因爲部署錯誤而承受相當損失的時候,金融戰、宣傳戰也才剛開始,美國權力階級全都自以爲勝券在握。作爲共和黨系的NeoCon,又準備要選最高公職,Pompeo怕被民主黨拿走全部“戰功”,所以必須搶先一步公開談NeoCon戰略的終極目標;其用意是出於黨爭,要能在選舉中邀功,而其所真正反映的,就是博客八年來所一直强調、美國兩黨一致反中以維護霸權的共識。
至於俄方勝利讓中國躺贏,其效應也早已明顯化:最近兩個月,Blinken多次主動要求外交接觸(當然只是暫時停火,而不是和談),前倨後恭的態度變化非常明顯。這個轉折不正完美地對應著我所指出的,美方權力核心理解在烏克蘭失敗的時段嗎?有疑問的讀者,可以回頭去復習五月我在兩個視頻訪談中所作的解釋和預測。
\subsection*{2022-07-08 00:37}

不是“最近”的演講,是今年三月3日給的,也就是俄軍還在被動挨打期間,NeoCon勝利在望,所以Pompeo有點得意忘形。我已經反復解釋過,NeoCon和傳統霸權主義者合流,挑起烏克蘭戰爭,目的是先整合歐盟、復興北約,然後才能統一戰綫,在臺海做決戰,徹底打垮中國。當然這裏的前提是必須先戰勝俄國;然而他們始料未及的是,Putin臥薪嘗膽20多年,準備充分,在金融、宣傳和軍事三個戰綫都大獲全勝,那麽接下來歐美自然無力在西太平洋搞事,反過來是西方國際體系和各國政權的崩潰,這不是正文已經詳細討論了的嗎?
此外,Pompeo也不是Blinken的密友:NeoCon内部本來就龍蛇混雜,有瘋狂的Ideologue、也有愚蠢的白左跟班、更有一些是財閥或政要的代表。Pompeo和Blinken雖然同歸最後一類,卻分屬兩黨,所代表的權力精英集團並不重叠。Pompeo去年減肥成功,很可能是準備2028年選總統(含2024年選副總統的可能)。他至少智商正常,可以理喻;當然這裏的“理”,指的是有射程的真理。
\subsection*{2022-07-05 00:58}

謝謝更正。視頻訪談無法暫停,全憑記憶,細節出錯,在所難免;請大家繼續依照《讀者須知》的要求,分辨主旨和旁支,做出適當的反應。
你對雅典治理效率的觀察是正確的;我鼓勵有閑暇的讀者也去閲讀那段歷史。我在最新的《八方論壇》采訪中,特別强調英國民主體制其實起源自Viking部族分贓的慣例,是貴族和武士階級制衡國王、以及彼此之間做利益交換妥協的平臺。後來商業活動興起,開始有平民致富,慢慢要求參與政治分贓,才事後找出希臘前例來裝點門面;就如同俄國建立Odessa,用了希臘文學的典故,然而除了名字之外和古希臘毫無關聯。這裏的重點在於,英國政治在幾百年崛起過程中相對高效,其實正因爲它是假民主(如果古希臘那種事事都要全民公投算是“真民主”的話),否則早就自我毀滅了。即使到了20世紀,終於有了全民投票,因爲是間接民主,而且社會階級分層明確,所以還不算太糟糕;最近信了白左,開始搞公投,不就很快面臨分崩離析了嗎?
\subsection*{2022-07-03 10:42}

歐洲的決策不可能在事先預料確定,是因爲它取決於一個人的一念之間。Von der Leyen和Baerbock再怎麽瘋狂愚蠢,如果德國總理能看清形勢、下定決心,完全可以否決、壓倒她們的妄動。當然,Scholz是律師出身,素質和Merkel差一大截,但問題在於主官本人不須要懂事,只要能重用有才華的幕僚,一樣可以做出正確的選擇。例如我在瑞聯銀的時候,總裁的智商不見得比Scholz高,但他對我的老闆完全信任,我的老闆又對我言聽計從,結果實際上執行的戰略就是我提議的計劃,自然是足夠高明的。預先判斷Scholz的素質當然不難,但我無法確定他是否有能幹的幕僚;事實上,民間觀察者是不可能獲得那麽詳細的信息的。
\subsection*{2022-06-25 09:17}

昂撒權貴利用霸權搜刮全球,利潤最高的辦法要求先搞亂目標對象,這是爲什麽NeoCon在過去20多年獲得重用、並通吃兩黨的基本原因。一旦成功地搞亂、搞垮,接下來提現有三個步驟:1)首先壓榨己方的納稅人,趁著群情激憤,透過緊急預算大幅投入軍工和其他費用,從中獲取回扣;Biden和國會之所以急著通過400億美元的援烏法案,就是爲了這道開胃菜。2)既有資源的掠奪,例如二、三月間,俄方被扣押的那幾千億美元海外資產。3)事後抄底,以破產價格收購戰敗方的優質民族資產;這其實才是最大、最重要的獲利,1990年代初在蘇聯、1997年在東南亞、以及過去一百年來拉丁美洲反復出現的金融危機,都是典型案例。這次運作的目標原本是俄國,沒想到反過來把歐盟、尤其是德國打殘了;這對昂撒權貴來説,反而是件大大的好事,因爲歐盟的財富比俄國還要厚實得多。在抄底之前先做空,不但可以確保加速歐盟的衰敗,而且獲取利潤,連日後的購買經費都預先解決了,不是無本萬利、一舉多得的完美運作嗎?
中國外匯的最佳處理方案,就在於等歐盟經濟崩潰之後,和美國人競價,至少逼他們多出點錢。
\subsection*{2022-06-23 18:21}
你是經濟系的學生嗎?美國經濟學教科書依舊被普遍翻譯采納,用來教學,這是一個很大的潛在問題。我並不是說Keynes(除了他的學説之外,我對他的個性也很認同,畢竟他不但也是理科出身,喜歡解奧數類的數學題,而且常常不給情面、直接罵人笨;他的一位同僚曾經抱怨,Keynes自封爲經濟學界的Einstein不是問題,他把其他學者當白癡來斥責才是問題所在;我想指出,Einstein有整個猶太媒體界自動自發當免費公關,Keynes可沒有那個運氣)之後,昂撒學術界就沒有具備洞察力的經濟學者,而是整個行業從Milton Friedman開始,被資本徹底滲透掌控,研究方向預先被過濾,結論則被有意扭曲或以偏概全,以服從昂撒體系的資本利益(尤其是跨國資本)為最高原則。這裏就是很好的一個例子:三元悖論的研究主要來自歐洲各國之間從50-90年代的貨幣交易史,當時那裏最大的經濟體是西德/德國,而德國馬克遠遠達不到壟斷獨霸,地位甚至還不如英鎊。實際上正確的結論應該是,一個沒有全面全球霸權(包括貨幣、宣傳、外交和軍事;參見1985年廣場協議)的國家,不可能同時完成那三項金融政策目標。你針對俄國這次金融戰戰術所做的評論,細節基本沒錯,但必須放在前述的理論認知下來考慮。\subsection*{2022-06-23 18:21}

你是經濟系的學生嗎?美國經濟學教科書依舊被普遍翻譯采納,用來教學,這是一個很大的潛在問題。我並不是說Keynes(除了他的學説之外,我對他的個性也很認同,畢竟他不但也是理科出身,喜歡解奧數類的數學題,而且常常不給情面、直接罵人笨;他的一位同僚曾經抱怨,Keynes自封爲經濟學界的Einstein不是問題,他把其他學者當白癡來斥責才是問題所在;我想指出,Einstein有整個猶太媒體界自動自發當免費公關,Keynes可沒有那個運氣)之後,昂撒學術界就沒有具備洞察力的經濟學者,而是整個行業從Milton Friedman開始,被資本徹底滲透掌控,研究方向預先被過濾,結論則被有意扭曲或以偏概全,以服從昂撒體系的資本利益(尤其是跨國資本)為最高原則。這裏就是很好的一個例子:三元悖論的研究主要來自歐洲各國之間從50-90年代的貨幣交易史,當時那裏最大的經濟體是西德/德國,而德國馬克遠遠達不到壟斷獨霸,地位甚至還不如英鎊。實際上正確的結論應該是,一個沒有全面全球霸權(包括貨幣、宣傳、外交和軍事;參見1985年廣場協議)的國家,不可能同時完成那三項金融政策目標。
你針對俄國這次金融戰戰術所做的評論,細節基本沒錯,但必須放在前述的理論認知下來考慮。
\subsection*{2022-06-22 19:16}

啊,原來俄方學術界也有人曾經想到這一點(我原本不知道,只是估算既然籃子貨幣是最優解,中方提議後,Nabiulina必然會支持),那就難怪今天出現消息,中俄已經準備要通過金磚組織搞正文中建議的合成貨幣,參見https://www.guancha.cn/internation/2022\_06\_22\_645913.shtml。當然這並不是完美的結局:如果依照我的建議,在三月俄方還打得很辛苦的時候,主動先正式提上外交談判桌,中方的姿態就會高得多。現在俄國已經大獲全勝,國際秩序的重組必須由俄方主導,並且必然讓印度坐上董事會,與中俄平起平坐。這是躲在後面讓俄國打前陣的缺點:風險低,但沒有功勞,在30年一次的國際架構重組過程中,平白損失了應有的話語權,徒然因爲俄方平衡中國的需要,讓印度白佔便宜,而且日後會有長期不斷的頭疼。
順便提一下,中方在這場第三世界反抗昂撒霸權的起義中,並非100 \% 以靜態的存在來“出力”:除了和印度一起購買俄國油氣、資助後者的外匯收入之外,其實還更進一步減低了自己成品油的外銷,所以對提高歐美通脹有一點額外的推動作用。不過這並不是出於外交大戰略的考慮,而只是一個純粹的巧合:亦即過去這年基於環保政策,持續減低煉油廠產能利用率,以致外銷量降低了超過一半;在當前的金融戰大環境之下,主要的影響是抵消了印度向歐洲轉出口產自俄國原油的成品油供應,維持全球煉油產能瓶頸,使得美國石油財團得以隨意加價,趁亂將煉油的毛利率從不到20 \% 一步提升至62 \% ,進一步增强了歐美的通脹壓力。
當然這種間接、附帶的貢獻,遠不足以拿上談判桌作爲決定未來國際架構話語權的籌碼。此外,這個作爲不但不是出於打擊霸權的考慮,而且反而是爲了取悅歐美白左,難免給予有識之士啼笑皆非之感,唉。
\subsection*{2022-06-20 21:44}

我已經說了,這是當前俄國政論界的共識,如果我不是看到相當數目的不同評論員都這麽說,是不能下那樣的結論的。參見這裏把它做成梗的圖畫:(網址過長,必須分段,請讀者自行連接囘不間斷的地址)
https://blogger.googleusercontent.com/img/b/R29vZ2xl/AVvXsEhpGf8avoXOVmH5BOyDyfMyRPZZFeO21\_
Tww6WXyga0aQ9OveaYxoCxWo89AUGEhlh4GRkjLSjrGV41rmh0fqk5JM1SX1osRztUUqzj4ZXXQ4AJttLSbBSvEST5zd2qPsij
WVzc6arMF45W56TcgFQqjZL7LbEa\_0cYEIcmujIZPWP0YwX6UjAZYxu1/w408-h338/Medvedev-Putin.jpg
今年俄方在二月攤牌、四月在貨幣戰勝出之後,Putin和Medevedev終於可以暢所欲言,說心裏的話。Putin講歐美還是純理性的批判,亦即如同博客這裏的論述;Medevedev卻直接破口大駡,威脅要消滅“European Degenerates”。
\subsection*{2022-06-20 13:05}

你對NeoCon的理解正確,但對Putin和Medvedev的印象有問題:當前俄國政論人士的共識是,他們兩人事先講好,由前者扮黑臉、後者扮白臉,實際上前者才是老成穩健的溫和派。Putin在2004年北約東擴之後,就明白無可退讓、開始準備帶頭起義、反抗昂撒霸權,但因爲雙方實力差距太大,暫時只能口頭抗議,即便如此,也引發NeoCon的警惕和反感,所以才會有讓Medvedev假扮親西方的需要,以爭取更多積聚力量的時間。別忘了,2008年金融危機和2010年希臘財政危機,俄國經濟受傷都很重。Putin重新上臺後,NeoCon又積極針對烏克蘭搞顔色革命,2014年事件爆發,Putin依舊覺得準備不夠充足,所以只拿下Crimea,把攤牌的時間又拖了八年。
過去這一周忙,不過有關NeoCon的那篇博文,我還是會寫的。
\subsection*{2022-06-13 07:21}

其實我前兩年已經都詳細討論過:這次美國經濟衰退,正規銀行界早有預期,留下非常足夠的資金儲備,所以完全不會有類似2007、2008年Lehman Brothers崩潰的事件。中產階級受益於股市和房市,儲蓄也相當充足。真正的軟肋在於影子銀行界;他們利用上次金融危機後正規銀行的監管被收緊,在過去15年大幅擴張,一般估計光是Private Equities一類,净資產就超過十萬億美元。不過他們的杠桿借貸周期平均是兩年,所以不會馬上出現周轉危機,必須等美聯儲持續加息、收緊銀根,先收回目前閑置的兩萬億Reverse-Repo現金,然後在一兩年後才會開始吃緊(這個遲滯效應也發生在恆大身上),引發體系崩塌。
所以替代美元,是這個經濟周期就打破美國霸權的充分條件,而不是必要條件。然而如果美元繼續堅挺,那麽美聯儲會有一年多時間來收拾爛攤子,期間若是因爲從歐盟吸血、或出現其他利好事件,説不定美國國内通脹提早溫和化,那麽就可能又可以矇混過關。畢竟烏克蘭的人力物力資源有限,雖然目前俄軍進展緩慢,但烏方要撐過年底而不全面崩潰,機率基本爲零;其後能源和農產價格轉為疲軟,國際通脹壓力會有所疏解。
\subsection*{2022-06-12 14:48}

20世紀末的學術界建立了共識,確認地質變動、氣候變化和生物演化等等自然歷史的現象都有一個共同的特徵,也就是絕大多數時段處於穩定態,當這個長期穩定態終於崩潰的時候,過程是迅速而突然、而且細節上是隨機、混沌的。抽象來説,這是大Scale之下Phase Transition的自然結果:只要系統的規模夠大,就會有這個趨勢。其實人類社會也遵循類似的規律,所以後世的歷史學人才得以將過去劃分出不同的時代。
我們正在經歷冷戰後建立的全球化國際秩序的崩潰過程。如前所述,這個過程在巨觀上是必然、但微觀上是混沌的;例如美俄之間的金融戰和俄烏之間的軍事衝突,雖然有客觀的戰略利益脈絡主導,但執行細節依舊取決於極少數人主觀的一念之間,而執行細節的取捨卻是勝敗的關鍵;這些勝敗得失稍有出入,就影響世界歷史的未來走向。中國如果選擇不參與金融戰,那麽作爲一個被動的旁觀者,更加只能靜待主動玩家之間的鬥爭塵埃落定。所以要談戰略環境是改善或惡化,端視俄方是否勝出、勝出多少、如何勝出。已知事實指向審慎樂觀。
上面只能“審慎”樂觀的主要考慮在於,雖然過去三年我一直估算美方這次的滯漲衰退應該足以導致霸權失落,但這個邏輯推演是建立在美國之外的主要玩家都能避免重要、明顯戰略錯誤的假設之上。現在歐盟如此積極、主動、反復地采納自殺性政策,把原本是美元最佳替代品的歐元搞成先死一步的墊背,美國熬過這場劫難忽然不再是極小機率的事件;而中國在貨幣金融上的不作爲,如果持續下去,更可能會把那個機率翻轉過來,容許昂撒霸權苟延殘喘到再下一場金融危機。
武統時段是邏輯鏈的遠遠更後端,連美國這波經濟衰退的程度都還在未定之天,要精確估算其概率所需的假設層面太多、不確定性太高,毫無實際意義。我覺得和幾年前相比,可能Expectation Value沒變,只是Standard Error反而大幅增加了。(學過統計學的讀者可能覺得有點奇怪,武統是個“Dummy”或者“Boolean”變數,談何SE?這裏我考慮的理論Framework是所謂的“Limited Dependent Variable”,内含一個“Utility”或者“Latent”連續變數,對應著所有主要玩家對引發武統的綜合貢獻;因爲它是連續變數,所以可以談Standard Error。有興趣進一步瞭解細節的讀者可以搜索“Probit”函數,那是Limited Dependent Variable的一個簡化特例。)
\subsection*{2022-06-10 12:34}

英國媒體比較嘴硬,美國政府、軍方、情報和媒體在最近幾周已經紛紛開始為戰敗卸責預做準備了。最新的是這篇《紐時》的文章:https://www.nytimes.com/2022/06/08/us/politics/ukraine-war-us-intelligence.html,居然宣稱Zelensky把美國蒙在鼓裏;這很顯然是情報部門怕被Biden/Blinken/Sullivan等人拿來當背鍋俠,事先準備擊鼓傳花、把鍋丟給烏克蘭;當然這是Biden自己批准並采用的策略(參見https://apnews.com/article/russia-ukraine-donetsk-education-business-85a2489b4fe1042cd03b0ae6f4421a04),也是上個月我在唐湘龍節目裏就預言過的。
\subsection*{2022-06-09 09:59}

你已經抓到一個重點了,亦即短綫操作、圖利頂層。剛好1990年代Jack Welch在企業管理上成功引領那個風潮;美國原本就是僞裝成國家的一個企業集團(Conglomerate),那麽在外交大戰略上也依樣畫葫蘆,不是很自然的嗎?
與此同時,他們可以壓榨盟友,靠的是白左教的興起,從而使17世紀以來的近代地緣政治鬥爭,退化成爲中世紀的宗教戰爭。東羅馬帝國(歐盟)衰落的轉折點之一,就在於十字軍(北約)在東征異教徒(俄國)的路上,順便攻破並搶掠了同屬基督教(白左教)的君士坦丁堡(德國工業);how is that for eating allies for lunch?
我宣佈新博文題目之後,請讀者剋制好奇心,不要提前東問西問,否則干擾寫稿,損害的是所有讀者的共同利益。
\subsection*{2022-06-07 23:31}

俄軍受困於國家經濟財力不足、以及電子工業技術落後,以致在預警機和精確制導彈藥上無法像美國那樣批量部署,固然是真實的缺失,但不是軍方本身的責任。俄國自尊心太强、加上對中方有忌憚,沒有事先引進足夠的零部件,這如同戰事第一階段的誤判一樣,也是政治和文化方面的問題,同樣不能怪到軍隊頭上。至於對基礎設施只做最低程度的打擊,則是政略級別的選擇,也不是將軍們能置喙的。我的核心論點,在於:1)純粹只看俄軍的戰略選擇和戰術執行的話,其表現至少是合格的,遠超網路視頻、照片和消息所給的浮面印象;這是因爲烏方不但有昂撒媒體的加持,而且把宣傳誤導視爲最高優先,對人命和誠信都毫不在意。2)俄方的整體戰力雖然不如美軍,但沒有代差,要做戰略防禦是足夠的。
\subsection*{2022-06-07 23:25}

我去年對英國政壇的分析並沒有修改的必要,這場不信任投票醖釀發生在5、6月正是博客估算那些幕後力量互相衝突的必然結果。至於爲什麽不信任案沒有通過,其實也很簡單:最近幾個月保守黨内的兩個儲君Truss和Sunak都自爆掉了(還記得Laschet在探視水災時開懷大笑的照片嗎?);前者一連串的蠢話蠢行讓她獲得英國版Harris的光環(Biden提前下臺的最大阻力,也在於Harris的公共形象),而後者更慘,夫妻檔出現了逃稅醜聞,而且還是利用外國國籍來玩的,其政治前途基本止步(這個爆料當然可能是支持Truss的Murdoch系搞的,但沒有證據)。在不信任案水落石出之前,沒有讀者留言討論英國政局,我也就沒有機會介紹這些新發展(除了《2022年國際局勢的回顧與展望》【後註九】);其實它們事先就代表著讓Johnson撐到今年底的原計劃再度成爲保守黨幕後支持者的選項,所以投票結果難以預料。現在結果出爐,也並不代表Johnson安全過關、或者Truss已經出局(雖然Jeremy Hunt呼聲上來了),只是將替換首相的時間稍作延遲,而且不會太久(最可能是2-6個月)。
目前的公開信息,的確指向中國對沙特是拆臺美國(美元)霸權的關鍵沒有足夠認識和決斷,以致在中東和巴基斯坦方向都犯了嚴重錯誤,白白放棄了大好良機。所幸俄國依舊在引領這場第三世界對昂撒獨裁的反抗,中國或許能繼續搭一班順風車,不過乘客一般是必須向操盤者付費的。
\subsection*{2022-06-07 22:45}

席亞洲在觀網的回復,我剛剛看到了,真是讓人搖頭。增程彈尾部噴火,是非常明顯的特徵,他硬是說那張照片“模糊”(其實幾周前照片刊在觀網自己文章的時候,是高清版,模糊是放在《八方論壇》視頻才造成的),所以無法確認,明顯是睜著眼睛説瞎話。這還沒有考慮到重炮從Mariupol城區向兩三公里外的Azovstal射擊,哪用得着增程彈?更別提增程彈所帶來的精度損失,反而是定點打擊堅固工事這個任務所不能接受的。
其實我批評大V網紅,素來都針對他們不尊重邏輯和真相的心態問題,知識不足反而不是重點,畢竟現代社會極爲複雜,再博學、再謹慎的人對細節掌握也會有遺漏,包括我自己在内。這裏我在前後文已經特別强調問題核心在於預設的“Narrative”,結果他不但無視辯證的主旨,轉換話題做狡辯,而且還故意造假撒謊。雖然每次我批評這些非理性的愚昧態度,總會有不少噴子立刻自告奮勇、强辯硬扯來示範自己的愚昧(光從這些人不知道、不遵從邏輯辯證的規則,既不舉證、也不談邏輯,就可以確認他們的愚昧),照理説我們應該感謝這些志願者才對。但席再怎麽説也是知名媒體的編輯,對中國軍迷群體整體理性程度特別低負有很大的責任;我以前寫過一篇博文專門批評過,後來以爲他有所改正,但顯然是過於樂觀了。
我認爲最合適我自己的頭銜是“王老師”;你把席亞洲也稱爲“老師”,對我和其他尊重事實真相的人都是莫大的侮辱。禁言一個月,再犯拉黑。
\subsection*{2022-06-04 00:44}

我一個體制外的人很難直接影響決策,必須三步走:民間人士=>學術界=>官方。經過博客和讀者群多年的不懈努力,最近中國學術界終於開始有國際戰略上的正確反思,不過我還沒有看到在外交戰術上的實用建議(連英美假新聞,都還在簡單否認,而不是直指媒體造假),大概是來博客學習的人還不夠多;我們只能繼續等待。

剛剛在《觀網》的社論欄一連看到兩位知名的學術和智庫界人士發表新文章,内容無關緊要,但是他們不約而同地在開場白討論自己最近的評論和推測準確性如何,這雖然是博客的日常,卻是中國社科界的新現象,剛好示範了我所談的三步走路綫,所以在此進一步解釋一下。
我人在體制外,對戰略、社會、行政、外交、金融、經濟等方面的建議,只能先教育民間的讀者群,通過大家的留言評論影響在大衆媒體發文的學術人士(“鄉村包圍城市”),然後慢慢傳播給整個學術界,一旦成爲主流正統意見,才能改變官方政策。不過我以往提過,除了正確有用的知識之外,我也試圖示範應有的理性態度和科學方法,而這些態度和方法中最基本的,包括了避免清談、堅持實用;而要有實用性,就必須做嚴謹、精確的預測,然後不斷回頭復盤分析。所以被有識之士如張文木教授反復批評為廢話專業的中國戰略學術界,現在終於感受到競爭對比的壓力,而開始談預測的準確性,是一件好事。
然而光是願意做預測,是遠遠不夠的。這裏有幾個常見的陷坑:首先,預測只在有爭議性的話題上有意義(亦即計算機科學所説的“信息量”或“信息熵”)。我在《讀者須知》一文中已經解釋過,預測的準確與否,必須看事前事後對與錯的機率對比,定量來説是對錯比值的對數差;否則可以天天預測明天太陽會從東邊升起,必然會有100 \% 的勝率。
其次,預測必須用詞精確。這個可能、那個也可能,等於什麽都沒有說。其實現在的物理和科技公關稿也特別喜歡玩這種文字游戲:“新理論可能推翻標準模型”在邏輯上自然也代表著“可能沒有推翻任何東西”,如果不詳細標明機率,純屬騙人的空話。
但是最重要的,還是我以前提過的Diagnosis->Prognosis->Prescription這個分析鏈,也就是預測(Prognosis)必須建立在嚴謹的事實證據和完整的邏輯推演(Diagnosis)之上,否則大家凴運氣胡猜,1000人之中就大機率會有一連猜對10次擲硬幣結果的幸運兒,然而他的實際智慧值依舊是零。這個謬誤其實正是中醫教的核心教義所在:屠呦呦不但做了雙盲實驗,證實青蒿的功效,而且進一步確認有效成分,把青蒿素分離提純,並對它的分子結構、化學性質和殺蟲機制做了詳細深入的研究。能對任何其他中醫藥材做出同樣研究的,才是科學;不在乎科學標準而純靠“經驗”、“感覺”來散佈傳言、妄做論斷的,就是宗教性的傻逼;明明有資源可以做科學研究,卻不願做、不去做,只管吹牛賺錢的,則是詐騙集團。
\subsection*{2022-05-30 02:39}

我說運氣,指的有兩點:首先美方完全可以在成功整合歐盟、重建北約紀律之後,適可而止,既不擅自扣押資產、斷送美元的信譽,也不强迫歐方自斷經濟命脈、試圖禁用俄國油氣。烏克蘭原本就不可能在軍事上獲勝,那麽自然也沒有必要(對國家戰略而言;實際上美國決策精英爲了大撈一筆,很有必要通過那個400億的援助法案)反復加碼、升級武器支援。不知分寸的後果,就是弄巧成拙、因小失大,徒然在金融戰和宣傳戰上碰了一鼻子灰,也讓俄方的軍事勝利更有分量。
其次是俄軍的表現可圈可點;雖然第一階段因爲政治誤判而采取了錯誤的部署方案,但人員損失有限(~1500陣亡),裝備補充不成問題。一旦進入第二階段,穩扎穩打,就無可挑剔。大家別忘了,烏方寧可枉送大批士兵的生命,也要爭取好看的TikTok視頻,所以只看這些剪接出來的片段,必然會得到扭曲的結論。俄軍戰術上真正的弱點,只在於對無人機還沒有很好的對應,但這是人類第一場雙方都批量投入無人機的戰爭,北約自己的反無人機激光武器預期2026年才能入裝,中國雖有原型,也沒有普遍列裝部隊,而俄國可以少量緊急部署,已經算是國際第一梯隊了。不論如何,俄方戰果遠超觀衆的預期,這必然會對北約的向心力做出進一步的打擊,從而在中方沒有主動出手的背景下,就預先擠壓了美國在臺海製造事端、升級施壓的運作空間。
\subsection*{2022-05-29 22:53}

《Defense Politics Asia》是一個新加坡人的視頻網站,他從原本完全中立的立場出發,幾個月下來自然而然地演變爲“Putin的走狗、反烏的先鋒”,据他自己說,是因爲“RT does propaganda with truth.”(還記得我對新冠起源陰謀論的批評嗎?)而烏方和昂撒媒體剛好相反。
Scott Ritter是長久以來一直反戰的人士,在這個議題上站對了立場,但他的人品很有問題,對事實真相也不夠尊重,我建議大家不要理他。例如Gonzalo Lira被囚禁期間,Ritter造謠說已被處決,這並不是對抗昂撒謊言帝國的正確手段。而且他私德很差,曾經因爲戀童癖進過監獄(和Assange不同,被正義自媒體人查證不是誣陷)。所以Lira自己和《Dreizin Report》都對他深惡痛絕,Larry Johnson則只是就事論事地批判Ritter的胡扯(參見https://sonar21.com/debating-scott-ritter/)。
中方的精英階級,的確有很大的思想、認知和態度問題,即使是外交部這樣有心做事的單位,都屢屢不得其法,像是人民銀行、中宣部、教育部、科技部,那更是等而下之,和Nabiulina形成極大的反差。我從2014年注意到她,發現我想到的策略她也在第一時間就想到而且執行了,所以才會在幾年前就評論過“她是個厲害角色”。中國體制内完全找不到這樣的人,自然也談不上主動開拓民族崛起的國際空間;這是爲什麽林毅夫可以只談保持内部發展,也是爲什麽我認爲美國自身戰略戰術錯誤、以及俄方的積極作爲,對中國的國運有特別重要的意義。
我對中國當前面對的内部難題,大部分還是審慎樂觀的,例如半導體和教育部,這些問題有目共睹,一般民衆也能簡單看得出來,反而必然會被解決。真正讓我憂心的,還是學術管理的一味放任,因爲這個議題的天然專業特性,我預期很難讓高層理解其惡劣和嚴重程度。我最近開始破例抛頭露面,真正的用意就在於盡可能廣汎地撒佈這方面的知識,希望能有奇跡,將正確的認知和策略傳達上去。
\subsection*{2022-05-29 01:37}

2014年Putin一念之仁,給了北約和烏克蘭八年的時間來重建機械化兵團,眼看著要被俄軍用全部總兵力的1/5花四五個月就完全消滅了。這次烏克蘭只剩下一半的領土、1/3的GDP、不到1/4的壯丁,你準備花多少時間和金錢來重建戰力?海空軍、裝甲和火炮損失殆盡,新武器哪裏來?美國新通過的400億美元援烏法案中,填補已發送武器空缺有90億,新的“援助”反而只有60億(其中還包含了美方説了算的“訓練費”),其他的分發給各式各樣的利益集團(包括搞宣傳顛覆作業的國務院和NGO);換句話説,這個夏天烏克蘭頂多只能拿到過去三個月所接受美製武器的2/3,這足夠扭轉戰局?
本周Kissinger出面建議烏克蘭割讓領土,然後昨天《Washington Post》登文采訪當了逃兵的一整個烏軍連;這都是在為即將急轉直下的戰局做鋪墊,讓西方讀者做好心理準備,事後才方便為一連撒了幾個月謊話的主流媒體脫罪,把責任扔到匈牙利、德國、意大利和Zelensky身上。
從戰略層級來看,這次美國安排烏克蘭挑釁俄國出手,其用意是預先整合北約,為打擊中國提前統一戰綫(若是直接在臺海搞事,歐盟置身事外的可能性太高);而其攻擊的手段分爲軍事、經濟和宣傳三個層面。現在後兩個方向已經完敗,等軍事上也全面敗退,甚至Zelensky政權崩潰,歐盟還會有胃口摻活臺海嗎?日韓還敢搞事嗎?我在唐湘龍節目中說,中方一向運氣極佳,這次可能又要躺贏,就在於此。
\subsection*{2022-05-27 11:10}

我指的是有正規訓練和良好裝備的野戰部隊快要見底了;現在緊急奔赴戰場的,例如本月新出現在前綫的四個旅,不是地方民兵(三位數番號)、就是後備役動員出來的大叔和老伯(番號是60+),頂多四五周的基本訓練,只有當炮灰的意義。
二月仗一打起來,就有幾百萬烏克蘭人出國避難,其中包含了大比率的壯丁。Zelensky所謂百萬雄師、70萬動員,都是説給歐盟民衆聼的宣傳吹牛。就算硬是拉夫拉出這個數字,也不可能有足夠的裝備和軍官來配套。就連烏軍正規部隊,最近都屢屢發生前綫軍官抛棄單位、私自逃生的事件,昂撒媒體故意無視罷了;這是因爲懂軍事的,自然能看穿烏克蘭當局的謊言。
Putin並不是不敢動員,而是沒有必要;其實俄國國内民意沸騰,要求盡速打垮烏克蘭的聲浪很大。俄國義務兵在法律上只能用來保衛國土,除非正式宣戰,不能出境戰鬥。當前俄軍的做法是接受義務兵中志願上前綫的人,帶著他們的重裝備成建制地分散開、配發給東烏民兵的步兵單位;因爲這些是二三綫的部隊,所以用的是俄軍最老的現役裝備,例如本周被昂撒媒體拿來説事的T-62。其實對面烏軍的機械化裝備基本打光了,坦克是否先進根本不重要。
\subsection*{2022-05-20 15:40}

借此處宣佈:剛剛上了唐湘龍的節目,https://www.youtube.com/watch?v=yD7HbsVBxYo\&t=5s,其中提到烏軍疲態畢露,未來幾周軍事局勢的演變,可能會對全球外交態勢有良性的影響;我下周末會上史東的節目,做進一步的討論。
此外,許多大陸聽衆對“過去20多年中國外交戰略上基本毫無作爲,所幸運氣極佳”那條論述非常不滿。博客讀者應該知道,這指的是2001年Rumsfield原計劃在2004年挑起臺海戰事、被911事件打斷;2008年金融危機後,美國虛弱不堪,以致雖然立刻開始宣傳動員,對中國做全面妖魔化,Obama/Hillary政權卻被迫反復推遲戰略圍堵,一直到卸任前才出臺TPP和TTIP;然後2016年Trump又意外當選,打斷了建制派的長期戰略部署,並且把對華敵意公諸於世;2022年,又是俄國主動觸雷,代替中方承受西方的新一波全面打擊。這裏的一系列正確戰略認知和反擊方案,都是我提前好幾年預先帶頭提倡,運氣好的,才有中國官方姍姍來遲,運氣不好的,例如當前的貨幣政策,當局還在拖泥帶水、猶豫不決之中。這其實是我一直不熱衷於上大衆媒體節目的主要原因之一:我做的是學術性、教育性的邏輯論證,那些節目聽衆習慣的卻是娛樂性、政治性的主觀反射動作,聽到新觀念,想的不是去原始來源追查細節、補充知識,而是坐地反噴。不過反過來看,明知普羅大衆是無可救藥的愚蠢,也要為國家和人類的前途,努力教育其中少數有理性思考能力的人,原本就是博客的使命;既然還有許多改革沒有被采納,我也只能强迫自己、捏著鼻子和非理性者打交道。

戰術被扭曲,是股市這類國内利益分配的體系會有的問題;貨幣政策則是大國博弈、人類歷史轉折的關鍵,即便有突破性的戰略建議,也必然需要體制内的專家先認可,才有被采納的可能。這裏的問題在於如果他們日常關心的是蠅營狗苟之事,哪可能有餘裕去考慮相關的宏大背景和複雜取捨?反腐固然是必要的糾正,但也會打破組織内部的理性氣氛,阻礙專業意見的客觀交流取捨,讓有心人藉機無限上綱,抹黑出事者的所有立場。例如孫國峰剛下臺,同一天《觀察者網》就有文章宣稱因爲孫反對Modern Monetary Theory,這代表著官方對MMT的肯定;還好幾個小時之後文章就下架了,應該是《觀網》編輯被提醒其作者別有用心,不應爲他站臺。
\subsection*{2022-05-18 15:03}

其實這個議題我以前談過,亦即冷戰期間美國曾經有系統地投資社科、尤其是國際研究,培養了大批瞭解各國風土民俗、熟悉當地精英結構的人才。冷戰結束之後,這個體系先被擱置忽略,近年來更是受到有意的排擠打擊。打擊的真實深層原因倒不是爲了省那點小錢,而是這些專業外事人員剛好就是能夠揭發美國政治領導階層國際貪腐管道(參見該系列博文)的群體。相對的,想在全球搞全面顛覆侵略的Neocon恰恰是最方便這類國際貪腐的政策藉口(參考最新的400億美元援烏法案連最基本的審計都豁免),所以不但被共和黨的Dick Cheney挑選重用,政黨輪替之後也可以改名為Neolib進入Hillary主管的國務院擔任副國務卿和助理國務卿等職務;Hillary退休之後,輪到Biden家族享受這些特權管道,所以他也順道網羅了這群幕僚,其中專門負責在東歐針對俄國搞事的,就是Victoria Nuland和她丈夫Kagan的整個家族。這些Neocon/Neolib的本質是非理性的蠻幹,懂事的專家當然必須先滅口,類似從李登輝開始、台灣執政政黨一直致力淘汰有良心、有能力的專業官僚。

Neocon和Neolib的重點不在於con vs lib的左右路綫之分,而在於那個“Neo”“新”字:他們特別著重製造國際衝突、混亂、腐敗和戰爭,其創教(此前Neocon、Neolib的教義和此後截然不同)聖經是美籍猶太人Richard Perle在1996年為以色列所制定的政策白皮書《A Clean Break》(參考https://en.wikipedia.org/wiki/A\_Clean\_Break:\_A\_New\_Strategy\_for\_Securing\_the\_Realm,以及https://www.youtube.com/watch?v=Z1LYNy8bhf8),執行則交給美國,所以自然是先從中東開始著手,而且即使後來把目標擴展到東歐和東亞,也依舊由猶太幫主導。他們不計代價也要搞亂世界的核心任務,特別方便財閥政要渾水摸魚,因而可以通吃兩黨、長盛不衰。
\subsection*{2022-04-19 11:49}

中國政府在過去一年多,一直默默地在清理地方財政和金融,尤其是地區小銀行被土豪當提款機的現象,排查起來非常吃力,又不討好。這其實是普世性的難題,其他國家只能等爆炸之後來收拾殘局(例如1990年美國的Savings \& Loan危機,以及2017年以來印度的影子銀行問題),中國勉强算是未雨綢繆,雖必須付出代價,但至少和外國對照下是較優的。招行當然不是小銀行,那麽或許清理金融界已經進入下一個階段。
Gonzalo失蹤之後,有些人說風涼話,認爲他太過招搖、自取其咎;其中大部分是原本就沒有什麽良心可言的,所以自然難以體會為傳播事實真相而犧牲奮鬥的思路,不值得我們理會討論。不過另外還有極少數是同路人,例如《Moon of Alabama》,多年來小心隱藏身份,似乎有資格做批評,但我想在此為Gonzalo辯解一下:和原本就專注在揭發政治黑暗面的MoA不同,Gonzalo並沒有事先計劃要做像是拆穿烏克蘭納粹政權謊言的工作,他的本行是經濟/金融方面的記者,多年來談的主要是歐美資本主義體系的剝削性,不但沒有必要匿名,而且必須實名報導才有公信力。他做公開評論已經很多年了,忽然戰爭在身邊發生、宣傳戰謊言滿天飛,他人剛好處在關鍵地點,有許多獨家的信息;不談的話,對不起良心,匿名談的話,沒人理,所以繼續用既有的管道是唯一的選項。對這樣的選擇,我只能說很佩服他的勇氣和執著;譏嘲他不夠謹慎的,連事後諸葛亮都算不上。
我自己在八年前開始寫博客的時候,也只想要糾正台灣公共論壇的一些錯誤認知,用實名絕對是有助於建立公信力的做法。後來博客討論的越來越廣汎深入,開始碰觸尖銳敏感的話題,要退縮或匿名重來,已經太晚了;這時只能把局面看作考驗自己良心和勇氣的挑戰,在不影響傳播真相的前提下,盡可能避免招惹事端,例如我至今仍然沒有從大陸的任何機構(不論是公家或私營)收過一分錢。這樣的慎重當然不足以保證絕對安全,但如果沒有願意為理想和公益犧牲冒險奮鬥的知識精英,一個國家民族還有前途可言嗎?
\subsection*{2022-04-17 20:13}

要是人類都能理性精算,世界早就大同了。 所有的宗教迷信,包括白左思潮、紅衛兵、中醫教等等,都必須對理性客觀的利害分析預先準備好預防性的套路説辭,才可能裹挾足夠的群衆在非理性的道路上狂奔到底。例如上個月我說能給德國利益和做出損害的都是俄國,然而綠黨和其支持者不但不理這個結論,而且對這類考慮完全嗤之以鼻,因爲那是“邪惡”的。類似的理性論述對臺獨也無效,因爲“人生不只是錢”。其實不道德的是為私利而綁架公共政策(“損人利己”、“壞”,參考這幾天有關新冠防疫的討論),解決方案正是一切都應該取決於整體公益。那些邪教故意將公益和私利混肴,把所有的利害得失計算都説成“不道德”,實際目的卻是爲了强加自己的迷信偏見於社會群體之上(“損人不利己”、“蠢”),才真是最可惡、最危險的。參見以往這方面的博文和留言。
\subsection*{2022-04-14 13:01}

毛時代曾經廣汎資助國際共產組織,甚至試圖和蘇聯爭奪主導權,這種當凱子還多面樹敵的事,以當時中國國力而言,顯然不是最優選擇,所以鄧掌權之後,外交國安就不再考慮意識形態了。但那是基於當時中國國情和世界局勢所做利弊分析的理性結論,並不是怕事,否則也不會和越南打那麽些年。現在中國政府、學界和民間,開口閉口就是“外交傳統”、“老祖宗的智慧”之類的胡扯,其實和歷史事實完全脫節,純粹就是懶得用腦又兼怕事的庸人在找藉口。
還是那句老話:一切都應該以事實爲基礎,根據第一原則來做邏輯推論,精算利弊得失。凡是無視事實邏輯,只談類比聯想(上面所談的假“傳統”,就是忽略時代背景,只看過去30年韜光養晦階段案例的無腦類比)的,必然是在自欺欺人。
\subsection*{2022-04-08 04:39}

我對亞元的構想,的確是中國主導,爭取主要資源國合作,容許友好並富裕的中型國家有限參與,小國跟著用就是了,至少亞元銀行不會像歐美那樣去沒收外匯儲備。這裏要小心的是,如何處罰濫發自己貨幣的成員國。一個非常重要而簡單的原則,是中國必須勇敢站出來占據舞臺中心的聚光燈焦點,不要畏畏縮縮地搞多邊平等,否則必然是邀請自私國家來白占便宜。這其實是比不建立亞元更大的危險,因爲要看出建立亞元是正確方向並不困難,The devil is in the details。從這届中國政府搞盲目金融開放、無限擴張上合、一帶一路跪求參與、亞投行自願當凱子等等前例來看,這樣子把正確戰略決定在執行細節上搞砸的機率很高,所以我也故意在文章裏留下一個綫索:直白地批評了邀請印度入股的思想,但放在最後那個章節,以避免若干讀者看到一半就起反感。不過如果幾個月後國務院公佈,亞元銀行創始成員包括印度,我們就可以簡單判斷國家和人類整體利益又被某些思想扭曲的傻子懦夫打包白送了。
Biden這次聽從Hillary系的NeoCon,斷送美元霸權,金融系的不滿已經明白顯示出來,若是Obama系決定藉機落井下石,就可以簡單推出己方的Harris上位,這是我最近剛剛詳細討論過的事,讀者應該仔細閲讀,不要重複發問。

Harris依附Obama,然後一起對Biden捅刀的可能性,共和黨媒體也注意到了,大家可以參考以下這個視頻:https://www.youtube.com/watch?v=j58Bx0Ebppw
\subsection*{2022-04-07 12:37}

亞元如何設計,牽涉到的專業考慮太過艱澀,所以我在正文裏只草草帶過,反正人民銀行的主管一定會被咨詢,他們懂得其中的奧妙,自然應該同意是最優解,這裏只粗淺解釋一下。像是歐元這樣沒有財政聯盟(亦即各成員國的稅收和支出相互獨立)的合成貨幣,最大的問題在於兩點:首先彼此之間匯率固定,無法視各經濟體的經濟健康狀態和貿易順逆差來做調整;其次各國的財政順逆差也有不同,嚴重逆差必然導致貨幣增發,最後爆發希臘式的危機。歐盟至少還有統一的市場監管標準,而且有長遠、持續的整合意願;中國建立新國際貨幣的主要用意,卻是在於取代美元和歐元,做爲鬆散國際社會的公平通貨,所以這個新貨幣聯盟也必須很鬆散,尊重各國的主權和自由。
用多個國家貨幣組成的籃子來作爲新合成貨幣的本位,自動滿足上面的所有要求:各國仍然保有自己的貨幣在國内使用,匯率可以浮動,增發完全自由,亞元銀行只要預先訂立規則抵抗要求救援資助(執行上是增發亞元來換該國貨幣,等於是人爲地維持其成分占比;所以股份不能固定或只依GDP計算,而必須考慮貨幣濫發的處罰)的政治壓力,就不會有金融上的大損失和危險。參股的國家甚至可以隨時退出,也可以選擇投入的深度,而且成立快、彈性大,是遠遠最優的方案。資源國家現在看來是最重要的成員,其實是世界經濟周期進入整體通脹階段所引起的暫時現象,十幾二十年之後局面可能完全不同,不過難關在當下,所以爭取他們確實是首要優先。中俄的軍事力量也的確會是必要的保障,雖然不足以全面對抗美軍的全球優勢,但在亞洲保護不被顔色革命輕易推翻,應該還是做得到的。至於不作爲,那等同於把幾十年纍積的國際資產送給美國當人質,是明顯的下下策。
\subsection*{2022-04-07 00:56}

有關Putin該不該在二月底急著動手,我已經多次論證過,在國際大環境之下不是最優的。他一個很聰明的人,爲什麽會那樣決定,當時就有傳言可能是烏克蘭正在集結大軍,要一舉打下東烏,所以Putin決定搶先出手;最近有更多的軍事細節被暴露,證實烏軍的確是在準備如此作戰,開戰日期定在三月底。但是我在戰前就評論過,即便如此,最佳的反應依舊是預先集中遠程火力,對烏方先做幾天殺傷再派軍入境,以杜絕國際口實。他的戰略戰術選擇,簡直像是有意要給英美所有可能機會來綁定歐盟,然後又不積極還擊,實在很難解釋。
在這樣的背景下,未來幾個月充滿了許多重要的歷史轉折點:德國會被迫立刻放棄進口俄國油氣嗎?放棄之後,經濟後果會有多嚴重?夠不夠推翻現政府?中國能不能采納亞元建議?能不能爭取到中東產油國加入?事後歐盟經濟所受的打擊會是致命的嗎?換句話説,歐盟在這場經濟危機中保持完整嗎?貧窮國家在新冠和通脹雙重打擊下,有多少個會淪入當前斯里蘭卡的困境?有多少政府會被革命推翻?會不會因而有戰爭?在俄軍完成殲滅烏軍主力之前,Zelensky會有足夠的智慧趕緊妥協嗎?如果沒有,那麽Putin會滿足於强制實施當前的要求,還是如我在一月討論的,拿下整個南方沿海的俄語區?法國大選的結果如何?新總統能對歐盟做出懸崖勒馬式的挽救嗎?英國保守黨在地方選舉的落敗會有多慘?Johnson是否因此下臺?女王能再活多久?上一次人類社會身處這個級別的分水嶺峰頂,是1989-1991年那個時段,開啓了後冷戰全球化階段,這次的轉折,會帶來什麽樣的國際秩序?一級强權之間能避免全面戰爭嗎?這些都是很有意思的問題,我們繼續觀察吧。
\subsection*{2022-04-06 02:22}

Putin這次前後一共只投入了19萬兵力(含東烏民兵、車臣部隊以及後期增兵),攻打光現役軍人就有26萬的烏克蘭,不得不開闢北綫戰場,分派5萬精兵佯攻基輔和Kharkov,另外還有2-3萬人在南綫佯攻Odessa,綁住了2/3的烏克蘭總兵力(亦即所有二綫和三綫部隊),勉强在東綫主戰場凑出1.5:1的局部優勢。但即便如此,也不足以全綫進攻,必須優先解決盤踞Mariupol的亞速營,並且撤回北綫佯攻部隊,然後才可能對剩下的6萬烏克蘭野戰兵團形成必需的3:1優勢,完成包圍殲滅;這將是未來一個多月的戰事主軸。
至於Putin用兵爲什麽如此摳門,我已經反復討論過了,可能是一方面低估了過去30年烏克蘭教育體系對國民宣傳洗腦的力量,另一方面則必須保留預防北約干預的戰略預備隊。不論如何,這依舊違反了兵貴神速的原則,所以引發了一連串的戰略和戰術代價,其中最基本的就是讓對手能回過氣來做出針對性的反擊。既然對手的幕後老闆是謊言帝國,這種造假栽贓的事也必然會發生;事實上博客在戰爭一開始就預測過了。
詠春拳的總訣之一是“千招千破,唯快不破”,軍事也如此,所以我也反復解釋過,正確的手段是做好萬全準備,一旦開戰就以絕對優勢兵力速戰速決,不讓對手有喘息回應的機會。這裏我指的不只是對手試圖做出對應性的兵力部署和調派,宣傳顛覆的造假手段也是需要時間來謀劃並執行的,所以一樣適用。

其實從目前已知的事實觀察,若真要做邏輯推演,還可以得到更多的結論,例如Putin並沒有把駐西伯利亞的遠東部隊大幅調動投入歐洲戰綫,這暗示著(亦即不是100 \% 決定性的嚴謹證據,只有機率上的指向性)事先並沒有和中方達成默契和協作。
\subsection*{2022-04-03 16:37}

The most likely cause of this inconsistency is again Scholz's idiocy. What Putin said was probably something like "I am not asking for ruble payment on April 1. It is too short a notice. " What he actually meant was of course that there would be a grace period of a few weeks.
On a side note, I did not publicly predict this ruble requirement because I was not sure Putin would be cool-headed/cold-blooded (in the "sangfroid" sense) enough to go all way, not because it was not the obvious right move. In fact, I myself would have done it much earlier and much more broadly, requiring "friendly nations" to do so as well. Accommodations can always be made on a case-by-case basis. Other differences in opinions include the bombing of Ukrainian oil storage: it's something I would have done in the first week of the war, not the fifth. Furthermore, the Russians still have not destroyed any major bridges, not even those near the EU border. Putin was clearly hoping for a rapid collapse of the Ukrainian regime and wanted to minimize both the civilian casualty and the rebuilding costs. While this line of consideration could not be ruled out a priori, it was not something that I personally would base a war plan on. The facts on the ground now prove that 30 years of anti-Soviet education leaves its deep mark on the Ukrainian populace. Recall that the brainwashing power of education systems is an old topic on this blog.
\subsection*{2022-04-03 10:31}

首先,Trump並不對國家機器有確實的掌握:William Barr的忠誠對象在於共和黨内建制派的一個分支,所以那台電腦上繳司法部之後,就沒有下文了。目前的消息都來自修理店老闆的備份,由不屬於共和黨宣傳體系核心的《NY Post》試圖揭發,但立刻被民主黨主流媒體圍剿,貼上“俄國栽贓”的標簽之後,天真群衆才會認爲“拙劣”,博客讀者應該看得出它不像是造假。其實不但民主黨政客和媒體知道那是真的,共和黨人也知道,問題正在於那家(以及還有類似的)公司是美國政壇行内人靠全球霸權為自己撈錢的日常管道(參見《從Manafort案談起》),兩黨都有大佬積極參與,所以不但Barr不想碰,共和黨黨工也沒人願意附和《NY Post》,只有金融系用不着拿這種酬佣性的小錢,可以收藏起來當把柄。
\subsection*{2022-03-29 21:12}

這兩天Abramovich忙得很,又是參與和談,又是陪烏克蘭代表一起中毒,中毒後還能跑去Moscow和Putin見面。好玩的是,這些消息的初始來源都是已知MI6幕後主控的宣傳管道(例如《The Times of London》),然後由《BBC》之類的國際主流媒體引用轉發,轉發的時候還扭扭捏捏,能不提出處就不提出處,例如這裏:https://www.youtube.com/watch?v=P1jlyVJ\_wdM,只説“據報導”;就連Google都還告訴讀者訊息的來源,國家級的媒體卻拿匿名的小道消息來大作文章。
現在我給讀者們出個練習題,看看以下兩個可能解釋裏,哪一個比較邏輯自洽:1)Abramovich有超人的金剛不壞之身,加超人的超音速飛行能力,加超人電影導演安排串聯重要事件連續發生的上帝地位;或者2)Abramovich不甘心放棄在英國的大筆資產,請當地的財閥政客“朋友”們幫忙扭轉民意,後者把洗刷公共形象的任務交給MI6和BBC辦理。
西方和附庸國的普羅大衆願意對英國媒體照單全收倒也罷了,中國的公共論壇從上到下依舊樂意撒佈這些明顯的謊言,我認爲有檢討反省的必要,至少當編輯的必須要求基本的真假分辨能力;參見前一條留言回復的必修課程建議。
\subsection*{2022-03-24 16:55}

當前國際儲備貨幣、管理機制、同盟結構都在迅速轉變、重組的過程中,美國本身則面臨通貨膨脹、經濟泡沫、以及多維的政治内鬥等等不確定性,的確把種種黑天鵝事件的機率從基本為零提升到不可忽略的地步。但是你所作的分析,假設臺海衝突的時程由蔡英文決定,這必然依舊是錯誤的:蔡所能做的,最多最多,只是在美國要求她打擦邊球的時候,在越界多遠的斟酌上,稍稍可以有點影響,要說主動選擇戰爭爆發時間點,是你太高估台灣的分量了。至於打游擊戰,烏克蘭的Neo Nazi勢力龐大,幾萬人被安插到各級部隊做政治基幹,寧可戰死也不願被俘;台灣有對應的死硬分子嗎?
臺海戰爭打起來之後,當然是昂撒集團聯合所有願意聽話的附庸對中國做全面制裁,這是博客從八年前一開始就解釋清楚的邏輯;這次對俄制裁的經驗,表明了將來會包括扣押外匯、取消欠債、沒收中國企業海外資產等等選項。然而在軍事上也有動手的可能,才是我最擔心的事。這裏我不是指派地面部隊到台灣參戰、或者設立禁飛區,那純粹是去送死,而是考慮戰略層面上中國和俄國的兩大不同:首先中國是昂撒霸權的立即和首要威脅,俄方不是;其次中國的核武力量比美俄小了不止一個數量級。所以在中方完成占領之後,美國會有動力和底氣來繼續升級。這裏有Not mutually exclusive的兩條路綫:首先可以對中國貿易做全球海上封鎖,欺負中方欠缺海外基地的弱點;其次是逐次升級挑釁(例如故意讓自己的航母被擊沉,以挑動國内民意)、結合作假栽贓(例如在臺北或東京引爆核彈),最終得以在西方暴民背書下對中國做全面核打擊,徹底抹除中華民族在地球上的存在。這篇正文在金融和外交的正題之外,特別去談核武備,你想是爲什麽呢?
歐盟在Merkel退休之後,全面倒向昂撒集團,雖然事先看是小機率事件,但這個機率並不是零;換句話説,雖然可以不預期,但必須做準備。這是爲什麽多年來我反復强調要把握Merkel的務實態度,加緊綁定中歐關係。但是歐洲徹底淪爲美國附庸的邏輯後果,推論到底,也同時會使核戰成爲可能,所以我也明確論斷過,加緊提升核反擊能力是保障中美霸權和平交替的絕對必要。這種需要至少十年全力以赴的努力方向,中方卻一直拖到去年才開始著手,這很顯然是無用的智庫和學術界又一次清談誤國的案例,我們只能希望國家足夠幸運,不必面對最嚴重的惡果。
\subsection*{2022-03-22 09:40}

當年陳水扁爲了炒弄選舉而公開搞臺獨的時候,美國還有意願踩刹車;現在的美國執政團隊,以及他們所操弄的第二綫炮灰如德、日、韓,都已經把刹車系統拆下來扔進垃圾堆,如果蔡英文不趕快長出一點戰略智慧,在面對戰爭懸崖時依舊只聽命於主子的口令,那麽軍事升級就成爲無可避免的選項。然而當前英美集團本身内部問題極度嚴重(參見過去幾年的【國際】和【戰略】文章,包括這篇正文和尤其是【後注二】;這些因素可能導致理性的自顧不暇,也可能導致非理性的狗急跳墻),俄國在烏克蘭也不尋求速戰速決,未來幾個月的歷史發展方向有很大的不確定性;不過在俄烏衝突期間另外向中國挑起嚴重事端,違反英美行爲模式,發生機率並不高(亦即即使栽贓中國軍援俄方,也只是爲了施加宣傳和外交上的壓力,並不是鬥爭的主軸),真正要造假來强迫升級對抗,可能還是會如【後注一】所討論的那樣針對俄軍以“化學武器”為藉口。
\subsection*{2022-03-12 07:13}

中國國内長期的難關是學術改革,短期則是整合先進工業技術(例如半導體和精密機械)研發升級;照理後者是舉國體制合力辦大事的理想題目,似乎是中央重視之後就不會有太大的困難,但過去兩年半導體方面依舊是一團亂,讓人不得不擔心。
在對外方面,我想若是層峰能嚴格要求、明確授權,還是有一點希望的;畢竟外交部的人員水平優於中宣部,年輕一輩之中應該有可堪大任的人才,關鍵只在於必須先把5、60嵗過度自卑的官僚邊緣化。
\subsection*{2022-03-12 00:57}

這次錄音失敗,我還在尋找原因;先向大家致歉。不過我説的是埃及有一點點可能性會和Ethiopia打起來,不是以色列。
我已經説過,Putin這次的大戰略以政治解決為方針,軍事行動的頭號任務是殲滅亞速營。目前後者被包圍在Mariupol,由東烏民團和車臣部隊負責攻堅,這也是當前俄軍唯一努力攻堅的攻擊方向。我原本估計的三周,還剩下一周時間,應該足以完結戰鬥。其後我預期俄軍行動會以運動爲主,占據戰略要點,對城市圍而不攻,等待烏方投降。
至於問題3,要求我評論爲什麽網絡言論那麽笨,違反《讀者須知》第六條規則,嚴重警告一次,再犯拉黑。
問題4的答案,其實也在正文裏:他們敢怒不敢言,只待有人帶頭,做出沉默但實質的抵抗。
\subsection*{2022-03-11 23:13}

你還記得博客已經倡議好幾年,要拉穩歐盟、孤立昂撒的大戰略嗎?當時雖然Merkel理性務實、Trump得罪盟友,歐盟爲美站隊的機率爲零,但我還是未雨綢繆,説過應該趁早讓利,簽訂投資協約,徹底綁定中歐關係;那個邏輯就是Merkel、Trump總會退休下臺,届時會有危險,即使危險看來不大,買個保險是值得的。結果中方斤斤計較,到Trump敗選才急著出手,沒來得及辦成;所以這次歐盟放棄中立,雖然似乎是所有人始料未及的黑天鵝,但我絕對是曾經考慮過這個可能性,並且做出針對性建議的,只不過是被中國學術、智庫界的噪音淹沒而沒有被采納實踐罷了。現在亡羊補牢,工程浩大,但我至少把方略講清楚,其他人看不看得懂、能不能做到,非我所能强制的。
\section*{【學術管理】中國的學術管理問題來自基本的邏輯謬誤}
\subsection*{2023-04-10 00:38}

“精英”這個來自“Elite”的翻譯,因爲英文本身就已經將兩個看來類似、實際並無邏輯相干的意義籠統涵括,導致中文也跟著思路混肴,在實用上有很大的危害,值得反復解釋澄清。這裏的類似共通點在於它汎指凌駕於大衆之上的少數異類,但顯然有兩個不同的標準:一個是財富權力層面,是“平民”的反義詞,博客一般用“政商精英”或“權力階級”來描述;另一個則是思想道德層面,是“庸俗”的反義詞,博客常用的“高級知識分子”和儒家所説的“士”,指的都是思想精英。當然這裏也有潛在的歧義:受過最高等教育的思想界、學術界大佬也可能道德低下、思想齷齪,但並沒有必要另開一路定義,可以直接把他們視爲知識精英中的敗類。
請注意,頭一類精英是政治方面的,第二類則是道德方面的;既然公益最大化同是政治和道德的最終目標,他們自然也分享同樣的公益責任,只不過實際行爲上的標準要求有所不同。把兩類精英混肴的惡果之一,就是可以簡單拿私德標準來抹黑政治方面的公共人物;例如Trump的私德固然極爲低下,政策選擇也極度功利,但NeoCon窮兵黷武、對全世界竭澤而漁,才是政治精英中最惡劣的。相對的,我雖是一介平民,但同時作爲像樣的高級知識分子,那麽當然必須在個人功利問題上有正確的態度;與此同時,博客做政策建議的時候,卻也必須優先遵循功利邏輯。這裏之所以能兩者兼顧,分割綫在於個人功利和公益功利的差別。事實上,正因爲這兩種功利常有直接矛盾,要能邏輯自洽一致,反而必須兼有非功利和極度功利的心理。這樣既細緻又突兀的邏輯心態轉換,顯然不是感性思路所能及,因而是理性思維和邏輯辯證應用在公共事務的重要性的基本來源之一,參考我説過的,昂撒體系爲了反理性而走上普選公投路綫的經驗。
這篇正文的核心論述,在於學術權力和權威不能混肴,在體制和管理上必須做出明確的分割。前述關於精英的兩種定義,是從另一個角度來看同一個道理的表述。此外,有一些望文生義的傻子可能拿“精英主義”的帽子給我帶;鄙視平民的“精英主義”當然不好,但我主張的是反對庸俗的精英主義,把這兩者混爲一談,是既蠢又壞的表現。
\subsection*{2023-04-06 07:52}

先評論一下現代中國的“功利”思維。所謂的文化,其實就是對真、善、美的追求,而金錢回報正是對一切優秀文化的最大腐蝕力量。我以前解釋過,科學的真諦在於求真,而藝術則在於求美;後者其實受衆遠遠更廣,一般人也能有普遍的接觸和體會,所以更適合拿來印證功利主義的惡劣影響。例如美國和韓國的影視產品,原本就高度商業化,但過去偶爾仍然有製作者願意投入足夠的愛心和精力,其產出自然就能額外地觸動觀衆,兼得口碑和票房,例如Cameron的若干早期作品;反之,如果是純粹100 \% 追求金錢報酬的最大化,像是迪斯尼化的漫威和星戰電影,即便是舊有的粉絲也會從疲勞最終倒胃口。
以上正是我個人早已不再看任何好萊塢電影的原因;過去幾年我若是想要看Fiction視頻,唯一還能感覺到創作者對作品有熱愛、對自己和觀衆有尊重的,反而是日本的動漫;而日本動漫創作,恰恰是影視界出名的報酬特別低、工作特別累的行業,凴的正是一代又一代Otaku文化培養出來的激情愛好者犧牲奉獻。然而即使是這裏,也因爲全球市場的逐漸開拓,製作公司為新市場而批量生產,很快就超越激情愛好者所能支持的輸出級別,因而平均品質開始明顯下滑。
我也曾因爲民族感情而試圖接觸現代中國的電影和電視產品,但感想和對台灣新聞媒體一樣,都是令人作嘔。唯一能熬上超過10分鐘的例外,是幾年前的一部動漫《全職高手》(我猜測動漫在中國還是小衆,仍然有容納個人理想的一點空間)。換句話說,中國藝術文化產品與迪斯尼極爲類似,但整體水準還要更低,而這裏反映的,顯然並不是什麽歷史文化、政治審查或個人創造力上的差別,而在於行業機制的極端功利化。中國男足之所以越整越爛,也正是出於同一個基本邏輯謬誤,以爲報酬越高、表現就應該越好,其實剛好相反:越是功利,則越鼓勵劣幣驅逐良幣。古今中外,金錢的激勵效應,向來只限於普羅大衆的經濟活動;真正頂尖的追求,不論是科學、文化、藝術、體育、政治或軍事,最優秀的人才都只能基於理想和激情(請注意,這句話的反向敘述並不成立,參考那個公開懟楊先生說,爲了資助她個人愛好,國家應該浪費千億美元來建大對撞機的研究生),而不是重金獎勵。
所以一個自稱是社會主義的國家,反而接受將資本主義無限上綱之後的絕對邏輯,一切以金錢至上,不但是極大的反諷,也是自我傷害的愚昧典型,難怪中國最大的問題、最落後的層面,正是基礎科研、學術、文化、媒體、藝術、教育等等這些思想上必須求真、求善、求美的領域。不過讀者先不要義憤填膺;博客自創立以來,一直堅持免費閲讀,更不斷拒絕創收的點子,爲的就是一方面避免誇大、扭曲或灌水的誘惑,另一方面示範中國士人應有的為公益理想而奮鬥的典型。然而中國社會的功利主義是如此的普遍而深入,群衆在痛駡國足的同時,卻不假思索地忽略博客多年的努力和犧牲,不斷拿其他追求流量和收入的網紅“學者”、“專家”來和我做比較,這不但是邏輯上的自我矛盾,也是對我和中國傳統文化的侮辱。換句話説,請大家捫心自問,如果沒有深刻感到博客非功利態度的可貴,那麽自己就也是問題的來源之一,沒有資格批評別人,包括國足在内。

回歸你的主題。
那篇《社會主義國家應該如何管理資本》雖然對很多重要事項做了深刻的解析和建議,但唯獨對功利主義的危害、學術管理的正道、科研人員的腐化沒有做任何涉獵;結果文章討論過的改革現在都一一實現了,就是教育、學術和文化方面毫無動靜。我猜測你是注意到這一點,所以特別針對這個話題來做討論。其實我自己也是憂心如焚,但正如我反復解釋過的,此事的專業壁壘過高,很難用一篇科普文章説服非專業者。大家群策群力,共同支持行内的良心人發聲,是我目前僅有的建議。
\subsection*{2022-08-31 17:08}

我每天殫精竭慮思索還有什麽可進言之事,教育學術的問題總是那座眼前的大山。然而敦促改革,必須所有的良心人一起盡力;如果指望我寫一篇像《社會主義國家應該如何管理資本》的總結文章,一次性地訂立新的認知綱領,那必然要失望。這裏的差別在於學閥有專業壁壘,可以簡單地用權威和術語來製造烟幕、混肴視聽。最高層在政治經濟學上原本就有相當的修養,對高能物理、核物理、量子物理、氫化工等等卻不可能獨立看穿專家集團衆口一詞的忽悠;所以只有聚沙成塔,由大衆輿論的不斷高調質疑引發行内良心人的實話,才可能揭開國王新衣的真相,例如《量子通信和計算是中國學術管理的頭號誤區:後注五》所談的那篇《FT》文章可能很快會有中譯版發表。
\subsection*{2022-08-30 21:38}

博客多年來反復强調,教育管理、學術管理和科技投資管理是當前中國内政最糟糕的重災區。這裏又分專業文化腐敗、戰略選擇錯誤和戰術執行不力三級問題,其中前兩者的危害遠遠更為重大,而且互爲因果,案例也俯拾皆是,如大對撞機、量子通信、量子計算、氫能、核聚變發電等等,都是在無形之中大幅虛耗國家的財力、物力、人力和公信力。即使是半導體產業政策,真正導致整體低效的錯誤,也在於放任地方自由招商的戰略決定,而不是個別管理人員的貪腐。
這裏你提起的教育、體育方面的政策失敗,也同樣來自專業文化的腐朽和戰略原則的失誤。然而每次中央出手整治,著眼點依舊是戰術上的貪腐問題,我認爲這是很不好的現象:一方面反腐本身有極高的重要性,不應該拿來作爲整治其他錯誤的藉口;另一方面,專業文化和管理戰略也需要針對性的反思和改革,不能只抓有明確金錢易手的案子。最近我的知名度上來了,在戰略方面的建議開始得到關注和采納,但是對學術文化腐爛的批評依然不見任何行動,是我日常憂心的事。
\subsection*{2022-08-25 12:31}

雖然幾年前討論過,這裏再簡單復述一次。美國的研究生教育分爲兩類:學術性和職業性,後者又分爲初始資格和在職進修。初始資格可以是博士(Doctorate),例如醫學院和法學院,但在職進修就必須只是碩士學位,這裏最知名的是商學院的企管碩士MBA,但教育學院也有MAT(Master of Arts in Teaching),對比著學術性的ME(Master of Education)。然後就是像Kennedy School這樣的特例,專門為國際政要做在職進修,同樣也是給碩士;博士也有,但那是為培育親美智庫的另一套課程,算是學術性的學位。這個體系背後的邏輯脈絡,我以前也解釋過,亦即學士教育的意義在於建立基本的Intellect,碩士則是專業知識的灌輸,博士代表著獨立執業的能力;既然在職進修的對象已經“獨立執業”過了,學校當然只能提供專業知識的復習和補充。
中國教育界拼命學美國,怎麽學的都是一堆糟粕,好的卻沒學到呢?
美國商學院教授的薪水,一般是同級文理學院的4-5倍,比當年芝加哥大學經濟系開出的3倍薪水還要誇張,那麽誰會想去揭穿財閥害國自肥的話語騙術呢?
\subsection*{2022-08-24 06:53}

這其實和學術管理的腐敗沒有直接關係,而是權力階級腐化後必有的現象:Commodus自認角鬥士冠軍、明武宗自封威武大將軍、Brezhnev不停地給自己頒勛。近代中國政客拿學位來自慰的傳統始自國民黨、歷久不衰,危害有目共睹;共產黨也淪落至此,實在是一大諷刺。至於改革的方法,幾年前博客已經討論過了(當時我以哈佛的甘迺迪學院爲例):如同商學院和教育學院為企業經理和學校教師提供在職進修,政治學院也可以專爲政治主管設立合適的學位,但是(1)必須和學術性學位有明顯分辨;(2)必須是一至兩年的政治管理碩士,而不是博士或其他專業;(3)必須從行政崗位上暫退(Sabbatical),全職(Fulltime)學習。
\subsection*{2022-08-17 01:39}

你説的,正是我讀到那條消息時,心裏的恐懼。一個簡單的中美戰略,尚且花了我五年時間,才扭轉社會共識到有實際政策轉向;這個學術管理問題,只怕我得鞠躬盡瘁、囉嗦到死而後已了。
行内人都知道,中國式的考試文化,導致不論用什麽指標,都會引發資源和精力被投入美化指標、而不是實質改進,尤其發學術論文的,正是這個體系中幾千次嚴格淘汰選拔出來、特別擅長把分數上推的考霸,所以現象遠遠更爲嚴重明顯。至於實際上是怎麽把論文引用數搞上去,除了偶爾造假之外,更為普遍有效的手段,是追逐無意義的熱門題目,也就是博客反復撰文批評的Ambulance Chasing,對科研學術界不熟的讀者可以復習參考。

有朋友來信指出,上面我所解釋的,官用指標會導致浮面扭曲、最終不但失效而且會有反作用,是經濟學裏的已知原理,叫做Goodhart's Law,一般被用來討論GDP對經濟政策的腐蝕效應,有興趣的讀者參見下面的網頁:
https://en.wikipedia.org/wiki/Goodhart \% 27s\_law
\subsection*{2022-06-07 23:34}

這個問題我們幾年前討論過了,我依舊認爲解決方案在於管理階層把打擊造假作弊提升為教育的最高優先考慮之一(另一個優先是保障階級公平)。要求底層教師執行紀律當然很難一步做到,但一旦問題暴露,各級繼續和稀泥就是明顯的瀆職;尤其必須從上層做起,院士和教授被抓包,不但應該馬上開除,而且有必要追究詐騙公款的刑責。
移風易俗從來都是艱巨的工作,但正因爲如此,在關鍵重點民風上能扭轉積弊的國家,自然享有國際競爭的優勢,容易保證長久的興盛。我以前舉過Prussia這個例子:19世紀前半德國工人階級的心態其實和現在的斯拉夫人很類似,是後來Bismarck花了一兩代人的時間才建立了負責、專業、細心、嚴謹的工業文化。
\subsection*{2022-05-29 18:16}

我認同你的分析。博客日常批評外交戰略、金融政策和學術管理問題,主要是因爲這些議題專業性很高,我既然是有相關知識和瞭解的極少數人之一,就應該把傳播的頻寬專注在這些方向。然而思想和教育上的問題,其實更為廣汎而基本,但這是許多有良心的知識分子都可以簡單看出的事,所以我以往只是點到爲止,真正追溯到底、貫徹改革,是大家的共同責任。
至於人性的分佈,我想三年前談香港的時候已經給過解答,這裏只簡略重述一次:一般可以假設人群有20 \% 的好人(亦即對損人利己的誘惑有相當抵抗力)、20 \% 的壞人(不論外部環境如何,總會想法損人利己,甚至損人不利己)、和60 \% 的普通人。建立健康的文化,在於提拔好人、抑制壞人、然後通過這些獎懲機制來影響普通人,讓他們做出和好人一樣的選擇。
\subsection*{2022-01-23 04:36}

GaN的效率和切換速度都比SiC更高,但是爲了高頻輻射(例如AESA雷達和通信)開發的既有GaN是所謂的Lateral GaN,亦即只在Si或SiC晶片上長很薄一層GaN,結果在電壓(目前只能做到650V)、熱性質和壽命上都完全不能滿足下一代電動車的需求。例如進一步提升充電速度和電路效率最簡單的辦法是提高電壓,BYD剛宣佈用800V的新車,有歐洲設計準備用900V,但是到2025年絕對會有1-2kV的要求,雖然可以用SiC凑合,最好的解決方案還是Vertical GaN,亦即整個晶片都是GaN。參見https://www.eetimes.com/vertical-gan-devices-the-next-generation-of-power-electronics/\#
\subsection*{2021-11-18 01:05}

我並沒有什麽特殊管道,所以這類有關腐敗、壟斷的内幕消息一般無法置評,不過本周看到《經濟學人》上的一篇文章(參見《Attack on the Tycoons》,https://www.economist.com/finance-and-economics/china-attempts-to-clean-up-its-sleaziest-regional-banks/21806193)有所感觸:中國改開40年,全面引進市場經濟卻沒有配套的監管系統的結果,是重蹈19世紀英美的覆轍,朝向Gilded Age演進。美國能夠慢慢吞吞、用20世紀的前2/3來對自私的資本做出若干限制,是因爲擁有霸權,沒有太大的外來危險。中國在崛起階段,現任霸主已經在全力打壓,然而學術、金融、商業管理上都是一塌糊塗,欠債極深,不改不行。習近平的改革速度已經是超過我以往認知的人力極限了;上面那篇文章討論的是全國幾千家地方金融機構被私有資本侵占成爲Piggy Bank,這似乎是現在整頓的重點方向,我完全同意應該是最優先,所以大家稍安勿躁,有什麽不合理的問題提出來公開討論、以便提醒執政階層是應該的,但政府只能一步一步來也是我們必須體諒的現實。
\subsection*{2021-09-13 11:28}

歐美過去300年的世界霸權,其實是建立在Age of Enlightenment爲了推翻舊有宗教思想桎梏而推行的理性思維和科學文化之上,但到了20世紀,英美爲了對内對外忽悠,把功勞偷來往自己的政治、經濟體制上貼金,反而成爲新的宗教(政治方面成爲白左民主教,經濟方面成爲私有市場教),才害人害己,造成過去半個世紀的迅速衰落。
日本、台灣和中國的經濟崛起,也同樣是建立在理性政策之上,但前二者已經跟隨英美墮落,中國有必要深刻檢討,認清治國的核心方針(亦即理性和科學),徹底排除外來和自發的各種不切實際、不合邏輯的制度和習慣,才有長治久安的可能。我若是能教育幾千名未來的精英,他們又各自去影響幾百個身邊的人,那麽或許可以讓理性思維在中國更加普及深入,最終延長國祚幾十年,這裏的潛在貢獻的確有可能比政策上的直接建言更大,所以我一直兼顧著博客作爲培訓班的任務。
\subsection*{2021-09-12 13:56}

過去20年新設的這些由Billionaire資助的科學獎項,無分中外,都是典型的現代美式資本運作:名義上是科學,實際上是行銷。一開始行銷的是幕後金主的行内朋友的職業聲望,最終如果有需要,也可以演變成金主自己的商業炒作。這裏我最痛恨的是Breakthrough Prize in Fundamental Physics,幕後金主是一個原本學超弦、後來轉金融的私募基金大亨,所以這個獎實際上就是超弦獎。
ACM相對好多了,畢竟計算是個工程議題,不可能100 \% 靠詐欺。但“不是100 \% 詐欺”並不代表“100 \% 不是詐欺”;遇到量子計算這種整個行業都在騙,而且還有好幾個超大型企業做後盾的東西,總還是有走上前臺的時候,届時就必須挑一個最玲瓏八面的人來得獎,Aaronson於是脫穎而出。我説過了,中方的科研投資完全沒有經過理性的論證,英美財閥的公關宣傳被當成發展藍圖,這裏正是細節示範。
風雲之聲當年不也是支持過王貽芳嗎?這種不談科學、不談事實、不談邏輯,直接講權威的論點,而且用的還是繞了幾個彎才能搭上邊的權威關係,不很明顯地是在轉換話題(尤其是要避免回答“八個數量級”這個質疑;這裏的關鍵在於他們反復吹噓的,始終是在特定題目上比傳統計算機強多少,這本身就是一個僞命題,真正的議題是什麽時候有用)做狡辯嗎?他們純粹就是既得利益者的傳聲筒、國家民族的吸血蟲。
\subsection*{2021-09-10 14:54}

我也希望大内自有高人,但是十四五計劃白紙黑字寫著要重點投入量子計算和核聚變,這些高人顯然都去睡大覺了?其實這裏的道理我也反復解釋過了:不論有多麽聰明,都無法突破信息不對稱;我有些特定的經歷和視野,是中國政府的幕僚所不可能擁有的,例如在美國生活30多年,見過學術和金融高層的内部運作機制。
學術管理的改革,治標不治本的方案有無限多,連我以前都被迫只談打假,那篇《從假大空談學術管理》雖然傳述者眾,也可能是最近改革聲浪的起因之一,但並沒有明白解釋必須剝奪學術大佬的政治權位。一旦内部開始討論方案細節,既得利益者必然會想要棄車保帥,現在把握時機,把道理解釋清楚,事關中國未來3、40年的國運,怎麽不重要?
\subsection*{2021-09-08 09:02}

我已經解釋過了,將專業權威和政治權力集於一身,立刻使利益糾葛(Conflict of interests)無可解開。管理專業議題,當然不是容易的事,但不論怎麽做,都沒有放任行内人自我評審來得糟糕;而當前中共的體制,不但讓學閥獨霸專業議題,而且他們可以通過一般官僚系統來操控普通輿論和金融市場,這已經不只是學術管理的問題了。
正確的學術管理,必須建立一批專門培訓出來,有若干行業知識,但毫無利益人情糾葛的職業學術管理人才;例如用數學系出身的去管高能物理,用物理系出身的去管化學,化學系出身的去管生醫。
至於你說的騙保問題,在美國也有啊。所以保險公司要求補牙之前要拍X光存證。這類的防弊手段,只要想做,總是有辦法;中共體制的毛病在於,不但學術界自己不會想防弊,而且外人想檢舉都無門可入。
\subsection*{2021-09-08 03:29}

我以前總是說要從嚴格打假做起,其實那只是最基本、最簡單、最初步、最無爭議的治標步驟(當然,現今的中國學術管理,連這也做不到)。要根治這個頑疾,必須如正文所强調,把專業權威和政治權力切割開來,然後才可能以事實和邏輯,通過公開辯論來挑戰前者(別忘了,中國的政治權力是不容公開挑戰的),一旦有了科學和理性的背書,自然可以獲得政治權力的尊重。
四年前,我發文(參見《談悟空衛星》)批評悟空衛星團隊的虛僞宣傳,《觀察者網》立刻受到政治壓力,雖然勉强發表,後來對我的其他文章有了很大的戒心,以致當我試圖把《美國陷阱》介紹給華語世界的時候(參見《域外管轄權》),《觀網》原本是拒絕轉載的。這就是後注裏所提的,“被封殺的危險”。悟空衛星還沒有什麽政治能量可言呢。真正可怕的,是官居中央委員,把量子計算這個假未來科技,搞成十四五計劃中,科研的頭號重點;你說,只要他還是中央委員,中國怎麽可能對量子計算的發展前景做出客觀、科學的評估?中國人講人情固然是一個問題,但如果政治體制根本就不容許你質疑學閥,討論文化因素毫無意義。
我覺得很可笑的,是量子計算被吹捧為戰略性的頭號未來科技,但卻從沒有經過嚴謹的評估論證(否則以8個數量級的技術鴻溝,是不可能一路開綠燈的,參見《從假大空談新時代的學術管理》),完全就是中方對美國企業(Google和IBM)為炒作資本所發的公關稿,照單全收,拿來當成科技發展藍圖(氫經濟也是如此,大對撞機和核聚變則是拿劉慈欣的幻想小説來當藍圖);既然有機可乘,自然有人順水推舟,名、利、權三收。這裏的關鍵是他的政治地位,如果不改,量子計算絕對會持續做爲無底錢坑直到他退休爲止,任何學術管理上的改革也無從談起。
\section*{【國際】再談Biden任期内的中美博弈等議題}
\subsection*{2023-04-01 15:57}

這件事有點複雜(雖然要做正確分析所需的前提伏筆,博客都討論過了),以致華語世界沒有一個人摸得着邊。我原本想等等,到四月中上《龍行天下》節目再詳細討論,以測試公共論壇網紅和中國學術界自我獨立分析(亦即沒有博客評論做“參考”)的能力,結果果然慘不忍睹:多數避之唯恐不及,少數硬上的則純粹自由聯想、胡説八道。現在既然你提起話題,就稍作解釋;不過後續發問,請放在《歐盟内部的無色革命》一文之下,理由見下文。
首先回顧一下幾年前博客留言欄曾討論過的以色列人口結構問題:立國之初,首批猶太移民主要來自歐美的中產階級知識分子,但後來他們的生育率很快跌破2.1的平衡值,新移民卻以蘇聯農民和猶太原教旨主義者爲主,後者的生育率至今仍然超過6,比囘教徒的4還要高得多。幾十年下來,除了國會議員普選能部分反映人口結構的重大變更之外,政治社會體系中的其他權力機構,仍然牢牢地掌握在受過大學教育的白左自由主義分子手中。換句話說,以色列猶太人社會内部,有非常類似美國白人的嚴重撕裂(亦即都依大學教育為分水嶺),而且既得利益權力階級都是Liberal,差別只在於:1)美國白左是1960年代之後的新興政治思潮,以色列的自由派則從1940年代就是絕對主流;2)美國的長期人口結構演變趨勢,有利於白左,以色列則相反。
在所有前述的白左權力機構之中,遠遠最重要的是司法系統,尤其是最高法院。以色列並沒有憲法,只有一個很含糊的基本法,原本就高度容許自由心證,再加上規則細節特別方便各級法院“糾正”國會立法和政府行政,以致他們成爲實質上無可制約的太上皇。偏偏法官(包含但不限最高法院)的提名和任命還不像美國那樣分別由行政和立法系統主導,基本是司法體系内部自選,即便是最有耐心的政客想要慢慢逐步變革,也無從著手。
Netanyahu自1990年代崛起,成爲以色列政壇的常青樹,所代表的就是右翼民粹勢力。在這一點上,他類似意大利的Berlusconi、匈牙利的Orban、土耳其的Erdogan、以及印度的Modi;此外,如同Berlusconi,他在道德操守上,也有著相當明顯的彈性。我在《歐盟内部的無色革命》一文中,曾詳細解釋國際白左集團制約右翼民粹政客的第一道防綫,就是以貪腐為藉口的司法打擊,明顯的成功案例有捷克和奧地利,於是這些人即便只爲了自保,也必須反過來設法做“司法改革”,尤其是由政府指派最高法院法官(請注意,這其實正是美國施行200多年的體制),例如波蘭和匈牙利,然後白左集團的反反擊,也就自然是把“干涉司法”升級為罪大惡極的“反民主”暴行,任何剝奪“司法獨立”的變法,都必須絕對阻止,即便有人口結構引發的正當理由,以及美國憲法的範例,也在所不惜。
所以這次用顔色革命的手段來搞無色革命,核心本質是因爲Netanyahu的司法改革威脅了國際白左集團的絕對權威,是國家主權和霸權勢力的對立。Netanyahu本身原本屬於“不是自己人,但可以容忍”的範疇,是因爲膽敢挑戰司法體系,才引發打擊;參見最近《龍行天下》節目中,有關蘇格蘭Nicola Sturgeon的討論:Sturgeon只要不直接抗拒司法體系,就逃不出如來佛的手掌心,也就無關痛癢,可以被容許搞蘇獨。讀者可以順便對照《社會主義國家應該如何管理資本》一文中提過的道理:所謂“三權分立”和“權力制衡”,都是爲了弱化國家機器的行政部門,以方便資本在幕後制約反抗力量的體制選擇。
至於國際戰略博弈,在此次事件中頂多只算非常次要的間接因素,畢竟有獨立的外交政策考慮,是右翼民粹的共同特徵。Netanyahu當然比白左走狗更可能做出背叛霸權集團的選擇(參考上月我在評論伊沙和解時所提到的,客觀上以色列的最優解成爲加入上合和金磚),但這還沒有發生,不是無色革命的直接原因。霸權集團的打擊,針對任何能解脫絕對控制的反抗;例如波蘭和匈牙利,一個選擇反俄、一個選擇親俄,但違反“司法獨立”的行爲,一樣是不被容許的。

Netanyahu在司法改革上吃癟之後,面臨被打倒抄家的危險,於是很自然地下令警察衝擊巴勒斯坦清真寺,以挑起新一輪的民族衝突,藉以鞏固自身的政治支持。如果在國内搞事不足以解決問題,下一步就是對外做軍事打擊;這值得中方的外交單位嚴密關注並預做準備。
\subsection*{2021-09-28 05:19}

還沒有100 \% 跛鴨化,但已經大約95 \% 了,剩下的5 \% 機率對應著目前卡在國會的財政刺激法案和警察改革法案,它們還有一點機會起死回生。Biden必須在明年期中選舉之前力挽狂瀾,否則國會多數一旦喪失,跛鴨進一步成爲死鴨。
Biden政治地位和權力的衰弱,有相當機率會導致中美關係暫時穩定化,美國内部民意較不在乎的制裁可以鬆綁,例如關稅,這不只是“象徵性”的,所以外交斡旋仍然值得努力,但“原則性”議題很難動得了,例如香港、新疆和科技禁運;台灣問題應該會被美方試圖擱置,但蔡英文是否會模仿陳水扁主動鬧事,則無法理性預測。與此同時,因爲Merkel的退休,東歐國家有可能趁機鼓動歐俄衝突升級,反而成爲未來一年的地緣戰略爭議熱點。
\subsection*{2021-09-25 08:20}

在回答你的問題之前,我們先復習一下AUKUS事件。整個核潛艇交易最關鍵的疑問,在於他們原先計劃的主要技術來源是英國還是美國。當然現在還沒有簽約,所以並沒有確定的方案,但三方首腦會談了幾個月,總是要有個大致的概念,而這個概念中,核潛艇的設計製造由英國還是美國主導,對瞭解幕後三方的戰略動機和心理進程,非常有幫助。
我想了想,覺得答案應該是英國。原因如下:英國核潛艇有重點部件和技術依賴美方,反之則不然;如果是美國主導,那麽根本不干英國什麽事,完全沒有必要把後者拉進來,反之如果是Johnson的主意,那麽他沒有選擇,必須要求美國參與。
如果Johnson真的是發起人,那麽很好解釋爲什麽核潛艇交易升級為新同盟,以及爲什麽AUKUS事先對法國絕對保密:這是因爲Johnson的首要執政目標,始終是討好脫歐支持者:一個新的三國同盟當然可以對國内的無知選民大肆吹噓;別忘了,英國反對派對脫歐的批評,正包括了外交孤立這一點。另一方面,從法國手中把訂單搶下來,軍工收入反而是次要的,最重要的是能對國内聽衆宣傳,說獲得了一個羞辱歐盟的大勝利;現在英國面臨食物、汽油和電的短缺,脫歐派媒體想要為Johnson開脫而不可得,能夠轉換話題對他們是極度渴求的。正因爲羞辱法國才是重點,所以不但不先安撫,而且要保密到最後一分鐘,讓Macron從新聞報導中得到消息。
不過這麽一來,法國得理不饒人,歐盟自然也站出來支持。Macron拿剛剛脫歐的英國沒什麽辦法,連召回大使都嬾得做,但可以擱置歐澳自貿協定的談判,更重要的是推進建立歐盟自主部隊。後者是Macron從上任就不斷鼓吹的政策,但因爲它顯然是廢除北約的前奏,所以歐盟中附和者寡,美國更是多方阻撓。昨天Macron和Biden電話長談,很可能開出的條件就是美方必須默許。
Biden在前天會見Johnson的過程中,始終保持一副肚子痛的表情,也可以簡單解釋了。他原本只是放行英澳核潛艇交易,但因爲沒有細思,莽撞地接受了Johnson的附帶建議,不但畫蛇添足,因爲成立新同盟而疏遠了其他盟國,而且把AUKUS對法國瞞著,捅出大婁子又是美國買單。難怪他對Johnson完全不待見,對英美自貿協議直接說NO,Johnson想要重談脫歐條約中的北愛條款,Biden簡單禁止,連接受記者發問都不願意。
我談了這麽多,想説的是,Biden在過去這幾周,内外交困(別忘了除了同盟動搖之外,還有放棄阿富汗、三萬億財政刺激法案卡在參議院、低迷的就業率,以及最重要的,他的民意支持率剛剛跌破Trump同期的最低紀錄),對華强硬的心態很難維持,所以雖然目前沒有太多直接證據,但對你的問題做比較樂觀的估算依舊是合理的。
\subsection*{2021-09-22 19:15}

表面上類似,實質上相反:Lehman Brothers是整個體系在極度狂歡、而執政者卻繼續煽風點火前提下頭一個爆破的倒霉鬼;恆大是整個體系在低度狂歡、因而政府決定主動刺破泡沫後頭一個爆掉的倒霉鬼。所以兩者雖然有相似之處,亦即2008年金融危機是美聯儲多年寬鬆政策和對行業作爲不聞不問的結果,2021年恆大破產則來自多年來(尤其起自2009年過度財政刺激所導致的)錯誤房地產管理政策,但後者的泡沫還沒有吹大到不能倒(Too big to fail)的地步,因而和美國相反,中方決策單位這次是主動出擊,為經濟的全面轉型做準備。
恆大的杠桿雖然極高,但它的投機性資產(主要是造車)佔少數,即使中國整體房地產價格下跌,只要能維持既有的融資速度(亦即不必像老鼠會那樣指數成長),它就不會倒閉。換句話説,恆大的危機,是個Liquidity Crisis而不是Solvency Crisis。依照美式經濟學理論,政府完全應該出手救助;相對的,Lehman則是典型的Solvency Crisis,所以美國財政部才會任其倒閉。恆大危機最驚人之處,在於政府不但不救,而且原本就是有意造成它破產的。這裏的證據很明確,中央在一年前出臺的“三條紅綫”,立刻打斷了恆大所有的融資來源,是這次危機的起因。如果政府不想讓恆大破產,早就可以和許家印交涉,以融資交換產權,並要求他逐步消化杠桿;結果中央選擇堅決打擊,連許在地方政府的關係戶也被嚴格禁止出手幫忙,恆大拖了一年,終於撐不下去。
既然在戰術上選擇犧牲,那麽中央必然是在戰略上有更重要的目標;這個戰略意圖也不難推測:是要為房地產市場消火,將經濟建設的資源,尤其是資金,轉投到實體產業(如半導體和電動車)之上;而且這些投資,必須是華爲式長期性的全神投入,而不是恆大造車那樣的短期投機;換句話說,這是“雙循環”裏建設内循環的那一部分。其實回顧過去這一年的諸多新政策,這個轉向是明確而且全面的,例如打擊互聯網公司,就是完全一樣的思路。在考慮過去四年的中美博弈之後,我們可以進一步推論,這是2019-2020年之間,高層汲取教訓,統籌規劃出來的一盤大棋,雖然和《中國製造2025》目的相同,但邏輯思路遠遠更深更基本。
所以恆大的破產,早在一年前就是定局。破產後的資產分配和企業重整,對中國經濟的資金配置,更有開創健康先例的意義。至於媒體的所謂什麽崩潰啊、什麽爆盤啊,大家就當成娛樂看吧,現代社會裏的新聞報導原本就是笑話佔絕大多數。在恆大事件中,中國政府從一開始就是幕後的編劇、導演兼製作,對未來的進程必然會繼續輕鬆主導,維持秩序更是不在話下。
\subsection*{2021-08-21 00:44}

這也是我幾年前就討論過的:一帶一路執行過程中,外交部只在乎業績任務,沒有深思熟慮,事先排除不合適的對象(例如Nigeria);對可行的國家,姿態也擺得太低,反過來求外人合作,一開始就沒有建立健康的平等互惠關係,第三世界普遍有著占便宜的心態,反而對一帶一路的長期發展,造成極大的阻力。巴基斯坦還算好的,雙方都有戰略需要,至少巴方政府絕對不會翻臉;既然已經投資下去,只好堅持到底。
至於阿富汗,我不是剛剛才反復强調“開國容易治國難”嗎?所以絕對不能假設Taliban有能力像1975年的越南那樣貫徹統一和集權,事實上連現在敘利亞的大體維持秩序都不一定能保證,也有可能淪爲另一個利比亞。中方應該靜觀其變,對Taliban處理東突稍作獎勵即可。
\subsection*{2021-07-18 08:42}

這次大選CDU的問題原本在於貪腐:去年新冠流行期間,一連兩個CDU的國會議員被曝出參與醫療用品采購招標的過程,綠黨因爲形象清新,所以大幅受益。然而過去兩三個月,輪到綠黨的女黨魁被曝出有瞞報收入和僞造履歷的案底,於是民調就反轉了。這兩天德國西部大雨引發洪水,可能因爲氣候變化再次成爲新聞焦點,會對綠黨有點幫助,但應該不會有像去年貪腐案那麽大的影響。
歐美的體制權力分散,非常低效,這是我談了幾千次的話題。不過德國和美國的情況也不完全一樣:美國分聯邦、州、和地方三級,互不統屬、叠床架屋;像是Miami樓房倒塌,地方根本無力處理,卻還是非要拖個幾天才向州政府求救,聯邦更加緩不濟急;但是至少那些國家級的行政力量是存在的(據説有兩個聯邦搜救隊的基地就在Miami車程一小時範圍内)。德國的聯邦政府則基本沒有什麽内政上的組織和資源,只負責協調(Coordination)和建議(Advisory),權力徹底下放到地方,不論是新冠防治,還是天災救助,都完全不是聯邦的職權,所以出了洪災,不管處理得再糟糕,也不會直接影響執政黨的選情。

話聲剛落,Laschet就犯了超低級的錯誤,在視察洪災的記者會上大肆談笑。雖然客觀來看,這不應該影響他是否適任,但在競選期間,卻是左右主觀民意的關鍵;直選體制下選拔人才的標準,類似網紅,那麽選上之後行政能力和效率也類似網紅,不是理所當然的嗎?
\subsection*{2021-06-10 05:26}

這是一個很複雜的問題,不過我個人認爲1)這些對中國仇視度較低的年輕一代人數依舊太少;2)他們的態度可能會隨新發展(例如臺海戰事)而改變;3)而且中美博弈在5-15年内結果就會明朗化,等他們成爲社會主流,勝負已分。
至於這個現象的來源,我想特別提出一點來討論:亦即大學教育的普及,使得有學士學位的美國成年人占比,從80年前的1/20,上升到現在高於1/3。其最重要的影響,是知識精英原本和政商精英高度重叠,但現在大部分的知識分子(廣義的,亦即受過高等教育)只是中產階級、甚至下中產階級。近年來美國兩黨政治路綫的重組和選民成分的分歧,基本是由這個機制所推動的。這些“Educated Poor”對美國國内的財富分配不公平、不合理,特別敏感,基本不相信“China stole our jobs”的右派托詞,所以在和平時期的民調中並不把中國視爲頭號敵人。然而他們也是白左教的天然信徒,所以前面我已經提到如果臺海、新疆、西藏、香港這類議題成爲新聞焦點,他們可能會比紅脖子更加瘋狂。
高等教育普及化也發生在大陸和台灣;它基本是爲了滿足先進經濟的人力需求,但對未來的政治和社會發展自然會有無意卻又深刻的影響,值得學術界預先探討。
\subsection*{2021-05-06 12:20}

她説的是常識啊。Yellen是學術界出身,改不掉清談的習慣;其實做了主管,就沒有這樣説話的自由,因爲聽衆搞不清楚那是客觀評論、還是主觀政策。
因爲美元是國際儲備貨幣,美國的經濟規則和管理,都是獨一無二的。全世界只有美國才能自己大印鈔票、通貨膨脹卻主要出現在其他國家。所有中央銀行中也只有美聯儲,才能無視任何國外因素,只考慮自己的經濟和貨幣,來決定收放的方向和程度。
美國經濟管理的基本問題,在於自20世紀中期開始,其宏觀經濟學受芝加哥學派影響,有意地忽略了外包風潮所帶來產業空心化的後果,實業被挖空的問題當然不能靠貨幣政策來解決,然而手裏的工具只剩下這一個(亦即美聯儲放水來資助超額的聯邦赤字)。它一方面極爲强力,另一方面卻完全無助於根治經濟的痼疾;換句話説,美聯儲的策略在短期内完全可以剋服所有其他因素,自由決定美國經濟的走向,然而長期來看,目前越是死撐、泡沫爆破的後果就越嚴重(除非又能轉嫁給歐盟和第三世界)。這個自主和宿命兼具的二元現象,使得觀察者很難做出準確的預言;我一直想要討論這個議題,但不確定性太大,遠超我平常寫作的標準,所以一直拖著。
\subsection*{2021-04-26 17:29}

你沒有讀懂我這幾年有關美元霸權的討論。
美國復蘇快於歐盟,正因爲前者印鈔票是後者速度的五倍;所以歐元獲得若干美元的額分,不但是歐盟經濟亟需的助益,對中國的大戰略,也是建立多極世界的重要步驟。
至於歐元對美元的競爭力,光看經濟表現當然不足,否則前者也不會在過去13年步步倒退。然而美國的泡沫越吹越大,已經進入指數成長的階段,就算美聯儲能填補SPAC這個坑,其他類似的浮誇資產不勝枚舉,我估計2-5年之間會終於壓不住。下一個金融危機應該就是美元的末日,但歐元區還是必須預做準備,尤其是避免投資美元資產,靜等作爲避風港,則自然會受益。
中國必須爭取歐洲,是國際形勢下的必要,不是因爲方便而順水推舟。在貿易全球化的背景下,國際話語權是有很大的實質影響的;中俄聯盟依舊是弱勢,只有歐洲中立,才可能對抗英美外交宣傳體系。
\subsection*{2021-04-25 23:27}

正文中已經强調,美元霸權是美歐之間最大的矛盾,更高於中歐之間爭奪高科技工業主導權的競爭態勢。中方一直沒有拿來利用,是非常不智的浪費。
至於如何利用美元來離間美歐的手段,我也已經明確列舉出來,亦即當前美國股市的SPAC狂熱。其實如果這篇博文討論的重點是金融,那麽雖然美國經濟有明顯的過熱現象,但正因爲美聯儲可以無限印鈔,所以泡沫爆破的具體導火綫和時間點都有很大的不確定性,SPAC只是其中的可能之一;畢竟三萬億雖多,但仍然在美聯儲可以遮掩的範圍之内。我特別把SPAC挑出來在正文中討論,是因爲它最明顯、最簡單、最易懂,非常適合中方拿來對德國人説事。在金融、貨幣的管理上,德國是全世界主要經濟體中最保守的,沒有之一,所以是這類離間法的天然受衆。
\subsection*{2021-04-24 10:12}

德國對華爲的處理決定,是一種人爲、非理性、不可預知的事項,既然即將定案,我們應該靜等水落石出,再做評論。有關德國的對華政策,真正嚴重的新發展,在於CDU/CSU兩個黨魁爭奪聯盟主導權公開化、持久化,以及綠黨在民調上奪取第一,使得九月的大選越來越不樂觀。
我從博客一開始就强調美國無可和解、歐洲才是中國戰略外交工作的關鍵對象。但中方的實際策略顯示,他們到2019年底才明白美國的真實用意;對必須把歐盟和美國陣營徹底分別對待的要點,則至今還沒有明顯的理解跡象。雖然我的讀者群中,傻乎乎地重複“大内自有高手”的人是絕對少數,我還是再解釋一次:君子可欺以其方,正是因爲再聰明的人面臨訊息不對稱也一樣沒轍;如果對基礎事實的瞭解是一片空白,任何邏輯演繹都無從做起。這是爲什麽中國人口多於歐洲人,演化論、相對論、蒸汽機、汽車、飛機卻都是後者發明的原因,爲什麽美國股市最賺錢的對衝基金靠的是内綫消息,也是我堅持勤於閲讀的動力(我非常厭惡面臨訊息不對稱的感覺)。
然而如果有足夠的科學素養,就應該能夠有自知之明,知道自己沒有觀察歐美内部運作細節的機制,在嚴重的訊息不對稱前提下,不能只拿西方的宣傳稿閉門造車,全盤接受對方自我美化的造神運動;也不能因爲他們的口氣相似,就認定為統一的團夥。歷史上只有鄧小平、王滬寧和習近平這個級別的頂層人物才分別在經濟發展、政治體制和對外姿態的戰略選擇上展示了這樣的智慧,在負責執行外交宣傳等戰術任務的官僚體系中,則完全看不到這個方向的認知和努力。
我雖然仍舊盡力把真相解釋給相關人士,但很可能爲時已晚、無濟於事。幸虧歐洲人應該還記得Trump的四年任期,必須防範美國外交政策波動,所以他們最可能的選擇,是在公開場合越來越接受美國的説辭,並且經常性地采取象徵性的敵對行動,但會避免實質決裂。中國總有廣大的第三世界做依托,只要歐洲不完全決裂,崛起過程被打斷的機率還是極小的。
\subsection*{2021-04-18 12:33}

我在七年前的《習安會有譜嗎?》一文中,分析安倍上任後美日的幾次高峰會議和聲明,就覺得是各懷鬼胎,結果OBama太嫩,被安倍佔了便宜,獲得若干鬆綁。這次菅上臺和美方互動,兩邊的動機應該還是完全一樣的;提到台灣,毫無實質意義,純粹是個方便的藉口來凸顯“中國威脅”。
中方的軍力早已足夠拒止美軍的干涉,所以美國不可能全國下定決心和中國直面對戰,這也是我在2014、2015年就反復解釋過的。美軍目前針對共軍所設想的武器采購計劃,大多要到2030年之後才能實現,届時共軍戰力必然又提升了幾級;美國軍事精英不是不明白,然而他們的訴求原本就不是爲了打勝仗,而是爲了要經費。未來5年,美國還是會繼續鼓動東亞盟國當炮灰、同時忽悠歐洲準備做經濟制裁。所以中國的外交工作,也必須針對這些方向來反擊。
\subsection*{2021-04-17 23:22}

我在前一個回復已經提過,用王安石新黨來對比現代美國,是太擡舉後者了。美國現在的問題,不是改革錯誤或失敗,而是幕後的財閥土豪根本不容許改革。
美國近年的財政管理,有點像個深陷圈套的賭徒,每一局都要輸錢,但每一次都可以從家裏(亦即使用美元的其他國家)凑出更多的財富來加碼。2008年之後,美聯儲印鈔票離譜的程度,可以簡單從貨幣供應(也就是M0、M1和M2)曲綫看出;這其中前兩者的定義改來改去,部分原因可能是爲了欺騙世界,雖然臺面上的理由是貨幣管理的態勢和方法變了。M2的曲綫比較乾净,有興趣的讀者,可以到這個網站看看:
https://fred.stlouisfed.org/series/M2SL
\subsection*{2021-04-16 13:04}

我對普羅大衆、甚至知識分子的智商和邏輯能力,沒有你這麽樂觀。中外歷史上虛僞宣傳因爲種種原因,沒有被立刻反駁之後成爲“常識”的例子,不勝枚舉。然後這些虛僞的常識,自然成爲進一步討論的錯誤根基。就連原本應該是絕對理性和極高智商的高能物理界,也因爲發論文的壓力而自發地走上超弦的歪路,堆砌出幾十層的荒謬假設和理論。核聚變的虛僞邏輯,也沒有什麽太難揭穿之處,但到今日,我仍然是唯一一個站出來批判的人。
我今天才又讀了一篇歷史學界的新論文,討論公元三世紀羅馬帝國的知識階級對基督教的正反辯論。我以前提過,三世紀的羅馬内憂外患、動蕩不安,基督教作爲慈善組織,不斷壯大起來,最終成爲國教。但在被政壇頂層接受之前,保守勢力曾做出最後的反撲,這篇論文談的就是那場文化宣傳戰役。反方指出既然耶穌被釘在十字架上,那麽就是最低級的罪犯;面對這個指責,基督教的公關人員發明了Pontius Pilate(一世紀早期羅馬在Judaea的總督)被猶太教大主教欺瞞逼迫的故事,藉著把猶太人當作替罪羔羊,解決了這個公關危機。事後猶太人被基督徒迫害了一千多年,這是主因。然而實際上耶穌是否曾經存在完全沒有實據,更別提什麽審判了。人類就是如此地愚昧,使得各國文化中充斥著虛假的敘事;所以我才會急著在新的陰謀論、謊言公關生根發芽之初,就盡可能趕快拔除。
當然,造謠一張嘴、闢謠跑斷腿;今天的另一個消息是美國官方承認過去三年炒作的“俄國在阿富汗和伊拉克懸賞美軍人頭”是假新聞,這對更正歐美民衆的印象已經來得太晚,但至少未來專業討論不會再引用這個説辭。
\subsection*{2021-04-14 17:11}

正是。這裏的問題在於如何包裝這個訊息:首先必須列舉日本核電企業和監管單位撒謊的前例,然後在國際機構追打。很不幸,國際原子能組織被北約完全掌控,不像WHO那樣能站在專業立場上說實話,那麽只好改用其他平臺。
我在這篇文章的續篇(還在獨家時期,暫不能發表)就已經提過,OPCW(禁止化學武器組織)幾年前調查白頭盔假造敘利亞化武事件,正式報告被土耳其裔總幹事篡改,内部的專業人員前仆後繼地想要揭露真相,被英美歐主流媒體持續封殺。中方一直以爲事不關己、袖手旁觀;其實如果早點聽我解釋,明白美國霸權從2009年開始就準備把中國當作頭號敵人,那麽這種手段必然也會被用在中國身上是極簡單的邏輯結論,中方顯然應該高調在聯合國譴責這樣欺騙全世界的准戰爭行爲,而現在也不會如此孤立無援、坐著挨打。
就算到現在,亡羊補牢,依舊可以拿OPCW來説事,然後再順便話鋒一轉,攻擊日本核電管理的誠信,要求聯合國調查日本和國際原子能組織。
\subsection*{2021-04-12 14:14}

哎,原來如此,難怪我在深圳的新朋友也不能來看我的博客。我原本還想偷懶,叫他們自己來這裏找既有材料,畢竟光是我已經明確討論過的議題,也需要幾年才能好好消化。現在只好希望趙立堅有時間看“世界對白”轉發給他的消息。
國内網絡其實或多或少也有這類的資訊,問題在於垃圾占了99.9999999 \% ,即使偶然撞見了有用的,也很難第一時間辨識。我一直强力確保這個博客的純净,就在於建立公信力,遇到一般人難以很快理解的道理,讀者想起自己在舊議題上一再經歷過從反對到服氣的心態轉變,就比較容易敞開心胸、接受新的事實與邏輯。
我記得有讀者志願把博客的内容打包,在國内分享;我堅持不收費,就是爲了方便傳播,那也包括了這類途徑。下載這樣的檔案應該沒有被禁止吧?
\subsection*{2021-04-11 11:19}

自從秦始皇建立中央集權的“現代式”政府,中國人先天就假設公共事務運作必須有内部紀律的一致性,然而英美完全不同,他們的政治體系其實只是各方勢力的談判桌,只有在很重要的國安和霸權議題上,拜兩次大戰和冷戰之賜,有系統性的組織和策略。
2006和2007年的時候,有沒有金融界内部人士看出危機將至呢?當然有的,而且還不少(區區在下也是其中之一),但這並不代表有任何機制可以敦促公權力出手挽救局面,連美聯儲都迫於既得利益集團的壓力而只想矇混過關,所以預見危機的人唯一能做的就是設法從中取利,例如高盛和好幾個對衝基金都大幅做空市場。正是因爲有這樣的例子,我才能斷言他們的體制是低劣的。你如果憑空想象出陰謀論,就太擡舉他們了;當然高盛他們佔的是德國中小銀行的便宜,但那不是出於戰略考慮,而是因爲後者天真幼稚,方便使然。猶太人要騙錢搶錢,連自己人都不忌諱的;不止在納粹時期有很多例子,近年的Madoff案也是一樣的。
\subsection*{2021-04-11 08:34}

《RT》在這方面已經做得不錯了;中國當然應該有自己的類似管道,但這是長期事業、緩不濟急。
目前的當務之急,是揭穿關於新疆的謊言,避免它成爲歐洲人的“常識”。外交部的“嚴正聲明”、“譴責”、“反駁”不但無濟於事,而且在歐洲人眼中像是默認。真正的有效辦法,是從細節上攻擊那些指控者的Credibility,正如我對韓飛龍所作的。這是有時限的;西方的公共輿論和新聞自由,其運作模式是任由利益集團出來漫天撒謊,而有利益衝突的對手則必須自行設法反擊。你在一段時間内,沒有做出有效的反擊,那些指控就會被社會接受為真理。外交部最近幾個月的手忙脚亂,基本還是對内的那一套,或許讓上級和小粉紅們滿意,但是對歐洲一點效果都沒有;偏偏正確的方向和方法,我都已經預告了好幾年,幾萬名專業外交的官員,只須要有一個來我的博客看看,自然會得到正解。他們堅持閉門造車,結果只是有見識的親者痛、而始作俑的仇者快。

我想補充一下;如果博客這裏所給出的謊言穿幫案例不夠多,網絡上到處都有類似的分析,尤其《RT》和《Grayzone》上連視頻都修編好了,我個人不喜歡拾人牙慧,所以沒有照搬轉述,但外交部不是在做學問而是在做應用,藉助既有材料真不應該是問題。就連國内的《微博》都有人傳播這些分析,昨天“世界對白”才發私信給我,說“那个被西方媒体竖为标杆的维族妇女,三次接受“主流媒体”采访的说词都不一样。而最新一次CNN全部视频只是在她护照签证的地方打码。结果被这位老兄识破——那位妇女之前说自己被关押时恰恰是给予她新签证的时候,而CNN明显知道这个漏洞才会这么处理。”我立刻建議他轉發給趙立堅。當然,官方要用這些材料之前,必須先小心確認其可靠性,避免英美傳媒倒打一耙,但資源這麽多,至少有一半是可用的,而外交部只運用了0 \% ,實在讓人扼腕。
現實世界不是低級連續劇,好人真不必是傻白甜。
\subsection*{2021-04-10 22:49}

這非常適合用太平洋的友邦作爲白手套來打擊日本之用。
你如果仔細去看這兩天新聞報導中的科普,就會發現討論廢水中污染物的時候,排名第一的往往是氚。
福島是第二代的輕水反應爐,裏面既沒有鋰、也沒有重水(參見《從假大空談新時代的學術管理》的後注二和後注三),一般物理學生就以爲不會產生氚。其實核反應非常複雜,幾乎所有的反應都有至少幾十種不同的可能結果(Branching);而普通教科書上討論的,往往就只是機率最大的那個。像是我以前提過的,有科幻作家設想要用氦3來做核聚變以避免產生中子,就是典型的紙上談兵;這是因爲實際上即使是氦3,也只有主反應通道沒有中子,次要通道都很高興地釋放高能快中子。整體來看,即使在最理想的估算下,也只把中子的生產減低兩個數量級。既然氘氚聚變產生的快中子比人類工業材料的承受極限高出7、8個數量級,那麽改用更貴上5、6個數量級的氦3來減低兩個數量級的中子,也是明顯的賠本生意。
我反復討論過,氚的厲害之處,在於一旦混入熱機管道,就無法完全遏制阻擋(Contain),而且它的化學性質基本等同於氫,在自然環境中很快氧化為超重水,不但人體對它有100 \% 的吸收率,而且吸收之後,就長驅直入所有組織細胞内,甚至成爲DNA的一部分。加拿大的CANDU反應爐用的是重水減速劑,每年無意中生產氚不到0.3克,在反應回路完整無缺的前提下,已經造成地下水氚含量超標。福島的輕水反應爐年產氚的數量又低了若干數量級,也就是在毫克的級別,一旦管路損壞,全部泄漏出來,一樣排名到所有放射性重金屬之前。這樣危險得不得了的超級毒元素,核聚變的推動者裝作不知道,還想要浪費幾千億、幾萬億來成千上萬噸地生產,其自私可惡,實在是罄竹難書。

我想特別澄清一下,我說的是這件事適合用來施加外交壓力,並不是說污染物的立即危害能被簡單、明確地測量出來。
從日本人的觀點來看,福島有六座反應爐,運行了40多年,假設每年每座生產30毫克的氚(亦即CANDU的1/10),那麽全部冷卻水所含的氚大約爲7克。加州對地下水含氚量標準最嚴格,上限是130TU,相當於每噸地下水中只能有0.015納克的氚,所以必須找到5千億噸的水(大約是30個洞庭湖)來稀釋,才能達標。這在福島縣的地下水層是不可能的任務,但是太平洋的海水總共有700000萬億噸,可以輕鬆滿足需求。
然而從國際社會的觀點,任何單一污染事件都可以靠扔進太平洋來解決,所以它絕對不是合理可接受的標準,否則所有其他案件的污染性基本都低於福島,更加有理由可以隨意排泄。而且這裏只估算了氚,還有其他幾十種放射性元素,在近距離高濃度的地下水中危害低於氚,但是大洋的稀釋作用就沒有氚那麽顯著,例如一些重金屬會因食物鏈而被重新濃縮起來,造成對水產的嚴重污染。
\subsection*{2021-04-09 08:22}

土耳其自認是Turkic民族的共主,與新疆的維族有實際的關聯,就像越南在南海有實際利益關係一樣。他們和英美純粹找不相干的藉口來造謠,有本質上的不同。
造謠編故事,以凝聚信仰、打擊異教,是宗教的一貫核心伎倆。例如整本聖經,幾千個故事,沒有一個是完全真實的。這並不是從基督教才開始,舊約裏的人物,從Abraham到Moses,歷史上都查無此人;Joshua征服Canaan,根本就是600年後才發明的胡説八道,當前學術界的共識是猶太人本身就是Canaanite的後裔,連上帝“Jehovah”,都繼承自Canaan宗教的主神“El”;King David和Solomon雖然算是例外,有可能真實存在過,但絕對沒有統一“大以色列”,頂多就是領導耶路撒冷一個城的小酋長。猶太人的宗教是Canaan宗教的延續;在公元前10-8世紀,從青銅時代崩潰復蘇起來的Canaan文明慢慢凝聚為兩個國家,其中較强大的Kingdom of Israel在公元前720年被Neo Assyrian Empire攻滅,很多教士逃難到Kingdom of Judah,帶來許多該國特有的神話和傳説。80年之後,King Jusiah根據這些外來的材料,結合Judah自己的歷史傳奇,針對當時他想要收復Israel領土的政治需要,編纂出一分完整的宣傳稿,這就是舊約聖經最早的版本。不過當時的原版聖經,還不是一神論,有些教堂的上帝居然有老婆(在這個女神的祭壇上,曾經發現過大麻的遺跡,所以白左真是一脈相承幾千年);這要等到公元前587年,Neo Babylonian Empire兼并了Judah,猶太貴族和教士在Babylon當俘虜期間,痛定思痛,充分發揚阿Q精神,依照精神勝利法原則,才把自家的主神無限放大到宇宙終極主宰的地位,學術界認爲這是猶太教Judaism的正式起源,此前的版本叫做Israelite Religion。但是嚴格來説,這樣的“一神教”“Monotheism”是假的,因爲所謂的天使其實也是神,只不過是不能被膜拜的小神,所以有些學者認爲猶太教、基督教和伊斯蘭應該算是“Monolatry”。
這些研究,來自現代以色列的考古學界,例如Israel Finkelstein就很有名,有興趣的讀者可以去找他的論文。真正的歷史學和考古學家,必須先有科學從業者的修養,永遠都以客觀事實證據為唯一考慮,即使這代表著否定自己國家的建國神話。中國的學術界,明明知道西方歷史學界並沒有歧視中國,真正的問題在於自家的研究結果太薄弱,不足以證明夏朝和之前的歷史細節,不但不知反省,反而縱容無良文人造假誣衊,編造出西方僞史論。這和中醫教一樣,都是自欺欺人的宗教,和以色列考古學界的科學態度相對比,實在讓人汗顔。
至於英美宣傳體系,只不過玩這個老掉牙的宗教神話把戲,以便聯合歐洲人打擊中國。如何破解,請大家等下一篇文章發表,再詳細討論。
\section*{【國際】【宣傳】如何破解當前歐美的宣傳攻勢}
\subsection*{2023-03-30 04:37}

這件事,博客以前詳細討論過了,請你找一找,回顧既有論證,這裏簡單總結一下:男女在幾百萬年游獵經濟下,各自演化出來的生理和心理差異,一下子放到高度城市化的工業社會,適應不良在所難免,是現代人類必須面臨的普世矛盾之一。既然女性針對男性最惡劣的少數做過激反應是錯誤的,男性反過來針對女性中最惡劣的少數做過激反應也顯然不明智。然而政府處理任何社會矛盾都應該追求公平合理,絕對避免偏頗,女權問題是白左思潮的重要工具,尤其必須嚴謹:和稀泥表面上似乎尊重以和爲貴的原則,實際上把矛盾深化、隱形化。正確的做法是制定明確的處理原則,公告全民,然後嚴格執行;反正中國社科界閑人多得很,讓他們鑽研這些處理原則的細節是有效的廢物利用。
\subsection*{2022-12-14 22:46}

Hanlon's Razor雖然沒有Occam's Razor那麽基本、重要(因爲前者其實可以看作是後者的應用特例),但在做社科分析時非常有用。它説:Never attribute to malice that which can be adequately explained by stupidity.
這裏顯然沒有足夠證據可以斷言是有意破壞,所以我自己始終談的是新舊交接期間的混亂。至於習近平,在專業議題上必須完全依靠專家建議,他本人直接關心的是政局穩定;如果行政業務班底沒有高效運作,出現慌亂是自然的,和他個人有沒有批示無關,因此而推斷治國能力不足更加是邏輯上的Overreach過度伸張。
\subsection*{2022-10-19 21:31}

是的,我有同樣的感慨。
這次美國設計俄烏衝突,原本就沒有指望在軍事戰綫獲勝,而是要靠金融經貿方面的打擊來推翻Putin政權,結果俄方準備充分、平安渡過,還進一步在國際輿論上有效反擊,英文自媒體出現了許多自幹五,幾個月下來串聯成爲“實話者聯盟”。這些人對昂撒霸權的本質完全看透,博客多年來解釋的道理,他們基本全部理解,唯一的例外就是新疆。而其原因正是,他們雖然疑心這裏也是昂撒捏造出來的口實,但事實真相完全沒有傳播出去,實話者也找不到任何證據,就只能默認Occam's Razor,而接受“主流説法”。這裏的責任,只能算到中方自己的宣傳體系頭上。

我想解釋得更直白一點:這裏中方錯失的機會,是聯係實話者聯盟的主要成員,提供詳細明確的駁斥和證據;至於選擇哪些成員,我可以簡單提供名單,可惜國務院官員不是博客的長期讀者,連這個機會的存在都不知道,更別提如何著手。
\subsection*{2021-07-19 20:29}

我也注意到最近外交部終於開始拿Anglo-Saxon集團的歷史和文化來做文章;至於他們的靈感是直接來自我的建議(參見《再談Biden任期内的中美博弈等議題》),還是間接地經由多年來部落格對華語公共輿論的潛移默化、引發其他作者的轉述演繹,我們無法確定。
中方應對美方壓迫WHO的正確做法,是由國際性英文媒體(例如《RT》或《Grayzone》)出手,尋找必然存在的WHO内部良心人士,暴露美國政治壓力如何運作,並且與2003年Iraqi WMD劃上等號。
俄國央行去美元的過程醖釀已久,和中方沒有什麽關係;反而是後者慢了不止半拍(當然中俄國情不同,中方的去美元化不能從自己的外匯儲備做起,第一步重點是貿易和援助),顯然高層有必要對專業管理人員强調不能只看金融財務、必須服從大戰略需要的考慮。
Taliban如果完成統一,就必須從革命奪權轉爲治理發展的模式。20多年前,他們采取毛式思路(亦即理念空想壓倒現實真相,剛好也是臺獨的基本心態,所以才會有紅綠衛兵的先後對照),結果自然失敗(在軍事失利之前,經濟和内政已經出現嚴重問題和挑戰)了;這次捲土重來,似乎有些反思。若是他們執政後有心搞實際建設,又能維持政局穩定,那麽中方小規模謹慎合作是合理的。
\subsection*{2021-07-08 05:17}
這裏的問題在於,學運除了很容易誤入歧途之外,即使偶爾撞上正當理由和目標,也沒有什麽卵用。像是反腐這樣的改革要求,牽涉到權力階級的頭號切身利益,不可能經由抗議或對話來改變,否則“占領華爾街”運動早在十年前就應該讓美國中興了。中國體制才容許習近平這樣有理想、有決心的領導人上位,並且上位之後容許他做深刻的改革,但是這個改革的能力恰恰來自權力的集中和對理性的尊重,剛好先天就不可能接受學運這種街頭運動的影響。這不是巧合:學生處在Dunning-Kruger曲綫的左側尖峰,如果體制愚蠢到經常性地接受他們的意見,國家早就垮了,例如烏克蘭;如果體制只是假裝接受他們的意見,選擇性地搞黨派鬥爭,那麽最終獲利的必然還是幕後的權勢階級;如果體制是誠實高效的,必然專注在實事實幹,以實際公益最大化為導向,那麽當然必須系統性地對愚蠢的學生意見置之不理。我們又回到民主政治的基本悖論:亦即基層民衆的政治參與程度,與同一批民衆的共同利益,有天然的抵觸;換句話説,因爲火鷄先天就是火鷄,被騙投票過聖誕節之後,到了餐桌上還是會繼續歌頌那個結果(參見現在脫歐派的輿論説辭),所以“By the people”和“For the people”是不可能同時做到的。量子物理的基本前提之一,是Uncertainty Principle:動量和位置的測量精度,有天然的抵觸,不可能同時獲得確值;這裏是類似的現象,或許可以叫做Democratic Uncertainty Principle?偏偏人性喜歡自己參與主導的感覺,而且越是靠近笨蛋峰的頂端,這個趨勢越强,那麽原本就坐在笨蛋峰峰頂的學生階級,當然非常喜歡搞學運了。學運不可能引發體制内的健康改革(這裏的Corollary是,學運能引發的體制内改革必然是有害公益的,例如太陽花運動),卻是革命奪權的必要手段。然而進一步考慮,革命奪權成功上臺的新政權,並沒有任何保證會比舊政權優越,而且是否優越,也和事先有沒有學運毫無邏輯因果關聯。別忘了,紅衛兵就是典型的學運;所謂“Taliban”的字面原意,也正是“學生”的意思。既然學運的對體制内改革的貢獻是負值,對推翻政權才有推動作用,那麽除了像是英、美、台灣這類政體不必擔心CIA搞顛覆,可以純粹拿它來哄騙人民、以作爲幕後Plutocracy+Kleptocracy的傀儡之外,哪個精神正常的政權應該去鼓勵它呢?中共在對舊有貪腐政權做革命的階段,當然必須采納奪權的有效手段;現在已經執政70多年了,還在鼓勵奪權的重要成分,難道是自認貪腐、無可救藥,必須為下一個革命做籌備?這裏的邏輯混亂真是令人咂舌。所以你的結論(“只有...才能防止學運...”)是徹底的顛倒因果;事實上是只有先避免學運,政策上才可能有理性的考慮和有效的執行,最終才有能夠反腐的客觀環境;這是鄧小平在89年,經過慘痛實踐後所領悟的真理,23年後也的確留下一個繁榮穩定的國家供習近平來發揮。\subsection*{2021-07-07 18:03}

離題太遠了,到此爲止。

西方制裁是一個外來因素,但絕大部分“社會主義”國家的内部組織和政策選擇也有嚴重問題:你不能說大躍進和文革是外加制裁的必然結果;Stalin的殘暴、Khrushchev的莽撞、Brezhnev的腐敗和Gorbachev的天真幼稚也不能怪到英美頭上。事實上,領先團隊對後來的競爭者無所不用其極地打壓,是自由市場的必然現象,有智慧、尊重科學思考原則的領導人,原本就應該把它納爲主要考慮,事先根據客觀環境來決定發展步調和方向;我不是還特別寫了《恐龍的起源》嗎?當然,這依舊是件極爲艱巨的任務,所以才會比革命奪權困難罕見得多;不過那不正是我所説的重點嗎?
至於你談體量,則完全Off the mark;剛好相反,越小的經濟體,越容易致富,畢竟真正賺錢的,不是完整的工業體系,而是有利基的高端服務業,尤其是金融,香港是很明顯的例子。小國既然不可能支撐全產業鏈,硬去追求,本身就是個錯誤。
\subsection*{2021-07-07 10:47}
我想了想,也有可能你不是在管閑事,而是切身利益相關。從這個角度,如果你受到生命、人身自由和財產的威脅,可以明説,我相信中共政權不會置之不理;然而你若只是不方便、不順意,或者不喜歡其他社會成員的心態,那麽對不起,世界就是這樣子,充滿了偏見和愚昧。正義之士能關心的,亦即國家爲了長期共同利益而必須改的,只限於有嚴重危害的偏見和愚昧,例如階級傲慢和學術腐敗等等。要是放著這些前十名的問題不管,而把很有限的社會對話頻道寬浪費在排名幾百萬的鷄毛蒜皮小事上,那麽這個國家民族自然會很快失去自我糾錯的能力(這正是媒體財團鼓吹白左的原始用意),開始迅速腐爛僵化,歐美是前車之鑒。學生閲歷不足,知識極其有限、自我認知卻無限膨脹,由他們主導的團隊來討論社會和政治議題,基本不可能有什麽理性、全面的思維,所以必然成事不足敗事有餘。請復習《爲什麽事實與邏輯對群衆無效?》中介紹的Dunning-Kruger曲綫;我在上月説過,即使考慮專業本行,本科教育也只算是簡單的入門介紹,那麽現實中極爲複雜、精微的政治議題,連政治系教授往往都摸不着邊(同樣參見我在上月所提的,這些教授關心“格物致知”嗎?瞭解科學思考原則嗎?聽説過Occam's Razor和Russell's Teapot嗎?學過邏輯悖論嗎?熟悉統計陷阱嗎?這些基本的分析工具都沒有,如何正確地理解政治社會現象?),你想學生處於曲綫的哪一部分?我也一再强調過,英美財閥操弄白左,主要就是通過“進步”“Progressive”媒體(因爲太過成功、普遍,後來被右翼民粹加上“主流媒體”的標簽),一方面事後美化、吹噓那些無腦、無用的示威游行,另一方面則是事先把選定好的鷄毛蒜皮小事(而且基本都内含謊言;事實上作爲謊言載具,正是它們的實際功能)像一塊紅布一樣在目標牛群眼前反復揮舞,激發“義憤”。工運至少推動的是工作環境和待遇等等實際議題,原本就是公益的基本成分,學運卻幾乎必然是在管閑事,而且是財閥通過白左體系精選出來的閑事,其最主要的特點,就是激化族群對立。回到你原本的問題,爲什麽在沒有和英美間諜組織直接聯係的前提下,這些搞LGBT的社團也受到政府的關注?答案很簡單:政府的責任不只是打擊間諜,反而主要是維護公益。而現代社會公益最常面臨的威脅之一,就是擁有公共發聲平臺或甚至決策權的笨蛋(參見有關龍應臺和馬英九的討論),尤其是接受白左洗腦的笨蛋;我不是才剛指出他們可以比邪惡無恥更糟糕?以上這些論點,都是我已經談過百遍以上的老常識,你卻毫無所覺,只怕在LGBT這個特定話題上,你也無力自行從全民公益最大化的角度來做分析,所以我在這裏特別詳細解釋一次。不過我實在沒有閑空老是給巨嬰喂奶,所以下不爲例,請你把博客從頭到尾讀上N次,直到有一點心得,再來參與討論。白左談LGBT,自動把它標爲“人權”,然後接著推論人權必然凌駕於民主多數決原則之上,所以必須賦予他們“Pride”“自豪”。這裏每一步邏輯,從前提到結論,都是Mushy Thinking(漿糊腦子)、胡説八道。基本人權屈指可數,生命保障(包括治安水平)、人身自由(亦即不會無故被關進牢房)和經濟機會平等(包括就學標準、托拉斯壟斷和政治性罰款)才是超越多數決意見的普世價值。LGBT在中國會被槍斃嗎?會像在美國那樣被暴民私刑毆打?會因爲性向而被關起來嗎?入學就業有專項標準以便減分排斥(例如美國的亞裔學生)嗎?會被政府起訴罰錢嗎?既然沒有違反基本人權,也無關生活水準(別忘了,GDP的損失也是隱性的人命,參見《政府的第一要务》 ),那麽次一項重要的公益考慮,就是文化上的主觀偏好。然而LGBT在自己臥房要怎麽搞,中國政府並沒有干涉呀,這裏的實際議題在於他們是否應該被容許公開高調地談自己的性事,或在大街上半裸跳舞擁抱。白左說這“不影響他人”,但那也是謊話;人類社會有著許多沒有特別道理的文化傳統,如果不違反基本人權,也不影響經濟產出,那麽它們不但不妨害公益,而且反而成爲公益的一部分,因爲它們是多數人主觀舒適感的來源。例如在公共場合當衆做愛,也同樣是不直接“影響他人”,那爲什麽不立法强迫大衆接受?如果一個LGBT的隱私,在無意中被公司發現,因而被開除,那是明顯地違反了他的基本人權,政府有理由出手保護(不過實際執行力道有多大,是另一個問題;畢竟中國的富裕程度和政府的行政資源都有限,當前就業方面的各種歧視很多、很普遍,包括醜人、胖子、病人、老人都很容易被解雇,有必要對LGBT做特別優先待遇嗎?)。但這並不是白左要求的重點所在;他們的訴求,被幕後財閥有意安排成爲過分、扭曲、虛假的極端,用來撕裂社會。那麽中國政府未雨綢繆,提前解除這些假社會議題背後的壓力,不是一件值得稱贊的明智之舉嗎?反過來看,你們這些搞學運的,才是國家民族的禍害;中共真正的錯誤,在於沒有把歷史事件用科學原則去想清楚、講清楚,爲了延續舊時的權宜,依然在美化五四運動。89年已經吃了一次大虧,還不知道學乖,將來只怕會有不斷的麻煩。\subsection*{2021-07-04 10:23}

“科學社會主義”這個標簽在19世紀的使用,我們幾周前已經討論過了。我現在談的是21世紀的重新利用(Repurposing),請不要離題。英美的“Democracy”和古希臘也不盡相同,2000多年前的老標簽不是在現代重獲新生?
革命奪權和長久治理是兩回事。打著社會主義招牌建立政權的,至今在一百這個數量級,但治理成功到全球一級富裕程度的,只有半個,亦即中國有望在20多年後達到目標。顯然後者比前者難得多;革命奪權的所謂經驗和理論(例如農村包圍城市),對後者基本沒有意義。這個博客近萬條討論,有哪一條是在談如何革命奪權的?這是因爲我的長遠志向,是要幫助全人類建立大同社會,所以社會主義如何進一步發展,自然成爲重要關鍵。讀者如果不一起面朝這個角度來思考,反而去扯無關的瑣事,對博客討論來説是負面的貢獻。
我昨天已經説過,“中國特色的社會主義”這個空洞的標簽對内不是大問題,主要的毛病在於外宣。隨著中方的國際地位提升,外宣的重要性也迅速提高,尤其在中美博弈的生死關頭,國際輿論已經成爲美國打壓的主要手段;你沒注意到那是這條討論鏈的起源嗎?這篇正文的標題呢?
再提醒你一次,不能無視博客既有的論證而自説自話。我做辯證的目的在於求真,不是求勝;如果讀者忽略已經鋪陳的事實和邏輯,自顧自地寫意識流,固然永遠不必認輸,但我們也永遠無法達成求真的共識,徒然浪費時間。如果你不想改,請離開,另找地方殺時間自娛。我的時間寶貴的很,只有追求和傳播真相才值得花費,讓你自我感覺良好不在我的職志之列。尤其我才剛批評因循苟且、固步自封的態度,你就來扯不相干的歷史細節,實在很像是被踩了尾巴的貓。禁言一個月,好自反省。

還有,“實踐是檢驗真理的唯一標準”,毛只在它方便自己的時候,才拿來當政治鬥爭的口號;他搞大躍進和文革期間,誰敢談這句話?對這個原則有誠心和一致性的是鄧小平,所以也只有鄧才有資格被引用為出處;畢竟實踐也是檢驗人格的唯一標準。
\subsection*{2021-07-03 18:51}

社會主義是什麽?馬列毛鄧所説所做的?它們根本就沒有一致性,互相矛盾,不可能充當定義,更別提成就參差不齊,導致你説的西方媒體可以很容易就把社會主義汙名化。而且從理論角度,一味低頭追尋前人的脚印,視野極其狹隘、不思進取,除了水八股文的篇幅之外,毫無實際意義。
我反復解釋過,政治的目的,在於公益的最大化;各式各樣的政治體制和理論,都只是爲了達到同一目的的不同路綫選擇。資本主義的基本假設是什麽?只要追求資本利益的最大化,全社會的公益也自然最大化;這在理論和實踐上都可以簡單證明是胡扯,我不多贅言。
相對的,社會主義的邏輯前提何在?政策必須始終並直接以全民整體公益為評量標準。這在字面上看來很合理,毛病可能出在哪裏呢?首先,政策設計與實際結果之間,往往有著複雜而且出乎意料的作用和反應,必須對經濟學、政治學、社會學、心理學等等社科學科進行深刻、甚至是開創性的思考,才能事先計算預估。如果做不到,就只好步步爲營,以實驗結果為導向(摸著石頭過河也有它的問題,亦即如果不高瞻遠矚、從理論上看出正確的大方向,就會被局限於局部最優解,而且組織内外的利益集團都可以試圖建立執行上的障礙,阻止國家社會追求全局最優,例如反腐、打擊中醫教,都不是例行官僚行政所能觸及的,參見下面關於人性的討論)。文革就是沒有考慮政策實際後果、强硬胡搞的典型案例,反而被資本主義宣傳機器拿來反襯自身的優越性。
其次,直接追求公益違反了人類的自私天性,公權力内部先天就暗藏極大的腐化壓力。這要求一個超政府組織,能夠强加嚴格的紀律。這個難關在歷史上更爲重要而且常見,蘇聯和絕大多數亞、非、拉的社會主義政權都是絆倒在這裏;中共在2012年之前,也曾面臨類似的困難。
所以“中國特色”這個形容詞不但毫無實際意義,隨便什麽版本都可以說是中國特有,而且歷史上實踐社會主義的兩個大坑,中共都曾經摔進去過,這算哪門子特色?反之,“科學”兩字,不但是對改開以來最高指導思想的精確總結,而且是避開和爬出大坑的最佳指南,和“廢話”剛好相反,你是怎麽扯到一起的?我沒有看到你的論證,就只有一句空口白話的論斷;嚴重警告一次,再犯禁言。
\subsection*{2021-07-03 10:18}

你説的很好,正是我建議“科學社會主義”之前,已經思量過的考慮;這裏我再補充幾點。
首先,Marx創立現代共產主義思想,著作中最大的空白遺漏在於政權(State,不是政府Government)應有的角色和作用,所以日後實際掌權的共產黨人,是在沒有理論根據的前提下自行摸索,其中Lenin最早奪權,他的經驗也自然被記錄下來充當後人參考,所以才會有馬列並列的説法。然而Lenin的選擇顯然達不到Marx的高度,重新嘗試探索在所難免。
在中國,毛的版本在文革後壽終正寢,鄧小平反思的結論是大家都知道的,以實驗和成果為導向的步步爲營策略。習近平雖然在對内對外的政策戰略都做了重要的改革和修正,在思想層次卻是完全繼承鄧的遺產。既然是摸著石頭過河,當然是因時制宜、與時俱進,難以事先歸納出一個簡單標題。我的思路是,可以更深一層來看(英文是A higher metalevel):現代中共的政策雖然沒有固定方向,但那正是不預設立場,遵循科學方法和理性原則,追求公益最大化的結果,所以稱爲“科學社會主義”再恰當也不過了。
自稱是“中國特色的社會主義”,是在前面討論的鄧思想前提下,有意做出的籠統説法。換句話説,它是毫無内涵意義的廢話(還記得上周有人提過,這個博客的慣例是把内容重點迅速高效地濃縮出來?),而且是故意的。這對内還不是問題,反正知識分子對改開以來的治國方針當然心裏有數,但是放在外宣就很尷尬:自由市場體制裏充斥著各式各樣、毫無意義的空洞廣告詞,歐美群衆久經訓練,一看便知,然後自然誤解為中共是在哄騙聽衆。用“科學社會主義”做出精確的總結,不但名正言順,在外宣上大幅加分,對内也時時提醒群衆要建立以科學理性為基礎的社會文化,那剛好就是這個博客的宗旨之一。
\subsection*{2021-06-30 20:24}

是的,不過英美内部民衆十年前就被洗腦完成,早已沒救了,畢竟“It’s Easier to Fool People Than to Convince Them That They Have Been Fooled”(一般說來自Mark Twain,不過查無實證,所以應該是後人僞托)。現在外宣的重點對象,只能在於歐盟和其他未站隊國家;即便如此,大部分民衆依舊不會聽得進去這樣的講理(雖然比八股文要好得多),必須是更簡短、有力、高調、反復的揭瘡疤才可能有實際影響。
這裏我舉一個近期的實際例子:中方批評日本排泄核廢水之後,外媒就忽然爭相報導中國核電站的一個小事故,這顯然不是巧合,而是有意發掘出來小題大做、用以矇蔽普羅大衆的宣傳烟幕。外交部發言人應該高調地强調,Fake News拿一個在美國每年發生多次的0級事故來説事,誤導受衆去和福島的7級事故相比,顯然是有意騙人的手法,句點。殺人誅心,公共辯論中最有效的伎倆是人身攻擊,Trump一輩子只靠編造謊言和强加幼稚標簽來攻擊敵人,結果在美國能做到總統,那麽中方用實話來直接打擊美宣體系,自然更加是名正言順、理所當然。換句話説,外宣的首要聽衆是歐盟,頭號打擊對象則是英美媒體;你引用的這個視頻認識到後面這一點,是我最欣賞的角度。
\subsection*{2021-06-30 20:03}

精神鴉片的確是我批評的對象之一(只針對知識分子而言,普羅大衆追尋娛樂無傷大雅)。不過我必須提醒你,這位Mac和上次的freeHK相比有一個本質上的差異,就是後者的思路完全基於感覺和意識形態,而前者卻是從自身利益出發,只不過沒有深思熟慮,找錯了發泄的對象。我以前曾經特別撰文討論“損人不利己”和“損人利己”的對比,並且解釋過前者其實可以更糟糕。上周有朋友來訪,閑談中他們也問起爲什麽我對龍應臺特別嚴厲;他們說一般人認爲龍只是笨,而其他藍綠政客則邪惡無恥得多。我的回答是,在現代媒體環境下,公共人物的笨,可以比邪惡無恥的後果更為嚴重。這裏的機制就在於愚化群衆、藉著身體力行的示範作用誘導他們走上損人不利己的歧途:壞人說謊話,總有痕跡可循,笨人說蠢話,卻是絕對誠心誠意的。 我並不是說Mac是壞人;自己的利益無端受損,想要抱怨是普世人性,而且不能算是非理性的反應。我之所以不太高興,是因爲這裏的正確利害分析,博客早已反復解釋過,不去找始作俑的美國人或爲虎作倀的法輪功和民進黨算賬,反過來怪到主要受害人頭上,才是他所犯的邏輯謬誤。這個謬誤是技術性的,而freeHK犯的卻是根本性的錯誤。
\subsection*{2021-06-29 16:11}

“這個國家在這方面的無能對我的利益是一種非常直接的侵害”這句話,讓我目瞪口呆;你有必要檢討一下自己,是不是被台灣或僑界的愚昧公共論壇所污染,把智商往下拉的太多了?我在2013年返臺,只讀了6個禮拜的《自由時報》,就深深感受到智商歸零的壓力。如果是中毒幾十年,我建議你把博客反復多讀幾遍;重建理性邏輯能力的過程,可能是非常緩慢而痛苦的。
你那句話有什麽邏輯問題呢?太多了,我得要一步一步來討論:
1)當前英美的抹黑攻勢,目標對象主要是歐洲、次要是其他對中美博弈還沒有站隊的國家。他們對内的洗腦,早在十年前就已經完成,和現在中國的反應毫無邏輯因果關係。
2)至於其反中的原動力,來自Anglo-Saxon的種族優越感和自私無恥傳統,中國唯一“討打”的作爲,是高效的政治和經濟,威脅了美國的霸權,在外宣上的抵抗與否根本不是原因。事實上剛好相反,如果中國的外宣都聼我的,見招拆招,把英美在國際上不斷羞辱,固然對中國本身有大益,但居住在英美的華裔僑民反而必須面對老羞成怒的暴民,處境會比現在還要糟糕。就連袁紹都不能就事論事,爲了自己的錯誤反殺了田豐,你覺得英美民衆有更高的智慧?
3)從道德觀點,這種爲了利益或仇恨而全世界造謠抹黑的邪惡行爲,真的應該被怪到受害人沒有更有效地抵抗嗎?你自己有沒有公開批評過《大紀元》或《自由時報》?如果有壞人闖進你家,你選擇躲起來保命,結果女兒被强暴了,你事後有臉來怪她沒有盡力抗拒嗎?
4)從時間來看,我在中學時期就注意到“民主”、“自由”那些敘事不像科學論述,反而類似宗教教條;來到美國沒幾年,就徹底看穿Anglo-Saxon言行之無恥。我有資格批評中國外宣的後知後覺是因爲自己比他們早覺悟了幾十年,而且還花了無數心血來做教育工作;你呢?
至於僑民該如何自處,這得視個人環境而定。我自己正在準備離開美國;其他人可能沒有這個選項,包括我兒子暫時都如此。我能事先建議的是,不要指望模仿黑人建立政治影響力,亞裔沒有那個本錢,徒然養肥幾隻踩著衆人頭上、換取個人私利的樣板忠狗。
\subsection*{2021-06-26 14:17}
你把一流和一般人才混爲一談了。如果全中國有大約一千多個一流天才,這個標準不正應對著百萬人選一嗎?你心目中的那些專家真的夠格嗎?那個公開質疑楊先生的研究生,顯然自以爲了不起,但他到了二十好幾,連自己行業的基本原則和定位都搞不清楚,把王貽芳這種四、五流學人當作神仙一樣膜拜,說他是九流可能都太寬厚了。雖然最近我把選拔、教育和獎勵機制並列,但那是因爲在現代專業細分的背景下,二、三流的人才也可以有重要的貢獻。其實對真正一流的頂尖人物來説,只有選拔才是最重要。人類的聰明才智固然有其上限,但這個上限依舊遠高於一般人的想象,只要思考訓練的大方向是正確的,格物致知的基本條件滿足了,自然有天才能脫穎而出,回應時代的召喚。當然專業細節的指數式繁衍,的確會把精煉的工匠技術和獨立的天才創造混肴起來,也可能成爲浪費時間精力的泥淖。尤其是行業組織一旦達到足夠規模,就自然有官僚慣性,即使無正事可幹,也必須繼續發明出新課題來忙碌,高能物理理論去搞超弦是典型的例子。你覺得真正一流的人才,是那些留下來寫出無數玄學論文的“大師”,還是有足夠見識和膽氣、願意堅持科學原則而轉行,另外開創出新事業的人?\subsection*{2021-06-24 22:05}
真實的科技探索,是10000個可能方向(“點子”;不過不是劉慈欣這類民科級別的點子,而是行内人做研究過程中想到的點子)裏,只有1000個經得起理論初步檢驗;這1000個之中,只有100個能通過第一級的實驗驗證;這100個之中,只有10個能在簡單的效費比估算後過關;然後這10個點子交給創投基金,基金只要10個裏有1個搞成,就算很了不起。科學的根本在於求真,技術的關鍵在於實用,民科級別的胡思亂想,反其道而行,那麽流行之後成爲騙術橫行的思想基礎,在所必然、是事先就可以簡單預見的。這也是我昨天說“科幻的核心從來就不是科技”的邏輯根據:既然是民科幻想,當然是假的、不實用的,這樣的“科”和“技”,其價值是絕對負面,越是當真、越是受歡迎,對社會民心的腐蝕就越大。科幻作爲虛構小説的一個類別,有什麽特別的價值?一般小説的優劣,取決於對人性的描寫是否深刻全面。人類歷史上可能出現的遭遇,已經包含了所有文學創作的需要,根本不必再去憑空創造虛無的假科技背景。所以硬要去假想科幻,就只能在科技或者其社會影響上做文章;既然前者是危害極大的錯誤方向,做社會探討不只是理想中的目標,而且可以用嚴格的邏輯推論證明是唯一值得寫、值得讀的作品類別。以上我用三個段落就完成科幻發展方向的論證,糾正了大陸科幻行業40年的迷思。在這之前,中國的文學和社會學學者沒有一個能達到正確的結論,這不是整個文科教育出了大問題的又一個表徵嗎?\subsection*{2021-06-23 22:57}

Bravo!你所寫的是在理性、誠實的前提下,為劉慈欣做開脫的極限。不過濃縮起來,你的論述終究只説了《三體》這類作品其來有自,並不抵觸反方的核心論點。這裏我為讀者方便,再總結一次:1)在文學上,《三體》是低級的爽文;2)在思想上,《三體》基於非理性的美宣謊言;3)在人品上,劉慈欣冒充科技和戰略專家,是個騙子;4)在國家政策上,《三體》有極大的危害,若不是楊先生和我挺身而出,差一點就要讓中國的學術科研完全出軌。
兩年前,我曾在《觀網》的新聞報導下留言,不過博客的讀者不一定看到了,所以也在此復述一次。那篇報導談的是劉慈欣挂頭牌出席科技發展座談會的討論,我的評論則是:“請科幻作家來談科技發展,和投票給演總統的演員當總統,有什麽差別?”
\subsection*{2021-06-23 15:57}

是離題了。原本我會刪這條留言,不過剛好本期的《Economist》有一篇相關文章(參見Hard truths about SoftBank | The Economist),值得推薦。它談的是孫正義和軟銀,原本一年多前已經瀕臨崩潰,但受益於美聯儲的無限放水,又滿血復活。這裏最有意思的細節,在於孫正義在過去幾年高調起用一批來自Deutsche Bank的交易主管,而這批人的專長就在於佔投資人和雇主的便宜;這正是我20年前和Deutsche Bank打交道的經驗。而且孫所雇的,居然是一個印度人團隊,他們比美國人還要更無恥、自私、糟糕。如果美國真如我所預期的,在未來5年内出現經濟金融危機,那麽孫正義和這批印度人或許可以抱著掠奪的贓款舒服地退休,但是軟銀這個機構除了由中方監管維持的阿里巴巴之外,只怕會面臨連續而全面的爆雷。
我不太想要繼續這條討論鏈;如果讀者真有深刻的心得,也請移步到金融類文章下留言。
\subsection*{2021-06-23 15:50}

我不是文人,不説沒有實證的空話。我在留言欄論證過程如果有看似跳躍的邏輯,不是因爲那些步驟以往已經詳細討論過、無需重複,就是旁支末節,除非有人追問,否則不必詳談(這一類涵蓋更深一層邏輯的探討,如果處處求全,整體論述被多次打斷,會違反博客這裏文字簡練易懂的原則,例如昨天我第一次談上個月的外宣成果,曾經直接忽略空間站,後來有機會才又深入解釋),再或者是有公開、明確的事實證據,讀者可以簡單自行查證。這件事屬於最後一類;謝謝你花時間解釋給其他讀者,不過我自己當然知道,所以才敢斬釘截鐵地說《三體》得獎完全源自白左思潮的政治考慮。當時他們正和右翼民粹做激烈鬥爭,還有什麽比推選來自中國的三流作品更能侮辱對手的?當然如果《三體》沒有實質接受美宣的前提,反過來對中方外宣有益,白左絕不會考慮提名。
至於小説的優劣,認爲它是一篇純粹為幼稚讀者所寫的爽文,才是高級科幻愛好者的主流意見;只是由於特殊的歷史背景,在白左政治考慮下,不公開批判罷了。換句話説,小粉紅以爲是中國之光的作品,其實在歐美行内人心中,反而是中國文化落後、品質低劣的例證;就像英美在過去20年對印度經濟發展潛力的吹捧,是在確定後者永遠追不上的前提考慮下,所説的宣傳反話。
\subsection*{2021-06-23 03:35}

你説的這些中國社會的文化現象和歷程,我在和大陸讀者交流之後,已經有所認知,但這並不影響《三體》的核心敘事處處顛倒是非的事實:不但在基礎對應應用科研的價值評估上,他説的與正道剛好相反,就連宣傳“謀略”這些Cheap tricks,全世界哪有玩得過Anglo-Saxon的民族?《三體》浮面上似乎是要重建民族自信,實際上是對美宣的虛僞前提照單全收,然後再藉用好萊塢的慣常技巧,扭曲邏輯、人性和常識,硬是拗出一篇小粉紅能自我感覺良好的爽文。這種表面上有自信,本質上卻繳械投降、讓中國年輕一代落入美國式愚民陷阱的思想毒藥,偏偏在現實政策選擇上又有超大的危害。十四億的華語世界,我居然又是第一個出來説清楚的人,唉,當今華人思想界的水準,真是讓人搖頭嘆息。

當前中國人口和三國時代相比,至少多出40倍;如果只算受過高等教育的,比率會更加極端。三國時代能有兩位數字的一流謀士,照理現在應該有四位數字的同級人才。真相只有一個,因此英雄所見略同,我也渴望有其他人能互相聲援,但是在一件接著一件的重要公共議題上,我都是第一個解釋清楚的人;難得有清華教授出來批評劉慈欣,居然是基於龍應臺式的文學欣賞角度。這明確揭示了在人才教育、選拔和管理體制方面的嚴重危機,即使算入現代大衆媒體和互聯網的愚化作用,都不是正常合理的現象。
\subsection*{2021-06-22 16:04}

我只嘗試讀過《三體》,而且不論如何努力提醒自己要堅毅,依舊只能簡單看完第二部,第三部説什麽都沒辦法强迫自己開始,實在太痛苦了。這裏最大的問題,我以前提過一次,也就是他對科技的描寫,只有大一的程度,對人性則連初中二年級都不到,至於你所謂的“生存之道”,全都是美國人説的“Jumping the shark”,或者英國人會説“Deus ex machina”,這對一般理性受衆應該是Turnoff,如果有人甘之如飴,只能是品味低下的表徵。
《三體》在國際上得獎,其實是個典型的白左政治正確;別忘了,白左正是一手打擊中共官方,另一手則提拔可以渲染的亞裔樣本人物。《BBC》最近雇了黑人女星來重拍《Anne Boleyn》,大陸網民各種譏嘲怒駡、自我感覺極爲良好;劉慈欣得獎本質上是一樣的道理,他們的反應卻剛好相反。Obama喜歡《三體》,也不能算是什麽有力的正面背書,畢竟他並不以戰略修養和科技知識著稱。
至於外宣效果,我已經説過了,科幻片一點用都沒有,你以爲《Netflix》做決定時沒有考慮這一點嗎?雨果獎給獎的時候呢?你若是被美國人扔了一根毫無實際營養的骨頭,就興奮得不得了,這反映的不是極度自卑和不自信嗎?和台灣群衆的心態如出一轍,五十步笑百步,所以我對假理性的網民沒有好話。
最後我想提醒你,不要被幾句胡亂編造的假哲學句子唬住,不論它們是如何氣勢磅礴,也應該先動動腦用點邏輯:傲慢是無知的自然後果,不是獨立的現象;喜歡《三體》的人,不正是因爲無知,所以才會爲它感到驕傲?弱小和無知才是歷史上被淹沒的族群中,99 \% 的通病,剩下的1 \% 是運氣。

或許有人會想到Asimov作品中,對科技的描寫也不實際;這沒有錯,但這裏的真正差別在於Asimov從來不假裝那些科技不是純幻想。而且他文章的核心其實是人群社會對新科技的反應,這才是一流科幻的題材。在似乎無解的危機下硬是翻出一個方便到曖昧程度的奇跡,在《Foundation》是糖衣包裝,在《三體》裏卻是故事的硬核。換句話説,Asimov其實談的是社科議題和群體現象,劉慈欣卻是在扯科技和戰略,而他對這兩者都沒有任何真正深入的理解,又偏要裝逼、自稱專家,把群衆帶往災難性的錯誤方向。
\subsection*{2021-06-22 04:31}

其實一般人沒注意,過去這個月中國最成功的外宣,是關於雲南野象群的報導,無形中就讓全世界知道,中國也有人對野生動物付出相當的尊重和保護,稍微抵消了多年來中醫教爲了莫須有的功效、追采獵殺野生植物動物而造成的生物環境破壞。
然而這類新聞,可遇而不可求,要把社會的真相(更別提美化的版本)呈現給世界群衆,電影和電視才是最高效的管道,可惜中國的影視娛樂界淪爲撈快錢的另一個資本游戲場,爭先恐後地追逐最低級的觀衆,藝術性和娛樂性兩失,根本沒有國際上的競爭力。

説到這裏,我聯想起上月有位清華教授批評劉慈欣的作品,說不如金庸的小説。他的論點是三體的人物描寫極爲扁平,沒有任何真實的人性。然而我覺得金庸的人物雖然比較鮮明些,也不是文學裏最豐滿、最深刻的呈現。我同意金庸遠優於劉慈欣這個結論,但真正的理由是,現在沒有人會想要放棄部署隱形戰機,改用降龍十八掌來訓練軍隊,但劉慈欣的胡説八道,卻有意地(他多次以未來科技專家的身份出席會議、發表談話;金庸先生可從來沒有自稱是武術教練)欺騙了幾千萬大學程度的讀者,想要把可以用在半導體、發動機和基因技術的經費,浪費到大對撞機、核聚變、火星殖民這種純幻想的空話上,讓其他騙子有機可乘,國家在崛起的緊要關頭,還要自縛手脚,罪莫大焉。
\subsection*{2021-06-20 20:16}

這些都是次要原因,實際上的主因是現代傳媒環境從日報的普及開始,至今不到兩百年,完全被英美霸權時期覆蓋,所以只有他們才有資源、能量和動力去搞外宣。你看那些抹黑扭曲的Fake News伎倆,全都是19世紀在英國國内政治鬥爭中就發明;所謂“自由”、“民主”的敘事和詞匯,則源自20世紀初,打擊新興德國的需要;冷戰時期對付蘇聯的成功應用,把英美外宣推上巔峰,自然成爲財閥想要直接控制的資源,從Murdoch開始,逐步走偏,越來越有走火入魔的跡象,所以中方其實不能、不必、也不該企圖模仿對方、一簇而成,只要針對英美宣傳的偏斜之處,做出持平、科學、簡短、明確的對應,自然就會獲得全世界的理性知識分子作爲天然盟友。
\subsection*{2021-06-20 10:44}

我以前討論過國族認同的來源,是因爲結成黨派組織先天有競爭優勢,那麽原本沒有組織的群體,最終也會被迫模仿以求對應和生存。既然地域、文化和宗族關係是普遍而既有的基礎,根據它們而組織起來的國族認同就成爲人類社會的常態,並且由於它有益於國家内部的共同利益,而道德正是利於公益的個人行爲規範,所以幾千年下來,國族認同也往往成爲道德規範的一部分。
不過你如果仔細讀這個博客,就會發現我從未基於國族認同來做推論;剛好相反,我一再强調我的論證是無關國籍、立場的。這是因爲國族並沒有明確的定義:臺獨可以簡單把族群的邊界劃在臺海,當然理論上任何人也可以隨便自劃邊界,選擇有無限多,那麽實際上勝出的定義,就看宣傳洗腦的功力,所以美國對内可以强調愛國,可以爲了維持統一打南北戰爭,不惜殺死殺傷5 \% 的全國人口,對外卻搞各式各樣的顛覆分化。
唯一有時空普世性和邏輯一致性的是非對錯標準,必須以全人類公益為評價標準(亦即經濟學裏的Utility Function,效益函數),這也是我一貫的論證前提。
\subsection*{2021-06-19 18:45}

美式民主獎勵自私和愚昧,雖然是我多年來反復討論的議題,但最近歐美民衆對新冠疫情的反應給予我更深的體會,確認這是英美政治體制對社會文化最核心的長期影響,而且是有意、必然和普世的。最近沒有精力寫正文,剛好有人頭送上門來,開啓這個話題,就藉勢發揮一下。
我十八嵗讀《Foundation》,印象深刻,其中有兩個細節左右了我一生自我改進的方向:第一,當然是“Psychohistory”能以科學方法來預先計算政治和社會的歷史進程,雖然是純科幻,但其基本原理卻是合理、可實用的,後來我一直努力推展、實踐、並探求這方面的極限,這個博客就是成果報告;其次,是你所提的,將一堆無意義辭藻迅速排除,只濃縮留下信息的核心,這是第一集中Foundation還很微弱的階段就已完善的研究,對我的學術和職業生涯有更大的影響。我反復説過,現今高能物理理論的典型名教授(例如Brian Greene、Lisa Randall、Michio Kaku、Nima Arkani-Hamed)一生幾百篇論文,濃縮下來什麽都不剩,就是我應用這項技術處理後的結論。
這個能力需要以龐大的知識庫以及精確迅速的邏輯思維為基礎,是幾十年不斷自我改進的結果,沒有速成的法門。
\subsection*{2021-06-18 11:48}

錯誤的結論必然來自錯誤的事實和邏輯。你這裏的錯誤前提很多,不過最基本的是“我感覺怎麽樣,就是對的”;這正是美式洗腦的核心,必須推翻客觀絕對的是非對錯標準,建立對自私態度和愚昧心理的徹底崇拜(不止是寬容!),才能對其的所謂“自由”、“民主”價值觀自圓其説。事實上你的感覺,唯一的意義在於呈現了你自己的人品和智商,與公共事務沒有其他的交集。
反映到你的論述上,先有了那個感覺至上的前提,才能接著推論“中共外宣説得不夠漂亮,就應該成爲人類公敵”的奇談。評論一個政體的價值好壞,當然並不是一個淺易的話題,但所用的標準至少必須與實際民生有所相關。尤其現今的人類社會,依舊是高度的不平等,有至少1/7的人口屬於赤貧狀態,也就是經常性的沒飯吃。把個人的感覺應用在公共議題上,必然要先忽略這些底層民衆的痛苦,所以前面我說這是極端自私的心態;而且這種極端自私,是英美傳媒有意推廣的:所謂的一人一票、民意至上,這個民意顯然不是公益,甚至不是私利,而是愚昧的意識型態,只有這樣,財閥掌控的媒體才能可靠地扭曲選舉、竊取公權力,英國脫歐是一個典型案例,新冠疫情下有歐美群衆不願戴口罩則是這種洗腦的副作用。
當前人類社會的高度不平等,是過去兩百多年英美霸權有意而且用心維持、創造出來的問題,藉以廉價榨取這些落後國家、地區、族群和階級的資源和人力,畢竟如果人類有了大同社會,人人都是吃飽喝足的中產階級,Matt Gaetz要到哪裏找17嵗的援交女孩?Jeffrey Epstein怎麽能批量收買15嵗的性奴?如果你不想理解這個殘酷的事實,而拒絕學習經濟原理,那麽更淺顯的案例也很多,例如Obama剛上臺,就對阿富汗增兵,其中一個附帶的小作業是對巴基斯坦北方的Taliban活動地區進行無區別(Indiscriminate)的無人機攻擊,結果當然是多次將婚喪禮集會搞成人間地獄。當年聯合國還和美國公開爭辯,前者估計每年被殺的1000多人中,至少一半是無辜的,後者則宣稱能從5000米高空100 \% 絕對準確地辨識受害者的政治屬性。這只是美國每年在外作孽的一小部分,主流英文媒體固然視而不見,但像是《Gray Zone》這類的良心組織已經把真相整理出來,放在YouTube上,你連5分鐘都不願意投入,卻自認有權在公共論壇上爲虎作倀,幫助邪惡帝國打壓國際政治上唯一的希望,成爲屠殺和搶劫全人類的共犯,除了是被自私和愚昧徹底洗腦之外,怎麽有可能呢?
我的文字被轉載到大陸之後,也有小粉紅和你一樣,視我為異端邪説;他們的論點,説來説去,就是我的立場不堅。其實我哪有什麽主觀的立場?我的一生奉獻給對真相的追求,事實如何,我就如何說。當然你可以說,那些小粉紅的本質和香港的黃絲、以及台灣的綠衛兵一樣,都是先確立立場,然後把主觀偏見無限放大升級。但這裏的差別在於,從理性客觀的標準,也可以論證出中國遠優於英美的結論;這不是預設的立場,而是普世皆准的邏輯論證結果。
\subsection*{2021-06-17 22:20}

是的。其實這事我從部落格早期一直説到現在,不知多少次了。半年前那篇《從假大空談新時代的學術管理》,重點雖然是基礎科研,也還特別把美國半導體產業政策的成功經驗又談了一遍;後來回響很大,聽説有不少讀者轉發給認識的官員,事後中國產業政策的計劃人還出面接受訪問,為自己辯護,所以上達天聼是合理的推測。再加上最近這些省級胡搞的結果紛紛爆雷,做出政策轉向雖晚不遲。
基礎科研和外宣雖然獲得新的重視,改進的步驟還不明確,我依舊擔心;這個半導體產業政策卻是已經走上正軌,多年的囉嗦終於有了實際影響,讓人欣慰。
請尊重《讀者須知》第5條規定,選擇相關正文來發留言。

在可能的主導人選之中,劉鶴是唯一兼有經濟學底子和層峰的信任,又不迷信自由主義市場的人,所以他來做這個工作很合適。
\subsection*{2021-06-17 07:18}

目前美元依舊是絕對强勢,只能蠶食;若是美國跌入嚴重的經濟危機,則可以鯨吞。這兩者的實際戰術,有所差異,尤其是下一個經濟危機的細節還屬未知,所以我只討論前者。
貨幣的國際地位體現在許多方向上,包括貿易、定價、儲備、兌換、銀行等等,彼此互相加成,所以打擊美元也必須全面出擊。
在貿易上,中國顯然還沒有用心用力地去美元化;照理這是最基本的手段,俄國已行之多年,所以從這裏就可以簡單看出中共高層並沒有下定戰略決心。其實正如我以前討論過,不只是貿易,像是援助和貸款都不應該繼續使用美元。
我說“定價”,指的是大宗貨品,尤其是期貨。中國近年建立了内部自有的期貨市場,固然有些許斬獲,但金融市場有很强的慣性和網絡效應,如果只靠自己對抗全球,注定會頂上很低的天花板。“兌換”也是同樣的情況:我以前提過,建立外匯期貨市場很容易,吸引交易量卻極難,連人民幣換歐元都還是以美元為中轉最方便划算,那麽人民幣直接兌換其他貨幣的期貨當然也是擺設而已。正因爲金融市場的强大網絡效應,交易越集中於少數貨幣、效率越高,所以連兩種國際貨幣都不容易做到,更別提“多元”;而偏偏中美博弈已進入高峰期,打擊美元的關鍵時段是未來15年内,所以中方實在沒有餘裕慢慢提升人民幣的地位;短期内唯一有實用性的可能替代,只能是歐元。
在貨幣儲備上,各國中央銀行原本就趨於保守,IMF又被美國實際把持,中方能做的,主要是在美元熱機循環到了收割階段可以出手截胡。中國的私營財團受體制規範,先天只想著洗劫國内經濟,然後伺機轉移資產到國外,在美國財閥眼中是韭菜而不構成競爭力量,所以只能由國營金融財團出手;這需要人才、組織、資源和策略的預先準備,我怕的是如同最近的外宣一樣,臨時抱佛脚會弄巧成拙。
在銀行方面,美國削減歐元威脅的手段之一,是用各式各樣官方和非官方的手段來打擊歐系銀行,在過去十幾年非常成功。這其實是中國對歐外交的另一個嵌入點,中歐投資協定是很好的開始,但中方似乎是出於外交戰術考慮而做出的讓步,並沒有理解到金融戰略上的意義:當前全世界勉强算是能挑戰美系投資銀行的,只有歐系,這是又一個必須藉助歐元來打擊美元霸權的原因。中系銀行和影子銀行可以出手的方向,是到亞非第三世界去資助、控制新生的互聯網支付系統,這顯然對去美元化的努力會有反哺作用。
以上談的是中方單獨執行的可行方向;國際上的聯盟協作可以提供新的可能。兩年前曾有金磚貨幣的相關討論,這因爲印度公開全面投入美國陣營而自然作廢。但俄國是去美元化的急先鋒,中國與其合作是理所當然的事,至今遲疑不決是很嚴重的失誤;南非和土耳其等等區域性强權也可以爭取,畢竟他們正是美元對外收割的主要對象。
\subsection*{2021-06-10 11:42}

你的留言讓我很爲難:基本思路是你讀博客的心得,例證應用卻加了一些你自己的觀察,整體論述沒有大問題,如果是獨立的文章,可能比許多網站上的評論精確深入得多,但放在博客這裏,就只是一篇模擬作業,並沒有本質上的新意突破,裏面也沒有什麽博文舊有論點需要修正。
《讀者須知》裏第3條禁止讀者自説自話,你的留言不算犯規,但是一樣是我沒有興趣深究的細節和復述。第4條禁止讀者以導師的口氣對我下指令,你反過來以學生的姿態把一份我沒有出的作業交上來,要我批改,但這個博客不真是網課,教育性的效應,只能是附帶的。換句話說,我對留言的回復應該是新的解釋,或至少對既有細節有所澄清,總之是必須對博客的内容有增益。偶爾有讀者留言超出我的能力或所知,那麽也可以Stand on its own,但你所談的還達不到那個程度。
你寫了兩篇,顯然花了不少心血,雖然我都不想評論,還是只刪了第二篇,這一篇留下給大家參考。請體諒我時間精力有限,專注在對多數讀者有意義的話題上,並且精簡你的論述。
至於我最近幾年對中方逐漸有嚴厲的批評,正如DesertFox所説,是因爲美方攻勢急速升級,不但中國舊有的韜光養晦、以靜制動路綫不再合適,既有官僚體系的動員學習過程卻漫長低效,而且美國這些猛攻也自行露出要害破綻,然而時機稍縱即逝,不加把握實在可惜。
習近平一再證明他有足夠的眼光、決心和手腕來做最困難的改革,像是貪腐官僚和壟斷性企業,全世界都只有中國能有效打擊壓制;封建學閥和商界騙子相形之下,抵抗力量微不足道,處理難度照理是低得多,然而他們卻能長期、反復地做出極爲嚴重的危害,這顯然和前述的對外策略一樣,是認知盲區的問題,剛好都是我能貢獻、也的確花了許多年解釋的重點,只是正面的成果還有待確認。
\subsection*{2021-06-10 10:47}

如果一個江湖騙子,能夠一連在兩個省份,用同一套伎倆,各騙了幾千億,不管背後的機制如何,這個體系有必要做出嚴格、深刻的反省。尤其這個騙子和他的前輩們,沒有一個被追責,更是匪夷所思,典型無政府體制的徵兆;所以我以前曾經説過,中國政府的管理往往不是太嚴,而是太鬆。另一個類比,是2008年金融危機之後,美國銀行巨頭也是沒有一個被追責,但那顯然是巨富竊國的後果,中國政府對財閥能出手、肯出手,對騙子卻百般維護,我已詞窮,或許可以說“匪夷所思的平方”?
同樣的,既然以往高鐵和電力供應已經實踐過正確的發展路綫,而汽車工業則完美地闡釋了反面案例,還不把半導體當一回事,放任自由市場機制來搞,同樣也是匪夷所思的決定,其失敗是完全可以預測、也的確被反復預測的定局。
\subsection*{2021-06-09 23:15}

這個博客的終極目標是對人類社會做出最好、最大的可能影響,所以文章寫作向來是以政策建言為基本設定。我最近解釋過幾次,博客建言的重點在於診斷和預後,至於處方,執行者應該有自行決定細節的空間,所以我一般只簡單舉例示範。
美元霸權是個典型的案例:中方顯然連應該出手的認知都沒有,所以解釋其必要性遠比詳列執行方法重要得多。其實Trump早已不分青紅皂白、不顧中方的妥協和退讓,得寸進尺,把美方的牌都打光了,中國打擊美元根本不會有更壞的後果。這種死豬不怕開水燙的邏輯,俄國人早已實踐示範,公然高調地去美元化,結果美國根本就沒有出手、也無從出手。這裏有部分原因是美國經濟學界有共識,知道美元霸權的副作用之一是貿易逆差,既然減低逆差、重建本土工業在Trump之後成爲全美政壇無分左右的政治正確,美元霸權也連帶被視爲雙面刃,在政權對外折衝的過程中,優先順序反而還不如美國互聯網企業在歐盟的逃稅自由來得重要。至於如何把美元推下神壇,俄國也同樣做過示範;他們那個中央銀行行長是個厲害角色,中方只要願意合作,她自然能想出各式各樣的花招,用不著我來越俎代庖。
外宣實在讓人看得着急,我才被迫反復地談策略細節,其實道理簡單得很:堅持理性的陣地,把中美對抗重新定格為科學專業與政治宣傳之間的搏鬥,然後中方自然獲得一群極爲强大的盟友,亦即全世界所有沒有預設政治種族偏見的科學從業者。一旦明白這個原則,如何執行就順理成章;我真沒有預期這麽清晰淺顯的道理會如此難以理解,唉。
我知道台灣的有識之士,早已被愚民噤聲;兩代之後,回顧“綠色文革”期間的瘋狂,或許歷史學者會引用博客這裏的討論,來證明台灣的理智火苗並沒有被徹底毀滅。
\subsection*{2021-06-09 20:56}

你的觀察不是錯覺。
還是以半導體產業爲例:我在2015年曾估計臺積電的霸業穩固,要到2025年前後才會有變數,但是現在來看,把期限延展到2030年(林毅夫最近說中國三年就可以替代ASML,我絲毫不知他的依據從何而來)應該都還不夠大陸的晶圓廠追上臺積電的製程(然而要把那個預估算入我出錯的10 \% ,還言之過早,因爲我當年的預測不只基於產業競爭,也包含了國際政略和軍事情勢等等考慮)。這裏内外的因素都有:對外是Trump啓動對中國的貿易戰,中方在戰略本質上都錯估,把謀殺當成搶劫,沒有立即對等反擊,鼓勵美國持續升級,最終找到科技戰的軟肋要害;對内則是最核心的產業政策也迷信西方自由主義經濟學,荒腔走板,徒然圖利財閥,這不但已經浪費了幾千億經費、大約10年時間,而且把從國内外引進的專業人才錯置,一次性地消耗了他們對先進製程的暫時掌握。這兩方面都被我事先就反復評論,預先解釋正確路綫,但高層智囊顯然有盲區,一而再、再而三、三而四、四而五、五而六、六而七、、、(單在半導體投資上,至少已經有7、8個大規模爆雷)地重犯同樣錯誤,這是把改開政策初期有效的外交和經濟原則,推到無限極端,因循守舊,不知變通的結果。我曾講過Colonial America一匹馬跳河的故事,這些人的思路就和那匹馬完全相同,而我的確是沒有想到負責14億人福祉的執政精英,會連續多年堅持向美國馬學習智慧。

年紀大的讀者不一定記得每個留言回復,這裏我重述一次那個故事:17、18世紀美國有一個醫生,一天晚上收到消息,必須緊急出診,他騎馬來到附近一條小溪,發現暴雨把便橋冲壞了,他不得不冒險强迫那匹馬跳躍過河,還好沒有出事。第二天他從患者那裏返家,到了小溪旁,便橋已經修好了,他正鬆了一口氣,跨下的馬忽然加速,在前一天的同一地點奮力一躍。。。
六年前我假設智商等同馬匹是普選國家才有的問題,或許我是錯了:中國的崛起可能還是要靠大家比爛。當然我會繼續努力教育華語世界,但是提升民智,不能只靠一個人,所有的理性知識分子都有責任。
\subsection*{2021-06-06 11:50}

雖然我向來不喜歡囉嗦,同一個道理説過一次就不太想再重複,但因爲這個議題很重要,特別一連寫了三篇文章,卻沒有立竿見影的結果,有點灰心。不過或許中國體制對行政細節講究共識,所以需要額外的時間來消化吸收合理的建議。
此外,我以前解釋過,診斷和預後主要看客觀環境而定,所以旁觀者只要有足夠的邏輯分析能力,也可以達到很高的確定性;處方卻必須考慮許多主觀因素來優化,所以外人頂多只能給出例子來做啓發。在這件事上,未知的主觀因素包括團隊既定的任務、既有人員的專業偏向、外交總戰略的明確定義、與外國政府之間的默契、和國際組織的私下協作、以及最重要的:國内研究人員對病毒起源的追蹤分析。我通常會偷懶,把診斷、預後和處方都放在同一篇文字裏,但是讀者應該留意,明白前兩者可以有很高的精確度和完整性,後者卻只供參考。
美國當然不會在這件事上放過中國,暫時的聲浪消退只是在為下一波攻勢做醖釀準備。這一次民主黨主導的宣傳打擊,吸收了去年的教訓,把重點放在尋求科學專業人員的支持,中方失去了以往Trump政權的助攻,如果仍然不好好應對,後果會很嚴重。

既然民主黨捲土重來,把重點放在尋求科學專業人員的支持,那麽中方自然也應該考慮如何爭取國際學術界的認同。這裏我們必須設身處地,理解學術大佬的心態:首先,部分有種族或政治偏見的,無可挽回,可以徑行忽略。其他沒有先天偏見的,則自然會重視科學誠信原則,中方必須盡可能以事實和邏輯來服人。但即使是政治中立的學者,往往也在乎行業本身的前途:如果每次有新病毒疫情爆發,附近的實驗機構都自動成爲嫌疑犯,顯然不是學界人士所樂見,這是中方可以運用的心理按鈕(Button),不過比較適合在專業論壇上提起。
\subsection*{2021-06-05 23:47}

你如果反復重讀我對重要事件的長篇討論,可能會注意到我的考慮比其他人深了不止一個層次,而且包容全面的相關細節。這其實是我的思考習慣,只是在博客做評論的時候,往往一開始會偷懶只把最終結論(去年我不是簡單地說,趕快認了“野味市場起源論”嗎?)簡單寫下來,這是因爲要解釋給別人懂,比我自己想清楚,要辛苦得多;只有在被讀者細究或者事關重大的前提下,才會花時間去討論深入和細緻的邏輯推演。例如那個東南亞免疫假説,去年我只寫一個段落,等到有人再問,我又剛好有時間,才囘了四個段落的版本。當然那也不是全部的考慮,但要寫得更詳細,工作量隨指數增加(別忘了,我寫作和一般學術人剛好相反,追求字字是乾貨,廢話佔比是零,在這樣的自我要求下,解釋同等細緻的内涵,文章壓得越短,所需的工作量越大),而我的時間有限,必須要用在閲讀和思考上。
“新冠起源”是個重要話題,我原本在準備寫正文,剛好有人談起,就直接囘在留言欄。你所歸納的,正是“一個段落”的版本,雖然抓到重點,但不可能包括所有邏輯推演的細膩之處(當然我只寫了七個段落,所以也不可能是真正完整的論證,例如美宣的强勢,也體現在它可以犯錯重來,中方沒有這個餘裕,因此更加不應該去搞陰謀論),在轉述的時候,必須留下鏈接,讓程度高的聽衆能滿足自己的疑問。
做視頻也可以。我對文章轉發的要求是給出鏈接,並且不做扭曲;但是視頻不是念稿,必然得做出新詮釋,所以你必須對聽衆特別强調你的視頻是你個人對我文章内容的二次創作,不能保證精確或完整。如果聽衆有疑念,可以追溯原文,但務必記得,我只對自己的文字負責,別人的詮釋不是我所能控制的。
\subsection*{2021-06-04 23:37}

中方覺醒太晚,一方面任由外宣墮落,另一方面也以爲美宣抹黑俄國事不關己,所以既沒有組織、人才、能力,也欠缺國際上的協作。英美的宣傳機構有百年的經驗,二戰和冷戰更是綁定了歐亞非的一大群死忠聽衆,所以宣傳戰還未開始,中方已經面臨少輸就是贏的局面。
一旦仗打起來,中宣部、學術人和媒體界不堪一用,只好把非專業的外交人員趕鴨子上架。他們不懂現代營銷和公關的技術,依照既往的習慣照本宣科,其實是必然的,無可深責;你看正文這裏給的建議,每一條背後的概念都來自商學院,外文系沒有教過嘛。不過習近平把任務交給外交部,最低的期許應該是要遵從國際交涉的基本原則,不能拿槍打自己的脚;換句話説,要明白在話語權上敵極强、我極弱的現實,盡可能將損失局限在無可避免的最低限度上。
很不幸的,就連這個最低期許也沒有成功;我舉一個博客這裏已經詳細討論過的例子:新冠病毒的起源。一年多前疫情剛在武漢爆發的時候,因爲活體野味市場是一個爆發點,全世界很自然地依照SARS的往例,懷疑病毒的物種跳躍在那裏發生。這其實是天賜的好運:自然突變是隨機的,根據國際慣例,國家完全沒有責任;如果美宣硬要找茬,不但歷史上有1918和2009年的瘟疫源自北美,可以簡單反擊,而且2003年SARS的外宣經歷也沒有傷筋動骨,只要再高調取締野味交易自然就能化解敵方的抹黑,獲得國際中立人士的諒解。
然而外交部明明知道中國對冠狀病毒的研究中心之一剛好在武漢,卻一點預見能力都沒有,不瞭解趕快把“野味市場起源論”板上釘釘已經是最佳結果。好在Trump政權極爲幫忙,不但立刻泄露了可能的抹黑方向,試圖扯上武漢實驗室,而且還特別示範如何拿槍打自己的脚,用上了“生化武器論”。既然新冠病毒本身的基因組合顯然是天然的(這是又一個天賜的好運;要是事實相反,中美雙方互相指控生化武器來自對方,美方外宣必勝無疑),那麽全世界的生化專業人員自然成爲中國外宣的無償盟友,紛紛高調出面批評共和黨胡扯。
在這樣的背景下,外交部只要簡單重複一句“我們尊重科學專業常識”,就可以把對方的陰謀論攻勢化解於無形。偏偏有人把習近平“不能退縮”的指令,誤解為要無腦出擊。中國有14億人口,其中有多達7億的智商在100以下,而且正是這群人特別喜歡上網胡扯,養肥了無數網紅大V;但是14億人中只有個位數字的外交人員負責代表國家發言,他們理應展現出比全國中位智商高出5、6個標準差(假設有7個外交發言管道,7/14億=1/200000000,對應著5.73個標準差,也就是智商186的等級)的戰技水準,你怎麽能去在乎那些既不懂生化、也不懂外宣、更不懂戰略的低等半數人口的集體幻想?
Biden上臺之前,已經明智地和共和黨劃清界限,執政一開始也沒有在這方面出手的跡象;我原本私下指望外交部發言人扯上陰謀論的失誤,已經被Trump政權的類似錯誤所抵消。很不幸的,民主黨的外宣機器在幾個月的深思熟慮之後,認清了正確的戰術方向,決定重新發動攻勢,而且這次有備而來,所提出的指控,是與已知科學事實全無抵觸的“學術樣本意外泄漏論”,那麽依靠美國的既有外交影響力,再加上歐美學術大佬的愛國情操和種族偏見,很自然地扭轉了國際輿論的態勢,造成新一波的外交壓力。其戰略目的是要强迫中方開放所有原始病理樣本,如果能和武漢實驗室扯上任何關係,那麽法理上中國就必須對整個新冠疫情負全責,這是災難性的後果;如果扯不上實際關聯,可以設法栽贓,最不濟也可以含糊其辭,在世界人民心中留下對中國的疑慮和敵意。
因爲後果太嚴重,中方不能冒那個風險,所以必須堅決拒絕進一步深入的國際調查。於是外交部的責任,就在於對這個決定來向國際社會做交待,把它合理化、合法化。回頭去炒作“生化武器論”是絕對的自殺性行爲,立刻自外於全世界的正義和專業力量。至於正解,這裏事關重大,不能拘泥於科學和道德原則,必須把我以往反復討論過的狡辯術和破解法,正反兩方的技巧一並綜合運用。最基本也最有效的狡辯術,是隱性地轉換話題。在這個案例上,對應歐美大衆媒體,應該把他們去年指責中國人吃蝙蝠的文章高調挖墳鞭尸,指明其與新指控之間的邏輯矛盾;然後一有機會就提起伊拉克的WMD,嘲笑美國在聯合國撒謊的歷史,並且反問那些媒體在2003年支持美國指控的報導是否依然有效。若是被逼到墻角,那麽只好自説自話,假裝美方的指控依舊只是共和黨的“生化武器論”(好在Biden偷懶,讓共和黨勢力打前鋒,中方可以挑選對方最離譜的論述;你看,即使在外宣上强勢如美國,搞陰謀論也只有反效果,何況是弱勢的中國),復述病毒天然來源的科學證據。遇到專業聽衆,則必須回歸理性論述,引用Russell’s Teapot原則,特別强調沒有正面證據,空口說白話,被控方沒有責任應答,就像美國右翼指控Hillary曾經在DC一家Pizza店的地下室,關押著未成年的性奴,Hillary也並未出來提供反證。
\subsection*{2021-05-29 16:13}

Anglo-Saxon靠殖民屠殺致富、種族滅絕立國,是博客一貫重點介紹的歷史事實;New Zealand沒有做得徹底,只是The exception that proves the rule。偏偏他們還有臉在歷史課本裏指控西班牙的Conquistador全面屠殺南美印第安人;然而現代Bolivia的人口,90 \% 以上還是原住民血統,美澳除了Alaska是14 \% 、Northern Territory 25 \% 之外,哪一個州超過10 \% ?而且美國的數字來自人口普查,由國民自報,然後官方有意誤導,把任何有一點血緣的都算成100 \% 或50 \% ;甚至純白人冒稱原住民來騙補或自我陶醉,例如Elizabeth Warren,也是很普遍的事,所以實際比率很可能是低於官方數字(全國號稱純種原住民佔人口0.9 \% ,混血0.7 \% )。其實就算相信這些不嚴謹的統計數字,NZ也只有15 \% 、Hawaii 9 \% 、Tasmania 4 \% ;大家可以看出,這裏列舉的幾個例外,都是小島、沙漠或苦寒之地,之所以沒有把原住民趕盡殺絕,是因爲農業條件不好,沒有太多人想要移民,也就沒有必要做種族滅絕,並非忽然轉性施仁。非常可笑的是,正因爲Anglo-Saxon的種族滅絕最徹底,所以現代殖民地原住民所造成的麻煩也就最低,反而給他們膽氣來頤指氣使,在國際上冒充白左聖母。
英文輿論系統不必官方出面撒謊就成功建造錯誤印象,是典型的宣傳誤導伎倆,博客的讀者應該都能簡單看出是個騙局。
\subsection*{2021-05-28 21:36}

懂德文和法文的讀者,應該會看到些事先的公開討論。我沒有這樣的外語能力,原本不適合談這件事,還好他們的内宣不一定是真實考慮,所以可以在這裏做點簡單的猜測和評論,雖然必然有細節遺漏,但仍可能對真相做部分的描述。
如同美國,歐洲的先進核心近年也經歷了左右黨爭撕裂社會的窘態;雖然各國内部政黨的立場不完全相同,所有右派的主要訴求卻都是反對移民。2016年英國脫歐和本周瑞士放棄與歐盟升級條約關係的基本民意動力,都是重商主義國民不願意將社會福利賦予來自東歐的歐盟公民。但是其他地區主要的移民問題卻在於外來的穆斯林,包括德法在内。
極右派憑著反移民迅速擴張勢力,與其對立的白左也就可以旗幟鮮明地走向相反的極端(這裏討論的是正常智商國家,現實中也有奇葩,像是捷克,就能夠歇斯底裏地在國内排斥移民、同時卻也心安理得地在國際上搶當聖母;相比之下,瑞典反而還有邏輯一致性),傳統的溫和派主流政黨幾乎都在急速萎縮之中。有興趣的讀者可以拿歐盟的前五大國家(德、法、意、西、波)來詳細檢驗一下:他們的極端政黨不是已經執政,就是在下一個選舉裏面舉足輕重。
德國的中左派SPD已經式微,今年選舉是執政的中右派CDU/CSU對抗白左主導的綠黨;後者幾乎可以確定會是下一個執政聯盟的成員之一。法國傳統的兩大政黨LR和PS更慘,同樣都已經淪爲小黨,明年大選看來會如同上一次,是中間派的Macron與極右派的Le Pen之間的競爭。
在白左思想已經正常化、主流化的政壇大環境下,德法中間派執政黨雖然分別面對著極左和極右的對手,但同樣有必要做出左右挪移,盡可能搶占民意光譜上的空間,所以做出象徵性的道歉和聲明,不但可行,而且是很自然的。

我預期有些讀者會把中國在非洲的影響力和這件事聯係起來,但這是典型的自由聯想,沒有嚴謹的邏輯因果關係。認爲德法是爲了與中方爭奪影響力才道歉的人,必須先回答:1)爲什麽早已撤出非洲一個多世紀的德國,會有和法國一樣的考慮?2)英國在2013年就做過類似的道歉,如何解釋?3)爲什麽他們只針對一個小國或一個部族的單一屠殺事件,而不是對整個殖民歷史做出道歉?4)爲什麽同屬歐盟的意大利、西班牙和比利時沒有發聲?5)Macron只説“法國有責任”,這真的對非洲人的心理有足夠的安撫作用嗎?
中國的存在,當然幫助非洲國家在對歐談判中能夠挺直背脊,但他們的訴求主要是在投資和貸款,不是道歉,所以頂多只能說有點間接的效應。
\subsection*{2021-05-28 02:38}

1. 同一句話,是否對照歷史背景細節,會有不同的理解。這裏的重點在於,有特別歷史的不是New Zealand,而是澳洲。澳洲在二戰期間開始脫英入美,二戰後迅速與美資進一步結合,然後又出了Rupert Murdoch,成爲現代Anglo-Saxon右翼民粹的發源地。這些因素,在NZ都不存在,所以相對看來,後者可以説是“有獨立自主的傳統”。
2. “Anti-semitic”在1970年代之後,早被以色列游説集團濫用到與字面意義完全脫節的地步;現在它的意思是指任何批評以色列的人。
猶太人對美國政治、經濟、法律和社會的全面掌控,我在2015年就仔細解釋過了。你會再問這個問題,顯然是沒有讀完博客。違反規則,禁言一個月。
\subsection*{2021-05-27 14:15}

Fauci一輩子在聯邦官僚體系當主管,當然是政治至上。
目前真正用力的推手,還都是共和黨系,民主黨系只在助攻;這並不一定代表後者真心參與這個宣傳攻勢,也可能只是想看熱鬧,所以還不能確定會鬧到多大,WHO的態度尤其是一大變數。我覺得這事又閙開,中方反應依舊不得其法、以致火上加油是主因:否認本身就是默認話題合理、值得討論,連帶地也接受闢謠證僞的責任;其實很簡單地根據Russell's Teapot原則,强調美方有義務提出具體證據就行了,順便可以指出美國情報機關的報告 Is not evidence but in fact more often evidence to the contrary。這些都是正文裏已經反復强調的重點,但是中方智庫顯然不想懂就是不想懂 。
至於後續發展,這個博客的討論,在於政策建議,順便做公民教育,向來要求有足夠資訊來達成大機率的邏輯論斷(參見《讀者須知》的70 \% 最低標準)。因爲關鍵事實還未發生,我已經要求大家靜等幾天,觀察事態發展。按耐不住、非要立刻擺龍門陣的人應該到其他網絡管道去。
\subsection*{2021-05-16 21:36}

要掩蓋真相,並不只是對它扭曲,而且要散佈無數彼此矛盾(而且有不同專業層次,見下文)的謊言,讓一般人接觸真相的機率降到極低,然後才針對性地抹黑實話和說實話的人,如此一來,連偶然撞見真相的讀者也會心存疑慮。正是因爲這個危險,我寫作特別嚴謹小心,避免留給有心人口實。
散佈謊言也有很精細的技巧:必須層層佈防(Defense in depth,又譯為縱深防禦),視讀者的教育層次和專業水平,各有適恰的説法。這裏底層不必太費心,靠著民主、自由口號和開國神話,自然會有火鷄(例如龍應臺和楊安澤)主動帶領其他火鷄要求過聖誕節;主流媒體的偏見也無須太過隱蔽,可以用“新聞自由”為藉口,把主觀和客觀議題混爲一談;關鍵在於最高端的大衆媒體,例如《經濟學人》,得要先在專業議題上維持超人的資訊搜集和深刻的學術探討能力,然後夾帶少量的私貨,而且持之以恆、長期保持一致性,才能成功洗腦教育程度最高的讀者群。
\subsection*{2021-05-16 13:28}

在我解釋清楚之前,(至少在公開管道上)只有我一人懂;解釋過之後,網絡上自然有無數人轉述。例如美元如何搜刮世界,在2014年之前,華語界沒有人能把機制說對的,現在《觀網》隨便一個讀者留言都談得八九不離十。他們大概不知道知識的源頭何在,但真理一旦說明白,感覺就是天經地義,即使幾分鐘之前還是完全想不到的。
做爲一個台裔美籍的陌生人,我不可能有直接可靠的政策建議通道,但是如果我的解讀成爲網絡上的常識,那麽至少幕僚和智庫會受影響,間接地加速決策層采納正確的認知和對策;例如我談美國的社會真相和對華戰略意圖,很明顯地在2016-2019年期間有星火燎原的效應。
我寫了七年,並不是像學術界人士那樣反復推銷同一個理論,而是解釋清楚一個道理之後,就轉而批判下一個公共意見的誤區,周而復始,至今已經澄清了幾十件重要的政治社會議題。在這個過程中,唯一一再重複强調的,是科學的方法和原則,這是因爲它們真正有永恆的普世性,而且是獲得正確結論的關鍵工具。
\subsection*{2021-05-15 17:30}

有關該如何建立獨立分析時事的能力,以突破謠言和誤解的烟幕,以前片片段段地反復討論過。這裏你針對性地問,我就系統性地回答:這需要三個條件:1)對社會經濟規則和人性自然趨勢有足夠的經驗和認識,能簡單看出不合理的論述;2)對各類主要議題的關鍵細節,不論屬於什麽專業,都至少有定性的掌握;3)熟悉邏輯思辨規則,能高效地抽絲剝繭,抓出因果關係的主軸。
你問如何獲得這樣的能力;很可惜的,這裏的答案是你的問題有一個隱性前提,就是一般知識分子也能做到,而這並不成立。你只要仔細想想就會明白,上述三個能力中每一個所需的天分、機遇、時間和努力,都不是普通人能企及的。我個人向來懶得去追熱點、人云亦云,正是因爲整個華語世界只有我一人能把這些事解釋清楚,才被迫花時間來寫作。
寫作的意義是什麽呢?一方面是對當權者建議正確的策略,另一方面則是教育華語界的高級知識分子,在培養前述三種能力的過程中,做出示範和引導,因爲網絡雖大,很不幸的這類示範和引導卻只有這個博客才能提供。然後視個人資質、環境、心態和努力,在我親自回答各種疑問之後,或許有人能在幾年内學得我自己花了幾十年培養的能力,從而成爲建設科學理性社會文化的尖兵。
回到你問的Q1,在西方宣傳和網紅文化不斷有意無意扭曲新聞、製造謠言的環境下,如何獲取可靠的信息?答案正是這個博客的宗旨:堅持事實和邏輯的科學態度。我之所以幾十年不間斷地努力提升這方面的能力,是因爲天生喜歡求真,破解那些愚昧和虛僞的謊言是很有趣的挑戰。做爲新來者,你慢慢地學習嘗試,有疑問在博客這裏可以得到可靠而且深刻的解答。
至於Q2,從功利的角度來看,個人的確沒有必要關心世界大勢;多數群衆絕對可以做出這個選擇,我只建議他們不要站在笨蛋峰上,高聲回應網紅大V創造的謠言。但是對在乎公益(別忘了,這正是“好人”的定義)的少數人來説,有了這個博客的指導培訓,就沒有藉口坐視壞人和蠢蛋占據輿論的制高點,侵占腐壞我們共享的人類社會。
\subsection*{2021-05-08 15:44}

很困難、需要很多步驟、時間跨度太大、國際環境背景要求很高,和博客這裏要求高度確定性的原則有矛盾,沒辦法一次寫出完整而嚴謹的策略建議。反過來看,因爲它們都是正確的施政方針,我在留言欄裏常常會討論其中的各個成分。換句話說,其實都提過了,只是沒有整理出來、放在一起、加上“解決貧富不均指南”的標簽。
簡單來説,首先必須全面清除資本在國際間逃稅和躲避監管的隱藏處所(例如香港),而這必須有中美歐三方共識才可能發生,那麽在中美霸權博弈階段就不可能有大的進展,但是像最近美國提議制定全球企業稅最低標準,起碼面朝正確的方向,所以中方應該予以支持。
然後在國内經濟監管上嚴打獨占性企業(別忘了,獨占是自由市場經濟的必然結果);全面整頓稅務,從收入稅和交易稅轉向財產稅(包括房地稅;可以有較高的個人免稅額,避免傷害中產階級,不過中國民衆沒有誠實繳稅的文化傳統,所以依舊會有很高的政治代價)。
在宏觀貨幣管理上,換檔到較高的通貨膨脹平衡值(用意在於減低資本纍積的速度,但這只是一小部分,更重要的是消除大戶和散戶之間投資報酬率的差距,亦即必須解決割韭菜現象,所以就有接下來的金融改革項目);在金融上,消滅所有不為實體經濟服務的不勞而獲方法。
在科技研發上,維持高效和競爭力,這要求建立良好的學術界風氣,消弭造假、誇大,打破學閥山頭勢力。這是博客近年的重點話題之一,讀者應該很熟悉了。
在社會階級的管理上,必須提供優質、廉價的全民公共教育,以及入學和參政的絕對公平門檻,以保證垂直流動性;在醫療之類的基本社會服務上,加大投資,盡可能消除城鄉差距。
前面提的每一項,都是巨大的改革工程,而且面臨既得利益集團的强大阻力。
我光是談學術風氣問題,就用了五年時間、十幾篇文章才解釋清楚;中國政府采納實行,則一點苗頭都還看不見。解決貧富不均的難度和複雜度,又高出幾十倍,目前我只能指出大致的方向、以及可以立刻實現的部分政策。這個龐大艱巨的任務,在我的壽命期限内是不可能完成的,那只好等待年輕世代持續的努力。
\subsection*{2021-05-07 19:55}

“中性”指的當然是大致的、整體的中性,而不是絕對的、片面的。瘧疾篩選出鐮刀血球基因,幫助其擴散繁衍,正代表著整體上不止是中性,而且是正面的;如果這個基因曾經只導致貧血(亦即瘧疾是全新出現的環境大變化;事實上瘧疾歷史極爲長久,鐮刀血球突變應該是事後發生的突變,根本就不適合作爲我們討論的案例),或者其致命程度大幅提高,它早就被演化過程淘汰了。你想不出合適的例子,是因爲Survirorship Bias:真正負面的基因必然早已消失,所以不容易找出例證。
我認爲在最高學府結黨收徒,傳播敵人有心設計來顛覆其他國家、並且已經反復成功過的惡性宣傳,是絕對負面的;那麽中國或者自清、或者將其隔離、或者等候演化天擇來將整個國家文化淘汰,是嚴密邏輯的自然結論。把這種Clear, severe and immediate danger和“王道”之類的文化趨向混爲一談,是典型的邏輯謬誤;你應該重讀《常見的狡辯術》。
\subsection*{2021-05-07 10:41}

歐美真正的純學術界,其實反而沒有什麽逃避“動搖西方根基的研究結果”這類的現象;例如對美國資本主義制度的内部矛盾,向來都有深刻的討論,只是被排斥出主流,上不了臺面,有興趣的人必須下工夫去找罷了。這裏我所針對的,是應用類、也就是學術和媒體的交界面,各式各樣扭曲印象和結論的小動作,到處可見。
中國現代文人研究西方制度,是極度的懶惰,不但只看主流、臺面上的半學術性宣傳稿和宣傳性的僞學術論文,而且連獨立的邏輯判斷都不做,學而不思,只獲取浮面印象,正好落入對方有意設置的陷阱。
建立自己的統計數據庫,正是獨立自主分析思考的基礎資源。中國剛剛在芯片生產器材的對外依賴上吃了大虧,國家治理的哲學和原則,怎麽可能不如一個工業類別重要呢?
\subsection*{2021-05-06 17:19}

其實大陸也是一樣的:一群被英美宣傳徹底洗腦的教授,盤踞在北大、清華這樣的最高學府,不斷毒害未來國家精英階級的腦子。就算是要尊重學術自由、保留異見供參考,不能把這些人送到沒有學生的研究機構(例如智庫,反正這些智庫的產出原本就和現實脫節)嗎?任由他們以指數繁殖,真的是維持多元意見的必要手段嗎?
在自然界的演化過程中,一個健康的族群必須事先纍積突變,維持基因的多樣性,在環境發生大變動之後,才能有適用於新環境的現成基因特徵來提供生存力。一個大國的學術界,也是如此。但是,適合纍積的突變/異見,必須至少是中性的,不能先天就有極大危害,否則就應該迅速淘汰。在中國面臨現任霸主的深刻敵意和全面打壓環境下,保留這些帶路黨實在是極爲奇怪的選擇,更別提放任他們污染下一代領導人才。
\subsection*{2021-05-01 05:41}

德國的情報系統是所有政府單位中被英美滲透最嚴重的一個,沒有之一,不能對其有任何指望。
2003年,德法能夠看清真相,有幾個特殊原因,這裏我依重要性的反向來羅列:1)在阿富汗之後,體驗了海外戰爭的難處,對入侵另一個中型國家有疑慮;2)美國内部主流傳媒已經有少數披露WMD情報可疑的討論;3)負責監督伊拉克核子設施的國際原子能組織有不少歐洲籍主管,可以拿到第一手的資料;4)當時德法的領導人水準高,有戰略眼光和魄力。
2021年Biden試圖拉攏歐盟,在經濟外交上對中國宣戰;德法決定對中國采取口頭譴責、實質合作的姿態,原因只有類似上述的(1)(已經經濟制裁俄國,明白兩敗俱傷的道理,而中國體量更大)和(4),並沒有(2)和(3)。九月之後,若是綠黨上臺,就連(4)也不見了,所以是個可預見的危機。
\subsection*{2021-04-28 23:17}

這裏的問題在於,一方面我沒有時間和興趣做這種整理闡述的工作,另一方面我對這些問題的思路,很大成分是天然自發的,要完整把結論歸納出來,反而並不容易;以下我詳細解釋一下。
讀者可能因爲我老是說“事實與邏輯”,就以爲我有一套複雜的、系統性的Doctrine,其實剛好相反,正因爲我只抓最基本的原則,遇到任何議題,就必須從最底層的事實現象重新用邏輯推導一次。一個簡單的類比,是大學數學課裏,有很多人是靠背誦定理和應用來過關;我卻從來不會死背任何結論,學習過程只留下一個大致的印象,明白正確的大方向和邏輯架構,解答問題的時候,往往必須把所需定理重新推導一次。但是我的邏輯推演感覺像是反射動作,並不多花時間;好處則是絕不會誤用定理,而且容易舉一反三。
這種思考方式用在社科方面,雖然容易看出新的、全面的解答,但也就沒有固定的敘事結構,也就很難歸納整理。換句話説,不但我每次重訪議題就可能看出新的關聯和推論,即使是既有的描述也是從多個角度出發的,和一般學説的抓出現象一個片面、然後反復搜尋佐證來填充篇幅,完全相反。
\subsection*{2021-04-28 05:59}

你的理解大致正確,但是第一層次不是普世價值。我上月討論過,普世價值只可能來自史前演化過程,所以也必然不完全適用於現代工業社會。真正最基本的目標是社會公益最大化,普世價值做爲精神生活品質的一部分,只是公益的許多成分之一。
三個層次,作爲政治和倫理的基礎,並不是有意設計來專門適用於中共,而是普世真理,任何有心向上的國家社會,都能受益。我做這些建議和分析,不是爲了圖利中國,而是爲了全人類的福祉,必須有一個誠實、理性、和平、公道的新霸主來取代美國。
前天有讀者私下寫信,說我會是一代大師,我回復說我永遠都不會是。這是因爲要成爲宗師,必須花大功夫做兩件事:第一是要把學説整理出來,然後針對當代主流學術界反復闡述推銷;第二是結黨收徒,創立山頭。即使是孔夫子也不能免俗。我則不但志向不在此(而在於有實際影響和貢獻),而且沒有耐心寫學術論文,更加不想玩政治性的門派操作。要是有體制内的年輕官員,想拿我的建議去實用,我會非常高興;如果是學術人士,願意把這裏的思路整理、演繹,成爲主流能接受的形式,然後作爲論文發表,我也歡迎。一個簡單的Citation,指引更多人來博客學習這些道理,就很足夠了。
\subsection*{2021-04-27 23:53}

他們沒説錯啊。國際輿論的確是中國力量最微弱的方向,和英美差了不知幾個量級。美元霸權也的確是相當牢固,連美聯儲都覺得以往的那一點矜持和謹慎其實沒有必要,完全可以全力開印鈔機。這一輪交鋒,中方完全無視我的建議,不吃虧才是奇怪的事。
我的籌劃都假設中方采納理性最優解,這次看來又是做不到了。過去20年,中國的機運實在不錯,就算自己沒有做好(例如沒有準備好外宣、沒有整頓學術界、沒有及早理解美國的用意、沒有盡全力打倒美元),也會有突發事件,至少緩解問題,更經常是把問題代爲解決了(例如911解除了美國主動挑起臺海戰爭的計劃,2008年金融危機延緩Obama的外宣攻勢,Trump當選更是打破美歐既有的同盟默契)。但Merkel的退休可能是個轉折:如果綠黨在九月執政,然後全面聽從Biden,連劣質美元資產都照單全收,那麽即使泡沫爆破,也依舊只是2008年的重演,歐元區受傷比美國還重,反而加强美元的霸權。這次美國的貨幣運作,比12年前放大了4倍,有了十萬億美元級別的新財富注入,完全可以滿血復活,再拖十幾年到下一個泡沫。
我已經50好幾,人生的挫折經歷多了,知道很多事急也沒有用。這些國際策略的道理,七年來講得明明白白,一直免費供華語界參考,其用意是就算我得不到任何利益和名聲,至少把正確認知和策略傳達給掌權者;結果中國14億人,居然連一個能學習上傳的都沒有。有管道、有權力的只拼命想著爭名奪利,吸取公共資源;絕大多數知識份子渾渾噩噩,全然不顧環境的險惡,只想著滿足反射迷思(例如對中醫教的迷信),在公共論壇上製造噪音、妨礙理性聲音的傳播,成爲改革的絆脚石、既得利益者的貼身護衛。我自己孤軍奮鬥、無償付出,已經絕對盡力,無愧於心。如果中國人能改,自然最好,這一波失利應該還不會傷筋動骨,中美博弈只不過有所延長;如果不能改,那麽中國社會的確有很多比美國還要腐敗得多的成分(例如學術界),為整體人類著想,這樣的霸權交替沒有足夠的正面意義,也就沒有什麽好在乎的。
\section*{【公共健康】【財政】一個嚴重的公共健康問題}
\subsection*{2023-03-18 12:55}

你說我古怪,完全正確:世界上事事以是非真假為最優先考慮的人,是極少數中的極少數;非理性或甚至反理性,才是人性的自然。Homo sapiens演化過程中,每次社會規模擴大、分工合作變細,平均腦容積就減小一次,是博客以前介紹過的已知事實。這是因爲一方面,理性邏輯在所有思維模式中是遠遠最困難、最高端、最消耗時間精力的(參考關於AI的既有討論);另一方面,人類社會組織擴大之後,就可以簡單依賴習慣、法規和文化來解除絕大部分人的理性思考負擔。博客之所以反人性而行,在於社會公益恰恰是必須絕對堅持是非真假的重要議題,而在能夠直通執政者之前,有必要先理順輿論、教育知識分子,但這當然只限自願負擔追求事實邏輯巨大投入的少數人。你不願意繼續,是你的自由;而幾年時間下來,博客對你的理性教育徹底失敗,我也的確沒臉收你的束脩,請通過Paypal通知,我會全額退還捐款。
不過把對理性的堅持,也描述成“Cult”“迷信”,就純粹是狡辯術的顛倒黑白了。迷信之所以是迷信,不在於堅持,而在於沒有事實邏輯的根基。因而真正的科學精神,先天就是宗教的反義詞;這是爲什麽我在形容假大空、僞科學詐騙的時候,特別不用“科學教”這個Oxymoron,而是“科技教”的考慮。
\subsection*{2023-03-14 14:25}

好,你是醫生,我理解你對非專業人員發表醫學意見的自動反感;然而你既然是博客的讀者,就應該也知道我不是普通的非專業人員,我説的話都有堅實的學術憑據,而且用詞盡可能的精確。這裏我說的是“只對少數人有效”,你卻似乎看成“只應該對少數人開處方”,其實邏輯上很明顯是兩回事:即使服用SSRI只有小於50 \% 的機率能產生統計上有意義的療效(實際上是顯著低於50 \% ,參見《https://www.ncbi.nlm.nih.gov/pmc/articles/PMC9344377/》),但是因爲(1)事先不確切知道個別病人會屬於哪一類,(2)沒有其他更好的療法,那麽拿SSRI來治療絕大多數的病患依舊是合理的。至於副作用,減量過程有危險當然也應該算。
我對就事論事向來不排斥,不過你先是違反《讀者須知》第八條,這是自動禁言一個月,然後又爲了硬拗寫了長篇大論,進一步浪費我的時間,必須追加一個月。如果你沒有足夠自制力的話,可以再次發言,我不介意幫你拉黑。
\subsection*{2023-03-12 13:55}

據我所知,的確是一個Meta Study,而且不但規模極大,還探索得極深,把製藥商幾十年來挑選數據、以扭曲結論的作爲暴露出來。
你是這個專業的嗎?爲什麽不知道如此重要的新結果?或者我誤解了那個研究?請詳細解釋。

想了想,應該把我所知的一些相關資訊列舉出來。如果我對其有所誤解,請賜教。
1)科普文章《https://www.economist.com/science-and-technology/2022/10/19/how-to-make-better-use-of-antidepressants》
2)數百篇質疑SSRI的論文列表《https://www.survivingantidepressants.org/forum/16-from-journals-and-scientific-sources/》
3)去年新發的兩篇重磅論文,含Umbrella Study《https://www.nature.com/articles/s41380-022-01661-0》和驚動英國輿論的《https://www.bmj.com/content/377/bmj.o1362》
\subsection*{2020-09-08 23:03}

抑鬱症有很多不同的類別,更有許多不同的程度;上次我說得太簡略了,或許給讀者錯誤的印象。
我所指的,是心理學裏的一系列叫做“Illusory Superiority”的實驗,也就是用問卷來調查群衆對自己能力的評估。最早的知名結果來自1977年Patricia Cross(參見https://onlinelibrary.wiley.com/doi/abs/10.1002/he.36919771703)詢問一批大學教授他們的智商是否高於同儕的平均,94 \% 回答是的。後來有人對一家大型軟件公司的程序員發問卷,說你是否屬於最強的5 \% ,結果有32 \% 給出肯定的正面回答。這個現象在各式各樣的行業和群體中不斷被複製(我一直懷疑這和文化有關聯:大部分的論文來自美國,印度的調查結果應該很有趣),唯一的例外是抑鬱症病患,大約只有50 \% 認爲自己强於平均。
至於更廣義的世界觀和人生觀,心理學有Optimism/Pessimism Bias樂觀/悲觀偏見的定義,對抑鬱症的輕重程度也有定量的評分方法(叫做Beck Depression Inventory)。前者對後者的函數,顯示一般人有樂觀偏見,嚴重的抑鬱症患者有悲觀偏見,但是中輕度患者才是無偏見的理性客觀平衡(參見例如https://www.sciencedirect.com/science/article/pii/S0960982211011912)。科學家已經追查正常人樂觀偏見的來源為amygdala杏仁體,而抑鬱症的主要生理機制就包括了抑制甚至縮小杏仁體(參見https://www.nature.com/articles/nature06280;目前對抑鬱症的機制還沒有徹底的瞭解,一般看法是長期焦慮使腦化學尤其Serotonin失去平衡,進一步促使部分腦結構出現不可逆的轉化,包括杏仁體)。
抑鬱症對生活品質、人際關係和生理健康都有很不好的影響。我並不是說這些患者才是正常的,而是人類因演化自然有樂觀偏見,抑鬱症所帶的悲觀趨勢(亦即不是Bipolar躁鬱症),可以與之對消,給出更精確持平的預估能力。當然極爲嚴重的病患(尤其是有强烈自殺傾向的),已經完全壓倒的樂觀本能,落入悲觀偏見的範疇。現有的SSRI系列藥物效果不錯,讀者如果受抑鬱症所困,千萬不要諱疾忌醫,除了吃藥之外,經常運動(產生Dopamine)和情緒管控都有助益。
\subsection*{2020-09-07 12:20}

我有一個在紐約長大的表弟,是過去20年來交往見面最頻繁的血親,每年兩三次他會來我家過節或度周末。這個周末是勞工節,他帶著未婚妻來拜訪,見面第一句話是“你瘦了好多”;我說去年夏天囘台灣被媽媽罵太胖,所以不得不開始實踐自己早已明瞭的養生之道,也就是寫在這裏的戒糖和每餐有足量的蛋白質。這真的是很簡單有效的科學辦法,請大家認真考慮。
抑鬱症並不讓患者悲觀,剛好相反,是一般正常人有很强的盲目樂觀和盲從獸群心態,抑鬱症病人只不過是對未來和世界的認知沒有偏頗罷了;這個事實,早就被無數的心理學實驗確認過。問題在於,人也是演化過程的偶然結果,人生並沒有什麽天然的神聖意義,一旦剝除了非理性的盲目樂觀或宗教迷信,自然只能在Hedonism、Nihilism和Cynicism中選擇一個。這裏的關鍵,在於不要問“Why?”,而是該問“Why not?”那麽就容易得到正面的人生觀。
\subsection*{2019-12-28 03:30}

美國民間的健身風氣很盛,所以營養學、代謝學的知識遠遠領先全球。有興趣的人,很簡單就可以找到已經被無數次證實的資訊,介紹如何長肌減肥。 
正文裏討論的不能吃糖(因爲要完全避免果糖)只是第零級的重點,再下一步的第一級常識是長肌和減肥必須兼顧(因爲運動和肌肉是促進代謝作用的主動力),卻又不能同時進行。這是因爲肌肉必須有胰島素才會增長,但是胰島素也會同時刺激脂肪細胞的增肥。所以必須以兩個月為周期,做長肌和減肥的交互輪替。長肌階段是偏重蛋白質的半均衡飲食,熱量攝取高於平衡值,減肥階段則是以脂肪爲主(!!!!!這叫做Ketogenic Diet,其原理是誘發體内燃燒脂肪來提供能量的代謝回路,這樣才能强迫脂肪細胞釋放儲存的脂肪),輔以蛋白質,完全杜絕碳水化合物(沒有葡萄糖,就沒有胰島素),並且把熱量攝取降低於平衡值。當然,運動是不能間斷的,卡路里的總攝取量也必須精確。 
對認真的人來説,減肥比長肌要容易多了。搞健美的人,爲了突破人體天然不喜歡長肌肉的趨勢,只好大量使用類固醇;網絡上的肌肉男和演超級英雄的電影明星,都不例外。 
取消一胎政策,晚了十年,後果很嚴重。我不希望十年後,又在糖尿病的問題上,有類似的追悔,所以才急著現在就提出這個話題。 
至於你提的稅收歸中央還是地方的問題,這不是我能置喙的,但是可以預見會有廠商逃稅,所以什麽樣的規則最方便執行監管,就是最好的選擇。
\section*{【歷史】歐系文明的起源}
\subsection*{2023-03-17 19:03}

你這是脫離實證的空想,而且沒有針對Mythology演變為Religion的關鍵轉折點來做思考。Religion和更早的Mythology普遍存在於人類社會,固然有其來自Evolutionary Psychology的强大因素(參見這篇總結性論文:《https://www.nature.com/articles/4551038a》),但迷幻藥劑顯然是在原始宗教發展早期,加强組織性、紀律性和擴大規模的重要助力。這個論點有著極多的實際例證和非常合理的邏輯基礎,基本在部族社會有普世適用性(當代原始部族仍有許多例證,我沒有在節目中提起);當然一旦文明發展到能支持哲學的程度,迷幻藥劑的重要性會有所下降,但瞭解宗教的最早起源,依舊是有很高引導性價值的。
\subsection*{2020-09-06 10:00}

權貴階級(資本只是工業革命後權貴階級的一大類)的推波助瀾,是最近民族國家成型的一個重要因素,但不是唯一的因素。更基本的原因來自Game Theory,亦即優勢族群形成更緊密的集團來謀求徹底獨占,經常是利益最優解;一旦一個這樣的集團出現,其他人的最優解(Nash Equilibrium)自然變成也組織類似的集團與之對抗。中國古代上至廟堂黨爭、下至同鄉情誼,都是這個機制的體現。侵略征服從新石器時代就是一門好生意,它所產生的集團組織就是後世所謂民族和國家的胚胎。既然歐洲在航海殖民和工業革命之後,以民族國家爲單位尋求對全球利益的剝削和獨占,受害者沒有選擇,只能也強化自己的民族向心力與之對抗來保護自我;階級鬥爭在這個背景下,不可能完全取代國家觀念。這個道理我以前已經解釋過了,請不要每隔兩個月重提一次。
\subsection*{2020-09-05 22:22}

真科學的進程是,每當發現一個新的事實證據,就可以試圖以其為基礎建立邏輯假設。但是如果這個假設與其他已知事實相矛盾,我們必須先嚴重存疑,這是因爲真相是複雜的,事實證據卻是片面的,研究人員也可能做出誤解。這時學術界應該進一步搜索更多的樣本,並且仔細檢驗舊有的實驗、觀察和理論,靜待足夠的證據澄清邏輯上不自洽的論點,而不是像何新那樣利用暫時的不確定性來推銷無中生有、毫無根據的陰謀論,或者像超弦論者那樣堅守已經被否定的錯誤假設,只是把參數空間無限放大來包容負面證據。
這裏,有關Botai馬匹的研究與其他已知事實沒有任何衝突,所以很快被接受為科學共識;但是對於Yamnaya移居軌跡的説法和許多既有的證據有矛盾,在過去兩年也沒有出現新的正面證據,所以並沒有改變主流意見。
\subsection*{2020-09-04 17:28}

基本正確,但是雅利安人沒有文字(日耳曼人一直到公元8世紀才有書寫能力),也沒有自己的青銅技術(始於Uruk Period;歐洲和西亞的銅礦比東亞要多得多而且容易開采,品位也高,問題在於錫的來源。上古時期除了Bulgaria有一個小礦,錫主要靠Afghanistan來供應,1000多公里的山路由驢子運到兩河流域,你可以想象多麽珍貴)。他們征服歐洲、希臘和土耳其(建立Hittite帝國),用的是新石器箭頭;這純粹是蠻族打敗先進農耕文明,就像五胡亂華和元、清兩朝一樣。後來東亞北方黃種游牧民族反過來向西征服擴張,所以現在新疆、中亞和土耳其的語言並不屬於Indo-European語系。
從敘利亞擴散到歐洲的農夫,可能説的是閃族語言,但他們沒有文字,所以不可考。這裏順便提醒大家,語言和文字是兩回事,同一個語言可以用不同的文字拼音出來(當然這麽一來,就很容易分裂成不同的語言,例如正文中所列的那些古語),同一套Alphabet也可以用拼好幾個不同的語言(例如Cuneiform被用來寫下好幾種古語言)。Indo-European是一種語言,它被散佈到極廣的範圍後,各地選用不同的字母在不同的時代寫下來。梵文是Indo-European的一個方言,但是書寫方法卻是後世的印度人參考臨近文明自己發展出來的。
在被雅利安人征服之前,歐洲的農耕民族已經有自己的問題,他們的村落開始築墻,考古學家也發現過被屠村的遺跡;這不一定是游獵民族幹的,也可能是他們自己人選擇當土匪。不過在北歐他們的確被游獵部族完全驅逐出來。
雅利安人征服希臘的過程,是目前最有爭議的部分。古希臘文顯然是Indo-European的一支,公元前12世紀地中海東岸青銅器文明大崩潰(Bronze Age Collapse)的過程中,也必然有來自希臘方向“Sea People”(埃及人對這波蠻族的稱法)的貢獻,此外希臘口傳歷史本身就描述一個“Dorian Invasion”(讀者可以參考“Return of the Heracleidae"),這些都是正面的證據。但是因爲確切的年代衆説紛紜(我猜測可能是因爲有兩個波次:2000 BCE和1200 BCE,前者帶來雅利安語,後者才是Dorian Invasion,而Sea People則是第二波外來者和本地難民的合集),考古挖出來的城市遺跡只證明有戰爭,不能排除内亂的可能,現在學術界的主流共識是嚴重存疑。這是西方歷史學者並沒有對中國搞雙標的又一個例證。
一般科普作者找到一篇外國人的文章,自己並不懂,翻譯過來就了事。我則一向是要從多方面徹底瞭解一個議題,才會寫下自己的心得。這篇博文我原本沒有計劃要寫,是因爲臨時出了“西方歷史造假論”的議題,我才覺得應該解釋清楚正反兩方的層次差距有多遠。這裏簡單總結了大概4、50篇論文的發現,所以沒有詳細列明出處;但都是有憑有據,而且經過我自己分析過濾,認爲自洽合理,與其他證據吻合,才寫下來。如果你有什麽疑問,可以進一步討論。
\section*{【國際】【政治】21世紀之民粹}
\subsection*{2023-03-12 16:22}

操弄民調是英國人發明的老技術,在《Yes Minister》節目中就被反復譏嘲過。其實我覺得那還算是對“民主”這塊招牌做表面上的尊重(Lip Service);幾年前有一個重磅的研究,分析美國國會法案通關機率是否取決於公衆支持,發現影響是零!請注意,這個分析所用的“公衆支持率”,正是已經被扭曲過的民調結果,所以實際上應該是越不受選民支持的法案、通過的機率越高。
那篇論文探討得很詳細,值得對美式民主的真相有興趣的讀者去學習理解。請搜索Martin Gilens and Benjamin I. Page,《Testing Theories of American Politics: Elites, Interest Groups, and Average Citizens》(2014)。
至於McKelvey–Schofield chaos theorem,一般是應用在代議制和選票的設計上,可以論證間接的層級越多、委任的範圍越廣,對民意的扭曲就越容易。不過我認爲民選制的問題,遠超這一個數學定理,即使事事公投,並容許無限多選項,依舊有很大的操弄空間;再加上民意本身往往就不是最優解(例如美國人對於享受全球掠奪的紅利,先天有廣汎的支持,他們反對的是分配不均),所以沒有必要過度强調這個角度,否則很容易重蹈Fukuyama在2017年狂讚英國民主體制的覆轍,事實上我曾指出英國的代議制在開始公投之後,決策水平反而摔下斷崖,不是嗎?
\subsection*{2020-07-01 05:57}

我什麽時候説過高經濟成長率導致高利潤率?美國在二戰後,工業一枝獨秀,以致成長率和利潤率雙高;我在正文只提了前者,並不代表你可以自己樹靶自己打。
你老是要堅持資本論裏面利潤率隨時間降低的論調,其實那是馬克思在一些隱形前提下的結論。他的假設主要是把歐美工業國家當做一個封閉系統,其他國家只能做爲原料供應地和成品傾銷的市場,而不考慮科技和產能向全世界擴散的過程和效應,也不考慮形成跨國托拉斯的可能。實際上實體產業的利潤率當然是逐步降低,尤其是外包給後進國家之後,那些真正從事生產的工廠利潤特別微薄,但是做零售、服務和尤其金融,完全可以維持舊有的高利潤,例如美國的商用房地產在過去30年的年報酬率在7 \% 以上,一直到今年在Amazon和新冠的雙重打擊下才有停止的勢頭。Amazon自己的利潤率很低,但這是爲了在行業興起早期奪取市場額分,目的是一旦沒有競爭對手,就可以盡情收割顧客和供應商,届時利潤率反而會隨時間增高;這一個道理廣爲人知,所以它的股價才能不斷高漲。
科技創新靠的不是自由民主,而是資源和組織,這一點我寫過博文《科技發展和美式自由無關》專門討論過了。
中國現在還是有邪教的,其中市場教和民主教已經被反復批判,還在被鼓勵的是中醫教;這一點我也已經仔細討論過了。
\subsection*{2020-06-30 15:51}

民粹是西式民主的終極形態,特徵就是政策不再取決於利害得失,而是只追求一時爽,然後大家自我感覺良好。所以印度對中國做出某種形式的“制裁”,是必然的事。民進黨政府的治理哲學,和印度一摸一樣,大家對這類思想模式應該都很熟悉了。
這次新冠疫情,終於打破了全世界對美國的崇拜敬畏心態,歐盟已經決定轉向,與美國保持距離;我在兩篇《後新冠世界》裏詳細論證過了。雖然年底Biden很可能會勝選,但中國還有至少半年時間來和歐盟簽署各類協定;這次香港國安法,只有英國和瑞典出來跳了幾跳,德法都低調處理,是很明顯的正面跡象。
我討論的議題越深刻,能馬上完全讀懂的讀者就越少;這一篇博文一年前刊出的時候,回響並不大,但是現在應該比較容易理解了,所以我把它拉出來置頂,提醒大家復習。
\subsection*{2020-06-30 15:47}

我對台灣讀者是額外容忍的,但是基本規則仍然須要尊重。這裏最重要的,是你自己必須先把邏輯理清楚,不能把錯誤成見當口號來喊,隨便加上兩句明顯不通的道理。
我在《讀者須知》的正文還特別强調,大陸網絡上的謠言不能作爲事實證據,結果你第一句話就是引用匿名、未標識的“大陸文章”。掩護修路必須越界,這個論斷背後的邏輯完全是空白,如果你看不出來,請先回去重讀高中數學,多做些證明題,再來我的博客。我並不要求讀者是邏輯大師,但是基本的能力和正確的態度是必要的。
總統制下,總統的權力有多大,請參見Trump和蔡英文的執政經歷。有阻力不是做不了的藉口,就像我走路也會遇到空氣阻力,但是不能因此而說一步都走不動。這也是很簡單的邏輯道理。
中共不急著統一,是因爲打不過美國;等到國力壓倒美國,政略方程式自然有不同的解。這個道理我至少解釋過幾十次了,簡體字和繁體字都用過,你到底是爲什麽看不懂?
\subsection*{2019-07-14 11:23}

美國在二戰後的經濟周期,原本是比較有規律的5-7年,但是冷戰結束後有一波全球化過程,美國内部的通脹壓力得以藉外力紓解,周期也就增長到10年左右。 
西方民選制度下的政客,一般是用減稅、減息、印鈔和貨幣貶值這些手段來人爲吹大泡沫,希望把破滅的時間延遲到他們退休之後。當然如此一來,其結果是經濟危機更爲嚴重。 
我沒有看《流浪地球》,但是兒子看了,他説比漫威的電影還幼稚,勸我不要浪費時間。我讀《三體》斷斷續續三年了,現在總算看到第二本,覺得還不是太糟糕,就是對物理和數學的描述,只有大學工科一年級生的程度;邏輯的破洞太多,讓人無法融入;人物的心理和個性也很單極,不是故事裏頂尖戰略人才應有的複雜多面。總之,劉慈欣的長處在於想象力豐富,對一般讀者來説,很有娛樂性。但我不是一般讀者;一本書裏每隔三四頁就看出一個學術、技術或人性上的嚴重錯誤,自然無法融入故事之中。
\subsection*{2019-07-02 10:16}

我覺得這種“微貸款”方案,構想很美,執行卻很難。非洲和南亞的幾個嘗試,在得了一堆國際獎項之後,無一例外在三四年内就破產,包括上個月剛停業的一個孟加拉機構。
其實原因不難理解:要分辨一個貸款戶是否能夠還錢,需要懂經濟金融的人才,花不少工夫才能研究完成;這個交易成本是固定的,不隨貸款金額大小而變動。所以銀行提供貸款,先天就是金額越大、效率越高。一旦爲了社會政治理念而强迫反其道而行,那麽自然無法負擔這個交易成本,結果必定是有過高的違約率,從而入不敷出。
中國在這方面,一直是依賴國營銀行的獨占性,由央行做政策性要求,强制增加對中小企業的貸款。那麽因爲貸款數額仍然不是太小,而且國營銀行的人工成本比較低,所以負面效應還可以忍受。如果真的照抄外國的微貸款,那麽後果反而會是金融危機。
\subsection*{2019-06-04 20:02}

Piketty的結論,是大資本家每年的稅前資本利得,在和平時期必然高於GDP成長率。以往歐美獨霸世界的工業產值,那麽因爲GDP成長率夠高,大資本家還勉强同意通過纍進稅率、工人高薪和福利政策,來使稅後資本利得接近GDP成長率,結果是貧富極端化還不明顯。 
在1960-70年代,因爲德、日的復蘇,瓜分了工業產值,然後又有石油危機,英美的GDP成長率忽然掉下去了,這時大資本家就不可能坐視每年只拿2 \% 或3 \% 的回報,他們決定盡全力對社會主義福利政策做反撲是必然的。所以英美資本家對中產階級的殺鷄取卵,其實早在中國現代化之前就開始了,只不過後果因爲1980年代的負債消費和1990年代的冷戰勝利紅利而暫時沒有浮現而已。 
當然,中國的體量比德日加起來還大得多,對工業產值的瓜分作用也遠遠更强,即使考慮到世界經濟不是零和游戲,總體和長期來説,先進工業國家在21世紀還是必須面對一個更大的負面衝擊。
\section*{【美國】【經濟】從回購利率暴漲談美國經濟周期}
\subsection*{2023-03-11 11:43}

SVB是地區小銀行,主要服務矽谷當地的初創企業;這次出事原因在於主管對美聯儲QT和升息毫無預期和準備,資金大部放在國債和MBS(Mortgage-Backed Securities;這兩者正是美聯儲QE/QT的主要管道,表面上很保守,實際上對貨幣政策非常敏感),升息後價值下跌,資不抵債的危險最近被公開,因爲初創企業的存款遠超FDIC保險限額,立刻就引發擠兌。總之是很特殊的個例,並不會導致系統崩潰。

剛剛有朋友私下通信來做進一步討論,把我的意見在此也和大家分享:
1)對若干初創企業存款戶會造成困難,但嚴重程度要視財政部如何善後,而且對美國經濟體系整體來説,初創企業原本就是可以隨時倒閉的高風險行業。換句話説,SVB破產的真正影響,在於風投業,而不是金融體系。
2)至於富國銀行(Wells Fargo)也出現擠兌現象,以及美國銀行(BOA)股票大跌,過去幾天紐約證券商的確正在風聲鶴唳,一些基金急著脫身,另外有一些想做空趁機牟利。不過美式金融體系原本就是高度不穩定的,謠言橫飛的實際影響,主要看財政部和美聯儲出手的早晚和强弱。目前的大銀行(含富國銀行)財務狀況,純粹只是Liquidity、而不是Solvency 的問題 ,Yellen和Powell無需總統和國會授權,就可以簡單解決,他們必須是異常的蠢才會放手不顧、任其惡化,所以我預期富國的最糟糕脚本是美聯儲注入資金,不會到破產的地步。
3)對金融的真正負面效應,在於逼迫美聯儲提早結束QT;不過考慮到民間美元現金的存量仍在高位,地區性銀行倒閉的作用有限,而且引發的通脹惡化問題是慢性的,必須等待其他更嚴重的黑天鵝和灰犀牛事件(例如金磚貨幣)做出打擊,才會有顛覆性的連鎖反應危險。
\subsection*{2022-05-25 01:48}

首先大家都不看好資本市場,例如股市下跌,很多基金趕緊縮表。原本現金的短期儲存主要靠買短期國債,但目前Treasury bill的yield是0.51 \% ,而Excess Reserves剛剛從0.4 \% 上調到0.9 \% ,Reverse Repo則是0.8 \% ,顯然是較優的選擇。
至於爲什麽Reverse Repo的利率稍低,反而更受歡迎,這是因爲Excess Reserves是大銀行才有的特權,而現在這些現金主要來自影子銀行界,Reverse Repo是他們的最優解。
Excess Reserves的利率在五月4日之前是0.4 \% ,遠低於其他管道,所以總量下降是正常的。Excess Reserves的利率調整,必須由FOMC開會通過;Reverse Repo則是NY Fed一家説了算,所以後者要靈活很多。
\subsection*{2022-01-24 09:09}

我一再説過,我沒有魔法,只能聼其言而觀其行。三四年前,美方對Nordstream II還沒有施加强壓,公開信息只顯示Merkel對美國百依百順,如果我徑行假設她另有圖謀,那只算是毫無事實根據、一廂情願的幻想。現在她艱苦卓絕地把管道建成了,這不是我的主觀意願有改變,我還是一樣的理性分析,只不過新的客觀事實指向新的邏輯結論。
烏克蘭要打起來,原本只有兩個脚本:1)美國教唆支持Zelensky;2)Zelensky自行貿動。Scholz的表態只消除了前者的可能,後者必須Biden主動明確嚴禁才可能阻止。當前的局勢,已經足以讓理性的美國決策者下禁令,問題在於Biden和Blinken都不是理性決策者,所以Zelensky開戰可能性雖然不大,依然不是零。
\subsection*{2022-01-23 13:05}

因爲當前烏克蘭局勢剛好是美英霸權的典型操作,所有昂撒媒體必須統一口徑、全力推動假新聞,結果不但人在國外的讀者往往面對洪水般的謊言,習慣照翻美國國際新聞稿的地區如台灣,更加是胡扯蛋的重災區,所以我上周和史東做《八方論壇》,特別選的題目就是討論此次美俄對峙的幕後真相。最近幾天又有一些新消息(主要是Scholz的決策),我正準備再做另一期訪談來追加評論;這裏只簡單提出事實的綱領。
首先,Putin絕對沒有主動打烏克蘭的意圖,任何拿著《NYT》或《Washingtong Post》文章指指點點的人,必須先回答兩個基本疑問:1)打烏克蘭對俄國有什麽好處?2)即使你假設Putin是個衝動的蠢蛋來回答前一個問題,那麽爲什麽去年春天不打、夏天不打、秋天不打、現在隆冬期間反而升級衝突?並且還遲遲不真正出手,平白賦予美方做外交和軍事準備的時間?英美宣傳機構欺負自己國民智商低、容易忽悠,所以編出來的謊話漏洞百出也不成問題;旁觀者如果也接受那些明顯的胡扯,就太辜負父母賜給的腦子和師長辛勞的教誨了。
當然,這個事件即使忽略全球地緣戰略態勢的大局背景(亦即中國崛起、霸權交替),只看局部的互動,也是4方(俄、美、德、烏)博弈的問題,有相當的先天複雜性,所以光知道美英宣傳不靠譜並不保證正確認知會自動浮現(參考當前充斥大陸網絡的胡猜,更別提所謂智庫的分析)。還好博客這裏已經反復解釋過其中三者(俄、美、烏)的戰略考慮和決策習慣,先簡單為大家復習一下。
Putin對自己的戰略意圖和戰術運作一直很公開、直白,沒有什麽猜測的必要或懷疑的餘地:他的目標是短期内遏止北約東擴、長期則試圖收復若干被侵占的勢力範圍,而所選用的手段則是所謂的Strategy of Tension(這真的是俄方自己的用語),亦即既然美英靠製造事端來打擊對手,俄方在終於補好所有罩門之後,可以反過來維持或甚至提升衝突緊張的態勢,讓美方承受不住自己引發的麻煩。
美國的霸權伎倆我更是已經討論過幾百次,總結起來就是忽悠“盟友”當炮灰。這裏又分第一綫的軍事外交炮灰,和第二綫的經濟貿易炮灰:先讓前者挑起事端,然後見死不救,再鼓動後者去做傷人傷己的制裁,美國作爲“仲裁者”,可以從中多方揩油。體現在對俄方向,烏克蘭是前者,德國則是後者。這套伎倆固然無本萬利,但必須有一、二綫炮灰都配合才運作得起來;這一點正是理解這個事件脈絡的關鍵。
從前面的討論,可以看出美俄博弈的成敗,取決於德國的選擇。2014年Putin在烏克蘭失手,落入美國的陷阱,被迫出兵。當時Merkel受到外交和輿論的多重壓力,又兼被BND(德國聯邦情報局,那份報告後來被泄露出來)忽悠,說只要配合美國做全面經濟制裁,CIA會夥同俄國的Oligarchs發動政變推翻Putin,於是她咬著牙忍痛接受德國企業的巨額損失,結果卻是Putin的民意支持率衝破90 \% ,權力更加穩固。到了2015年,她已經明白自己上當,於是特別訪問Moscow,和Putin進行了一場秘密會議(這裏所謂的“秘密”,指的是她下令摒棄所有德方的幕僚、助手和翻譯,獨自和俄方會談幾個小時,所以全世界都知道他們談了,卻誰也不確定談的是什麽),然後德俄之間隨即有了兩個公開的外交發展,一個是NordStream II上馬,另一個是Minsk協議,要求烏方容許東烏高度自治以換取和平。
其後的七年裏,Merkel一直是棄車保帥,不求取消對俄制裁,不圖貫徹Minsk協議,只求建成NordStream II。這裏我認爲是她人單勢孤,在美國全面滲透掌控德國政治、情報、宣傳體系的背景下,連和幕僚討論的餘裕都沒有,只能獨自默默地為解除這些桎梏做最間接隱性的努力(德國政治人物誠實討論戰略議題的空間,可以從昨天海軍總監只説了兩句客觀評估就被迫辭職看出來)。她的第一優先考慮,自然是預期美國會重施故技,利用烏克蘭挑起衝突,再次强迫德國去當經濟炮灰,而NordStream II是讓德方能置身事外的關鍵前提,只要有它作爲備用,德國的天然氣供應就不受東歐局勢制約,可以獨立選擇理性的外交政策。
所以烏克蘭之所以又在2021年發生衝突,並且一路拖延惡化到2022年,是烏、美玩弄敲詐的老把戲,卻沒有想到俄、德都已做好準備,願意奉陪的結果。上周我上《八方論壇》討論這件事的時候,還不能確定Scholz是否有足夠的智慧延續Merkel的策略,過去幾天的一系列新聞,徹底解除了我的疑慮。例如昨天Blinken和俄方會面,居然是空手到,拿不出承諾的Counter-Proposal,只能要求延展日期;這裏的幕後機制,是美方原本對Scholz做了三點要求:1)譴責俄國侵略;2)軍援烏克蘭;3)公開承諾將發動新制裁;結果被德國全部否決。雖然博客讀者應該看得出,Scholz若是同意了,反而等同為Zelensky開一張空白支票,導致戰爭必然發生,但在當代歐美的民選體制下,出現有基本常識的領導人,依舊算是一個驚喜,畢竟不但日本和澳洲做不到,連Biden政權原本都自信滿滿,沒有意料到德方會有自保的舉動。

雖然整個事件在2021年3月的起端,以及冬天的情勢惡化,都是Zelensky爲了國内政治需要而主動挑起,以爭取烏克蘭民意支持,順便對歐盟進行訛詐,上面的回答卻省略了這一點,這是因爲上個月我才詳細討論過,所以讀者應該自行復習,參見《歐盟内部的無色革命》一文下的留言討論。
Biden並沒有參與挑起和升級衝突的決策鏈,因爲沒有必要,整個體系原本就到處鼓勵類似烏克蘭的挑釁舉動,從國務院到CIA,專業負責搞顛覆的中低級官員數以千計,其中總有一些不是純粹吃閑飯的;Zelensky在自己國内調動軍隊,在美國沒有明令禁止的前提下,也不必事先報備。
MI6也是如此,對Kazakhstan的Alyazov做工作是已被確認的事,但這指的是平常的聯係、協作、信息交換和互相利用,這次政變連Alyazov事先通知了MI6的跡象都還沒有,在事實證據為零的前提下,自動假設後者參與謀劃,不是理性知識分子所當爲。
\subsection*{2021-06-20 05:26}

長期通貨膨脹的主要動力,正在於人群的預期心理,所以它不但是非綫性的,而且是Path Dependent(路徑依賴,亦即不是一個State Function態函數),還可能是Chaotic。很不幸的,這也代表著它難以確實預測,只能約略估計危險程度高低。
中方沒有理由再為美方壓低進口物價。反過來看,進口物價也不是美聯儲的頭號麻煩,當前真正最讓他們憂心的導火綫,其實是美國服務業的大重整:新冠不但導致暫時的歇業,也永遠地改變了消費習慣和上班常態,網上購物和遠程工作並不會因爲疫苗普及和店面重開而消失。這大幅延長了勞動力重置(Worker reallocation)的過程,以致出現了就業率低迷、而同時企業卻雇不到人的矛盾現象。這裏的危險在於後者會導致薪資上漲,而前者卻迫使美聯儲繼續放鬆銀根。
我個人認爲,美聯儲騎虎難下,早已把自己逼到墻角(Paint self into the corner),現在連國債市場原有的通脹估價職能都被洪水般的美元流量(Liquidity)所淹沒,卡在同一個極低利率的讀數上。這篇正文討論的2019年回購利率暴漲問題,現在剛好反過來,成爲銀行爭先恐後地搶購“反回購”(“Reverse Repo”,亦即用現金向美聯儲交換債券,最新的數字高達7000億美元),反映的正是金融系統裏現金充斥,而國債回報率又過低的窘態(至於爲什麽Reverse Repo的回報率居然會高於直接買國債,讀者別忘了,前者是美國國内金融機構的特權,後者卻是美國對外吸血的管道)。
美聯儲的難處是一方面已經印鈔過多,實體和虛擬經濟都無力繼續吸收,另一方面低就業率和Biden的巨幅赤字又預先排除了積極緊收銀根的選項,只好硬著頭皮持續放水,靠喊話來安撫群衆,同時指望爆雷後果由外部經濟體來承受。拿破侖曾説過(這裏采用常見的英文版,我確認過這合理地反映了他所説法文的原意):“Never interrupt your enemy when he is making a mistake.” 可惜2008年,中國不但打斷了美國的錯誤,而且進一步獻身擋子彈;這次至少應該設法置身事外,以自保為優先。當然更好的做法,是聯絡其他利益相關國家,將美元的破壞力盡可能局限到自己境外,這需要多國中央銀行的協作,可以從俄國做起。
\subsection*{2021-03-07 06:19}

兩年前全世界還不知道會有新冠的時候,我已經明確指出美國經濟泡沫會在2020年爆破,其後美聯儲必須放棄只完成1/3的銀根回收過程,直接返回大水漫灌政策,而且這會是美元周期循環收割全球的絕唱。後來疫情大幅加劇了經濟衰退的程度,美聯儲的量化寬鬆至今比上一輪增快了4倍,而上次2014、2015年的回收階段力量已經很微弱了,所以這一輪吸力近乎消失是很自然的事。
這個現象體現出來,除了美聯儲從主動落為被動之外,另一個特徵是利率和匯率的脫鈎。如果你仔細去看這兩年至今4萬億的新美債是誰在認購,就會發現傳統的外國買家(亦即以中日爲首的中央銀行)基本沒有增購,人民銀行反而在慢慢減持,專業金融機構也頗爲小心,結果除了美聯儲之外,真正購買的大頭就只是零售和半零售(指沒有什麽技術和知識含量的金融載體,例如Mutual Fund和Pension Fund)投資人。如果美聯儲被迫在財政部還在狂抛債券的階段就提升利率,那麽不但匯率的反應會很有限,而且中產階級投資人要吃大虧,一旦債券利率飆升失控,聯邦政府的利息支出將以倍數成長,反而要敲響美元霸權的喪鐘。中國應該預做安排,和歐盟、德、法、歐元銀行等相關機構事先準備好協同預案,以便迅速出手保護中歐的共同利益。
至於美國國内加稅,你放心,有共和黨在,這麽重要的正事絕對幹不成。尤其是財政崩潰眼看著要發生在Biden任内,剛好是Trump把責任賴在民主黨頭上的天賜良機;早先我說過,Trump面臨幾十件州級和民事官司,只怕不再能夠自身出任公職,但如果有了世紀級別的財政危機,自然另當別論。
\subsection*{2020-03-30 18:12}

這次美國的經濟衰退,一個合適的歷史前例,是1990年前後日本經濟泡沫的爆破,兩者同樣是浮腫的繁華假象被徹底戳穿。
當時日本股市崩盤,一般人也以爲已經過度强勢的日元(在1985年的Plaza Accord,美國强迫日元升值,從250:1一年就升到120:1,正是日本經濟吹起泡沫的主因)會開始貶值,但是真正内行的人應該看出事實會剛好相反:正因爲日本的公司現金流開始出問題,他們必須賤賣海外資產,把錢匯回日本,所以後來日元不降反升,從1990年到1994年,匯率從130:1升到80:1,其後才慢慢貶值,上下振蕩到近年的120:1。
現在也是一樣的:一切其他資產都比美元現金的風險更大,再加上美元的地位比當年的日元還更強得多,不只是美國人,連其他國家的資本也會想要換成美元來“避險”,這樣一來,美元反而有很大的升值壓力。當然最終美國經濟空洞化、貨幣過度發行,這些利空的長期基礎因素會顯現出來,但是那必須等到一個替代美元的新國際儲備貨幣有能力接收百萬億級別的資本流動才會發生。
\subsection*{2020-03-13 10:24}

這次的經濟危機,早已遠超Trump和美聯儲能堵上的程度;事實上大前年的減稅、去年的放水和三年來預算的濫用,都徒然吹大泡沫,讓他們自作自受。
我已經一再説過,和2008年相比,這次的差別在於大銀行沒有參與狂歡,所以不會有金融界的連鎖反應。爆炸的核心是企業債,中小銀行會因此而倒下一片,但是知名的國際銀行不會需要2008年級別的聯邦救援。但是我原本拿2000年的衰退來相比,這在新冠這個黑天鵝出現,又被Trump政權胡搞因人禍而擴大之後,必須稍作修正:震央仍然是股市和非金融企業,所以性質類別相似,但是程度會更嚴重許多。這是因爲美國的經濟和國力比20年前衰弱不少,而新冠不但會打擊經濟需求(Demand),而且會大幅遏制供給(Supply),甚至會影響社會穩定。
\subsection*{2020-03-10 12:27}

美國經濟在過去這年,靠的是三個支柱:私人消費、股市泡沫和美聯儲放水。疫情會直接打垮消費,股市泡沫已經開始爆破,美聯儲花了12年都沒有辦法收回上一輪的量化寬鬆,這一次繼續印鈔票必然會面臨收益遞減的現象。所以一個至少等同2000年的經濟衰退是跑不掉了。
我已經解釋過,2008年先倒閉的是金融機構,這次大銀行早早就置身事外,所以問題會集中在非金融企業,尤其是資本密集的產業,例如頁岩油氣。Putin和沙特決定不減產,最終的考慮就是要落井下石,一舉消滅美國的中小型頁岩油氣公司。
如同2008年後,美國一直到2014年才喘過氣來,這次美國至少也會經歷三四年無力外顧的階段;雖然輿論上已經完成仇中的全面動員,有心無力也是無法可施。例如法國在二戰前,不是看不出德國國力復蘇,但是國家客觀力量不足,政治主觀内鬥不止,也就不可能主動出擊,只能選擇戰略收縮和防禦。中方恢復元氣,也就是一年半載,届時可以安享幾年的冷和平,持續改變雙方的力量對比,直到美方知難而退爲止。
\subsection*{2020-02-23 02:00}

這些説法和我過去這一年所做判斷的大方向是一致的,或許那位教授直接或間接看到這些意見。我寫作的目的,一向是要影響意見領袖,才可能扭轉輿論,所以如果真是如此,是件好事。
不過我要修正這裏的若干細節。現在美國貸款利率其實又接近了歷史性的低點;不但長期利率因爲大銀行看衰經濟所以低迷不振,短期利率也只有百分之一點多,還有美聯儲拿出几千億美元來針對性地放水。企業就算不走借貸這條路,股市的狂歡也打開了其他融資方法的大門(例如Preferred Shares,Convertible Bonds等等)。
我不知道今年七月企業債到期有多麽集中,不過如果真的造成融資瓶頸,美聯儲必然會再一波放水,輕鬆解決。金融的特性是表面上容易看到的一級效應,自然會吸引資本套利或央行補償,真正會引發麻煩的,是慢性問題由二級或三級效應引爆。這是需要實戰經驗才能領會到,所以也往往是學術界的盲點。
\subsection*{2019-10-03 21:20}

1. 對Trump這人實在很難做出預測;照理説,他早應該和中方妥協了,但實際上,他是現在美國右翼民粹總操弄者,又正在全力發動他們抵制罷免,所以整體來看,還是只能指望回歸到今年六月G20後的短暫休兵狀態。 
2. 這其實正是正文裏,那句“與其去研究2008年的金融危機...”的用意,亦即消費者固然不像2008年那樣負債纍纍,企業卻是被淹沒在垃圾債券之下。 
3. 同樣的,這次回購利率暴漲,危險並不是2008年金融危機式的連鎖反應,而是銀行手頭現金不足,必須削減貸款量,這不但會使經濟衰退無法避免,事實上本身就可以是衰退的導火綫。 
一年前,我忽然去討論印度的一個影子銀行破產的事,就是覺得它對印度經濟會有同樣的作用。現代社會,每天的新聞以百萬計,但是我每隔幾天才會寫一篇文章,其實已經是從無數沙礫中去挖出鑽石了,只不過偶爾沒有把方方面面的細節全部直白討論。你如果有時間再多讀幾次,我想還是可能會有繼續的收穫。
\section*{【工業】【能源】再談氫經濟}
\subsection*{2023-03-07 18:41}

自從去年我寫了《從SWIFT制裁俄國,看中國的對應之道》和《社會主義國家應該如何管理資本》之後,就明確説過,外交戰略和金融經濟議題的研討和建議已經基本大功告成,博客未來的建言任務將專注於科研和教育管理。本來我一直計劃針對這個題目也寫一篇文章,總結多年來纍積的辯證結論,但始終無法起筆,原因是基礎科研的專業壁壘太高,目標聽衆又沒有像在外交和經濟方面那樣的知識經驗,實在難以兼顧簡單易懂又涵蓋所有重點。結果有關國務院“兩大毒瘤”(科技部和教育部)的討論,依舊零零星星地散佈在幾百條留言對話之中,這是爲什麽我一直說,希望有潛水的中央幕僚或官員願意來博客詳細閲讀理解的原因所在。
今天看到科技部的改組計劃,是意外的欣喜:那些細節,的確與博客既有的意見驚人地一致;金融改革和數據局的建立同樣是博客多年來反復的建議,再加上兩周前外交部發表的《美國霸權霸道霸凌的危害》也是如此,我覺得博客應該已經成功獲得若干有實踐能力的特殊潛水讀者。你曾經積極參與關於科技管理改革的討論,有資格分享成就感;將博客内容搬運到國内的志願者,也有相當的貢獻,我在此代表全世界人民對你們致謝。
\subsection*{2021-09-02 09:20}

我所羅列的那三個新政策,其共通點是都有可以拖延的藉口,但那並不代表它們的性質或重要性一致,或者拖延的藉口是同一個。
整治壟斷性平臺當然原本就是該做的關鍵政策,對產業升級也有間接的好處,但依舊不是必要條件,而且對國際資本體系公開宣戰,立刻就在金融貨幣政策上造成額外的壓力;既然爲了完成對美國的承諾而剛剛放寬金融管制,那麽整治壟斷性平臺的時間點選擇當然可以說成應該或早或晚、但不是現在。
雖然我用字一直很精確,但總是有邏輯思路模糊的讀者,把不同的觀念混肴起來。我並沒有說那些新政策不重要,剛好相反,是雖然中共政府早該出手,但它們依舊被拖延多年,那麽當然完全有藉口繼續混下去;然而一旦習近平注意到這些事,就立刻下了決斷,即便不是最佳時機,也一樣劍及履及,所以由此也可以看出未來他對臺海問題的處理態度。

你誤解我的文字,已經觸犯了《讀者須知》第八條。念是初犯,先禁言一個月;請你利用這個機會把文章仔細重讀,習慣我的寫作方式。博客雖然有上千萬字,但沒有一句廢話;讀者若是有疑問,請先檢討一下自己的理解是否精確。如果你真的自認能力高到可以隨口對我的邏輯推演挑錯,那麽你絕對夠格寫自己的博客,根本就不該來這裏發言。
\subsection*{2021-08-22 04:01}

有關AI用在互聯網對整體經濟無益的討論,的確沒有出現在正文,但在留言欄我一直支持做這種論斷的讀者。
美國的經濟“奇跡”,其實300 \% 是印鈔票買的。這裏有一個巨觀經濟學的簡單概念,叫做“Velocity of money”,亦即多注入一美元現金,能產生的額外GDP其實是好幾倍(以往至少是3倍,但最近兩年實在印的太多,超出實體和虛擬經濟的容納量,所以我估計大約已經略低於2倍);例如金融巨鰐從美聯儲拿到錢去買游艇,游艇商賺了錢又可以去買Ferrari,汽車經銷商賺了錢又去買別墅,等等。美聯儲在過去兩年每年印3萬億,相當於GDP的15 \% ,但考慮Velocity of money大約是2,灌水效應其實應該是讓GDP浮腫30 \% ,而今年的實際成長率遠遠不到10 \% ,所以我說這些GDP成長,300 \% 是印鈔票買的。
\subsection*{2021-08-19 05:21}

讀這個博客,不能拿你在其他媒體上的習慣,只看浮面或局部的句子或段落就試圖反駁。因爲我討論都是很複雜深刻的議題,雖然從一開始就有一貫正確的主體論斷,但每次重提一個話題,會針對當天的關鍵細節而對正反兩面繼續做詳細的剖析和辯證,所以我用詞必須很精確,相對的,讀者遵循博文的邏輯到哪裏,也就只能確定到哪裏,任何自己領會的言外之意都有可能不是我的原意,至於只得到大致正負印象就以爲我在一刀切,那更必然是錯誤的。
我對AI的整體評價一向是很高的,在《從假大空談新時代的學術管理》一文中還特別强調是政府和社會應該積極投資的科技方向。最近幾天的討論,旨在去蕪存菁,讓這些投資更爲高效,這自然也包括對現階段AI技術的局限性有正確深入的瞭解。
舉農機爲例,只不過是爲了凸出可行性高的應用方向:亦即無關人命、容許若干錯誤率的人工替代;另一位讀者還用白話把這個道理說清楚了,你從哪裏得到“機器人不用deep learning”這樣的聯想?我還特別給了那個AI除草機的鏈接,只要稍做瀏覽,就會知道它根本和你所講的Robotics扯不上邊,就是一輛卡車搭載一個激光器,用AI來識別雜草;以往的程序能做到AI的辨識率嗎?
AI目前在商業應用上當然主要是靠對網絡消費者做細分來提高營收榨取,但我們不是已經分析過,這對經濟整體沒有什麽正面貢獻?所以話題才會轉到必須找提高農工效率的應用。其實網絡服務不但對經濟效率沒有意義,對AI技術的基礎理論進步也沒有什麽反饋,你自己提起“可解釋性”的研究,難道那是Alibaba發的論文?真正做這方面研究的,還是學術界;那麽水論文怎麽不是一個嚴重的問題?
我說“野蠻發展才有治理的本錢”的時候,前因後果都解釋清楚了,其中特別著重的核心前提,是1.改開初期沒有足夠資本和技術纍積;2.所以當時只能先搞消費性產業和引進落後幾代的技術。相對的,我一再强調過,一旦進步到國際第一或第二梯隊,就必須依靠研發效率,這時學術文化和路綫選擇成爲勝負關鍵,管理方式也因而必須做180°的轉彎。你把這些道理都扔到一邊,就為了杠?至於你寫“後脚就説,AI研究可以跳過...”云云,更是莫名其妙,我的哪句話給人這樣的印象?
正因爲AI是重要的未來科技,我對這行業的專業人員有先天的重視,希望能通過對話來交換心得。但是你顯然是一個“假理工人”,能在一個短短的留言段落裏,有四五個無中生有、或顛倒黑白的論述,即使龍應臺都做不到。換句話説,你的專業詞匯可以成千上百的串聯在一起,但背後的邏輯思路卻是一團漿糊,實際上是在以不入流的文藝生心態在工程領域混飯吃,還好你的專業不是土木或機械,否則必然會鬧出人命。
雖然你已經反復違反《讀者須知》的條文,我還是在禁言一個月和直接拉黑之間做了考慮,不過結論是你欠缺邏輯思維習慣的沉厄深重,不是短期内能改進的;容許你發言對我的時間精力增加太多負擔,所以決定選擇後者。這篇討論我依舊詳細回復,以供其他讀者在閲讀《常見的狡辯術》時作爲參考。
\subsection*{2021-08-18 02:21}

我對這些“可解釋性”的研究不太樂觀,因爲Neural Network説穿了,就是高度非綫性的Optimization,它的效率原本就來自超乎人類Pattern Recognition的細節歸納;强行用人類思路去解釋,當然很容易事後説得頭頭是道,非常方便發論文,但並沒有什麽預測力。這個道理,你只要看Alpha Zero下西洋棋就會明白,每一步棋都可以找到人類既有的棋理來解釋,問題在於既有的棋理很多,而且自相矛盾,AI的厲害之處正在於選擇合適的那一條。人類看AI下棋20多年了,整體棋力只提升了不到50點,而現代最新的Engine已經比世界冠軍高出800多點了;要是Alpha Zero的棋路能用人類思維來“解釋”,就不會有這樣的差距。
既然AI在若干方面效率高於人工,就應該脚踏實地地去實用化,即使並不Glamorous。爲了收割韭菜誇誇而談,對工業發展、國家財政和人民生活都不是好事。
\subsection*{2021-08-17 23:53}

這種實用性的比較,只有實際操作試用才能決定。從理論分析,只能確定液流電池(雖然Ambri的液態金屬電池在科學上並不算是液流電池的一類,但工業特性類似,可以混爲一談)比鋰電池遠遠更適合大規模儲能(因爲在Cycle count和Storage time兩方面的先天物理優勢)。我擔心的是,商業因素造成先到先得,結果次優的技術路綫反而勝出;這在科技史上屢見不鮮,像是Microsoft的作業系統,從一開始就遠遠不是技術上的最優,勝出後獨霸市場,其内含的低效率就只能由全世界消費者買單。
本周有公關炒作的新液流電池廠商,還有美國的Agora Technology,他們的技術是二氧化碳液流電池。此外,利用其它化學作用的液流電池技術也多得很,例如鐵基的就有好幾家Start-up。中國的科技管理核心如果有智慧,就應該把浪費在氫經濟和核聚變上的錢,轉投到各式各樣的液流電池技術上,對國家和全人類都會是件好事。
\subsection*{2021-01-20 13:28}

電解水製氫的效率,的確是氫能源前途的關鍵所在。我所引用的數字,是工業化條件下的總效率,它必須達到80 \% 以上,最好是90 \% 以上,這樣整個氫能量存儲的循環效率才可能達到70 \% ,才有實用的意義。你說的數字,我不清楚,可能是實驗室的數據,也可能是新的結果,更可能是Cherry Picking的數據按摩。
拿美國NIF激光聚變堆爲例,它報告產耗比超過1的時候,你不能當真,因爲它用電雖多,卻只有一部分用來驅動激光,然後又只有一小部分真正成爲激光,然後又只有一部分激光真正到達目標區,然後只有大約1 \% 的激光能量成功轉化為X光,然後只有一小部分X光打到靶標,這個耗能值已經比實際用電小了五六個數量級,但他們就是有臉敢拿它來做分母。然後分子也是同樣按摩過的:我們真正在乎的,應該能發多少電,但是他們報告的,是多少聚變能量被釋放,其實只有一小部分能被捕捉成爲熱能,然後熱電轉換效率理論上頂多是40+ \% 。所以NIF把產耗比誇大了八九個數量級,你説如果有美國憤青來這裏留言,宣稱激光聚變發電已經接近實用化,我該怎麽回答?
我做分析的時候,很小心要及早發現這些騙術,希望讀者們也能有所警惕。

據我所知,要提升電解水製氫的效率,真正可行的解決辦法在於高溫,大約900-1000°C,那麽80 \% 或90 \% 的效率是可以做得到的。當然,這裏假設燒水的熱能價格為零,實際上要達到這樣的經濟性,非常困難,或許高溫氣冷堆可以做到,但低估新技術的價格太容易了,所以我不太願意談。
\subsection*{2020-02-21 05:25}

太陽能和風能即使在20年前,技術路綫已經很明確,沒有基本的工程難關或安全性隱患,所需的研發純粹在於減低成本和提高效率,這只要有政府和企業肯花大錢來支持(例如爲了環保),逐步量產後持之以恆,成功沒有疑慮。換句話說,它們失敗的可能性,主要來自市場選擇,這和氫能源以及核聚變是技術本身内建的固有缺點,完全不同。
像是我以前提過的全釩氧化還原液流電池(Vanadium Redox Flow Battery,VRFB)也沒有什麽嚴重的固有缺點,但是爲什麽我說它成功的機率只有10-50 \% 之間呢?就是因爲VRFB面臨鋰電池的競爭,至今沒有政府和企業的全力支持共識。雖然鋰電池先天不適合電網級別的作業,但是因爲消費者電子產品、PC、然後電動車一共30多年的量產和投資,成本和效率上已經高度優化,而VRFB仍然停留在第一代設計,電極、電解液和尤其是薄膜都有問題。你只要拿第一代鋰電池和現在的比一比,就會知道爲什麽VRFB被最新的鋰電池壓著打。但是鋰電池在電網儲能上的發展潛能顯然不如液流電池,所以這是真正需要政府出手,突破初始難關的一個襁褓技術,如同20年前的太陽能一樣。
\subsection*{2020-02-15 12:15}

三周前我對《觀察者網/風聞》的這篇文章(https://user.guancha.cn/main/content?id=232130\&fixcomment=19962682)做過評論,其實是以往我在博客留言欄回復的總結,所以沒有重發到這裏來。請先讀完《讀者須知》然後小心遵守規則;我不是馬英九那樣的膿包,執行紀律並不會手軟。 
=================================================================================== 
有關稀土,還有一個重要的細節你沒提,就是中國在儲量和技術上領先(也就是你引述的那些數據),指的是傳統稀土礦中經濟效益最高的一類,其實還有其他沒有被大量開采或仔細研究過的礦石,從工程觀點來看,完全沒有實用性的障礙。它們之所以被忽略,正是因爲稀土並不稀缺,所以基於歷史、經濟和環保的原因而被閑置至今。 
這其中對中國主宰稀土供應威脅性最大的,是和釷共生的礦石。恰恰是因爲釷的放射性,所以處理起來困難而昂貴;但是如果釷基的裂變發電技術成熟(參見我的文章《熔鹽堆簡介》),那麽這個缺點立刻會轉化為優點。這種釷和稀土的共生礦石,在地表的總可開采量比中國特產的傳統稀土礦還要大,而其中儲量遠遠最高、品相最好的在印度。。。
\section*{【金融】【戰略】後新冠世界(二)}
\subsection*{2023-02-18 00:48}

中國的經濟學界,在所有社科行業之中,顯然是水平最高的。即使在外交戰略議題上有同樣的天真趨勢,但最起碼在1990年代就拒絕全盤接受華盛頓共識,為人類留下一個能證明昂撒謊言的希望種子,是非常驚人的先見之明。00年代社會思想混亂,經濟系難免也出現雜音,但並沒有整體屈服。10年代國家對美國發動新冷戰後知後覺,是大政略上的認知錯誤,並不是經濟專業的特有責任。
30多年前我還只是個年輕學生的時候,曾經迷過《銀河英雄傳説》。看到楊威利攻占Iserlohn要塞(宇宙曆七九六年的第七次Iserlohn戰役)過程中,帝國艦隊指揮官當斷不斷,任由楊威利自由行動,等到大勢已去才無謂送命(換句話説,這位蠢蛋完美地展現了“戰略定力”和“傳統智慧”),覺得實在蠢得太過,小説家描述失真。後來年紀大了,見識廣了,才明白這是人性之常:正道説明白之後,固然會讓人覺得極爲簡單自然,但若沒人指點,絕大多數人一輩子都無法自己想通。過去十幾年的中美關係,尤其經濟貿易戰綫上的指導原則,也是如此;讀完博客覺得理所當然的邏輯,其實遠超一般學人思維所能及。
\subsection*{2022-04-04 18:58}

致死率和傳染率是獨立的變數,你所引用的論述是文科生的誤解,這件事還是留給懂統計和機率的人來考慮爲善。至於防疫政策的取捨,我已經反復詳細討論過了,沒有重述的必要。
最近兩天,烏克蘭又無中生有、炮製了“Bucha屠殺”,然後不但歐美自動認定俄方的“戰爭罪”和“反人類罪”,華語網絡上許多人開始爭辯是非對錯。但是在討論是非對錯之前,必須先分辨真假(事實上真假往往是對錯的關鍵),而烏克蘭提供的視頻和照片很快就被證僞:首先視頻中有“死者”在微微呼吸移動;其次,俄軍撤離三天之後,烏軍才“發現”路上堆滿死尸,與此同時,鎮上有水、有電、有電信網絡、有幾乎全部居民活著;第三,Gonzalo Lira剛剛發表了分析,證明烏克蘭政府在過去六周發佈、四個名義上不同地點的“民宅被毀暴行“視頻,其實是同一個建築從不同角度加上臨時改裝而拍攝的,這樣已知有系統撒謊造假的消息來源能相信嗎?
當然一般人沒有能力和時間去搜索這些反面證據,但即使是完全無知的人,也可以通過邏輯思辨,獨立考慮“Bucha屠殺”的可信性。這裏最基本的思路,在於做這個屠殺對俄方有什麽好處?如果沒有好處,他們有沒有這個傳統習慣?相對的,假造這個屠殺對烏方有沒有好處?有沒有這類慣例?連這種簡單考慮都沒有能力做的人,完全沒有資格在公共論壇做評論。當然提升一個Metalevel來看,正因爲普羅大衆太笨,所以自然不知道自己有多笨(反之,知道自己笨的人,反而已經遠超群衆的平均程度;參見前文《爲什麽事實與邏輯對群衆無效》中有關Dunning-Kruger曲綫的討論);然而這個機制正是昂撒謊言帝國幾百年成功運作的客觀基礎,是危害當前人類社會最緊急、最關鍵的頭號問題,所以即使必須逆天(指人類愚蠢的天性)而爲,有理智的少數人依舊得要努力闢謠,並且鼓勵沒有能力獨立做邏輯分析的人盡可能虛心閉嘴。畢竟每一個浮面、錯誤的公共論述,都是為壞人開闢危害公益的空間,昂撒體制下的所謂“言論自由”,正是爲了讓謊話、傻話充斥,你以爲他們對實話和智慧有興趣嗎?進一步總結,這些沒有專業能力,卻拼命語出驚人的網紅大V,其實都是爲了一點小小的私利(包括金錢和虛榮),而損害國家社會的公益;受過本博客教育的讀者,實在不應該受這些歪論所迷惑。

有人指出尸體移動可能是視頻錯覺。我接受這個質疑,但其他的論據依舊有效,而且遠遠足夠得到同樣的結論。
此外,Gonzalo Lira報導有烏克蘭議員宣稱Bucha事件由SBU(烏克蘭情報局)和MI6合作製造;很巧的是,俄國在安理會提議要求國際聯合調查,正是被英國否決。
順便提一下,我對Gonzalo Lira是非常佩服的。這倒不是因爲他預測分析的能力超高(其實在邏輯推演未來發展上,我常有不同的意見),而是他以一個典型的文科生出身,在美國受高等教育,能自主看出整個體系的腐朽虛假,已經是難能可貴;然後幾十年不輟地努力揭發真相,不惜放棄高薪職業生涯,更加顯示高尚的品格;至於現在人處烏克蘭納粹政權的管轄之下,還天天公開高調地揭他們的瘡疤,爲了對真理的執著而不顧自身安危,絕對需要莫大的勇氣。智仁勇兼具,真英雄也;只希望他能幸運地活過這場戰爭。
\subsection*{2022-01-13 03:00}

唉,“Group think”在越大的官僚體系裏越強勢。至少有翟教授也開始批評此事,或許慢慢會有其他眼光較高、能獨立思考的學者也加入關注的行列,然後終於説服高層進行必要的改革。
我擔心的是,中美實力反轉在即,如果到了2025、2026年,中方的自信再上升一級,可能躊躇滿志,對這種隱性問題不予理會,而且進一步把已經極度腐化的内在制度反過來向全世界投射,那麽人類社會從昂撒集團桎梏下解放的救贖過程就必然夭折在襁褓之中,兩代人之後只怕中國反過來成爲問題的根源。
這裏的邏輯並不複雜,不過我以往沒有詳細討論這麽遠,所以還是簡單描述其骨幹:中國的基礎科研如果繼續被造假、誇大、玩弄騙術、圖利山頭的學閥把持,那麽中國作爲霸主,全球的科技尖端水平都會停滯不前,導致一方面人類整體的生活水平無法提升,另一方面中國也不可能同時持續雙贏外交和内部穩定,必然只能優先選擇後者,被迫轉而模仿英美靠榨取外國血汗來維持相對優勢和收入水平;這會把19世紀至今全世界無數志士的犧牲努力完全浪費,不見得比再搞一次文革的災難性低。
\subsection*{2021-06-13 22:51}

台灣防治新冠,其實政府的表現大約與西歐(英、法、德)同級,優於印度,而明顯劣於韓國和新加坡。去年的統計數字好看,倒不是作弊(否則病毒在社區自由流行,必然會以指數成長),而是因爲:1)海島容易封關(參考澳洲和NZ);2)政治心態促使及早切斷與大陸的交流,碰巧是防疫的正確措施;3)台灣人民素質高於歐美,至少願意戴口罩(可能是多年前為防治B型肝炎而推動公筷母匙所建立的防疫習慣);4)本省人口可能有歷史瘟疫所賦予的先天免疫力,但對病毒新變種有所失效。
至於自產疫苗,別忘了人類歷史上原本沒有在全新病毒出現三年内實現全面疫苗接種的前例,這次中、俄、歐、美、英(剛好對應著安理會五常)都是不計金錢和人力花費、全力以赴,才如此迅速地進入大規模生產階段,台灣這種規模和層次的工業體,18個月做不出來,是普世現象,先進如日本也同病相憐,有什麽好大驚小怪?民進黨真正的罪過,在於拒絕來自大陸的疫苗,這很明顯是爲了滿足自私政治心態而犧牲許多無辜生命的謀殺行爲,受害的台灣群衆卻因爲媒體宣傳和群體壓力而不知反抗,這正是我多次討論的美式民主的邪惡之一。
印度人的生醫論文,必須等先進國家反復驗證才能信,目前應該暫時把Ivermectin與牛尿和蓮花清瘟放在同一級別。
\subsection*{2021-06-03 04:41}

這種“瘟疫考古學”,比你想象的還要難些;例如1889-1890年流行的Asiatic Flu(又稱Russian Flu;請大家不要對名字過度解讀,當時的科學界連病毒是什麽都不知道,基本是哪一個英語系報紙先公開報導疫情,就可以隨便取名),到底是H2N2還是H3N8的流感,吵了近一百年,到本世紀分子生物學工具進步了,被發現反而最可能是OC43,這正是一種冠狀病毒!
目前能做出全基因解碼的最早瘟疫病毒,是1918-1919年的Spanish Flu,被證明是現代H1N1流感的老祖宗。病毒面對露天的氣候和環境很脆弱,所含的分子非常不容易保存,在1997年研究人員特別到Alaska北部凍原裏一個在1918年被團滅的村落去挖墳,才終於獲得完整的病毒基因。後來在美國和德國的醫院又分別找到當年對受害者解剖的病理樣本,得到部分基因組合,所以學術界剛剛可以開始比較當年疫情大流行過程中,相差幾個月不同變異種的傳染力和致命力差別。
前面提到的OC43冠狀病毒,如同H1N1流感一樣,也仍然還在人類社會中流行傳染,只不過毒性(Virulence,不是Toxicity)已經大幅降低,致死率低到可以忽略不計。事實上我們一般所謂的感冒(Common Cold),並不來自單一的病原,而是許多種毒性溫和的呼吸道病毒的總稱,其中包括至少4種冠狀病毒。從病毒-寄主共同演化(Coevolution)的角度來看,很可能每一種感冒病毒都曾經在剛突變獲得人傳人能力後,造成過全球性的瘟疫大流行,在人類社會以大批死亡為代價獲得群體免疫之後,慢慢降級為不致命的騷擾性(Nuisance)感染。新冠是人類第一次在疫情尖峰期就靠著全面疫苗而壓制的呼吸道瘟疫,雖然救了很多性命,卻也打斷了共同演化、互相適應的天然進程,其中包含病毒毒性逐步降低的傳統機制。所以當疫苗的使用結果被統計出來,發現仍然有相當大比率的輕症患者,並不見得是壞事,因爲演化作用或許還能憑藉這些輕症病例而正常運行(亦即毒性低的病毒對毒性高的變異種有排斥淘汰的作用,而且後者可以比較容易辨識,然後隔離而消滅)。
我所猜測的東南亞/南亞瘟疫歷史,並不對應著那4種已知的感冒冠狀病毒,因爲後者是全球性感染,無法引發區域性免疫力的差別,而必須是類似SARS和MERS那樣傳染力較弱、在傳播過程中有可能會自行消失的病毒,向西不進入伊朗,向東、向南沒有傳入菲律賓和印尼(這兩個島國有他們自己的蝙蝠和冠狀病毒,但品種和亞洲大陸差距較大,免疫效果必然也較低;事實上這兩國在第一波疫情下和中南半島國家的對比,是支持我這個假説的最强間接證據);至於台灣,雖然也是海島,但多數人口是最近400年的移民,可以從福建把免疫力帶過去。不過這些古代甚至上古的區域性疫情,既然沒有造成超大規模傳染,又已經自然消失,歷史上根本沒有記載,要直接去找基因證據,基本是不可能的;尤其我談的不是絕對免疫,而只是稍有緩和作用的部分免疫效果。唯一可行的考古研究方向,是對現代人口的免疫基因做跨國性的統計分析,然而這需要對人類基因和免疫系統有極度深刻而全面的瞭解,明顯超出當前學術界的知識範疇,或許再過2、30年會有可能吧。

我在前一個回復裏,談到歷史上冠狀病毒所造成的瘟疫流行,可能賦予部分人口額外的先天免疫力,但是研究不容易做,需要許多年的持續努力。這裏是今年四月發表的一篇英國論文,宣稱找到第一個已知的新冠免疫基因,有興趣的讀者可以去閲讀參考:The influence of HLA genotype on the severity of COVID‐19 infection - Langton - - HLA - Wiley Online Library
\subsection*{2021-06-02 10:36}

幾個月前,DesertFox提過這個話題,認爲台灣醫生社會地位高、醫療體系健全可能是疫情被控制的主因。當時我已經談到我懷疑蝙蝠傳人的冠狀病毒可能在過去幾千年曾經在東南亞和南亞反復發生,導致這些地區的群衆先天就有若干免疫力,雖然沒有任何實證,但這是唯一一個能同時解釋爲什麽台灣、越南、Laos、Cambodia和印度在第一波疫情的防治上都特別輕鬆的理論。
現在新冠的變異種已成主流,推動了新一波的疫情流行,而前述的那些地區無一例外都面臨創記錄的患病率和死亡率。這依舊不算實證,但那個理論又一次成功解釋了所有的新現象。在科學地探索未知機制的過程中,如果還不能直接達成絕對的邏輯結論,就只好樹立幾個假説,然後慢慢搜集間接證據,根據Bayesian Statistics來更新計算它們的機率。這個“既有免疫力”的假説,剛剛因爲台灣疫情的新發展,而進一步獲得可能性的提升。
\subsection*{2020-09-04 21:25}

大戰略上是對的,但是在戰術上沒有選擇最優。
這裏最大的問題在於他把重點放在德國,但是德國一直是白左的主要基地之一,媒體和智庫又被美國人徹底滲透,所以中德外交不能高調進行,否則必有反彈;果不其然,德國人現在出來評論說他們“不願當超强爭霸的玩物”,要修改亞洲政策,“分散風險”。中德關係原本自有經貿作爲錨碇,法國人才是喜歡玩合縱連橫的民族,所以我一連寫了好幾篇文章强調外交重點應該放在法國。
有關白左對中國的抹黑,倒不是外交部的錯。香港的事西方擺明要搞雙標,並沒有回擊的好辦法。但是新疆基本是兩三個造謠核心(即世維會和澳洲的一個智庫,都明顯是CIA資助的組織)經過偏頗媒體不斷共鳴放大的結果,像是《Grey Zone》就能很簡單地把他們的底挖出來,一個14億人大國的中宣部,沒辦法順籐摸瓜,把整個假新聞的演化過程整理出來,交給外交人員用來反擊,實在是匪夷所思。
\subsection*{2020-07-09 06:30}

你顯然沒有讀懂我的博文。
我在博客上一直反復强調,中國要能持續、甚至加速崛起,除了打鐵自身硬之外,外交政略上的決定性突破點就在於爭取歐盟。只要歐盟不和美國一鼻孔出氣,美國身邊的小集團實力已經不足以遏制中國。
拜Trump和新冠所賜,在過去兩個月,德、法和歐盟一同明顯地改變了二戰後75年來的親美政策,表明要在中美鬥爭中保持中立;這為中國未來幾十年的發展,掃除了最後的外部障礙。
英國的脫歐,不但成爲自身崩潰的起點,也是歐盟進化的轉捩點。德國雖然一直是歐盟的經濟核心,但是後者的政治權力以往分散在英、法、德三國手裏,而英國遇到强化歐盟的提議必然反對,所以德國不可能進一步主導歐洲事務,那麽自然也沒有動力在財政上犧牲小我。像是這次歐盟針對新冠的7500億歐元財政補助,如果英國還在,根本不可能通過。原本德國也和北歐國家一起反對,但一旦德國人明白失去英國的歐盟有著加深統合的機會,他們就改變心意,把錢掏了出來。那些北歐小國鬧幾天是可以的,到了表決日還是得支持全票通過。
所以在可見的未來,歐盟會對内加速整合、對外采取獨立政策,天下三分,正是中國最理想的發展環境。
\subsection*{2020-07-07 20:13}

一)最近有少數幾篇論文宣稱部分患者的抗體會快速消失,這還沒有被完全確認;即使是正確的,抗體計數和免疫力也沒有絕對的關係。 
二)新冠是所有RNA病毒中最大的一類,具有對基因複製做校對的功能,所以變異速度並不高。最近風傳的重要突變依舊在於Spike Glycoprotein的小更動,似乎能提高傳染力,但是對後天免疫系統的影響還沒有任何實證。 
既有抗體幫倒忙的前例是有的,最有名的是登革熱,第二次感染的病情反而最嚴重。較少爲人知的是西班牙流感,有理論認爲1890年開始出現的“Russian Flu”如果是病人一輩子第一個嚴重流感,也會促成類似的效應,從而造成1918-1919年青少年的死亡率特別高。不論如何,目前沒有任何證據顯示新冠也有這個效應,如果不幸真的如此,最大的麻煩是在開發疫苗上。我們暫時不必杞人憂天,靜待研究單位發表進一步的結果。
\subsection*{2020-05-31 16:04}

五年多前Michael Brown被槍殺之後,我已經針對這個議題一連寫了好幾篇博文,解釋這才是美國素來的真相;所謂的“體制優勢”,其實是因爲美國掠奪所獲的財富和特權纍積太多,百姓和移民有錢賺,體制再不合理、社會再不公平,也可以忍著。
本月這兩篇文章,則討論在經濟下行、貧富不均、種族歧視、社會分裂的背景下,美國因爲財閥掌控宣傳,不可能做出根本性的改革,反而養虎爲患,被一個民粹政權加速消耗以往的國際信譽和行政能力,幾年下來只剩下美元這個最後的老本+堤防。
這些社會動亂,還不到能直接掀翻美國的地步,但它們是國際輿論上壓死駱駝的最後一根稻草,應該會幫助歐盟的務實派説服白左勢力脫美向中,一旦中歐聯手推翻美元的國際儲備貨幣地位,美國這個紙牌屋就要轟然倒塌了。
\subsection*{2020-05-30 01:46}

這是你在這個博客留下的評論最好的一個,原因是你自已做了獨立的邏輯分析,而不是拿書本上的理論來硬套。
最近有不少人討論新冠會帶來通縮還是通脹,這當然是連前提都沒搞清楚的無意義爭議。我在四五年前已經特別寫了博文解釋過,討論通脹/通縮必須指明資產的類別和性質,工資和原材料價格沒有必要走同一個方向,工業成品和金融資產更完全是兩回事。
一個嚴重的經濟危機自然會壓低大部分實業的上下游價格,然而量化寬鬆卻會捧高金融資產,所以必然會加劇貧富不均。2009年中國政府的四萬億財政刺激,最大的毛病就出在沒有考慮這一點,容許這些國家給的資金去追逐金融資產,尤其是房地產。希望他們這次能換上有點主見的經濟管理人員,而不只是照搬美式經濟學理論;畢竟美國人作死胡搞,有美元托底,中國可沒有這個餘裕。
\section*{【戰略】【國際】對俄烏戰爭的新觀察}
\subsection*{2023-02-18 00:44}

是例行公關。
Bakhmut打到現在,勝負毫無疑義,繼續讓Wagner進攻下去,已經不再是最優的配置(尤其最近炮彈供給受限,可能是爲了囤積足夠的基數,為全面攻勢做準備)。這裏的問題在於巷戰攻堅是非常專業的活兒,一般由特種兵(不是SEALs或Delta那個級別,而是像一戰德軍的Stormtroopers和二戰美軍的Rangers)執行,普通的機械化部隊只適合打運動戰;而整個俄軍戰鬥序列中,遠遠最大的攻堅專業戶就是Wagner。偏偏他們的老闆Prigozhin和俄國國防部彼此看不順眼,所以Wagner轉移戰綫固然有必要,但實際細節選擇值得觀察玩味。國防部當然希望把那四萬精兵打散了,交給各前綫戰區指揮運用;而Prigozhin只有在Bakhmut這樣由Wagner獨當一面的時候,才有指揮權和公關機會。昨天有目擊報導說幾千名Wagner戰士向南綫調動;考慮到俄國的一個精銳機械化旅剛剛在Vuhledar吃癟,Zaporizhzhia正面的攻勢也是雷聲大雨點小,顯然Putin對那裏的進展並不是特別滿意,但Wagner是去幫忙還是替代正規軍,目前還沒有定論。當然,Wagner的合同只有六個月,幾周前又停止招收監獄犯人,所以也可能會在夏天之後逐漸縮編淡出。
同樣在昨天,Zelensky宣稱俄軍的大攻勢作戰已經開始,這當然又是胡扯;真正打響要看Sumy方向的大規模投入。
\subsection*{2023-02-13 10:30}

Seymour Hersh的描述,在美國政治外交戰略運作原理這個大綱上,是完全精確的,但是在細節上往往並沒有(事實上也不太可能會有)足夠的佐證。我一再强調這一點,並不是對他的批評,而是預先提醒大家對揭發昂撒虛僞宣傳洗腦的資料,必須有正確的學習心態,不要因爲一點枝微末節的問題而無限上綱。博客本身也一樣,所以才會有《讀者須知》的8A條規則。
Hersh的文章,當然比Pattberg還更爲重要;好在這次他終於在華語界打開知名度,希望不用我推薦,也會有後續的翻譯、引入。
美國專門揭發政府醜聞的媒體明星,被主流社會接受、甚至吹捧,是1960末、70年代初的短暫現象,有其獨特的時代背景。最有名的兩個人是Hersh和Bob Woodward;後者很膚淺,完全接受自由主義價值觀和殖民帝國世界觀,所以實際上一直都只是民主黨用來黨爭的工具。Hersh才真正理解昂撒的邪惡本質;但正因如此,反而沒有Woodward那麽出名。媒體人要做好爆料真相的工作,成敗的關鍵在於能否有足夠可靠的消息來源,亦即體制内的良心人。但從良心人的觀點來看,他們並不瞭解媒體界的内幕,要找到可信任的爆料管道,往往是最難的關鍵,所以這也自然成為Deep State嚴厲防範和打壓的焦點,近年對Assange的迫害,正是爲了殺鷄儆猴。Greenwald在Snowden事件之後,成爲監控的重點,而且這些監控並不受掩飾,也是爲了產生嚇阻作用。Hersh的低調,反而有助於這次内幕消息來源與他聯絡;但他垂垂老矣,幾年後很可能就後繼無人了。
至於資本主義體系下,權力分散,有許多獨立山頭,彼此同時做鬆散合作(殖民壓榨)和競爭(利益分配),這源自資本本身就有多元性的特質,博客已經反復解釋過。白宮當然是山頭其中之一;而其決策的非理性因素,在近年急速擴張,完全壓倒理性思維,其背景(歐美思想文化的整體腐化)和細節(Deep State和NeoCon的持續興起)也都是博客素來的討論重點,請自行復習。
\subsection*{2023-02-06 14:56}

你拿俄烏戰爭和越戰做對比,是個很好的觀察;這裏的差別,在於前者不但是謀定而後動,而且已經退守到己方邊境勢力範圍之内,道義和後勤上都與遠程侵略相反。此外,我還要補充一點:越戰也是美國貨幣超發、迫使Nixon打破Bretton Woods的主因,從而引發了全球通脹、美元石油霸權和後來美國經濟金融化(爲了扭轉滯漲,除了美聯儲嚴控需求面之外,也必須從供給面壓制通脹,對工資循環上漲做釜底抽薪,是80年代開始產業外包的重要背景因素,否則當年不少政經精英對去工業化有所質疑;參考Volcker自己的演講詞:“A controlled disintegration in the world economy is a legitimate object for the 1980s.”)。
至於實際執行,俄軍動員了至少50萬機械化部隊,現在下場的從去年九月的1/5提升至1/3,就已經扭轉戰場態勢,讓烏克蘭窮於招架,後續純粹是貓玩老鼠,最終結果沒有任何疑問;你沒注意到上周我上節目,完全懶得討論戰況細節嗎?
\subsection*{2023-02-05 05:25}

談基礎服務的話聲剛落,就有報導(參見https://thedreizinreport.com/2023/02/04/ukraine-hr-the-praetorians/)烏克蘭開始將警察編組為“Assault Guard Regiment”“近衛突擊團”,派上前綫應該是未來一兩個月的事。烏克蘭原本就黑幫橫行,一旦警察人數大幅下降,治安必然落下斷崖,亦即索馬利亞化。

烏克蘭名義上有4000多萬人口,實際上2014年之後人口向歐洲流出一大批,去年開戰前後又跑了一批,這些都以年輕力壯的勞動人口爲主,然後東烏親俄民衆也有向俄國逃亡的,所以計算戰爭潛力所用總人口基數,其實只相當於2000萬上下。當前的戰損人數可能已經達到1.5 \% ;通常工業化國家的極限在2-3 \% 左右(二戰期間,蘇聯的農村人口比例依舊很高,而且政治管控極爲嚴密,所以可以大幅超越極限,光是陣亡就達到5 \% ;美國南北戰爭,南軍的陣亡人數則佔總人口2.7 \% ,但當時傷兵救活比率低,而且絕大多數是農業人口,還有等同白人人口50 \% 的額外黑奴勞動力),這是因爲有許多基礎設施和服務不可能因戰爭動員而放棄(例如鐵路,在戰時反而更爲重要),即使烏克蘭不必在乎糧食、能源生產,也不在乎財政收支,物資的運輸、供應和分配還是需要人力的,而且其政府高度腐化低效,所以我認爲烏克蘭兵員枯竭應該是年内的事。
俄方沒事不會跨越國界主動出擊,反正主宰北約的Neocon必然會無限升級,届時見招拆招便是。至於最終目標,說是打垮美國或歐洲都不精確,瓦解北約這個侵略軸心才是關鍵。
\subsection*{2023-01-06 11:01}

Surovikin顯然覺得在Bakhmut一天殺傷幾百人,是筆很划算的生意。原本在12月動手是個選項,但既然天氣不容許,對烏軍放血也值得繼續,拖延一段時間也是合理的。
去年底我用二戰戰史來參照解釋雙方策略謀劃的時候,曾提到Donbas易守難攻,若要做運動戰就必須迂迴。因爲南綫走廊太窄,二戰期間不適用,不過去年俄國已經預先占領了Mariupol/Melitopol走廊,那麽這次要從那兒出擊倒不成問題。上次我之所以沒有提從南綫北上的選項,是因爲當時烏軍還在夢想著向Melitopol發動攻勢,部署了重兵,又可依托長期建構的堅固工事。不過過去幾周的新發展,是烏軍指揮官也預期會有北綫攻勢,在Kiev和Kharkov之間準備了預備隊,反而是Melitopol方向的部隊被調到Bakhmut去填坑了,結果現在南北合擊成爲新的可能。這又一次印證“兵者,詭道也”;用現代科學語言來説,就是戰爭謀略,是連“回合”都沒有明確定義的開放性、連續性、全面性多玩家博弈,戰情瞬息萬變,直覺、反直覺、反反直覺都可能是最優解。當前的“正道”是在一月底或二月初發動南北夾擊;至於其負面因素,雨季倒還是次要的,最大的問題在於對手已經事先預期(參見Zaluzhnyi接受《經濟學人》的訪談)這個時間點,所以在兩翼佯攻以牽制烏軍預備隊的同時,多花幾個月在Donbas做正面硬攻以徹底耗光烏克蘭人力資源,反而在戰略層面算是出奇制勝的選項。
\subsection*{2022-12-22 17:49}

不對。國防建設必須基於底綫思維:如果有危險,只要經濟負擔得起,就必須做準備。而且準備充分反而是嚇阻敵方不做升級挑釁的不二法門。這裏俄國建軍所針對的5-10年期國安威脅,首先是波蘭的瘋狂擴軍,可能對Belarus或未來殘存的烏克蘭尋機動手;其次是短程核導彈,如果部署到芬蘭或波羅的海三小國,飛行到莫斯科時間低於10分鐘,預警、攔截、反擊都會有困難,更別提St. Petersburg,唯一的有效威懾正是非核突擊;換句話説,在傳統軍力上獲得絕對優勢,恰恰是避免三戰的最佳手段,而俄方以軍備競賽壓倒北約的可能性,是過去這一年剛剛才被歐美軍工的囧態所揭示的。在經濟層面,中印都已表態,保證了未來的能源出口基本收入,所以節衣縮食、撐過美國霸權消亡的垂死掙扎,並非不合理。
\subsection*{2022-12-05 22:49}

好的,我誤會你了。
不過正文也解釋了,Putin的優先目標,在於重建國民的國家民族意識,戰場上打得再怎麽不好看(對一般外行民衆來説),只要不影響最終結局,又能激起愛國回憶,仍舊是最優解。這應該事先解答了你質疑的大半,而我素來不喜歡忽視既有答案的留言。
至於戰損比本身,反而有相當不確定性。我自己也只是大略估計,若是有人想做理性的反駁,只要有新的證據或論述,絕對可以。但如果只是主觀認定有差異,那麽多談無益,到此爲止。
俄軍在Putin想要動員群衆的政略前提下,在第二階段以殺傷對方人員為目的,絕對是最優戰略選擇,而且這個目標原本就需要時間和耐心。歷史上蘇軍的抗戰花了三年,才達成掏空德軍的任務;1944年的Operation Bagration摧枯拉朽,其實靠的是那之前許多戰役對敵方所作的消耗。反例也很多,像是1940年的英倫空戰期間,如果Hitler不被英軍轟炸德國城市激怒,將空襲目標從英方空軍基地轉爲倫敦,其實後者當時已經很接近徹底崩潰,那麽勝負説不定就逆轉了。事實上昨天我討論俄軍即將發動的冬季攻勢,不敢對時間點把話説滿,主要擔心的也是如果Surovikin評估對烏軍的消耗還不夠,就必須繼續再磨幾個月;否則俄方自己的準備是完全充分的。
\subsection*{2022-12-04 02:47}

你顯然是中國網絡上的“軍事評論”看得太多了,智商被拉下來到他們的程度。要警惕,這個大衆媒體的愚化效應不止適用於港臺傳媒,一般大陸網絡也是一樣的,尤其軍事論壇還特意僞裝為專業,是個格外危險的陷坑。
我已經反復解釋,像是席亞洲之流,連基本的科學分析原則都不知道,實際學術研究經驗恰恰爲零,一輩子對真正深刻廣汎的邏輯思維毫無接觸,他們所依靠的,除了幾十年土法煉鋼、纍積了許多類專業的詞匯之外,主要是和面熟的幾個普通中級軍官喝酒聊天,讓他們拍腦袋回答問題後再拾其牙慧。談的若是新武器的獨家細節,那還純屬復述簡單内幕消息,沒有什麽太大的問題(不過這所需的腦力運作,等同集郵,不能與科學分析混爲一談);一旦觸及戰術,讀者就必須小心;要是談起戰略、甚至政略,則基本可以事先確定是軍事幻想小説。我以前强調過,絕對不能把科幻小説當學術策劃的基礎;同樣的邏輯在這裏也完全適用,爲什麽會有人把明顯是娛樂用的虛構文章拿來當真呢?
以俄軍近年廣汎采用的BTG(Battalion Tactical Group, 營級戰術群)爲例,這只是戰術級別的議題,但中文論壇一樣把它扯歪到事實的母親都不認識的地步,幾百篇評論反復批評它頭重脚輕等等缺點。但俄軍將領並不是不懂軍事原理,而是面臨國家特有的合同兵+義務兵制度,他們的服役年限和專業程度都有不同,爲了知識傳承、高低搭配,必須混合在同一個旅裏面;然而若要出國作戰,卻因法律明文規定而只能派出一個集中合同兵所構成的營級單位,所以平時訓練也必須考慮到這種臨時性編制。這個先天的外加限制,是俄軍設計采用BTG的決定性核心因素;討論BTG的得失,怎麽能不反復强調這一點呢?何況這個考慮連Wikipedia都有記載討論;我十月上《龍行天下》,還剛談過,純靠Wikipedia來做分析是小學生級別的行爲,連Wikipedia都不查就妄作評論,還不如小學生。
正文裏已經詳細解釋了,俄軍今年的戰略運作,是在國際情勢和國内現狀嚴格限制下,當代政略大師Putin所選擇的最優解。中國軍事論壇充斥著自我臉上貼金的胡扯,例如所謂第一階段顯示俄軍後勤能力不足:你若是受命要在敵軍遍佈的環境下,沿公路分秒必爭地强行軍1000公里,那當然不可能停下來處置抛錨的車輛;但這是戰略需要,和後勤一點關係都沒有。現在俄軍在十個月内發射了1000多萬發重炮炮彈(相當於美國年產量的30倍)、近3000發導彈(超過美軍歷史上纍積的總發射數量),這樣的表現還被稱爲後勤不足,顯然純屬嘴炮無敵。
真正入流的分析,比這些網紅的智力所及,要高出至少兩個層次(亦即要從偽專業提升到半專業,然後才是專業)。但是即使做到專業水準(指軍事院校中,鳳毛麟角、對戰史、戰略、戰術有獨到心得的學術專家,不是街上隨便拉個穿軍服的)的深入分析,也不可能完全確定戰事的未來走向;這是因爲“兵者,詭道也”,虛虛實實,正反兼用,反而才是最優解,而今年俄國的表現,至少在政略和戰略層次,都達到世界第一流水平。
\subsection*{2022-12-03 21:06}

這個15:1是上限;若同時考慮東烏民兵傷亡,以及烏軍的大量MIA,總交換比可能在5:1或6:1左右。然而東烏部隊雖然有俄軍軍官幫忙指揮,在訓練、紀律、裝備、火力上依舊比正規軍差一級;如果只看正規部隊在第二階段的對戰,我想10:1才是最佳估算值。
至於接下來俄軍會怎麽打冬季攻勢,剛好我先回答下一樓的時候,提到即使做出專業性的深入分析,也只能給出正道方案;然而兵者詭道也,所以並無法事先確認俄軍會做何選擇。不過正道是一切斟酌、設計和變化的基礎,所以這裏我也稍作介紹。
俄烏戰場其實也是二戰歐洲東綫的重點,屬於德軍南部兵團集群(Army Group South)的責任區。北部兵團集群始終是助攻,中部兵團集群則在莫斯科外圍焦灼了兩年多,大部分出名的運動戰和攻堅戰都是在南綫打的;這是因爲與前兩者所面對的森林沼澤地形不同,烏克蘭和其東面的南俄主要是所謂的Steppe地帶,地勢平坦開闊、人口密度不高,易攻難守,只有城市區才有利於防守方;而面積最大、對阻滯大規模機械化攻勢最有效的城市帶,正是因爲富產煤礦而發展起來的Donbas,尤其是南起Donetsk、北至Sevorodonetsk、東到俄烏邊境、西至Kramatorsk,長200公里、寬150公里的一個橢圓型區域,有丘陵起伏和許多臨近的中小型城市,互爲犄角。
在二戰的實際戰史上,德軍在1941和1942兩次向東攻打烏克蘭,然後蘇軍在1943和1944年也兩次向西反攻。這四次大會戰中,德蘇都試圖先硬吃Donbas,全部失敗。要繞行包圍的話,南面的Mariupol走廊太窄,明顯會被簡單側面打擊,只能走北面。但是西北方150公里處,剛好有一個大城,亦即Kharkov,如果不先拿下來,硬是從其間穿過(也就是Izyum走廊),會被守軍兩側夾擊;蘇軍在1942年和1943年都犯了這個錯誤,兩次從Izyum出擊,兩次慘敗。所以先占領Kharkov,然後打敗對方必然的全力反攻,才是正道;歷史上也因此而的確有了四次Kharkov戰役(Now you know why they happened)。一旦守方的反攻失敗,Donbas就面臨被包抄的危險,只好主動撤退。
我想提醒讀者,上述的道理,雖然中國網絡軍評不懂,卻是蘇聯軍事學院教材的重點;當前交戰雙方的將領,都是熟讀這段軍史出身的,所以他們的作戰計劃,必然基於以上的考慮,然後再決定是走正道還是反其道而行。外人要做分析預測,也必須從這裏出發,否則必屬空想。例如2014年,Putin決定放棄屏蔽Crimea的Kherson和Odessa,卻堅持保護Donbas,很可能就參考了軍方的建議。今年夏天,俄軍從Izyum突破,立刻被烏軍全力阻擊,而前者並沒有持續加碼,應該也是雙方都考慮了戰史。烏克蘭的九月攻勢,事先宣傳會是Kherson方向,然後實際主攻是從Kharkov向Kramatorsk夾擊其間的俄軍,後者也居然自願唱空城計,戰史的教訓顯然有心理影響。
當前俄軍已經占有Donbas的3/4,正在重點打擊Bakhmut。如果Surovikin采行正統運動戰策略,這個攻勢可以吸引牽制烏軍精銳部隊,方便俄軍主力從北面插入Kharkov以西,距離Kiev250公里的Sumy走廊,從而包抄Kharkov,由北向南席捲整個Dnipro河東岸。一個簡單的變招是可以反過來用北綫攻勢來威脅助攻,然後打破歷史常規,輪番吃下烏克蘭在Donbas的最後三個主要據點Bakhmut、Kramatorsk和Slovyansk。這是上月在《龍行天下》,我說預期俄軍會從Donbas和北綫兩面夾擊的論證依據。當然,Surovikin也可能會想出出人不意的怪招,或者因政略要求而做其他選擇(例如如果Putin要求故意拖延戰事以徹底打垮歐盟經濟,那麽可以采納第五次剿共戰爭的步步爲營戰法;不過機率實在不大);我向來不願意保證軍事議題的預測準確率,正因如此。不過這不是輕信網絡騙徒的藉口:如果他們連二戰戰史都不詳細考慮,顯然只是胡猜,可以客觀認定只有隨機誤打誤撞的意義。
\subsection*{2022-12-02 02:27}

你“一直懷疑”的原因,很可能是因爲我在四五月間開始就反復在留言欄和視頻訪談裏强調,俄軍第二階段作戰的目標是殺傷。一般人如果有多個信息源頭,往往一段時間之後就不記得哪裏來的;不過博客是全華語世界正確原創分析的主要來源,我希望至少老讀者不要拿來和網絡上那些天下一大抄的大V或學者混爲一談。
其實俄軍在過去八個月專注在殺傷對手這件事,在我所分析的許多議題之中,算是相對重要、因而頻繁重複討論的;自然也方便被廣汎抄襲,很快成爲“共識”。然而其他還有很多是一筆帶過的細節,所以《讀者須知》第一條建議大家多多復習,特別有其價值。例如意大利新首相Meloni十月上任以來,除了為難民問題和法國政府發生衝突之外,在對俄政策上完全遵循歐盟主流,昨天剛剛宣佈接管俄方所擁有的煉油廠,並且加碼援助烏克蘭;這個政策取向我在九月底意大利大選結果剛出爐的第一時間,就已在留言欄做出精確的分析和預測,然而只提了一次,不用心的讀者可能早已忘懷。能可靠地提前幾個月獲知天機,對有學習欲望的人,是極爲珍貴的事,如果偷懶不復習而徑行忘卻,非常可惜。
\subsection*{2022-12-01 03:58}

這個比較不成立:在軍事層面,海灣戰爭的兵力對比是3:2,美方占優,不是1:3。在大戰略層面,美國以世界霸主領導所有主要軍事强權,打擊一個孤立無援的第三世界國家,雙方的GDP比例超過100:1,而現在俄國卻是反過來獨力對抗整個北約,即使加上提供武器的伊朗,GDP比例依舊是1:40。

有個合理的質疑是東烏民兵的傷亡並不包含在俄國國防部的數據裏;如果我們接受主流估計中的上限最高值,亦即陣亡兩萬,那麽雙方戰損比是4:1。不過烏軍一直不用心為己方陣亡將士收尸,連俄軍多次建議局部停火以方便清理戰場都反復拒絕,所以失蹤人數比例遠高於歷史上的其他戰爭,而這些MIA是沒有算進陣亡數字的;因此總結來看,俄國在以一敵眾、以少攻多的背景下打出這樣的結果,必然是非常優秀的。
\subsection*{2022-10-09 09:53}

首先,烏克蘭的雨季剛剛開始,現在去占領並防守大片的空曠平野,不但攻勢必然後繼無力,連持續補給都不容易,而且前綫部隊住在泥濘的戰壕裏,日子必然很凄慘。
其次,最近俄國的民生狀況相當好,危險在於油價會因全球經濟衰退而下落。但是OPEC+已經決定減產200萬桶,這是Putin所建議的兩倍,可見其外交大獲全勝,足以在短期内保障國内財政和經濟平穩。
至於防疫政策,Omicron的天然致死率已經比以往的COVID低了一個數量級,只要針對性的疫苗普遍施打過了,把致死率再壓低一個數量級,就沒有必要繼續强迫歸零,所以開放應該是明年上半的事。我建議開放之後,國家統一發放抗血栓保健綱要、由醫生視情況開藥給患者,以減低Long COVID的危險。
\section*{這個部落格}
\subsection*{2023-02-11 02:30}

你說的並非沒有道理,但我不特別擔心,這是因爲中小學的教程原本就只含學界和社會已有共識的結論,所以只要教育部不是異常地無能腐敗(然而這麽基本的要求也不一定滿足),AI所導致的偏見應該局限於整體社會共享的迷思。迷思的動力來源又分兩類:利益集團的推動,例如昂撒、學閥和商業吹噓,以及人類自然演化出的内發心理偏差,例如宗教(含中醫教和科技教)和流行(剛好看到這篇新論文https://link.springer.com/article/10.1007/s00265-023-03289-8,證明鳥類築巢也會追求新季的流行式樣,這顯然不是高等邏輯思維)。這些都是博客反復討論的問題,因爲是以小博大、而且逆天行事,可能必須有執政高層的理解才能解決。
\subsection*{2023-02-10 06:16}

《The Economist》讓ChatGPT去解高中數學題,只能做到1/3正確(亦即常見,所以網絡上有多重討論的1/3題目);然而用來寫企業研究報告,已經達到全美排名第一商學院Wharton School的水準,更別提龍應臺那類無病呻吟的散文,幾分之一秒鐘之内寫幾萬字,完全不在話下。你所提的長篇小説,難處在於敘事的邏輯結構,而不是文筆上的問題。
我雖然寫博客、上節目,但工作的本質並不是典型的傳媒,而是學術性研究。這裏的差別在於我用的事實固然來自閲讀(但不是方便的主流媒體,而是被昂撒體系有意淹沒的少數真實來源),分析卻是自己根據第一原則和嚴謹邏輯獨立做出的;如果必須引用別人的看法,和證據一樣,都會列舉出處。當代媒體網紅,即使有教授頭銜,依舊剛好相反,分析抄別人(但拼命掩飾抄襲),只有所需的證據反而是自己腦補編造的。想一想,做原創邏輯分析、和抄襲他人説法,哪一個比較難?AI可以替代的是後者,但選擇的唯一標準是量大,連人類網紅對優質内涵的有限理解都完全喪失,所以AI將抄襲、然後總結的工作,廉價化、普及化的社會效應是持續拉低大衆的愚昧、隨時間等而下之;願意並且能夠欣賞真正學術分析和事實真相的少數理性知識分子,並不會直接受到影響,但是和佔人口絕大多數的庸人,其差距隔閡會不斷擴大。這固然對選票制社會有立即的負面作用,對内宣薄弱、中層治理思維水平不高的中國,也不是好事。
\subsection*{2023-01-24 13:06}

我對中國社科學術界不熟,不能確定你的描述是否精確,不過應用在1960年代後的美國社科界是完全貼切的。現代社科界量化分析的潮流始於經濟學,而其鼻祖是MIT的Paul Samuelson(亦即Larry Summers的伯父)。Samuelson本人雖然和Friedman同代,倒不像後者那樣是100 \% 財閥代言人;他引入數學模型,似乎並沒有惡意,而只是爲了追求新奇、方便發表論文。然而即使是高能物理這樣的自然科學,脫離現實的數學模型(超弦是典型)一樣成爲無意義的玄學空談;社科議題絕對不對應著簡單的數學模型,那麽要在論文中建立後者,就必然要脫離現實。換句話説,自然科學裏的數學模型還有好壞之分,社科界也搞數學模型,除了教學和發假大空論文方便之外,不可能有實際意義。所以美國社科界在過去半個世紀極度追求量化分析,有兩個基本推動力:除了學人内發的“Publish or Perish”考慮之外,也有外來的資本財閥要腐蝕學術界的意圖。畢竟深刻正確的邏輯辯證,自然會揭穿新自由主義的謊言,那麽鼓勵學術主流去搞量化玄學,就是幫助抹殺淹沒實話的必要方向。
數學在社科的正確運用,在於統計資料庫的建立和過濾;這是因爲未被虛僞模型假設所扭曲的統計結論,是做任何邏輯分析的事實基礎。博客討論過幾個常見的統計悖論,但還有更多沒有提及的細節必須小心應用,因此若要建立龐大完整的Database,在學術人力的質和量上都有很高的要求。不過我所知的社科人物,都遠遠沒有足夠的統計學常識,完全談不上過濾數據;再加上收集資料一點也不Sexy,無法反復發論文,所以自然無人問津。很不幸的,獨立自主的資料庫,正是國際政治經濟話語權的基礎;任何基於西方資料庫的研究,必然會自動包含他們有意無意埋下的扭曲和偏見。我反復提倡中方重視這方面的投入,但中國學術界似乎同時兼有著崇外和論文至上兩個迷思,如果沒有最高層的强力糾正,不可能有所改變。
\subsection*{2023-01-09 06:00}

Google Translate連在簡體字翻為繁體字的應用上都偶爾出錯,翻譯俄文必然會把Nuances(尤其是官場語言,例如Putin的演講)搞丟;俄方訊息必須是能説俄文、又沒有撒謊動機(排除了歐美記者群)的人翻成英文才能信。
我的小孩剛和我抱怨,同學的作業稿都是由AI復閲更改過的,結果把文章的Polish程度提到很高,導致助教對他的微小文法失誤扣分,很吃虧。
我其實很不願意拉黑的,只要有一點教誨成長的可能,連“大一統理論”都容忍了好幾年,不是嗎?

稍微離題。我的小孩一年前開始自發學俄文,所以也試圖閲讀有關俄烏戰爭的第一手報告。他今天才剛剛跟我說,自從上周Makiivka的俄軍宿舍被襲之後,官方和部分媒體的俄文戰報忽然風格一變,成爲烏克蘭式的浮誇胡扯;我們簡單討論以後,同意這應該是Putin震怒之下,若干俄軍將領卸責自保的掙扎。
\subsection*{2023-01-07 05:14}

是的。不過不止是60年代的經典科幻低估了邏輯思維的重要性和難度,現代的好萊塢和日本影視作品,也多的是以感情來定義人性的溫情價值觀論述;事實上感情是本能(避免飢餓、疼痛)之外最廉價、最低級的,基本所有群居的哺乳類動物都有感性,所以龍應臺之流因爲具有等同老鼠級別的思維能力而自鳴得意,居然還有一衆粉絲接受她對世界實際事務的胡扯,非常不合理。隨著Deep Learning技術的持續發展,我們可以預期藝術家級別的雕塑、作曲和詩詞創作AI也會很快出現普及;這是一件好事,因爲文藝領域的本質先天就不屬於真正的學術(社科才是,所以我一直說大陸教育體系把社科和文藝歸爲一類,是個嚴重的錯誤),而是嗜好意興引發出消費需求,是一種技藝;有AI能將這些技藝自動化,自然會消弭非理性的崇拜。畢竟既往社會因文字取士,與選舉知名運動員和演員來當總統的心理並無二致。
\subsection*{2023-01-05 19:39}

我自己對嚴謹的邏輯思維模式習以爲常,覺得理所當然、不言而明,反而沒有解釋得像你這麽清楚;這個總結應該對很多新讀者有幫助。
不過我必須補充,大部分的社科議題並沒有足夠的事實根據來做絕對肯定的結論,所以博客更經常示範了幾千次的,其實是基於Bayesian Statistics的機率估算。這是更高階(不過整體來看依舊是基本技能,和組織大規模邏輯結構然後形成完整的認知框架,顯然還有相當的距離)但相對模糊的分析方法,你既然已經掌握最基本的邏輯運作,可以在復習博客的過程中留意觀察相關的研究案例。
雖然Deep Learning AI的應用已經到了井噴的階段,但其本質仍然局限在類比聯想、然後歸納,只不過是比人腦高出許多個數量級的廣汎類比聯想罷了。要真正達到Sentient的地步,所差的正是上面兩個段落所討論的1)嚴謹的邏輯分析;和2)在無法達成100 \% 結論的時候,做出持平而且精確的Bayesian估算。目前的技術只能替代工匠(例如畫匠和文字匠),過濾論述是做不到的(所以ChatGPT其實很危險,它只能接受並復述主流媒體的宣傳文稿,然後做簡單漂亮的包裝總結,結果必然是更方便昂撒體系對一般讀者做洗腦;不過話説回來,這不正也只是許多中國社科“學者”一輩子所能幹的嗎?),博客所反復示範的從第一原則出發做深層思考,然後進一步建立正確的認知架構,則更加遠超其所能。
\subsection*{2021-11-16 11:37}

你是打算用我的學習模式來創建一個AI系統嗎?那你必然知道這叫做Expert System,是最老的AI嘗試之一,並沒有什麽成果(因爲很難Scale?),不過或許他們的架構選擇不得法、硬件速度也不到位。你是否設想要和現代的深度學習結合起來做新的嘗試?
你的理解沒有錯,我的閲讀不只是完全不管作者的觀點,事實上是要用心剝離他們的主觀意見(誠實科學家的專業評論例外,但這樣的人很少,而且一樣必須小心過濾);這對AI可能是個大難題。學習過程中固然也會注意有趣的新思路,但世界上可能存在的思路方法數目原本就遠低於事實細節,經過幾十年的纍積,更加難以遇到前所未見的新邏輯,所以絕大部分時間,我看文章(即使是專業論文)都是儘快跳過邏輯論述,直接去找新觀察現象/實驗結果/内幕消息,所以才能每天掃描一兩百篇嚴肅的話題;如果遇到事實細節豐富、自己不可能全部記住的,再花一分鐘把它分類存入電腦的硬碟,供日後參考。當然這些“事實”,必須經過過濾,可疑的要另找佐證,如果找不到,就得加貼“存疑”的標簽;不過這類Bayesian算法,反而是AI領域日常的技能,你大概不需要我的囉嗦。
\subsection*{2015-10-25 00:00}
中时博客的留言已经这么难用,你能想像发正文的麻烦吗?不过软体这种东西,用惯了就好,现在我有自己的一套习惯,先在Word里写好,再一次拷贝过去,也就懒得再改了。你若是要留言,最好也这样做,否则网页经常会重启,那么写过的东西就会消失。

现代大众传媒越来越偏颇,《观察者》已经算是最好的了,否则大家也不会都常去。不过前几个月,《观察者》先是写了一篇吹捧金立群女儿的文章,把一个靠关系进哈佛读书、再找到伦敦教职的庸才夸张成中国第一天才,后来又特别刊出金刻羽的经济评论,基本上是美国教科书的节录版,让我看得直摇头。这种拉关系人为造势来捧学术明星,是美国经济学的常态,Larry Summer就是如此成名的(他是Paul Samuelson的侄子;后来克林顿对金融放松管制,他居功甚伟,所以2008年的金融危机也可以算是他的杰作);中国既然自称是共產党,《观察者》又自诩是左翼,实在不该去搞权钱勾结的背后交易。\section*{【學術管理】從假大空談新時代的學術管理}
\subsection*{2023-02-04 15:29}

幾年前我剛開始公開批評中醫教,就曾指明因内心深度自卑而盲目尋找心理補償,是中醫和功夫(“國術”)在清末民初聲名大噪的時代背景。現在功夫在擂臺實證之下,已經泡沫化了,但醫療效果卻沒有簡單明確的驗證手段,必須對心理作用(尤其是Placebo Effect)和統計原理有正確而深刻的理解,才能分辨好壞,所以過去幾十年,不但沒有因經濟工業化而被淘汰,反而在商業吹噓之下持續壯大。這和量子計算背後的“科研教”(參見《Foundation》;請注意,科研和醫療一樣,也有著很高的專業信息障壁),其實都類似“高等金融”“High Finance”,同是自由市場所帶來的虛擬化、空洞化效應之一,除了堅持並普及科學原理和求真態度之外,並沒有很好的解決辦法。
至於我對華語輿論界中,三觀相對正確的一方有影響,也是幾年前就開始明顯化;現在正方的聲量或許還沒有壓倒性的優勢,但論證水平早已遠高於反方,是我喜聞樂見的成果。其實如果只有智商低於100的一半人口還迷信昂撒宣傳洗腦,那就已經很接近和平繁榮時期人力可及的理想境界了;要再進一步端正視聽,沒有經過戰爭苦難,的確不是容易的事。當然,博客的知名度上升,也有負面的效應,亦即新來的讀者很可能以爲那些販賣二手或三手分析的網紅才是創新者,那麽讀者群的增長自然也隨之放緩。
\subsection*{2022-09-28 11:23}

是的。所謂的“Big Bang”“大爆炸”,指的是倒推宇宙歷史可以基本確定它曾經是密度和溫度都極高的等離子體,但實際上即使接受理論(主要是廣義相對論和標準模型)的間接證據,其已知可靠應用範圍也頂多只能容許推論至還不到10\^-30秒;這時的密度和溫度都還是人類科學可及的有限值。然而正因爲至此還是40多年前的理論已經講明白的事,對發論文沒有幫助,所以自然整個行業就形成默契,開始憑空捏造新幻想,其中最有名、最成功的就是“Inflation”“暴漲論”。原本假想新理論是科學研究的正常步驟,但這裏做暴漲論的人越過了兩條紅綫,以致從科學研究轉化為集體詐欺:1)在暴漲論的邏輯被證明不自洽之後,依舊不肯放棄,百般尋找藉口繼續發新的論文。2)不正面囘對專業質疑,轉過頭去重寫科普(這是爲什麽學閥要養袁嵐峰這類倀鬼的原因;而倀鬼的存在又反過來標識了詐騙的意圖),欺騙大衆;這裏我覺得最可惡的,在於重新定義Big Bang,無限倒推到時間零點,人爲地創造出Singularity,以便混肴視聽,為自己的玄學理論背書。
科學是根據事實和邏輯來理解這個世界的努力,超越已知事實和邏輯的,就只是玄學。Big Bang很簡單就可以依據科學原則、正確定義為一個真正的科學理論,這些人爲了自己的私利,背叛科學來散播錯誤扭曲的認知,是科學界的罪人,也是全人類的罪人。
\subsection*{2022-02-02 23:08}

其實不止周院士,上周還有另一篇類似的文章:“再这么玩下去,我国的科研就真没戏了”(可惜是某所長匿名發表,參見https://mp.weixin.qq.com/s/492jiGF5nj14w2A0iVvAgQ),我覺得更加直擊要害(可能是匿名的好處)。
學術改革這件事,牽連極廣,敵對勢力盤根錯節,尤其掌控了官方和非官方的話語權(對比昂撒宣傳體系),一般民衆和官員自然崇拜“權威”、相信“主流”,有素養和節操的科學人反而因爲只看清自己專業内的問題,而不敢貿然批評其他案例,難以聚合力量,更無法深入探討中國學術界全面腐敗的根本原因。
我公開反復談這個議題所希望的結果,如同過去八年對抗英美假新聞那樣,是作爲一個引子,一方面提醒所有愛國人士注意學術腐敗的重大危害,另一方面也讓良心從業者確定這些問題不但真實存在,而且極爲廣汎深刻,從而鼓勵他們發聲。我以前提過,在公益和私利的鬥爭中,好人雖然是少數,但不需要利益共通就能自然結成同盟,而且先天站在道義的一方,只要再加上一些智慧和勇氣,智仁勇兼具,挑戰邪惡勢力並非蚍蜉撼樹、螂臂當車;昂撒宣傳網雖然鋪天蓋地,一旦足夠的良心人在網絡上溝通起來,不也是(至少在理性知識份子心中)被擊破了嗎?
\subsection*{2021-12-29 13:47}

你應該理解JWST的科學意義,但一般讀者可能不熟悉,所以這裏我先簡單解釋一下:因爲宇宙自138億年前Big Bang之後一直在不斷膨脹,最早的星光(大約是大爆炸後幾億年)隨著空間本身的延展,波長等比增長,當年的可見光至今已經成爲中紅外綫(Mid-infraRed),這不但無法穿透地球的大氣層,只能由太空望遠鏡來觀測,而且探測儀器必須冷卻到6K以下,不是Hubble的繞地軌道和普通傳感器能滿足的。爲了觀察宇宙第一代的銀河(Galaxy)星系,JWST是唯一的辦法。
和大對撞機相比,JWST有幾點不同:
1)JWST的目標明確,只要工作正常,100 \% 會得到極重要的新科學觀測成果,而高能物理的新粒子,可能要提升能級1000000000000000倍才會出現,大對撞機只增能7倍,空手而返的機率很大。
2)JWST的原預算是5億美元,後來超支20倍,才成爲100億;現在歐洲計劃的新對撞機預算就已經是200億歐元(高能所的睜眼瞎話,不值一哂),將來超支數倍也是必然的,而且倍數無法保證比JWST低。
3)JWST在光學、材料和航天上所開發的新技術,遠比對撞機要容易轉化到實體工業用途。
4)如同高能物理在1970年代確立標準模型,天文物理界在1990年代也建立了一個標準模型,叫做Lambda-CDM(Lamda指Cosmological Constant,亦即暗能量,CDM則是Cold Dark Matter的縮寫,指速度遠低於光速的暗物質)。這裏的差別在於高能物理界在過去近50年做了無數極爲昂貴的實驗,依舊沒有超越標準模型的明證;而天文物理的Hubble Constant Problem(Hubble Constant就是宇宙當前的膨脹速率,有兩類測量辦法:可以直接觀測,也可以從宇宙幼年期遺留下來的跡象做間接推斷,後者當然必須依賴理論,亦即Lambda-CDM;然而這兩種方法卻產生互相矛盾的結果)最近超過了五個標準差的確定性,證明Lambda-CDM有嚴重而且立即的問題。
總之,JWST不但有强大的理論支持和緊迫的科學需要,而且價格比對撞機低一個半數量級,成功機率更是高出好幾個數量級,還有額外的溢出效應,所以兩者是不能相提並論的。當然,100億美元依舊是一筆大錢,在中國當前基礎科研預算極爲有限的前提下,投入JWST這類的Big Science Projects,必然會影響數以萬計的其他研發方向,所以即便它有重要的科學意義,也很可能不是效費比最高的投資方案,必須要先做審慎、理性、公平的評估,切忌只讀了科幻小説、熱血沸騰而自願被利益相關者游説出片面的決定。
\subsection*{2021-11-14 04:54}

這裏是美國行内人的評論:https://www.youtube.com/watch?v=-j1-5P-0qG4
因爲量子計算是整個行業一起自欺欺人,你不用管他們的主觀態度,直接找客觀事實就行了。這裏最重要的客觀事實,是和兩年多前的Google Sycamore(參見前文《Google的量子霸權是怎麽回事?》)相比,主要的改進在於從53個量子位元提升到56個,增加了5.6 \% 。你用對數去算一下,可以簡單得到以這個改進速度,每提升一個數量級,要花大約84年;8個數量級就是672年。別忘了,Moores Law不但改進速度是對數綫性的,投資也成指數增加,例如最新的5nm Fab耗費200億美元一座。換句話說,中國政府只要不斷依指數增加投入(100年後就可能達到每年一億億元),在2690年代前後應該可以破解2020年代的古典密碼。
至於所謂“特定領域”有相對優勢,我以前已經反復解釋過了,是先射箭再畫靶的騙人花樣:這些特定領域,從來都沒有什麽用處,被Scott Aaronson特別挑選出來寫論文,唯一的考慮只在於它們是量子計算機自然運作的結果。什麽幾億億倍的速度,純粹是56單元多體系統的先天複雜性所造成的。即使是古典力學的三體問題都沒有簡單的解答,那麽56體的量子作用,模擬起來當然很花時間。我家這隻狗,自帶10\^30個電子,它只要一秒左右就能完成排氣過程,這要比用電腦來模擬快上10\^N倍,N可能大於一個Google。依照Aaronson的邏輯,這隻小狗是超超超級量子計算機,在仿生模擬的科目上,有著近乎無限的量子霸權。
正文裏說量子計算頂多循半導體的前例,每18個月密度加倍;依這個速率,兩年下來位元數要成長為2.5倍,亦即130多個。所以中國量子計算的新“突破”,其實反而代表著其進展速率遠低於半導體,也就讓前景更加大幅黯淡了。他們越加誇下海口,說要在5年内解決量子糾錯的難關,應該類似全球核聚變團隊都在競相加碼吹牛(最近有一家叫做Helion Energy的公司,剛剛承諾要在2024年實用化,並且藉此騙到了五億美元),是因爲情勢緊迫,必須在謊言被拆穿之前盡可能大撈一筆。我的這篇《從假大空談新時代的學術管理》,據説被官員們廣爲傳閲,那麽等離子所和潘委員之所以開始着急,我可能有所貢獻。至於《觀察者網》和我保持距離,是我事先就預期的説實話的代價,只要有人聼,都是值得的。
\subsection*{2021-10-08 08:24}

這是一個很大的難題,只能從本土的學術文化和大局需要出發,做不間斷的改革試驗。但是以下幾個前提是可以確定的:(1)國家必須保留進一步改革的彈性和自由度,所以不應該預先賦予過多的特權和承諾;(2)要容許學者潛心研究冷門課題,並沒有事先給出高級身份和待遇的必要(參見Perelman和張益唐的前例),終身保證更加難以自圓其説;(3)必須把真實的應用價值和論文被熱烈追捧分別開來(參見博客過去幾年的討論)處理。
這裏是幾點我現在想到的具體建議:(a)把基礎科研(理學院?)和應用學科(工學院?)分別開來,前者要做徹底的改革(討論於下文),則後者可以根據研究成果在實體經濟的應用成效,由國家統一審查賦予終身正教授職位,地位在普通正教授之上,此外也可以考慮以接外來的研究任務(尤其是國家指定的),取代發學術論文的責任;(b)任何從事基礎科研的組織都應該盡可能扁平化、平等化,避免待遇(指個人薪資和Benefits,研發資金、儀器和實驗室則視個別研究計劃決定)的厚此薄彼,或者内部管理權力的過度集中;(c)理學院應該模仿美軍的Non-commissioned Officer Track,專門設立一個不要求發論文的升遷管道(講師、常駐講師、資深講師、特級講師?),給予較低但仍然足以糊口的待遇,並且從第二級開始,賦予中長期合約,這些講師可以承擔本科教學和日常行政的大部分責任(但不能被視爲天生應該996的變相奴隸;換句話説,仍舊是一學期教一門課,只不過是學生人數多、工作量高的本科課程),教授(要求發論文的升遷管道,但頻率標準不宜過高)的員額因而對應地削減,只負責帶碩士、博士生做研究或講授高級課程;講師和教授兩個管道之間,可以依據最近的表現或偏好做調動(亦即不想再定期發表論文的教授,可以轉資深或特級講師;有重要研究成果的講師也可以轉為教授)。
至於完全剝離中科院院士的政治權力,更是我反復强調的必要改革,不再贅言。系主任也應該架空,由特級講師每一年或兩年輪替一次是個可以考慮的方案。
\subsection*{2021-10-07 22:12}

你聽説過一本叫做《The Mythical Man-Month》的書嗎?它談的是軟件開發的管理,其結論是任何一個軟件開發計劃所用的工作人數和時數(亦即“Man-Month”)在很低的時候,效率可以隨其做近乎綫性的提升,但一定有一個臨界值,一旦團隊規模和投入接近這個臨界值,加人加班的效益就會很快遞減,超過這個臨界值之後,邊界效益不但會趨於零,而且有時會成爲負值。一個最近的例子,是因爲《Witcher 3》而名聲大噪的CD Projekt Red,以爲有了經驗和資金,就可以靠著多雇用程序員來加速完成下一個產品《Cyberpunk 2077》,結果成爲超級災難,不但品質慘不忍睹,連公司本身都面臨分崩離析的危險。
上面所談的臨界值,是可以通過合理高效的管理原則和運作規範來推高的。但是這些管理原則中的第一條,就是不能急功近利、强人所難;第二條則是必須事先規劃好正確的發展路綫、明確訂下分工的職責和進度里程碑,不能做出非最優的選擇,然後再試圖修改彌補、甚至被迫推翻重來(連軟件開發都如此,在十四五規劃中點錯科技樹,可能是沒有嚴重後果的嗎?)。歷史上所有失敗的軟件開發計劃,基本都是栽在這兩個錯誤之上。
前天我想要解釋的,是學術創新比軟件開發還要極端得多,畢竟後者沒有思想或原則上的不確定性,純粹是工程方面的執行問題,至少在理論上沒有理由不能事先做出完整、高效、明確的規劃。科技研發當然也有這類的項目,尤其是國外已經實現的成果,例如半導體和發動機,或者雖然是全新的工業,但基本技術難關已經解決,到達應用井噴的階段,例如電動車和AI深度學習,那麽完全可以投入大量的人力物力財力,然後挑選有眼光、有能力的主管來領導大型團隊,進行全力攻關。但是如果任務是要在思想、理論或基本技術上,做出徹底超越現有層級的突破創新,那麽不但人力臨界值必然很低(參見發明相對論或者藍光LED的團隊規模),而且連研究方向都是大海撈針,專注在其中之一反而是成功機率最低的選項。一般人不理解這個道理,只看行業的大標題,以爲都是一回事,其實同一個題材之下,完全可以有兩個不同的目標,對應著剛好相反的最優解。例如我多次討論過的深度學習,先天依賴既有數據做非綫性優化,所以必然帶著很嚴重的局限性,雖然已經可以有很多實際應用,但若要衝破現有的局限,必須開創出下一代的AI原理。做應用的,和想突破的,其工作的性質就完全不同,所需的高層投資和管理原則也剛好相反。
其實在所有的科研管理上都是同一個原理:應用性高的學科,可以搞全力攻關,其原因是事先就能確定最優的極少數幾個可能路綫,對其做出全面的覆蓋,例如我在上月所談的,科技部在過去十年引導動力電池產業的過程中,可以簡單兼顧磷酸鐵鋰和三元電池。但若是那個前提不滿足,就必須反過來做最廣汎的嘗試,基礎科研先天就必然屬於這一類。然而驕縱學閥、放任造假、和急功近利的論文至上主義,剛好就是把整個行業都逼進熱門但卻無用方向的手段,所以我說增加1000倍的資源,不但毫無功效(因爲已經超過臨界值),而且會產生額外競爭壓力,反過來讓造假誇大和追逐熱門的現象更爲普遍,反而使效率進一步降低十倍。
\subsection*{2021-10-05 23:10}

你聽説過Terence Tao陶哲軒嗎?他很可能是人類有史以來智商最高的數學家,很小就被發掘,本人也十分努力,又能夠動員21世紀的科技(尤其是計算機和互聯網),但是除了把現代所有相關的獎項拿了個遍,他和17、18、19世紀的大師相比,對人類社會的實際貢獻卻微不足道。
同樣的,Edward Witten很可能是有史以來智商最高的物理學人,把Weinberg都感動到願意追隨他的程度,但是他玩超弦30多年的結果,不但沒有搞出任何進步,反而是把高能物理帶下懸崖。
這個現象的基本原因,的確是這些老學科可行的進展已經非常接近完全耗盡的地步。而且數學和高能物理只是先行者,其他的物理分支也開始有類似的跡象。不過真正讓問題惡化的,在於人類經濟過去200年以指數發展,然後教育的普及,帶來前所未有的人力物力資源,可以反饋到基礎科研上。再加上國際間的競爭壓力,導致先進工業國家的決策者不管科目前景,一味强加投入。偏偏開創性的研究,需要的是思想上的突破,人海戰術不但毫無效果,反而因爲Group Think人云亦云,而有反作用。這裏我給個簡單的估算:和100年前相比,尖端數理難了100倍,但是人員、物資和技術的投入增加了1000倍,那麽進展速度應該增加十倍嗎?不是的,實際上速度減爲1/1000,這是因爲資源提升帶來割喉式的競爭,研究人員即使能抗拒造假和誇大,對追逐熱門題材也不可能免俗,人人都必須選擇容易發論文、而不是真有開創性的研究方向,這樣的逆淘汰反而使發現新概念的機率降低十倍。
中國的基礎科研管理尤其糟糕,理念剛好與最優解反其道而行:不但嬌寵學閥、放縱造假誇大,而且迷信人海戰術,對論文重量不重質。我一再譏笑他們是以生產粗鋼的心態來管理學術創新,就是基於以上的考慮。這裏的解決方案是打擊學閥、嚴懲造假誇大,並且只對有明確應用前景(亦即經過完整、徹底、客觀論證,沒有基本理論或技術問題)的學科做專注、大筆、重點的投入。凡是必須等待理論或技術上基本突破的科目,不論是衰老垂死、或者還在妊娠期間的研究方向,就應該反過來以低預期、低資金但持續廣汎而且Hands-off(亦即不要求定期發論文)的支持,畢竟你希望培養的,是張益唐這樣的人物,而不是現在中國科學院那些假裝是學者的政客。
\subsection*{2021-10-03 12:11}

不只是騙子人多、聲量大而已,而是中國科學界本身和政府科技管理人員都一味崇洋媚外,隨便哪個有頭銜的高鼻子伸手往懸崖一指,就閉著眼睛跳下去。明明歐美資本主義體制的基本原則,就是私利至上、Greed is good,行業和公司這些利益集團的公開論調都是由狹隘的私利來決定,所以只要先看看説話的人有沒有利益牽扯,就可以立刻排除99 \% 的噪音和忽悠。不過在歐美,光是找到真相沒有用,因爲利益集團也控制了絕大多數的傳媒和政客,像是那篇歐洲議會的報告,並沒有成功阻止歐洲把大量人才金錢浪費到ITER這個坑(原預算50億歐元,現在進度不到1/3,已經上漲400 \% 到200億,等到20多年後達到全狀態運行,需要一個奇跡才可能讓總花費不進一步成長500 \% 超過1000億)。反倒是美國,因爲資本主義路綫走得更極端,學術行業相對於資本集團處於絕對劣勢,結果在核聚變上只忽悠到一個NIF。這個計劃的直接預算只花了42億美元(2001年GAO的估計,然後沒了下文,只剩下NIF自己發表、明顯不靠譜的公關數字,不過已經至少相當於超支400 \% ;這個比率似乎是大科學計劃超支比例的底綫),而且雖然搞聚變發電先天就不可能成功(我在1997年計劃剛開始就預見了,只是當時還沒有公開寫博客;奇怪的是,核聚變這個忽悠不論被證僞多少次,願意被騙的傻瓜永遠繼續佔絕對多數,或許是爲他們的愚蠢買單的是公帑,不必心疼),但是對模擬氫彈有點用處,所以政治上獲得了軍工集團的支持。
照理説,中國政府的權力凌駕於所有利益集團之上(這正是中式政治體制的優越性所在),而且在政治、經濟、外交這些社科問題上早已知道不能盡信國外的理論和宣傳,結果在自然科學這種可以依客觀邏輯來嚴格論證的議題,反而是照單全收,根本不試圖找利益不相關的專家來做獨立分析。例如中國核子工程的專家這麽多,隨便找一個人,都能簡單指出聚變發電的不切實際(不是Hossenfelder所談的等離子體物理角度,而是工程方面的諸多問題),偏偏就是只聽等離子體所的一面之詞,這樣層次的科技管理,連中學生都不可能罩得住,真是讓人無語。
\subsection*{2021-09-26 08:58}

科學理性的思考方式,不但是一個違反人性、逆天而行的技巧,而且必須與深刻卻又廣汎的專業知識配合,才能有效達成正確的結論。一般人哪可能有空天天在找有趣的專業問題,然後研讀論文直到有足夠的心得爲止?至於幾十年不輟的邏輯思辯訓煉,以及對事實真相的絕對尊重和堅持,那更加不是熱門的習慣和心態。純為鍛煉思維能力而到老還在天天解數學題的前人,我只聽説過林肯曾經反復研讀《幾何原本》數十年。
不過知識份子至少應該有“知之爲知之,不知爲不知”的修養,在沒有檢驗過完整嚴謹的邏輯論證之前,對無根的宣稱(Claim)保持健康的質疑,不讓主觀意願影響對真假的判斷。這是我對博客讀者的期許,換句話説,就是你最起碼必須進步到Dunning-Kruger曲綫的中段。很不幸的,現代新聞評論和互聯網文化完全是反正道而行,把99 \% 以上的群衆牢牢鎖在笨蛋峰的頂端;這一點我也反復寫文章詳細討論過了。
至於你所提的永動機,剛好切中這個技術議題的另一個要害:這是因爲熱力學第二定律,除了如同我在昨天解釋過的,通過濃縮CO2的困難,預先否決了從CO2合成有機分子來解決全球暖化問題的任何嘗試,即使是直接使用引擎廢氣,在CO2還沒有被稀釋之前就試圖回收轉化,那麽這個引擎+回收的總系統,也會自然等同一個永動機。正因爲這種先釋放再回收的構想,是如此地不合理,所以才會有8個數量級那種級別的價格差距。
那些研究人員不懂熱力學嗎?我想不是,他們完全瞭解這條技術路綫的實用前景是零,只不過有少數人想要直接向投資人騙錢,另外的多數胃口小一些,先騙一個論文,然後再騙一點公款。説實在的,這類實驗除非是建工業級的廠房(例如本月在冰島投產的那個),否則光是做學術研究,花費還真不太多;若不是有人問起,我不覺得有必要主動寫文批評。
至於他們論文能登在《Science》,學術界的行内人當然知道其中奧妙,但我為其他讀者解釋一下:基礎科研的期刊,對應用前景是一點也不在乎的,新奇有趣反而可以是唯一的考慮。不過我昨天也說過,這種容易又新奇的題材極度難找;中國的院士們,一般是走另一條路綫,亦即拿巨額公費來做昂貴但科學意義不大的實驗,因爲科學意義不大,所以國外實驗室一直要不到錢來做,但因爲昂貴,所以做出來頂級期刊依舊另眼相看;悟空衛星和墨子號量子通信衛星都屬於這一類。他們能在頂級期刊發表論文,並沒有作弊,但一旦搞公關來吹噓無中生有的所謂實用前景,那就是純粹詐騙,值得正人君子鳴鼓而攻之。
\subsection*{2021-09-24 22:16}

這項技術是否有用,取決於產品的市場價格:如果人類對每克十萬美元的澱粉有足夠需求的話,可以算是極爲重要的技術突破。既然澱粉是人類生存必需的養分之一,要引發這樣的需求,也很簡單,中科院只要先把地球上所有能進行光合作用的植物和微生物全部消滅就行了;我們對中科院的後續配套突破(無藥可救的植物病毒?)翹首以待。
這種異常容易,而歐美先進國家在發明現代化工之後百多年卻都始終沒有認真嘗試的技術方向,其實極度難找。畢竟稍稍投入,就可以在全國大肆宣揚,提升中科院的政治地位和公款收入,這樣的投資報酬率就算不是史無前例,至少也是獨步天下。這個層次的智慧,我自己絕對望塵莫及;而中國科研學術管理之鶴立鷄群,也又多了一個例證。

好吧,我想你不像是學理工的,所以中科院這個假突破有多少漏洞,你大概無法自行體會。這些假未來科技越來越多、越來越離譜,偏偏普羅大衆太笨,明明什麽都不懂也敢胡亂站隊,污染公共論壇(博客的讀者當然不在其内,否則早就被踢出門外);説實話原本就吃力不討好,遇到中國網絡上那些拿科幻小説當成教科書的無知愚蠢群衆,更加是自找苦吃,難怪現在就剩我一個人堅持。其實我很累了,所以早先只說了幾句反諷。
這裏我爲你解釋清楚一次:植物光合作用本身的轉換效率的確是很低的(略高於1 \% ),所以過去十幾年大家開始關心碳排放之後,就有很多歐美的實驗室搞出直接用陽光來把CO2轉換成甲烷;中科院這次宣稱稱比光合作用效率高幾倍,對稍有這方面知識的人都知道是不足爲奇,因爲早就有號稱效率高出20倍的過程,而且用的是免費又清潔的陽光。中科院可用的是傳統能源,所以應該把發電時的碳排放加入考慮,這麽一來,它反而是做得越多,CO2排放越多。當然中科院的最終成果是澱粉,而不是甲烷,但是有機分子之間的轉換是老技術了,這部分完全不是重點。
即使我們假設下下下下下下一代的技術可以改成用陽光,而且達到20 \% 的轉換效率,農業是不是即將被取代了呢?你記得我討論NIF做公關,宣稱激光聚變的產耗比超過1,但那是對分子分母都動了手脚,實際上還差8個數量級嗎?這裏是完全一樣的道理。這些學術騙子第一步,就是先忽略熱力學第二定律,只談能量、不談熵;別忘了,植物是直接吸收一般大氣,其中只有0.04 \% 是CO2,所以它效率低,其實包含了將CO2濃縮所必須剋服的熵差異;而這些騙資金的論文都假裝把大氣中的CO2濃縮2500倍是無需代價的。事實上光是CO2濃縮集中這一步,人工技術就比光合作用貴了百倍左右。
作物(例如玉米或蘋果)的光合作用不只是自帶CO2濃縮機,它也自動提純生產出的澱粉,再自動包裝成顆粒,而且整個過程對環境的净污染可以做到接近零(因爲碳固定),所需的資金投入很低(考慮農地對比高科技工廠的單位價格,差異至少是6個數量級),成果更自動是食品級的純度,還外加一些調味的分子,增加產品的價值,至於規模化生產,幾千萬噸都不成問題。生化實驗室剛好相反,所需的資金極高,濃縮、處理、包裝全都要另外建生產綫,而且化學污染的危險很大(你願意幾公斤、幾公斤地吃實驗室從無機分子合成出來的有機物嗎?)。當然只要願意扔錢,這些問題都可能解決,但是每一步都把費用以倍數提高;我的估計是即使批量生產,也會比玉米澱粉貴至少8個數量級,所以每克大約十萬美元。中科院如果不同意,我非常歡迎他們把價格估算寫在論文裏,讓大家進一步討論;但是實際上他們對這些關鍵議題都是故意回避的,指望他們談經濟性,和接到詐騙電話時問騙徒的身份證號一樣,是不可能得到誠實的回答的。
\subsection*{2021-06-26 15:10}
我説過了,“爽”的成分純粹是糖衣包裝,兼有娛樂和吸引效應,本身並非負面。科幻作品中充斥假科技、真幻想,也是天經地義,原本就在於探討人類群體對新奇環境背景的反應,背景條件是否切實際不是問題,反應是否合理、與現實社會的對照是否有意義,才是重點所在。甚至原創性都不是決定性因素:我近年遇到最好的科幻作品,是《The Expanse》,其中關鍵的科幻前提假設是人類找到早已滅亡的外星人所遺留下來的超光速星際旅行系統。對科幻稍有涉獵的人,都應該立刻注意到這個點子正是1978年雨果獎得獎作品《Gateway》的核心假設,後來已經有無數小説和游戲模仿過(例如《Mass Effect》)。但是這幾個系列所描述的人類社會各各不同,對“新發現”的反應也各異,所以各自有其價值。這和劉慈欣照抄Niven的點子,純粹是爲了唬沒見識的讀者,根本沒有用來探討社會議題,完全是兩回事。人類社會的發展,始終受到許多客觀條件的限制,其中有自然法則、天然資源、也有技術能力,而且越到後來,後者的重要性就越高。科幻在超越現實的新奇前提假設下,才能用以往未有的角度來觀察人類社會的作用機制,其實可以比只談人性的傳統小説更爲宏偉、深刻。劉慈欣反其道而行,爲了“以弱勝强”的爽,必須在故事後段硬是掏出“新科技”來推翻既有的邏輯限制,如此幼稚低級的作品,有腐蝕人心、危害家國的效應,不是很自然的嗎?至於要培養年輕人對科學的興趣,的確正是科幻的實用貢獻之一,但是必須是好科幻,不把幻想成分硬拗成真實的。換句話說,一般讀者成長超過初中程度之後,就應該普遍明白科幻作品中的“科技”不能當真,否則反而有大害,而始作俑者自然成爲理性知識分子應該全力打伐的對象。\subsection*{2021-06-07 11:03}

Abductive Reasoning是在不完整信息條件下,依照科學原理來追尋最佳解釋。除了這個博客之外,華語界還有哪一個公共論壇能做得到?這些包括種種名教授的發言群體連Abduction的基本邏輯思維工具(如Occam's Razor、Russell's Teapot和Bayesian Probability,更別提邏輯學的各種佯謬和統計學的諸般陷阱)都全無所知,根本連從哪裏著手都不懂,你指望高中老師來傳授這個能力?小學和中學教育的主要任務是教導公民基本的文、史、數、理等等常識,並賦予正確的三觀,不是培訓全世界最頂尖、極其稀有的思想家。
中國院士級別的研究人員,在被Publish Or Perish的壓力逼急了之後,尚且會想要作弊,你覺得美國的中小學學生能創新出什麽大道理來寫幾百篇報告?最終訓練的,還不是粉飾自己無知的諸般技巧,尤其是含糊其詞、以聯想來替代因果的壞習慣?美國的記者、公關、律師和MBA個個巧言令色、擅長内捲,所謂Kiss Up,Kick Down的專業技能不正是小時候不重事實或邏輯,而專注於表達能力所培養出來的?以Boeing爲例,你覺得中小學寫報告特別漂亮、脫穎而出的學生,後來成爲他們的氣動和材料工程師、還是MBA管理階層?
\subsection*{2021-06-06 13:19}

1.不合適;造假和誇大主要是行爲不端,除非有明確的詐騙行爲,例如漢芯,否則還是以行政處分爲宜。何況我們談的是很明確的不當行爲,並沒有什麽羅織陷罪的危險。
2.教育民衆,讓他們瞭解學術不端的代價嚴重,而且是由全民承受,的確是正確的方向,我個人也在努力。
3.職業運動員的價值,是純粹相對的:只有勝負,沒有好壞,能力、水準和貢獻全不相干;這是典型的零和游戲,導致尋租式生涯,類似的還有演藝界明星、律師和一般公司主管。一個健康的學術界,顯然必須有實際產出,是非零和經濟發展的基礎;如果行業大佬不願意走上正道,那麽政府公權力絕對有必要出手糾正。
4.這是很好的建議,希望中國政府早日認識這些問題和解決方案。
\subsection*{2021-06-06 08:44}

我先提醒大家,這個留言討論了兩個有關高等教育和學術論文的話題,但是都和前一個讀者的問題不相干。這兩個新話題是:1)中式教育忽視學術報告的寫作訓練,是否有偏頗?2)忽略寫作、專注於考試,是合理的入學標準嗎?
首先,我認同對這個中美差異的觀察:不但我個人剛到美國念書覺得最吃力的事,就是學著寫論文(其實不但在學生時期下了大工夫,畢業後又投入多年的努力才真正習慣寫作;這是我以前談過的經驗),而且後來我的小孩升小學四年級就忽然開始拿到成堆的作業,每周都要交好幾篇報告,結果他從10嵗起,就幾乎每天都要熬夜寫稿。
差異是實在的,但是美式做法真的優於中式嗎?我覺得寫作表達的技巧,對學術人、律師和商業管理階級非常重要,但對其他公民沒有什麽意義,對社會整體也沒有基本的貢獻;反過來看,一個專注於解題的教育體制,如果有心好好設計,可能會更方便培養邏輯思考能力,而邏輯思辨才是建立現代科學理性社會的剛需。
至於入學標準,當然可以針對博士班、法學院和商學院來特別要求寫作能力,但是一般的本科就沒有必要,而且高考還兼有維護階級流動性的關鍵價值, 絕對客觀的優劣標準是不能輕易放棄的。
\subsection*{2021-06-06 03:12}

這個議題,我還是研究生的時候,曾經和哈佛物理系的教授討論過。他們給的解釋,我覺得是美國學術界的普遍共識:亦即學士教育只是專業的入門介紹,碩士教育則是對專業内某分科的深入學習,要到博士班教育目標才是培養出獨立創新的研究能力,而且不一定每個拿到學位的人都成功得到那個能力。説到這裏,當時的系主任Howard Georgi還特別澄清,(即使是資深研究人員)每篇論文所含的創新觀點,數目絕對不能超過1,但是0是完全可以接受的。雖然他是在半開玩笑,但其背後的道理,類似於我反復解釋過的,“空”是尖端學術研究無可避免的可能結果,無可深責。
99.9999 \% 的本科生不可能獨立做出創新,强迫他們寫畢業論文,是典型的把學術研究當成粗鋼生產一樣來管理,那麽產出的論文價值類似同重量的粗鋼(大約不到一美元)是可以事先預期的。至於揠苗助長、逼良爲娼的效應更加應該是常識;台灣清華大學的管理階層這點常識還是有的,但是國務院的管理機構(教育部?科技部?)似乎對學術、創新和人性都毫無所知。
\subsection*{2021-02-14 18:20}

她是當前高能物理界罕有的,既誠實又有見識的人,比起Nima Arkani-Hamed、Sean Carroll或Brian Greene這些名教授一輩子沒有説過一句真話(亦即整個職業生涯幾百篇論文之中,被證實為科學的共有零篇;即使放鬆標準,把有可能在未來被證實的非僞科學也算上,依舊是零篇),她大部分言論是真實合理的,可以説是不知高出多少倍去了。
不過她受自身經驗和知識所局限(例如她對社會科學,尤其是經濟學,一無所知),一旦成爲網紅,開始談物理之外的話題,我就每每要爲她捏把冷汗。其實就只談物理,她有時也會説出很奇怪的邏輯,例如我寫了好幾篇博文討論、有關量子測量上Decoherence的詮釋,她的理解(“巨觀測量是對量子相位的平均”?!What?How?)就極端冷門離譜、似是而非。不過我想,對非高能物理專業的讀者來説,這些精微奧妙的議題並非重點所在。
\subsection*{2021-02-10 14:53}

你這個問題問得太大了,讓我猶豫了幾天,還是想不出能在合理篇幅之内做出完整解釋的辦法,所以只挑主要重點來談。
這裏最基本的毛病,來自經濟的成員是Homo sapiens,一種由演化自然產生、高度非理性的個體,所以經濟產出的價值先天就是極端主觀、隨機、易變的。以食物爲例,理性的要求是沒有毒性、營養豐富均衡,但市場上的食材珍貴與否,顯然和那兩個條件沒有什麽關聯。這還是第一產業,如果談到高度抽象、虛擬的第三產業,誰能用理性解釋爲什麽郭敬明是名導演?
所以雖然經濟產值的實際内含品質差距很大,最終還是只能用會計手段事後簡單計算成交的總金額,很難排除虛胖。當然這留給奸商很大的運作空間,但理性人士頂多也只能靠發表真相來聊盡心意;例如減肥和中醫,我都解釋清楚了,但即使是已經相對很高階的博客讀者群,也沒有完全接受,那麽自然更無法指望普羅大衆在乎並且瞭解真相。在國家層面上,既然GDP的品質差異很大,那麽當然不應該只拿總量來當作執政的主要指標;中國政府不是不明白這個道理,而是實在很難找到好的替代罷了。
\subsection*{2021-02-09 20:52}

很好的討論。不過這又帶回我想我們都同意的結論:整頓學術風氣是當前中國内政治理的重中之重。
產學勾結是資本主義體制的必然結果,在美國的演化發展過程,我反復詳細解釋過了。這種腐化的第一步,一定是創造一個容許、甚至鼓勵“假、大、虛”(請注意,不是假大空;空只是空洞、沒有内涵,虛卻是沒有内涵、還要強做錯誤結論)的學術界和媒體環境,所以我在這方面的發言,是試圖從根本上遏止腐敗的進程,建立對惡勢力有抵抗力的健全文化。如果主政者有無限智慧,能夠一眼看穿專業忽悠,當然可以省卻這些麻煩,但在現實裏,沒有人是全知又全能,那麽就只能靠有良心、有見識的人聯合起來,為國家和人類的前途,與自私的貪腐勢力做鬥爭。
\subsection*{2021-02-08 21:24}

我自己也覺得很奇怪:這許多失敗的案例,不但在金錢上是巨大的浪費,在時間上也拖了國家戰略需要的後腿,在實際發展上更打壓了真正有潛力的研發團隊,等於是花大錢請人來割自己的肉,而且是事前就可以輕鬆預期,實際上也有足夠的警告聲浪,但偏偏就是一而再、再而三的發生。全國的公款吃喝加起來,還不到這類損失的零頭,怎麽體制内就沒人想要提醒高層注意?
其實從經濟的基本理論原則上,就可以簡單看出,民資在乎的只是報酬。這種戰略性產業,所需資本極大、周期極長、風險極高,還要面對外國敵對政權的打壓,任何頭腦正常的資本家都立刻可以看出是最糟糕的投資項目。單是有補貼,並不改變其難度和需時,那麽民資的理性選擇自然是騙補。所以唯一可行的道路,是技術上集合既有團隊,但國家必須牢牢掌握財務和經營權;這可以是間接的,但必須確定管理人不能自肥、不能跑路。主管單位似乎是有思考盲區,對國企和半國企有反射式的排斥,但自由市場和私有企業其實只適合消費性產業,戰略性行業是沒有火藥的軍事戰場,其目的不是利潤、而是勝利;在這裏也迷信自由主義經濟學,難道解放軍也要轉為民營、自負盈虧?放任只在乎創造就業和GDP的地方政府搞,則是另一條保證失敗的路綫,就像把解放軍完全拆解為互不統屬的省、縣單位一樣。
至於事後不追責,更加匪夷所思。以弘芯爲例,民資拿了經營權,卻違約沒有注入資金,這麽明顯的竊盜國庫,怎麽能不追究?我是局外人,看的一頭霧水;有明白内幕的讀者,請解惑。
\subsection*{2021-02-08 12:42}

其實不能這麽簡單地完全否定一整個行業或學科,因爲他們在應用普及上必須遷就前綫底層人員以及基本無知的顧客,不可能把真實的考慮層層傳遞下去。例如我談過美國的減肥業,是一門每年720億美元的生意,典型的業務是把一天三頓飯送上門;這裏必然會有許多假的“健康食材”(例如石榴)被用上,以滿足一般顧客的心理預期。實際上有沒有效呢?有效,但效果來自戒糖、高蛋白、高纖維的組合,和常被吹噓的健康食材沒有關係,這個秘密,企業高層的專家是心知肚明的。換句話說,只因爲他們沒有説實話,並不代表完全沒有價值。
中醫的現狀比這還要差一級,就是它連實效都不敢去探討,如果真有幫助純屬誤打誤撞。然而如果能好好整頓這個行業,强制進行科學研究,把所有常見的處方都仔細做雙盲實驗,我相信像是青蒿素這樣成功的例子,還會再出現。但行業本身擔心的是人員的就業,而不是對真實療效的研究,所以指望他們主動改過自新是不可能的。
\subsection*{2021-02-08 02:21}

其實這件事,本質是急著成名的研究人員,利用近20年生醫界有89 \% 論文結果無可複製的背景,連續打擦邊球,事先就選擇很難重做實驗、又沒有立即影響的題材(所以不會有其他團隊急著複製),故意造假,等自己功成名就之後,再加緊建立封建山頭,以便抵抗質疑。換句話說,饒毅教授所面臨的舉報困難,並非巧合,因爲這些人並不是誤入歧途,而是從一開始就是詐騙集團。
我兒子雖然只在生物系念大一,但高中就在Columbia和Yale的實驗室打工過三年,和世界許多國家來的生醫研究生都有合作的經驗。他對中國學生的評語是,遇到同行就本能反應,自動視爲競爭對手,可以想見成長過程中學術環境所給的競爭壓力特別大。當我和他提起這次作假風波時,他完全可以想象他們在極高壓力之下會想要作弊抄捷徑獲勝的心理。
“允許試錯”,在這裏是絕對錯誤的評語,原因是生醫界的論文圖片性質,和陳平教授習慣的物理學界和經濟學界都不一樣。後兩者的論文插圖,都是對數據做統計分析後的呈現Presention,包含了不小的主觀成分(因爲統計本身就很容易按摩,參見前文《一個重要的統計悖論》),本來就只是作者人爲選定的表達工具,先天有錯誤的可能,所以“誤植”或“PS”都沒什麽大不了的。生醫界的插圖,也有這類的統計結果,但這次出問題的主要是另一類,亦即原始觀察照片,是人爲分析之前的基本事實,光是“誤植”就可一不可再,必須撤稿道歉,重新評審升等、升級;至於PS,那絕對匪夷所思,沒有任何藉口。
這次科技部的報告中,說對比了原始數據,沒有發現造假;這顯然是避重就輕,只討論是統計結果的造假圖片,完全不提觀察照片的問題。這裏的關鍵,在於這些造假的人,事先就小心挑選了很難重做實驗的議題,所以可以輕鬆地連數據也造假,那麽科技部只把數據和圖表做比較,自然不會找到任何問題;這也是饒毅教授在明知生醫論文原本就大部分無法複製的背景下,還一再要求重做實驗的基本原因。
陳平教授在經濟學上的造詣不淺,觀點很正,我一直視爲同道中人。但他近年開始做科普工作之後,在媒體壓力下必須對自己專業之外的時事做評論,就不可能維持以往那個高水準。一年前,他是背書新冠陰謀論的大V之一,現在又再次在生醫話題上略為走偏;不過我相信他不是有意和稀泥,而只是受到自己專業經驗的局限。事實上,我把他的誤判看作對自己的警惕,一方面如果我的專業知識不到位,就一定明説(例如兩周前,有行内人詳細解釋饒毅教授的指控,我就特別强調自己沒有能力獨立驗證其細節);另一方面,我對網紅身份極度排斥,深怕追求流量的壓力影響寫作的品質。你在閲讀陳教授文章時,如果遇到無關經濟學、尤其是生醫方面的討論,必須小心,但不要自動無限上綱、提升到質疑他人品的程度。
\subsection*{2021-02-02 22:52}

你是文科人吧?文科成品的好壞,先天就很主觀,學術論文的内涵和品質無法鑒別,在歐美也早爲人詬病,所以才會有一些閑得發慌的理工科教授用原始AI隨機纂綴文字來投稿,以羞辱那些期刊。不過封建壟斷的現象,達到你所描述的地步,的確是中國學術界特有的奇觀,是傳統文化留下來的包袱。反過來說,“壟斷”本身不是適合的批評著手點,因爲從這個角度,你必須解釋他們的集團如何組織運作、更優秀的人才如何被排擠壓抑;既然文科成品的品質評鑒,主觀成分很大,那基本是不可能的任務。
其實只要有政治意願,要解決這件事很容易,從期刊的編輯著手就可以了。中國歷史上如何保護考官的公正,隔絕他們受金錢權勢影響的經驗很豐富,可以簡單借鏡。不過文科對國家的前途只有微弱、間接的影響,不論是體制内、還是體制外的改革者,都自然會把精力、資源和政治能量優先投入到自然科學和社會科學上,所以方方的案例並沒有引發官方的反思和改革;我自己也不想深入討論文科的學術環境。
研究數學基本是要證明定理,有時一個特例很難證明,但是擴大適用範圍之後那個廣義的定理反而簡單。這裏也是一樣的:我作爲一個媒體人,要指出高能所一個特定學術單位的把戲很困難,因爲討論的細節必然是極爲專精,不論我再怎麽努力,能看懂的讀者少之又少,但是如果能舉出許多不同專業都有類似的弊病,反而可以從人性心理來解釋,方便業餘公衆理解。一旦成功,就不再是圈子外的民科挑戰專業學者,而是腐敗的封建山頭必須證明自己沒有參與惡行;換句話說,舉證責任Burden of Proof一下子換邊了,這對改革中國學術界的大業是極爲重大的推進。
所以在2016年,我誤打誤撞引發有關大對撞機的公共爭議之後,就很有耐心地從大戰略角度來佈局,準備把公衆的注意力擴大到整個中國科技和學術界的諸般腐敗現象。例如5樓的讀者,就看出博客早已暗中留置了不少伏筆。這是因爲我有華爾街的工作經驗,在公關戰術上比起那些學閥對手有量級的優勢(例如丘成桐,連“不予置評”這種理虧時的基本反應都做不到),但一般人仍舊迷信權威,所以我必須先纍積足夠的信譽度Credibility,並把諸般專業話題慢慢解釋清楚,然後待機而發。
這次《科工力量》邀請我做視頻,我終於決定把握機會全面出擊,運氣也不錯,剛好撞上饒毅教授的發聲。再下去,只要有足夠有良心的知識分子願意聲援、推動改革,不讓議題被遺忘消失,我覺得還是可以審慎樂觀的。這裏的重點是所有的好人都必須站出來、盡自己的一分心力,否則壞人自然勝利。這是我幾年前就引用過的Edmund Burke名言:“The only thing necessary for the triumph of evil, is for good men to do nothing.”你如果對自己行内的風氣有所不滿,我鼓勵你參考我的例子,在深思熟慮之後,聯合志同道合的人推動變革。
\subsection*{2021-02-02 18:11}

我認爲這不是高層的意見,而是内部的異見。如果普羅大衆中都有20 \% 是好人,爲什麽黨校的教授就不能有至少20 \% 是熱心公益的?
這裏借機會强調一下,好人的正確定義,是不願意損人利己,所以自然會在乎公益。壞人則是很願意大損人小利己,所以會想要侵占公益,典型的例如高能所那些人。至於損人不利己的,只能算是又瘋又蠢,例如英美的紅脖子,和台灣的臺獨。
華語(至少以往台灣的)電視劇裏的所謂“好人”,往往是傻白甜,這是英語裏所謂的Cardboard Cutout(紙板假人),連“人”都算不上,好壞根本無從談起。當然這是因爲行業環境惡劣,製片人不願在劇本上花錢,智商稀鬆平常的編劇自然無法真正描寫出比他自己更聰明的角色。其實現實社會裏的好人,不見得不比壞人精明,否則後者不受法理道德所規範,先天就有較大的行動自由,好人原本就是少數,如果又都是癡呆,世界哪還有希望呢?好人的優勢,在於即使從來未曾謀面,即使利益毫不相關,自然會互相親近支持,但這也得有足夠的道德勇氣出來站隊才算數。請大家參考前文《自斷後路的狷介清官》。
\subsection*{2021-01-30 06:29}

所謂的“自媒體”,指的是中方嗎?我對中宣部的内部管理邏輯,一直不是完全理解認同;不過這兩天,外交部明顯地對Biden政權伸出橄欖枝喊話攻勢,應該是劉鶴級別的決策。這並非錯誤,只是成功的機會和影響都不太大,美方頂多表面敷衍一下,讓關稅戰降級容易一些,但法律戰和宣傳戰不會有任何鬆懈。
至於情緒性消息渲染,這裏的問題不在媒體,而在於群衆:絕大多數的老百姓對事實和邏輯沒有興趣,他們上網是爲了尋個開心,而且是Quick Thrill立即快感。只要媒體自負盈虧、在乎收入,理性的經營方式自然是Race to the bottom爭相搞低級娛樂。世界對白老是叫我做什麽做什麽,以便提高流量,最後我不得不和他講明,流量提高才是應該警惕的事。我不收錢,除了避免網紅心態,也是爲了有教無類;但是這裏“類”如同夫子當年一樣,指的是社會階級,不是受教者的素質。剛好相反,那些無心學習、純粹來看熱鬧的,早走早好;他們的參與,除了製造噪音、妨礙正事之外,毫無意義。儒家哲學始終是愛民如子,而不是西方的Bread And Circus;前者照顧的是民衆的需要,後者討好的是民衆的欲望,照顧過小孩的應該都知道兩者有根本性的不同。
“溫哥華的魚”研究態度嚴謹,在同道中鶴立鷄群;希望他保持初心,不要因一點小利而過度擴張,尤其是對自己能力範圍以外的話題,不要爲了評論而評論。例如去年新冠成爲新聞熱點,結果沒有任何生醫知識的名人也被迫參與討論,為許多誤導群衆的陰謀論背書,至今我還在清理善後。
\subsection*{2021-01-29 20:02}

我自己的金融生涯就是這樣被毀掉的,對英美權貴階級肆無忌憚、目無法紀的真相,自然是有深刻的體會;我不是一直說英美的所謂法治,最終也是人治嗎?我想老讀者都知道我從來不會空口白話,這類論斷都是有深刻廣汎的證據才會説出來,但是這裏的真相太過離譜,所以我以前很不想談自己的經歷,談了也沒人相信;現在現成證據確鑿,才順便一提。
其實只要睜開眼睛,例子到處都是,只不過他們掌控媒體宣傳管道、又極端不要臉,所以幾百年撒謊洗腦、顛倒黑白,在沒有能力獨立檢驗事實與邏輯的一般群衆心中,種下了一個完全虛幻的假象。
所以讀完我博客,就是“take the red pill—you stay in Wonderland and I show you how deep the rabbit-hole goes.”英美的宣傳,正是現實版的Matrix,the world that has been pulled over your eyes to blind you from the truth.
\subsection*{2021-01-26 03:47}

AI的確不是萬靈丹,市場經濟和國際競爭也自然會吹捧過度,所以我對Google AI的宣傳也持保留態度。正文不也說要等一年,讓塵埃落定嗎?
AI其實是一個大筐筐,技術路綫上有很多彼此之間沒有交集的不同選擇。在2012年發生的突破,來自GPU的算力終於達到Neural Network(NN,人工神經網絡)所需要的基本程度,於是忽然開始有了實際應用結果。但NN有它的極限,它在象棋和圍棋這些每一步都只有少數幾個邏輯可能的問題上,效率很高,然而遇到開車這樣有真正近乎無限可能遭遇的環境,就沒有辦法完美處理每一個小機率事件。事實上,這也是所有當前大數據分析的共通弱點:在現實人生中,總有許多統計數據沒有涵蓋的真正黑天鵝,那麽任何基於對經驗來做非綫性歸納的研究工具先天就不可能做出合適的反應。你需要的是更高階的抽象思考能力,NN不屬於這類。
但是反過來考慮,像蛋白質摺叠這種十分重要、有用、又困難的問題,是比較像圍棋呢,還是比較像開車?我覺得有可能是前者,所以我們還是靜待專業人士一探究竟再做結論吧。
\subsection*{2021-01-24 13:49}

其實詳細分析,又分理論和實驗。做純理論的,照理應該是像數學家一樣,靠黑板和粉筆就能活得下去(有個好玩的真實故事:幾年前,全世界最高端的粉筆供應商,一家日本的小公司不賺錢、不做了,美國數學界爲之大幅震撼,到處搶購存貨)。所以在理論界,大佬的影響比較間接,主要在於安插工作,尤其像是高能物理理論,實際研究進展徹底乾枯,超弦這樣的科幻產出,完全沒有客觀的標準,全憑行内大佬主觀認定好壞,就像人文藝術學科一樣,自然推動了學閥的形成和强勢。
實驗,尤其是大規模的實驗計劃,就多出人事管理的維度,内部的分工、評比、資源分配,樣樣都賦予帶頭人極大的權力。這時實驗組的老闆往往不必關心實際學術細節,他或她的責任專注在對外要錢和對内督工兩方面上,工作的實際性質和一般企業的分部經理很類似,如果管理系統對他們像中國這樣完全放任,那麽腐敗絕對更快更嚴重。
\subsection*{2021-01-20 09:47}

隨機找一大群人,一般可以估計大約有20 \% 會是好人(即相當不願意損人利己的)、20 \% 是壞人(即非常願意大損人小利己的),其他60 \% 介於其間。所謂的社會風氣,主要影響這些中間多數;但是主導風氣的,卻是公權力對極端少數的獎懲。中國學術管理單位一方面放任作假,這是鼓勵壞人,另一方面嚴懲不發論文或空洞論文,即使只是在鑽研困難議題或運氣不佳,這是在懲罰好人,逼著多數學者造假或誇大,所以不但沒有整頓風氣,反而是在有意創造惡劣環境,那麽學術界裏有80 \% 以上的人偷搶拐騙是很自然的後果。
四年前,我批評悟空衛星的誇大公關,就是一個典型的例子。這個題目雖然原本成功(亦即發現新物理現象)希望就不大,但卻是扎實的基礎科研問題,值得做;而悟空團隊本身執行效率也很高,沒花多少錢,實驗也很順利;後來結果只得到負面的結論,並不是他們的錯,應該容許他們誠實報告,然後適當獎勵(頂級期刊讓他們發表論文,就是這個“獎勵”)。在這裏,所謂的誠實報告,也有它的奧妙,也就是統計上總是會有些噪音,論文就以這些噪音做總結,然後說這個疑似正面的結果有待進一步驗證。其實行内人心知肚明,它就只是統計噪音,那個結論只是場面話。最近20年,生醫論文有89 \% 無可複製,其實就是這麽一回事:實驗努力完成了,卻沒發現新現象,只好拿統計噪音來做結論,大家心照不宣,並不是要欺騙自己的同行。
所以悟空衛星團隊的實驗和論文都沒有問題,毛病出在事後的公關,把場面話拿來對外行人做欺騙,也就是從“空”升級到“大”,這才逼我出面批評。而他們論文已經發在頂級期刊,卻還覺得有必要搞虛僞誇大的宣傳,這顯示出很惡劣的學術環境,顯然是主管單位的錯誤獎懲逼迫他們撒謊,所以我對中國學術界的前景才會如此憂心。
\subsection*{2021-01-17 10:10}

唉,打假這事是非常基本的,但中國學術管理階層連這都做不到,的確讓人灰心。我對此事沒有什麽特別的洞見,所以就只簡單提一提。不過我談的處理誇大公關,如果類比成尖端科研,很容易被專業利益集團狡辯忽悠,那麽打假就像是加減乘除一樣,要瞭解是非對錯完全沒有困難,問題只在於做不做罷了。
暴漲論如同超弦一樣,一開始是一個合理的假設,但稍作研究之後,邏輯推論被證僞,於是就只好放鬆自由度到近乎無限,以便自圓其説。照理説,這時就應該把它標示為失敗的理論,扔進垃圾堆裏,但是全球幾萬名這個行業的教授仍舊搶著做研究,然而又沒有其他的理論方便發論文,所以就被學術界主流繼續抱著當寶貝,並且發展出各式各樣的藉口塗脂抹粉、對外推銷。這其實是正文裏所提,世界强權分立、科研重複投資的另一個後果。
我一直沒有急著詳談暴漲論,是因爲它是一個純理論,沒有什麽像大對撞機那樣的財務黑洞,所以大家把它當作科幻作品對待就行了。事實上超弦、暴漲和可控核聚變都是純科幻,但是有現實影響的當然必須優先解釋。
\subsection*{2021-01-14 21:04}

是的,鄧小平在對文革做修正和反思的時候,決定給予學術界極大的自治空間,在當時就已經矯枉過正,導致了他一輩子唯一的主要錯誤:讓半導體、大飛機和發動機下馬,卻去投資北京的對撞機。40年下來,更演化出既封建、又官僚,兼有二者缺失的惡劣體系(這個“封建+官僚”的描述來自《觀網》的讀者留言,我覺得非常簡潔恰當),亟待習近平來撥亂反正。
不過雖然糾正此事,沒有反腐那麽危險宏大,卻需要深刻的專業知識,目前甚至還看不出中共領導階級對問題有任何認識,所以是十分嚴重的隱憂。
我盡自己所能,想要提醒在位者,但既得利益集團的政治能量實在太大,連這次的視頻,都被合作的媒體機構草草處理。說實話來挑戰國家社會的吸血蟲,在哪裏都是極爲困難的。
\section*{【歷史】根深蒂固的誤解(三)}
\subsection*{2023-02-02 09:46}

這裏的大時代背景,在於高等教育的普及:我曾提過,美國的大學學位佔人口比例,在20世紀的100年内,從4 \% 上升到34 \% ,那麽當然不可能都成爲理性知識分子(其實4 \% 一樣是不可能的,不過那時候蠢人沒有數量上的壓倒性優勢,所以不敢公然獻醜)。然而平均程度急劇下降,自我認知卻依舊高高在上,與此同時還失去了知識精英的羞恥心和責任感,參見博客既往關於美國文化腐朽的討論。
中國高等教育體系的急速擴展,遠遠更加突然,與其配套的社會文化也更加功利,再加上不但忽略邏輯教育,而且還反過來縱容鼓勵作弊、造假、僞托(例如中學作文要求學生無中生有),所以一輩子只曉得求爽、不知道什麽是求真的人,自然佔絕對大多數。兩年前我撤離《觀網》的《風聞》欄,就是已經撐到人類忍耐力的極限,所以你的感受我完全理解,但教育群衆這個理性知識份子的社會責任,一個人不可能承擔得起,只能繼續請求大家分擔痛苦。
\subsection*{2020-09-07 01:06}

這是我一再提起的網紅效應。
我對像是中醫教和陰謀論,不但直接揭穿他們的謬誤,對迷信盲從者也完全不給情面,做出精確直面的批評貶抑。這不是在發泄情緒,也不是不知道會引發反感,從而喪失追隨者,而是博客原本的目標就是要提供當前人類社會最高級的理性討論平臺,一旦開始擔心得罪部分讀者的非理性偏見,那麽立刻淪入網紅效應的陷阱,成爲又一個假裝理中客、實際上是50步笑百步的科普網站。
有讀者私下向我贊許這個博客的嚴格紀律;其實他還沒有注意到,對讀者的風紀管理只是一個附帶結果,真正的重點是我寫作的絕對自律;這裏的前提是評論内容不受任何追求金錢利益的欲望干擾,也是爲什麽你在網絡上極難找到其他類似管道的原因之一。這個世界充滿了幾千萬爭名逐利的名嘴,有我這麽一個堅持事實真相的評論網站並不算太多。
\section*{【政治】社會主義國家應該如何管理資本}
\subsection*{2023-01-20 14:55}

首先提醒讀者,放烟花這種民生小事,素來屬於博客不予置評的範疇。我們在此討論的,純粹是制定任何政策時,應有的前瞻性預案和配套措施。民衆不配合,私下放烟花,就如同與美國簽碳排放協定後,會有對方廢棄、作弊、長臂管轄等等問題一樣,是可以簡單預見的。如果能解決、肯解決、而且值得解決這些問題,那麽當然可以考慮采納政策選項;反之,沒有任何解決方案,一廂情願地拍腦袋硬上,就是明顯的瀆職了。
中學教材内容在全世界都是國民人格教育的重點、未來三觀的規範。然而我對中國現有的版本沒有任何接觸,不適合討論其細節,能說的只有:1)社科是求真的學術,文藝是求美的技藝,兩者絕對不應該混爲一談,尤其不能通用同一個入學標準,亦即前者必須特別强調邏輯思維能力;2)早年我曾經對廢除高考英文持保留意見,但那是出於理工科方面與國外做知識交流的需要,考慮到歐美的社科和文藝人物是如此的腐爛,斷絕外來的殖民主義假新聞和頹廢文化影響反而是好事,所以文科除了語文系之外,的確不應該再考英文。
\subsection*{2023-01-11 13:54}

你說中國“息事寧人”的傳統危害很深,我完全同意;不過這不是儒家思想,而是官僚體系自我腐朽的結果。過去100多年的中國其實一直還在為晚清社會政治文化的徹底腐敗還債;毛澤東過於激進的矯枉過正,反而引發80年代後反動派的復蘇。我們只能希望現在這一代人能做出明智合理的改革。
昂撒體系是在國家整體出了大問題之後,資本爲了避免覆巢之下無完卵,可以容許愛國志士挺身而出做改革;上世紀大蕭條之後如此,現在美國霸權有失落的危險,也是如此。但請不要忽略了一個至關重要的客觀背景,亦即賈似道、文天祥之所以失敗,而美國的林肯、羅斯福之所以成功,基本差異在於美國獨占歐洲白人500年世界霸權所搜刮資源中的精華,有史無前例的豐厚財富纍積(而其最重要的政治效應,在於因而有足夠的財富分配給足夠的中產階級),地緣戰略上的位置不但極度安全,而且還方便幾百年的持續擴張,一直到最近才撞上極限,因而出現内捲問題。中國始終沒有那樣優越的條件,只有全體人民的更高努力和知識分子的更高智慧才足以作爲崛起的基礎。
\subsection*{2023-01-04 00:52}

幾年前博客曾討論過英國心理學會的一則研究報告,統計發現商業界高級主管的平均智商只在120左右;這個現象的基本原因,在於智商差距30點以上就會有日常溝通交流的困難(這是我一生的困擾,所以後來極度依賴文章寫作,因爲可以預先濃縮、精簡、組織,方便讀者消化吸收,就像為幼鳥喂食前先咀嚼食物,如此一來,只要有足夠理性思維能力的人都能理解),而在一般組織裏,和基層溝通的能力遠比自己高瞻遠矚更重要得多。
這個道理的一個自然Corollary(系理),是越依靠Popularity(聲望?知名度?嘩衆取寵?參考博客以往關於“選拔網紅”的評論)來做選拔的組織,領導階層的智商就越低,例如當代的歐美政壇。反之,但凡不是因出身而預先固定單一人選(例如英國國王Charles III),即便只從少數幾個候選人之中擇優提拔(例如金正恩、MBS和二戰的德國高官,其中海軍司令Raeder和空軍司令Goring智商都被美軍測量在140以上),能力也會更强得多。這裏中國體制也是正面案例,愚蠢政客不可能進入中央;習近平本人更加出類拔萃,和前任相比都能拉出代差,他明辨是非、察納雅言的能力是毋庸置疑的;所以問題出在現代中國廣納言路、探察真相的機制不但不如李世民或雍正,連歐美和台灣都比不上。
以美國爲例,政客天天和富豪們交際應酬,除了套交情要錢之外,自然也要瞭解“民”情、解決“民”怨,只不過你必須是豪門才算得上真正的公民罷了。至少到了州長級別,就會開始纍積可信的顧問;事實上他們所謂的“行政經驗”,指的就是幕僚和智囊團隊。在國家層級,還有大量民間的額外腦力資源(不一定附屬於政黨),可供總統挑選;不過請注意,這裏不是讓幾千人輪流寫報告,而是面談之後選拔少數幾個志同道合的謀士,做爲高層的日常資訊和策略來源。
至於你所談的,從基礎教育做起,這是博客以往的建議之一,對國家、民族、社會的長遠前途有重要的影響,但是緩不濟急,對未來幾年關鍵時段的外交和内政改革不可能有助益。而想要憑藉公共輿論來影響當前政策,若是政府放鬆控制,則非理性雜音汎濫;若是收緊管理,則被利益集團綁架,正面建議反而最先被封鎖。過去八年,博客一直很耐心地等待官僚體系逐步改進,但2022年出現明顯的倒退,尤其體制内外的建言通道都被堵塞,難免讓人灰心。
\subsection*{2023-01-02 11:54}

你說的是防疫政策所出許多細節問題中的一個。與其逐個檢討,限於篇幅和時間,我在此直接評論其背後的深層因素。
除了最後驚慌失措的棄守之外,三年防疫政策的主要問題並不在制定,而出在執行、宣導和糾正之上。中共在改開之後,對地方政府做了相當程度的放權,這在早年有其必要,容許各地靈活招商,但在過去十年,中國工業化層次逼近世界前列之後,就反過來成爲潛在的阻力:汽車和半導體產業政策都是明顯的負面案例,同樣放任給各省級單位競爭,同樣浪費了20年時間和無數的人力財力物力;後來汽車工業終於在電動化過程中成功升級,所依靠的卻是源自中央的靈活詳細管理,這剛好指明了正確的解決方案。新冠防疫是比產業升級更需要全國統籌的戰爭性任務,中宣部不統一内部意見、衛生部不制定細節規範、國務院不積極糾正地方缺失,不論實際細節如何發生,必然反映了【後注十二】所指出的國家級管理單位欠缺深思預謀和明確規劃的能力和意圖。
再看得更深一層,是什麽導致了前述的現象呢?首先是整個體制過度專注於防民,卻沒有足夠的防吏和防地方官的意識;這一點博客反復討論,這篇正文中又特別提起,請讀者自行復習。其次則是【後注】想要强調的,由官僚體系一步一步提拔出來的高層官員,由於選拔標準在於執行能力而不是全面智慧,沒有理由期望他們能夠自行選擇正確的政策方向和細節。換句話說,中國制度下的高官,和歐美資本選舉制推舉出的領導相比,只在執行能力和經驗上有優勢,在政策規劃上,同樣需要一流專業人才的輔助。
實際上,正因爲歐美政客的本質是演員和網紅,反而更願意並注重尋求一流顧問團隊,也因此在歷史上,能夠采行積極主動靈活的戰略;一直到最近2、30年,整個學術思想界的文化徹底腐朽,才摧毀了歐美政府做出明智選擇的能力。相對的,現代中國高官並沒有建立顧問團隊的傳統,所謂的幕僚純粹是執行助理,負責做建議的智庫水準參差不齊不説,還受到官僚體系層層節制,文章必須經過多級復閲才能上呈,名義上鼓勵暢所欲言,實際上根本不可能做到:如果你的建議不但一開始就不知道聽衆會是誰,連内容都很可能會被刪除修改,那麽你願意冒險得罪人嗎?就算有膽量敢做職業自殺,批評教育部的諫言極可能正由教育部處理,你還會有下次的機會嗎?這種糾正高層體制和管理的建議,古今中外都是由最高領導熟悉信任的人私下提出、直接送達,才方便理解采納,指望官僚體系下的智庫系統承擔起重任,顯然是不切實際的。
過去十年,中國政府逐步加强了監管,但針對官吏的只著重於反貪反腐,結果反而賦予非直接貪腐的詐騙集團更大的謀私空間,而且進一步提高了輿論反饋真相的難度。例如在中宣部加强過濾輿論的同時,中科大趁機要求所有公立和私立傳媒停止刊登任何有關量子計算的負面新聞;這在理論上似乎沒有關聯,實際上過濾輿論必須由底層執行,也就賦予他們自由心證的權力,這個權力是很難監督管理的,結果必然會被利益集團(包括非資本的集團)收買利用。換句話説,加强監管本身或許有必要,但必然有兩個副作用:第一是權力下放,第二是阻斷言路,所以必須有配套的彌補措施,一方面節制公權私用,另一方面開通建設性的批評管道。過去這一年中方有關量子計算和新冠防疫的表現都非常不理想,值得所有在乎國家人民福祉的人憂心。
\subsection*{2022-11-28 14:49}

防疫政策被地方官員層層扭曲,是官僚體系的必然(屬於“懶政”的變型,多年前留言欄曾有相關討論);這是全世界的普世難題,昂撒的作秀哄騙體制是鋸箭療傷,不止無濟於事,而且反而阻礙真正的監管和改革。
英美針對中方做顛覆,是建國以來的例常,在香港暴亂之後,更加是全民常識;現在第三世界尚且普遍有了抵抗力,中國若是連自己國内這點小事都搞不定,只能找中宣部和教育部問責。
我覺得這裏最值得注意的,在於帶頭鬧事的果然又是學生。我在《冷戰、學運和五四運動》一文中已經詳細論證,學生在所有社會族群中,生活經驗最欠缺、政治知識最貧乏、理性思維最薄弱、自我信念卻最高,所以學運對政府的運作,尤其是理性政府的運作,有百害而無一益。換句話説,學運的唯一用處在於阻撓合理政策、推翻合法政府,那麽中共建國掌權已經70多年,還在美化吹捧學運,豈不是和歐盟今年的自殺性作爲有異曲同工之妙?
\subsection*{2022-10-26 21:32}

你扯的這些都是無關緊要或者沒有實證的臆想(“不讓專家發聲討論”?你看不見並不代表不存在!),警告一次。我嚴重懷疑你是在學舌反中論壇的抹黑,違反《讀者須知》第六條規則,禁言1個月。請所有讀者注意,你若是沒有過濾無中生有虛假論述的能力,就應該避免去看;非要去看,就不要來博客,更加不能在留言欄直接當作事實來復述。
清零政策的邏輯合理性來自權衡生命的價值,一旦新冠的毒性下降,方程式的參數變了,最優解也跟著改變,這個邏輯思路並沒有錯。但全國性的政策變革,而且一旦開放就無從反悔,必須謹慎爲之,自然需要時間。我最近才説過,估計會在明年上半開放(亦即發生在今年的機率小於一半);時間過去還不到一個月,來此絮絮呱呱,所欲何爲?
\subsection*{2022-08-15 10:53}

This theory runs counter to my understanding of the current Chinese reconnaissance capabilities. Besides, if China wanted to track the jet and failed to do so, there would have been visible signs of the efforts. I would simply label the "report" as a typical wild speculation by uneducated newsmen. The educated analysis was that China refrained from the intercept option largely out of strategic consideration. On the tactical level, they believed they probably could but with slightly less than 100 \%  confidence, which further dissuaded the decision makers, who tended to be very conservative.
\subsection*{2022-07-30 11:07}

其實成品又分自產和進口兩類,其價格波動完全可以是互相獨立的。例如當前美國自中國進口的東西,價格依舊穩定,甚至因爲過去一年過度訂貨而疲軟,但是他們國内自產的貨品就因原物料和勞工的短缺,成爲這一波通脹的主要動力。一旦基層百姓的生活費用不斷升高,自然會有要求工資隨之上漲的動力,然後反饋到日用物品和服務的定價上,這就是我所談自我加强的循環效應。這個效應會使通脹固化:原本在初始階段,可以用2-4 \% 的真實利率打斷的循環,徹底固化之後就非得10 \% 才有效果(參見1980年的經驗)。然而現在的美聯儲貼現率仍舊只加到2.5 \% ,在通脹率已達8-9 \% 的背景下,對應著大約負6 \% 的真實利率,這正是我不相信美國光從歐日吸血就能解決通脹問題的主要考慮。
中國智庫的主要任務之一,是建立自己的統計資料庫,因爲統計數字是最重要的學術話語權;這是我以前反復談過的事。我知道上了年紀記性自然衰退,或許你該設法抽空復習博客的舊文。
\subsection*{2022-07-30 03:21}

關於“父為子隱,子為父隱”,那是因爲在中國古代的家族社會中,直系血親關係最爲密切,即使爲了社會公益也不能强迫切割,是必須要有的特別例外。現代美國家庭以夫妻為單位,所以也明文立法,規定一般公民對刑事案件做目擊證人和轉述招供的責任,對被告的配偶是豁免的,這叫做Spousal Testimonial Privilege和Marital Communications Privilege;它並不足以把美國劃歸人情社會。當然論語對這個道理說的太過簡略,實在應該要進一步稍作解釋,澄清它背後的考慮和應用範圍的極限;結果2000多年下來,原本的特殊例外被不斷擴張,來為家族利益集團的非法行動正名,的確為人情社會提供了助力。
爲了對抗資本,必須强化國家權力,然而執行這些權力的人轉過頭來以權謀私是自然的結果。但我也已經解釋過無數次,公益與私利之間的矛盾鬥爭是永恆無止境的,原本就不存在一勞永逸的體制方案。正文中提到的e-gov,是可以為正義一方提供一點助力的機制。此外,可以禁止二等親内在同一個組織内服務等等。歷代帝王做過許多不同的嘗試,個別來看,都可以簡單繞過,而且必然會因年久而荒廢,但那並不代表它們不是有用而值得考慮的政策。
\subsection*{2022-07-29 14:00}

是的,土豪和體制無關,是普世問題,到處都是;比起全球相對少數的國際財閥,處理起來麻煩得多,但對一般民衆的影響卻也遠遠更直接、更切身。尤其中國傳統文化是人情社會(不是真正的“儒家”!),所以土豪的生存活動空間也更加寬廣,例如台灣的政治和社會,從中央到地方基層,都由大大小小的土豪把持,司空見慣、習以爲常。
改開40年,縱容得太久、太過分了,就連習近平反腐,都沒有針對性地處理土豪,更別提建立誠實風氣。尤其對學術造假和誇大的絕對放縱,更是貽害深遠。我只不過看得遠、看得清,結果一般人反過來指控我迂腐、情商低、不懂人性。換句話説,許多群衆可以前一分鐘對周劼這樣的事件破口大駡,下一分鐘就轉過頭來對真正有效的根本解決方案嗤之以鼻,渾然不知他們自己的這種反射式的愚蠢、自私和雙標(參考幾個月前的上海防疫經驗),其實才是社會諸般問題的基本根源和進步的最大絆脚石。那麽由既得利益者自由影響和挑選“民意”的西方體制,自然不可能做出深刻的自清改革,其腐化的速度和程度也遠高於一般人的想象。

有人發私信給我,指出周家的地位和權勢並沒有他所誇耀的那麽大。我沒有第一手資料,無法置喙;不過我做上面評論時,已經避免針對特例,而是從社會大局來做討論,只要土豪真正存在,就還是切題的(Relevant)。

又有私信建議,希望我不要對群衆的愚性多做批判,以免有失自己的身份。以下是我的回答,和大家分享:
要真正治理腐敗,必須雙管齊下:一方面從上向下嚴打,另一方面在基層建立公共道德觀念,由每一個人監督周邊的害蟲。如果不先承認群衆愚蠢自私的劣根性,基層公德從何建立起?人情社會,你好我好大家好,正是鄉愿世界,德之賊也。海瑞的清廉兼具秦檜的圓滑,不但違反歷史事實,在邏輯上都是不可能的。
\subsection*{2022-07-23 10:28}

原本臺積電和三星就都是在7nm轉用EUV,也就是說,中芯這次做出來的,是運用DUV的已知經濟極限。這是因爲製造同級製程的成本,EUV和DUV在7nm做了交叉,實際上如果基於特殊背景考慮,非要繼續用DUV不可,很可能還有一點進步的空間。
至於中芯成就的技術分量,當前所發佈的那個芯片是極度簡單、微小的特殊用途處理器,並沒有什麽真正的實用價值。當然,要進一步纍積經驗、優化製程,應該不會有什麽跨不過的難關。事實上中芯從14nm進到7nm的速度並不慢,更加指向樂觀的預估。這裏的實際工作,主要來自前臺積電的人員,我猜有不少台裔老青年把最後剩下的一點肝功能貢獻出來了。
從戰略角度來看,7nm或甚至再進步一點,已經能滿足除了頂級CPU、GPU和APU的需要,對保障戰略自主和經濟持續發展,提供了大約5年的喘息空間,而且在當前的國際局勢下,剛好可以大部補償第三世界集團受歐美訛詐封鎖的損失,所以美國必然會試圖禁用設計軟件和設備售後服務以干擾這些應用,後續發展視其國内政治鬥爭的進程細節而定,目前還無法斷言。此外中方要自主開發EUV機器,5年是不夠的;反而國際局勢在5年内有突破性的轉變,例如某些技術先進國家,出現政治經濟社會的全面崩潰危機,是機率更高的解決方式。
\subsection*{2022-07-22 01:12}

我對Greenspan印象不好,在2015年的舊文裏已經詳細解釋過其根據緣由,請自行復習。對Outsourcing和中國崛起的作用毫無感覺,絕對不是美聯儲主席應有的表現;當然,幾年後有Yellen和Powell對照襯托,可能相對好看些。
美國的印錢始自1971年Nixon打破Bretton Woods,要説50、60年代興起的Monetarist Theory純屬巧合,就太小看宣傳洗腦的效果了。90年代的低息政策,和2008年後的QE都只是同一套思路自然演化的結果。
中文的“市場萬能論”並沒有與之對應的英文名稱;這也不是巧合:我說Friedman是傳銷天才,而在名稱上玩花樣,避免被抓小辮子,是傳銷學的基本功。
\subsection*{2022-07-17 04:15}

這方面我考慮很多年了,所以字裏行間也會引發若干讀者產生相關的疑問;不過因爲自己不懂俄文,沒有第一手資料,一直不敢妄做論斷(別忘了,我做邏輯辯證的要求極高,例如爲了瞭解英國的民意,以對其政局發展做出精確理解和預測,曾花三年多時間,讀了幾萬條《Daily Mail》的讀者留言;那可多半是臭不可聞的愚蠢評論)。如果非要給個結論,我目前傾向“兩者皆有”那個答案:從Yeltsin爲了不讓Primakov掌權,轉向一個似乎是親歐派的Putin,結果一樣是以國家社會爲重的愛國者,可以推論出俄國政治精英的向心凝聚力一直都是存在的。然而Putin能夠破格拔擢Mishustin、Nabiullina、Lavrov這些人,顯然他不止自己是千里馬,也兼任伯樂。中國在這兩個方向,都有可觀摩、學習、反省的空間。
\section*{【空軍】即將出現的新裝備(一)}
\subsection*{2023-01-07 18:32}

要成功干擾必須暫時致盲,隨實際投射至目標(亦即必須有高效可靠的瞄準和聚焦系統,激光發射器在這個任務需求下反而不是技術難點)的功率大小,效果有明顯的優劣差異。不過確實的功率數據是軍事機密,我卻只能從公開訊息來做推敲,只知道這次部署的CIRCM已經是美國的第二代干擾機;更早的LAIRCM雖然被海空軍(小規模)采用,但陸軍卻誠實地揭露它的實用效能很低(參見https://www.thedrive.com/the-war-zone/17969/us-army-hits-setbacks-trying-to-add-new-infrared-countermeasures-to-its-helicopters),又强制廠商多花了幾年研發才接受新一代的產品,所以CIRCM可能比俄軍的干擾機更成熟些,但細節未知。
\subsection*{2021-07-27 00:27}

美宣中胡扯得最離譜的口號,正是“自由”兩字;人類是群居動物,個人怎麽可能有絕對自由?自由只能是在社會公益最大化的計算過程裏,衆多考慮中非常局限的一項,否則立刻自我矛盾:你有侵害他自由的自由,他有沒有不受侵害的自由呢?“自由”做爲白左宣傳説辭的重要成分,結果反而成為歐美右翼民粹反對戴口罩、打疫苗等等反智行爲的藉口,正是因爲它原本邏輯就不自洽;流行性瘟疫來襲,每個人都是傳染鏈的節點,是否參與防疫,影響的不只是自己,也是未來所有可能接觸的其他人生命所繫,怎麽談得上“自由”決定?
“學術自由”指的是研究過程中,誠實地應用邏輯以達到正確結論的自由,不是滲入私貨冒充結論、或甚至直接撒謊、造假的自由,更加不是人事管理和資源分配上的自由。你只要看看美國自己是怎麽做的,而不是怎麽説的,就會明白這個道理(尤其在外宣需要下,美國人對扭曲科學結論也不手軟,事實上連應有的“學術自由”也不絕對尊重)。中國在科研層次、專業知識纍積、對外來人才的吸引力、經費資源投入等等都還落後美國。後者容許一些浪費,尚且慢慢地喪失領先地位;中國有後來居上的態勢,靠的是無數基層人員的犧牲奉獻(還記得我談996嗎?),對研發效率事關緊要的學術界風氣上反而遠遠更加糟糕,政府卻拿出一個無腦的口號當藉口而撒手不管,這是哪門子邏輯?Putin有一句名言:“你把權力扔在地上不用,自然會有壞人搶著去撿。”他説的是沙皇Nicholas II和Gorbachev兩次亡國的歷史,華語界對馬英九也記憶猶新,如果學術管理還去學他們,真是腦殘至極。
\subsection*{2021-07-25 12:52}

SSMB是趙午教授的新發明,是又一個新的變種,試圖將高能物理加速器的技術,轉用於實用目的上。還記得2017年前後,趙教授也曾經出面反對建大對撞機嗎?他不但是行内人,所以面臨的同行人情壓力大得多,而且身體力行,把自己的專業研究也投入對人類社會有實際回報的項目,實在讓人佩服。
趙教授他是新竹清華畢業的學長校友,但似乎和北京清華有密切的合作,所以後者在SSMB的研究上處於世界先進的地位。不過SSMB是全新的技術方向,目前還在概念驗證的階段,距離工業應用,即使是全力投資,而且運氣好,也至少要15-20年,所以作為第二代EUV光源的備選是可以嘗試的,但和已經實用化的LPP技術相比,還是不能同日而語。
的確,和趙午教授一對比,很明顯可以看出中國科研學術界有著嚴重的逆淘汰現象,脫穎而出、被選拔出來作爲領導階層(亦即院士級別)充斥著太多玩政治的專家,只想著騙經費、方便自己發論文,進一步建立山頭圈子,反過來打壓幹實事的研究人員。這是幾十年Laissez-faire絕對自由主義學術管理下的天然後果:政治資本纍積集中之後,必然發生的尋租和托拉斯現象。中國政府自稱是共產黨,口頭上復述Marx對資本主義的批判,在工商發展方面也懂得要監管,卻沒有想到同樣的經濟原理適用於任何既合作又競爭的群體交互作用。換句話説,美國在Reagan之後搞絕對自由市場經濟,結果帝國迅速腐敗、中國得以崛起,但中方卻早已在科研學術上放任一模一樣的絕對自由市場原則,以致未老先衰。現在發動機和半導體的窘態(還記得我説過,當年鄧小平不聽楊振寧先生的勸告,解散這些有用的研究團隊,把節省下來的寶貴資金浪費在北京對撞機之上,是他一輩子最大的政策錯誤?),只是反映出來的冰山一角,如果不好好整頓科研學術管理,未來這類國家吃大虧、必須投入無數人力物力財力資源來彌補的案例,還會不斷發生。

剛剛看到《Nature》的這篇論文:《Free-electron lasing at 27 nanometres based on a laser wakefield accelerator》(https://www.nature.com/articles/s41586-021-03678-x)。這是中國物理學家試圖將自由電子激光微型化的成果,波長距離光刻機所需很近,整個裝置只有12米長也很合適,問題在於功率和頻率距離工業標準還差好幾個數量級,所以依舊只是概念驗證,不過是比較成熟的概念驗證。
\subsection*{2021-07-24 11:54}

目前ASML的EUV光刻機所用的美製光源,的確是以大功率CO2激光將錫(Tin)的顆粒瞬間氣化(Vaporize)成為等離子體,然後自然輻射出峰值在13.5nm的極紫外光(這叫做LPP,Laser Produced Plasma),剛好對應著Mo/Si Multilayer Mirror反射鏡的工作頻段。中方是否有合適的CO2激光,我不熟,但印象中有過神光計劃,製造大功率激光的專業能力不像會是個問題。
美國研究EUV光源是從1970年代就開始了,但一直到1990年代末期才慢慢專注到錫顆粒等離子體光源(Tin Droplet Plasma Source)上,然後又花了十幾年才成功實用化。之所以原本不看好,後來又很花工夫,主要有兩個原因:1)錫等離子體(以及必然伴生的氣態、液態和固態殘渣)的回收不容易,尤其必須避免污染反射鏡;2)光刻機要求每秒照射幾萬次,平均發光功率達到250W以上,這樣的頻率和功率下,錫顆粒的饋送和激光的瞄準都很困難。然後反射鏡也要求極端的工藝,其表面粗糙度必須低於0.45nm。我覺得以上討論的這些技術問題,才像是中國研發單位真正面臨的難關。
我的專業是高能物理,EUV光源的關鍵問題在於工業上的實用化,很多細節只有行内人(例如ASML的員工)才可能知道。我聽説業界除了LPP之外,還考慮過DPP(Discharge Produced Plasma,這是直接用電流來產生等離子體,不過好像很難把功率做上去)、FEL(Free Electron Laser,自由電子激光,我在2017年的那篇《回答王貽芳所長》裏,就提過它是最佳的硬X光光源,後來上海建了一座;當然要降低能級到EUV,在理論上也是可行的,問題可能還是在於功率,或者是廠房空間)以及LSS(Laser Synchrotron Source,又稱Inverse Compton Scattering Source,自由電子激光的一個現代變種,理論上可以大幅減低空間和能量消費,但技術相對不成熟)。或許這是中國物理人報國的大好機會。
\section*{【美國】海湖莊園抄家事件幕後的美國政治鬥爭}
\subsection*{2022-11-12 10:28}

其實要遵循邏輯規則並不難,真正的難點在於尋找正確的認知框架,這往往需要天量的事實和無數的試錯。以這次中期選舉爲例,如果你瞭解了正文所揭示的認知框架,就會明白並不是民主黨打敗共和黨,而是建制派大獲全勝,消滅了不受Deep State掌控的民粹派候選人。因爲這些民粹派是Trump的權力根基,真正的輸家正是Trump。至於Deep State安排DeSantis做潛伏臥底,分析起來反而很簡單,因爲這是理所當然的一步棋:反正Trump也不是真心反對貧富不均,那麽找另一個假心假意的民粹領導人,完全沒有被識破的風險。所以選舉結束,Trump立刻感覺到威脅,大嘴轉向攻擊DeSantis,但這徒然揭露他自私自利的本質,只會加速他被淘汰的步調。
\subsection*{2022-09-23 23:56}

一.是的,這是Deep State“勸阻”Trump參選的手法之一,正文裏已經簡略提過。以家人為人質,的確很下流。不過美國的司法體系對白領犯罪非常寬鬆(是資本多年來有意造成的;參見《美國式的恐龍法官》系列),Trump一家要脫罪的可能性並不小,所以算不上是什麽殺手鐧,説不定反而鼓勵他積極參選,以便用特赦來對聯邦刑事案一勞永逸,總統權力對州級的民事案也有間接影響。
二.這其實是一周前為上唐湘龍節目準備過的第三個話題,時間不夠沒有談;這裏我簡單列舉大綱。美元匯率和利率調整,影響的主要是國際貿易和資金流動(對國内需求面的影響,很容易就被通脹上升所引發的消費潮所淹沒;參見兩周前我在另一篇文章後注中討論的企業和家庭現金儲蓄);問題在於美國經濟對進出口的依賴相對來説並不高,而其國内供給面面對的卻是40年的舊債必須一次還清。原本美聯儲和聯邦政府如果在2021年政權輪替之後,立刻以控制通脹為第一優先,還有機會暫時壓下、維持既有經濟格局;現在已經太晚了,所有的國内通脹因素一起爆發,最近的經濟資料都指出通脹已經普及到整體經濟的每個層面(指薪資和成品,過去40年一枝獨秀的資產通脹必須做出若干囘吐,大宗貨物則有起有落,見下文),並且正在繼續深化的過程中。短期内美國通脹指數的稍微緩解,除了美元强勢的作用之外,最主要是能源價格回落(而這裏的最主要原因是中國經濟不景氣,能源消費大幅降低)。雖然在對未來一兩年的中期預測上,歐、日都將面臨世紀級的經濟衰退,進一步減少全球能源消費,然而農產品因氣候因素而短缺的問題還在,美國國内的通脹因素也足夠壓倒美聯儲當前這種一次75基點的升息,而且還有最具決定性的去全球化帶來對美元的替代,所以Yellen所説的明年徹底解決,是癡人説夢。
至於昂撒體系通過IMF收割第三世界,博客已經談了八年,討論亞投行的時候,說得尤其直白。當時中國知識思想界和金融管理階級都不聽,現在臨時抱佛脚怎麽可能來得及?只能眼看著他們在金融上被搜刮,財富被用來填補美國的國力空缺;不過政治上倒不是問題,因爲這些國家有了全新的教訓,對建立替代性的國際體系應該會更積極。
\section*{【美國】【國際】2022年國際局勢的回顧與展望}
\subsection*{2022-11-09 16:47}

這麽重要的訪問,涉及雙邊戰略關係的升級和全球外交格局的改變,事先必然需要好幾個月的籌備和交涉。三個月前Reuters放話,當然是昂撒集團及早曝光做警告;現在傳言又起,則指向恫嚇MBS的企圖並未成功,那麽進一步動手的可能性當然不小,尤其是類似2016年土耳其的政變。希望MBS能渡過那一劫。
前天有老同學來台南看我,聊天的時候談到我做的預言往往提前數月到數年之久,遠超一般群衆的記憶範圍,所以博客自然就局限在極少數聽衆;我回答說那正是我想要的。事實上電郵詐騙犯會故意寫下不通或幼稚的文字,以便預先挑選最笨的對象,省得後續聯絡浪費時間;博客也遵循同樣的邏輯,只不過是反過來挑選最有理性思維能力的那一小部分人口,所以對犯蠢的讀者絕對不能假以辭色。新讀者如果不想成爲勸退的目標,就應該反復復習舊文。例如這裏所討論的,是幾個月前視頻訪問和留言欄中評論過MBS的處境,要是不記得,就必須警惕自己,知道沒有資格插嘴。
\subsection*{2022-11-05 09:28}

表面上看,Scholz和Merkel很類似:兩者在上任之初,都曾經接受昂撒白左的仇中抹黑宣傳,然後才面對現實、逐步轉而專注在實利考慮之上。不過正如你似乎已經理解的,Scholz不可能複製過去20年中德經貿合作互利共贏的經驗,這有幾點原因:首先,Scholz賦予綠黨在聯合政府中外交和經濟的主導權,並且從一開始就完全放任,未作節制,早已尾大不掉。其次,歐盟對俄制裁實際上是自我犧牲,在50年一見的歐美通脹危機下,主動將資金和產業奉獻給美國;這場危機和它的特性是我在過去幾年反復强調、預測並討論的議題,不再贅述。然後,中德平等合作的時代背景,也就是全球一體化的國際體制,已經壽終正寢,取而代之的是Golden Billion和人類社會的對立,而中德分屬兩邊。
綜合上面的考慮,我們可以對雙方關係的發展做出幾點簡單預期:1)Scholz政權不可能長久;2)歐盟和德國都將陷入短期極度嚴厲、長期無可扭轉的經濟崩潰;3)未來的中德經貿不再是國家間的對等互惠,而是中國和德方經濟組成部分,亦即企業、資本、專家等等化整爲零的合作。
很不幸的,正因爲德國的民主體系較爲健康,資本並沒有構成對政府的凌駕和掌控,反而給了昂撒集團通過文化宣傳洗腦竊取政權的機會,將國家置於萬劫不復的死地,現在要談救贖,爲時已晚。不過這次Scholz拗不過企業界的絕望要求,將對中對俄戰略分拆開來處理,甚至不惜公然打破歐盟團結假象,拒絕Macron同行的建議,不但對德國產業技術向中國轉移提供了便利,也使中國在這輪反昂撒霸權革命中有了不必站上前綫、可以留在相對安全的後方繼續深化改革的餘裕。這並不代表中方應該置身事外、守株待兔,相反地,中方在非軍事手段上積極出手,成爲更加順理成章的作爲,尤其是金融貨幣方面,博客從成立之初談到現在,老讀者耳朵都長繭了。好在這個行業的親美主管最近在二十大終於提前退休,我對未來幾個月持審慎樂觀的態度。
\subsection*{2022-08-10 21:55}

Putin在出兵之前有三個選項:1)封鎖圍困、以戰逼和,亦即實際上的第一階段作戰,因爲烏克蘭與北約直接接壤,所以必須占領交通要點和首都外圍;2)以絕對優勢兵力速戰速決,但這會在其他戰綫留下空隙,賦予北約突襲的機會;3)實際上的第二階段作戰,鈍刀子割肉,花幾個月時間持續大幅殺傷頑固份子,否則讓他們投降反而不方便處置。如果這些策略能夠成功,對烏方的老百姓來説,傷亡損失是(1)<(2)<(3)。然而(1)有相當機率會失敗(實際上也失敗了),那麽(2)反而才是最仁慈的選項(但對俄國來説,若是失敗有亡國的危險)。
以下是台海和烏克蘭在戰略地理局勢上的最大差別:後者接受北約補給支援靠的是陸路,北約威脅俄國也是陸軍突擊,然而台灣是海島,補給支援靠海路,美軍威脅共軍也靠海軍。Putin不能把陸軍大幅投入烏克蘭戰場,必須保留强大的預備隊,而中方卻沒有這個顧慮,因爲對方不可能登陸突擊,預備隊基本只需要海空軍。所以客觀來看,中方的最佳策略顯然是(2)。然而從這次軍演來倒推決策幕僚的心理,似乎依舊偏愛(1)這類自縛手脚的婦人之仁,那麽若是失敗,守方就有時間多做準備,連帶著影響升級為(2)的成功機率。
言歸正傳,從百姓的觀點來看,最大的變數還是在於守軍是否拿居民當肉盾:若是接受美國智庫的建議,學習烏軍的焦土戰法,派出大批反坦克小組到公寓裏埋伏打巷戰,並且堅守每一個地下室,攻方唯一的解決方案是把樓房轟掉,差別只在於是俄式的一次幾公斤高爆藥轟穿、還是美式的一次一噸直接轟平;這正是我在兩個月前已經討論過的:保存城市不被摧毀的責任在守方。當年納粹德軍尚且有足夠的職業道德和人性,事先宣佈巴黎、羅馬等古城為不設防,現代這種狂轟濫炸的新戰爭傳統,恰恰始自二戰的美軍。所以關心自己身家性命的台灣民衆,從烏克蘭戰事所得到的頭號教訓,應該是必須堅決反對任何打巷戰的計劃;至於野戰起來,要用火箭彈還是掃把,倒是次要的議題。
\subsection*{2022-07-21 16:06}

這是因爲政治是權力精英的游戲,而且必須拉夥結黨,玩家數目並不多,例如在最新後注的討論中,英國政壇真正的勢力就是五個(昂撒金融、非金融、猶太金融、愚民直覺反應、和少數殘存的良心人,而且後者在Corbyn下臺之後已經無力參與權力的游戲);而經濟裏做獨立動作的參與者,數目遠遠高得多,這不但立刻高度複雜化,而且容易產生囚徒困境等等悖論。
另一個差別是,玩政治的對己方的行動目標認知很清楚,如果能觀察上一段時間,通常可以確定他們的Modus Operandi;經濟則由於題材複雜、深奧,再加上主流學術理論被長期扭曲,個別玩家做出不可預期的隨機非理性動作是家常便飯。
所以總結來說,我對前者的預測有時可以精確到個人和月份,對後者則只能談必然的大趨勢,而且有+-一兩年的不確定性。
\subsection*{2022-05-07 10:31}

我説過很多次,預期經濟衰退,就如同預測雪崩一樣:只能斷定它必然會在一個大致時段内發生,但實際發生的確切時間和方式,卻不可能準確説定,因爲這類崩潰過程是混沌現象(Chaos Phenomenon)。
美國第一季的GDP負成長,是美聯儲去年底終於明白自己放水太多之後,緊急Taper的後果;一旦發現經濟承受不住,Powell又反過來只加息50bps。客觀評估只能確定他們無法走鋼索到底,至於是從哪個方向墜落深淵,要視美聯儲未來的決策而定。既然這些未來決策還沒有發生,旁觀者當然不可能準確預言。不過這裏是二選一,所以不明就裏的傻子反而可以很高興地隨意猜測,也有50 \% 猜中的機率;相對的,一個理性的邏輯分析者,面對絕對隨機的混沌現象,應該指明這個事實,然後拒絕參與這個胡猜游戲才對。
我的確擔心中方繼續配合美元霸權的吸血機制,主動幫助美國渡過今明兩年的金融財政難關。博客不是一直說,蠢往往比壞還要糟糕嗎?
我以前也早解釋過,除了軍事和金融之外,想不出世界要如何擺脫昂撒霸權的桎梏。軍事衝突的風險太高,但願是貨幣革命解救人類。
\subsection*{2022-05-05 23:45}

墮胎這件事,原本就不是什麽基本人權,而是取決於社會結構、宗教文化和經濟層次的政策選擇,應該通過正常立法程序來做變革。當年Roe-v-Wade利用美國財閥打擊行政權的鬥爭(細節我反復解釋過,這裏指的是其中藉由無限擴張司法權來挖聯邦監管單位的墻角,參見有關Ralph Nader的討論),趁機偷渡成功,名不正言不順,已經種下禍根。幾十年下來,更加成爲共和黨系財閥轉移話題、動員保守派群衆的最佳藉口;而相對應的民主黨系財閥也自然發現可以用來反動員,於是也有了故意不解決問題的動力。否則以近年70 \% 的民意支持,民主黨又多次掌控白宮和國會,早可以簡單立法,一勞永逸地消除爭議。
不過這次中期選舉,經濟因素太過明顯强烈,墮胎這種社會議題並不足以影響結果;所以我對未來幾年美國政局的估算,不因此事而有修正的必要。
我說德國政府可能提早垮臺,當然不是綠黨自願下臺,也沒指定是今年下半,而是要等經濟崩潰、民怨沸騰,然後工商界精英聯合工會對親美的知識界做出反擊。把自己的想法硬往我嘴裏塞(putting words into my mouth),警告一次。
\subsection*{2022-03-16 19:51}

歐洲的歇斯底裏反應,Putin固然必有預案準備,卻也不是事先能確定的,更別提有意促成。昂撒媒體將好萊塢超英電影的故事結構應用在新聞敘事之上,歐美台港民衆照單全收,已經是極爲可笑;然而你用華語連續劇的邏輯來做推論,一樣的完全脫離現實。我的建議是,少看大衆娛樂,多讀世界歷史。
從地圖看,波羅的海三國當然位置尷尬,但北約是軍事上越不過的檻,現實裏沒有快速解決的可能。
歐美企業的撤離,是因應暴民情緒,目前還只是口頭説説,要關門裁員或脫手轉讓,可以簡單拖上幾個月,届時若局勢緩和,必然有人會想反悔。然而我不認爲Putin公開宣佈要徹底脫鈎是情緒發泄,所以應該只有俄方最有需要的(例如航空和重要的實體產業)才會被允許留在俄國。
Putin連Zelensky都願意接受,和約有歐盟的背書當然會是加成。目前的問題在於法國大選還要五周,在Macron確定勝選之前,外交折衝不可能有明顯進展(然而Putin自己都不急,旁觀者急個什麽勁兒?);若是其他人當選,自然又是黑天鵝,後果難以事先預測。
至於和約條件,現在俄方提出的正式要求,基本和我在過去20天的評論一致,唯一的差異在於Putin還要求烏方承認兩個東烏共和國獨立。我懷疑這是用來討價還價的籌碼,是烏方(如果夠聰明;現在Zelensky連放棄Crimea和裁軍都不願意)可以讓Putin做出讓步的方向。
\subsection*{2022-03-09 14:19}

通脹,尤其滯脹,必然大幅壓低各式各樣金融資產的價格,這是Stiglitz鼓吹印鈔的考慮之一。問題在於有辦法找到能抵抗通脹的實體或國際資產的還是財閥,我已經給了Gates的例子。
在2008年金融危機之後,美國國會通過了Dodd-Frank法案,包含Volcker Rule,禁止投行自己下場炒作資產。後來雖然被各大投行聯合起來修法,挖出一個大漏洞(參見前文《富豪口袋裏的國家》),但已經來不及挽回炒作金融資產的統治性市場額分,被新興的Private Equity取而代之;後者的體量在過去十幾年中成長五倍,超過了10萬億美元,例如總量1.8萬億的Junk Bond中Private Equity就佔了過半。換句話說,2022年Private Equity的市場角色,相當於2007年的投行;所以我懷疑在這一輪將至的新危機中,扮演Lehman的會是一個Private Equity Firm。届時聯邦政府和美聯儲必然又會在事後拿國家資源來補貼大財閥,這才是他們竊國的模式,不是像你所想的那麽有遠見。
\subsection*{2022-03-06 22:34}

因爲新博文可能還要十天才能刊出,我在這裏補充一點細節,解釋一帶一路爲什麽和如何過時。
一帶一路的基本思路,是由中方提供技術和初始資金,幫助落後國家進行開發,尤其是基建。那麽一方面這些國家得以加速發展經濟,成爲中國出口的潛在客戶;另一方面中國則短期能利用過剩產能,長期可以間接獲得更大的國際影響力;尤其中歐之間的交通綫,更是一帶一路建設的重點,兼具整合歐亞大陸的潛能。當年建立亞投行,遵循的是世界銀行的扶貧模式,就正因爲它被構思為一帶一路的一個環節。
這個策略的問題,在於它築基於一個隱性假設,亦即全球外交經貿規則的制定和執行,在時間軸上相對穩定,在空間維度基本完整,在原則上大致公平互利,以及中歐經貿關係不可能完全破裂。然而實際上,當前國際政經治理體系是歐美先進國家的俱樂部,對新興國家的參與只做了非常表面的敷衍工作。在冷戰結束後的全球化階段,美國在面子上有矜持,歐盟也有務實的領導,所以一帶一路還可以發生作用。一旦Trump撕毀昂撒集團的假面具,隨意出手對外做出經貿壓榨,立刻就暴露了國際體系的Impotence無能,WTO和國際法庭的癱瘓是最突出的案例。現在歐盟為英美站隊,則是更進一步將既有國際行爲規則從“無力制約昂撒”演變為“昂撒對外打擊的工具”,如果繼續埋頭做生意,是保證會被先搶劫後謀殺的被動反應。在這個新時代、新環境之下,中國外交戰略的重點必須改爲對歐美掌控下國際規則體系的所有主要環節做出替代,例如SWIFT。進一步考慮,我們可以拿IMF來和世界銀行對比:前者其實是不挂名的國際破產法庭兼Lender of the last resort,享有後者完全欠缺的規則制定權和執行權。當年亞投行專職搞扶貧的時候,客觀威脅還沒有明顯化,我只表達了潛在的憂心,說那是金融和外交資源的次優使用;幾年發展下來,已經可以認定的確是完全錯誤的選擇。

答案其實都在新博文裏,這裏我先簡單回復:
1.是,一帶一路已經過時(其實從2017年,Trump撕破“Rule-based international order”的假面具,針對性地直接出手打擊挑戰者時,就已經過時了;歐盟為昂撒霸權站隊,只不過是徹底消除了中方以拖待變、矇混過關的戰略選項);不過新博文不能直説,所以選擇“必須更進一大步”這樣的委婉用語。
2.哈哈,現任霸主以心狠手辣、下流無恥著稱于世,又剛剛得到所有老工業國的背書,你覺得中國有避免戰略投入的餘裕?
3.不但不必考慮避免接盤,而且反過來不得不另找替代;中國的實質盟友不嫌多,而是怕不夠。俄國若孤立無援,尚且會有經濟全面崩潰的危險;以中國對外貿依賴之深,哪可能承受得起被踢出國際經貿體系的打擊?美國有歐盟站隊,會找不到全面制裁中國的藉口嗎?中方應該賦予美國對不聽話國家各個擊破的閑暇,還是提前主動聯合所有潛在受害者,預做防範?這裏和昂撒霸權前例的根本差別,在於經貿同盟只要擴張有序、管理得法、不揠苗助長,可以有益無害,只有軍事同盟才必然會有反噬自身的危險。
\subsection*{2022-03-06 22:15}

這兩份稿像是別人準備、你代發的。。。注冊六個月才能發言的規則,雖然對一些想問正經問題的新讀者造成不方便,但對博客整體來説是件好事,尤其是賬戶被拉黑忽然有了嚇阻力,不再是另外注冊一次能簡單規避的。
1.歐盟從來就沒有“Great Again”的可能,一直都是在慢性衰退之中,問題只在於:A)衰退得多快?B)在衰退到無關緊要之前還剩下的十幾年之間,對中美霸權更替的過程有何影響?Scholz的選擇決定了新的答案:A)加速;B)歐盟從對立雙方的緩衝成爲其中一方的打手,中方必須做出針對性的外交措施和戰略回應。
2.是的,尤其低估了俄國復蘇的可能性和高估了自己在21世紀全球經濟中的分量,以爲可以予取予求,利用北約、歐盟東進來對歐亞大陸内地吃乾抹净。
\subsection*{2022-03-05 00:54}

原65樓被刪了。請較為新進的讀者注意,這個博客不是容許隨便胡扯的地方,欠缺内涵、沒有評論價值的留言自然會被清除,其他讀者只有在我已經回復過之後,確認那會是留言欄的一部分,才應該加入討論鏈。
過去八年,博客的教育工作從底層做起,影響的基本是民間的輿論;中國的國關、戰略學術界有成見的人要自己覺醒,反而需要更多時間。這是爲什麽當前中國公共輿論對國際議題的討論,出現了明顯的上下顛倒現象:非學界的讀者留言,往往比智庫、學者的眼界要高、思想路綫更靠譜,而且後者之中,名氣越大、資歷越老的,越是抱殘守缺。這個機制自然引發官方策略的滯後現象,就在兩周前這還無關緊要,但國際局勢的突然演變,已經將它凸顯為我未來一兩年必須努力彌補的方向。
原本歐盟外交穩步務實,英美經濟虛胖無根,中方與昂撒集團的鬥爭大勢已定。去年秋天Merkel退休前我寫的《美國制華歷程分析及對中國政策調整的建議》已經開始為後霸權世界預做籌謀。與其同時,我卻也私下擔心中國崛起過程太順利,會提早引發自滿心態,難以貫徹所有必要的改革,利益山頭將得以固化内部的不合理制度和慣例,尤其習近平可能會在2027年退休,届時更加可能有社會進展全面停滯或退化的危險,權力階層轉爲專注於内耗。所以我急著挑選最嚴重的危害,高調批評,參見《量子通信和計算是中國學術管理的頭號誤區》。
Scholz把歐盟外交的理性底綫徹底擊破的後果,除了引發當前的歐俄對立危機之外,也使未來幾年中國面臨的外來威脅大幅惡化。短期内,中方必須積極采納一系列新的對應措施(參見下一篇新博文),以促使美國霸權衰退的歷史進程回歸正軌,但長期來看,反而有益於中方維持對内改革的動力,賦予解決内在威脅的努力一些額外的時間。
\subsection*{2022-02-28 22:52}

我同意沒有必要對軍事進展太過關注,但原因不是難以預測,剛好相反,是因爲俄軍勝利是絕對的必然,唯一的問題在於需要兩周以内、還是三周左右的時間。目前有進一步的消息(來自長居烏克蘭的誠實網絡媒體人Gonzalo Lira,參見他的Youtube頻道)指出,俄軍不但避免損壞公共設施,而且以二綫的非精銳單位爲主(我的確也注意到他們配備的主要是老式裝備),此外許多軍事評論員指出實際投入作戰的部隊規模十分有限,雙方軍力對比並不顯著高於1:1。這又一次指向Putin並沒有强調速戰速決,所以一方面,戰場上的實際進展在前述背景下,可以說是順利的,代表著烏軍的抵抗總體來説很混亂、薄弱;另一方面,Putin可能預期北約有參戰的可能,所以留了後手。
至於戰略層面,我已經反復解釋爲什麽我不同意、也沒有預期Putin做出這樣的冒進,然而現實已經發生,而且昂撒媒體充分利用德國新政府的愚昧無知,陷歐俄於同歸於盡的死地。我們現在應該專注的,是中方如何因應新發展,為俄國解套,從而讓歷史回歸昂撒霸權持續衰落的進程。我正在撰寫一篇新博文,請大家稍安勿躁,在文章發表後再做討論。
\subsection*{2022-02-27 06:18}

現在是21世紀了,瓜分國土的事在公關上非常難看,更何況波蘭對俄國有反射式的仇恨,而Putin連自己士兵的性命都願意犧牲,寧可減慢攻擊效率也要保護基礎設施,這只能解釋為企圖爭取人心。既然要爭取人心,就不方便侵占國土;所以從他不打民用設備這件事,就可以看出事後分裂烏克蘭國土的機率不大。
我已經説過,Putin這次的決策可能沒有做到完全理性,而是基於歷史情懷,對情勢抱有一些幻想。從絕對客觀的角度來看,不但事先應該再忍耐幾周,繼續做外交斡旋,一旦開戰,反而不能客氣,必須一切以進軍速度爲重。日後重修發電廠太貴,那麽可以打擊綫路和變電所;水庫影響人命,那麽就炸管道和自來水廠;至於通信網絡,更是早該一步摧毀,省的方便烏軍組織抵抗,以及外媒持續抹黑;靠近居民區的軍事目標,也不能猶豫。畢竟任何拖延,都是給予對方額外的時間來製造混亂和破壞,例如現在Zelensky在Kiev大發自動步槍,立刻導致幾百起私人仇殺,等俄軍占領後要恢復秩序,豈不必然會有更多殺傷。是以,聖人不仁,以百姓為芻狗。孫子兵法只説“上兵伐謀”,後世卻往往和“攻心爲上”混爲一談,其實兩者毫不相干,後者是一個不懂軍事的東晉文人裴松之的清談胡扯。
\subsection*{2022-02-27 00:59}

烏克蘭國土東西狹長,大半部隊卻部署在東烏前綫,俄方又從北東南三面對其包圍,所以俄軍作戰方略必然是南北夾擊、從中切斷;唯一的疑問是夾擊的鉗形攻勢有幾道。現在我數的,第一道是Kharkiv/Melitopol;第二道是Kiev/Mykolayiv;如果Lviv也是目標,那就是第三道了。占領全境或許不是Putin的戰略目標,但他必然想要徹底消滅烏軍戰力,那麽攻占全部領土還是值得認真考慮的步驟。
至於政略,俄國既然動手了,就只能頂著被駡;未來的主要問題,在於如何重建烏克蘭政府。中方只要口頭堅持中立,就可以簡單置身事外,坐看美歐虛張聲勢、色厲内荏的醜態。事實上,白左洗腦宣傳用力過度的後果,是美國反而可能被迫自戕:原本把俄國踢出SWIFT系統是反射性選項,後來據稱是美聯儲嚴重警告會有逼迫歐盟和第三世界加速去美元的危險,才忽然銷聲匿跡;現在昂撒媒體罵的太過慷慨激昂,美國國會議員要求Biden做出更强力的制裁,後者不得不重新考慮爲了選舉權宜而自毀長城。類似的鬧劇其實在英國已經上演過了:因爲昂撒媒體始終抹黑Putin,說那些Oligarch都是他的Crony,甚至是挂名爲他藏錢的,結果自然有反對黨議員義正詞嚴地要求扣押Oligarch在英國的財產,並且驅逐出境;實際上,正因爲他們是Putin想要整治的對象,才會出逃到倫敦,然後和當地的財閥合流,成爲保守黨的金主,所以Boris Johnson只能支支吾吾地矇混過關。
\subsection*{2022-02-25 22:16}

“結論錯了”,我和大家一樣都很失望,但是正如你所説的,那和“分析錯了”是兩回事。我也同樣復習了自己過去幾個月的推演,並沒有發現任何邏輯上的錯誤,需要修正的是兩個事實假設,亦即Scholz會選擇最優解,和Putin會選擇最優解。目前只能確定前者必然不成立,所以我根據“Putin依舊是孫子兵法定義下的理性玩家”做了討論;但當然也有可能Putin基於戰鬥民族的文化習慣,對使用武力有非理性的先天愛好。既然分析的基本假設前提剛被部分證僞、部分存疑,我想還是先重新觀察一段時間,再重建邏輯架構。
博客的老讀者應該記得,只要是和人類主觀意識相關的議題,要做到100 \% 的預測正確率原本就是人力所不可及的,這是因爲我只有公共訊息,即使邏輯永遠不犯錯,也必然只能做片面的解讀;尤其在研究非理性玩家的時候,更加難以準確預測他們會如何犯蠢;我不是説過,我喜歡從中方的角度來做分析,因爲中共是全世界理性程度最高的執政集團嗎?從博客的教育任務來看,我希望讀者吸收學習的,不是類似收集郵票那樣的零碎結論和教條,更不是網絡論壇的凑熱鬧和情緒發泄,而是源自自然科學的一系列邏輯方法和思路,即使應用起來還是達不到100 \% 正確率,至少它們永遠都是事先預做分析時的最佳選項。
至於有關烏克蘭獲取核武器的説法,我覺得是俄方事後找的藉口,這是因爲烏方打嘴炮是日常;美國也不可能同意;Putin要反應儘可以等到烏克蘭開始籌備再動手,届時反而名正言順。
\subsection*{2022-02-25 04:59}

是的,我認爲如果俄軍長期占領Kiev,扶持傀儡政權,或者試圖大規模清洗烏克蘭西部的右翼納粹份子,都是明顯的Overreach。
話説回來,即使Putin繼續采納理性策略,也沒有保證Scholz能一夕之間多長出腦細胞來。Merkel終究是理科生出身,擁有若干邏輯思維能力,能從經驗中學習正確的教訓;這在現代歐美民選政客之中,算是非常罕見的異類了。

補充一點:我一直認爲出兵應該是用盡外交手段後的不得已選擇,一個重要的原因在於事後難以找到可以服衆、又願意修憲(至少要軍事中立,並且立法禁止納粹)的政府。現在Putin急著動手,軍事勝利是沒有疑問的,但我還在納悶他要從哪兒找來一個可以信任的新烏克蘭總統。台灣不一樣,可以直接當直轄的地方政區處理,就像當年的Chechnya,要寬要嚴、悉聽尊便;烏克蘭卻是有主權的聯合國成員。
\subsection*{2022-02-24 14:42}

你説的其實也是2014年殲滅烏軍前綫部隊的重演。中國人往往反射性地認爲攻心爲上,但正如我們以往對國臺辦策略的討論,攻心和懷柔是兩回事;Putin在2014年出兵,對Merkel就有很强的教育性作用,原版的Minsk協議正出於此。這是分析當前俄方策略考慮時,必須參考的重要前例。
兩天前我說Putin的分寸拿捏有不確定性;現在雖然他一次性地從駐軍、轟擊、直接升級到出兵殲滅對方有生戰力,卻還不足以客觀認定是在戰略上脫離了原先的目標(亦即一方面保護東烏,另一方面對德國施壓,離間歐美),轉爲搶占“實利”。這是因爲那些策略選項博客也都討論過,只不過我偏愛分幾步逐次升級,以便在其間嘗試外交斡旋。Putin選擇跳過外交步驟,不但有軍事作業突然性的需要,而且也可能他所擁有的事實認知,超出外人能看到的公共訊息,尤其是對德國領導階層的評估,是關鍵中的關鍵。
一個多月前我上《八方論壇》的時候,一再談到當時俄方内部並沒有達成必須訴諸武力的結論,反而是英美想方設法要升級衝突。後來也的確是美國安排烏軍發動炮戰,才導致事件急轉直下。但是Putin明明看穿了美方的意圖,爲什麽會自願放棄不戰而屈人之兵的原本路綫呢?請回憶一下,過去幾周,Macron和Draghi曾先後表態,願意背著北約和俄方做和解。同時如果事件繼續升級,損失遠遠最大的其實是德國,美國還主動對德方攤牌,指明要求廢除北溪,Scholz隨後的拒絕代表著德國工商精英已經發聲。所以即使Scholz的智商比Macron/Draghi低一個層級,照理也應該順水推舟,簡單跟隨法意,以Minsk協議為根基來解除整個危機。然而上周Scholz到莫斯科會談,卻是明顯的不歡而散,其後事態純粹因爲冬奧而拖延了幾天,一旦沒有那個顧忌,Putin就一步升級達到、或甚至可能要超越戰略最優解的暴力極限。
所以綜合來看,我們還不能徹底否定以下的可能:亦即Putin依舊是絕對理性,只不過外人看到的是Scholz有點愚鈍,而Putin卻有兩個小時的面對面經驗來確定他是絕對的無可理喻,任何外交折衝的理性利害分析,只會是對牛彈琴;面對不見棺材不掉淚的白癡,反而只有實際行動才有效果。其實德國被美國滲透太深,即使是Merkel在2014年,也是等到俄軍亮劍之後才學乖的;不過現在有那個前例,Scholz的冥頑不靈並沒有太好的藉口。
整個危機還在迅速發展之中,Putin的主觀意志何在,有待繼續觀察。然而我同意你的看法:客觀上這場軍事行動最好是有限的,也就是只求把烏軍打垮,解除烏克蘭未來幾年主動用兵的能力,在占領土地方面只將Donetsk/Luhansk兩個共和國擴展到對應的整個州。要是俄軍長期占領Kiev,就是明顯的次優解:雖然可以重塑烏克蘭的政治結構,但俄方必須承擔社會、經濟和安全等責任,並且在外交宣傳上大幅失分;換句話説,如果爲了軍事考慮而暫時深入烏克蘭内地,也應該在肅清抵抗之後迅速撤離,畢竟就算Putin認爲Scholz連在美方再一次坐視“盟友”被滅的前提下,也沒有膽量履行類似Minsk協議的方案,烏克蘭攤子太爛,承擔起來的代價實在很難Justify;所以勉强合理範圍内的最最最激進方案,是我在上個月討論過的,占領黑海沿岸的俄語區,分裂烏克蘭。
\subsection*{2022-02-23 18:11}

我以前解釋過,俄烏油氣管道過境協議,有效期至2024年届滿;在那之前,北溪純屬備用,只在烏克蘭出事的前提下,才有需要。Putin能夠和美方玩輪流加碼,最大的王牌正是天然氣供應,所以這整個冬天一直拒絕現貨交易,只滿足長期契約的最低要求。派兵進入東烏,管道自然關閉,如果不是運氣不太好、過去三個月歐洲天氣相對溫和,德國經濟應該已經進入緊急狀態了。另一方面,雖然美方的要求是永久性地關閉北溪,Scholtz一直在這個議題上虛與委蛇,其邏輯是想要用庫存天然氣熬過這一波烏克蘭衝突,坐等春天到來;反正北溪審批還可以“暫停”兩年,届時公衆記憶早已淡忘。這正是前幾天我才剛討論過的,普羅大衆沒有記性的特點,是民選制政客的天賜禮物,庸碌之輩自然充分利用,並且養成習慣;這個現象在歐洲政壇尤其明顯。
這篇正文的重要建議都已經被采納,我預期外交重點向法國傾斜也不例外。
你來博客,還只有五個多月,照規則字面解釋的話,這條留言應該直接刪除;考慮你在新規出臺之前已經發問多次,法外施仁,下不爲例。請等幾周,賬戶滿六個月歷史,再做發言。
\subsection*{2022-02-22 15:11}

俄方的戰略目標是什麽?局部上是要維持東烏人民的安全和自治,整體上則要離間歐美,讓德法同意停止北約東擴。英美呢?則是迫使德國放棄北溪二號,重新成爲溫順待宰的外交經濟附庸。爲了逼迫歐盟和俄國對立,可以簡單預見美國會要求烏克蘭在軍事上不斷做出有限的升級挑釁(但不是全面開戰),所以幾天前開始的炮戰,是Putin必然早已做好預案準備的進展。
Putin的預案有哪些選擇呢?這次美俄對弈本質是外交宣傳上的鬥爭,直接揮兵攻入烏克蘭本土是下下策,但見死不救也不見容於國内輿論,因此至少必須做出我一向對中方建議的“對等反擊”。;再加上俄國已經在過去八年做好準備,不懼任何經濟外交上的制裁,反而是歐洲尤其德國會承受絕大部分的損失,所以俄方從衝突一開始就公開實踐“Strategy of Tension”,這可以解釋成“對等升級”策略;用撲克術語來説,就是不止“跟注”“Call”,而且等量“加注”“Raise”。博客留言欄兩周前討論過用遠程炮兵對烏軍陣地做嚴重殺傷,但那是在烏方發動大規模攻勢下的保底手段;現在的炮戰還只在騷擾的層次,Putin必須留下餘地,爭取更多外交折衝時間,所以目前只選擇了相對溫和、人道的步驟,派兵進入東烏做肉盾;然而要名正言順地駐軍,就必須先外交承認那兩個共和國。
如果我們檢驗最近兩周的諸般局勢發展,就會發現Putin的策略不但合理,而且是必然,只有在分寸的拿捏上有一點不確定性。這裏的關鍵是歐盟的態度:Macron很積極地想要促成歐俄和解,但並沒有什麽實質進展,原因是德方在扯後腿,不敢直接違逆美國。然而美方要求的,公開承諾放棄北溪,又是德國經濟不可承受之重,所以Scholtz拖拖拉拉,想要靠私底下承諾來矇混過關,這才有了美國下令Zelensky發動炮戰的後續。Scholtz上周訪問莫斯科時,Putin已經失去耐心,在聯合記者會上對其公然斥責,是這一波情勢惡化的前兆;後來事件又多拖了幾天,以慢動作發展,可能是美俄考慮到冬奧,不想在宣傳上失分的結果。現在冬奧完美收官,雙方都沒了顧忌,德方的拖延策略顯然玩不下去了。
我從一年多前Merkel正式確定退休就反復預言,這會導致歐盟的決策真空,從而誘使美國在東歐製造事端。這次的烏克蘭事件始自去年十月,正是德國大選後籌組新政府的無主時段,這不是巧合;即使一開始有其他的導火綫,其後美國之所以肆意加碼、步步升級,其最重要的動力和目標,必然是要利用Merkel的去職來扭轉北溪二號問題上德國的外交反抗。Scholtz拒絕放棄北溪,原本還讓人有所期望,以爲他能繼承Merkel的長期戰略,主動爭取外交獨立,但最近的發展,證明他完全沒有那個能力、眼光和企圖,純粹只是被幕後的德國工商精英嚴重警告,所以從短期經濟損失的角度來做的政策決定。我在整個事件的過程中,反復强調關鍵在於德國的選擇;對昏庸無主見的德國執政集團加壓,再一次是這一輪美俄先後升級衝突的基本考慮,也是當前無法理性預測勝負結局的原因。
\subsection*{2022-02-20 02:10}

我剛剛增修了《讀者須知》,請大家移步瞭解新近詳列的規則,特別注意第6和第9條。
我對匿名大V的瞎猜胡扯,向來沒有興趣;熟悉博客原則的老讀者還喜歡去看,已經是對事實和邏輯的不尊重,轉述在此,更加是浪費空間和時間。例如這裏,烏克蘭國内狂徒的妄想,在國際局勢現實壓力之下,毫無意義;一旦大國達成決議,内部反對、抗議到怎麽亂都沒有用。這麽簡單的邏輯,只要想想如果美國不撐腰,臺獨能多硬就知道了;或者參考Armenia最近戰敗後割地認輸的慘狀,以及尤其是當年希臘被逼大幅削減預算的故事,我還寫了不止一篇博文來討論。
嚴重警告一次,禁言一個月,再犯拉黑。

過去幾個月,我沒有嚴格執行規則紀律,結果留言欄發問越來越不像話,我越是多花時間精力來彌補對話品質,違規的人越是習以爲常,把我的客氣(尤其對老讀者)當作鼓勵;你沒注意上次我雖然說“正解”,實際上是做了重要的更正嗎?
這些匿名大V,好的是自娛娛人,差的是自愚愚人,但同樣都不會對事實和邏輯保有任何責任心,所以原版的《讀者須知》第6條就已經禁止引述。我寫博客表面上也是公共論壇,但對我自己是做學問,對國家社會是做建言,對讀者大衆是做教育,每一字一句都要對三者同時負責。你反復拿不入流的言論來博客轉述,輕佻心態表露無遺,對我和在乎事實邏輯的讀者都是極大的侮辱。博客討論的是如何治國平天下,讀者就算沒興趣,至少先試圖格物致知、誠意正心吧。
所以大家請注意,若是有人喜歡到互聯網上的信息糞坑裏玩耍,我不可能禁止,但這個博客不是讓人散心玩樂的地方,入門之前必須先把心態和認知洗乾净。我最近被熏得頭暈腦脹,不能再容忍了。
\subsection*{2022-02-17 06:30}

我和Stiglitz並沒有意見上的分歧,純粹只是談不同的話題:我談的是客觀上“最可能的未來發展”,而Stiglitz談的卻是在《政府的第一要務》是扶貧的前提下,主觀上美國“最應該采取的政策”。
普羅大衆對公共人物的言談毫無記憶,是我多年前就指出的大毛病;這不但方便詐騙集團和學術娼妓事後遮掩,更是讓平庸之輩得以兩面下注、矇混過關的基本因素。只要稍稍用心去追究早先的原版言論,就會發現99.9999 \% 的學者專家和名人權威,對未來事件的事先預測正確率,如果是獨自則不高於隨機,如果是跟風則不高於流行。但即使證據確鑿,愚蠢的公衆一樣會找“戰略忽悠”這類的藉口,或者徑行忽略忘卻錯誤的細節;神棍和騙子之所以能吃香喝辣,靠的正是人性中先天常見的這些非理性弱點。
\subsection*{2022-02-16 11:28}

經濟和金融由許許多多利益對立的玩家組成,一般是高維度的博弈論,不能只做初級的綫性擾動分析。你所談的美元利率上漲超過實體產業投資報酬率,就只是初級效應,一旦廠商無利可圖,首先淘汰弱者,接著必然導致報價上升,反饋成爲新的通膨壓力。
利率和通膨有很強的交互作用;美聯儲利率會提升多高,取決於控制通膨的需要;但是美國已經不再生產許多民生消費品,提升利率固然會壓縮大筆開支,但基本消費沒有什麽彈性,一旦因上一段落所討論的機制而持續漲價,美國國内的底層員工生活水平就會兩頭受壓,逼迫聯邦政府增加緊急福利支出,從而擴大赤字,讓美聯儲在利率決定上陷於兩難。這其實是我以前反復討論過的美國經濟、金融和財政胡搞40多年的惡果,説得稍微詳細一點罷了;你的論述只是這個邏輯鏈的一個小環節,對博客的主軸議題來説沒有什麽重要意義,爲了賦予把你留言保存在博客的價值,我已經很耐心地一連詳細回應了三次,請你暫時收斂,不要占用太多時間。
至於美國企業界的垃圾債券,這倒是值得討論的議題,不過答案也很簡單:總額不到2萬億美元,其中一半是過去兩年内發行的,所以還不會很快到期。雖然SEC打壓SPAC為股市泡沫提前泄了一點氣,使得債市成為當前的焦點之一,但既然短期内會到期的只是1萬億美元,對美國經濟體量的占比實在不大,只靠它一個泡沫並不足以引發連鎖反應,所以即使金融危機表面上從此開始,也不代表它是危機的真正源頭;美國的財政金融困境,是40多年來全方位腐敗衰弱的結果。
\subsection*{2022-02-16 07:55}

美國利率在過去40年波動過程中,一波比一波低的現象,其實主要反映的是通膨壓力的消失;而通膨之所以消失,是工業外包、導致中國廉價高效的勞動力取代美國工人的結果;參見博客既往的評論。Trump針對中國所設立的關稅等等貿易壁壘,固然遏止了中低產業向中方轉移的進程,但It's too little, too late,而且反而帶有加劇通膨的副作用(對這一點我也早就預期、並且反復强烈批判過Biden的不關注態度)。所以未來兩三年,美聯儲加息必然會超過2018年的尖峰,甚至2007年、1999年的記錄都有可能被打破,但並不保證美國經濟100 \% 會摔下斷崖:這裏的不確定因素,除了美國自己之外,中、俄、歐的貨幣和貿易政策都會有影響;這也正是我寫這篇正文的主旨。
\subsection*{2022-02-15 17:10}

在冷戰期間,美元並不是霸權的支柱,反而反過來,必須靠其他方面的美國霸權力量來强迫歐日持續接受美元的搜刮。美國在1990年前後,擊敗蘇聯、日本兩個挑戰者,都是靠文化、宣傳、經濟、軍事的全面壓力,誘導並脅迫對手采納自殺性政策而成功地從内部瓦解他們;美元匯率的操弄,是整個過程的通道,而不是壓力的原始來源。冷戰後初期,美國的政治軍事霸權更進一步踏上頂峰,美元得以超越以往受害的其他工業國,轉而搜刮新興國家,才有了1997年的國際金融危機。其後全世界爲了避免重蹈覆轍,普遍開始纍積超量的美元外匯儲備。
其實在Nixon打破Bretton Woods體系後的1970、80年代,歐日就已經爲了試圖減輕美元通漲的搜刮力道,而努力改用自己的貨幣,到1990年,美元的國際額分已經降到40 \% 。1990年日本的經濟崩潰和1997年的亞洲金融危機,徹底反轉了這個趨勢,到2000年,美元額分漲回到72 \% ,然後歐元的出現和2008年美國金融危機又使其轉爲下降。然而2010年起始的歐元危機,卻大幅減緩了這個進程,所以經過20多年,美元額分也只降到59 \% 。
多年前我在《美元的金融霸權》系列,向華語世界解釋清楚美元一放一收的周期性搜刮機制,這個波動周期對應著美國國内經濟的起伏,大約是5-10年。然而正如AM無綫電的高頻載波承載低頻信號,在美元搜刮過程中,也有一個更長周期的波動,對應著美元的國際額分,這個低頻周期大約是30-40年。當美元額分處於高點,搜刮的重點在於“放”,亦即印鈔通漲;等其他國家反應過來,開始改用其他貨幣,就必須把重點放在“收”,力求以金融結合政治軍事手段,製造各式各樣的國際危機,反過來逼迫世界建立高額外匯儲備,保障美元的地位。
所以單從貨幣角度來看,美國目前的處境,類似1979年美元超發再加上供給鏈問題(當時是第二次石油危機,現在是COVID)所引發的嚴重通漲;我認爲不論美聯儲主席由誰來當,拿著當年Volcker留下來的教科書案例,都會選擇專注在收緊銀根,所以未來兩年經濟蕭條的機率很大。新興國家雖然總體的外匯儲備有了顯著提升,但因爲40多年來工業經濟的普及和發展,可供壓榨的對象也大幅增加,其中金融體質較弱的,如土耳其、南非、巴西和墨西哥,依舊足以讓美國大量的吸血。
中國的應對之道,在於一方面聯合金融體質較强的所有非昂撒工業國,以高於當年歐日的速度進行去美元化(這是最近幾年博客評論,包括這篇正文,的核心建議),另一方面在美國成功引發弱國經濟崩潰之後,出手截胡,阻斷美國財團賤價兜底收買優質民族資產的企圖。這裏最理想的手段,是有自己的國際金融穩定機構,例如亞投行;問題在於正確的運作,是對國際金融危機的既有仲裁機構International Monetary Fund做替代或至少競爭,而當前的管理人卻把亞投行搞成專責扶貧的World Bank的低級代用品,拿國家的錢在金融戰綫的和平時期就當凱子亂花掉了。我在多年前已經反復批評過這個錯誤,現在事到臨頭,來不及再改,所以正文中也就沒有提及。緊急機制和其他管道也還是有,但可以等這一輪經濟危機態勢明朗化之後,再做針對性的建言。
\subsection*{2022-02-14 20:21}

烏克蘭和台灣的對比,乍看之下都同樣是心甘情願為美國赴湯蹈火的第一綫炮灰,但這裏的差別,在於前者對應的第二綫炮灰(德國)已經在2014年學乖,並且花了過去8年積極佈局(Nord Stream II)來擺脫鉗制,而後者的第二綫(日本)卻沒有一點覺悟。還好,正因爲德國在反抗,所以美國的注意力不得不集中到歐洲去。至於日本,它比歐盟更虛弱,而中國比俄國更强大,這才是美方在東亞早已準備好第一和第二綫炮灰,卻不敢如烏克蘭那樣迅速升級事端的原因。中方的解決方案,正文已經詳細給出了:短期内雖然成功機率不大,依舊可以試圖和岸田和解,緩和局勢;中長期來看,日本的國力即將摔下斷崖,中國則繼續上升,威脅只會隨時間而自動消失。
\subsection*{2022-02-12 23:08}

前面的回復有點簡略,可能引起誤解,我在此澄清一下:為大選而裝腔作勢,指的是高調到莫斯科和基輔轉一圈。實際上,真正的重要外交還是有的,而且Macron的確是核心人物之一,只不過溝通用電話就夠了。
當前的僵局是:俄國不會打,除非烏克蘭先動手;烏克蘭不會動手,除非美國下令;美國不會下令,除非德國承諾制裁俄國;德國不會承諾制裁,除非被逼無奈。所以一邊是英美在拼命造勢加壓,另一邊則是德法意急著要以最不得罪美方的形式來和俄方和解。因爲這必須推翻二戰後以來77年的内政慣例和國際格局,是真正艱巨的工作,而德國新總理剛上任、還未站穩脚跟,Macron的確是領頭人物,邀功非常合理。可以譏嘲的是,他不能只管辦好正事,還必須出場作秀;但這不是他的錯,而是民選體制下外行選民所逼迫出來的,請讀者理解清楚我批判的對象。
至於Biden政權的操作,目前正反面的敘事都有許多證據。照理説,歐盟擺脫昂撒集團桎梏,會是極爲惡劣的損失,不值得冒險;即使在國内愚民汎濫的背景下,理性玩家也應該趕緊一方面公開施用障眼法、一方面私下設法收官。但美方自相矛盾、不斷搖擺的言論,顯示他們不到黃河心不死,一脚踩刹車、一脚踩油門,至今仍在對德方加壓,可能是認爲自己可以吃乾抹净,在歐俄協議出臺的前一天再出手阻止;這正是正文中所討論的貪婪心態和素質問題。

的確是正解。前幾天沒人問,我心裏還想,讀者經過過去幾個月的解釋,已經能夠自行對新發展做出正確的分析和結論,因而為自己的教學能力頗感自豪。
此外,Macron本身也是攔路搶功的截胡手,因爲更早幾天,匈牙利的Orban才是第一個到莫斯科“斡旋”的歐盟國家首腦。很巧的是,他也是四月有大選。
自兩周前起,大陸的公共論壇忽然轉向,全面理解整個烏克蘭“危機”來自英美的炒作,似乎正是因爲博客讀者群包含不少意見領袖。
\subsection*{2022-02-03 22:33}

韓國社會對美國的崇拜和溫順,更勝德國,而且沒有歐盟和歐元的屏障,即使從理性利益角度,外交内政的頭號考慮也必須是可能來自美國的懲罰。然而它地理臨近中國,對中方的經濟依賴和軍事糾葛也同樣更勝德國,所以文在寅的平衡外交已經是最優策略,如果執政黨候選人繼任總統,應該會延續在中美之間尋求平衡的原則,這是中國可以接受的。反之,如果反對派上臺,那麽有可能再度出現類似2016年THAAD事件的外交挑釁,必須如同對待澳洲和立陶宛那樣,予以嚴懲。
中國對經濟制裁,始終保持曖昧態度,只做不説。這在過去韜光養晦的階段,需要依靠國際自由經貿體系來培養國力,是合理的政策;然而在現今昂撒集團公然發動冷戰2.0的背景下,還堅持同樣的手段,就是自縛手脚,同時徒然賦予外宣抹黑的藉口。再考慮未來中國在全球外交和經貿的地位,只會隨時間繼續上升,正確的做法顯然是模仿美國,公開立法,堂堂正正、系統化地以“國安”理由對外交、軍事的威脅做經濟制裁。因爲美國早已在國際法體系中開拓好這條道路,反而更加名正言順。
\subsection*{2022-02-03 03:09}

你所說的“陽謀”,就是孫子兵法裏的“未戰而廟算勝”、“上兵伐謀”、“不戰而屈人之兵”、“善戰者先為不可勝,以待敵之可勝”等一系列原則。這不但是博客做戰略分析和建議的慣常思路,Putin也剛剛做了完美的示範:準備萬全,所以能夠對美烏的挑釁見機加碼、借力打力,連自己的戰略意圖(遏止北約東進)和手段(Strategy of Tension)都無需遮掩,可以以堂堂正正之師公開宣佈。相對的,中方始終幻想或假裝仍能和美國維持“友好”關係,反而自縛手脚,無法在戰略機遇下做出足夠迅捷强力的反應。
至於Biden和Blinken,原本俄方謀定而後動,美方的最佳回應只能是大事化小、小事化無、私下軟服、做出承諾(而且一開始趨避得越快,後來所須做的退讓就越少),至少保住北約團結的表象和世界霸主的面子,結果幾番騷操作之下,居然反過來聽任英方主動升級加碼,把歐盟核心逼到準備與昂撒集團公然決裂,這是Putin自己都不敢事先奢望的完勝結局。這個過程中最離譜的是,美方幾個月前才在AUKUS一事上吃過Johnson的苦頭,還不學乖,其魯鈍愚昧的程度,確實僅見於已被民選體制高度腐化的反智社會,例如台灣。

俄方和法意的溝通在3日一公佈,美國的態度就立刻軟化,不但白宮急著改口號稱“俄軍入侵烏克蘭不再緊迫”,國務院也願意和歐洲核心協調對俄策略了;這是典型的不見棺材不掉淚。
\subsection*{2022-02-01 11:33}

對一般公共議題做預言大致分爲兩大類:二選一和N選一,這裏的N往往是20左右。對後者能做到50 \% 勝率,其實比對前者做到90 \% 還要難,參見我以前詳細定義、解釋過的log(win/lose)指標。我個人的目標是對後者也做到90 \% 以上,共勉之。
我對Merkel的評論,如同博客歷來千、萬個分析一樣,是純粹理性的論證,所以隨著新實驗、新觀察、新角度和其他相關議題上新知識的出現,有所增補修正,是科學方法的天然内建過程,大家熟悉的例子不可勝數,像是廣義相對論中的一個項Cosmological Constant是否爲零,在Einstein生前就有反復,至今也不能說蓋棺論定,但這並不代表百多年前原版的相對論有什麽錯誤。
當年我說Merkel把德國搞得外强中乾,指的是内政和經濟上的改革,今日依然有效。德國在她任内,始終沒有引領新興產業,基本是吃機械方面的老本,連汽車電動化都只有VW一家勉强跟上,核心的鋰電池技術也完全沒有掌握;雖然正文專注在日本即將面臨的問題,其實德國也只是多活幾年罷了;這個慢性衰退的政治責任最終是Merkel的。
\subsection*{2022-02-01 00:25}

這種内參當然不會公佈,但通過簡單邏輯分析,一樣可以達到基本100 \% 的確定性:1)他們有沒有做這個研究?自己說有,所有報導都證實有,常理也必然是有,所以可疑性是零;2)他們有沒有找到有效的手段?美國官方一直更換説法,提起一個制裁手段,幾天後又不見了,周而復始,來自歐洲的報導也是一系列的德法否決,客觀上更是可以確定沒有有效的制裁手段,所以可疑性也是零。(1)+(2)自然達成我說的結論。
要擔心科技禁運對俄國有影響,必須先假設Putin比Blinken笨得多,花7年時間還沒有想到後者7周就搞出來的花樣,並且對俄國近年的工業應用研發毫無所知(有空去搜索一下“PD-14”是什麽)。至於電子產品,俄方的確很難自行開發替代,但這只是簡單從歐洲和韓國供應商轉向中國的事,反而幫助中方被制裁的企業擴大市場,對美國得不償失,只不過多繞一層邏輯,方便騙騙無知的老百姓,可以作爲下臺階罷了。
這個博客是理性知識份子的論壇,留言發問者對人類社會運作規律全無所知的話,每條論證的幾十、幾百個基本細節步驟都可以在這裏囉嗦,不是我在精力上能負擔的,對絕大部分讀者也純屬時間上的浪費。你雖然有博士學位,這些討論超越你的專業範疇;請沉潛至少一年,反復閲讀博文、用心深入思考,自覺有所成之後,再來發問。
\subsection*{2022-01-31 22:15}

1.NGO無關緊要;學術管理的不作爲,有其深刻複雜的歷史背景和理論原因,我已經反復論證過了。
2.把去美元化和推動人民幣,混爲一談,的確很可能是引發不作爲的基本誤區。我一直懷疑(當客觀分析指出反復的低級錯誤,所以非蠢即壞的時候,我傾向於假設非專業人員是前者、專業人員是後者)是金融方面的主管人員利用專業暴政(Tyranny of Expertise)來欺瞞最高層,亦即當後者要求有動作時,他們故意把資源和時間浪費在提升人民幣額分的無用功(“無用”指在相關時段内)之上,例如數字化人民幣,以逃避采納直接打擊美元的有效手段。然而所有的金融工具,都有極强的網絡效應,亦即既有市場額分越高,交易摩擦(Friction of trades)就越低,所以試圖拿佔2 \% 市場的人民幣,單挑佔60 \% 的美元,已經是不自量力,數字化貨幣更加是全新的概念,指望全球金融系統迅速采納,毫無成功可能。去美元化是未來幾個月國安上的頭號緊急要事(我從七年多前就開始公開高調地倡議,現在已經被那些人拖到最後關頭,無可再拖),中方反而重點投入需要十年左右時間的長期發展,不但緩不濟急,而且已經構成資敵的賣國行爲。我雖然預期這些謬誤,在正文中再次指明正道,結果卻馬上聽説有部門又重施故技,把上峰的意志斗轉星移,轉投到不可能在一年内有任何結果的方向上去。這其實和假未來科技異曲同工,偏偏我是外人,即使事先把道理解釋清楚,内賊事後要在細節上做扭曲,我也沒有管道置喙。
3.這事對中國崛起來説,還算是相對的小事,該説的我也早都説過了,台灣内部鉗制實話的新政策,更加不方便我繼續評論,請不要在此發泄情緒。
\subsection*{2022-01-29 02:01}

短期的原因是危機若搞大了,美國卻不敢也不能出兵,世界霸主的面子再次挂不住;英國可沒有這個顧慮。長期的原因則是英國媒體幾個世紀以來一直都只是Partisan黨派鬥爭的工具,撒謊造謠、鼓動暴民是幾百年的核心任務,不像美國在20世紀中期進行過深刻的反思,出過Walter Lippmann和Edward Murrow這類人物,普及過“超越黨派、堅持事實”的理想(不論是否真的做到,至少表面上大家同意,而且在大學新聞系一本正經地如此教育學生),背景裏的整個國家社會更是曾經相信過羅斯福主義(亦即公益必須由政府和好人挺身而出來維持)而不是絕對放任、自私自利的雷根主義(其實就是社會達爾文主義)。換句話説,美國新聞業的腐化是7、80年代財閥奪權、有意腐蝕人民思想的一部分,相對晚近,英國傳媒界卻從來沒有健康過(所以香港的思想文化,中毒深入骨髓臟腑,原本就無可救藥,現行的以政治手段强力維穩是唯一的可行方案)。
指桑駡槐、拐彎子駡人是我的老毛病之一;雖然昨天字面上似乎只批評了台灣媒體,我原本就希望聰明的大陸讀者能自行引申,注意到大陸和台灣社會50步和100步的對比。畢竟我早就指出過,海峽兩岸公共輿論的品質差距,根本原因不在於人民素質,而來自政治體制的差別。
\subsection*{2022-01-26 15:01}

可笑的是,一旦英美劫持了這個“俄軍即將侵略”的敘事,並且爲了自己國内的聽衆而繼續加成,導致烏克蘭民衆和經濟發生恐慌,Zelensky本身反而必須趕快跳出來闢謠,公開承認俄軍在過去幾個月根本就沒有增兵。當然,不論烏方如何着急,政府説了多少次實話,只看英美國際媒體的聽衆是絕對聼不到的。
照理説,如果一個人懶得去主動發掘事實真相,只看方便的昂撒主流媒體,那麽自然沒有資格對國際事件做公開評論;沒有努力就沒有發言權,這是求真的最基本修養。但是台灣輿論界,表面上熱鬧非凡,卻沒有一個人具備這個修養;這裏的始作俑者當然是李登輝引進了英美財閥用來愚化民衆的説辭,那套“民主多元”、“自由思想”的胡扯蛋,畢竟不但“民享”和“民選”有先天的矛盾,“事實”和“多元”、“邏輯”和“自由”更加是不兼容的。這是因爲一般人的認知理解雖然是片面、偏頗、多樣、互相矛盾的,真相本身卻是唯一、完整、自洽而且絕對的,所以如果這個世界看來充滿未知或矛盾,那是個人的程度問題;當然,術業有專攻,一般人不懂政治經濟外交軍事並沒有錯,但若認定自己的主觀認知等同客觀的事實真相就升級成爲傲慢和愚蠢。台灣人面對這樣的誠實批判,往往還會回應“喝過台灣水、吃過台灣米,怎麽能說這種話”,這是把團體中部分個人的傲慢和愚蠢趨勢,升級擴展到整個族群所有成員資格的層次,已經是愚蠢的平方,而且自動引發“只有説謊者才算是台灣人”的邏輯結論,等同極度侮辱“台灣”這個字眼,居然還毫無知覺,在博客也反復有人敢用上這個説法,説是愚蠢的立方都還太客氣了。
\subsection*{2022-01-25 12:43}

你還記得4、5年前我說對英美外宣的抹黑必須直面以對嗎?結果到了2020年,外交部終於挺直脊梁,囘對新冠來源問題卻用上了陰謀論。我只建議正確的原則方向,尚且在執行上會出現那樣的偏差;在處理兩岸折衝的細節上,拿捏必須絕對精準,我一個局外人,不論怎麽説,都必然會被扭曲,所以不談是有意的選擇。
中俄在國際戰略上,原本就在互補長短,要和印度交涉,自然有俄國去惹那身騷,中方繼續保持距離就行了。
\section*{【邏輯】一個重要的統計悖論}
\subsection*{2022-10-27 16:35}

有關這類話題,能100 \% 確認正確的結論只有“生醫研究很難做”一句話。最近兩年的行業共識,的確是Serotonin在腦中的濃度不是抑鬱症的直接因素,但(1)請注意“直接”兩字;(2)這個共識只有70-90 \% 的確定性。不論如何,SSRIs確實對至少部分患者(尤其是重症)有效,所以但凡有自殺意圖或明顯生理症狀的人,不應該排斥醫生開處方藥。
另一個正在被翻案的舊共識,是Alzheimer's和Amyloid  Beta的關係。20多年前的幾個研究確定兩者有極强的統計相關關係之後,全球頂尖藥廠都自動假設後者是前者的因,投入幾百億美金來控制Amyloid Beta,結果全都打了水漂。最近幾年的共識轉爲後者可能是病徵而不是病因,然後幾個月前出現一款針對Amyloid的新藥,早期實驗結果非常正面,這麽一來,大家又只好存疑了。
其實生醫和社科一樣,太過複雜,所以很難得到簡單明確的結論。我認爲正因如此,更須要堅持科學原則,强調嚴謹的分析和實驗方法,一方面理解機率規律、不怕結論錯誤,另一方面追求邏輯的絕對嚴密。博客八年來所做的,正是這樣的示範。
\subsection*{2022-04-27 12:25}

因爲世界極其複雜,博客專注在解釋其幕後的深刻道理,所以目標讀者被定爲有理性的知識份子,直接面對普羅大衆的一般科普,我通常並不涉獵。然而這並不代表它沒有價值,事實上把正確的事實介紹給最大數目的民衆,顯然是非常有意義的事;考慮到絕大多數的所謂“科普作家”和網紅大V都以欺詐群衆、收割韭菜為真實目的(例如袁嵐峰),和詐騙份子對著幹的人的確值得尊敬。
在蓮花清瘟這件事上,理性知識分子只要考慮過去百年來,有幾千萬名中醫,發表過幾百萬篇文章,其中有至少幾萬篇是以科學論文的格式寫的,卻沒有一篇是對蓮花清瘟這類東西做出成功的雙盲實驗,就自然知道沒有道理會在2021年才出現例外;如果真出現例外,科學界的反應也必然會是極度熱烈,而不是只在非專業媒體轉發。當然這樣的邏輯推理,對信仰中醫教的無腦民衆來説是太過奢求,所以“睡前消息”能發掘謊言詐騙的確實來源和脈絡,雖然一樣無法扭轉蠢人的信念,至少讓理性人批評起來更有力些。
\subsection*{2020-07-16 13:23}

我討論日韓是因爲他們比你用的韓美對比要合理的多;你現在又回頭用兩個小城市來作比較,人口和GDP都不對。我連這種分析根本不入流都剛剛説過,你反過來倒打一耙,真是讓人無語。
我也已經說過許多次,中醫正是屬於那種惡劣結果不能簡單馬上看出的類別;正文還特別討論爲什麽統計難做。至於毒性測試,被盯著看的時候當然可以過得了,但是量產時就挂羊頭賣狗肉的事還少了嗎?就算真沒有毒性,什麽時候無效的治療變成合理的了?
我從未抱怨過中醫“浪費”國家經費。你再仔細讀讀,我之所以提過中醫界有經費,純粹是爲了論證錢不是不做雙盲實驗的藉口。幾億的經費和腐蝕民心以及謀財害命比起來,根本不值一提。
這個博客不是任何人可以撒潑的地方。剛好你志願示範僞科學/反科學的信徒是如何無法理喻,我就讓你出醜賣乖。不過我的時間寶貴,不能陪你死纏爛打;反正你已經決心一輩子留在賊船上,這個博客非常不適合你。我只希望你少對中醫教未來危害那許多人命出力。
\subsection*{2020-07-16 09:02}

正文裏面反復强調,必須先控制主要因素,結果你還是拿基因和飲食習慣完全不同的國家來做比較;果然是反科學本色,事實和邏輯擺在眼前,說不管就不管。
日本和韓國血緣接近、富裕程度類似、生活飲食基本相同、醫療水準相當,但是前者的人均壽命多1.5年,對應著1.8 \% 。韓國人口5100萬,1.8 \% 是93萬條人命,這還只是一個中型國家。當然這種簡單的比較很不精確,我自己是不會主動使用的,這裏只是最小程度修正你的錯誤;不過至少可以看出我原本估算的數量級並不離譜。
中醫教如果依托中國國力而撒佈全球,最危險的是亞非那30億貧窮人口,他們沒有可靠的現代醫療體系,那些可以亂加莫名其妙成分的“中藥”可就不只是像在韓國或中國這樣在陰影下偷偷吸血,而是能夠大開殺戒了。
\subsection*{2020-07-15 20:32}

要指望中醫界采納科學方法和原則來提升水平,首先他們必須改變心態、願意試圖證僞自己的假設,但是中國中醫界的既得利益階層規模太大了,比實質内涵能支持的高出三四個數量級,所以任何科學化、真實化的企圖,都等同消滅99.9 \% 的道友,就算有人有良心、有見識想這麽做,也必然會被多數人立刻打壓噤聲。
我並不指望消滅中醫教;這個世界蠢人太多,連法輪功都能越做越大,何況其他?但是由政府來强制要求任何醫療咨詢和藥物販售必須達到同樣的現代醫學所確立的安全性和有效性標準,完全是合理可行的。退而求其次,至少社會上有理性的批評聲浪,那麽或許這些騙子要消費中國的聲譽到世界其他國家騙錢的時候,被害者有機會事先獲得警告;只要能減少1 \% 的被害者,就是救活幾萬條人命,這樣的功德,捨我其誰?
\subsection*{2020-07-13 18:00}

利用切身經驗來糊弄人是一個已知的心理效應,早被詐騙集團廣汎應用。例如1930年代一本介紹華爾街零售生意的書就提到,股票分析師會在第一周隨機選定A股,然後向一半股民發大批垃圾郵件說A股會漲,另一半郵件則說會跌,事後針對猜對的那一半再切分成兩半,分別預測B股會漲還是跌;如此重複十遍,假設原本有100000股民,那麽大約100000/2\^10 ~ 100人會連續收到10次精準的預言,這時就可以向他們直接販售自己操弄的股票,Take them to the cleaners。
在中醫界,除了業者有意的選擇性記憶之外,一般未受過嚴格科學訓練的民衆,也會自然趨向搜集正面證據,這在心理學上叫做“Confirmation Bias”,是宗教的天然起因之一;有興趣的讀者應該自行去瞭解細節。我一直說中國的中醫界已經成爲一個宗教,是有很强的邏輯根據的。
\subsection*{2020-07-12 00:22}

Simpson's Paradox比你想的更複雜、深刻,它的根源在於統計只告訴你相關性(Correlation),我們真正有興趣的因果關係(Causation)卻必須依靠其他的考慮來確定。一般被研究的社會現象有著許許多多的維度,到底哪一個是因、哪一個是果,不能簡單論斷;即使把所有的變數依照Correlation的强弱一一列出,也只是減小搞錯因果關係的機率,事實上完全可能有某一個與最終結果無直接因果關係的變數,通過與幾個真正因子的聯動而獲得最大的Correlation。不過這已經考慮得太深入了;近年一般社會科學的論文基本上都是只搜集關於自己有興趣的單一變數的資料,然後做個簡單的綫性回歸(Linear Regression)分析,這樣天真的研究幾乎必然是錯的。一個典型的例子是三個月前《經濟學人》拿新冠死亡率和“民主程度”來做統計,發現有正向的相關性,然後斷言民主自由有利於防疫,就是忽略了富裕程度這個主變數,所以得到是非顛倒的答案。可笑的是如果他們本月再重做一次,連富裕程度都無須考慮,美國、巴西、印度等“民主國家”一樣名列前茅;所以他們必然不會重做,做了也不會登出來。
科學不是題材,而是態度;詳細來説,是人類總結用來求真最有效的一系列方法和原則。現在的中醫學院拿做生意廣告的態度來搞中醫,那自然是成爲宗教而不是科學。
\subsection*{2020-07-12 00:12}

科學是求真的,即使沒有理論基礎,如果有確實無疑的實驗證據,依舊可以接受為真理,慢慢再修改理論來解釋這些現象。
中醫的理論當然是胡説八道,但是若干偏方歪打正着的可能還是有的,像是青蒿素就是一個例子。這裏的問題是屠呦呦是科學家,她拿到青蒿並不是吹噓其歷史和實驗效果,而是開始分析其成分,然後確定作用原理;然而現在中國學界那些人根本就不做這些工作,拿了蓮花清瘟做個很小規模的實驗,碰出了一個似乎是正面的結果,就宣告勝利了。居然還有人問我,知道有效成分有什麽用?答案就是這篇正文:統計太容易給出錯誤結果,屠呦呦進一步找有效成分和作用機制,其科學上的意義就在於消除統計噪音和誤解的可能,確定結論的正確性。像是中醫學院不去進一步分析蓮花清瘟,生怕否定了這個偶然撞出來的正面結果,這哪裏是做科學的態度,純粹就是騙吃騙喝的混子。
\subsection*{2020-07-11 17:17}

所謂的“真”中醫,也只是自欺欺人的又一藉口。他們和“假”中醫唯一的差別,在於他們不是存心詐騙,所以他們是蠢而不是壞。我對中醫教的其他批評,仍然完全適用。嚴格來説,唯一可以算是“真正醫學”的中醫,就是屠呦呦,但既然青蒿素是現代醫學的一部分,你就不用找任何中醫來抓藥。
不能100 \% 搞清楚,不是把1 \% (蓮花清瘟)當作99 \% 的藉口,更不是把0.01 \% 吹噓成國粹的理由,否則犧牲的是許多人命、環境、社會風氣和國家民族本身。
我以前説過,邏輯上上帝隨時有可能召開記者會證明他自己的存在,但這不成爲信教的理由。我還介紹過Russell's Teapot。一個老讀者還説出這樣的話,你得要好好自我反省;理性批判的對象應該從自己開始,如果你只能看到別人的誤區,那麽你並沒有學到什麽東西。
\subsection*{2020-07-11 16:19}

Epigenetics是真的科學,但它依舊是遺傳學的一部分,遵守物競天擇的演化規律,而且從來沒有和高階的行爲模式有關係,這是因爲1)許多高階行爲是現代社會的產物,在演化歷史上並不存在,也就沒有理由引發遺傳壓力;2)心智是一種湧現現象,人腦的設計原本就是以靈活可調為優先,演化壓力反過來層層下傳到分子級別,必然所剩無幾。
中醫教的危害之大,除了每年成千上萬的人命、以及對生態和物種的不必要摧殘之外,還有消弭理性、腐蝕人心的作用;後者雖然間接,但是後果其實是最嚴重的。台灣到處敬拜鬼神,原本就是發展民粹的沃壤,李登輝上臺之後,把迷信也擡舉上臺面,成爲選舉文化的重要成分,其實是有意的。中醫教的本質正是宗教性迷信,中共卻有兩套不同的態度,這完全不合邏輯。
\subsection*{2020-07-11 15:34}

正因爲中醫教每次做嚴謹的科學統計,有99.99 \% 的機率得到負面結果(剩下的0.01 \% 是青蒿素),所以爲了自己和師門的生計,只剩下兩條路可走:1)避免去做科學實驗,然後把一生精力集中在發明自欺欺人的藉口上,並且到處收集完全沒有統計意義的Anecdotes;我遇到的每一個中醫從業人員,都或多或少有這個態度。2)偷偷作假,以便發表論文;國際上生醫界論文的可複製性已經只有1/9,中國學術界是誠信的重災區,比率必然更低得多,而且正確的統計分析難做得很(參見正文),連有意扭曲都不必,只要隨便試幾次,很自然會撞上名義上有效、實際上是隨機噪音的結果。
這次新冠疫情,引發全世界的非理性半原始文化都在吹噓本土的偏方,如果當地的學術界沒有嚴謹、誠信的傳統,這些偏方更加會出現在學術期刊上,例如在印度就非常普遍。很不幸的,上面這個描述,也適用於中國;昨天還有人要我去讀一篇“證明”蓮花清瘟防治新冠有效的論文。姑且不提一個可信度先天就遠低於1/9的結論是否適合當證據,新冠病毒固然是全新的,蓮花清瘟卻是歷史悠久,它如果有效,必須先適用於其他的病毒,那麽問題就來了,過去幾十年,幾十萬個中醫研究人員,數以億計的研究經費,爲什麽沒有證明它對流感或任何其他病毒有效,爲什麽在新傳染病剛出現,嚴謹實驗還來不及反復檢驗結果的時候才出現方便炒作的“正面結果”?生醫界的論斷,幾乎不可能有100 \% 的確定性,但是拿著頂多只有百分之幾這個數量級的可信度來吹噓,依舊是很明顯的非蠢即壞。
\subsection*{2020-07-11 15:20}

我在解釋量子力學和高能物理的時候,曾經提過,理論本身必須邏輯自洽、並符合Occam's razor,才值得進一步研究。如果理論本身從邏輯上就明顯是胡扯,那麽或許個別處方/預測可以瞎貓撞上死老鼠,但這個機率必然很小,遠低於背景雜訊,因而在分析這類信噪比極低議題的時候,就一定得追求比平常更高得多的統計置信度。物理界尚且要求五個標準差,對應著1/3500000的P值;生醫界的傳統是只要求1/20的P值,這當然就導致8/9的論文無法複製(亦即把噪音當信號),以致連當年率先創立P值<1/20這個標準的教授都出面建議改進,但是整個行業已經習慣這個方便發論文的慣例,結果自然改不掉。
你自身這個經驗,在統計意義上之欠缺,我想不用我再深入解釋;這對應著億萬次未受報導傳述的中醫效能等同安慰劑的結果。即使只針對個別事件來談,不適應新氣候的反應,也很自然只會是暫時的。換句話說,你喝不喝苦茶都不影響後續的發展;這正是我在上面所説的“背景雜訊”。
如果“中西醫各有所長”是正確的,那麽你必須事先定義中醫的所長何在,不是事後來找現代科學的邊界。因爲說那句話的人始終是先射箭再畫靶,而我對這類狡辯術先天沒有好感,所以拿它來搪塞不是這個博客能接受的邏輯論證。
\subsection*{2020-07-10 05:39}

Simpson's Paradox之所以值得大家學習,除了它有廣汎的應用之外,還有其難以預防的特點;這是因爲除非做出深入徹底的研究,否則很難確定主要變數何在。一般研究人員總是會想偷懶:既然我只在乎性別差異,那麽就只針對性別來做統計分析。Simpson's Paradox的含義在於:即使你只在乎某一個變數的影響,你仍然必須先完全瞭解所有主要相關變數的作用,然後依它們的影響大小,一一排除。這比一般學者準備投入的工夫要大得多了,所以學術界(尤其是社會科學)到處都是違反Simpson's Paradox的錯誤結論。
中醫教教徒之不合邏輯,已至極點;我在正文最後一段所指的是,他們連“違反Simpson's Paradox”都不夠資格。
順便討論另一個觀察:照理說,Absence of evidence is not evidence of absence,所以中醫教的諸般教旨沒有通過雙盲實驗的記錄,表面上似乎並不是否定他們的理由。但這裏的背景是中醫教有幾億名病人、幾百萬個祭司和上百個學院,幾十年下來還做不出雙盲實驗的正面結果,只有兩個可能:1)反復嘗試、始終無法超越安慰劑,只好把結果藏起來、假裝不存在;2)原本就是存心騙錢,所以連試都不用試。
\section*{【藝術】【基礎科研】藝術與科學的衰敗}
\subsection*{2022-10-23 00:27}

我在十幾年前就不訂有綫電視了。原版的《GoT》是因爲小孩在初中階段喜歡,所以陪著看;現在他離家上大學了,我對這些劇集的興趣,基本只出自社會學研究的觀察需要。在任何一個有意義的專業裏,天生有才華和願意下心血,都是非常難得的事,而且對自己才華和投入的驕傲正是培養職業道德品格的心理基礎。然而職業生涯是否成功,卻是現代人生的極重要評價標準,很多人無所不用其極;所以一旦有了任何像是Wokeness之類的終南捷徑,自然就會劣幣驅逐良幣,但那還只是第一層效應。在與群衆認知有關的行業裏,例如新聞、娛樂、政治、學術、科普等等,資本會進一步有意采行偏離才能和品德的提拔標準,以充分發揮它們作爲投名狀的作用。換句話說,正因爲這些人明白自己無才無德,所以他們對幕後財閥老闆的忠誠度是絕對的,而越是侵占公益的利益集團,對忠誠度就越重視。這個道理的適用性很廣,袁嵐峰這些娼妓被中科大推捧出來,也是出於同樣的邏輯考慮。
\subsection*{2021-04-22 22:31}

這是合理的擔憂;我自己也早就特別寫了《什麽是科學?》來提供精確定義,並且反復强調,科學是原則(亦即對事實與邏輯的絕對尊重)與方法(例如Occams Razor和Russells Teapot),而不是人員和科目。這是博客討論的重點方向之一:這裏有很多篇文章,專門用來批判表面上是科學從業人員、甚至整個科學專業,實際上是僞科學和反科學的亂象。甚至連“邏輯狡辯”和“統計撒謊”這樣的細節,我都已經詳細解釋過了。
至於范老師的寫作,我已經説過,我自己沒有去讀過,所以不能置評。讀者願意遵守博客的規則,精簡轉述外來的論述,那麽我就針對讀者的文字來做評論;這個對話當然不保證轉述者做到100 \% 的精確,所以不能反推應用到范老師的原文上。
\section*{【醫療】我的新冠經驗}
\subsection*{2022-10-14 23:14}

科學研究,尤其是生醫類研究,往往需要幾千個實驗、數十年的時間,才能達成定論。在那之前,雖然不能100 \% 確認一個結論,但往往大趨勢很明顯,在實用上基本可以提早接受。
Omicron的毒性弱,尤其引發急性肺炎(舊版新冠致死的主因)的比率很低,是非常明顯的事實;研究這個議題的論文有幾百篇,絕大多數都得到正確結論,例如Real-life Evidence of Lower Lung Virulence in COVID-19 Inpatients Infected with SARS-CoV-2 Omicron Variant Compared to Wild-Type and Delta SARS-CoV-2 Pneumonia | SpringerLink。然而如果硬要去找,早期做錯實驗或統計的必然也有,但這必須是有心自欺欺人的才會故意專門去找結論錯誤的少數,歷史上是反科學集團(如基督教原教旨者攻擊演化論)的典型伎倆;這當然是博客所不容許的。
病毒的突變,是真正基於量子力學的隨機事件,細節先天就不可能預測。你違反了《讀者須知》第六條規定,不但引用並且推薦一個匿名大V,而且還是明顯在跳大神的大V,我不得不把你拉黑,請利用未來六個月自我檢討,再考慮參與博客討論。
\subsection*{2022-03-18 23:15}

策略最優解,不論是政策、商業、戰略、戰術還是人生,從數學來看實質上就是Constrained Optimization Problem,亦即在參數空間内求某個函數的最大值,但必須受許多綫性方程式所限制。在幾何上,這一般可以簡化為對高維不對稱多面體(Polyhedron)尋找距離原點最遠的頂點(Vertex)。客觀條件的演變,對應著那些綫性平面的滑動;所以即使這些滑動是逐漸(Gradual)而連續(Continuous)的,最優解也會出現Discrete jump(離散跳躍?)。
原本清零顯然是中國的最優解,但Omicron的高傳染性、低致死率,以及世界其他國家的躺平態度,重新定義了兩個很重要的綫性條件;這些變動足夠造成最優解的跳躍嗎?我個人覺得還沒有足夠的證據做正面的論斷,合理的反應是繼續嘗試清零一段時間,持續觀察並評估。我相信這也正是中國政府當前的處理原則。
\subsection*{2022-01-15 21:35}

典型的猶太家族企業,錢賺飽了,留下公司金蟬脫殼,然後和聯邦檢察官達成默契,交出2 \% 左右的贓款換取全部高管免責,和波音的故事如出一轍,唯一的差別在於忘了收買法官,結果後者在去年底宣稱和解條件太離譜,拒絕批准。但是不要指望真能把他們繩之以法,頂多是再多交點罰款,更可能直接換掉法官。
你們以爲我寫《美國式的恐龍法官》是挑選特例嗎?博客這裏從來只有懶得重複,不會試圖誤導讀者:我舉出的每一個案例背後,都有成千上百個類似的現象。

有關和解條件的新聞報導,因爲是典型的美式公關,一般讀者很可能會被其中所含的烟幕所矇騙,所以我想解釋一下,爲什麽我說“2 \% ”。
這裏的分母有三種選擇:第一是國家經濟所受的損害,這難以精確定義,所以我沒有采用,不過美國任何一個試圖估算的經濟學人,得到的結論都超過10000億美元。其次是Purdue Pharma賣了幾十年OxyContin的總收入,這大約是370億美元。第三個,也是最保守的算法,是Sackler家族從Purdue獲得的財富,至今還剩下140億美元。至於分子,所有的美國媒體都說罰款是45億美元,這是假新聞!因爲這其中只有2.35億來自Sackler家族,其它是Purdue Pharma的責任,而後者早已被掏空,甚至都已經申請破產,根本不可能交出40多億美元。所以罰款比率即使最誇大也只有2.35/140=1.7 \% 。
\subsection*{2021-08-03 20:55}

去年疫情剛開始的時候,我曾經提過,一個病毒做出物種跳躍之後,就立刻開始和新寄主進行共同演化(Co-evolution),其長期結果一般是感染力的提升以及毒性/致死率的下降。但是這兩個趨勢之中,前者可以發生得很快、很自然,後者卻必須靠嚴重疫情對寄主幾波大幅殺傷,將抵抗力弱的寄主和毒性太强的病毒一同淘汰。典型的例子是1918年的Spanish Flu,1919年第二波的感染力和毒性都大幅上升,到第三波之後才逐步溫和化,後來演變成爲每年固定發生的H1N1流感。
新冠的感染力提升,和百年前的Spanish Flu基本同速,但毒性溫和化的進度還很難說。這是因爲它是人類擁有現代化醫學之後的第一次全球性瘟疫,在高峰期已經有疫苗開始大規模接種;這雖然挽救了許多生命,但也打破了共同演化的自然進程,使得病毒突變的趨勢難以預測。此外當前的疫苗面對最新的Delta變種,都有類似的防護力下降問題;英美媒體故意選擇性拿國產疫苗來説事,純粹是政治性抹黑。我們做科學理性分析,必須專注在可靠的專業資訊上:從英國國内的新數據可以看出,AstraZeneca對Delta的效果,和中國疫苗在南美的表現是同一個級別,保護力都在50 \% 上下,其價值在於壓低致死率和重病率。
不論如何,生命是社會公益的最重要成分,而當前新冠的毒性還居高不下,既然疫苗有若干效果,就必須全力全面接種。然而現有疫苗固然能大幅減低致死率和重病率,卻無法完全截斷感染鏈,如果放棄其他防治手段,不但依舊會有成千上萬的死亡病例,而且病毒感染往往對器官和免疫系統有不可逆的傷害(“Long Covid”),再加上隨時有可能發生更新的突變、進一步削弱疫苗的保護力,所以既然中國有能力做到零感染,就必須繼續努力。
人類社會在新冠防治上的下一個重點,是把疫苗普及到第三世界。白左内含的宗教性反智原則,是近年來反疫苗民粹的文化根基,它毒化歐美群衆的思想,固然是自作自受,但許多貧窮國家也受了騙;中國應該一方面高調批評反科學的西方文化,另一方面全力提供廉價的疫苗,儘快將亞非拉的接種比例提升到歐美的水準,同時加速開發針對Delta的下一代疫苗。視病毒的下一步突變方向,這樣的疫苗開發/生產/分配可能要無限持續下去。在幾個周期之後,如果最新的疫苗對流行的病毒變種有了極高的防護效果,或者病毒毒性有了明顯的下降,才能放鬆社會隔離和防疫的手段,接受新冠成爲流感之外的另一種年度性流行感染。
\subsection*{2021-07-04 15:21}

請你復習《當前世界的公共危機》一文。和《熱帶風暴之後》解釋長周期事業的特性相比,那篇文章談的是私營企業/自由市場經濟的另一個盲區,亦即當某個行業的關鍵商品不應該任由買家自由出價競購;除了醫藥對應著人命這個關鍵商品之外,還有法律對應著正義,以及教育對應著人生前途等等,博客以往在留言欄已經反復討論過了。
美國學界在過去40年被商業界滲透腐化極爲嚴重,其中也包括醫學界,所以你所描述的學術人對醫藥公司的自私惡行置若罔聞、甚至爲虎作倀,的確已經是常態;然而依舊有少數良心人堅持要揭發真相,建立專門吹哨的網站,只不過大衆傳媒故意忽略罷了。很不幸的,一般人去找這些極爲專精的訊息,往往只會遇到自以爲是的民科;你如果是生醫專業,我可以去咨詢行内的朋友,給你一兩個特定的鏈接建議。
隔行如隔山,分辨民科和吹哨人,需要極爲深入的專業知識。我盡可能依賴自己和可信任的朋友,來為讀者解惑,如果有額外的可靠相關資訊,讀者也可以在這裏和大家分享。作爲一個無私心、有見識的Moderator,我可以幫忙過濾這些討論。

朋友回復,建議你看看這兩個網站,專注在討論藥物實驗被扭曲宣傳的結果。
https://medshadow.org/
https://www.drugwatch.com/
至於Big Pharma的陰招,報導太普遍,反而沒有專人維持網站的動力。
\subsection*{2021-04-22 21:16}

這個博客default是學術性、公共視角的宏觀討論,不是針對個人生涯前途的建議。例如最近有關宗教的一系列批判,指的純粹是政府政策和社會風氣,而不是個人的信仰選擇。
幾年前我在批評香港民意和治理的時候,有内地來的本科生發問,是否還能到香港念研究所,我說大環境不是決定個人生活遭遇的唯一因素,這裏也是一樣的道理。華僑在東亞亞被屠殺、迫害,歷史久遠,但絕大多數沒有能力離開。我們可以確定美國未來20年,華裔整體所受的待遇會每下愈況,但是會不會發生類似1965年印尼對華僑大屠殺,或者1942年美國把所有日僑公民送進集中營的事件呢?可能不會那麽糟糕,但是馬來西亞式的系統性政治經濟歧視卻是合理的終局之一。
當然美國真正最核心的内部矛盾,是階級而不是種族,所以如果你是億萬富翁,能雇傭一隊保鏢,每年花幾千萬收買地方政客(如同我兒子一個猶太高中同學的家長),華裔也一樣能過得滋潤。
\subsection*{2021-04-22 04:44}

亞裔可以組織起來?亞裔和美國黑人不同,來自幾十個互相不服氣的文化語言背景,沒有實質的一致性。
亞裔組織起來就可以影響政局?韓裔比華裔團結多了,有影響嗎?日裔從政歷史久遠,有結果嗎?
亞裔有集中地區,可以局部發生作用?亞裔集中在兩岸人口密集的城市和郊區,民主黨的鐵板地盤;目前亞裔的求學和工作機會被剝奪,是爲了滿足安撫黑人的需求,民主黨正是幕後的政治主力,它有可能放棄黑人選票來保護亞裔嗎?
美國地大物博,總有機會?天然資源再豐富,財閥不會嫌多;過去40年,連白人中產階級都被逐步剝削,這個過程還在加速之中(例如過去十年的大醫院兼并整合,醫生的職業環境已經忽然惡化),你能指望在失去美元對外吸血的能力之後,他們忽然醒悟,對亞裔手下留情?歷史上被自私的利益階級從内部腐壞的帝國、王國至少有幾千個,哪一個曾經爲了佔人口6.5 \% 的少數賤民而懸崖勒馬的?
中國崛起能幫助美國的亞裔?我建議你讀讀歷史,看看德裔在20世紀前半、俄裔在冷戰期間、日裔在80年代(你沒聽過陳果仁案嗎?)所受的待遇,紅脖子會對黃種人特別寬待?美國愚民如何仇外排外,大衆讀物當然不提,但嚴謹的歷史研究還是存在的。
\subsection*{2021-04-21 23:33}

一般人的確只想渾渾噩噩地過日子,所以凴直覺來思考、只看膚淺的表象,是普羅大衆的常態。不過一個健康的國家社會,必須有清醒的高級知識份子,既然他們正是本博客所針對的目標,那麽我當然不能滿足於盲目的樂觀態度。
以亞裔在美國所受的待遇爲例,除了一些口惠而不實的“Thoughts,prayers \& moral Support”之外,就是極少數樣板面孔(例如趙小蘭和楊安澤)消費族群福祉來謀求個人私利,實際上最重要的人身安全以及求學工作的機會,反而在持續而且迅速地惡化之中。換句話説,亞裔目前的處境(相對於自我努力、能力以及社會標準),還不如1865年的黑人。美國在未來十年,很可能要失去美元的金融霸權,從而無法繼續對外收割以維持内部生活水準,届時社會動亂必然會進一步激化,而所有弱勢族群中最無助的就是亞裔。所以樂觀的説辭有點像是火鷄到了11月,看到兩隻同類訪問白宮、上新聞節目,就自我麻醉,以爲很幸福。
\subsection*{2021-04-20 14:10}

是的;美國最值得自豪的歷史,就是憑著二戰而稱霸世界(20多年前還流行了一段“The Greatest Generation”的封號)。從小羅斯福任期開始,定義一個或幾個外部死敵就成爲全國凝聚力的來源,所以十幾年前的仇中宣傳,是順水推舟,很自然地被一個(即使已經嚴重撕裂的)社會全盤接受。
這麽重要的思潮,當然在文化的各個角落都體現出來。1990年代,我還有興趣看電視劇的時候,追過幾個系列的《Star Trek》劇集,劇中的“Federation”當然就是美國自己,而“Klingon”則是俄國,所以劇情講的是原先的仇敵和解了。至於中國呢,則影射為矮小、暴牙、滿腦只想著生意的“Ferengi”,這正是當時美國人對中國人的印象成見。原本編劇想要拿Ferengi來作爲劇情裏的主要壞蛋,結果觀衆不買賬(因爲覺得可笑而不可怕),所以後來才發明了“Borg”。如果那齣戲晚20年播出,我想就會有Ferengi+Borg融合爲一體的大壞人。
\subsection*{2021-02-11 09:34}

我在博客已經反復解釋過了(參見《當前世界的公共衛生危機》、《現代醫療的大倒退》、《有關環保和全球暖化的幾點想法》、以及對新冠疫情的一系列討論),資本主義體制追求的是企業利潤最大化,而不是社會公益的最大化,所以絕對需要政府的嚴格監管和扶正。
這裏問題最明顯的行業有兩大類:1)把私人開支轉嫁為社會成本的,經濟學上叫做Exogeny,例如污染排放、全球暖化等等。這一類,美國經濟學界很願意討論,並且建議用市場化來解決,像是碳排放交易,名義上是把社會成本計入賬目,實際上是找藉口為金融巨頭發明新的游戲平臺。這是因爲這些問題的真正關鍵在於準確監察記賬,市場化根本與其無關,也就是整個課題上的實際工作依舊留給政府,收錢發合格書的權利卻分割出來交給財閥,是典型的芝加哥學派把戲。
2)行業先天就有比利潤遠遠更重要的服務目標,例如軍事是爲了保護國家、醫療是爲了救治人命、法律是爲了維護公義、教育是爲了下一代的心智。這類矛盾基本無法解決,所以芝加哥學派只能避而不談,堅持市場化只能靠宗教式迷信,在英美之外的忽悠就沒有像碳排放交易那麽成功,把軍隊、醫院、警察和學校全面私有化仍然是很極端的政見。不過正因爲有著基本而且絕對的矛盾,即使只是部分私有化危害也極大,必須靠知識分子群起而攻之。我已經把道理講得很清楚,傳播出去是你們的責任了。
\subsection*{2020-05-18 05:14}

I have explained before that when something is truly unique, you cannot expect it to be appreciated fully.
My brother has decided to retire early in order to better look after my parents. This will grant me greater freedom in traveling. I should be able to commute to Taipei from time to time. We'll see what kind of institution can use someone like me.
\subsection*{2020-05-15 08:55}

要掌握事實,只有不斷廣汎閲讀、嚴格過濾,成年纍月地下工夫。我以前也提過,我自己是一直到40嵗之後,才覺得有足夠的知識儲備,能看清世界主要事件的背後脈絡;在那之前,我只聽不說,知之爲知之,不知爲不知,是持續學習上進的重要前提。年輕人先建立學習的欲望和習慣,保持踏實求真的心態,然後追蹤閲讀能做可靠分析的學者,是短期内可以速成的捷徑。
至於邏輯,一樣也是要一輩子不斷地自我訓練,從最基本的定義和規則開始,反復練習,務求達到反射式的熟練。成功之後,可以開始研究常見的狡辯術,並且在日常新聞報導敘事或者留言討論裏,尋找狡辯和扭曲的痕跡。有成之後,要補足常被用來欺矇外行人的假專業論述,例如統計。再進一階,是要瞭解人類心理的自然反應趨勢,綜合趨利避害的動力,解讀場面話背後的真義。更進一步,則要開始考慮一些反直覺的博弈論現象,不過這已經遠超一般知識分子能及的範疇了,年輕人不必太擔心。能看透狡辯術,對絕大多數人來説,已經夠用,畢竟連楊逍也只練到第二層。
\subsection*{2020-05-14 17:31}

沒有錯,我也覺得除非是已經立志要進學術界的,在中國至少先讀完本科再考慮出國,是最合理的選擇。 
借這個留言談另一個話題,也就是我昨天提到文人其實是藝術家的那個討論。我想大學入學文理分科的傳統,造成很多人的誤解,值得特別澄清:台灣和大陸分文理科,是依照所需的數學難度來分的。結果理工類的自然科學被獨立分割出來,但是同屬科學的考古學、政治學、經濟學和社會學卻被劃到文科裏去,和藝術性人文科目混在一起。這就引發一個很大的矛盾:科學的終極目標是求真,而藝術卻是求美,把兩個格格不入的領域硬是綁在一塊兒,就像把鯨和魚類畫上等號,結果就是出現很多受了一輩子文藝訓練、毫無邏輯辯證能力的人,莫名其妙地自以爲有資格做社會批判,甚至管理公衆事務;例如方方和龍應臺之流就是典型,幾條不是特別壯大的底栖魚類連水面都沒見過,卻一天到晚批評哺乳類不用鰓來呼吸空氣是絕對錯誤,實在是既奇怪又可笑的。 
正確的做法,是把社會科學從人文科目解放出來,另設政經學院。藝術裏參雜著社會評論,往往是獲得認可和光環的終南捷徑,但這頂多只能視爲科普,絕不能和事實與真理相混肴。我再次借用科幻小説那個例子:娛樂、審美、觸動情感才是它的實際意義所在,少數足夠嚴謹的創作(例如Arthur C. Clarke的若干作品)可以作爲科普,但是絕對不能用來作爲理工科的教科書或甚至國家學術政策的根據(例如Brian Greene的著作,無恥吹捧超弦,視爲科幻都還屬於低級的那類)。
\subsection*{2020-05-14 08:51}

因爲台灣和大陸的文化背景和社會體制很類似,而發展又早了2、30年,我們應該看看前者的教育系統的演化歷程:在1949年之後,有不少大師級別的學者必須在高中教書,其後十幾二十年大學吸收了這些優良的師資,所以整體高等教育的素質還是很不錯的。80年代台灣富了起來,並開始建立世界級的高科技產業,所以能比較容易地吸引從歐美返國的科技學人,同時培育出合格的本土人才。但是過去20年,在中產階級家長持續投入更多課外資源的背景下,學生的平均素質卻一落千丈,這形成了當前台灣教育界的主要危機。 
造成素質下降的主要因素,並不是過强的愛國教育或填鴨教育。剛好相反,是引進了美國過去40年的快樂教育理論,使得中小學失去了對傳授客觀知識的注重,而去追求遠超學生智力發展程度、虛無縹緲的“獨立思想”和“創造力”。很反諷地,提倡這種快樂教育理論的高官和學者,本身就正是把英美的糟粕囫圇吞棗,毫無獨立思想和創造力可言。 
與這個歪理配套的倒退“改革”,如減負、課外“成就”作爲入學標準、無限增設大學、傳授自我主義(雖然他們叫做“自由主義”)和道德相對論(叫做“多元化”),無一不是美國財閥在可以自由外包製造業、不再需要高素質勞動力的背景下,爲了節省公共開支、圖利自家子弟、安撫普羅大衆(因爲他們進不了常春藤,卻有了許多新學校可選擇)、誤導白左力量,而鼓勵資助教育“理論家”和智庫編造出來的禍國殃民政策。它們的禍害是多方面的:除了固化社會階級之外,還有降低勞動素質、助長民粹思想等等極大的毒害;這次新冠疫情下,英美的反智社會被暴露無遺,正是基礎教育被斷根的後果。 
所以你所討論的,都是小打小鬧,可以慢慢解決修正。中國義務教育最重要的任務,在於提供全民公平的升學機會、足夠的基本常識和集體利益的概念,同時灌輸對客觀絕對事實的尊重並試圖建立做理性邏輯思維的能力。這些目標,無一是民營學校能更好提供的。
\subsection*{2020-05-13 14:21}

你説的沒錯,有錢有資源的家庭,即使是在中國式的高考制度下,一樣會有優勢,但是任何其他的入學標準都會更加地不公平;英美是前車之鑒。 
我自己從幼稚園到初中,都是在鄉下學區的公立學校念的,課餘時間除了導師强制要求的補習之外,我一律不另外花錢找資源。考上中一中之後,我的確注意到城裏來的同學們佔了一些優勢,但沒有超過我的天分和努力所能彌補的地步。 
我想在這裏强調一下,公平(Fairness)和平等(Equality)是兩回事:人生不可能是完全平等的,市場經濟尤其會不斷放大這些不平等,但是不平等也有許多不同的級別,如果國家已經在不犧牲整體經濟效率的前提下,盡全力弭平不平等,那麽這個社會仍舊是公平合理的。換句話説,一個好的制度是盡可能壓低這些不平等的現象,讓出身貧寒的優等生仍然有公平競爭最頂尖學府的機會。 
所以我認爲基礎教育一律國營才是正確的制度;我一再提過,醫療、法律和教育不能聽任市場自由買賣定價,就是這個道理。與此同時,國家還必須刻意加强對鄉下的投資,以彌補先天的城鄉不對稱;這在醫療和教育上,都是中國亟須努力的方向。新冠疫情已經提醒了當局要重視前者,但是後者其實更爲重要。 
我反復地說,中國當前内政最大的問題在於教育,這裏包含了兩個亟待修正的錯誤:在基礎教育上,必須有力鏟平貧富不均和城鄉差距;在高等教育上,則必須建立專業倫理,嚴打假大空。我希望高層能夠在百忙之餘,儘快注意到這個議題,並做出正確的政策改革。
\subsection*{2020-05-10 12:29}

我可以理解你談的現象。我在台灣的親戚朋友,很多也仍然把美國想成人間天堂,拼命花錢送小孩來美國讀書;我自己反而是情願兒子去念清華,可惜他的中文太爛,所以不可能。
所以這個現象背後的原因之一,是距離產生美感:其實美國對華人公開歧視非常嚴重,尤其在入學過程,更是已達可笑的地步(SAT滿分的華裔小孩,進不了常春藤的比比皆是;白人若滿分,則基本沒問題;黑人即使是成績掉尾,一樣也能拿全額獎學金進哈佛)。只看高中,性價比也很低,好學區光教育稅就是一年兩萬五,這還是公立的。
但是另一方面,中國當局對教育界的腐敗和假大空,向來不聞不問,以致劣幣驅逐良幣的現象很嚴重,我已經多次説過,教育系統是當前中國内政最大的錯誤所在,我們只能希望它早日得到足夠的重視。
\subsection*{2020-05-09 12:17}

隨著邏輯思維的精煉和人生經驗的纍積,閲讀歷史就像研究案例一樣,一方面可以看出背後隱藏的推手和脈絡,另一方面也為未來的研究提供新的參考。不過這必須出於興趣,先花上幾十年建立堅實的知識基礎和邏輯能力,切忌揠苗助長,在能有真正的洞見之前,硬搞語不驚人死不休,那麽自然是陰謀論連篇,對人對己都沒有好處。 
你提的這問題基於一個類比,並沒有絕對的因果效應。我的意思是,能理性面對問題、高效執行對策的,就是好的政治社會體系,這和執政者的髮色、膚色和所用的語文、哲學等等都沒有直接的關係。中國在清末到文革之間半個多世紀,和歐美文明的差距達到歷史上的最高點,文人普遍對中國文化、體制和人民素質做全面否定,就是只做簡單類比、而沒有嚴謹的邏輯因果分析的結論。
\subsection*{2020-05-05 01:31}

美國民情是底層民衆欠缺儲蓄、教育、常識和對官僚系統的信任,長期隔離原本就是不可能的,所以要不要在疫情未被控制前復工復產並不是真正的議題,能討論的只是稍早稍晚之間的選擇。
不過復工復產只能部分(因爲消費人潮一定大幅下降)解決零售業從業人員的收入問題,目前供不應求的物流鏈(如食品處理包裝業),早先沒有優先派發防護器材以致病毒汎濫,現在要改進生產綫也不可能一蹴而成,整體經濟復工復產對他們並沒有立即的幫助。
在抗體檢測器上,聯邦政府一樣展現出零先見、零準備,所以不可能有法規上的要求和系統性的執行。有需要的地方和企業,再次只能八仙過海、各顯神通。
四月裏我估計美國各地的實際感染率在5-15 \% 之間,這指的是不同州的城鎮和郊區;一些真正偏僻的鄉下和村落,病毒可能還沒有立足。前天的報告說紐約市民有20 \% 左右有抗體,基本和我的估計符合。這裏我要特別提一下Specificity的概念。它的意思是檢測器材不可能百分之百可靠,有僞陽和僞陰的現象,一般用Specificity和sensitivity兩個指標:前者定義為每測試100個實際上陰性的樣本,會報告出幾個正確的結果;後者則是對全部陽性的樣本正確檢測的比率。目前美國的抗體檢測器材,Specificity有些低到只有95 \% ,所以名義上20 \% 的抗體陽性率代表的可能其實是更接近15 \% 的實際感染率。
不論如何,這樣的感染率距離群體免疫還差一大截,理論上解除隔離必然會引發又一波疫情。但是這裏還有另一個很大的變數,就是大部分群衆的行爲模式也改變了,不但普遍戴口罩,而且有意保持距離,避免室内擁擠現象。如果全美國都能像我住的康州這一帶這樣小心,病毒傳播就會顯著減緩,那麽疫情維持在可承受的範圍内是完全可能的。
\subsection*{2020-05-04 06:54}

你一次問了兩個很大的問題。
首先,在重要的產業升級方向,國外既有的領先企業占據了長時間、大投資所獲得的技術纍積、人力資源和市場額分,可以很簡單地承受短期的殺價競爭,遏制新對手於襁褓之中,所以任何以利潤為導向的新來者,基本不可能成功;這其實正是爲什麽英美的資本普選教對外也要强調自由主義經濟理論的主因。所以在高科技工業上放任地方政府向國營銀行借錢搞自負盈虧,注定會人、財、技三失。要成功,只能以長期立足市場為唯一目的,集中授權並投資給有理想、有能力、有誠信(尤其如果“自主可控”都是哄人的,那麽事先就很明顯沒有什麽誠信可言)的主管專心致志地去建設成長,至於國營或是民營反而不是決定性的因素,例如高鐵、核能是國營,京東方是半國營,華爲是民營,他們的共同點是都以公司的技術能力和長期成長為第一優先考慮。相反來看,公營的和私營的一樣都可以短視近利,以幫助主管們大撈一票為核心目標,前者如貴州和高通搞的幾個合資企業,後者如聯想,除了成就出一批黃皮膚的國際富豪之外,完全沒有其他的意義。
至於美聯儲大印鈔票的事,你要記得經濟危機越嚴重、全世界對國際儲備貨幣的需求量也越高,所以一年印5萬億(我的預估)雖然在幾個月前還是駭人聽聞,美元體系並不會因此而自行崩潰。要推翻美元,最終還是得靠被剝削的其他國家聯合起來反抗虐政,共同發行有替代性的新國際貨幣。這件事我們也是討論了很多年,我個人覺得人民銀行可以更積極些,對内對外都應該加速推行加密貨幣,尤其是金磚貨幣。不過中國的貨幣管理人才,素質還是要遠高於宣傳公關,或許他們内部有合理的疑慮和爭議。我們再等兩年看看是否會有所行動。
\subsection*{2020-05-01 15:33}

庸人無法超脫信仰危機,不是中國的專利;名校教授欠缺智慧,也是普世現象。
英美陣營對“共產”、“獨裁”的妖魔化,從蘇共成立開始(對“專制”則更早,在19世紀末,德國開始威脅英國海洋霸權的時候就已發動),冷戰期間更成爲國家社會的頭號思想教育議題,這其實是多個世界首要政權傾全力來建立一個新宗教(“資本普選教”?),就像Henry VIII搞英格蘭國教會一樣,表面上冠冕堂皇,實際上爲的是統治階級的私利。
資本普選教搞了一百多年,早已根深蒂固。英國在等同的歷史時段,有繼任國王是信天主教的,馬上發生了光榮革命,貴族和富人寧可從荷蘭請人來當國王,以避免羅馬教廷瓜分他們的特權,而被徹底洗腦的老百姓也很高興衝在最前面。在這樣的背景下,希望靠直接論述那個宗教的愚昧和非理性來要求他們回心轉意,是緣木求魚。只能先指出他們嚴重違背客觀事實和自身利益的謬誤,讓他們在考慮狂熱反擊的時候有所猶豫,就已經是很大的成功了。
\subsection*{2020-04-30 13:21}

最近兩周,中國外交部的發言有明顯的進步,似乎是内部有過檢討,建立了新的組織、程序和資源來更專業地處理公關。盧大使在巴黎的發言,從宋先生的節錄來看,也沒什麽問題。
至於解釋“中國的民主”,我覺得步子可能跨得太大了些,無法讓西方人民一下改變認知。我認爲應該先向那些讀者解釋他們的媒體報導中國事務時經常故意欺瞞。我已經舉過幾個例子:幾乎所有的主流英文媒體都曾經用印尼市場販賣蝙蝠的照片來誤導讀者,而事實上中國沒有任何一個地區是吃蝙蝠的。本月稍早,他們要求東亞國家禁賣活體野味的時候,也故意不提中國已經做了。反擊的方式,可以由大使先問采訪者:根據你的印象,吃蝙蝠在中國有多普遍?得到答案之後,再針對該國媒體的假新聞做發揮。
\section*{【美国】【戦略】从乌克兰看今日美俄的政略与戦略}
\subsection*{2022-09-30 23:47}

我沒有能力對一個議題做評論的話,當然也無法仲裁後續討論。忍不住非要談那個話題的人,應該另尋發泄管道;有一個“大統一理論”反復污染褻瀆博客這個公共園地,已經夠煩人,我們不需要其他讀者模仿這種自私行爲。

NeoCon是不知道適可而止的:他們只有一個伎倆,就是步步升級進逼。你若是忍讓不回應,他們心裏認爲得計,佔了便宜,可以進一步挑釁;你終於忍無可忍、强力回應,他們就鼓搗宣傳,説你“侵略”、“違規”,然後推炮灰盟友上陣。所以我素來都說,正確的對美策略,是在第一時間立刻對等反擊。但這裏的“對等”,指的不是方向而是力度,所以並非簡單地在美方選定的議題上回擊,而是要找出有利自身的戰場,針對美國資本的關鍵利益,做出力道相當的打擊。
Putin不速戰速決,就必須面臨許多風險,其中包括給予美國和北約時間不斷升級。烏克蘭正式申請加入北約只是這方面的最新發展,本身只有形式上的意義,不可能被立刻接受。
Putin大律師安排東烏四州入俄,是爲了名正言順地升級戰事,和要不要Odessa沒有關聯。客觀上只有升級增兵之後,俄軍才有足夠力量打下Odessa。
我的興趣廣汎,所以博文能談的話題很多,然而這並不代表我對自己的認知理解範圍沒有掌握。事實上,“知之爲知之、不知為不知”是博客對讀者的重點要求之一,我自己當然也身體力行。農村政策屬於我沒有足夠專業瞭解的科目之一,所以向來對這方面的討論敬謝不敏。
\subsection*{2021-07-10 10:23}

我想澄清一下,前面有關政策責任的討論,專指中方拿熱臉去貼印度和越南的冷屁股的外交策略選擇。烏克蘭和印度、澳洲、甚至日本那些先天仇中的國家還是有所不同的,它只不過是極度跪美罷了,對中方倒沒有什麽特別的惡意成見。《再談Biden任期内的中美博弈》中討論那些在中美對抗下積極站隊的國家時,我尚且建議不要與日本完全決裂,和烏克蘭做生意當然不是不行,只是必須時時把政治風險計算在内。

這是典型的官僚慣性:把上面交下來的任務,依字面解釋,循慣例執行。要改,不能指望層峰(因爲他既沒有專長,也不是他的職責),也不能苛求中下級官員(因爲他們既沒有權力,也沒有足夠的視野),而是專職負責對外工作的最高主管、以及他的貼身幕僚的責任。
\section*{【國際】歐盟内部的無色革命}
\subsection*{2022-09-27 23:45}

There will be no direct financial subsidy from China. It's neither the Chinese way nor to China's interests.
My previous writing on Europe was based on facts on the ground at the time. Now things have obviously evolved. The three key differences are, of course, 1) the new German leadership turns out to be the opposite of reason; 2) Russia has survived the economic blitzkrieg, quite well in fact; 3) the entire 3rd world has stood up against Anglo-Saxon hegemony. Therefore, the following issues are moot: 1) a potential strategic understanding with Europe; 2) making sure that Russia does not collapse; 3) winning support from the 3rd world. The focus in the near future should be on, 1) setting up a new international governing system; 2) replacing dollar as the reserve currency; 3) surviving the current global economic difficulties.
With the big picture explained above, we can easily see that supporting Hungary makes sense, but only in the form of indirect means, fully utilizing the power of Chinese private enterprises.
\subsection*{2022-09-25 02:37}

如同正文中所解釋的,歐盟對内部做無色革命已經有幾年了,只不過Merkel在的時候還不敢太招搖離譜;今年的新發展在於對民主、自由、人權等等口號只一筆帶過,徹底形式化,直接搞霸權干預。反正連制裁俄國這樣的集體經濟自殺都可以在傳媒上説成是英雄式的政策,剋扣援助金和干預選舉這種小事她們有恃無恐。
不過不要因爲有迫害就對Giorgia Meloni抱什麽希望:她和當前歐洲的那群白左女領導是一個模子出來的,被標識為極右的唯一理由是反移民,其他政見例如認同歐盟、仇視俄國、支持墮胎和同性戀、印鈔維持經濟等等都是典型的白左政策。此外,就算强勢如Orban,也不敢只靠掌控議會絕對多數和法庭人事任命而對歐盟的打擊掉以輕心,幾個月前特別宣佈國家進入緊急狀態;意大利的政壇微弱分裂,誰上臺都沒用,何況是一個被昂撒媒體洗腦過的網紅式領導。
\subsection*{2022-04-06 17:09}

很不幸的,匈牙利已經孤掌難鳴。歐盟在去年連番出手,波蘭執政黨一路敗退,例如原本拒絕美國財團收購媒體企業就被迫退讓,後來靠著和白俄閙衝突轉換話題,烏克蘭事件再起,又給了他們繼續回避尷尬話題的機會。既然波蘭被實際打服,歐盟也就順水推舟,暫時擱置“民主改革”的要求,轉而集中對匈牙利施壓,做出財政制裁。匈方沒了波蘭幫忙否決,並無抵抗能力,這正是去年本文所描述的歐盟攻略計劃(參考【後註一】、【後註三】和【後註五】;Orban大選大獲全勝,擊敗的不是歐盟,而是Soros支持的反對派勢力),現在基本大功告成。所謂否決能源制裁案,那是德國和意大利自己需要又不好意思明説,才縱容匈牙利當釘子戶,否則依歐盟規則,這類決策只有一國反對是可以相對簡單剋服的。
\subsection*{2021-12-12 13:49}

你指的是Merkel退休所引發的東歐動蕩,還是Boris Johnson面臨的内部反叛。兩者都是我提前很久就警告過的,雖然前者還有一些新的演進值得解釋(例如最近幾個月來自Lithuania、波蘭、Belarus和烏克蘭的新聞),後者卻基本沒有超出我在過去幾年反復預測過的内涵。馬後炮式的事後評論,只需要自由聯想,是隨便哪個清華、北大的教授都擅長的事;我只做因果性的邏輯分析,而這種分析的特點就在於其預測能力,所以對當前英國政局有興趣的讀者,請自行復習幾年前的討論。
博客還在繼續吸收新讀者,而其中絕大部分一輩子沒有見過因果性的邏輯分析被應用在社科議題上。然而我已經討論這些議題多年,任何新的評論必然是對既有幾百萬字的内容來做補充,所以片面印象可能會是天外飛來,而且極度傲慢。對於前者,我只能建議他從《讀者須知》開始,耐心看完舊文再做論斷;對於後者,我想指出,我對自己分析結論的高度自信,並不是基於那些分析是“我的”,剛好相反,是由於我不斷在做深刻的自我質疑,以維持對事實和邏輯的絕對信任和尊重。客觀事實和因果邏輯不止凌駕於個人或者社會(例如台灣)的主觀意見之上,它們甚至超越時空局限,是真正普世而且永恆的真理。堅持事實和邏輯、而不是自恃聰明的人(參考我對Witten的評價),才能有超越群體和權威的正確洞見。
\subsection*{2021-12-08 02:51}

幕後的真相是這樣的:Zelenskiy治下的烏克蘭,因爲沒有無限印錢的選項,在過去幾年面臨政治、經濟、社會的持續惡化,已經被迫把反對黨關起來(西方媒體當然完全無視),然而還是眼看著要全面崩盤,剩下唯一可抓的稻草是對東烏用兵,希望憑藉軍事勝利來維持自己的政治地位。今年三四月間,他就已經準備動手,結果Putin立刻陳兵邊境;當時Biden政權剛上臺,還處於狀況外,趕緊叫停,只有英美外宣體系出面聲援,渲染俄國“準備侵略”。
過了半年,國際上出現能源短缺危機,Putin手裏的能源牌更加强勢,烏克蘭内部的經濟和社會壓力也更大,眼看著Zelenskiy很可能拖不過這個冬天。所有的跡象顯示,在過去兩三個月,他和美國的國安外交情報和宣傳體系達成共識,計劃要在明年初(冬奧可能曾經是個被優先選擇的節點)用兵,所以雖然俄軍並沒有大幅調動(上次事件緩和之後,俄軍把重裝備留在邊境附近,部隊回歸原駐地,原本就可以很快反應,不必在前綫等待),英美媒體開始無中生有、反復報導俄國的“侵略”計劃和動作。
因爲軍事上早就準備完善,Putin現在可以簡單在外交上做出絕對强硬的警告,甚至要求北約簽約保證不再擴張。這當然不可能被接受;Putin的用意,在於提醒Biden政權裏的那些白左蠢蛋,他們手裏根本無牌可打。而他之所以能有這樣的絕對戰略優勢,正是自2014年之後,根據己方利害考慮,主動選擇脫鈎方向的成果。現在不但俄國的銀行金融業完全不懼被踢出SWIFT系統,能源業有中國作爲替代顧客,高科技和軍工產業轉向東亞零件供應商,甚至連農業都徹底獨立重建;這都是忍受了幾年極大痛苦才換得的長遠利益,也是我所倡議的針對性、選擇性脫鈎戰略的最佳示範。
\subsection*{2021-12-07 20:44}

你似乎是把博客的敘事完全吸收,能夠獨立觀察這些現象,來獲得同樣的結論。其實除了診斷、預後之外,我也早已多角度給出處方(例如前幾天又再强調一次,應該忽略政客、針對假新聞做反擊),但正確的觀點只在網絡一角傳播是沒有用的,在任何一個大國、任何一種體制之下,正規管道都是學術和智庫界的責任,而且正解不能只是極少數聲音,必須是主流,才方便獲得執政者的注意和采納。換句話説,政策選擇只有兩個關鍵節點:學者團隊和決策高峰;因此我們根本不必瞭解中國政府内部研討政策的運作細節(把問題怪到某些虛無縹緲的未知官僚體系運作細節上,似乎是中國社科學術界最喜歡用的藉口),只要證實高層並不排斥正解,就可以從簡單邏輯確認學術和智庫界必須對多年來的颟顸窘態負主要責任。
\subsection*{2021-10-27 19:42}

我什麽時候説過新歐洲會被邊緣化?當前歐盟内部的衝突,本質上是聯邦和國家主權之爭,明面上的議題則是白左教和天主教之間的文化差異,和對俄政策一點因果關係都沒有。從邦聯向聯邦轉變,權力上升的是中央政府,對應著位於Brussels的歐盟,核心大國的地位和重要性反而會視聯邦體制而有不同程度的下降。像是歐盟這種現代民選制度,尤其最高決策單位是一國一票的European Council,越是中央集權,核心成員國的影響力就損失得越多。你看美國紐約州和加州對西部那些人口只有它們1/50的小州完全沒轍;美國的總統大選還沒有到一州一票的地步呢。
至於歐盟軍隊對俄國的效應,還用問嗎?你先看看歐盟内部,只有法國能派出旅級的完整機動部隊到境外作戰,很多小國連營級都談不上,更別提戰役支援單位(電子、運輸、偵察、加油等等)基本全靠美軍,所以歐盟軍先天就只有打治安戰的能力。
\subsection*{2021-10-21 03:29}

因爲拿人的手短、吃人的嘴軟。意大利原本是歐盟創始國之一,地位僅次於德法,但是自從2010年希臘貨幣危機之後,歐豬國家忙著跪求經援,就算是想在其他議題插嘴,也徒然惹人恥笑。例如兩年前意大利前政府不知好歹,公開支持黃背心,結果不但法國立刻召回大使,其後歐盟乾脆趁亂指派剛退休的歐洲銀行總裁Draghi空降為新總理,連選舉都免了。我覺得任命Draghi很可能是德國為歐盟共通債券背書的條件之一,當然Macron也很樂意。不論如何,連總理都是歐盟指派的,你還能有什麽話語權?
至於天主教被歧視,我想不是主要因素(因爲西北歐已經高度世俗化了,每周上教堂的人很少,與其把他們歸類為新教國家,還不如説是白左教來得精確),至少不是直接的;有影響也是間接在同性戀婚姻這件事上引發衝突。奧地利、波蘭和匈牙利也都是天主教國家,只有捷克的世俗化程度接近西北歐;然而他們和歐盟的衝突,明顯不是基督教教派之爭,而在於抗拒歐盟的權威,白左教只是這些權威中的一部分,參見正文中的詳細討論。
\subsection*{2021-10-19 18:20}

照搬50年前冷戰時期的策略,怎麽可能有邏輯因果意義?你的第一句論述就明顯是牽强附會、硬凑預設結論;這個博客不適合你,請自行離開,另找同類。
如果理性思維能自動成爲主流,人類社會早就進入大同世界了。問題在於能做嚴密邏輯論證的已經是極少數,願意不趁機參入私利考慮的更是絕無僅有;指望民選出來的政客做到這一點,顯然是緣木求魚。
即使是從中方的觀點出發,在歐洲白左的强大聲量下,要確認他們其實也是英美宣傳的受害者,必須分別對待,也非易事。過去幾年博客在外交方面的建議,最核心的信息就是這一點。這一直到今年年初的那幾篇文章,才被官方采納。雖然一般讀者沒有注意到,但我自己心裏有底,知道這是在大對撞機和《美國陷阱》之後,我對現實世界的又一次可確認的貢獻,所以最近也就沒有必要再反復分析這一點,而是進一步談更深入的細節。

我想順便提醒讀者,這個博客是針對理性知識分子的論壇,一般網絡的留言慣例不適用。發問的讀者自動成爲共同作者,所以必須先自問是否具有足夠的知識和邏輯能力,以及是否已將博客既有的討論完全吸收。換句話説,請體諒我是無償做輔導,理解自己在提問前有責任預行準備功課。
像是這條留言,既影響我的心情,又浪費大家時間,事後我又得另花10分鐘算一道簡單的奧數題來調節情緒,實在不是我希望每天看到的。
至於讀者中若有好奇的,那道奧數題是土耳其的:證明方程式 m\^4 + 2*n\^3 + 1 = m*n\^3 + n 只有一組自然數解。
\section*{【太空】嫦娥5号T1顺利返回地球}
\subsection*{2022-08-30 14:34}

我並沒有說一定會出問題,而是這類載人航天計劃應該追求低於0.1 \% 的重大事故率,最不濟也必須實際壓到1 \% 以下,但NASA這次爲了政治考慮而趕鴨子上架,而波音主導的承包體系又早已嚴重腐朽,若不接受反復延期,則重大事故率必然會超出10 \% 。財團或許樂意拿航天員的人命和國家的顔面來賭這個機率,對社會整體來説卻絕對不是最優解。
這裏我順便指出一個常見的認知誤區,亦即一般人往往以爲工程議題的不確定性必然小於政治,這是錯誤的。人類固然有著比普通物理系統更複雜的生理和心理作用,但也先天就建立在減熵原則之上,而且一旦組織出國家社會,更加必須遵循固定的規則,因而人爲地壓縮了隨機性。我在去年底對英國政局的分析預測就遠遠比當前美國登月計劃的成敗更為Precise;這除了後者有著較高的不可約不確定性(Irreducible Uncertainty)之外,也因爲其誤差是更難處理的Poisson Distribution,而人文政治方面的機率分佈往往是相對簡單的Bernoulli Distribution。
我十幾歲的時候,也曾經迷過《Foundation》裏的Psychohistory;現在年長,自己可能是當前全世界最接近那個理想的人,也就明顯看出了Asimov沒有理解到的問題,亦即人類歷史演進是一個混沌現象,所以1000年的未來發展,不是靠著中央極限定理N趨近無限大而能精確預測的。還好正如Keynes的名言:In the long run, we are all dead!能正確預判未來幾年的國際形勢,已經可以對國家和人類有重要貢獻了。
\section*{【金融】我對引入美國投行的一些看法}
\subsection*{2022-08-27 09:16}

雖然博客一直只用一兩句話給出簡單結論(若不對基本定理一筆帶過,金融經濟是如此的複雜,光是這篇正文討論的内容就至少需要十倍於1+1=2的證明篇幅,而Bertrand Russell的那本書已經有一千多頁),如何測量國際貨幣份額,其實是個很深的實用性議題,完全足夠寫出多篇學術論文,然而至今我沒聼過任何經濟系的教授,含那群諾獎得主,能正確解答(這當然可能是昂撒經濟學腐朽的間接後果之一);一個數學系出身的也理解得八九不離十,難爲你了。
是的,正因爲貨幣原本就有多種功能和特性,所以不同國際貨幣的相對重要性,也反映在遠遠不止是國際貿易一個層面之上,還包括了儲蓄、融資、大宗商品定價等等。而各國中央銀行的外匯貨幣選擇,剛好就對應著該國經濟與國際體系對接的總體需要,所以外匯儲備是國際貨幣額分的最佳標杆。這裏的誤差主要來自對各國做平均的權重,理想上應該大致與GDP成正比,最好能與戰略性天然資源挂鈎,實際上偏重的是進出口順差;這個相關性Correlation是正的,但不是100 \% 。不過這是高階修正,在初級近似上,外匯儲備很明顯是最佳指標。
\subsection*{2021-08-15 07:04}

《經濟學人》剛剛有專文(China’s future economic potential hinges on its productivity | The Economist)介紹中國在2020年3月出臺的32條提升全要素生產率(Total factor productivity)的政策,很奇怪我用中文搜索不到;不過從《經濟學人》的評論來看,還是自動化和城市化那些老套的宏觀措施,完全沒有考慮到地方亂投資和不追究騙補所帶來的資本浪費。其實投資效率,的確是生產率的決定性因素,只不過這靠的不是美式經濟學所鼓吹的金融“創新”和“自由”,反而是監管和紀律,而中共中央管經濟的人,似乎還沒有理解到這一點。

Risk-adjusted return和Risk premium都是廣爲人知的金融概念,理論上當然也應該納入行政效率評估和人事績效考核;但在實際執行上難度很高。這是因爲風險比回報(Return)的估算還要難得多,就連計算GDP成長率(屬於Return)都有一大堆貓膩(參見《談GDP數字的局限性》),風險的計算公式更加容易被扭曲。現代投行裏都有龐大的風險管控部門,但金融危機一樣反復發生,就是這個道理。與其强行數量化和公式化,不如把行政的其他考慮因素直接列為指標,例如環保成果、地方政府的隱形負債等等。
\subsection*{2021-08-10 15:45}

原來如此。那麽的確可能是19年談判時太心急,做出過度的退讓。當時很可能也有想要藉機開放金融的官員;在最近的金融監管改革浪潮下,希望那個危險已經消退了。
最早的投行生意,原本只是幫一般企業上市或重組,後來因爲往往需要自行處理一些IPO股票,開始參與證券交易,越做越大。到20世紀後期,這些大宗證券交易的部門早已喧賓奪主,成為利潤的重心。然後從80年代起,有人開始獨立出去,以基金的名義搞同樣的交易策略,成爲影子投行。過去40多年的幾個主要金融危機都有真假投行在搞大宗交易的過程中,為追求利潤把杠桿調得太高的“貢獻”。
“證券、基金管理和期貨服務領域”固然有面向小投資人的零售銀行業務,但即使在那裏,交易單(Order Flow)匯集起來之後,幕後也要有人集中處理,這就屬於投行的大宗交易了。20年前我在UBS創建程序交易的時候,當然也準備要順便把這方面原本外包出去給像是Madoff那類公司人工作業的小額多頻交易(就是因此和Madoff的弟弟、兒子打過交道),藉著自動化收回自理。那個商業計劃,後來被整個華爾街廣汎參考複製,很快就完全普及。JP Morgan的團隊據説做得不錯,利潤率很高(處理客戶交易單的利潤,當然最終還是來自客戶,差別只在於吃相難不難看)。此外,不知他們這次會不會連消費者端的門市生意也一起做。我以前提過,在2000年我曾建議UBS買下一家既有的美國證券公司來一次性獲取更高的交易流量(後來高層選擇兼并Paine Webber),但是中方可能不會(也不應該)容許美國人買下大型的零售證券商,除非這也是外交承諾的一部分。
\subsection*{2021-08-10 02:08}

正文有一整個段落,專門論證金融行業的任何效率提升(除了最基本的業務人事管理品質之外)都會自動投入尋租,而不是反饋經濟。你的整個論述的基本假設卻正是美國投行的高效是在資本配置上。我知道這是美國經濟學課本裏面的基本教義之一,但我在部落格批判美式經濟學沒有一萬也有上千次了,讀者沒有藉口。而且美國這些投行搞出全球性經濟危機遠遠不止一次,任何客觀的人都可以看出他們的盈利高效純粹是杠桿作用,大賺10年,還不夠危機期間一年賠的,最終全靠美元霸權來兜底。
迷信課本、重複口號的人,選擇忽略我已經寫過的幾千萬字,參與留言討論有什麽用呢?難道你指望我專門爲你寫下幾億字的回復?我追求的只是真相;既然美國經濟學教科書在你心中高於真相,那麽就請你繼續和那些“學者”討論,這個博客與你無緣。反正你已經違反了《讀者須知》的第八條,我們依照規則處理。

要提高資本配置的效率,我以前也解釋過了,必須先對奸人利用信息不對稱來詐財做出打擊。中國資本配置效率低,不是因爲沒有美國投行這類玩金融游戲的高手,剛好相反,是監管不足,例如美國都沒有的刷單產業,卻在中國欣欣向榮。錢被騙走了,談何效率?要解決像漢芯那樣的負效率案例,美國投行能有什麽貢獻?
犯罪學早就有一個共識:定罪率若是低於30 \% ,對一般罪犯的嚇阻力就近乎0。中國對商界的詐騙行爲比美國還要放任,定罪率別説30 \% ,連1 \% 都不到,才導致資本配置效率低下,你們這批被美國洗腦的帶路黨居然還在鼓吹要進一步去監管,罪莫大焉。
\subsection*{2021-08-09 11:06}

有任何報導證實這是第一階段協議的一部分嗎?請提供鏈接。
2015年股災的時候,規則就不是中國自己訂的?我早説過,只要是有大賺大賠,金融管理就還不到位。中國的管理到位了嗎?就在這篇正文裏,我還特別强調“國内奸商”汎濫,然後又詳細解釋了美國投行會教導他們提高尋租、鉆漏洞的效率。如果監管單位稱職,會有這麽多奸商?正文最後一句的“精神美國人”,真的在中國決策和執行單位裏一個都沒有?
美國投行裏,人人都是哈佛、耶魯的頂尖學生出身,繼承了百年來無數伎倆的知識纍積,爲了每年幾百萬美元的年終分成而日以繼夜地殫精竭慮,不斷繼續發明新的花樣。中國的金融監管單位也都是清華北大的畢業生嗎?有實際金融游戲的經驗嗎?獎勵機制呢?這些金融花樣,不説穿,連其他業内專家也想不到(如果他想得到,就應該自己應用來賺大錢),監管人員怎麽可能?如果沒經驗的人也能想得到,美國投行怎麽會笨到要每年花費幾千億美元獎金?反之如果監管人員看不懂,那又談何制定規則?
\section*{【人生態度】知識份子的基本修養}
\subsection*{2022-08-09 10:27}

是的。我一輩子的習慣,是在做任何評論之前都會先做兩個必要的考慮(這是我在視頻上有點支支吾吾的原因之一,當然實際執行上一兩秒鐘的思慮不可能保證100 \% 成功):1)是不是可以邏輯論證到基本確定是正確的結論?2)說與不説之間是否影響社會公益?如果這兩者有一關過不了,我就不予置評;然而反過來,如果兩關都過了,那麽再怎麽難聽、再怎麽突兀、再怎麽“有失身份”,我也會想方設法找機會來大聲疾呼。
把理性思考能力爲零的人稱爲“沒有資格對公共事務做評論的蠢蛋”,正是直接應用以上那個原則的結果。你指出這些人朝三暮四、前後説法不一致,那還只算愚蠢的間接徵兆。其實從他們的藉口來分析,就可以簡單看出他們連最基本的邏輯都不懂。例如說我的專業權威不夠高,固然是典型的以人身攻擊來轉換話題的狡辯術,2016年大對撞機論戰中,丘成桐也曾搞過同樣的把戲;但這裏即使接受話題轉換,批評我權威不足的邏輯前提也必須是發言者本身的權威夠高,丘至少在這一點上邏輯自洽,而那些網絡噴子不但示範邏輯矛盾,而且毫無自覺。光從以上的論證,就可以得出他們是貨真價實蠢蛋的結論。
請注意,罵人與否,本身可對可錯:如果是嚴謹邏輯分析的結論,而且有益公益,那麽不但是對的,而且是大大的好事;如果沒有邏輯基礎,純粹是潑婦駡街,而且阻礙公共事務的邏輯辯證,那麽不但是錯的,而且是大大的壞事。這其實和鄉愿是一體兩面:後者是沒有邏輯基礎硬是給出謬贊,同樣阻礙公共事務的邏輯辯證,所以是“德之賊也”,那麽噴子很明顯是“德之盜也”。
噴子另一個喜歡用的藉口,是我一輩子沒去過大陸,所以沒有資格發言。我以前已經反復用重力方程式爲例,來解釋這裏的邏輯缺失,但實際上這個論述不止是有缺失,而是也達到了自我邏輯矛盾的地步:當前世界人口有將近80億,其中只有15億左右曾經去過大陸,這幾乎包括了被優先選拔來寫内參那幾千名精英的全部。其他的60多億人,想要被邀請來寫内參的標準遠遠高得多,總數多少我不知道,但應該可以用一隻手來數。那麽一個生長在大陸、被自己的政府認定沒有資格寫内參的人,卻自封有資格拿這一點來批評我,不又是明顯的邏輯矛盾嗎?
當然這些噴子實在太過愚蠢可笑,所以還有更直接的駁斥,也就是上面已經隱約帶到,我的邏輯論證過程都是詳細公開列出的,要做任何批評之前,不是必須指出我錯在哪一個步驟嗎(參見《讀者須知》第8條規則)?指不出錯誤,卻在不相干的話題上謾駡,正是典型的撒謊造假、混肴視聽,自然沒有對公共事務發言的資格,否則影響國際社會的治理,是人類中的害蟲。
\subsection*{2022-07-22 02:06}

或許吧。不過我除了早年那位科技編輯之外,總覺得他們事事對我吹毛求疵。
此外,我已經得罪不少人;利益集團那沒有辦法避免,但是有閑人群體也以噴我為樂,是干擾新讀者的噪音源。這裏有兩個心理因素在作祟:首先,有些大陸人聼到台灣口音比他們聰明,就全身不舒服;等這個台灣口音開始批評大陸的知識和執政精英,那更加是可忍孰不可忍。不過我早年就説過,不論我的寫作有什麽實際影響,至少先消除“台灣人全是白癡”的錯誤觀念,算是一個小小的成就,所以這其實是個好現象。其次,就是我對網紅文化的反感和批評,自然引發那些網紅粉絲的仇恨;這叫做Parasocial relationship,是近代大衆媒體被發明後不久,就被心理學家注意到的非理性病態。不過我不是他們的心理醫生,所以也沒有治療他們的責任。
\subsection*{2022-06-28 11:08}

我對網絡騙子沒有興趣,不會去鑽研他們的背景。不過以網紅的邏輯來看,如果“似乎”、“可能”的經歷是真的,不大吹大擂的理由何在?和Scotland Yard的人熟,是必須隱藏的事嗎?這麽簡單設身處地的分析,一個受過高等教育的人也想不到,邏輯辯證在中國教育系統中的被忽視,又多了一個實錘。至於“台灣方面可能有事”,這是典型Gypsy算命人的模棱兩可伎倆,博客留言居然也有高興受騙的,這不是我的錯、就是你的錯;我回頭反省淘汰讀者是否過於寬鬆,你也應該好好檢驗自己的思考模式。
要預先判斷Johnson何時面臨逼宮危機,難處其實在於他每個禮拜都有兩三個新的醜聞,正確選擇臨界點極爲困難,隨機猜中的機率比一般的5 \% 還要低(指短期預測;若是要提前一兩年,等同是對未來的幾百個醜聞都做影響評估,機率還要隨指數下降)。英國的政評,若是保守黨的,永遠不會承認有真正的倒臺危機,若是反對派,每次發言都會說Johnson即將下臺;光暗示自己有英國公務員朋友,所以會有洞見,是可笑之極的藉口,受衆必須内心先是崇洋媚外,誤以爲英國政壇信息誠實公開、而且一般公務員(警察能知道國會黨派内鬥的内幕消息!?你們三流電影看太多了)素質等同政治學專家,才會上當。你只要想想,中國幾千個國關專業人士,其中有相當數目和英方知識界精英有交流,而且這裏任何一名英國學術精英都比什麽Scotland Yard的警官更懂國會運作千百倍,卻沒有一人能做出正確判斷,一個必須故意掩飾自己教育和工作經歷的網紅,自主得出解答的可能性何在?
網紅愚弄群衆,固然可能只是爲了一己私利,但惡劣的非理性愚昧風氣成爲日常,卻必然是為壞人侵占公益做預先鋪墊;自願受騙者沒有一個是無辜的。

雖然前面提到的算命師伎倆照理應該是衆所周知,但博客讀者留言不斷突破非理性思維的下限,所以我還是簡單解釋一下:他們靠的是從浮面信息來做推測聯想,例如從對方手指是否有繭、穿著打扮化妝、年齡性別、口音用詞、和情緒緊張程度等等信號來選擇一個模糊的方向做試探。應用在這裏,網紅可以從網絡上或者甚至直接觀察得到有關東南方向軍事基地和機場戰備狀態有動靜的訊息,然後做概括猜測;故弄玄虛固然明顯可笑,但非理性群衆自然會有上當的。
所以理性分析的預測,不但必須給出明確精細的結論,而且要把事實和邏輯根據講清楚;就像職業撞球必須指定球、指定袋,用力亂撞、碰運氣進袋不但是不入流的作爲,也方便剽竊者找藉口。
\subsection*{2021-06-21 11:02}

自從我幾年前開始和史東合作做視頻,首次接觸我思路的觀衆偶爾會在反響中以孔明相比。你既然也提了,我想在此澄清這個議題。諸葛亮是中國歷史上不世出的特級人才之一,我是遠遠不如的。不過“遠遠不如”不在於一般人直覺中未卜先知、屢出奇謀這方面的比較,因爲真實歷史中、也就是《三國志》裏的孔明,和小説《三國演義》有很大的差別:後者爲了群衆娛樂性,大幅誇張、虛構了戰術性的奇謀詭計,已經到達妖術的地步。現實裏人力可及的第一流診斷、預後和處方能力,固然遠高於現代學術界和智庫的水準,但還是有其極限,而且在正確的選拔、培養和獎勵機制(亦即有伯樂,而不是習慣管理粗鋼生產的官僚)下,可以同時找到不止一個有足夠天賦的人。例如《三國志》裏,只因爲時代有需要,一流的謀士就一下出現了十幾個,孔明固然是其中之一,但還有田豐、荀彧、賈詡、徐庶等人;我曾經自比田、徐,讀者可以自行評估這樣的比較是否完全貼切,但我想至少不是太離譜。
諸葛亮真正了不起之處,在於他出將入相,文治武功都達到中國歷史的頂層級別。雖然常人以成敗論英雄,他卻沒有成功做到“平天下”,但這純粹是因爲他所擁有的資源比對手低了一個數量級,而且即便如此,他一次對抗兩個未來開國創代的梟雄(曹操和司馬懿,孫權還不夠格),竟能維持戰略攻勢達數十年之久,顯然本領比他們還要高出至少半截。有後世的Armchair Strategist批評他過於激進、虛耗國力,那其實是形勢所迫:蜀漢的人力、物力條件太差,一旦孔明離世,被消滅只是時間問題;徹底利用指揮能力上的暫時優勢,嘗試扭轉資源的不平衡,是不得已之下的唯一可行戰略選擇。如果對手是庸碌之輩,如曹丕和司馬昭,很可能他就成功了。
我自己除了毛遂自薦、借箸代籌,做政策建議、當個謀士之外,附帶的長期任務,是要引導華語界的政治學、經濟學、社會學、管理學理論,走上科學理性的正道。博客的讀者,應該自認為未來重新審視這些學科思想的種子。
人性容易嫉妒、眼紅,中國人也不例外。大陸網絡輿論曾經偏執於楊先生的婚姻私事,一直到2016年的大對撞機爭議才有所更正,明白要著重在他對公益的貢獻。其實我也羡慕楊先生的婚姻遭遇,但不是因爲他後來有年輕妻子,而是他願意續弦,顯示出第一任婚姻必然很幸福;像是我這樣遇人不淑的,想到女人就嚇出渾身大汗,哪會有興致再婚?我母親曾責怪我弟弟一生不娶,他說你看大嫂這副德性,我幹嘛自找罪受?我過去幾年,獨力撫養兒子的關注重點之一,就是怕他被他媽媽嚇壞了,影響一生與異性交往的意願,所以對交女朋友特別支持鼓勵,兩口子吵架之後,我也用心輔導,希望他能把這些經驗轉化為正面的學習過程,建立利於健康家庭的超然心態和習慣。一個成功的婚姻,需要雙方的共同努力,我期望他至少做到自己的本分,至於配偶的個性和智慧,運氣成分是難免的。
\subsection*{2021-06-10 10:09}

真抱歉,我的人生鏈條從“格物、致知、誠意、正心、修身”做到“齊家”就被前妻打斷了,一個被離婚的理工男實在沒有資格談男女關係,何況這也脫離博客宗旨太遠。
不過我發現誠懇、無私的愛,加上不間斷的身教、言教,對建立孩子的正確三觀很有成效。當然這只算一個采樣點,沒有統計意義,或許我的兒子天生人格優於我的前妻。
人生短暫,肉體離世之後,家人親友心中的記憶算是延續我們的精神生命,但頂多也只是幾十年,如果要在百年之後讓自己曾經的存在仍然有些價值,只能靠著對群衆做出長久的貢獻。我的天分在理論思維上,所以試圖為人類社會留下智慧遺產;其他人可以依照自己的特質,選擇合適的努力方向。這裏的重點不是要留下文字記錄,而在於盡自己之力、無愧於心。
\section*{【台灣】再談統一}
\subsection*{2022-08-03 18:02}

顯然我說的還不夠清楚,那麽就再重述得更細緻、更直白一點:這麽多民衆指望强力手段,主要是受外交宣傳體系高調發言的鼓勵,大半還是感性的反應。然而若是提升一個Metalevel來理性分析這個高姿態,恰好也會得到同樣的結論,亦即在未來一兩個月,應該會有不斷的持續升級;這裏的邏輯是我昨天反復解釋的,和躺平相比,事先放狠話而事後無大作爲,需要更多的努力來得到遠遠更糟的結果,實在不像是中國政府會犯的錯誤。
其實如果聽衆讀者真正用心留意我的既有評論,就不會有不切實際的預期。上周我才剛剛明確解釋1)胡錫進所説的不是最高層的決策,而是他所接觸到的許多選項建議;2)在這些選項建議之中,被采納的不會是伴飛這樣的和美國人針鋒相對,而是戰機飛越臺北上空這類針對台灣壓縮防禦和外交空間的舉措。然而更早我就做過一個更重要的預估,很不幸的似乎被更普遍地忽略無視,也就是三四月間上《八方論壇》談俄烏戰爭,雖然我有不談武統等等敏感議題的原則,依舊因爲事關重大而特意跳出美俄鬥爭的原主題,對民進黨當局做出友誼提示,說美國當前的政壇已經沒有一絲智商和理性的存餘,在通往毀滅自我和世界的懸崖邊狂飆,能夠踩刹車的只有蔡英文自己,如果不果斷出手、阻斷局勢升級惡化,就必須準備提早赴美當寓公。這裏已經精確地預告了本次事件,並且還包含一個不留心就錯過的重點,亦即承受挑釁後果的會是臺方自己。
現代國際社會極度複雜,爲了方便讀者領會,我儘己所能用最簡潔的文字來為大家解釋,一方面不說任何廢話,另一方面往往必須對多層次、多維度的複雜考慮簡單一筆帶過。然而如此一來,固然可以傳播給更廣大的聽衆群,但是要徹底領會其中的道理,也要求更高的資質和更多的付出。留言欄主要就是爲有額外領悟的讀者澄清細節用的,而不是遇到新聞熱點,來重複已經被解答過的問題,純粹發泄自己的反射性發言欲望;參見《讀者須知》第三條規則。
至於Pelosi訪台的美國國内政治收益,根本不存在,純粹就是她為自己的面子和形象而硬闖,連帶地加速昂撒霸權的垮臺。
全球性經濟衰退早已進入現在進行式。你閲讀博客多年,進步不大,有必要反省Self Improvement的方法,少説多聽,有空做做數學題,並且研讀正規入門性學術課程,例如經濟學和近代歐美歷史。
今天還有好幾個其他類似的留言,爲了保護篇幅簡潔,我利用這一樓一並解答,請自行參考思索。要再提問的讀者,請先反復細讀這個回復,並參考《讀者須知》規則,除非真有新細節值得討論,否則會被直接刪除。
\subsection*{2022-04-12 08:19}

我想這件事各個方面的邏輯論證,博客已經做得很詳細完整了。甚至我事先預期反方的論述會主要采用狡辯術裏的所謂“Slippery Slope Fallacy”(參見https://en.wikipedia.org/wiki/Slippery\_slope),也就是把正方的論述推到極端,遠遠超出合理的範圍(這其實還是移花接木、自己樹靶自己打的一個變種)。在這個案例裏,就是直接指控我要違反“不干涉内政”的原則,所以昨天很高興有讀者發問,給我機會特別針對這類敘事做了消毒。請大家在輿論上廣爲傳播這一系列思路,希望能先端正公衆視聽,然後間接影響智庫和幕僚。這些年來,我不是已經在很多其他議題上,成功引導公共輿論,以致《觀網》社論欄的讀者留言水平有時凌駕於學術訪談的正文之上嗎?
\subsection*{2022-04-11 13:33}

你是去年才來博客的吧?現代網絡文化對社會有極大的腐蝕作用(參見前文《大衆媒體的内建矛盾》),再聰明的新讀者也難免談著談著,一些壞習慣就露出來了。我用文字雖然極度精簡,最忌無病呻吟,也盡可能避免重複論點,因而不是每個人都能一次看懂全部涵義;但也正因爲精簡,可以每一個論述都追求細節上精確和整體上自洽。換句話説,讀者試圖舉一反三、自行引申是好事,但是引申的過程必須純粹從第一原則和既有博文敘述出發,根據因果邏輯嚴格推導,絕對不能無中生有、望文生義,或者斷章取義、隨意聯想,更加不應該自我局限於其他人的老生常談套路。
在這個案例裏,我什麽時候說過要放棄不干涉内政的原則?不干涉内政,不但是王道的基礎,是未來中國主導的國際體系優於昂撒霸權的重要因素,而且原本就是當前中東產油國主動向中方表示親善的原因之一。我説的是,不能放任英美顛覆不順從的政府而無所作爲。這和不干涉内政有矛盾嗎?完全沒有!事實上顛覆就是最惡劣的干涉内政,所以任由英美作惡而無所作爲才是違反“不干涉内政”原則的錯誤方針,是庸人怕事懶政的藉口。
我的建議其實是要放棄被扭曲的既有政策,轉爲真正貫徹“不干涉内政”、“互相尊重”等等正確外交原則,並試圖將其普及、應用到昂撒集團身上。換句話説,以往中方執行的是“中國自己不干涉内政,但對英美顛覆侵略只做壁上觀”,我認爲必須轉變成爲“中國不干涉内政,也反對昂撒霸權對第三世界做顛覆侵略”,這不才是中國真正的傳統外交原則嗎?在實際應用上,當然必須視情節輕重而選擇適當的反應,但目前中國政府和社會對基本原則有著根深蒂固的誤讀,只有先破除錯誤的思想,才談得上針對性的理性策略。
施行到巴基斯坦這個事件上,我們的確面臨一些複雜的細節,例如整個爭議至少在表面上依舊遵循法律程序;美國的黑手相對隱蔽;現任總理除了對民衆打打嘴炮,並沒有正式要求外來幫助,等等。但是我認爲正因爲中方不必過度反應,這反而是修正過往縮頭烏龜策略的大好良機。在新聞報導證實美方資助反對派,然後Khan批評有外來影響之後,外交部至少應該表示關注,並且把握機會嚴重批評美國任何“政權更替”(“Regime Change”)的作爲,將其與“侵略”(“Aggression”)劃上等號。私底下則可以考慮通過軍方,向Sharif和最高法庭施壓,要求他們公開宣稱拒絕美方壓力。這種純粹耍嘴皮子的表態是國外政壇的日常,然而不論實際後果如何,從中東國家的觀點來看,中國至少願意直面對抗英美的顛覆企圖,也就成爲更可靠的朋友。
提升一個層級來看,中方外交應該把握“政權更替就是侵略”這個論點,狠抓猛打。例如上周Biden在演講中提出要推翻俄國政府(“Cannot remain in power”),連歐盟都不敢應和,然而中國外交部卻一句話都不吭,就是錯失了大好良機。更積極的做法,包括各式各樣聯合宣言,以及在國際組織提起決議,要求把任何政權更替的主張都明確標識為非法的顛覆侵略,並且把昂撒假新聞體系也拉進來痛批。這樣一來,不但立刻增强第三世界對中國的向心力,而且真要出手,也師出有名,甚至可以是多國聯合的協作。設定前例以後,這還可以作爲歐亞經濟共同體等新組織(參見《從SWIFT制裁俄國,看中國的對應之道》)的國安輔助機制,有百利而無一害。
\subsection*{2022-04-09 20:18}

從中方對美國成功在巴基斯坦做出顔色革命(第一級的,亦即純粹體制内,還沒有用上暴亂)毫無反應來看,似乎是部長以上知道這是“百年未有的變局”,但副部以下並沒有準備好要做“百年未有的動作”。

我原本以爲前因後果和邏輯主軸都剛剛在博客反復討論過,所以點明思考方向就夠了。但還是收到一些回饋,説什麽“巴基斯坦不論誰上臺都必須維持和中方的關係”等等的傻話,所以在這裏囉嗦一些照理應該是不言自明的論述,有邏輯思辨能力的讀者如果覺得是廢話,請耐心包涵;需要別人來解釋這些基本道理的人,則遠遠還沒有足夠資格對國際戰略議題做評論建議,請先確認自己已經走下Dunning-Kruger曲綫左端的笨蛋峰、開始虛心學習、向知識理性方向進步。
這裏Khan和Sharif哪個親中、哪個親美,根本無關緊要,就連Sharif推動不信任案的背後到底有多少美方的幫助都不是重點。中方真正必須關心的,是中東產油國的觀感:既然Khan剛剛高調拒絕支持歐美對俄國的制裁和譴責,即使他所説的“進口政府”言過其實,以巴基斯坦和中東政壇之熟悉、中巴關係之密切、以及美國在波斯灣的駐軍,都必然會引發聯想,成爲沙特和阿聯酋等國掌權者在考慮和中國進一步親密合作時,安全方面的重點觀察參考事項。我反復解釋過,全球化已經正式終結,國際體系即將分裂為兩個陣營,中方爭取的頭號對象就是中東產油國;然而歐美沒收外匯、極限制裁的做法固然引發强烈反感,但如果任何動作都會帶來顔色革命,而中國又必然會見死不救,那也只好忍氣吞聲。這才是中國在對此事做決策上的第一優先考量。我知道庸人思維有高度慣性,原本不能指望基於第一原則的隨機應變,但目前局勢演變的速度太快了,非得當場立作決斷不可,這才是我說必須要做“百年未有的動作”的真實涵義。
\subsection*{2022-02-27 01:19}

這和韓戰沒有可比性:當時美歐日仍然代表著全球絕大半的工業經濟,雙邊敵對性大幅上升的結果是經濟貿易體系的徹底分裂脫鈎,然後華約陣營自然落後越來越遠。現在單是中國的工業產值就相當於美、德、日之和,印度、土耳其、沙特、巴西、南非等區域强權也都不願為英美站隊;俄國作爲資源出口國,靠著這些顧客一樣活得滋潤,同時可以憑藉國内向心力的增强,繼續改進社會風氣和工業基礎。中國則完全可以披著軍事中立的表象,一方面保持對俄國的經貿資助,另一方面推動全球體系向有利自身的方向演變。
綠營中的確可能會有些不是完全腦殘的人,從這次的案例理解到美方策略就是存心讓他們當炮灰;然而我對普羅大衆的平均智商一向不抱太大的希望(而且我對人物評估即使出錯,向來也只有高估,而不是低估),尤其是台灣這樣經過30多年反智教育洗腦的社會,這樣的反省極可能是Too little too late。
\subsection*{2022-02-26 01:26}

我覺得中方面臨的經貿局勢和俄國不同,決策階層的思路也更偏保守,提早到2022年動手的機率實在不大,所以我七年前的估算依舊有效,並不需要修正。
俄軍的進展不是特別迅速,在Kiev和Kharkiv都有激烈戰鬥,可能是因爲臨時起意,只有不到兩周時間做部署,所以準備不是完全充分,尤其在電子戰方面,還沒有徹底壓制,因而沒有在24小時内就獲得絕對制空權,這讓我有點驚訝。我想共軍必然正在密切關注做筆記的過程中;當然戰爭越是摧枯拉朽,平民受災越小。

今天有報導指出,俄軍對民用基礎設施非常小心保護,電廠、水廠、電信中樞等等都完好無缺,這可能是讓進度稍慢些的因素之一。至於優先占領Chernobyl,那明顯是爲了避免烏方炸毀防護盾、釋放放射性物質來泄憤。這些考慮都有簡單的臺海類比。
希望你打字沒有困難。
\subsection*{2022-01-07 20:22}

War is the perfect justification for taking territories. Besides, I was referring to the option of occupation, not annexation. Okinawa can and should be fostered into independence, since after China defeats the US/Japan alliance, there will be no one left to impose diplomatic costs. Quite the contrary, beyond allowing China to completely dominate East Asia and thus pull South Korea off American orbit, the occupation serves as a reminder, even to the US and Australia, of the very real risk and cost of continuing to oppose China.
\subsection*{2022-01-06 19:50}

While your statement about the two countries' military weights is largely correct (i.e., their participations in a 2025 war of the West Pacific would not change the outcome), there are some fine points worth mentioning: Australia can serve as a base to launch strategic bombers and long-range support aircraft, and Japanese navy/air force is not exactly negligible, particularly in regards to submarine/ASW. Furthermore, should Japan enter the war, it would be logical for China to mount an amphibious assault on Okinawa, which would certainly require allocation of additional resources and other preparatory efforts.
As to why these two nations exhibit such irrational hatred towards China, we have had discussions scattered over multiple entries before. In short, it starts with a deep-seated racist mentality, then exacerbated by propagandist smears over the years.
\subsection*{2021-12-30 12:58}

博客早年就解釋過,美國建制派向來的長期戰略計劃,都是如果發生武統,就只在事後做出外交、經濟的制裁。但我一直沒有談美國可以封鎖臺積電的選項,原因是只要臺積電依舊是芯片代工業的龍頭,美、歐、日都承受不了一夕之間喪失全球過半產能的衝擊。現在臺積電被迫在美國和日本建廠,其實是美方擔心武統之後,反過來被中國掐脖子,那點海外產能並不足以扭轉整個行業對台灣的依賴。
我正準備再寫一篇新年的回顧和展望,思考的問題之一就是Biden的低能外交團隊是否在Sleepwalking away from the established strategy;換句話説,他們是不是笨到沒有先和國防部協調,就想當然爾地以爲美軍可以輕鬆干涉臺海戰爭並獲勝,因而會繼續主動鼓動兩岸衝突升級。照常理説,料敵從寬,中方不應該假設對手會蠢到那個程度,但美俄過去幾個月針對烏克蘭情勢的折衝,偏偏就指向這個脚本;例如最新的消息是美俄雙方都有軍方人士參與高峰會談的準備,這只能是美國外交體系太過癡呆,嚇壞了國防部的結果(因爲Putin自己掌握全局,沒有讓軍人參與外交談判的需要,所以這個建議必然來自美方,而Blinken和Sullivan又不可能主動讓權)。至於進一步的考慮,大家稍安勿躁,等博文刊出再來討論。
\subsection*{2021-09-20 21:22}

人生不如意事十之八九,希望你保持樂觀積極的態度。這裏的老讀者群和你神交已久,我相信大家都在祝福你傷勢早日好轉。人入中年之後,肉體必然是不斷纍積各種傷病,只能在智慧層次上繼續追求進步;我想過去幾年你在這方面是有所成就的。
年初我對中國的外交宣傳多所建言,擔心的只是短期戰術上的得失,在長期戰略上中方的優勢從來沒有任何疑義。而且新疆議題的炒作,雖然暫時有所消停,卻也絕非雨過天晴。事實上,今天才又有美國媒體發文,檢討比較各種打壓中國的手段,結論是新疆人權依舊是最高效的方向,參見https://www.theatlantic.com/international/archive/2021/09/us-china-human-rights/620112/
多年前我曾解釋過,對象越是愚蠢,越難以邏輯來推論他/她失敗的方式過程,雖然最終的失敗是可以簡單預見的定局。Biden上臺之初,我們當然必須對他的外交政策團隊從寬預料,結果他們竟然比Trump政府還要離譜,屢屢跌破任何合理估算的下限。美國的政治體制是利益集團的分贓平臺,對内永遠都不可能真正解決貧富不均、種族歧視、性別對立那些不平等問題,對外也自然會堅持殖民霸權主義,但在戰術上並沒有非得胡搞瞎搞的先天必然性。如今三億多人選出來的兩黨執政精英,卻都擁有類似太陽花群衆的高中生思路水平和政治手腕,這只能歸咎於整個國家社會全方位的腐朽衰敗,其中又以社科學術界的墮落,影響最爲重要深遠。
我也認爲新版的Taliban具有遠高於英、美、澳、台、印執政階級的思維和視野。
\subsection*{2021-09-11 20:36}

不是“拜習通話”,而是Biden特別打緊急電話給習近平。其他的話題都是過場話,真正的用意類似今年四月,烏克蘭準備對東烏動武,俄國隨即調兵列陣,Biden也是緊急打電話給Putin,保證美國不想開戰,然後邀後者在六月見面,又當面談了一次。在長期問題上,雙方依舊各説各話,什麽改變都沒有;Beden唯一的用意,就是避免軍事衝突,以免又在國際上被羞辱。
這次打電話給習近平是完全一樣的道理:中方利用這個機會抱怨美方的敵對性行爲,然而Biden根本不在乎,整個交談的用意,只是要求中方不要馬上打台灣。這並不代表台灣、Lithuania和美國自己的小動作會停止,美方的信息純粹就是求習近平不要在Biden的任期内動武。
和平條約?爲什麽不順便要求中國割讓福建省?再加賠款一千萬億美元好了,否則國軍在一天内掃平大陸。
\subsection*{2021-09-01 21:58}

不會了;自從2014年香港占中之後,中方就開始慢慢明白一國兩制根本沒有實際的可行性。參見我在2015年的討論,以及《再談統一》這篇正文。
説得更詳細一些:正文裏的條件2,早就成立,只不過中方曾經有些官員單相思、自作多情,不肯面對現實。條件1在2015年還未滿足,所以當年我另給了10年的時間,以等待中國的國際地位進一步提升;後來Trump執政四年再加上Biden主導的Afghanistan大潰退,提前完成了中美國際影響力的消長,也已經不成問題。今年中方内政一連出臺了許多新政策改革:習近平連打擊壟斷財閥、整治娛樂圈子、限制網絡游戲這些表面上完全可以先放下,等完成產業升級再出手的議題,都不願意拖延,那麽你想條件3還有疑問嗎?六年前我說2020年代中期武統的機率是50 \% ,現在很明顯又上升了一大截(75 \% ?剩下的25 \% 對應著條件4),以致連蔡英文都急著大買軍備,那麽我勸台灣群衆最好不要繼續做一國兩制的白日夢。
\subsection*{2021-01-26 10:45}

我的看法剛好相反:Trump在沒有名正言順的Casus Belli之下,把所有可能傷害中方的招數都嘗試了一遍,其中只有半導體禁運真正有效。這一方面讓後任美國領導人明白哪個制裁武器有用,另一方面卻也讓現任的所有中共官僚都清楚脖子在哪裏被掐著;你覺得這對哪一方有更大的教育性貢獻?哪一方有較高的執行力來做針對性因應?
別忘了,美國掌握的是半導體產業的最上游製造器材,而台灣的TSMC卻是中游晶圓廠的世界頭號廠商,下游的應用則是全球都需要的。戰爭是極端敵對的態勢,大家都撕破臉,什麽知識產權都抛諸腦後;這時上游的器材被切斷,只代表晶圓廠無法升級換代,而不是不能生產,只有掌握晶圓廠的人才能控制芯片的供應。當然,韓國的三星也有相當的產能,但仗打起來,駐韓美軍會置身事外嗎?中方在消滅THAAD的過程中,會不會不小心誤擊三星的晶圓廠呢?韓國針對這類誤擊的可能,是否要事前做出妥協呢?這些可能性需要我們詳細考慮。
\subsection*{2020-09-22 22:18}

我已經解釋過,這種話題沒有深入的分析,非常容易自説自話,陷入邏輯謬誤。這裏你由美式經濟學出發,只看交易的立即得失,自然是互利的。但是中方真正的國家利益,不在於今天省個幾毛幾分錢,而在於產業升級,建設全面的工業實力,台灣的項目專長和技術層次都首當其衝。如果不是政治考慮,很自然地會優先替代來自台灣的進口,別說2、30年,只要兩三年就會使台灣對中的貿易順差大幅萎縮,而且連世界市場上都會被中國企業取代。在這個背景下,位於台灣的企業反而必須更加急著搬到大陸。
英國和歐盟的經濟實力對比大約是1:6,脫鈎的結果尚且是前者即將分崩離析、後者不痛不癢;台灣只有大陸體量的1/30,一個簡單的經貿脫鈎會使前者經濟徹底崩盤、後者反而有利可圖。
\subsection*{2020-07-19 04:12}

這是你第一次留言,我不想太爲難你。不過你既沒有言簡意賅,問的也是我解釋了幾十遍的議題。請你回去仔細讀懂了,再來參與討論;暫時先禁言一個月。
我從2014年開始寫博客,就一再揭發美國的醜陋真相;到了2017年Trump準備打貿易戰,我馬上就反復强調這是生死搏鬥、不是能花錢解決的事。這在當時不但不是主流看法,事實上有一兩年沒有任何中方的學者、智庫、媒體應和這些正確的意見,直到中國政府碰了一鼻子灰,這些後知後覺的普羅大衆才看出真相。好笑的是,你去看他們現在的文章,個個都假裝自己從一開始就明白這些道理。
不論如何,我絕不是這方面的外行。我最近明確告訴讀者,歐盟和美國分道揚鑣,中方已經獲勝了。你讀了若是不懂,就多讀幾次;如果還是不同意,至少把邏輯理清楚,用簡潔的語法提出問題。我不是心理醫生,這個博客只接受理性論證,請不要做感性呻吟。
\subsection*{2020-06-07 18:19}

這是我以前提過,局部戰爭和全面戰爭的差別。如果要遠程封鎖,不但必須徹底撕破臉,而且會是持久戰,這兩個條件也代表著對美國本身的嚴重傷害。美國在二戰之後,對外出兵幾十次,在阿富汗的泥淖中更是掙扎了十幾年還無法脫身,但從來沒有事先就明知勢均力敵、而且無法速戰速決的例子,和擁核强權直接對戰更是不可思議。如果歐日積極鼓噪要一起出兵,還稍微有那麽一點點可能,但是歐洲的態度已經轉向了,所以未來歷史學家回顧,2020年必然會被認爲是中國崛起的轉捩點。
Trump再瘋狂,權力也是有限的;一旦觸及財閥的核心利益,就必須打退堂鼓。上月談賴掉國債,本周說要大規模部署軍隊來鎮壓示威者,都是很快就灰溜溜地夾著尾巴回頭,假裝從來沒有發生過。和中國打全面戰爭,比這些例子還要嚴重至少一個數量級,自然更加不可能。
\subsection*{2020-05-03 23:03}

你的擔心是有一些道理的:Trump今年大選民調落後,已經越來越有狗急跳墻的架勢,雖然戰爭對美國自己也很不利,這樣自私、愚蠢、瘋狂的人難保不會破罐破摔。還好民進黨不見得會急著配合,畢竟他們在今年選舉裏以大勝結束,下一場還有兩年,期間各種執政“紅利”已有保障,現在真的閙獨立,豈不是無端去冒損失這些紅利的危險?要閙也應該是選舉之前閙,尤其是2024年的大選;不過届時國際情勢會有很大的發展變動,中國説不定還希望他們鬧大些。
其實比起鼓動台灣獨立,Trump還有許多稍稍不那麽完全瘋狂的選項,例如三天前被《Washington Post》泄露的,要賴掉中方持有的美國國債。不過因爲這些國債並不是可以隔離開來單獨處理的雙邊貸款,而是全球美元貨幣架構的通用基礎工具,賴債就是Default違約,而從1944年開始的國際貨幣交易系統,是建立在美元國債永不違約的假設上的。如果一意孤行,不但美國金融界會受重擊,連美元的地位都可能因此立刻被顛覆,所以金融財閥才會授意大衆媒體提早暴露消息,以便聯合力量對Trump政府施壓,結果是一天之後他們就慌忙地否認。我想台灣獨立引發中美直接軍事衝突,對許多國際財閥的影響更大,所以機率也就更低。
\subsection*{2020-01-14 20:35}

其實在半導體產業的發展上,中共的政策作爲也不高明,事倍功半,過去15年浪費了很多時間和金錢給美國企業的馬甲和帶路人。Trump發起貿易戰,總算讓他們知道美國公司不可靠,但是投資對象仍然沒有絕對優化。
這件事不是我在放馬後炮;5年、10年前就有不少批評的聲浪,但是中央一直放任地方政府去胡搞,補助美國人來打壓自己的民族企業。我在Trump上臺之前也寫過文章討論此事:引進先進技術,不能靠地方政府和市場運作,必須是像高鐵、核電這樣統籌規劃;汽車、半導體搞了幾十年還沒有成效,就是因爲投資的效益都被帶路黨騙走了。現在小汽車正在電動化,這是百年一次的技術更新、後來居上的絕佳機會,千萬不能錯過。半導體則還沒有到摩爾定律失效的地步,所以紫光、中芯等少數真正做硬核技術的企業暫時只能艱難前進,看是否2025年大環境能好轉。
\subsection*{2020-01-12 23:06}

其實對台灣態勢的正確評估和處置,我從這個博客剛開始就反復解釋過了;四五年前的文章,現在回去看,哪一篇不是仍然切題相關?但是台灣的體制正是被設計來保證蠢人當道,再精確的分析、預測和建言,對台灣的政治、經濟、社會一樣毫無效果。沒有被本土臺獨迷昏的,已經是少數,偏偏主要還堅持國民黨建制派的那一套。其實只要馬英九、龍應臺那批人還有市場,本土臺獨就會不斷持續壯大。
台灣的前途,在馬英九白混了八年之後,早已沒有任何台灣人所能左右的餘地。現在只能看中美折衝的結果。每次我看到大陸的一些極左派SB說出“大内自有能人”這種話,就想到中聯辦、國臺辦、中國教育部和中宣部在過去2、30年的胡搞;大陸的對臺政策是否能展現出智慧,並不是一個可以確定的未來。
\subsection*{2020-01-12 00:21}

其實我也早有這樣的感覺,最近看到一些藍營評論員(如雁默)對選舉的樂觀預期,就一直搖頭。不過我對台灣政局的瞭解,來自一些内部深層的人士,反而不方便在博客上公開討論。 
馬英九改正課綱是台灣扭轉命運的最後一次機會,但是那個蠢蛋的八年都白混了,所以我在《八方論壇》上說蔣經國過世前後,台灣對美國過於崇拜,已經種下了禍根。 
總之這次選舉並不是意外,沒有什麽好震驚的。而且即使韓國瑜當了總統,也不見得能有正面的改變,反而有一些可能的負面影響,也就沒有什麽好懊悔的。我這幾句話,有其深意,但是如上所説,必須保護信息來源,不能進一步解釋。 
你如果能換到好工作,就搬到大陸去吧,否則不要意氣用事。我必須以家人爲先,自己的安危禍福倒不是特別重要。
\subsection*{2019-07-20 18:10}

這是一個比較高深的邏輯佯謬,原本我不想在留言欄做長篇大論,所以沒有試圖去解釋囚犯的邏輯錯在哪裏。既然你問了,我簡單討論一下。 
首先你應該注意到,囚犯的結論是原設的前提不可能被完全滿足,但是實驗的結果是至少在某些情況下可以滿足;這是這個佯謬的核心。 
但是一個更聰明的囚犯會注意到,如果前提不被滿足也是一個可能性,那麽反而會使得無法確定老虎的所在變得不但可行而且很容易。 
所以這其實是一個論述無法邏輯自洽的例子,如果你假設前提是對的,那麽邏輯結論是自我矛盾;如果你假設前提是錯的,邏輯結果卻可以推得它應該是對的。這和“本句是錯誤的”一模一樣,只不過更為複雜而已。換句話説,一個邏輯論述,除了對或錯之外,還可能是非對非錯、無法自洽的。
\subsection*{2019-07-16 19:56}

你所描述的,當然也是一種可能,但是我覺得機率小於一半。首先時間的確是在中國這一邊,但是等待不可能是無限的,最終總會有中國夠強,台灣夠爛的時刻。 
有一個很有名的思想實驗,它說一個囚犯面對著五道門,從1號到5號,他必須順著號碼去開門;他也被告知有一扇門後是一隻餓虎,而且在開門前他將無法100 \% 預知老虎的所在。這個囚犯就依邏輯論證如下:老虎不能在5號門後,否則開了前四扇門後,就能事先確定老虎的位置。既然5號門被排除了,那麽在開過前三道空門之後,就可以確定老虎必須在4號門,這也是不容許的。同樣的邏輯說老虎也不能在3號門後;然後再排除2號門;最後連1號門也不可能。他的結論是老虎不存在,於是他很開心地去開門,在第2號門他被老虎吃掉了,也的確事先沒有預期。 
其次,非理性的群眾忽然發現理性,是非常非常難得一見的事。我以往舉過末日教會的例子,預言的末日過了而世界沒有終結,這難道不是一個極大的臨界點嗎?然而80%的教徒不會醒悟,這是人性。
\subsection*{2019-07-12 15:10}

地主階級因為在歷史上和知識分子是同義詞,彼此之間是講情面,禮貌和客氣的;現在大陸人往往被視為自私,無禮,不知收斂,也是土改和鬥爭的結果,因為傳承下來的文化其實來自舊社會底層的貧農。 
我並沒有任何主觀褒貶的意思,純粹是客觀描述一個既有現象。這些文化上的差異不一定是好是壞,要看在21世紀現代工業社會裡,是否能促進生產效率的提升。不過有部分的確是現在就可以說是不好的,例如考SAT和托福,世界上作弊最普遍的是韓國和中國並列,這顯然是大陸人必須反省的事。 
台灣的毛病剛好相反:不但仍然由中國自古以來專門腐化政府的土豪階級主宰,還歡天喜地地接受了英美在近代編造出的全系列洗腦謊言和非理性體制。相對來說,台灣的問題要嚴重得多,所以新疆模式會是一個大大的改進。
\section*{【科技】量子通信和計算是中國學術管理的頭號誤區}
\subsection*{2022-07-29 13:41}

2015年股災我判斷監管單位的處置不合理,結果果然是貪腐。過去幾年抱怨半導體產業政策胡搞,現在證實又是貪腐。狹義來看,一次抓了四五個,是清理門戶的好事。然而拖了這麽多年,騙補都已經形成產業鏈了,依舊讓人難過,更別提對國家的嚴重危害。中國縱容思想腐化太久,笑貧不笑娼的風氣深入各個領域;我批評學術界造假誇大,並不是迂腐,而是簡單預見了這類思想腐朽的後果。光是事後抓人,猶如竹籃提水,空費力氣;必須事事嚴抓,以建立健康文化,讓業内人自相監督,才有長遠的效率可言。
昨天上唐湘龍節目,不知道直播已經開始,以爲還在私下聊天,談到此事,其實並非我的本願。這是因爲這類突發新聞,我處於信息鏈的下游,在能夠多方搜證、補齊視野之前,不應該妄自評論。節目内討論半導體的部分,我就有意回避中方的管理問題。
此外,一般人可能沒有注意到,我上《龍行天下》,視頻訊號通過網絡繞行地球,來回有近半秒的延遲,尤其錄下來的是台灣視角的版本,所以雖然容易出現我和主播搶著發話的情況,並不是有什麽“不禮貌”或“欠缺默契”的現象。
\subsection*{2022-02-17 13:36}

我實在已經盡力遷就讀者的留言,但很不幸的,現代社會中智能低下的人也能接受高等教育,然後在互聯網又進一步被鼓勵養成隨口胡説的習慣,來到博客自然肆意污染。偏偏蠢人的蠢法還各個不同,不花時間心力無法簡單確認;這對我的精力和時間是極大的浪費。
自本日起,新讀者不得留言;過一段時間,我再考慮解禁。

你的這段說法,是典型的行業内小卒子或者是炒股的底層韭菜所接觸到的,由“Entrepreneur”和“Venture Capitalist”互動編造出的自欺欺人騙術。像是你所謂用量子算法“可能”在未定未來可以對AI做“潛在”加速這種理由,隨便幾千個行業都能説得出來,憑什麽只有量子計算是全國五年計劃的頭號優先?如果這些堆砌術語的説辭,真有科學意義,怎麽會每兩年就面目全非?
我實在很想對你的邏輯做逐一駁斥,但你的文字很特別,表面上是理工科論述,但實際上是文藝寫法,輕輕一捏就什麽實質都不剩,連個別因果關係都找不到(唯一的例外在下一段落討論),説來説去,就是“未來”“可能”“潛在”“或許”,這一般是資本炒作公關的寫法,如果把你所用的技術詞匯改爲宗教術語,量子計算換成基督教,也完全説得通。所以我非常希望你是商學院或文學院出身;如果你是工科人,那麽這麽差勁的邏輯思維能力代表著中國(或者台灣?)大學教育的徹底失敗。
至於最後一個論述:“中美都投入那麽多資源”,也是騙徒常用的狡辯術之一,叫做Circular Logic,“因爲投資已經發生所以投資必然是對的”,類似的結論在基督教思想中叫做Theodicy:“因爲世界存在所以其中一切必然都是最完美的”。Theodicy的邏輯前提是創造世界的上帝是全知、全能、而且完美至善的;你的論述也需要類似的前提,亦即科技部和美國投資人都是全知、全能、並且完美至善的上帝。
\subsection*{2022-02-14 19:54}

其實他們過去兩年,拿“玻色子采樣”,在中外反反復復大吹特吹,說比經典計算快了多少多少萬億倍,用來比較的經典電腦,你猜是經過優化的專業程序,還是隨便搞出的低效算法?這裏他們利用的,不止是量子計算機的自然運作剛好就可以描述為“玻色子采樣”(我以前提過,一隻狗排氣,也可以抽象化為量子計算),而且正因爲是無中生有的“用途”,根本沒有什麽傳統的計算機科學研究來優化算法,要等到量子計算的公關炒作在大衆媒體成爲海嘯之勢,才會有學者開始思考如何優化經典計算機與其對應的算法效率。然而就是這樣强行用全用途計算機去模擬狗放屁的競賽,而且還是後發反應的第一波,依舊很簡單就基本拉成平手,參見Physics - Race Not Over Between Classical and Quantum Computers (aps.org) 。只不過這樣的新聞,和大企業財團的利益對著幹,大衆媒體不會登,想找事實的人必須盯著專業學術期刊才看得到罷了。
\subsection*{2022-02-12 07:00}

核聚變法國、日本和英國都在用國家的錢搞,中國政府不能落人後,必須浪費比他們更多的錢?雖然我們討論了中國學術管理和投資,比美國還要腐敗、不智得多,這裏的核心論點,不在於中外的比較,而在於客觀的技術和經濟評估。Tritium污染如何解決?内壁材料用什麽?經濟效益評估能優於核裂變?實用化趕得上減碳需要的時程?這些都是足以否決整個科技路綫的問題,不能空口白話、打馬虎眼過關,必須拿出詳細、確實、工程化的方案,尤其如果出現我昨天討論的“奇跡邏輯”反而應該當作詐騙信號來看,否則我們乾脆花錢多建教堂,以預期上帝現身,爲人類解決全世界所有問題。這是因爲事關國運,Russell's Teapot原則更加重要,用邏輯證明有足夠的前景希望是正方的責任,要求反方提供數學級別的絕對證明,本身就是狡辯,因爲邏輯連證明上帝不存在都做不到。此外光有科普公關(聯想)、或外國例子(類比)都不算因果論述,絕對不能拿來替代真正的專業邏輯論證。
評審的專家不但不能只是本行,而且必須是對應主要質疑的行外人士,例如核污染專業來評審Tritium問題,材料學專家來審查内壁結構,核裂變項目主管來做經濟比較,核工程總師來估算時程。即使這樣合理建構的評審團通過了,也必須把論據公開來讓整個學術界和工業界挑戰。光是幾個“權威”拍胸脯、打包票沒有用,因爲事後不可能追回浪費掉的款項,更不可能挽回虛耗的時間和人才。解決方案必須做到凡是科學界願意去理解的人,基本都同意有10 \% (或在國安相關議題上,1 \% )以上的機率能做得出來、有實用效益才行,否則必然是誰的政治能量大,誰就可以雇人在公共傳媒上胡吹,同時霸占五年計劃的資金,推遲有實效的替代路綫。
\subsection*{2022-02-11 22:00}

關於對Copenhagen詮釋的執迷,以及找科學上不值得花大錢的外國剩飯來滿足科技管理階層好大喜功、欺上瞞下的欲望,我批評的不是潘建偉一人,而是整個中國物理界,尤其是基礎物理。你說物理界會不服,那是必然的,我怎麽寫都不會有影響,But who cares?不但高能所本身是個規模更大的賣國詐騙集團,而且整個中國物理行業也是追救護車跟班式研究的重災區,組織上更加個個都是潘建偉的mini-me,只不過沒有潘那麽成功罷了。我早已公開批評過大對撞機和悟空衛星,如果在這場新論戰中還指望一群學閥站到反腐一方,那就如同台灣人把美國當成救星一樣,不只是天真無知,而且是自殺性的愚蠢。
提升一個Metalevel來看,你的論證本身就隱含著完全錯誤的前提,亦即把學術管理問題説成學術問題。依這個思路再推演一步,就可以說反對潘建偉的意見應該發表在學術期刊上。然而當年漢芯事件中,陳進自己發了一堆論文,熱烈支持他的評審會委員發了更多的論文,揭發騙局的記者發了幾篇論文?即使假設想揭發騙局的是行内專家,你覺得他應該把exposé發到報社還是期刊?有期刊會登嗎?歷史上揭發學術不端的,向來就都是通過報社、電視、博客、國會等通道,哪有寫論文的前例?更何況我批評的是整個行業再加上管理單位的問題,怎麽能當作體制内的日常運作?這是大對撞機論戰中高能所也試過的狡辯術,在邏輯上完全站不住脚,只能騙騙頭腦簡單的普羅大衆。我知道你並沒有迷惘到那個地步,但畢竟還是陷入同一個誤區,所以我再總結一次:我關心並挑戰的,是當前中國的整個學術管理體制,不是潘建偉個人,所以討論的Frame和目標聽衆都和你所假定的不同。
\subsection*{2022-02-11 06:05}

你的理解是正確的。
其實例如徐令予(亦即正文中的“旅居海外的良心科學家”)早就努力多年來解釋這些技術考慮,但又不能説得露骨,否則沒有媒體敢登。即便是委婉、純邏輯的論述,一樣被學閥倀鬼鼓動無數網絡上的無腦噴子攻擊得體無完膚。實話陣營早在2016年我剛因大對撞機爭議有了一點名氣,就邀請我一同發聲,但我覺得潘建偉詐騙集團勢力太大,有著太多外圍的宣傳打手,貿然下場只會白白犧牲,必須等到其他重要議題都解決了,有了可以豁出去的自由,才能暢所欲言。
這篇正文別説《觀網》這類正規管道,就是《微博》也得先經過刪改節錄才能刊出,留言欄討論的細節更別提了。我終究在體制外,可以第一個帶頭衝鋒吸引火力,但後續和邪惡勢力以及他們所操弄的許多蠢蛋的肉搏戰,就只能靠大家一起努力了。這個道理,很多讀者已經唱了一輩子,應該耳熟能詳:“起來,不願做奴隸的人們。。。”
\subsection*{2022-02-10 16:02}

我說這些人是賣國詐騙集團,你以爲是文人誇張用語嗎?老讀者應該知道,我的寫作風格,向來是文人的相反,追求絕對的理性、精確和直白;所以表面上同樣是駡人,博客這裏是經過嚴謹邏輯推演得到的必然正確結論,亦即“事實=》邏輯=》結論=》立場”這個順序,我的對手(即使名義上是“科學家”)卻都是反過來從“立場=》結論”出發,先亂駡一通、隨便扣帽子,如果被迫進一步囘懟,自然只能是“=》狡辯=》造假”。
你舉的例子,查個水落石出照理說是一個像紀委那樣單位的責任。當然一方面這個單位不存在,但另一方面賣國詐騙集團爲了遮掩自己的罪行,狡辯和造假都是必然的;不論你的論述是否精確,實際上必然有許多真實的案例。我一個外人,卻也沒有能力去深究到博客慣常要求的嚴謹精確標準,不過他們是賣國詐騙集團的這個結論,從既有事實證據和邏輯推論,就可以絕對確定。

我以前解釋過,博客的任務分為“建言”和“教育”兩個方向;這裏剛好有個機會做後者的工作,所以我囉嗦一下。
以袁嵐峰所喜歡用的“民科”標簽爲例,首先可以簡單地依照Russell's Teapot原則看出它的事實和邏輯根據爲零,純粹是簡單類比。我以前已經反復强調過,類比聯想不是邏輯,因果關係才是。所以論證中舉例,即使細節正確,也絕對不能以偏概全,從單一的案例推論對所有情況都適用,從而“證明”正面法則。這裏實際上唯一可得的邏輯結論,是正面論述並非不可能,所以可以否決反面法則。在這個例子之中,“沒有學術職位-》民科-》胡扯”的推論鏈,就明顯不是因果關係,而只是部分成立的類比聯想。
其次,邏輯能力較强的讀者,可以提升一個Metalevel抽象層級,把袁嵐峰本人和他的論述也放到事實和邏輯的顯微鏡下來做分析,那麽那個論述其實可以濃縮成爲“王孟源自大,他不可能有資格做評論”。這樣的結論需要什麽邏輯前提呢?已知事實是人類的智商大約成高斯分佈,經歷和教育也是一山還比一山高,一個人的能力和思維層次如何,照理是無法脫離特定議題和長期觀察而事先籠統論斷的。唯一能達到袁嵐峰那個結論所需的前提,正是“我有能力判斷人類思維的上限”,而這個論述所需的必要前提,卻是“我就是人類思維的上限”。所以很Ironic諷刺性的,不就事論事、直接指控別人自大的人,反而在正確深入的邏輯分析下,可以證明恰恰才是真正最自大的。
\subsection*{2022-02-09 19:25}

我之所以把正文那樣寫,是因爲剝離了破解RSA之後,沒有國安考慮的加成(其實是加了幾個數量級的重要性),量子計算就純粹是商業炒作,如同Metaverse一樣,頂多只該交給私有資本去炒,輪不到向國家要錢,更不夠格作爲十四五科技發展的頭號項目。此外,量子通信完全沒有商業潛力,100 \% 靠國安藉口,把這方面的騙術揭穿了,同一個團隊就必須先回答幾年前的案子。
從技術觀點來看,量子計算其實還不如Metaverse,後者完全沒有技術壁壘,而前者需要一個奇跡(亦即一夜之間找到無代價糾錯方案,相當於在qubit數量上一步前進3-4個數量級),才能夠至少在紙面上繼續假裝10年内會有實際應用;這類似氫經濟(所需的奇跡是找到廉價高效耐久的電解方法),而“優”於核聚變(光是有一個神跡出現,讓磁場强度提升10倍,依舊沒用,必須在放射污染、内壁材料、等離子體穩定性和經濟費用上同時出現神跡才行;不過我把等離子所的詐騙排名放在潘建偉之後,是因爲前者至少沒有到股市去騙錢)。問題在於依賴未來可能奇跡的邏輯,已經不算科學,而是宗教了:理論上基督教也只需要一個奇跡就可以完全正名,亦即上帝一夜之間決定明天召開記者會,向世人展示他的存在;個人因此而決定提前信教“重生”是他的自由,但科技部能因為這個可能性而把基督教列爲國教嗎?
\section*{【科研】流行病的起源(下)}
\subsection*{2022-07-24 15:59}

人類破壞原始生態,侵占野生動物的生活空間,是農業興起之後就持續發生的事,而且其規模和效率成指數成長,到了19、20世紀,終於突破臨界點,古代數百年一次的zoonotic瘟疫,間隔步步縮短,現在是幾年就來一個。這不但可以簡單預期,而且實際上也的確被有識之士反復警告,只不過一般愚民事先置若罔聞、事後再編造陰謀論,反過來栽贓拉響警報的人(例如Bill Gates)。
Monkeypox其實是新發現時有誤解、以致取錯名字的又一個科學案例:它的真正原始寄主並不是猴子,而是鼠科動物。Zoonotic疾病當然要求病毒針對新寄主做出基因優化,一旦擴散開來,突變的機會更多,優化的速度也就越高。
不過Monkeypox和新冠有個很大的差別:它是接觸傳染的,所以傳播的速度和方式不一樣,接近我以前在博文討論過的梅毒和艾滋;換句話説,比較慢,而且相對容易隔離。當然,以歐美政府近年的表現,自己拿槍打自己的例子太多,“相對容易”可能還是不夠的。
\subsection*{2020-05-31 23:47}

目前對新冠的研究還在進行之中;這是一個很不尋常的病毒,一方面靠飛沫傳染,像流感,另一方面攻擊對象卻是人體内很關鍵的細胞受體,能夠產生嚴重的病理後果,像HIV。學術界對各國疫情發展的不同情勢,並沒有完整的解釋。我覺得熱帶和南半球,因爲在過去幾個月日照充足,沒有維生素D欠缺的危險,人群對新冠自然有較高的抵抗力,這可能是台灣、越南、印度、印尼、澳洲、紐西蘭和非洲在防治上相對輕鬆的原因之一。當然,像是巴西和墨西哥這樣積極作死的國家,一樣不能幸免。 
我們對人對事,必須堅持科學、客觀、理性的態度,不讓既有成見矇蔽事實和邏輯。台灣控制疫情成功有多方向的證據,尤其新冠能成指數散播,光靠撒謊遮掩,不可能撐上幾個月,那麽即使造成這個成果的確實機制仍屬未知,只要沒有明確違反邏輯之處,就必須接受它是事實,至少是暫定的事實。所以這裏的結論是,民進黨政府的行政水平顯然高於巴西和墨西哥,可能不低於越南、印尼和南非。
\subsection*{2020-04-30 02:15}

我也同意目前對方方只能容忍,但是如果她在其他方面(如逃稅)犯法,並不須要視而不見。至於其他尚未被國外媒體吹捧的砸鍋黨文人、官員,至少將其辭退是該做也可行的。英美財閥在過去40多年改造社會契約,最重要的細節之一,就是强調雇主可以隨意解雇員工,所以打擊這些人的荷包,完全符合英美自訂的普世公理,沒有“人權”問題。
至於中外生活水準上差距所造成在思想戰綫上的不利態勢,那是博客反復討論的核心議題之一。我曾提議利用載人登月來治標,現在新冠疫情已經更好、更快地揭露歐美體制低效無能的真相。要治本,除了積極產業升級之外,還要對美元的國際儲備貨幣地位主動出擊:前者的主要問題是近年引進的美式經營理論,我在談波音的一系列文章裏詳細批判過;美元則也因爲新冠而加速衰落,但仍然是未來幾年中美的主戰場之一。
\subsection*{2020-04-29 15:03}

I always think that China's government-owned media should have devoted some manpower to maintain a database on all major foreign news reports, particularly those related to China. Once these foreign news outlets attempt to rewrite history, China would be able to retort with their own words from the past. Sadly, the propaganda department seems staffed with traitors and saboteurs. We have to go to CBC for a useful catalog.
\subsection*{2020-04-29 00:54}

原本是有點兒危險,會在英、澳這些地方有呼應;但現在Trump身陷清潔劑風波,對中國找茬開始被許多民衆認清是甩鍋行爲,會無疾而終的可能性越來越大了。
我剛剛花了一個多月來强調,官方公關必須堅持理性、占據道德高地。如果對等調查/求償,就等同於認可那是合理、合法的行爲,是跟著對方跳進茅坑打爛泥仗,非常不智。從現實利害來算,中方的美元資產遠大於美方的人民幣資產,而且真的撕破臉對中國崛起進程極爲不利,所以更加必須著重於用巧、而不是用力。這事的關鍵在於其他國家是否呼應,所以合適的中方反擊,其實是拿Trump的誠信和動機來大作文章;這個話題美方絕無勝算。中國爲了日後見面的餘地,並且避免被視爲參與美國黨爭,一直沒有針對Trump本人發聲;但現在Trump的大選局勢急轉直下,不但右翼群衆可以聼得進對他的批評,他本人也開始慌不擇路,中方可以考慮在維持道德高點的前提下,針對性的反復指出Trump甩鍋的企圖。中方越是被認爲和Trump有私人過節,Trump的反中言論就越像是巨嬰的情緒發作。
\subsection*{2020-04-26 06:40}

他們願意爭論H1N1和Spanish Flu的起源地,中國就已經贏了,因爲他們不可能100 \% 證明自己理論的對錯,那麽整個議題就成了“有爭議”的那一類,不再有“普世公理”的光環。 
其實這是公關裏的頭號技巧:細節不重要,要緊的是把焦點引導到正確的討論框架(Frame)。只要選對了框架,爭論得越久越激烈,不論細節勝負如何,從整體局面來看就對己方越有利;而且設定框架可以是隱性的,對手上了當可能還不自覺。例如前天Trump對著Deborah Birx問是否應該注射清潔劑來治療新冠;這似乎是無可救贖的錯誤,但是如果Birx有公關專長,又願意把Trump從他自己挖的坑裏拉上來,其實可以很簡單地轉換框架,挽回局面。 
她應該解釋:新冠是一種Enveloped Virus,亦即它有一層Lipid Membrane,結構與人體細胞膜相同,所以能殺死病毒的藥劑往往也會殺死人體細胞,目前無數生醫研究人員還在辛苦開發特效藥,就是難在要從對病毒的毒性和對細胞的毒性之間找到快樂的中庸(Happy Medium)。而清潔劑和肥皂都靠著攻擊Lipid Membrane來殺死病毒,所以這樣的快樂中庸應該是不存在的。至於爲什麽我們可以徒手使用清潔劑和肥皂,那是由於皮膚是特別演化出來保護人體内部細胞,所以表面是一層死細胞所構成的蛋白質障壁,可以隔絕許多化學品。 
這裏我利用的是一般人只知道清潔劑不能内服,卻不知其所以然,所以把話題扯到其背後的科學原理,Trump的問題就不再是違反常識的傻話,而是探討科學精妙的好奇,這不但是一個完全不尷尬的框架,而且為整個發言討論加上智慧專業的光環。 
中方應該堅持H1N1和Spanish Flu這個框架,因爲不論它們源自何方,美國都沒有對外要求賠償,所以至少也是雙標。另一個合適的框架,是新冠的突變和爆發明顯屬於自然災害,所以外交部可以說拿新冠疫情來求償,就好像恐龍埋怨墨西哥沒有好好抵抗隕石一樣;不論歐美在細節上怎麽扯皮,這個框架先天就把新冠定位為自然災害,然後中方可以悠閑地指出2008年的金融危機卻絕對是人爲的禍害。 
最後,我想提醒你,正文裏沒有說知識分子必須誠實,我説的是知識分子必須對自己誠實。上月我批評外交部搞陰謀論,也不是爲了它是謊話,而是因爲它是會導致論戰嚴重失分的謊話。
\section*{【工业】【能源】永远的未来技术}
\subsection*{2022-07-21 13:24}

是的;推崇電動車的人,以前就有,但最早明白指出這場產業革命對國際地緣政治有重大影響的,的確是這裏。事實上,特殊管道的聯絡人在看到博客那條留言評論之後,理解其重要性,特別指定要求寫入後來邀稿的正文(參見博文《2022年國際局勢的回顧與展望》)之中。
很不幸的,在大衆輿論場上,我早年的正確評論有些被忽略,以致謬誤的胡猜以訛傳訛、至今不衰。這裏我指的是世界主要工業國家之中,只有日本全力投入氫汽車路綫的決定:大陸有很多人腦補聯想為獨占專利的影響,其實2010年代早期豐田曾經是Tesla的最大股東,而且計劃很快推出自己的鋰電池汽車,因此他們在電動車方面的專利一樣領先。更早,半導體的專利也沒有阻止產業從美國轉移到日本、然後從日本轉移到韓國和台灣。這裏全世界選擇電動車而不是氫汽車的理由,和專利一點關係都沒有,純粹就是技術的優劣問題。
\subsection*{2021-10-23 12:52}

電動車要從40、50 \% 的占有率,上升到80 \% 以上,最大的阻礙在於充電樁的普及,而且這不是產業自己能解決的。我很擔心中國政府沒有足夠的遠見,關注和投資不足,事到臨頭才試圖亡羊補牢,如同現在的缺電問題一樣,所以正在計劃專門寫一篇文章來詳細討論(當然,博客對能源問題已經提早幾年警告得極爲明白,但言者諄諄、聽者藐藐,也沒有用;這正是所謂的Cassandras curse)。
另一個附帶的觀察,是電池固然是電動車的頭號關鍵技術,但是Power MOSFET也是一個重要部件。當前的主流是SiC,五年後可能會進化為GaN,中國在這兩方面的商業應用都還沒有達到第一梯隊。要做追趕,在行業營收全面爆發之前會容易得多。這又是一個中國政府需要先見之明的角度。
至於電池的技術選擇,反而不成問題;私企達到規模之後,完全可以自行嘗試解決。鈉離子電池只是可能的路綫之一,更熱門的還有Solid-state Electrolyte(雖然我個人不看好;因爲電池最大的技術難關,向來都是電解液和電極之間的界面,把電解液改爲固態徒增其困難,感覺上是自找麻煩)等等,最終哪一個勝出取決於成本、安全性、甚至是推銷能力那些細節,政府不必操心。
\subsection*{2021-10-19 12:30}

一般人只關注發電容量(Generating Capacity),其實越環保間歇性就越强,應該看的是實際年發電量;換句話說,不能拿不同類的電力來源以MW和MW相比,至少得用MWh。一個簡單的Rule of thumb是:核能持續發電能力為100 \% ,水電、風電2/5,光伏只有1/5。然後這樣的間歇性要求儲能上的配對投資,耗費一下子加倍,這還沒有考慮長程輸電所需的資金。
不過正如你的分析,光伏/風電+儲能的大規模全面應用,不但沒有任何基本的技術難題,連價格都已經接近可以和煤電直接競爭的門檻,只要繼續批量投產,經濟性完全達標指日可待。國家的責任,在於推動儲能和輸電的建設發展;這樣明確的邏輯結論,我在過去六年反復解釋傳播,結果歐美政府沒有理性倒也罷了,中國也被忽悠到核聚變和氫能源的歪路上,直接導致儲能和輸電建設落後需求,這是現在缺電限電的直接原因之一,經濟損失以百億計。然而若說肉食者鄙,就抓錯重點了,畢竟這是一個專業性高的議題;真正的問題在於學術界沒有人敢出來爲國説實話,縱容騙子滿足小圈子利益,明顯錯誤的認知被接受為“常識”。唉,楊先生這樣的國士終究還是單獨的異類。
\subsection*{2021-09-30 11:19}

他說的基本沒錯,只是省略了一些新發展和細節;整體來説,不算離譜。
(1)目前批量投產的儲能電池,95 \% 以上是普通鋰電池,這的確有安全性問題,但是行業已經開始向磷酸鐵鋰和液流電池轉換,實際上沒有人認爲十年後還會繼續用傳統鋰電池。(2)氫能的危險性在於零售應用,用在電網儲能時的問題是經濟性,這一點沒有錯,不過我始終也是這麽解釋的啊。(3)水電如果存在,當然是最理想的,問題在於一方面它很有限,另一方面水資源是一個比電力更爲短缺的東西,發電/儲能往往不能是水庫的最優先任務。(4)同上,如果建成了,抽水蓄能的運作效率高、成本低,這也是實話,但前提是必須忽略修建的花費和選址的困難。
我以前已經解釋過,抽水儲能如果是在河川截流的水庫發電站做雙向運行,經濟性還可以説得過去,如果是另外專門挖蓄水池,那就是庸人自擾;以下針對後面這一點做詳細論證:
抽水儲能站必須滿足幾個條件:在一個豐富可靠的水源(中型以上的湖泊或河流)邊邊,剛好有一個懸崖,懸崖上有一個丘陵平臺,既平坦又寬廣,足夠建造人工湖,湖水冬天也不能結冰,而且這個山丘地質必須很穩定,能夠承受山頂新增的重量而沒有土崩危險。這裏我簡單估算一下:假設儲能容量為100GWh,這相當於3.6*10\^14J,再假設高度落差為200公尺,這對應著大約0.2個Gigatonne的水容量,相當於600座帝國大廈,對地基的要求非同小可。此外,假設人工湖平均深度為4米,則這個人工湖直徑大約為8公里,山頂平臺上若是有如此寬廣的平地,早就住滿人了。
這裏的基本問題在於重力位能的密度遠低於化學能,實際上選址只能找到直徑250米級別的地點,那麽容量只有前面計算的千分之一,亦即100MWh,用電池的話,相當於1200台Tesla 3的電池組,堆不滿一個小倉庫,建造價格更是差了兩三個數量級(這還沒有考慮地價和電力傳輸系統),所以相比之下,抽水儲能完全沒有競爭力,根本不可能普及。

有讀者反應,張談的應該是在同一個峽口建上下兩個水庫,既發電也可以儲能。但是他給的鏈接會導致UDN的留言欄發生格式錯誤,所以我必須刪掉,只好在這裏回復:
是的,上面我早先的回復版本可能有誤解/會引發誤解,我已經做出合適的增刪修訂。至於爲什麽我原本直接談人造池塘的儲水方案,這有幾個原因:(1)博客以前已經解釋過,水庫發電站逆向儲能的方案,在經濟性上一般是可行的,問題只在於適合建的河流不多,以及水資源的運用有超越發電或儲能的考慮;(2)水庫雙向發電儲能是很老的技術,不應該還有什麽爭議;(3)現在美國的Start-up,談抽水儲能時,指的就是新建人工池,這樣選址比較有彈性,至於浪費錢,那原本就是他們的用意。

爲了徹底澄清我的看法,在此再加一個總結:建雙重水庫的儲能站,原則上絕對是值得考慮的,執行上必須依個案的特點,做性價比的分析,但是在大局規劃上,水電儲能先天就嚴重受限於合適地點的Availability,不可能成爲低碳能源體系的主力。
\subsection*{2021-09-28 21:47}

一般人往往不知道,核聚變比裂變還要早發現,根本不是什麽“下一代”的突破;剛好相反,裂變才是當年取代聚變、解決困難的神奇新科技。聚變之所以被無知群衆拿來當未來技術,恰恰是因爲它先天的基本缺陷無可跨越,導致90年下來還不如裂變發現後頭5年的進展。那些科幻敘事,把裂變到聚變說成“進步”,等同於說人類終究必須進化成猴子,或者汽車必須換成老鼠拉車,所以要求幾十年和幾千億美金來對人和鼠做基因改造;對外行人或許可以裝扮成高科技,但實際上是純粹的虛功,偏偏他們有辦法把這樣的計劃寫進十四五。
過去半個多世紀來,聚變研究人員的慣例是說還要“30年”(說“50年”的也有,是廉恥心沒有100 \% 被狗吃光的人,但不是多數)。不過最近5年,恰恰在NIF這類“大科學”計劃Crash \& Burn的背景下,誇口越來越離譜,對新的金主開始改口為10年或甚至5年,其基本原因是氣候變化的證據越來越明確,所以減碳政策越來越緊急,眼看其他真正有用的科技(太陽能和風電已經實用化,由裂變輔助的儲能電池是最後一步)即將完全成熟,10年後這套騙術將徹底失去市場,所以只好拼命撈最後一把了。
\subsection*{2015-10-20 00:00}
中子的穿透力很强,这是因为它是电中性的,所以所谓的中子屏蔽,衹能靠原子核藉强作用力来与之反应。但是强作用力的距离很短,反应截面很小;反应截面最大的是氢原子核,所以水是很好的中子减速剂。

理论上可以把水管装在磁綫圈内部,但是这就有了新的问题:1)磁场必须做得更大更强,但是人造磁场是有极限的;2)内壁或许可以用砖块,但是水管却是承力结构(里面是极高压超临界水)。

仔细想想,中子携带了核聚变反应后的大部分动能,电厂必须把它吸收到水里,所以你看到的那些砖块并不是用来屏蔽中子的,而是用来屏蔽等离子的。这是因为等离子体遵守波茨曼分布,不论磁场多强,总有一些离子的动能特别高,能够突破电磁障壁。

因此,真正的中子屏蔽其实正是那些水管,问题的核心也就在这些水管,它们不但是承力结构,还承受了全剂量的中子轰击,因此必须经常更换。人类还没有发明能在高放射性环境下,经常更换承力结构的技术。

这些水管是远远不足以屏蔽全部中子辐射的,所以连磁綫圈也会需要定期更换,衹不过是没有像水管那么频繁;不过总体来看,仍然是每隔几个月就必须把已受高辐射污染的整个反应器拆开重建。人类若是有这个技术,还是先把车诺比和福岛的反应炉清理一下吧。

正文里的那句话的确写得太简略了,我已更正。\subsection*{2015-06-05 00:00}
你不是学物理的,请不要拿一知半解的认知来哄人。我的时间有限,不能一个一个人地教。文章已经写了,补充材料得你自己去找。如果你的基础教育不够,不能理解,那就不该在这里下断言。美国人写文的步调和中国式的不一样,翻成中文很容易被误解。你只凭着一篇翻译过来的科普文,就要来否定有专业教育的人的意见,未免太自不量力。

磁场控制等离子体的困难,是可以靠技术和材料的进步来解决的。如果ITER还没有完全解决,下一代也必然会做到。

中子的处理问题是物理困难,这才是工程手段无法解决的。

最后说一句,为了自己面子而在留言栏死鸭子嘴硬纯抬杠的,我会直接删除。我的部落格是为了人口中前0.1 \% 的精英而写的;如果没那个水准,要旁听我不介意,要发言就必须以不打扰正题为前提。\section*{【金融】【战略】美元的金融霸权(五)}
\subsection*{2022-07-21 13:11}

我自己也常常只記得説過,但找不到哪裏。
順便和大家解釋一下,新讀者往往以爲我喜歡自誇;但如果你去復習博客早年,就會看出我原本屬於内斂型的中國南方文化(參見前文《訪意大利有感》),連自己曾經領導瑞聯銀的創新團隊,發明世界第一個全自動程序交易系統的事,都是博客寫了幾年才提起的。近年來必須反復、高調、明白指出自己高明正確之處,其實是被逼的:整個華語圈就我一個能可靠地對時事做出成千上萬個原創性評論,自然吸引了一大堆盜用智慧財產的人(很不幸的,這就是當前的大陸網絡文化),有意而且習慣性地竊取我的創新論點、假裝成自己的。我曾專文討論過(參見前文《高能物理的牛屎文化》),丁肇中的諾獎研究被盜竊之後,性格轉了180°,從密不發聲變成又一個胡吹亂蓋的Bullshitter;我至少還繼續堅持事實。不過如果是自己的功勞,明知大陸有幾十、上百個網紅盯著想偷,不特別指明反而是不合理的。
\subsection*{2022-03-23 23:02}

其實這個做法非常理所當然,我前天上《八方論壇》的時候,還寫進自己的筆記裏;但當天談的亂,很多本來準備的議題沒有機會觸及,也包括這一點在内。一直沒有在博客討論,則是因爲Putin過去一個月已經連續幾次有脫離最優解的記錄,不能再和中國政府一樣被歸納於“接近絕對理性”的類別,所以也就不能再對他的選擇做“預測”,頂多只能當作“建議”來發表評論,然而博文和留言的建言一般都是從中方的角度來做的,如果沒人問,就沒有機會談起。另一個類似的手段是對油氣管道做“維修”;今天消息傳來,俄國國内一條主要管道在Novorossiysk附近因“風暴毀損”,必須“關停維修”一個月以上。大家可以拿來和盧布交易這件事一並考慮一下,是否俄方主動升級了反制。
盧布面臨美國的貶值打壓企圖,原本就抵抗得很成功:一開始掉了40 \% ,但很快就恢復了15 \% ,然後穩定下來。既然踢出SWIFT和扣押外匯等金融制裁的作用通道都是要逼迫盧布的持有者趕緊賤賣,俄方除了管制貨幣交易之外,要求歐盟國家以盧布來購買油氣,自然是釜底抽薪的合理策略;畢竟在當前西方濫用貨幣霸權、包括濫印鈔票的背景下,大宗貨品如能源才是真正的硬通貨,拿來和歐盟交換自己被禁用的美元和歐元實在是極度明顯的不合理。不過這是歐美貨幣霸權動搖的反映,不是對其推動促進的重要因素:歐美自行打擊既有國際金融體系的信譽,第三世界國家普遍想要改用替代性國際貨幣,這是一開始我就强調的客觀事實;能否把握機會趕緊建立一個大家都能接受的新貨幣選項,才是未來歷史發展方向的關鍵。
\subsection*{2021-12-17 21:30}

我以前提過,金融是所有學科中非常特別的一類,這個特殊性來自:1)它是高維度、極度複雜的博弈問題,必須實際操作才能掌握細節脈絡;2)一旦有了獨門的理解,可以轉換為大量真金白銀;3)但必須對上述知識嚴格保密,一旦外泄,就不再有價值(例如内綫消息)。所以學術人不可能真懂金融;真懂金融的人有對衝基金搶著要,不可能留在學術界或對大衆做營銷,也就不可能白紙黑字地寫下有用的心得;就算真有異類寫下這類心得,有99.9999 \% 的機率在你讀到之前就已經因爲公開而失效了。總結來説,你所要求的學習資料雖然表面上汗牛充棟,但基本都是無用的假話、廢話。
至於要看破美式經濟學的謬誤,其實對有嚴謹邏輯思路的人來説,容易得很。剛好我這兩年輔導了小孩的幾門經濟學課,他最不喜歡的就是我會自然而然地對著那些教材開罵(不是因爲他反對我的論證,而是對他準備考試有反作用)。我罵的是20世紀中期以後的大批經濟學理論,它們普遍基於與現實完全無關或甚至相反的邏輯前提。例如Coase的Property Rights Theory,號稱侵占社會公益的Externality問題來自產權不清,所以只要把全部社會公有財產私有化,然後讓資本在法庭互告就能解決。這個論述之離譜,是真正的罄竹難書:首先,社會公益先天就很難、甚至不可能以金錢衡量(例如正義、人命和文化思想,對應著法律、醫療和教育),所謂“產權”根本無從定義;其次,如果去强行談出一個價格,富者必然有不成比例的話語權;然後,實際執行和法庭紛爭都只在所有信息都是透明的前提下,才有可能,然而如果信息透明(事實上,信息不對稱本身就經常是Externality的來源之一),哪還須要用上訴訟這種效率極低、有許多攔路者來分一杯羹、並且可以用資源來做極限施壓(例如Trump就很習慣用專用律師來訛詐、欺壓商業對手)的手段?所以實際結果,就必然是資本利用Coase的歪論,反過來為侵占社會公益正名;這正是我一再解釋過的,美式經濟學人作爲資本的文字打手,奠定了腐化英美社會的理論基礎。
\subsection*{2021-12-16 10:48}

美元霸權除了靠金融系統本身的慣性之外,Nixon用石油來綁定歐日,然後再和歐日一起壟斷工業品的交易,從而間接控制第三世界。現在歐洲有自己的貨幣,早就主動進行過美元替代,第三世界的原料供應商和工業品消費者也都有了替代美元的意願。俄國的脫鈎準備主要針對歐盟:在於金融上脫離SWIFT(中方也已完成),以及在能源出口和工業進口上獲得替代;這裏中國剛好相反,需要的是能源和農產品的進口來源和工業品的出口市場,所以這些外貿關係的關鍵在於亞非拉,不但不是不作爲的藉口,反而是影響第三世界、對美元主動出擊的杠桿和底氣。至於什麽美方會翻臉云云,更純粹是瘋人院的夢囈:美國打擊中國還有矜持嗎?有什麽剩下的牌在手裏?別説俄國替換美元毫無外交後果,就看歐盟只是美國軍事外交上的附庸,都有足夠的勇氣和見識來用自己的貨幣做外貿替代,中國如果怕美方的報復,就是典型的拿自己的影子來嚇自己。
\subsection*{2020-06-25 15:30}

有時我也覺得人生沒有什麽好眷戀的,支持我活下去的就是兩件事:自己的家人,和我對人類後世所能做的貢獻。
一個人如果相信了英美和瑞典式的自由主義和個人主義,就主動割斷了與家族社會聯繫的心理臍帶,這時只能自己為自己隨意(Arbitrary)定義出自我實現的方向。雖然這可以是很崇高的為全人類著想,但更可能是財富的囤積或享樂主義。即便是前者,也因爲背後的考慮最終還是在於滿足自我,而受到扭曲,因而發出奇異的氛圍(Vibe),Greta Thunberg就給我這樣的感覺。
説到底,我的人生觀還是儒家“修身、齊家、治國、平天下”由内至外的哲學,既接受先天的群體關係,也定義了先後次序;這個順序只是依難度而列的,其實由輕而重:為國家和全人類服務,並不只是自我實現的手段,而是比提升自我更重要而崇高的終極目標。
\subsection*{2020-06-23 10:32}

1)目前中國對外的FDI有兩大類,第一是為技術,第二是為存錢/洗錢;但是其實FDI也可以用在購買一般的資源、產能、品牌和銷售管道,就近服務當地市場。如果有計劃、有策略地做投資,並不須要產生强大的去工業化壓力。
2)成爲國際儲備貨幣,其實會大幅減低匯率波動性,對實體貿易是有幫助的。真正的問題在於持續的升值趨勢,這的確會對國内的低階產業有殺傷力。
3)人民幣要成爲儲備貨幣,必須先自由兌換,這也的確是很危險的。國内的金融管理人員雖然聰明,但畢竟沒有與華爾街對戰的經驗,只怕無法或無力接下所有的陰招。中國國内自己的炒家,只怕也能繞著監管單位團團轉(Run circles around the regulators)。
4)使用慣性只代表推倒美元不容易,和人民幣適不適合國際化沒有因果關係。在大戰略上,中國沒有選擇,必須與美國做生死鬥爭,而在宣傳上打不過、全面戰爭代價太高,只能從美元著手。換句話説,不論表面上再怎麽穩固,美元依舊是美國霸權的命門和軟肋。
整體來説,擔心人民幣國際化的部分論點在短期内是切題合理的,這也是爲什麽我一直提倡先捧歐元、然後用上國際合成貨幣(例如SDR或金磚加密貨幣)的考慮。當前的戰略目標是打倒美元,以終止美國對外吸血,至於取代它的,並不須要是人民幣;等到中國在國際上全面領先,許多問題能得到緩解,再重新檢討人民幣的定位也不遲。
\section*{【外交】【傳媒】現代英美的假新聞體系}
\subsection*{2022-07-09 02:14}

如果你仔細去研究那段新聞史,就會發現Lippmann的用意是爲了正面批評、敦促改進;而《紐時》在其後的30年,也的確有明顯的進步,到了50年代冷戰高峰,居然能夠在全美主流媒體一致詆毀共產集團的同時,獨善其身、樹立理性媒體的榜樣,盡可能地只用中性敘述。當然,這樣的操守並沒有維持太久,在60年代,學運世代進入社會,也把意識形態至上的愚昧自大帶進新聞界,迫使《紐時》當時的總編發出内部備忘錄,要求基層記者不要强推白左思潮;然而我們半個多世紀後復盤,可以簡單看出他力挽狂瀾的努力毫無功效。
美國由盛而衰的種子,其實早在60年代就種下了:有效市場假説、白左學運和1968年開始的黨内普選,都為當前美國社會和政局的混亂,奠定根基;而其共通點,就是非理性、不嚴謹的自以爲是。我對中國社會的非理性因素素來不假情面,正是因爲長期的危害實在太大了。
\subsection*{2022-01-02 22:52}

I find Google News to be a very helpful tool, since it not only learns from my viewing habits but also allows me to manually adjust its recommendations. But ultimately, the deciding factor is how you interpret the information. This requires an ocean of pre-existing knowledge, the correct cognitive framework, as well as speedy yet accurate logical deductions.
\subsection*{2021-11-01 15:10}

這和正文裏的例子有很大的差別:英國的媒體說是“自由”並非虛假,他們撒謊造謠歷史悠久,完全沒有法律責任。當然這其實是爲了保護媒體背後的財主,但不論原因何在,爲了八卦新聞和整個體制+傳統對著幹不但無濟於事,而且惹人嘲笑。正確的反擊是在媒體/自媒體的層面反過來嘲笑《Times》,這時像是《RT》這樣有足夠受衆的英語管道,價值就再一次呈現出來。
正文裏談的是違反國際法的嚴重指控,而且對方無法用“誤解”來開脫,所以升級訴訟才值得考慮。
不過可以在外交部指定一個專人來盯China Africa Project,收集這類笑料/醜態,下次有記者問起“中國歧視非裔”、尤其是這個組織的新指控,就可以把這件事挖出來鞭尸。
\subsection*{2020-05-13 12:24}

利用這次新冠提供的機會,把中國對外的包裝定位為“官方負責任”和“人民有實惠”是值得深入考慮的方向。 
疫情是人類歷史的一個轉捩點:既有的國際霸權醜態畢露,必然會惱羞成怒,加倍努力來抹黑中國。世界即將因此而加速分裂為兩個陣營:親中和反中,後者包括美國(含英、加、澳、新)、歐盟核心(法、德和白左國家)和印度/越南這類與中國有邊境爭執的國家。對反中國家的一般百姓,因爲他們媒體的持續謊言轟炸,中國説什麽都沒用;況且世界的新任頭號强國,不可能對人人都討好。所以中方的對外公關,必須以中立或親中國家為目標聽衆,兼顧歐美的少數理性思想精英。那麽以往的否認八股固然不合時宜,陰謀論也有很大的反作用,只能主動、正面、誠實地理性敘事,並且避免糾纏在詞匯定義這樣鷄毛蒜皮的細節上,簡單明瞭地直指歐美的痛楚。既然對方已經扭曲、霸占了“自由”和“民主”,中方去爭執定義會像是在吹毛求疵,不如用“負責”和“實惠”來反制,其中自然已經包含了歐美是不負責任、口惠不實的涵義。
\subsection*{2020-05-12 11:02}

我的猜想是因爲60多年前大鳴大放被秋後算賬,接著文革更被無限上綱,很多合理專業的諍言成了整死學術界人士的罪證。於是到了中共十一大,寬容來自學術界的批評被確立成爲隱性的政策原則,就像高幹不再判死、罪不及家屬一樣,都是文革之後深刻反省的結果。 
其實全世界到處都是這種不成文的規矩,在民間叫文化,在政府組織就叫做慣例。我們要研究瞭解一個人類社會的體系,除了表面上的法規制度之外,不成文的規律往往也有同等量級的重要性,而且因爲它們常被外人忽視,所以更加容易成爲造成誤解的盲點。例如我最近討論瑞典的那篇博文:一般人的評論常常止步於抱怨瑞典太過白左,所以我才會想要解釋這個現象背後的文化基礎,以及爲什麽它會有極大的國際影響。至於中國人應有的反應,留給大家當習題。
\subsection*{2020-03-09 00:06}

中國人對國際外交的基本哲學態度,的確和西方文化不一樣。你說的這個現象,主要是自卑感作祟。中國雖然在人類歷史的大部分進程之中,是先進國家,但是過去200年卻因爲歐美的殖民和工業革命而落後了。這對應十代人的時間,中外雙方都因此而有了世界觀的扭曲,雖然必須修正,但其實是很自然的結果。
另一個差別是西方人沒有天下爲公的概念,他們的歷史、宗教和文化都是築基在人我之分上面,從近到遠來決定是否親近和善。中國人剛好相反:鄉土觀念使得一般群衆勇於私鬥而怯於公戰,越是臨近的越容易互相眼紅,對外人反而極度慷慨大方。不過我覺得要糾正這些舊中國的醬缸糟粕,不應該靠引進別人的傳統,而是應該提升人民理性思考的高度,從邏輯出發,客觀冷靜的考慮每件事的利害得失。如同中醫、西醫之爭是個僞命題,能通過雙盲實驗的就是科學,通不過而依舊吹捧的就是胡扯忽悠;西方文化也不一定是精華,理性邏輯的態度才是人類思維高度的結晶。
\subsection*{2019-12-27 23:19}

是的。爭取國際輿論上的立足點,不可能通過英美既有的假新聞體系,必須建立自己的英文宣傳通道;這是一個長期的工作,在站穩脚跟之前,可以和《RT》合作,借用他們已經成功吸引的讀者群。
這裏的目的,不是要直接衝擊歐美大衆主體,而是要把我們的觀點放出去給少數願意聼的人,一旦能打臉假新聞的事實被公開陳述,即使大多數群衆聽不到,職業媒體一定會注意到自己被羞辱了,那麽逐漸地就必須收斂,過濾掉特別離譜的謊言。這也是爲什麽英美對《RT》如此咬牙切齒的原因。
這種國際宣傳上的發力,在外交上有直接的後果。歐洲民意裏被白左洗腦最徹底,所以也就最反中的前三名國家,分別是瑞典、捷克和德國。我們在新聞看到它們一而再、再而三地對中國外交製造麻煩,其來有自。
\subsection*{2019-12-25 18:05}

上次越南人偷渡的事,沒有效果,是因爲事實打臉之後,英美媒體的報導仍然是為自己文過飾非。民衆是愚蠢的,你不把論點點明,他們就不會自己想清楚。 
這個問題的根源,又囘到我一再提過的,中宣部沒有建立類似《RT》的英文宣傳組織。如果你有部分英美民衆願意閲讀的通道,就可以拿“假新聞”來做文章。《BBC》和《Guardian》即使假裝沒看到,下次也會收斂。 
現在要羞辱他們,也不是不可能,只要把消息交給《RT》就行了;他們很喜歡報導英美假新聞的醜事。 
很多細節,我以爲大家都想得到,所以沒有在正文裏解釋;但是有些人(例如下面那個留言甚至在質疑宣傳戰的重要性)非要被牽到河邊、而且用琉璃杯奉上才肯喝水,我們也只好浪費時間反復討論。不過群衆不懂是非不是新聞,真正的責任終究還是在於官僚體系本身。
\section*{【外交】【經貿】後新冠世界(一)}
\subsection*{2022-05-12 09:38}

國際貿易上的實質影響很小,這是因爲美國東西兩岸的海關原本都大排長龍;上海出現阻塞,反而有助於他們梳理擁堵。美國政界、學界和輿論拿上海封城來説事,純粹是因爲自己印鈔引發的通脹急需替罪羔羊,所以中俄自然輪流上榜。
國内經濟當然會有相當的損失和影響;不過和躺平所代表的全國醫療體系受衝擊相比,顯然高層認爲兩害相權取其輕。我看不到内部的詳細資料和評估,但感覺這個選擇應該是對的。
有關“信心”受到嚴重打擊,的確是一個驚人的發展,不過我指的不是民衆對政府的信心,而是中央對上海地方官僚、外地對當地人民素質、海外華人對國内教育思想的認知和信心被顛覆。博客反復解釋白種人稱霸全球和種族優越性毫無關係,純粹是地理偶然和歷史遺留,先造成貧富不均,然後在社會達爾文主義之下,他們得以拼命佔他人便宜,使得富者越富、貧者越貧。這裏最可笑的是白種人還花了幾百年來往自己臉上貼金,發展出全套的政治經濟理論,自封為世界燈塔、人類終極。作爲中國對外的主要口岸,上海必須反思自己的幸運,感謝内地同胞所做的犧牲(百年來的城鄉差距,和近年的武漢和吉林封城,都是例子),而不是學昂撒去當精緻的利己主義者。
\subsection*{2020-07-14 02:41}

你不用客氣;Dunning-Kruger微笑曲綫的最低點,其實一般就在博士班程度;換句話說,在拿到博士學位之前,學得越多,對自己的無知認識越清楚,要當上教授之後,才能慢慢重建自信。
我覺得美國體制先天效率很低,但是穩定性很高,出現社會秩序完全崩潰的機率很小。疫情如果繼續惡化,真正須要擔心的還是經濟。美聯儲能做的都已經做了,Trump也不吝於花費公帑來刺激消費,但是一方面新冠對經濟的打擊是全球性的,另一方面美國的GDP太依賴零售,所以五月和六月的復蘇趨勢基本不可能長期持續。我認爲有儲蓄的中產階級所面臨的最大危險還是就業。
在公共衛生方面,Trump的確是在向瑞典式的群體免疫策略靠攏。還好目前維生素D對免疫的重要性已經被一系列雙盲實驗反復確認,甚至在新冠的季節性和黑人易感性上都很可能有重要影響,我建議你開始每天服用,輔以抗氧化劑(如NAC)和抗凝血劑(如Aspirin),並且保持睡眠充足,應該可以把家人的生命危險降到最低。
至於高等教育的因應措施,我其實一直在很密切地注意著。知名大學最擔心的,是如果全部改爲網課,許多新生會申請延一年入學,那明年基本不用招生了。所以很普遍的一個方案是只讓一年級新生進駐校園,高年級生才以網課爲主。Trump政權原本要求留學生的網課不超過3學分,幾天後就退讓為有3學分不是網課;我想這已經很容易繞過去,大部分的留學生不會因此而停學,只是教書的要辛苦些。
\subsection*{2020-05-22 06:21}

財閥在70年代開始資助的右翼民粹口號,就已經包含了對美國現有體制、憲法和開國元老的絕對神聖化,其背後的目的正是要阻斷改革之路。40多年下來,建立了40 \% 人口的鐵板基本盤,所以正如你所説,任何有實際意義的改革都絕無實現的可能。那麽美國的衰落,也就絕無扭轉的餘地;而中方最基本的大戰略任務,是熬過這個過程,而不被美國拉著一起落下懸崖。
這個隱患的一個體現,是中美之間發生全面戰爭。美蘇之間的冷戰,是人類歷史上首次有霸主和挑戰者經歷數十年的對峙而始終沒有大打出手,其原因很明顯地是熱核武器所帶來的MAD(Mutual Assured Destruction,相互保證毀滅)。中國要消除全面戰爭的危險,就必須有能力保證美國的徹底毀滅。反對投資在戰略武器的論調,不是完全無知地以爲武裝越多就越容易開戰,就是拿蘇聯破產為例子,誇大投資的程度。這是狡辯術裏“無限上綱”的技倆,其實在足夠的嚇阻力和破產之間,有很大的中庸空間,而美國完全瘋狂的可能性是不容忽視的。
\subsection*{2020-05-21 02:00}

中國過度依賴美國市場不是問題,畢竟美國是世界的最終買家;問題在於中國過度依賴美國技術,這是有識之士在過去十幾二十年一直大聲疾呼要留意的危險,但是官方一直沒有下定解決問題的決心。
臺積電也在觀望。5nm是2020年的技術,到2024年轉移到美國,說快不快、說慢不慢;不過最起碼在今年年底美國大選之前,臺積電不會真正投入任何真金白銀。即使Trump連任,臺積電必須開始真正執行建廠計劃,明年光是找地點、找員工就會非常頭痛,變數還是很多的。
要抓住中美脫鈎的契機,前提是對兩邊都維持良好關係,以便左右逢源;越南和泰國都準備如此,連日本都很可能會被迫走上這條路。民進黨政權卻是意識形態挂帥,這必然是要出大錢維持的;轉機從何談起?
\subsection*{2020-05-20 13:12}

英國的製造業,因爲起源太早、獨霸太久,其實效率一直很差,小作坊非常普遍,極其依賴廉價勞工,所以德國一工業化,馬上在生產效率上超趕英國,在1870年代來自德國的進口工業品已經汎濫成災。一開始還可以假裝德國產品品質低,所以搞了個法案要求進口商品標明生產國(“Made in Germany”),但是很快連這個藉口都沒有了,於是只能玩政治,在原材料和市場上封鎖德國。
所謂的自由市場經濟,從來在實踐上就沒有真正成功過。效率最高的經濟,如果不是由國家來組織生產力(如德國、日本、中國),就必須靠大財團來執行這個統籌規劃的職能(如美國和南韓)。我在批判美式經濟學的時候,重點之一就是指出大企業對他們的經濟運作至關重要,然而企業的内部卻是絕對極權的反市場、反民主機制。
\section*{【美國】Trump的權力萎縮}
\subsection*{2022-04-26 12:47}

很好,你能把博客解釋過的道理和設定活學活用,應用到新現象上,這正是我反復示範、試圖教導讀者群的推理能力。
光是有錢,並不保證能被既有的幕後權力集團接受成爲自己人,例如Trump成名幾十年也一直被當成笑話來看,中俄富豪更只是刀俎上的魚肉。收買地方政客雖然容易,也只代表你是個土豪,距離真正的“精英”還差得遠。要把手裏的錢轉化為内綫政治(對比著臺面上的政治)的勢力和籌碼,最高效的手段就是買下或建立掌控輿論思想的管道,有耐心的例如Rockefeller設立芝加哥大學、Carnegie創建大學和各種研究機構、以及由許多巨富在1970年代建立或擴張的各式各樣智庫(參見1973年啤酒大亨Joseph Coors設立《The Heritage Foundation》,Koch兄弟在1977年設立《Cato Institute》,《American Enterprise Institute》則是衆籌的典範),急於速成的就必須買既有的主流媒體。《華郵》原本在民主黨系中算是相對獨立、有良心的,Bezos買下後根本不須要主動幹什麽,只要在高層人事上“尊重”建制派的推薦,很快就把整個報系轉化成爲最不要臉的傳聲筒,Bezos也得以從網紅暴發戶升級為國際精英;先例不遠,Musk試圖仿效是理所當然的事,而且《Twitter》不像《華郵》,原本就沒有節操可言,現在換手只是從一個勢力轉換到另一個勢力罷了。
\subsection*{2018-12-26 09:48}

陰謀論很容易編造,但若是沒有證據,就只不過是一個猜想;如果連情理都說不通,那麽就純粹是胡扯,你所提的這個“大牛”就是一個例子。買空賣空搞的都是小股票,一般市值幾千萬美元;而美國的高科技股,光是Apple和Amazon就各是萬億美元級別的公司,說要賣空,那是完全不懂金融的人説的胡話。如果華爾街真要整Trump,花個幾億給他的對手做競選資金就可以保證他的落選,哪兒用得著擔上幾萬億的風險? 
美國的經濟成長週期即將告一段落,2019年再升息已經晚了。 Powell怎麽決定都無關緊要,即使他因爲慣性而在年初加息,反正到年底就反過來必須考慮降息,所以隨便路邊的擦鞋童亂猜也會有接近100 \% 的勝率,因爲不管往哪邊猜應該都能自稱是對的。 
至於世界其他區域的經濟,由於油價反轉,忽然下跌,所以像是印度這樣的大消費國自然受益。中美的互鬥也會讓許多國家漁翁得利,尤其是越南、印尼等等低端製造業的新出口國,可能會因之而完全避免衰退。所以受害者必然是有的,但是發展成97年或08年那樣的大災難,機率很小。
\section*{【基礎科研】高能物理界的新動態}
\subsection*{2022-04-08 16:40}

這裏最大的問題是,即使只看直接測量W-boson質量的實驗,不同的對撞機(包括比Fermilab更新、能級更高的LHC)也已經做了將近40年,以前的4個主要結果都一致(其實還有一大堆間接測量結果也一致),只有這次Fermilab特別高,那麽照理在能解釋與既有實驗結果的差異之前,不應該直接拿來和理論做比較,但是現在的高能物理界根本不在乎什麽職業道德或科學修養,既然“打破標準模型”這個標題比“違反其他實驗”要吸睛得多,誰管你誠不誠實。參見https://non-trivial-solution.blogspot.com/2022/04/do-we-have-finally-found-new-physics.html?m=1
即使假設以前4個直接實驗加無數個間接實驗都做錯、只有這次的實驗做對了,也不會顛覆標準模型;這是因爲W-boson質量對Higgs很敏感,標準模型其實只說會有一個“Higgs Sector”,現在的計算都必須另外假設只有單一的Higgs粒子,但那是十年前LHC實驗的結果,不是理論的要求,被推翻也只是那個實驗結果被推翻,不是標準模型這個理論基礎。
\subsection*{2021-09-02 13:57}

是的,這些恆等式有部分不在乎質數與否,而且適用於多個數值,上面的那個例子是我隨意挑的,剛好屬於這一類;但即使如此,一般人也會簡化結果,那麽只有n是質數的時候,平方根才會存活下來,引人注意。這裏是另一個更常見的例子:Σ tan(m*Π/(2n+1))\^2 = n(2n+1),where the sum is over m from 1 to n;雖然等號右邊沒有出現平方根,但等號左邊的tan必須取平方,可以説有相似性。
只適用於特定角的三角恆等式其實更多,有些用到tan(Π/n)的,形式可以更簡單(例如tan(3Π/11)+4*sin(2Π/11)=√11);而且所包含的,不一定是√n,如果n是p\^q,出現的依舊是√p或p的立方根:簡單的例如sin(Π/4)和tan(Π/8);較難的有cos(2Π/9)\^(1/3)+ cos(4Π/9)\^(1/3)- cos(Π/9)\^(1/3)= [3/2*(3\^(2/3)-2)]\^(1/3)。歷史上,印度數學家Ramanujan特別喜歡去找各式各樣冷僻的恆等式,上面最後一個等式就是他的發現。
這裏的總結是:咋看下似乎有“美麗”的規律,實際上是一系列偶然的巧合,沒有什麽深刻的道理。高能物理走上超弦的邪路,一個很重要的隱性動力是居於領導地位的頂尖人物(例如剛過世的Weinberg)自我心理膨脹,先天假設高能物理必須是全宇宙最深刻、最美麗的自然規則,所以每次遇到路綫選擇,總是基於前述假設來推進,把Occam's Razor抛諸腦後,結果才會反復被大自然打臉,被迫反過來搞出無限多個自由度。
\subsection*{2021-08-27 17:07}

從行内人的觀點來看,我必須再加兩點評論:1)文章裏强調高能物理的哲學成就,但那是50年前的遺產,而且我最近也提過,其實量子場論和相對論也都只是等效理論,這些哲學詮釋並沒有無限深刻的道理,只是某些未知的數學巧合所產生的偶然近似規律;2)試圖替代超弦的新理論,雖然還沒有走上被證僞=》放寬自由度=》失去預測力=》成爲僞科學的道路,但那是因爲這些理論的計算極爲困難,基本都還在出發點附近原地踏步,事實上當年超弦會流行,原因就正是它方便計算,所以拿它們出來做論證,不是嚴謹的作爲。

有關“數學巧合產生偶然近似規則”,我在這裏補充一個淺顯的例子:考慮θ=Π/p或者Π/p\^2,p是任意質數,那麽θ的三角函數值往往會扯出√p。你可以簡單查證p=2、3、5的案例,但是p=7、11的時候其實也都適用,例如sin(Π/7)*sin(2Π/7)*sin(3Π/7)=√7/8。這背後的道理,來自複變函數論裏面的一些巧合恆等式,沒有什麽特別神奇的哲學意義。
\subsection*{2020-08-13 18:35}

Dirac Medal有四種,都是個別三流大學發的,規模很小,其中兩種是給物理的。超弦既然早在30年前就已經鵲巢鳩占,獲得高能理論界的主導權,這行業大大小小的獎項當然都是他們在得(除了Nobel Prize之外;因爲Nobel Committee很頑固地堅持要求必須是真科學)。
上個月網絡上有人議論哈佛物理系的尹希,因爲他一副全心投靠美國的姿態,還說出“科學無國界”這樣的話。這整個爭議我覺得很可笑:他是做超弦的,這種人留在美國少禍害中國年輕學子,是大大的好事,有什麽好氣憤的。此外科學有沒有國界另當別論,但他是沒有資格談的,要談也只能說“僞科學無國界”(剛好這句話我同意,參考王貽芳)。
\section*{【海军】【空军】美军对抗共军的作战计划}
\subsection*{2022-04-08 14:04}

不是,俄軍根本沒有想要癱瘓對方機場。這是因爲搶修跑道並不需要太久,癱瘓機場的用意是在閃電戰下爭取最初十幾個小時的時間,而俄軍原本就計劃打長期消耗戰,根本沒有足夠彈藥來一連幾個月天天去炸十多個機場。
真正奇怪的是至今依舊沒有炸橋梁,而且到第二個月才開始炸油庫,至於煉油廠更是等烏軍先突襲了俄方的才回擊。這裏有些蹊蹺,我討論一下:烏軍派了兩架直升機,趁黑夜偷襲了Belgorod的煉油廠,俄軍根本沒有防備,照理這是值得大吹特吹的事,結果Zelensky居然矢口否認,反而轉過頭來開除了兩個將軍的職位,説他們“叛國”;但是叛國爲什麽只解職不槍斃呢?Alexander Mercouris認爲是Putin事先照會過烏方做了君子協定:雙方都不攻擊對手的煉油設施,結果烏克蘭的兩個將軍自作主張,違反協議。雖然這個敘事有點詭異,但考慮到Putin的幾個預案中必然包括了長期占領Odessa的選項,不捨得打爛在那裏的煉油廠並非不合理;再加上實在沒有其他合乎邏輯的解釋,所以我覺得很可能是正解。
現在俄軍的鉗形包圍已經明顯化,北路是基輔撤下來的5萬人由Izyum走廊南下Barvinkivska,南路則是打Mariupol的4萬部隊開始轉向Bogatyr,應該不到一周就可以打下來,然後整個烏東兵團只剩E50高速公路一條補給綫,接下來戰事熱點可能是對Krasnoarmeysk的攻防。這些城鎮都不大,基本會以野戰爲主,我預期俄軍機械化部隊越打越順。
\section*{【基礎科研】大對撞機不是好的基礎科研項目}
\subsection*{2022-03-18 02:28}

傳統土木工程的問題不在於技術風險和專業忽悠,而是經濟效益的評估,這只能由當地當事團隊仔細分析,我一個外人無從置喙。這類議題應該留給直接瞭解真相,而且願意實名作證的人來討論;除非你有第一手核實的論據指出官方評估結果有錯,並且看過/能反駁支持建造工程者的意見,否則就應該閉嘴,以免傳播噪音污染公共論壇。尤其在這個博客,如果你引用的論述來自匿名的網絡評論,直接違反《讀者須知》第六條,嚴重警告一次,禁言兩個月,再犯拉黑。

在此特別提醒大家,不尊重我的時間精力,把這個博客的留言欄,當成其他網絡論壇一樣隨便發言的人,都有被立即拉黑的危險。這裏的一切討論,必須字字以求真為目標;網絡上到處都是網紅“學者”,專門滿足求爽的意願,一般人用不著來這兒騷擾真正想做學問、求真相的絕對少數。
\subsection*{2020-07-06 22:48}

我可以感受到你建立模型的動機和樂趣,但是這個模型和已知的宇宙相差太遠:你試圖解釋的題目是牛頓之後、Maxwell之前那兩百年的研究對象,這些問題已經被完全解決,沒有任何疑義。目前困惑物理界的是量子力學如何從更基本的原理湧現(Emerge)出來;許多玩了幾十年量子場論的真正大佬(例如Gerard 't Hooft)嘗試過,但頂多只能做些Hand waving,連Toy Model都弄不出來。
我知道這對你是很重要的,所以特別多等兩天,讓訊息在自己腦海裏沉澱過才回復。然而我依舊想不出其他評論,只能建議你放棄幻想,回歸到現實生活。一輩子擡頭仰望天空、尋思宇宙的奧秘,是少數極其幸運的教授才有的特權;低頭看著看脚下的人生之路、享受與家人手牽手的溫暖,是遠為切實的快樂。
\subsection*{2020-06-23 21:30}

你這樣讓我爲難了。我並不是貪戀錢財的人,設立自由捐款的賬號是仔細斟酌後的折中:一方面這個博客其實是一個講座,我花大量時間和精力來教學,收束脩是合理的;另一方面,我不願落入網紅陷阱,為追求流量而犧牲品質。但是即使是這樣的安排,還是有一個小小的危險,就是偶然有讀者會陷入非理性的崇拜,從而付出超越自身能力所及的資源。我在博客一向冷酷無情、往往不給情面,這不但是爲了與鄉愿如馬英九做對照,而且也是在不斷提醒讀者堅持理性態度。
客觀來説,每一百萬個民科裏,頂多有一兩個能做出真正突破。這不是因爲他們智力不足,而是沒有經過多年的科班教育,欠缺相關的“常識庫”(參見兩天前的留言討論),自然無法進一步做正確的推論。
既然你已經沉浸在這個研究議題許多年了,那麽爲了幫助你厘清人生裏的優先順序,我願意花時間看看你的研究。但是我很忙,八小時是不可能的;請你花幾天時間把它整理成兩頁以内的敘述,然後用訪客簿的私訊功能發給我。我只能承諾我會盡力理解你的論述,然後給你一個客觀的評價;事先警告,這個評價很可能不是你想聼的。我想一般人能接觸到的學者中,沒有人比我更具資格來做這類鑒定;希望你能虛心接受我的結論。至於酬勞,還是堅持我的慣例,大家以能輕易負擔得起爲準,自行考量決定。
\subsection*{2020-05-24 21:41}

當簡單的粒子大規模聚集在一起,會產生新的集體現象,這些新效應來自粒子的互相作用,而不是個別粒子的特性,這叫做“Emergence”“湧現”。它乍聼之下很神奇,其實人類的日常經驗裏到處都是,整個凝態物理就是在玩弄原子的排列組合,看能湧現出什麽有意思的新效應。從同樣的原料出發,配方或製程稍有變異,做出的材料就有完全不同的物理、化學性質,可以看出湧現先天就是一個Chaotic phenomenon(混沌現象),也就是最終結果對起點非常敏感,所以要做邏輯倒推也是非常困難的。
你的模型,接近雪花形成的機制,你可以自行找資料深入研究。不過既然它是混沌現象,有無限多的排列組合,用人力一一嘗試不是有意義的工作。
高能物理和擬態物理不一樣,是從可以測量的現象出發,要剝離湧現效應,藉以推測背後基本粒子的特性和彼此之間的相互作用。所以對撞機做爲研究工具,要求了極高的能量級別和投資金額,卻仍是行業主流鼓吹的目標,就是因爲它搞的是一對一的對撞,自動排除了許多極端複雜的湧現效應。歷史上也相當成功,在讓人眼花繚亂的實驗結果中,整理出標準模型。目前的主要困難,是要理解基本粒子和時空本身之間的相互作用,前者遵循量子場論,後者則是相對論。這兩個理論當然也可能是湧現的結果,但是它們在應用上太成功、太乾净,所以如果來自更深一層的基本作用,所需的能階提升必然以百億倍計算,這是不可能用更大的對撞機來解決的。所以王所長要花千億美元來建的大玩具,只提高能階7倍,並沒有任何真正的知識擴展前景,它的用意除了養肥經手人員之外,頂多就是幾篇毫無實際意義的論文。
\section*{【美國】再談美國的腐化}
\subsection*{2022-02-27 02:51}

你説的沒錯,“Way of life”和“Self determination”等等説辭,都是南北戰爭後南方州被占領期間,發明出來凝聚抵抗意識用的,後來不但成功逼迫聯邦政府撤軍,南方得以恢復種族歧視政策,到了20世紀還越演越烈,成爲美國文化的主流之一,參見《Gone with the wind》。
不過這並不代表昂撒文化撒謊抹黑、自欺欺人的傳統,是19世紀的新現象。你住美國,應該知道“Go Dutch”是什麽意思,有沒有想過那句話的起源?它其實原本是“Dutch treat”,意思是“荷蘭人請客”,指各付各的;另外還有“Dutch wife”,“荷蘭妻子”,指妓女;“Dutch courage”,“荷蘭式勇氣”,指爛醉下的瘋狂作戰,等等。爲什麽英語裏有這些痛恨詆毀荷蘭人的詞匯呢?那當然是戰爭需要,而那場戰爭發生在300多年前,亦即17世紀下半的第一、第二和第三次英荷戰爭。知道這個典故的人已經不多,但其中其實還有更驚人的細節:1672年第三次英荷戰爭爆發,英法聯盟攻打荷蘭,後者陸戰失利,共和國首相(Grand Pensionary)Johan de Witt被迫辭職,然後被暴民當街刺殺,吊起來燒烤分食。我一直有點好奇,爲什麽英國人不拿這來説事?難道他們認爲吃掉自己的首相是理所當然,遠遠沒有不肯請客來的離譜嗎?還是在日後神化大憲章的過程中,把“民主制”在歷史上的不光彩記錄也一並清洗乾净了?
\subsection*{2018-12-08 10:06}

你所問的這些玩家,如日本、美國學術界、美國的其他附庸國等等,都不以頭腦清楚、邏輯明晰見長,再加上種族歧視這種根深蒂固的偏見,所以很難用理性來做預測。不過美國很可能不會有英國的風度,自願退居二綫,那麽中美之間的衝突,就會延續超過實力已經明顯扭轉的階段,也就是不只一代人的時間。 
我在這個博客,一直想清楚解釋給讀者,美國不可能永遠滿足於短暫的商業利益,世界霸權所帶來的特權太優裕太舒服,所以一旦中國成長到明顯威脅它的程度,美國内部必然會達成共識,對中國撕破臉、出手做敵對性的遏制和打擊。好在習近平不是一個碌碌無爲的人,現在有如此明顯的挑戰,他必然會督促他的團隊準備好方案。這並不代表軍事衝突;美軍是美國實力衰退最慢的一部分,在未來10-20年内,中國的軍力仍處於劣勢,貿然開戰,徒然是以弱擊強。中方的長處在經貿,一帶一路是正確的著力點;Trump又自毀外交上的盟友關係,中方應該在這些方面,放下以往的含蓄隱忍,只要占有道義的制高點,就可以主動出擊。例如2008年Putin邀請中國一起抛售美債,當時中方認爲中美經濟同舟共濟,所以拒絕了;現在再有這個機會,就不必客氣。
\section*{【基础科研】谈量子力学(二)}
\subsection*{2022-02-12 20:05}

是的,這其實和熱力學第二定律有很强的相關:提高複雜度或自由度就是增加熵,這是自然趨勢,卻與科學研究的目的背道而馳,因爲科學必須能做預測,預測就是確定性,也就是熵越低越好。所以正確的科學方法,必須避免無謂的熵增,這正是Occam's Razor
問題在於中國學術管理體制選拔科研人員,並不在乎他們只知其然而不知其所以然,連Occam's Razor和Russell's Teapot兩個最最基本的邏輯原則都不要求,結果就是假裝是在做科學,其實只是論文生產綫上的組裝工。像是潘建偉這樣從國外帶回整條生產綫的,當然成爲大佬,誰管他懂不懂物理、能不能創新。

中國物理界一連出了高能所、等離子所和中科大這些超大型利益山頭,實在讓人生氣。然而這個亂象再糟糕,終究也是管理問題;良心人不是不存在,而是被逆淘汰出局了。我原本説了氣話,是自己修養還不夠,在此向還在埋頭苦幹的物理人致歉。
\subsection*{2017-09-13 00:00}
BM is not physics, but physical philosophy; the standard is not experimental falsifiability, but logical consistency, just like math. The alternative is the Copenhagen Interpretation, which is not even logically defined, much less consistent.

String is not billed as a philosophy, but a physical theory of everything. There lies the difference, which even 3rd rate college students should be able to see.\subsection*{2015-12-22 00:00}
Occam's Razor 要求最"简单"的解释,而"简单"有时是必须主观定义的。如果两个理论的简单程度很接近,不能明确分辨优劣,物理界就应该把Occam's Razor 放在一边,用实验来分辨好坏。

事实上Occam's Razor 的优先次序一直都是低于实验的,我们衹要求满足实验结果中最简单的理论。

Occam's Razor 的价值在于人的想像力是无穷的,在实验的尖端永远可以发明出Rube Goldberg Machine来满足已知的结果,代价就是要把自由参数增加一个数量级以上。歷史上把自由参数增加一个数量级以上的,从来没有对过;超弦这种增加了500个数量级的更不用提了。
\subsection*{2015-12-19 00:00}
1. 是的,你已经懂了Bohmian Mechanics的精髓之一。不像普通的量子力学衹是Deterministic down to the wave function,它是Deterministic down to the particle itself,也就是完全Deterministic。但是这是假设你能确知现在的量子波和粒子在量子波里面的确实位置。很不幸地,它有一个定理,说任何粒子碰撞很快地就把那个确实位置矇蔽起来了,所以有点像热力学。当然量子波本身也是不可能被确切决定的。

2. 不是的。波动方程式在时间的正向和反向都是Deterministic,所以如果知道现在的量子波,理论上可以倒推过去。\section*{【海軍】美國軍艦撞船事故的分析}
\subsection*{2022-02-08 03:02}

入伍當兵,是很多美國鄉下窮人家庭(亦即照全國貧富五等分的話,第二低的Quintile;軍官則主要來自第三Quintile)子弟謀生之路的首選,程度原本就不高,再加上海軍任務繁重、人手不足,訓練和睡眠時間被犧牲,出低級錯誤是自然結果。
上月落海的F35,也是駕駛員的人爲錯誤:在晴朗天氣下做例行降落(Case 1 Pattern),居然在發現高於Slope之後,大收油門、矯枉過正,導致降速過大,等發現不對勁,引擎來不及恢復全推力(即使是美製的渦發,也有幾秒的遲滯),以致撞上航母甲板後緣,把起落架削掉,機身滑過整個跑道,在另一端墜海。這也是低級錯誤,極可能也是訓練或睡眠不足的後果;不過身爲海軍第一個F-35C中隊的成員,照理應該沒有智商素質上的先天缺失。然而目前有未被證實的傳言(同一個消息來源在上周就正確描述“起落架被削掉”,證明此事的視頻卻是昨天才出現在互聯網的),說駕駛員是女性;這裏並不假設性別和駕駛技術有因果關係,但爲了政治正確而揠苗助長、强行選拔不適任者卻是事故的可能因素之一。
\section*{【美國】【政治】美國崛起時代的治理哲學}
\subsection*{2022-01-13 12:46}

針對資本主義弊病的反思,必須等工業革命進展到一定程度才有可能。歐洲是先行者,自19世紀中期就開始一波一波針對財富特權過度集中的改革,從Bismarck的國家主義到Marx的共產主義,大家應該都很熟;但階級固化的沉厄太深,一直到二戰後才普及福利社會(例如瑞典的Statare佃農奴隸制度一直到1945年11月才廢除!)。
19世紀末的美國才進入迅速工業化的Gilded Age,但因爲城市化比率依舊很低,貧富不均主要只體現在少數幾個工業城市區,南北戰爭又消滅了南方州的大地主農莊,再加上近乎無限的西部擴張容許新移民和赤貧階級無價獲得新農場,結果是佔人口多數的農民普遍比歐洲富裕平等,所以可以拖到老羅斯福執政才開始嘗試改革,於是在20世紀前半經歷了思想百花齊放的階段。
\subsection*{2021-04-13 01:18}

你説的對,這正是我討論E-Government的動力。我一再强調E-Government的重點在於對第一綫官僚的監管,正因爲紀律是嚴格執行公權力的必要前提。
不過“法無禁止即可”的離譜,遠遠超出這一點。自由市場下,不可能事事都有規範,而損人利己的花樣卻有無限多。美國試圖調解這個無解的矛盾,幾十年下來的結果是法條細如牛芒、自相矛盾、動輒其咎;換句話說,法律必然向所有行爲都成爲非法而演變。那麽實際上只有無權無勢的倒霉鬼才會被抓,真正佔全民便宜的,反而有足夠的資源來脫罪,這也是我寫《美國式的恐龍法官》的用意之一。我預期很多讀者看不出那系列文章的重要性;其實博文版面有限,我不可能把所有深刻的含義一次説完,爲了兼顧學習過程的可讀性,必須處處適可而止,先羅列基本案例事實,等待日後對照之用。所以即使自覺已經把博客内容爛熟于胸的讀者,再回頭去復習,也不見得不會有新的收穫。
\subsection*{2021-04-09 23:26}

宗教這種東西,越禁它反而越狂熱。正確的處置是打預防針,也就是拿我最近在留言欄討論過的舊約和新約故事背後的歷史真相來教育年輕學子,讓他們事先就明白這些神話是如何、爲何被編造演化出來的。這些宗教故事其實都很淺薄幼稚,放到真實的深刻歷史背景中,它們的虛僞立刻就被凸顯出來。迷信先天就很蠢,雖然信了之後很難跳脫,但如果在迷上之前解釋清楚,大部分人會覺得很可笑。宗教的一大特點在於虔誠執著,要破解這個吸引力則在於讓他們看來荒謬滑稽。
例如基督教的“歷史”上有所謂的“Theban Legion",說在羅馬帝國正式接受基督教之前,一整個6000人的軍團在公元286年改信基督教,拒絕叛教而被全部殺戮,成爲Martyr。這故事的細節錯誤百出(例如真實的羅馬軍團到三世紀已經大幅縮水,沒有超過5000人的)不説,它來自一篇號稱由Eucherius of Lyon在五世紀中葉所寫的文章,卻又很神奇地順便提到一個六世紀的蠻族國王。那麽簡單邏輯就可以歸納出,要嘛Eucherius有預知未來的超能力,要嘛這是後人僞作的。
我説過了,聖經裏面幾千個故事,很可能沒有一個是完全真實的;這其中已經被史學家確實證僞的,汗牛充棟,要編寫教科書,材料多得很,只在於願不願意找罷了。
\subsection*{2021-04-09 02:22}

我年輕時的台灣,不是這樣的。中文課很少作文,主要學古代詩文和經典;我覺得即使對理工科學生,也不是浪費時間。美國的語文教育,偏重感性,但對假造名人名言,還是有所顧忌的。
像你所描述的,的確是有害無益,難怪大陸網絡論壇上,編造、托僞的風氣盛行,而且技巧熟練、篇幅漫長,非台灣所能及。這種年紀輕,就反復假大空的訓練,貽害極深;那麽中國學術界,造假、誇大被視爲理所當然,甚至成爲職業生涯的絕技,再自然也不過了。願意出面反對大對撞機的人,都是國民政府教育出身,也就不是巧合。方方這類低級文人的普遍,更加其來有自。奇怪的是,現代中國以工業立國,怎麽會在中學階段就全面訓練律師和MBA?教育部腦路之清奇,誤國誤民之熟練,真讓人嘆爲觀止。
\subsection*{2021-04-08 19:23}

我對中國學術界的人事問題不熟悉,無法置評;不過除了你討論的歷史遺留問題之外,是否還有文理分科不合理的影響呢?以我所知的80年代台灣學制爲例,分爲甲乙丙丁四組,分別代表理工、文藝、生醫、社科。這裏社科雖然和理工分家,至少沒有和文藝混爲一談。我以前反復解釋過,只要是必須求真的學術,就是廣義上的科學,必須堅持對事實的尊敬,並且著重培養邏輯思辨能力。如果把社科和文藝混在一起招生,可能導致教育内涵抓錯重點,讓社科變成閑扯淡、純信仰的四不像。受西方宣傳欺騙的,或許還是少數,但我日常看到中國社科教授的文章,大多數誇誇其談、沒有邏輯脈絡、不知所云,實在不像是來自一個健康的學術環境。
中國古代的文人教育,雖然把文藝和社科放在一起,但其實是全科教育,以政治和歷史的學習爲主,書法、詩文爲輔,自然包含了邏輯訓練。現代教育體系分科過細,反而讓社科教育的素質下降,是很大的反諷。
請注意字體大小要與博客整體協調。
\subsection*{2021-04-08 18:13}

法律系統本身的低效和矛盾,在這裏還是次要的考慮,“法無禁止即可”的最大問題,在於危害公益的事情有無限多,而法條卻是極爲有限的。例如王貽芳和高能所詐騙公款,完全合法,但對國家民族卻是極大的危害,難道我就不能批評他了嗎?再說的極端一點,汪精衛依附日本,建立僞政權,也是完全符合日据區的日本法律,難道就沒事了嗎?試圖刺殺他的志士自然違反了當地的法律,難道他們反而算是壞人?是非對錯的標準,應該是公益的最大化;尊重公益,並誠實、深刻地討論如何將其最大化,是健全社會的必要條件。“法無禁止即可”必然干擾妨礙這類討論,進而扭曲群衆的價值觀,所以絕對是負面的。
即使是健全合理的司法系統,也頂多只是維護公益的衆多手段之一,連最重要都說不上,把它擡舉到至高無上、唯我獨尊的地位,是典型的本末倒置。而深究這個本末倒置發生在英美的原因,則是因爲最常侵犯公益的資本利益集團剛好可以掌控法律的制定和執行,結果司法系統反而腐化成爲顛覆公益的重要通道之一;這樣的腐化,和“法無禁止即可“的心態,是互爲因果、相輔相成的。
\subsection*{2020-07-22 10:51}

列舉一大堆有著各式各樣、多多少少關聯的現象,然後憑著現象的數目來驗證共通的論斷,是文科生的思維方式,沒有邏輯上的意義。我們真正必須尋找的,是因果關係,否則再怎麽學富五車,也無法跳出自身的偏見,總是能夠拿事實來拼凑出錯誤的結論,福山和Niall Ferguson是很好的例子。
在這個議題上,Huntington列擧的那些表現,和文明的衰敗,都只是同一個因素的不同後果,要做邏輯上的推論演繹,重點應該是討論這個因素,而不是滿足於列舉現象。這個因素是什麽呢?我認爲是國家社會失去自清和改革的能力;這時面對環境變遷帶來新挑戰,國家社會就無法做出最優的對應,甚至是往自己脚上開槍,年深日久逐漸纍積了不合理的結構和習慣,順便提供了政治學者近乎無限的寫作題材。
再往上追溯,美國是怎麽樣失去自清和改革能力的呢?這其實是本博客多年探討的重點話題之一,有幾十篇博文從各個角度來做分析。簡單來説,是富豪爲了占據社會產值更大的額分,所以必須全面腐化政治、社會、學術、媒體、經濟、文化、法律等等層面,才能消滅理性力量,讓不合理的財富分配安全地越演越烈。
所以最近中國那些學術官僚扭曲規則來幫助子女升學的現象,看來似乎無關緊要,但是事後的官官相護其實正在考驗體制的自清和改革能力,中國教育部不當一回事來處理,嚴重性遠比事件本身要高得多。
至於生活水平,日本和台灣都止步於大約美國的一半,光是這個目標就足夠中國以6-7 \% 的年成長率再努力15年。如果中共當局能夠繼續不斷整肅紀律、提高治理水平,那麽其後衝破那個極限絕對是可行的。
\subsection*{2020-05-12 09:26}

首先,美聯儲如果需要錢,印鈔是沒有上限的,光是今年就應該會印5萬億美元,去搶中國的那1萬億並沒有什麽實質上的財政意義。
其次,國債是不記名的;中方向來也廣汎使用歐洲的中介公司,尤其是比利時和盧森堡。要搶錢有實際執行上的困難。
1971年的去金本位,違反的是美國和小弟們之間的國際條約,美國面臨的純粹是外交壓力,這對霸主來説只要臉皮夠厚就可以簡單挺過去。美國國債運行的基礎卻是契約法(例如我以前討論過Magna Carta就是特定貴族和個別國王之間的契約,可見在Anglo-Saxon法系裏,至少在理論上契約法的地位高於國家元首和政權),這不只是國内法,而且是繼承自英國的Common Law,來自千年纍積的幾百萬個判例,遠早於國會的成立。若是硬要以新的明文法條去推翻它,那麽整個市場經濟的法律根基都會受威脅。
此外,過去30年金融衍生產品大行其道,其中包含了許多Credit Default Derivatives(主要是Credit Default Swap,CDS,信用違約交換)。這些契約要求在債務人違約之後,第三方要向第四方付出巨金。如果美國向中方賴債,光是CDS市場就會天翻地覆,金融界絕不會喜歡這種必然延續多年的極大不確定性。
總之,賴國債的實際意義是對選民作秀,凸顯自身的强硬,這有很多其他不會對自身體制傷筋動骨的選項,何況連建墻都無法逼迫墨西哥直接出錢,美國的統治階級嘴皮子動完,最終還是得優先考慮金融財閥和商業資本的偏好。
\subsection*{2019-12-25 11:12}

你的批評,有些是大家都認同的,有些則强於證據能絕對支持的程度。前者主要是精神建設和外交宣傳,確實必須加一把勁兒,否則無形的損失會越來越大。後者如中共對經濟金融的管理,是比我個人的偏好要更加放任些,因此風險也更大;不過人文社會的議題,極端複雜多變,我不是科班出身,或許所見有所不足,這件事又已經有陳平教授日常寫文督促,也就無須我置喙。 
至於習近平的中央集權,我想他繼承的政治局以及地方和專業官僚都不是完全可靠,那麽在他任期内暫時把以往下放的權力向上回收,不失爲階段性的正確選擇。畢竟我們不應該既要馬兒好,又要馬兒不吃草:既然還指望他進一步整頓紀律、提升治理品質,就不能抱怨他過度集中權力。事實上,權力的集中和分散,必須在一代與一代之間不斷輪替,否則不是僵化就是腐化。所以我覺得這裏的關鍵問題,不在於權力集散之間的選擇,而在於實際執行的細節是否合理到位。
\subsection*{2019-12-15 20:16}

如同我所有討論比較抽象議題的文章,本文也有很多層次的弦外之音;讀者能體會多少,看你自己的知識豐富程度而定。不過最直接的一點,應該是人人都能聯想到的,亦即人類歷史上其實從來沒有全民直接參政,而能高效面對國際挑戰的例子(想想古雅典);把它當作一個教條,强行實施,結果必然是全體的災難。瑞士、瑞典這些人口不及千萬的小國,能在大國競爭之中,置身事外,甚至從中牟利,是特例;沒有同樣條件的國家,自然不能盲目抄襲他們的制度。 
邏輯在文科教育中被忽視,是很不幸的事;就像入門經濟學一樣,應該是高中必修的項目才對。純數學,對一般學生的意義,不在於其自身的直接應用,而在於練習邏輯推理能力;這個能力是每個知識分子都需要的。 
我討論Lippmann,是取他的政治理論,至於他在國際戰略上的看法,那是另一個話題了。
\section*{【美國】【國際】新年的回顧與展望(一)}
\subsection*{2022-01-09 13:17}

今天Putin對Kazakhstan的公開評論,談到絕對不容許再有這樣的“Surprise”,這一句話之内就包含了好幾個重要信息:1)Putin和Tokayev事先可能並沒有確實情報;2)CSTO有條款,要求簽約國不針對彼此做情報工作,Putin應該是在指出這個條款已經不合時宜;3)所以CSTO國家可能將有統一的跨國情報合作。

謝謝你分享所聞。當前情勢還很模糊,新聞消息不可盡信;雖然基本可以確定Alyazov是幕後的金主,但Masimov的作爲還沒有得到證實,例如放棄機場這個説法頗有可疑,畢竟兩天後俄國運輸機就以每天超過百架次的規模在那裏高效進行起降,卻沒有任何槍戰的報導。
除了自發示威的民衆(包括剛被Chevron解雇的幾千名工人)之外,真正趁亂執行衝擊、占領任務的核心份子已知有土耳其來的傭兵、當地的黑社會等等。Ablyazov有13年來做準備,收買少數政客(但能否信任他們到可以事先通知的程度,是另一個問題)和幾百名死士完全在情理之中。
Putin和Tokayev事先就得到確實情報而有所準備的説法,並沒有足夠的事實證據和邏輯必然,反而有不少相反的論點,例如Belarus的前例是及早逮捕首腦,以防萬一(現實不是電影,1 \% 的額外風險都不值得冒)。而且Masimov顯然不站在Tokayev那一邊,他是情報主管,不論是否曾經參與暴亂的策劃,都沒有道理會在後者預知暴亂的前提下,坐視局勢如此發展。至於在兩天内出動70多架運輸機運送一個空降兵旅到邊境,原本就是典型的快反預案,並不需要特別提前準備。Tokayev剛剛接管國安會,還來不及替換中高層將領和警官,要求CSTO派兵托底(換句話説,俄軍不是去鎮暴,而是做監軍)也是理所當然。
\subsection*{2021-02-25 21:29}

這裏内幕消息完全欠缺,我先做一個純粹的揣測,請大家只把它當作許多可能的脚本之一:在我寫載人登月的文章的時段,中共參與決策的中高層官僚中還是有不少不願公開正面地與美國競爭,所以内部的投票可能是六四開、繼續拖著。等到新冠疫情之後,有大約20 \% 的人終於獲得信心,不再在乎美國人敵視的眼光,結果投票還是六四開,只不過贊成的成了多數。
好,現在接著談有客觀事實可以用來做論斷的相關議題。我先前已經解釋過了,載人登月是一個典型的形象工程,直接的經濟回報微不足道,但因爲人類歷史的特有進程,被賦予極大的間接廣告效益。新冠雖然對這個效益略有削減,對風險(亦即失敗的後果)減低得更多,所以決定投資進行並不算是錯誤的決策。
至於實際執行的技術選擇,兩年前我建議用長五多次發射、軌道連接,是基於風險高、時間緊迫的背景,以最小投資達成最快結果的小道奇襲方案。大約同時,專業團隊在計劃被裁的威脅下,提出抄襲Space X的多發小推力發動機並聯方案,則是病急亂投醫,沒有任何實際價值。現在既然政府高層有了自信,不再在乎美國的反應,那麽最佳方案又回到了慢慢發展長九,在2030年代中期實現載人登月的選項;這有培養維持團隊和減低技術風險等等的好處,是在決策已經下定的前提下,以長久眼光來看的最佳執行選項。
\subsection*{2021-01-26 21:21}

首先,這些步槍被出售的時候(假設是1986年之後製造的,在那之前生產的全自動武器仍然可以無限制自由買賣,全美總計63萬支合法的私有機關槍)是半自動的版本,全自動理論上是非法的,實際上當然有人自己私下改,而且販售改動器件在某些州是合法的。
其次,美國的執法完全由警察單位(別忘了,他們互不統屬,並沒有統一的執法規範)自由心證,所以即使帶槍游行在當地是非法,只要事先明白警察同情你的組織和理念(亦即種族歧視和右翼民粹),大規模違法就是安全、正常、而且公開的事。
你所看到的,應該是2020年4月30日右翼集團帶槍衝進Michigan州議會大廈的視頻,警察的確完全放任,事後也沒有法律後果。因爲媒體普遍報導,這顯然對2021年1月6日衝進國會大廈(更別提衝擊各州議會)的抗議者有激勵和示範作用;如果有些人兩個事件都參與了,我也不會覺得意外。
\subsection*{2021-01-24 12:03}

市場經濟的優點,在於靈活定價、激烈競爭,其好處是如果不計社會成本,可以提供較高的財富創造速率,壞處則是不但這些新生的財富集中在少數贏家的手裏,而且一旦這些贏家完成寡頭獨占,他們繼續纍積的財富不再是新創,而來自對整體社會的搜刮。所以政府對市場必須做至少兩個主要方面的調控(次要的問題,例如信息不對稱,我們就先忽略了):一方面人爲地把隱性的社會成本轉爲明顯的市場定價,另一方面必須抑制寡頭、確保社會公益,這在無需大量研發和資本投入的消費性行業尤其重要,因爲這裏受損的純粹是一般消費者,而受益的純粹是寡頭。至於名義上投資高科技產業,實際上做金融炒作的,則進一步危害國家的長期戰略利益,主政者如果迷信芝加哥學派的鬼話、尸位素餐,那麽就是國家民族的罪人。
這次馬雲撞上鐵板,正是因爲貪婪過度,把手伸向影子銀行業。我以前早已反復解釋過,影子銀行是特別危險的金融搜刮手段,暴雷之後,國家的損失可以是寡頭前期所得利潤的十倍甚至百倍;這是爲什麽現代銀行業有層層監管的原因。中國的金融監管單位,在2015年失守(參見當時博客的留言討論)之後,有所整頓和强化,所以這次出手並不是完全沒有徵兆的。
\subsection*{2021-01-23 18:35}

這條留言應該放在有関學術管理的文章之下。
其實中共早已認識到管理學術界和管理一般政府部門大有差異,問題在於他們所做的因應方案是完全錯誤的:他們以爲既然學術界專業性高,就給予高度自治,容許集團自我凝聚,非專業官僚只依據論文產出來分配資源,這就是“既封建、又官僚”體系的理論基礎。
當然我們知道這是基於完全不成立的假設:亦即只要是高論文產出的學術大佬,就自動會是新版的錢學森和楊振寧,無私睿智地為國家做籌劃。實際上這些人道德人品可能很低,甚至連所謂的專業成就都可以是造假出來的。既然他們現在已經想要利用身份地位來拼命搞公關,那麽解決辦法之一,就應該是確保這些對外解釋自己所作研究的報告,不能針對毫無辨識能力的普羅大衆,而是適合近似專業的人員來學習評審的材料。這樣一來,連負責主管的非專業官僚,只要有些相關的教育基礎,都可以慢慢理解細微的專業特性,從大局上來估算研究的前景。我談學術管理所揭示的基本原則,甚至可以整理成一個碩士學位的教學内容,專門用來教育下一代的學術管理人員。
\section*{【宣佈】Whoops,不小心刪了一些值得保留的討論}
\subsection*{2022-01-02 22:55}

是的,不論科技和社會如何演進,人性的弱點卻是不變的,尤其是人云亦云的Herd Mentality對群衆輿論的愚化作用,反而隨著現代通訊技術的普及而成指數增强。中國的發展太快了,民衆和官員都欠缺對假未來科技騙術的社會記憶和歷史沉積,全無免疫力,所以特別容易上當,那麽仔細研讀近代和現代先進工業國家的大衆傳播史也就格外重要。例如150年來,西方承諾要直接空運信件、包裹到每個公寓窗口的“未來學家”和“發明家”,至少有幾千個,技術從最早的熱氣球和仿鳥翼飛行器,到後來的直升機,再到現在的Amazon Drone Delivery。它們不可能實用的關鍵,從來都不只是技術上的,也來自經濟性和社會性,然而一代又一代的群衆就是自願被騙。
其實不只是假未來科技,當前世界許多最嚴重的問題,都是歷史的重演。例如英美自2016年後民粹挾持決策,導致國家走上自我毀滅,在熟悉1920、30年代日本軍國化過程的人眼中,完全就是老片翻拍;而德國末代Kaiser和Trump在智商和個性上的類似,也是我幾年前就提過的。
\subsection*{2021-07-04 17:10}

我是從30年前原版的《Civ》一直玩到現在的老玩家,這個系列至今依舊是我最喜歡的電腦游戲。奇觀(World wonder)在早期版本曾經是極爲OP(Overpowered)的,從《Civ IV》才開始被逐步Neuter掉,到最新的《Civ VI》的確有點“誤國”了。不過游戲的奇觀多多少少還有正面的效應,比起現實世界中大對撞機、核聚變發電、火星殖民這些純粹的錢坑+人才粉碎機,仍舊不可同日而語。
我想特別提醒你,《Civ》的Gameplay策略偏好,如同科幻小説一樣,也是主觀憑空設定出來的,不能直接移植到現實國際鬥爭之上。不過作爲作者觀察西方歷史的總結,它可以被當成理解Anglo-Saxon世界觀的一個參考點(換句話説,You must look at it from the next metalevel.)。
我以前説過,在那些學術騙局的討論中,雖然表面上是我一個人孤軍奮鬥,實際上我談的都是科學角度下早已明顯化的主流認知,只不過爲了圈子的共同私利,沒有中國專業人士願意或膽敢發聲罷了。你如果熟悉英語世界,就會知道在美國也有針對核聚變和火星殖民的批判聲音,不過被很成功地壓制住,只能偶爾出現在非常冷門的網站。對於這個中外文化的對比,有説法認爲西方個人主義文化鼓勵正義人士擺脫人情壓力,有助於揭發真相;這個看法有其根據,但不完整,因爲美國的利益集團雖然不能强迫每個人噤聲,但可以遏制(Contain)他們,然後用公共論壇上有意創造出的高分貝噪音來淹沒真相,然後殊途同歸。我當然希望能有其他人出面聲援,但是現實中原本就不能奢望到處都遇上楊先生這樣才德兼備的大師;至少大陸公共論壇的理性程度相對高於現代美國、噪音分貝數也較低,我的實話還能有一點市場,那麽就看是否能堅持努力下去了。
\subsection*{2021-07-02 21:15}

我說“科幻行業”,指的是“field/circle”,而不是“business”。它可大可小,即使80年代中國只有幾千個留學生和外交官能讀科幻,一篇公開發表的自主作品都沒有,也不能説它不存在。
這個博客討論的是成千個關乎國計民生的重要議題,而且目標是要達到華語世界獨一無二的深刻層次;我真沒有時間精力可以浪費在杠精身上。就算“圈子”是更好的翻譯,直説就好了;搞清楚是“40年”還是“21年3個月又5天”,有什麽價值?對核心論述能有什麽影響?無限上綱不是又多違反了一條《讀者須知》的規定?他一次觸犯了至少1、8A和8B三項,可能已經創紀錄了,如果有比拉黑更重的處罰,也當之無愧。
試圖為他開脫,徒然繼承他浪費大家時間的職志;念在正文因此而修正了幾個字,只警告一次,再犯禁言。
順便提醒大家,尊重作者和其他讀者,不浪費衆人的時間,雖然沒有列在《讀者須知》的條令裏,卻是在前言中明確談過的。想要留言之前,先自問你的話題有沒有資格出現在這個博客。如果已經有人被拉黑,沿著同一方向繼續囉嗦很可能不是明智的行爲。
\subsection*{2021-07-01 14:40}

你讀了《讀者須知》沒有?新讀者在這個博客發言,要先三思而行:個人的感覺在此毫無價值,論述必須是基於事實,並且符合邏輯,尤其不能無視博客既有的辯證結論。
我的確沒有去研究過大陸的科幻文學發展史。不過中國連甲士械鬥這種冷門到極點的嗜好,都能派得出像樣的團隊,十幾億級別的人口,武俠流行的國度,說沒有幾千萬對科幻有興趣的人是不可能的。被學術界鄙視,不但是Asimov在40年代的美國也曾有的經歷,從客觀標準來看,又有什麽意義?歷史上哪一本經典小説的作者是比較文學系的畢業生?文學學院派對科幻的意見,如同他們對科技的意見一樣,根本毫無價值,可以直接忽略。我的論述只基於兩個前提假設:1)劉慈欣是同代科幻作者中的前列人物;2)劉認爲科幻的重點在科技、而不是社會對科技的反應。科幻的正確方向,原本就只能經由作家和讀者的互動來決定;學院派既然選擇不當科幻愛好者,他們怎麽想、是否“指導”過、或甚至是否存在,都完全不相干。
高能所的所長、副所長、研究員,算不算一般人心目中的“高級知識分子”?在我出面之前,他們已經為大對撞機造勢多年,可從來沒有遇到任何負面的評論。科技部的決策算不算“影響”?你知道中國已經花了多少錢在核聚變上?給了多少獎?多少次出現在全國科技成就的列表之中?就在上周,還有總師上媒體,得意洋洋地說國家已經準備在2030年代要送人上火星;你知道這會浪費多少人力、物力、財力的資源嗎?國家和人類實際需要的,不是對於航天事業和基礎科研的“普遍”支持,而是對兼具可行性和經濟效益(不是短期效益,否則可以交給市場經濟來搞,根本用不著國家參與;但短期無回報和永遠無價值是兩回事,公款不能被有政治能量的人騙去造永遠無價值的大玩具)的那萬分之一的研究路綫做重點投資。當然,不可能事先準確地挑選,但至少先把10000個點子嚴肅論證,刷到剩下10個再投資,否則佔全世界70 \% 的R\&D經費,一樣不夠維持科技領先地位。
這些道理,我不但反復地在博文和留言欄説了幾百、上千次,而且到本周都還是討論的重點。你既不學、也不思,匆匆飛來,草草看了兩眼,拉出一坨“見解”,然後又匆匆飛走;這種海鷗式的留言,固然是互聯網的常態,在這個博客卻是絕對不容許的。請不要回來,浪費大家的時間。
\section*{【基础科研】什么是科学?}
\subsection*{2021-12-14 17:02}

只要其本質是求真,就是科學,可以而且必須依據科學方法和原則來做研究分析。社科的複雜性,代表著它遠遠更爲困難,但這不是搞僞科學、文藝抒情、自由聯想、主觀論斷、或耍嘴皮子擡杠的藉口,反而在要求從業者和參與者有更高的理性、邏輯修養。如果無法滿足這樣的高標準,最起碼應該閉嘴,不要污染公共論壇。英美體制的基本謬誤之一,就在於至少在明面上,鼓勵全民參與政治社會議題的決策,那麽民粹自然無法避免(這不是說要壓制民意;事實上百姓是反映政策缺失的明鏡,但民意偏好必須只供理性決策的參考,而不能直接作爲壓倒一切其他考慮的藉口)。中國的問題則在於社科界既沒有正確的選拔標準,也沒有優良的文化傳統,更沒有明智的管理制度;還好在中國體制下,決策集團的水準原本就高於學術界,所以危害較爲有限。
\subsection*{2021-05-08 02:35}

First Principles指的是極爲基礎、可靠性全無疑義的已知事實。回歸第一原則的意思,就是不在假設上再加假設,頂多只有一層不確定的前提考慮。世界上有趣的議題,很多無法完全確定,所以討論最可能或最優結論的時候,也必須估算誤差。如果纍積了多層假設,這個估算就幾乎必然會是錯的。
在自然科學裏,若是把假設拿來冒充為已確定事實,層層堆砌,像是超弦,就自然演變成爲僞科學。在社科上,因爲美宣對維持霸權有重要作用,這樣的冒充頂替更是有長久的歷史和充分的資源,被系統化和普世化了。例如美國人假裝直選制是普世價值,就不是第一原則;如果貿然接受作爲前提,後續討論就毫無意義。我從演化論去討論真正普世價值的特性,才是從第一原則出發。

“從演化論去討論真正普世價值的特性”,我在過去兩個月就提過兩次。留言回復你不用心讀,只管發自己的問題,嚴重違反博客規則,那則留言被刪了,禁言一個月。
\section*{【金融】【戰略】美元的金融霸權(一)}
\subsection*{2021-12-12 03:15}

因爲金融是我的專業本行之一,從博客一開始寫了《美元的金融霸權》那一系列文章,早已解釋過針對美元來做反擊的重要性。不過我一直沒有對人民銀行做尖銳的批評,是因爲他們管理貨幣政策,並沒有犯過證監委在2015年股災期間那樣的明顯戰術錯誤,所以在戰略上我也給他們Benefit of doubt。
幾年下來,世界經歷了中美貿易戰、新冠疫情、美聯儲超發,然後通脹壓力浮現表面,再怎麽仁慈的旁觀者都沒有藉口繼續假設人民銀行有什麽隱性的正當理由不對美元下手,所以我才終於開口批評,而且談的只限貿易替代。貿易替代是最最基本的應有作爲,我在過去介紹俄方政策的時候已經反復論證過其壓倒性的正面效益,就不再贅述。不過這些利害考慮,有許多是超乎人民銀行日常職務視野的中美博弈戰略問題,原本就應該由負責戰略分析的智庫來做,所以我並不是把責任完全歸罪於金融管理部門。
\subsection*{2021-12-08 02:59}

當時的先進工業國僅限於歐美日,而且對化石燃料有近乎絕對的依賴,所以美國簡單把美元和石油綁定,歐日只能被迫跟著印錢,把通脹全球化;這裏的重點在於整個過程中,美元的國際地位從未受到真正威脅,因爲根本沒有接近合格的替代選項。20世紀末,歐洲還有些具備戰略眼光的政治家,所以在事後(尤其是經過Plaza Accord的又一次打擊之後)推行了歐元,其用意就在於避免反復受美元的搜刮。
很不幸的,到了21世紀,歐洲管理階層的素質逐步退化,坐視歐系銀行、產業和歐元受美國的多方暗算而毫無所覺。現在美國又以無限QE來複製50年前的通脹搜刮,眼看著歐洲又要躺平了。中國的正確反應,在於聯合俄國、亞非拉和甚至中東,拒絕被動承受美國引發的通脹壓力,順勢挖美元霸權的墻脚。這裏的第一步,是要求(除了對美的)主要進出口產品立刻改用其他貨幣,尤其是人民幣;對這一個基本手段的任何猶豫,都是鼠目寸光、因小失大的非理性思維,也是中國社科學術智庫界,對國家人民的又一次辜負和危害。
\subsection*{2021-10-21 15:33}

啊,既然你給了鏈接,我乾脆剪貼過來算了:
美聯儲的任務之一,是安撫市場情緒,所以撒謊或者答非所問是必要也常用的手段。例如文中被問到量化寬鬆,他卻大談國際溝通。被問到新的國際數字化貨幣,他就强調美元的優越性和美聯儲的必要性。
當然也有他不小心透露口風的時候,例如最後兩段明顯地是在討論過去三個月的Repo市場:Dodd-Frank這種限制金融業胡作非爲的法案,被他説成”使(提供穩定性和流動性)變得更加困難“,指的是Excess Reserve裏有4000-6000億美元其實是Semi-required,不完全能由銀行界自由花用,但其實流動性短缺的真正問題在於過去三年,美國金融界損失了15000億的現金給聯邦赤字和美聯儲,他卻不敢提,只在最後一段說以往的緊急程序,現在已經變成“常規程序”了。這真的是正面的消息嗎?
\subsection*{2021-08-10 22:40}

1.不必然,尤其中美之間的敵意擺上臺面,或多或少的切割無可避免,隱隱浮現出兩個或三個陣營(現在中方的外交努力方向,不正是要把美國陣營做得越小越好嗎?那也是我寫《再談Biden任期內的中美博弈等議題 》的核心話題)是很自然的後果。國際儲備貨幣碎片化雖然在極長期來看經濟效率不是最高,但過度階段卻可以長達幾十年。歷史上最近的例子是一戰和二戰之間,英鎊和美元的額分上上下下,但基本在同一個數量級,達20多年之久。
2.這正是Keynes預見的解決方案;但如同80年前,美國人必然搗亂,基本不可能在下階段一步達成。中方或許可以預先打下基礎,不過成果是2、30年後才看得到。我一向不喜歡談那麽長遠的事,因爲它們的不確定性不是太低(例如美國必然衰敗,廢話)就是太高(美國衰敗的確實方式和時間,胡猜;要是20年後的細節都能預見,爲什麽不先處理未來一兩年的中美折衝?)。

要確保國家貨幣一家獨大,卻不反噬自身的實業,可以采用金本位(19世紀的英鎊)或半金本位(Bretton Woods系統下的美元)。但這在現代金融環境下,已經不可能行得通;就算行得通,也1)不完全有效;2)永遠會有誘惑,要把撲滿打破,獲得無限印鈔權。所以我的確贊成Keynes的跨國貨幣建議;在實際執行上,短期内讓歐元成爲輔助國際貨幣,等到美元獨霸的利益被削減到足夠程度,再和美國人交涉,建立真正的世界性貨幣。
\section*{【台灣】【工業】經濟的最後支柱}
\subsection*{2021-12-07 05:20}

我的研究習慣,是對每一個值得深究的議題,都把所有相關的事實和邏輯從全方位的角度仔細考慮,然後才提綱挈領,只將辯證的主幹用最精簡的語言解釋出來。從讀者的觀點來看,雖然實際上四面八方都被嚴密精確的邏輯層層包圍了,但膚淺的直覺印象卻可能只看到眼前幾個段落所描述的一個光桿論點。這裏的矛盾在於,要進一步領會,必須有相當的聰明才智,但是聰明人卻天生很難放下名利心和好勝心,一旦陷入玩小聰明、耍嘴皮子的擡杠模式(當然,很多笨蛋更喜歡擡杠,參見Dunning-Kruger理論),忽視佔99.99 \% 的隱性深層論述反而是必然結果。你們大多經過“好奇路過->心有不服->反唇相譏”的心理過程,如果我沒有耐心、不堅持理性討論,怎麽能勸服你們也用耐心把幾百萬字的博客讀完、讀懂呢?很不幸的,現代人類世界就是這麽複雜,即使是最優化過的詮釋,也不可能塞進幾周就能涵蓋的篇幅。一般網民的Attention Spam頂多以分鐘計,這是爲什麽大衆媒體反而有愚化作用的基本原因之一。
當然,理性態度還有遠遠更重要的意義:它不但是求真的不二法門,更是國家民族長期興盛的文化根基。我希望讀者廣爲傳播的,不止是對特定議題的正確解讀,建立健康的思維標準和輿論慣例的確可能有更宏大的深遠影響。
\subsection*{2021-12-06 05:51}

是的,這個問題在當前行政的優先順序似乎很低;然而一個議題是否優先,其實必須有至少兩個獨立維度的考慮,亦即重要性和緊迫性。很不幸的,雖然高科技產業升級的整體協調和引導,有著極高的重要性,但卻並不緊迫。歷史上中方在半導體和機床等方面始終沒有投入足夠的管理和資源,以致現在極爲被動,而剛好它們正是台灣經濟的主幹。
我和大陸智庫開始交流之後,瞭解他們被要求的研究方向,有了一個大家或許預期不到的感想,覺得習近平很可憐:他明明知道政策的正確方向何在,但是在執行細節上卻沒有足夠的學者和幕僚來貫徹他的意志。換句話説,國家花大錢養了一大批北大和清華的教授們,但這些人不只是尸位素餐,而且反過來充當帶路黨、誤國誤民。如果連國安外交這樣優先順序極高的議題,都充斥著金磚智庫那個級別的胡扯,那麽指望產業政策能獲得理性深刻的建言,當然是不切實際的。
\subsection*{2021-12-05 13:28}

2015年我寫這篇文章的時候,的確是低估了英美台民粹的愚蠢,沒有算到居然會有脫歐、選Trump當總統和想要以武拒統的種種自殺性蠢事;不過我已經反復解釋過,我沒有魔法,只能以邏輯來試圖推算蠢蛋的作爲,原本就不可能是完全精確的。
我在早期博文中,也的確曾經極度高估了中方智庫、幕僚和官員的水準,後來中聯辦、國臺辦、教育部、科技部和各類學術機構的種種蠢事、爛事不一而足,強逼著我逐年下調對他們的評估。例如2017年貿易戰剛有苗頭,我就已經給出正確的總結和建議,但是一直到2019年,清華大學的金磚智庫居然還建議“站在人类生存角度...维持美国的世界大国地位是十分重要的”(參見http://www.cbgg.org.cn/index/article/show/cid/20/id/72.html)。這一樣也是自殺性的行爲,一樣也是邏輯無法事先預估、或事後解釋的。事實上,我的文章被轉發到大陸,始終有人什麽事實邏輯基礎都拿不出來,也敢評論“大内自有高人”、或者“小不忍亂大謀”之類的空話傻話;所以如果你的論點是中方的學者、專家和一般知識分子之中也有許多蠢蛋,那麽事實根據很明確扎實,我無法反駁。
然而即使是菜鷄互啄,也有一方更菜;而且台灣連當菜鷄下場的資格都沒有,只不過是被爭奪的一隻小蟲罷了。正因爲台灣體量太微不足道,大國打個噴嚏,都可以讓它乘風飛起;你聽説過“風口上的飛豬”這個描述嗎?恆大風光了20年,平均年成長率達到33 \% ,一旦中國政府終於決定做早就該做的事,現在下場怎麽樣?台灣靠著美國打貿易戰而暫時獲利,能高速成長20年嗎?連恆大都不如,有資格自傲嗎?我的預測原本就是以三十年爲期、十年為一個階段,現在距離2025年還有四年,請大家稍安勿躁,短周期内的高頻噪音在所難免,但是歷史潮流的大方向,終究會呈現出來。
ps.我不想評論《Foreign Policy》的文章,因爲它並不是像《Foreign Affairs》那樣的學術期刊,而屬於“主流媒體”;我在討論假新聞的博文裏,該説的都已經説了。參見《讀者須知》。
\subsection*{2017-10-10 00:00}
我并不是说张忠谋撒谎,而是他说的话是故意过于简略来给臺湾听众错误的印象。

其实现实是很复杂的,而商场的宣言却很模糊。张所说的"好几年",是三、四年,还是八、九年?我想他给人的印象是后者。他説的"达到臺积电的技术门栏",是指挑战臺积电届时最尖端的技术,还是对臺积电的生意有实质的竞争影响?臺湾人的解读也同样偏向后者。

但是臺积电的生意不会全部是最尖端的制程(其实必然大部分都不是),八、九年后(可能在五年后就开始),中国的半导体制造公司必然会和臺积电在较老的制程上有激烈的竞争,对臺积电的营销和利润也因此会有明显的衝击。这才是我要强调的重点。

张忠谋好像否定了那个可能性,可是实际上什么也没説(除了三、四年内中国企业不可能赶上臺积电的最新制程,但是那是废话,没人真的宣称能赶得这么快),这就是生意人(或者政客)讲话的艺术。\subsection*{2017-08-19 00:00}
以蔡英文的个性,既不在乎科学现实,也不在乎百姓生计,所以必然是会想要矇混过关。臺电成了代罪羔羊,只能拼命买燃气轮机来烧天然气发电,但是厂址有限,费用又高,臺电会大幅亏损,不是长久之计。

等大停电成了常态,深绿大佬的逼宫也会更加急迫,这倒是蔡英文很在乎的事。所以她拖了一段时间之后(时间长短视运气而定,长则两年,短则数月),她应该会恐慌症发作,不择手段来搪塞悠悠之口。最可能的是花费极为昂贵的价钱来装设不靠谱的绿色能源,基本上是举债来作秀,不但电力问题没有根本解决,而且国债又要创新记录了。

这样的总统是自己选的,也就没有什么好怨天尤人的。这就是民选制度的真谛:不在于能办事,而在于永远都是自作自受。\subsection*{2017-06-23 00:00}
核数超过4,对一般用户就没有太大的实用意义。Intel过去十几年,就强在单核性能上。现在AMD的单核功能又追上了,Intel被逼上加核这条路,实在不是好兆头。

我在两年前,就想要写文章,预言Intel的衰落。当时Intel还没有AMD的问题,长期的威胁,来自ARM。Intel自我限制Atom CPU的发展,容许ARM独霸低功耗、大销量的市场端,自己退守高性能、高利润的伺服器市场,这是极为短视近利的不智策略,与30年前的DEC如出一辙。十年之内,必有清算(Reckoning)。

我原本每隔四年,会自己组装一部新PC,不过2016年到期之后,一方面没有新软体需要更强的功能,一方面新硬体没有太大的进步,再一方面GPU、RAM和SSD都正在被厂商哄抬价格,换新PC,实在不划算,也就作罢。\subsection*{2015-11-09 00:00}
If silicon technology peaks at 12"/7nm, it may be a blessing in disguise. For decades, the rapid doubling of performance has drained enthusiasm and resources away from other promising chip technologies and programming techniques. I look forward to the days when more frugal optimization is once again the focus of hardware and software engineers. This will encourage new ideas to surface. Furthermore, the decentralizing the competitive power from Intel will allow new players to emerge. The world may be better for it.\subsection*{2015-11-05 00:00}
Actually, I think the end of Moore's Law is a very good thing overall. First, as you pointed out, engineers finally have some incentives to try out innovative ideas. Second, the existing monopoly (yes, I am talking about Intel) will slowly fade away. New comers like those being fostered in China will have a chance to compete.\subsection*{2015-11-04 00:00}
18 inch technology has been scotched once already a few years ago. This depends really on an accounting equation; TSMC has to balance competitive pressure against the enormous costs. The technology itself is certainly possible, so the forecast tends to be rosy, but accountants do not usually share the enthusiasm of engineers.

Anyway, we will know for sure in a few short years.\subsection*{2015-09-02 00:00}
本来是离题的,不过我刚好想在这方面说几句话。

你之前引用的那篇文章的作者voyager\_ho在文化和社会两方面都比我专业得多,可能是台湾某人文社会学科的教授或博士班学生。他至少从2009年就开始写文章,但是并没有创出一个像这里一样的部落格,也一直必须保持匿名。

请大家注意,我有一个独特的优势,就是我不用担心同事或朋友的排斥和歧视,这才是我成为台湾极少数(如果不是唯一的)用实名说实话的人的主因。在我开创这个部落格之前,我常看到大陆的网友讥笑台湾人没有见识,其实是有见识的人不敢讲话。所以请大陆来的读者体谅绿色恐怖之下的受难者,不要强迫去猜测他们的真实身份。

我在此特别允许台湾读者回覆这个离题的留言。大陆读者请旁观,不过若是以后在别处看到有讥笑台湾没有国际观的,可以考虑出来说句话。
\subsection*{2015-09-02 00:00}
我以前就听说过这些研究结果;这是人脑很不幸地在演化过程中发展出来的弱点,因为事事都用脑,很耗热量和时间,所以演化出成见,就可以很快很简单地做出决定。不过我们已经不是随时可能会饿死的原始人(其实肥胖反而是现代常见的问题),所以这个演化的后果在热量上已经远远不是最优化的了。在时间方面,现代人很忙,没有空閒来学习社会与政治的议题;不过这和民主制度成了先天的矛盾,因为民主制就是让每个人都参与重要决定,只有在多数人是能做正确选择的专家时,国家才会上轨道。

我时时提醒自己必须以理性检讨任何成见,但是一般人似乎很难做到,那么他们就不应该争取投票的权利,因为他们的成见可以被有心人利用而从弱势群体榨取利益。\section*{【宣佈】如何支持這個部落格}
\subsection*{2021-12-06 21:07}

謝謝你的提醒。支付寶需要大陸銀行卡才能認證;多年來我的賬戶一直處於未認證的半違規狀態,而人民銀行最近開始收緊監管,所以限制越來越多,這個賬戶的確已經成爲僵尸。
今天我和支付寶的客服談了100分鐘。和客服打交道向來是我最怕的事之一:客服人員對Problem Solving的邏輯思路往往很陌生,他們自動假設客戶完全無知、犯的是最低級錯誤,於是會忽略對方的解釋,無限重複自己的基本解答套路。這是我第一次面對中國客服,發現她和美國公司雇傭的印度人有著基本相同的行爲模式;人性的確是普世共通的。
話説回來,我可能必須設法另開賬戶,這需要一點時間,如果成功再做宣佈。目前這個支付寶處於凍結狀態,有所不便請見諒。

終於拿到新的支付寶賬戶,已在正文中更新。謝謝大家的支持鼓勵。
\subsection*{2021-02-28 06:51}

謝謝。
我的文章太過直言不諱,大陸主流媒體除了《觀網》之外,不見得敢登。其實就是《觀網》,也往往瞻前顧後,嫌棄我的文風與衆不同,需要内部部分編輯力爭,才能登上《首頁》,而且越是重要的文章,越是如此。例如兩年前我介紹《美國陷阱》的那篇《域外管轄權》,他們壓了近一個禮拜,我以爲已經無望,還特別緊急聯絡史東,另外做了《八方論壇》的節目。《737Max必須重新認證》也是華語世界第一篇把事故前因後果解釋清楚的文章,結果第一次發,首頁主編輯說太短、不合適,我還得爲他們指出質和量的差別,又有科技編輯力保,才勉强過關。這次《學術界的假大空》,原本就是和他們《科工力量》合作的視頻,居然一開始也被草草處理,《觀網》上根本不提,音質又很差(這我必須承擔部分責任),後來是“世界對白”自行花時間、花錢做的字幕,我親自校對。還好因禍得福,既然有了文字版,就可以另外再發一次;結果又是科技編輯在内部會議上力爭,拖了一個多禮拜才發在《首頁》。
這個博客從來就不是普通的博客,我也從來不是普通的博主。人類社會運行的規律,對獨一無二的事必然會有很强的疑慮,如果我在乎的是名利,根本就不可能維持這樣的寫作風格。這又側面地證明了自由主義市場經濟理論的局限與偏頗:人類追求真、善、美,先天就和利潤最大化格格不入,沒有理想主義的利他精神,社會自然淪爲弱肉强食的禽獸世界,掠奪來的財富可以減輕、掩飾這股壓力,但只能是暫時的。
\section*{【外交】【戰略】美國制華歷程分析及對中國外交政策調整的建議}
\subsection*{2021-11-27 07:23}

對手的Utility所含的因素之中,選舉其實是很膚淺、次要的。這是因爲在官商勾結的背景下,實際行政細節有媒體遮掩,説一套、做一套完全是英美政治體制的常態,利益集團才是政策的真正優先。但那至少是理性的,我們真正須要仔細考慮計算的非理性因素,是大大小小的各種“次文化”。所謂“文化”就是沒有客觀道理的集體習慣和思路,“次文化”指不涵蓋全體國民,只適用於特定群體;這裏不同方向的民粹是其中最高調、普遍、明顯的類別,但影響往往更大的,是相關學術界和智囊圈子裏所流行的共識。例如我在討論大對撞機的過程中,最大的阻礙也是各種次文化:科技部的瞞頇、高能物理界的自私狡辯、一般國民對科幻和現實的混肴、高級官員對歐美名人和熱點的崇拜、以及體制對行内專家的絕對信任和放權等等。至於中國當前與美周旋,所面臨的Biden/Blinken/Sullivan陣綫全是白左外交理論的信徒,不去理解他們的選擇性脫鈎策略,當然會把他們的喘息當作善意、利用看成和解。
\subsection*{2021-11-26 07:44}

你説的沒錯,對付已經徹底民粹化的對手,不能假設他們會做出理性客觀的最優選擇。這裏有兩個解決方案:首先,你可以同時考慮博弈論的Nash Equilibrium以及把非理性對手當作背景設定(亦即沒有主觀意志的噪音)的非博弈論問題,在兩個解之間做一個連綫,那麽未來的發展應該會落在這條綫上,所以你應該尋求對整條綫都優化的解。但這是很複雜的計算,即使有合適的解,也需要大量的信息或時間才能發掘,一般只在大戰略原則上能夠做到。如果必須做出快速的反應(例如一般戰術折衝),可以采用第二個方案,也就是設身處地去理解非理性對手的Utility Function(效用函數?),把他們的非理性欲望濃縮到這個Utility Function裏面,再參考這些人過去的行爲模式和記錄,自然可以對他們的未來作爲的大致方向(但不是特定選項)推測個八九不離十。
\subsection*{2021-11-25 10:44}

技術上可行,經濟上頗爲可疑。台灣不是西藏,大型海島載人有飛機、載貨有船隻,真正重要的目的地,一個長三角、一個珠三角,坐高鐵都要繞一大圈,並無效益。我的估算是,在當前的人口分佈和經濟條件下,不可能在地震、颱風頻仍的海域,滿足建造100多公里的橋梁/隧道的客觀效費比評估,除非是以比港珠澳大橋更高的倍數,來假造高估流量。
至於維持國家統一的“政治意義”,更加是牽强附會、莫名其妙的非理性聯想:港珠澳大橋2018年開通,所以2019年香港動亂比2014年溫和嗎?英國脫歐是因爲沒有高鐵連接大陸嗎?歐洲高鐵里程第一的西班牙,面臨的Catalan獨立運動爲什麽是在通行最高級高鐵的Barcelona而不是離島呢?明顯應該從國安、法治、組織和尤其是教育(基礎教育對民心的重要性,我已經針對台灣、印度、土耳其等案例討論過無數次了)著手的問題,硬是要靠形象工程來解決,是典型的把科幻小説當成治國方略的民粹思維,我相信中國領導人的智慧遠超鍵盤俠的層級。
\subsection*{2021-11-22 22:19}

這種事,《DW》和《RT》自然會做,中方還是先把自身相關的宣傳話題做好吧;跑到地球另一邊和200萬人口的小國慪氣,反而惹人譏笑。
我覺得Biden團隊的下一步,可能會是在半導體封鎖上加碼。這一方面是因爲它是Trump最成功的對華打擊手段,另一方面美方剛從臺積電和三星獲得的商業機密資料,提供了更加詳細精確的打擊目標。中方的正確對應,在於拒絕隨美國的音樂起舞(例如過於專注在Lithuania),自主選擇脫鈎或談判的方向,並且放棄自欺欺人的中美雙贏未來,誠實地把美國標識為敵對勢力,充分運用損人利己和損人不利己的事務來爭取籌碼,準備你死我活的鬥爭(我不是建議公開這麽說,而是内部討論對自己誠實,然後在行爲反應上選擇最優解)。短期内,冬奧是個自找麻煩的痛脚,悔之已晚,只能設法拖到結束;這裏俄國有2014年的類似經歷,可以尋求他們的經驗教訓作爲參考。
\subsection*{2021-11-21 19:53}

好的,那麽《Reuters》談釋放原油儲備這件事,就和《FT》報導中美核武談判一樣,是美方有意造謠,一方面供内宣消費,另一方面對中國間接施壓。這種施壓方式當然是很幼稚的,但美國掌權的精英在高中和大學裏都是Bullies,習慣搞這些花樣;英國的Johnson也是如此。
未來幾個月,美方可能因爲通膨問題纏身,不得不在關稅上做出退讓,但這和中方是否接受政治外交上的和解假象毫不相干;事實上,如果中國也根據己方利益來對個別方向選擇脫鈎(外交)或談判(經貿),反而可以得到更好的條件。我預見沒有邏輯因果概念的社科“專家”會把關稅和解拿來吹噓他們的綏靖政策,所以提早在此澄清真相。
汽車的芯片需求,可以保障供給鏈的生產底綫,但要獨力支撐先進的半導體產業是遠遠不夠的。
\subsection*{2021-11-20 19:34}

戰略儲備是爲了戰爭(例如臺海)緊急需要而準備的,對中國這樣的主要消費國尤其重要,拿來平穩油價是本末倒置。所以中方的頭號選項是根本不要理美國的要求(除非美方提供重要交換籌碼,例如承諾要遏制Lithuania,但現實裏這是不可能的);如果非要答應美國空口白話的使喚不可(請注意,這個可能性正對應著我最近討論過的戰術和解的壞處),也應該只象徵性地少量釋放。與此同時,繼續積極建設油氣儲存設備不可鬆懈。
Biden的用意,主要在於平穩物價,以減低通膨。最新的統計結果依舊讓經濟學界的鷹派和鴿派各執一詞,而後者主要指出通膨集中在少數大幅漲價的產品類別。然而我認爲整體數字之所以還不太嚇人,正是因爲有些產業的明面價格反而下降了,真正的重點在於這些似乎通縮的產品(例如旅游業),才是受疫情影響而暫時失準的部分,實際的通膨壓力是高於賬面數字的。Biden一意孤行,不斷增加赤字,美國所面臨的經濟崩潰危險越來越大。
當前能源短缺有三個原因:首先是去年疫情開始之後,有全球性普遍的大幅減產(不是封存,而是停止探勘和開采),然後歐美的財政刺激反過來創造了消費的新歷史高峰,短期内供給無法跟上需求;其次,中東產油國和美國頁岩油氣企業的財務都有嚴重困難,現在一方面終於可以享受一下較高的價格,另一方面也還沒有意願追加投資;最後,是美國頁岩油田經過連串的瘋狂擴張,低價高產的優質油井已經所剩無幾,平均開采成本在隨技術進步而逐步下降了十幾年之後,反而開始緩慢回升,更加減低增產的速度。至於美國在中東的影響力,倒不是主要因素,例如你去看Saudi的產能利用率,原本就很接近100 \% ,並沒有短期内可以迅速增產的可能。
\subsection*{2021-11-17 11:36}

是的,這個議題我已經論證過幾百遍,事實上習近平本人在貿易戰的經驗之後,必然也已經同意正確的結論,否則不會對澳洲做出那樣的處置。兩周前,我已經評估過,當前高層内定的政策,很可能正是我過去7年多來所試圖解釋的正道,只不過出於冬奧的考慮,而暫時隱忍,順便繼續對美方好言相勸罷了。
所以最近這一波討論,其實是針對背景裏提供理論和選項的幕僚學者,他們在錯估美方戰略意圖被打臉之後,依舊只是簡單地轉換為另一類幻想和誤解,單憑戰略情勢下拖延時間有利己方,就天真地以爲戰術和解是最優解。這裏的根本問題在於假設對方是被動、固定、沒有自己思想、文化、習慣、戰術、理論和反應的,只要學過博弈論的人都知道這是大錯特錯,社科專業的精英也犯這樣的低級錯誤,對行政單位還是有些噪音污染的效應,所以有必要端正視聽。
\subsection*{2021-11-16 09:05}

因爲我在評論一個議題之前,必然要先想清楚全面的考慮,所以細節和用詞都是很精確的。一般人囫圇吞棗,然後凴反射直覺來回應(亦即我説過的“海鷗式”留言),已經不適合在博客發言;一個數學系出身的人,也犯這個毛病,就更加不應該。考慮到你一輩子在象牙塔裏成長,剛剛面臨陌生的現實世界,我暫且不禁你的發言;不過請自重,在發問之前先把這裏的論證仔細閲讀、徹底吸收。社科類的邏輯,不像數學是單維單綫的,而必須是多角度、多維度的考慮,才能剋服複雜度所帶來的先天不確定性。
美國對英國的財政釜底抽薪,也經過了40年、三輪的過程;我何嘗説過一次就能讓它從國際社會消失,剛好相反,我也討論過美國在霸權交接塵埃落定之後,應該會反過來和中國親善,届時中方也只好假裝已經忘卻他們的惡行。我和其他政評的差別,在於我認爲這種2、30年後的長期必然結果,是和當前政策考慮無關的廢話;建言的目的,應該是要在未來幾個月到幾年的進程中,保證並加速歷史的和平演進,尤其要從細節上選擇最優解,而其所引發的利害差別,從世代的觀點似乎不重要,但其實出入往往以萬億計,這才是評論國安外交策略的用意,不是學術性、沒有實際影響的清談。
回到你的問題,這裏的關鍵在於,財政金融上的打擊效果,原本就有不確定性,但其他人完全忽視它才是極端,我的意見是即使不致命,也會讓美國病在床上十年,而這正是一般空談的中國和平崛起所需要的餘裕。你看在一戰後,英鎊的國際額分只降到50 \% (現在的美元是60 \% ),英國一樣很快主動推動了《Washington Naval Treaty》,接受美國可以有100 \% 同級的海軍,這相當於美國自發邀請中國簽1:1平等的限制核武協定,保證了老霸主放棄以武力打壓新挑戰者的選項,不正是所有人一致同意的戰略目標嗎?我只不過指出什麽都不做、坐等老天眷顧,不是最優方案罷了。
\subsection*{2021-11-15 23:18}

做這類分析必須考慮美國行政效率的指數下降現象,典型的案例是加州高鐵的建造時程和價格,所以要靠印錢完成“下一次工業革命”,即使假設無限印錢毫無代價,可以指數性地提高投資額,時間上的拖延卻是無解的。而且不只是執行效率有問題,在資金配置上,現代美式體制内公關謊言充斥,拜30年前的遺產,沒有落入大對撞機的陷阱,並且留下一個强大的半導體產業,但2000年之後就因爲無限印錢而使市場紀律和政治文化進一步腐朽,當前幾十個核聚變公司幾億幾億地從市場吸錢,比起SPAC的危害居然還微不足道。換句話説,即使資金能無限成長,投資效率也會因應地下降,長期來看,很可能是絕對負面的事;這裏的典型案例是16世紀的西班牙,從美洲獲得比原本GDP高出一個數量級的金銀礦,結果卻是迅速腐敗,成爲新進强權的獵物。
我自己在大二開始對歷史有真正的興趣;在研一遇上六四事件,開始思考政治體制的問題;到研四明白高能物理是死路一條,爲了留下後路,開始閲讀經濟性文章。30多年的自我訓練,至今也不過是掌握了小部分最重要的議題罷了。例如現在經濟學裏的周期理論,還是依靠外加(Exogenous)變數直接驅動周期性(這竟然是近年諾貝爾獎的得獎研究),這顯然是錯的:實際機制必然是内生(Endogenous)自發的波動;如果我的專業是經濟學術界,當然會花幾年時間去把正確的理論研究出來,但我是無正式職位的散人,時間的最佳應用在於當中國的Walter Lippmann。一旦決定了人生的大方向,就應該一往無前、專心致志地努力,放棄遠離核心任務的課題,把可用的時間精力專注在傳播和教學上。
作爲數學家,要思考社科問題,的確是很極端的轉變;物理原本就是等效理論,連定義都是操作型的,如果對所學有正確深刻的理解(然而在論文至上的體制原則下,能做到這點的是極少數),並不難應用到社科議題上。我對你的建議是,先習慣操作型定義的思路,同時從歷史著手,吸收大數量的案例,然後慢慢地歸納出其中的脈絡,一步一步提升自己思維的Metalevel。這一般需要至少幾十年的全神投入,不過我的博客可以幫助讀者加速這個過程。
\subsection*{2021-11-14 16:26}

他們都是聰明人,但也都有兩個先天的嚴重缺陷:首先是受教育期間欠缺嚴謹的邏輯訓練,往往不自覺地以直覺、聯想、習慣和衆議來取代因果推演。其次是英文能力不夠,無法直接體會英美的邪惡文化和深刻敵意。前者可能是文理分科不合理的問題,我已經多次討論過了。後者在導致最終誤判的過程中,有幾個特別普遍的誤解,我在此稍作討論。
首先,他們雖然也看到英美傳媒對中國的造謠抹黑,但推己及人,自動以爲那只是給普羅大衆消費的普通内宣,決策階級另有一套信息系統,所以Biden、Blinken和Sullivan等人心裏應該明白新疆等等議題是無中生有的外交籌碼。然而真相是,英美體系下的事實,原本就是隨群衆的主觀意識而自由演化的,他們的造謠抹黑,是一種湧現現象,每一個官員、編輯、記者都貢獻0.0001 \% ,同時也真心相信其他的那99.9999 \% 都是事實,所以自己的造假是在為“Greater truth”服務。換句話說,國務院官員可以在資助、安排東突僞造指控的同時,良心非常滿足,認爲自己假造的證據,只是被中方遮掩而無法直接獲取,己方的努力只是跳過這個技術問題來“揭發真相”。因此決策階層其實也完全相信中國是反人類的邪惡軸心,即使在道德層面也有强烈的厭惡仇視情緒。
其次,中國是一個人情社會,尤其是體制内脫穎而出的人,必然很會搞關係,所以難免在外交上又推己及人,也想要搞好人情關係,以爲這能帶來通融方便。然而Anglo-Saxon的長袖善舞和中國的人情關係有本質上的差別,前者是基於個人主義、極爲現實的交易性關係,尤其在霸權爭奪和種族歧視的背景下,表面上的和善純粹是爲了有機會時捅刀的方便。如果中國也能日常勾肩搭背、稱兄道弟,等到美方有需要時就立刻落井下石,我倒也不會擔心了,但這顯然不是中國外交的行爲模式,所以戰術和解真不是適合中國當前戰略需要的正確路綫。
最後,有些人可能是想要拉攏“溫和派”(Kerry)以孤立“激進派”(Blinken和Sullivan),但疏不間親,這頂多只對溫和派主管個人有效,不可能影響其他部門。在Trump政府裏,Mnuchin能幫忙,不但是因爲他主管的業務與貿易戰息息相關,更重要的是他原本就代表華爾街的利益,所以總統也得讓他三分。現在全憑簡單聯想而試圖對Kerry重施故技,卻沒有考慮當前氣候議題並非美方打擊的重點,以及Kerry根本沒有什麽政治能量,真是太天真了。
\subsection*{2021-11-14 15:37}

唉,我想以前也提過,智庫文章上送,不像外人想象的那麽直接容易,底層根本無權決定送往哪個部門,一般是反過來由行政單位指定題目,限時提供。我的文章與衆不同,尤其不方便,智庫主管得先代爲翻譯成官話,任何批評性的論述基本就此消失。
我的希望,是體制内年輕的幕僚和學者能主動到博客來,直接吸收我的論點,然後再轉述上傳。畢竟大國博弈的戰略戰術,可能是人類社會最重要、最嚴肅、最複雜、最精奧的議題,如你所説,要做出正確判斷所需的先決知識和論證是全方面的,我在博客用來介紹這許多不同角度的文字已經極爲精簡、沒有一句廢話,但還是用了上百萬字。針對你的問題來做回答,老話獲得新意,其實是誠實嚴謹的邏輯辯證(Dialectic)所帶來的提綱挈領效應,是我重視留言欄討論的用意。
\subsection*{2021-11-14 12:55}

答案都是博客一再解釋過的:所謂的“合作通道/夾縫”,其實是白左外交理論中打擊/停火的單向自由選擇權。美國全民,而不只是共和黨人,一致同意中國是反人類的邪惡軸心,“良性互動”從何談起?至於“範式”更加不適用於美國政治,他們言而無信、出爾反爾,你以爲是Trump一人的問題嗎?Biden即使連任無望,也必須設法減少民主黨在國會席次上的損失,怎麽可能冒全民之大不韙和中方真正和解?而且他越是在對中打擊上猶豫,越會引誘共和黨對手在競選過程凸顯這個話題來做攻擊,那麽下一任總統也必然更加極端。
在美國民意和政治體制的現實條件下,中方頂多能爭取減低關稅,所以和戴琪談判我不反對。然而其餘的諂媚動作,徒然鼓勵美方得寸進尺,毫無任何實際收益的可能性:高峰視頻會議固然談不出真正成果,氣候問題中國自己做自己的事就好了,扯上美國幹什麽?難道你真指望他們可能捨己爲人、資助落後國家嗎?這還沒有考慮2024年黨派輪替,必然又有反復。當然,未來三年中美外交官聚會、吃喝、談笑的頻率可以大幅提高,有些人會想要把它與成功外交劃上等號,但是中聯辦正是以類似的標準執行了17年,後果是什麽?
然而鄉愿心態的最大危險,還是在於未來幾年美國經濟有摔落斷崖的可能,届時中方是應該處於“合作”模式,還是趁機解除美元霸權,一次性解決美國仇中的問題?中美夫妻論者喜歡談英美霸權和平交接,但是美國正是在一戰、二戰和1956年運河危機,一連三次趁人之危,對英國在金融財政上落井下石,才成功逼迫後者和平退讓。要讓美國人心甘情願地坐視中國崛起,只有兩條道路:一是像美國對日本和德國那樣在軍事上徹底打服,二是像美國取代英國那樣從金融財政方面釜底抽薪。我覺得在核子時代,第二條路遠遠更合適,那麽現在全心去搞和解,不正是事先放棄天賜良機,反而提高最終兵戎相見的機率嗎?
\subsection*{2021-11-14 03:03}

這些選項,雖然理想,但太過不切實際,完全沒有實現的可能,反而浪費外交能量和管道。在這方面,俄國同樣有先進經驗:Putin早就專注在最起碼的雙邊關係之上,尤其是大使館和領事館被過度裁撤的問題;這裏的始作俑者是Obama,在2016年大選後,以Russiangate(亦即號稱俄方干涉美國大選,Trump有把柄在Putin手裏)為藉口,不但驅逐俄國外交官,還把俄方在美房產扣押充公。Trump下臺之後,證據不斷被挖掘出來,發現那些指控都是Hillary團隊為選舉而捏造的,連如何層層外包的過程都已經被徹底確認,但Biden對最起碼的大使館復員都不肯談,畢竟黨爭重於一切。
只要對美國外交政策習性有點實際觀察瞭解的人,都知道和解就是一個幻想;中國在外交國安上表現得如此幼稚,實在令人扼腕。
\subsection*{2021-11-14 00:03}

你從第二段開始是讀書心得報告,無需評論。但是第一段有個很大的問題:雖然單獨來看,當前的和解步驟本身算是雙贏,但如果我們提升一個Metalevel,從大局背景以Game Theory博弈論(這裏的根本問題似乎又回到文理分科不合理,導致國安外交的智囊幕僚居然普遍欠缺數學訓練,沒有學過博弈論!)來看雙方決策的心態,就不再是正面的敘事。
首先從中方角度來看,戰術上尋求和解、戰略上準備長期對抗,固然也是可行的道路,中方智庫的言論卻不讓人樂觀:所謂美國的長期戰略利益在於接受中國崛起,只等新的理性政權上臺,這樣的説法在大國競爭下的戰略戰術謀劃中不應該有任何地位。畢竟如果人的理性能靠得住的話,就不會有兩次大戰,美式絕對自由主義經濟學也就應該行得通。事實上,我反復論證過民選制的强大反智效應,中方不但有英美這些例子,而且和台灣、印度、香港打交道的切身經驗記憶猶新,實在沒有藉口繼續堅持已經被證僞過無數次的錯誤假設。
從美方的角度來做分析,這些戰術和解的壞處就更加簡單明晰:Blinken和Sullivan所代表的白左外交理論,正是鼓吹美國可以吃乾抹净,單向自由選擇脫鈎和打擊的方向,然後對方仍舊會逆來順受,接受美方不想承受衝突結果時偶爾遞出的橄欖枝。所以任何和解都成爲這個外交理論的正面驗證,加强並支持這批人和這個理論的政治地位,保證後續的打擊强度不斷升級。這裏中俄面臨的正是古典的Prisoners Dilemma囚徒困境:前期和解的好處由中國一家獨享,後續美國升級打擊卻是俄方一同承受,短期看是中美占了俄國的便宜,長期整體來看是三輸;別忘了,囚徒困境又分一次性和反復性兩種,其最優解是剛好相反的,難道中方以爲世界到今年底就會停止運行嗎?如果這還不算是犧牲長期利益來換取短期享受的短視近利、飲鴆止渴,我不知道什麽才夠格。
\subsection*{2021-11-12 10:12}

先依據你的説法來評論第一段:
1.首先,美聯儲的忠誠不針對政黨或總統,而在於金融行業。美聯儲主席的決策,當然會配合政府開支,但這不是出於政治考慮的協作,而是被動地擦屁股;換句話説,優化的標準依舊是金融業的利益,而不是政黨的偏好。
2.其次,美聯儲對金融業的影響力,有50 \% 左右是裝腔作勢搞出來的“Confidence”信心(Greenspan時期曾經是90 \% ),如果換新人這方面立刻受損。所以如果我是Biden,就會讓Brainard留在副主席層級;但他是白左教的忠實信徒,爲了性別而讓Brainard硬上並非絕無可能。
3.美國濫發國債從來就不是直接違約的問題,因爲總有美聯儲來兜底;它的危險在於間接地推高通膨,然後導致貨幣政策徹底失控;這一點我已經解釋過幾百次了。
4.我很早就説過,一般人到35嵗之後,三觀就絕對硬化了,必須是有科學素養的聰明才智之士,用心壓制自己的反射性直覺,才能接受新的事實證據,扭轉成見,選擇理性最優解。你覺得Biden、Blinken、Sullivan這些人,算得上“有科學素養的聰明才智之士”嗎?
5.民選制的反智效應,隨時間而做指數增强。晚清沒有那個問題都撐了70年還死不悔改,英國打了兩次大戰、被掏空兩個世代才承認霸權衰落,美國獨霸資源豐富的美洲,要在沒有世紀級災難的前提下做出深刻反省,必須有更高一級的智商,這絕非目前就可以確定的事。
至於第二段:
1.我不是已經反復建議,徹底放棄中美夫妻論,主動挑選有利己方的角度來脫鈎?這除了替代美元之外,也應該考慮美國的經濟周期和金融需要,反其道而行。
2.美國的下一個經濟危機,是否會在兩三年内發生,尚且不能確定,所以更加不可能事先確認其主要導火綫來自何處。我的論點是,不同於2000年問題重心在股市,而2008年是房地產,這輪泡沫被美聯儲全力拖延,已經擴散到金融和經濟的每一個角落,SPAC只是股市方面最離譜的現象;災難固然可能引發自SPAC,其它的候選多得很。正因爲如此,我個人覺得能拖到2024年大選而不爆發經濟危機的機率,是小於一半的。
3.Mid-Term選舉之後,總統跛脚兼無能,只能靠行政命令搞白左花樣,要挽救經濟你想得太多了。
4.最近真忙得不可開交,只好優先學習,保證持續長進;連内定的下一篇博文都已經拖了一個月,所以要等有空才能再上《八方論壇》。
\subsection*{2021-11-11 15:55}

21世紀美國的兩大政黨,雖然都仍舊以選舉勝選為目標,其實在組織和運作上高度不對稱,有著本質上的差異:民主黨才是傳統印象中民選制下由各種基層利益集團組成的鬆散政黨,共和黨則由於過去半個世紀歸順屈從右翼富豪集團的徹底掌控和改組,獲得了若干革命性政黨的内部紀律和一致性。例如宣傳媒體,民主黨系的名目繁多、各自爲政,在政治任務上主要作爲對外鬥爭的喉舌;共和黨的主流卻就只一家《Fox News》,而且其攻擊撻伐往往是爲了對内肅清異見、維繫思想“純潔”。所以共和黨的“内鬥”和民主黨的是兩回事;像是這次13個衆議員投票支持基建法案,原本就只是圖利轄區内土豪,並沒有違反共和黨的基本教義,更不像民主黨的Joe Manchin那樣嚴重損害黨的核心利益,所以被罵純粹是内部紀律的需要,如果有實際懲罰,也只反過來更加强化向心力,不會影響中期大選的大局。
當然Trump是一個異類,他的絕對自私和自信往往違反了黨的利益,但是我認爲這是一個“exception that proves the rule”:2016年他在黨内初選中勝出之後,《Fox News》被迫進行了徹底的内部清洗和轉向,很明確地示範了共和黨系的紀律是如此的强大和重要,他們寧可被Trump綁架,也不願冒真正内鬥的危險。
1萬億咋看下不少,實際上完全不足以扭轉美國國勢衰退的脚步,反而在未來兩三年的關鍵問題(亦即金融財政)成爲額外的包袱。我認爲不只是Too little, too late,長期來算都不一定是正面的。
\subsection*{2021-11-10 00:51}

美國官員的基本修養,就是對著記者只説空話。這裏最起碼的要求,是内部決策即使要公開宣佈,也必須由專人專稿在特定場合發表。所以像是Sullivan這樣現場即興回答問題所提的,絕對不是確實的官方決策,頂多只是和幕僚聊天時曾談過的哲學論點,是對既有現實(亦即美宣早已沒有顛覆中國的影響力;這是幾年前博客讀者群就瞭解的事,參見前文《美國宣傳戰的新困境》)的認知,他自己都覺得是學術性的廢話,才能順口而出(如果有人因爲老年癡呆,而説了好像不是廢話的發言,事後官方還得趕快出面辯解“澄清”呢!),我不知中方政論單就浮面字義來斷章取義、穿鑿附會能得到什麽洞見。尤其Sullivan談話的主軸其實是中美文明體系的對抗,那麽閑談“不能指望中方改變”不正反過來暗示“只好直面對抗打擊”的心態嗎?能把這種發言做正面解讀的人,須要反省自己的邏輯能力。
至於中國的金融管理,雖然我作爲一個外人,很難管窺其内幕,但從2015年股災開始,就可以判斷其問題很深很廣(參見當時的留言欄討論)。這其實和學術管理的難處類似,都是行内人憑藉專業權威來忽悠決策者,而且因爲不像後者那樣有大批公開的論文可以檢視,以私害公的行爲更加隱蔽得多,如我這樣的良心人所能做的建議也就只限於大方向的原則性敘事。
\subsection*{2021-11-08 21:28}

我最近才剛提醒大家,美國人的公開談話毫無意義,無所謂聼其言,直接觀其行就是了。
過去這一周,美國外交國安政策是有值得注意的新動靜,但不是Sullivan的談話,而是CIA Director William Burns跑到Moscow去談烏克蘭。這事照理當然應該是國務院出面主導,卻由情報主管頂替,顯然事非尋常。這裏有三個可能;
1)Biden警覺Blinken的蠢笨,主動派Burns代行職權,那麽前者鞠躬下臺指日可待;但是依Biden的個性和智慧來判斷,這個可能性實在不大。
2)Blinken自己知道不受Putin歡迎,所以請Burns代爲傳話;但是考慮Blinken的智商和胸襟,這個可能性也很小。
3)九月初Taliban剛掌權,Modi就曾經邀請Burns和俄國國安會長官(Secretary of the Security Council)Nikolai Patrushev同時到New Delhi,表面上是分別會談,實際上必然有三邊聚會。有可能當時Burns就毛遂自薦要到Moscow進行後續討論,因爲兩個月後國際情勢有所轉變,東烏又成爲美俄衝突焦點,所以Burns剛好可以代表美方來談這件事。
不論如何,Putin自己也希望能藉機讓美國明白他保衛東烏的決心。換句話説,不打便罷,一打起來,俄軍不會停止在當前的停戰綫上,就算不直接滅國,也要至少占領烏克蘭所有的俄語區,包括主要港口Odessa和整個海岸綫。這個警告是有效的:美國國務院緊急派了中級官僚到Kiev,顯然是去下達指令給Zelenskiy,叫他有所收斂;而他也的確立刻乖乖聽話,開始大談和平共處。
我想提醒讀者,Putin在烏克蘭方向所面對的戰略局勢,其實和中方在臺海問題上一模一樣,也是時間有利自己:不但Nord Stream II即將啓用,而且到2024年和烏克蘭的油氣過境條約也會過期失效,届時可以簡單徹底關停供應,然後坐待烏方來跪求和解。然而Putin有足夠的智慧,明白正因爲戰略上求緩求穩,戰術上一步都不能退讓,結果雖然國力只有中國的小半,卻能得到美方的尊重和敬畏,從而確保國家利益和世界和平。
\subsection*{2021-11-05 07:30}

其實Virginia州長選舉,對Trump的政治前途是一大變數;這是因爲共和黨人勝選靠的是Biden的民意低迷,Trump反而是競選過程中被有意避免的話題。現在共和黨有可能會總結教訓,認爲中期選舉的決勝關鍵在於疏遠Trump,於是雖然後者對黨内初選仍有影響力,主流派會有更深更廣的共識,要盡可能忽略Trump,那麽他的政治生涯就將End with a whimper。
我在《新年的回顧和展望》和其他文章中提過,原本民主黨建制派在去年大選前的如意算盤是要橫掃國會,然後可以簡單推動一連串的白左法案,那麽爲了幹掉Sanders,推舉沒有決斷力的Biden和毫無能力和人緣的Harris就無關緊要。沒想到結果是慘勝,而且參議院出了兩個叛徒,連最基本的撒錢法案都無法通過。這一來,執政品質和黨爭技巧的重要性就被凸顯出來。現任政權裏不稱職的幕僚和閣員各自爲政、到處惹禍的後果,讓Biden優柔寡斷、無所作爲的個性被無限放大,自然導致四面楚歌、内外交迫。他會比Trump更惹人嫌,我想是白左做夢也無法預期的窘境。
從中方的觀點來看,美國内鬥導致行政癱瘓的長期影響當然是正面的,在中短期達成外交停火協議的動力同理增强,但反過來全面升級衝突的機率也從極小變成不可忽略。我建議在去美元化方向另闢戰綫,其實已經考慮了這一點,因爲這方面的早期措施都是中國單邊的内政,既沒有外交爭議性、也不像在軍事上針鋒相對那樣會有擦槍走火的可能。當然博客自一開始就反復解釋過,美元是美國霸權的基礎兼軟肋,在2014年美方的敵意絕對固化之後,去美元化就成爲中美博弈的決勝關鍵。此外當前美國經濟對量化寬鬆的依賴極深,明後年陷入金融經濟危機,將是在貨幣上出手的一大契機。這些都是論證過無數次的已知事實,不再贅述。
\subsection*{2021-11-04 01:16}

Sovereign Wealth Fund完全去美元化一點問題都沒有,因爲這基本就是一個國有的儲蓄基金,原本就應該隨意追求高回報、低風險的金融資產。不過這和中國沒有什麽相關,因爲中方的美元資產主要在於外匯儲備,和Wealth Fund是兩回事。
俄國的外匯儲備目前大約是歐元和黃金各25 \% 、美元20 \% 、其他佔30 \% ;這裏的美元比率依舊是遠低於中國的。不過討論去美元化的正確角度,不是要求人民銀行立刻大幅調整外匯儲備的貨幣比例(因爲外匯儲備是抵禦1997年式金融搜刮的盾牌,參見前文《美元的金融霸權》),剛好相反,俄國央行能在外匯上做到這一步,是多年來全面與美元脫鈎後的成果,例如其金融系統,包括融資,已經基本不用美元。我對中方的建議,在於先從貿易、外援、和尤其是金融等方向模仿俄國的經驗,盡全力替代、排除美元,然後能夠安心把美元在外匯占比壓到20 \% 以下,才算是有小成。
如同兩年前我警告必須儲備天然氣來彌補新能源的不穩定性,結果被當作馬耳東風一樣,留給中方的時間其實不多了,這是因爲美國有大約50 \% 的機率會在未來兩年内陷入滯漲性經濟金融危機,届時再手忙脚亂地搞貨幣脫鈎,賬面上的損失固然驚人,大戰略上錯失的自我保護和反擊作用則更爲嚴重。
\subsection*{2021-11-01 00:22}

我不同意。這次會談顯然如同自Anchorage之後的所有前例一樣,都是各説各話,這是美方心懷鬼胎、而中方想先拖過冬奧的自然結果,所以除了沒有在貨幣上出手可以質疑之外,中方的策略沒有大問題。
事實上,認真和美國談判本身就是自殺性的傻逼行爲,因爲他們的籌碼原本便是無中生有,遵守協議的字面規定毫無問題,反正另外找茬簡單得很。這不只是Trump的行爲模式,也不只是中方才面臨的困難,例如Biden在競選過程中大肆批評Trump退出伊朗無核協定,一旦上臺,不但對伊朗强加和Trump一模一樣的苛刻條件,而且當伊朗要求美方保證不再撕毀新條約,Blinken先回答說他沒有權力限制下任總統,伊朗方問:你們至少保證在Biden任期内不再另加制裁吧?結果就連這麽基本的前提,都被Blinken否決,那還有什麽可談的?現在伊朗根本完全放棄對美談判,偶爾出席會議純粹是給歐盟面子。如果中方連伊朗的智慧都沒有,還在做妥協和解的白日夢,以爲自己的善意(包括不打擊美元)會有任何回報,那麽被占盡便宜、甚至肢解消化,都是活該。
\subsection*{2021-10-31 01:39}

最近有一系列内幕消息,顯示Biden對外政策上的自殺性愚昧錯誤,全都由Blinken帶領中生代幕僚主導推動,連戴琪和他相比,都算是溫和派:據説戴還只不過是想延續Trump的策略,拿關稅做爲討價還價的工具,而Blinken卻毫無妥協的餘地,剛好相反,他是一個Closet Neocon,民主黨版的Rumsfeld+Wolfowitz,認定中俄是不共戴天的仇敵,一味堅持要全面升級衝突,包括越俎代庖、在貿易策略上也要求加徵更多的新關稅。
如果這個報導真實可靠,那麽基本不必指望這個政權能學乖了。這是因爲Blinken是Biden的長期心腹,外交和國安的總管,不論捅出多麽離譜的婁子,依Biden的個性都不可能在短短兩三年内下定決心來換人;例如AUKUS,他就只以一句“Clumsy”結案。換句話説,從中方的觀點來看,不但退讓會鼓勵對方得寸進尺,連對等談判都是浪費時間、給對方繼續升級的餘裕。後悔早先沒有及時對等反擊固然爲時已晚、無濟於事,不在對方軟肋(亦即美元)上另開戰綫,卻真讓人費解。這裏不但應該立刻加緊用歐元和人民幣替換美元,更必須和俄國中央銀行預先協調、定下預案,在下一輪經濟危機中把握時機,合力落井下石,但我沒有看到任何人民銀行做這類準備的跡象。
\subsection*{2021-10-28 22:47}

你説的大致不錯,但國債真不在Biden政府當前的考慮之中:中方所持的比率低於10 \% 的級別,過去幾年的購買量在1 \% 的級別,這些外匯存底主要是用來作爲預防美方金融戰的盾牌,除此之外你買多少美國人在沒有金融體系崩塌的威脅下毫不在乎。
中方當然不想主動過早武統,但問題在於事態已經升級到對等反擊會有擦槍走火的危險,那麽你冒還是不冒呢?不冒的話,美方必然又再升級,届時問題一樣沒有解決,反而更糟糕。這正是我反復預警過、可以事先避免的窘態。現在唯一合適的方案,在於打擊美元,但若是在這裏也猶豫不決,等到美國出現經濟危機,美方會主動把貨幣政策放在雙邊關係的核心,那麽他們固然有更大的理性動力來退讓,非理性的升級誘惑也同時增加。別忘了,如果美國人的理性思維能力能靠得住的話,我們根本就不會有當前的困難。
\subsection*{2021-10-28 11:19}

Biden的團隊實在太蠢,而我已解釋過多次,蠢蛋的行爲是無法依理性預測的,所以我們不能排除中方被迫武統的可能。不過真正打起來,美方無力保護台灣,一樣是丟盡霸主的臉,這絕對不算是美國的勝利。
今天稍早我提到冬奧在當前戰略決定上的影響,並不暗示它有什麽重要性。事實上從國家的角度來看,冬奧在這個中美博弈的關頭上掣肘的原因,不在於它本身的價值,剛好相反,正因爲它不痛不癢,因而抵制奧運向來是白左影響歐洲地緣戰略決策的後門,有著明確的歷史前例,歐盟内部的理性派無法阻擋。然而一旦有了這個裂痕,美國操控的白左親美派就可以順水推舟,進一步做出對中歐關係有實際嚴重破壞力的決策,這才是中方應該暫時忍讓的務實考慮。當然,在美國敵意盡露的前提下,申辦這類活動是授敵以柄,但2015年做決定的時候,14億華人似乎只有我一個確認中美衝突無可調解。
總結來看,我仍然認爲最可能的脚本,是美方在未來三個多月得寸進尺,中方則一方面放話警告(當然對美方完全無效),另一方面設法拖延。一旦冬奧辦成,中國自然有出手的餘地,美方應該會在震驚之下,主動謀求和解。不過這裏牽扯到不止一個蠢蛋(除了Biden政府,還有蔡),所以擦槍走火(例如國軍全力攻擊共軍機隊)並非全無可能,這也就是前面討論的被動武統。
\subsection*{2021-10-23 05:29}

剛好今天和小孩通電話,你這篇留言讓我感觸特別深。我對自己給兒子的家教是很自豪的;他大一社會主義課的老教授是比Sanders更左的老學究,上個月他看到我兒子出席他的座談會,第一句話就是“Of course XXX here would say we are not going far enough." 原因是小孩知道民選制的反智效應,而這些美國的“極左派”基本還都是選舉制的信徒,所以他們只批評市場經濟,卻對政治改革毫無反思。
一年多前,他女友的母親(哈佛畢業的白左)開口談新疆的“Genocide”和中國的“邪惡”,把他氣得當場爆發;事後我反復叮嚀,叫他不要公開談這些事,否則不但立刻社死,而且可能職業生涯還沒開始就先終結。然而年輕則氣盛,現在他又和室友為同樣的話題大吵一頓,被對方罵“中國納粹”(這裏的Irony在於那個室友是Trump的粉絲;兒子是爲了躲開白左才選擇這人)。我除了在電話裏重新開導一遍之外,也沒有什麽好辦法(很可悲的是,他說中國來的留學生比美國人更信仰白左那一套);畢竟我自己當年也是拿到博士,才足夠懂事,知道批評超弦是Career Suicide(其實光是不參與已經足夠摧毀自己的物理前途了),後來在金融界工作20年,始終對政治體制問題不予置評。衆人皆醉我獨醒,是人生最悲哀的事之一,我自己隱忍小半輩子,現在連累小孩面對同樣的孤獨、嘲笑和敵視,心裏很難過,可是又沒有其他對得起良心的選項。
美國或許能撐過下一場貨幣危機,但英國的集體瘋狂和自我毀滅應該是很接近臨界點了;我個人很期望大英帝國的崩解早日發生,因爲英國的軍事和經濟實力雖然不足為懼,但在宣傳洗腦上卻是和美國體量相當的頂級玩家,它的崩潰立刻將他們的話語體系Call into question,不但緩解中國面對的外交壓力,而且對身處西方國家而不認同其愚蠢瘋狂的人,也能減低彌漫社會的自動敵意。
\subsection*{2021-10-22 17:41}

本周美國國防部長飛到東歐去鼓動烏克蘭和Georgia反俄,公開做出(美式的)“承諾”支持他們加入北約,看看Putin是怎麽回答的:這是“Genuine Threat”,絕不容許;而且他對兩者已經都動過手,沒人能懷疑他的決心。
這裏的本質是美國人空口白話,騙小國去送死,而當地的政客也順水推舟,騙百姓去送死,能撈多少就先撈多少;指望這些人適可而止是緣木求魚。唯一現成有動機、又有能力遏制衝突升級的第三方是歐盟;但我也反復説過,Merkel的退休帶來一個歐盟戰略決策的空檔期(所以前一個讀者留言對Merkel做出完全負面的評論,是不用心讀博客的結果),是美國越加囂張的因素之一。不過Putin也不是喜歡打仗,他只是迫於形勢:越是不想打仗,越是必須及早强硬,讓美國考慮喪失卒子的外交損失,還有較大可能引發反對軍工集團的内部意見;否則多拖一兩年(明年底有期中選舉!),美方挑釁持續升級,反而是讓軍事衝突成爲必然,而且必須大打。
中國在美國越來越狗急跳墻的前提下,出於國内改革的需要,以及對東亞欠缺能夠遏制衝突的第三勢力的考慮,難免有人心存僥幸,想要矇混過關。要看出這會適得其反並不難,但要下定決心、打破官僚體系不作爲的習慣,依舊是絕大多數官員做不到的。我在2017/2018年的中美貿易戰一開始,就反復論證不能退讓一步,當時大陸主流學者、幕僚100 \% 對這個邏輯結論嗤之以鼻。現在中方知道反悔了,但一遇到真正的戰爭危險,非理性恐懼仍舊壓倒理性計算;這是過去幾十年教育失敗(亦即文理分科,導致社科專搞類比聯想,不重視理性思維訓練,尤其是邏輯因果論證)的後果。
\subsection*{2021-10-22 13:36}

Merkel的經濟管理其實還不算太差的,尤其是和其他歐洲主要國家相比,她至少算是謹慎(連諸葛亮都被如此稱許,這絕對不算壞事),願意聼工業家的意見,所以當白左思潮威脅到貿易實利,她有時能夠選擇後者,例如對華政策。真正的問題在於一般人沒有注意的地方:首先,她沒有足夠的專業眼界和魄力(參見廢核;這很可能是幕僚素質的問題,不過這也是西式民主的通病);其次是她沒有關心新增財富的合理分配,尤其是幾千家Mittelstand的家族企業,在她任内許多忽然成爲暴發戶,而老員工卻什麽都沒拿到。
至於給華語界印象很差的柏林機場貪腐事件,其實怪不到總理頭上;這是因爲我以前也提過,德國的中央和地方分權特別徹底,地方政府完全有辦法單獨搞出特大的婁子。事實上,它代表的是德國社會文化和國民紀律的普遍腐朽,比它更離譜的還有一個Stuttgart 21工程:一個簡單的火車站從1980年代開始計劃,1994年公佈(你如果以爲那個“21”指的是2021,就太高估他們了),2009年開工,到現在才建了一半,預算已經膨脹到100多億美元,學術界都把它寫成案例來研究(參見https://www.emerald.com/insight/content/doi/10.1108/JPIF-11-2019-0144/full/html),讀者可以參考。在這樣的歷史背景下當十幾年的總理,不試圖撥亂反正固然是遺憾,但維繫整體秩序不墜也不是每個人都能做到的事。
Merkel真正的貢獻,在於把歐盟帶大到當前可以積極準備進一步整合的地步;這個過程可不是風平浪靜,有好幾次停滯甚至崩解的可能,參見《歐盟内部的無色革命》。如果歐盟如願成爲聯邦,Merkel絕對有資格列為第一級的Forefathers(Foreparents?國父/國母?)。
\subsection*{2021-10-13 08:59}

有關台灣地緣戰略位置的重要性,我們六年前就討論過了。至於臺積電,的確有其價值,但是(1)它雖然領先三星和Intel,並沒有拉開到一整個代差的地步,而且半導體的所謂一代也就只是18-24個月罷了;(2)當前全球半導體有嚴重的供不應求現象,不過半導體這個行業素來大起大落,如果明後年出現金融、貨幣或經濟危機,供需態勢很可能立刻反轉;(3)中方自己剛剛開始推行一系列深刻、困難的政治和經濟改革,短期内會比較虛弱,要到2025年左右,電動車外銷大幅搶占日本的出口市場額分之後,才有足夠動力引發下一波强勁的經濟發展和國力上升。所以總結來看,我不認爲一個短暫的半導體銷售高峰應該被拿來作爲重大戰略決定的主要依據。
\subsection*{2021-10-12 00:00}

美國人的公開談話毫無意義,無所謂聼其言,直接觀其行就是了。如果她不是有私心成見,而能專心爲國(指美國),中美貿易談判早該有些進展;參考過去這一年的通貨膨脹和供給短缺。
Biden的原意可能只是要避免Hilary的團隊,用些新人是附帶的選擇。但是從AUKUS那個案例可以看出,選擇雖然是附帶的,並不代表它們不會有嚴重後果。至於新生代的水平低,並不是全國人口世代之間真的有智商普遍下降的現象,而是學術界進行了高效的逆淘汰,願意想實事、說實話的人根本沒法發論文,早就被迫轉行,連諾貝爾獎得主Stiglitz都被徹底邊緣化,過去30年才畢業入行的人,哪可能活得下去?你如果還有疑問,設想如果Biden找的不是國際關係和經濟貿易的專家,而是高能物理這行,那麽也同樣是連諾貝爾獎得主Glashow都被徹底邊緣化,60嵗以下全部都是做超弦的,你怎麽選?
\subsection*{2021-10-10 18:59}

她是做什麽思考?這還用問?她思考的當然只是自己的利益,而且是短期政治利益;前者是壞,後者是蠢。台灣政壇有權力的政客,無分藍綠,哪一個不是如此?我沒有興趣浪費時間精力來重複廢話,若有再問者,一律拉黑。
至於國際形勢,陳水扁是在美方明確警告之下,硬闖紅綫;現在美國國務院反過來為台灣撐腰,蔡只是看了主子臉色,叫幾聲附和一下,本質上和阿扁剛好相反。臺海局勢早已100 \% 成爲中美之間的二元博弈,台灣根本沒有任何插手的餘地,島内的媒體聲浪再怎麽鋪天蓋地,也只是關起門自娛自樂而已。再說一次,這個博客只試圖幫助讀者開闊國際視野和提升理論思維,台灣内部的爛事細節,不得在此討論。
美國政權高層從Reagan開始,越來越玩弄民粹,在Trump任期達到巔峰。但是不論有多麽反智和低級,至少Pompeo這些人還算是操綫的玩家。從Biden開始,所有主要幕僚反過來成爲被反智民粹操弄的玩偶,這是一個量變之後的質變,也是當前中方對美開展外交工作上,最大的戰術級別阻礙。
簡單來説,Biden的高級幕僚和他們的幕僚都是Ideologue,亦即白左邪教的忠實信徒。這本身已經是個嚴重的問題,但更糟糕的是,即使只看白左知識界,這群30、40、50幾嵗的“新生代精英”,智商也比上一代低很多,完全不能和Hilary原先預定的行政班底相提並論;像是AUKUS那個debacle,簡直就是初中生的思維層次,連Boris Johnson要求對法國事先保密,都看不出是個陷阱。我覺得這不只是民粹教條的直接愚化作用,而是教育界和學術界長期衰敗腐化的結果;參見以往我對美國社科界的幾千個批評。
最近幾個月的中美交涉,基本都是各説各話,除了釋放孟晚舟之外,並沒有什麽實質的進展,而且臺海問題還在不斷惡化之中。中方是從理性出發,説明不接受美方一手捅刀、另一手伸到中國口袋裏撈錢的美式雙贏(亦即單方贏兩次;很諷刺的,這原本是美國外宣用來抹黑中國的説辭;我一直覺得中國外宣可以簡單反擊說:“Same side winning twice” is the American version of win-win, not Chinese. After all, we don't proclaim "China First".);但是美方卻堅持認爲可以公然爲所欲爲、損害中國的核心利益,只要私下撒一句謊(“遵守一中原則”)就足以安撫中方。這個心態就像是習慣面對幼兒的初中生,和成年人發生爭執,卻拿出一個奶嘴叫對方安靜。從Anchorage會議開始,中方代表反復强調中國不是嬰兒,不吃這一套;但是美方無法理解,只是搔著頭奇怪,爲什麽奶嘴已經給了,對方仍然不滿足。
當然,在美國民意已經被仇中宣傳徹底洗腦的背景下,Biden的行動空間是很受局限的。但是若有足夠的智慧,依然可以做出比現在好得多(指為美國自身利益著想)的決策。有關解除關稅、結束貿易戰,是我上月已經討論過的雙贏選擇,而且由於有美國國内工、農、商界勢力的支持,並不會引起民意或國會的嚴重反彈。這裏的阻礙在於戴琪是個有仇兼有鬼兼有病兼有癮的反中鬥士,爲了滿足一己私欲而肆意危害公益。如果我是Biden,第一件事就是撤換貿易代表。然而現實裏的Biden和Trump剛好相反,對幕僚盲目信任、縱容、放權,被連串捅出這麽多明顯的大婁子卻依舊不知反省問責。
在軍事戰略方面,美軍早已反復試圖教育政客,指明無力打贏一場西太平洋的局部戰爭,但是Biden政府的高官一樣是聼而不聞,繼續活在自己的幻想世界,以爲只要持續加碼,總能逼中方在臺海問題上退讓。我想我的讀者們都有足夠的智商可以看出這必然適得其反:美方唯一的理性選擇,在於停止升級,凍結局勢,另尋下臺階。但是這群美國高官能否及時長出幾個腦細胞,是很大的疑問,也是當前危機的真正起源和動力。
兩個月前,Lithuania挑起事端的時候,美國國務院即使忽然人人中風癱瘓,什麽都不做,也不會出事,反而是中方必須耗費駐歐的外交資源來處理善後。偏偏他們突發奇想,沒事找事,宣稱要跟進,於是中方先禮後兵,由胡錫進出面,不但指出這有嚴重的後果,而且還點明了衝突升級的下一步,亦即中國戰機進入台灣空域巡航。這就像是被一個小流氓拿刀要挾,只好把槍掏出來,說你再不停止,我要射你大腿了。結果Biden和習近平通電話,依舊只有那句“支持一中”,事後繼續升級;換句話說,痞子一面嘴裏說大家是好兄弟、不要動手,一面拿刀揮刺。本周有中國戰機到台灣ADIZ打擦邊球,等於是鳴槍示警了。
戰機進入ADIZ這件事,讀者應該理解幾個細節:首先,西方媒體的報導,完全來自國軍的一面之詞,尤其飛機的數目,是地面雷達操作員的估算;因爲共軍的電子戰能力對國軍已經拉出代差,這很可能是在電子戰機干擾下的誤讀,實際上的規模可以低一個數量級以上。其次,ADIZ在國際法上毫無意義,純粹是美國發明的國内法,所以西方宣傳特意用“Airspace”“空域”這個含糊的字眼,以矇騙大衆。事實上,台灣的ADIZ有超過1/4涵蓋大陸本土上空,甚至深入到江西省境内,在客觀法理上完全站不住脚。這次中方戰機的航綫,距離台灣海岸綫超過150公里,相當於海峽的整個寬度,與其説是進入台灣空域,還不如說東沙附近來得精確,顯然中方仍處於先禮後兵的早期階段,並沒有重劃紅綫。
但是如我先前解釋的,美方始終把此事當作小孩子游戲場裏的打鬧,以小霸王自居,繼續霸凌耍賴,不但不知反思收斂,而且進一步升級挑釁,通過媒體傳聲筒,間接公開了美軍駐台培訓國軍部隊的真相。一般人以爲這是最近的操作,相當於派兵侵占中國領土;其實此事由來已久,是上世紀遺留下來的現實。當時美國國力完全壓倒中方,既然給了面子,小心遮掩,中國也無法可施,只能假裝沒看見。照理說,美國憑著上代的遺留,悶聲占便宜,中國雖然實力今非昔比,也不會爲此而挑起事端。偏偏國務院那些自認的天才們,居然以爲這是可用的籌碼,適合拿來羞辱對手。更可笑離譜的,是在此同時,白宮卻想方設法,要安排Biden和習近平之間的高峰會議,其心思剛好和七年前的安倍一模一樣(參見前文《習安會有譜嗎?》),是要吃中方豆腐,一面占便宜,一面向國内國外聽衆宣示馴服了中國。讀者應該注意到,這其實反映了短短七年下來,美國實力衰退之嚴重,居然已經淪落為和日本同一級別的角色。
所以當前的中美對峙,可以簡單描述為手裏有傢伙、心中計算實利的成年人,面對一個裝腔作勢、只在意同伴面前形象的小流氓頭子。前者雖然試圖理論、不斷給出警告,但後者以爲他也同樣在裝腔作勢,自己人多,可以威迫對方束手投降。在槍都已經鳴響一次之後,美方竟然又向前進逼了一步。上一次有類似的事件,是日本把釣魚臺國有化,我們都知道後來是怎麽收場的。
\subsection*{2021-10-10 06:36}

For people interested in the US debt ceiling, it is beneficial to understand its origin. The US started out with a barebone federal government. It was not until the Civil War did it begin to consolidate political powers familiar to the modern society. In fact, federal taxes were levied only on a contingency basis during major wars, and regular taxation became constitutional as late as 1916 with the passing of the 16th Amendment. This is an important historical background unknown to most.
In 1917, the US entered World War I and once again began to issue war bonds. Because the main job of a parliament/congress had always been considered to be the authorizer of taxation ever since the days of 《Magna Carta》, the US congress was naturally concerned that recent changes in the constitution and external circumstances would combine to lead to government's unchecked taxing power, by first creating huge debt loads as fait accompli. Out of consideration for practical feasibility, they simply set a ceiling and left the details to the treasury secretary.
Over the subsequent century, the original intention has long been totally forgotten, and the debt ceiling first became a formality and later a partisan tool. Unfortunately, as a partisan tool, it is not just counterproductive but also unidirectional, thus making its repeal effectively impossible. In fact, it got progressively worse, when in 1995, the congress instead got rid of the Gephardt Rule, which said that as long as the spending had been properly authorized, borrowing money to fund the spending was automatically legal.
The partisan utility of the debt ceiling is unidirectional because the Democrats never have a problem raising it, while the Republicans selectively obstruct the process whenever the White House is not under their control. Therefore, although the former have all the reasons in the world to repeal the ceiling completely, the latter are adamant in its preservation, and they have the senate filibuster rule on their side. The whole long-running mess is simply a reflection of the overall dysfunctionality of the US political system, and nobody expects it to fixed anytime soon.
\subsection*{2021-10-08 17:31}

Let's first do a little review of history. Unfortunately, I don't have reliable data on China's workforce going as far back as 1978, so we have to settle for close proxies.
Please visit https://pubs.aeaweb.org/doi/pdfplus/10.1257/jep.26.4.103 and read 《Understanding China's Growth: Past, Present, and Future》. Scroll down to PDF Contents 6 of 32 (page 108), and you will find Table 1, the last row of which shows that during the 30 year period of 1978-2007, for every 1 \%  of per capita GDP growth, only 0.0705 \%  comes from increased labor participation rate.
A simple look-up of historical data on the web shows that during the same period, the real GDP grew to 1738.70 (1977=100), corresponding to an annual rate of 10.02 \% , while the population grew to 140.52 (also 1977=100), corresponding to a rate of 1.14 \% . Thus real GDP per capita grew at the rate of 8.75 \% , out of which 8.75 \% *0.0705 = 0.62 \%  came from increase labor participation.
With both the population growth and the increase in labor participation taken out, China's GDP growth over those 30 years would have been 8.13 \%  instead of 10.02 \% , after a deduction of 1.76 \%  (a simplistic arithmetic sum would not be accurate because of the compound effect) annually. Sure, that would make China's rise somewhat less "miraculous", but would it be enough to change the historical narrative qualitatively? Clearly not.
The analysis at https://www.globaldemographics.com/china-labour-force concludes that the average annual change in Chinese labor force will be -0.28 \%  over the period of 2007-2037. Contrast this with the positive 1.76 \%  number from 1978-2007, and you will see that (1) yes, it will be a drag; but (2) it is not the end of the world. Further salvation can be found in the fact that urban (read: industrial and post-industrial) labor force will keep increasing all the way past 2037. Since urban labor is far more productive than its rural counterpart, overall "effective labor" (i.e., labor force adjusted by its intrinsic productivity) will keep rising. The doomsayers should dust off their introductory macro-economics textbooks and put in the study hours they skipped during college.
\subsection*{2021-10-02 02:12}

What's there to be gained by being unnecessarily proactive? Britain is falling apart, and the US is not far behind. China can simply stand its ground and refuse to commit suicide (Or, as Western propaganda calls it, "democratic and market reforms"), while waiting for its opponents to finish doing so.
\subsection*{2021-10-01 08:37}

車體、懸吊、刹車等系統,基本是承襲既有的技術,已經高度優化,不可能再大幅改進或壓縮成本。馬達也是歷史悠久、非常成熟的科技,只有電池和能量管理電路還有降價的空間,所以未來五年達到與内燃機的Cost Parity應該沒有問題,然後再慢慢隨量產效應以及中國廠商的市場額分增加(中國製造向來有為全球壓制通膨的作用)而進一步減價20 \% 左右也很有可能,但絕不會有倍數的差距。尤其現代汽車智能化程度越來越高,晶片等電子元件所占的成本額分從2 \% 一路上升,十年後可能接近20 \% ,這反而會引發零售價格的膨脹。
目前基本沒有任何回收動力電池的產能和技術;如果鋰的價格再上漲一些,或許會有經濟效益;不過這必然會滯後於電動汽車的銷售量成長。
前天我談韓國電池廠商點錯科技樹的事情,其實對細節大幅簡化,把所有的問題都歸罪於三元電池之上。現實細節當然是遠遠更複雜的:磷酸鐵鋰和三元電池的對比固然是鋰電池安全性的頭號考慮,但其他的因素,例如生產綫的精度和一致性,以及包裝技術路綫等等,都對自燃的機率有影響。造成GM停售電動車的LG Energy Solutions,就是不但All-in三元電池,而且選擇了最危險的Pouch包裝(BYD的Prismatic是最安全的,Tesla的Cylindrical居中),所以才會讓Chevy Bolt成爲自行火柴。當然安全性越高的技術路綫,能量密度越低,這也是爲什麽韓國人決定(讓顧客)冒險犯難的理由。我想内燃機被換代改爲電動的真正動力,是環保而不是性價比,所以一旦電動車的生產成本達到Parity,無需政府的補助之後,後續的技術改進和成本壓縮會主要被用在延申航程,或者拿航程來交換安全性,而不是繼續降低零售價格。長遠來看,汽車對大多數民衆都將永遠是一個帶著一點奢侈性的大筆消費;考慮到交通阻塞的問題,這並不是件壞事。
\subsection*{2021-09-29 16:59}

澄清一下,Mercedes在2030年轉爲100 \% EV的計劃,專指歐盟而言,對其他地區還在觀望之中。

先説德國。德國政壇當前群蟲無主,即使幾個月後塵埃落定,新執政集團必然也偏向弱勢。但是Merkel向來就是無爲而治的典型,所以基本不會有什麽實質上的大改變,頂多就是嘴皮上玩賤。中方體諒他們也是美宣的受害者,包涵一下也就罷了,中德經貿關係應該不會有嚴重的波動,畢竟不讓德國在中美博弈中全盤倒向美方是當前的重要外交政策方向之一。多年來我反復强調,中國外交戰略的重點在歐盟,而爭取歐盟的重點在法國;最近的新發展不是完全印證這個預測?
至於日本,其屬性是昂撒集團最重要的外緣附庸。我在《再談Biden任期内的中美博弈等議題》一文中,已經列舉出中方應該采取的正確認知和姿態,這些預測是長期性、宏觀性的,所以當然不須要每半年更動一次,請自行復習。以下只補充前文沒有詳談的細節,來和新的大選結果發展做簡單的綜合對照討論。
我在這篇《美國制華歷程分析及對中國外交政策調整的建議》正文裏,解釋了昂撒集團的核心(亦即美英兩國)在未來兩三年都面臨斷崖式崩潰的危險(尤其是英國),所以中方只要以治待亂,以靜待嘩(參見《孫子兵法軍爭篇》)就行了。其實日本也類似,只不過需時較長、程度較緩,而且有其獨特的背景,以往我一直沒有機會提及,在此詳細解釋一下。
日本經濟是外貿主導的,而其外貿的主力產業,原本在於消費性電子產品支持半導體、以及汽車支持精密機械和重工兩個大方向。然而前者在過去30年被美國有意打擊瓦解,已經大量流失到韓、台和大陸,現在日本的國力根基只剩後者。然而汽車工業正面臨創建以來最大的換代轉向,從内燃機向電池動力過渡。我以前曾多次談起這事,雖然只論證這是中國奪取造車市場額分的百年一遇良機,不過只要稍微再往前推論一步,考慮汽車是很老的產業,全球總需求量不可能有像電子產品那樣的急劇波動,那麽很簡單可以推論既有的產業龍頭不但必須讓出市場額分,而且會面臨營收和利潤的急速緊縮。當前國際上汽車外銷的龍頭是誰呢?德日再加上美韓。
電動汽車的興起速度,如同當年的光伏一樣,也因爲中國的全力投入而遠超產業早先的預期。在中、德、日、美、韓五個主要玩家之中,只有前兩者屬於第一梯隊,他們EV佔所有汽車的生產額分,在2022年前半就會超過20 \% 。不過雖然現在兩者並駕齊驅,卻只有中方掌握上游動力電池的關鍵技術和產能,所以德國不可能長期維持並列第一的局勢,參見《我對引入美國投行的一些看法》一文中的討論。
第二梯隊是美韓,目前的估計是到2025年他們生產的EV可以從内燃機搶占20 \% 的額分。這兩國不是不想加速,而是受制於韓國電池生產商點錯鋰電池科技樹的影響,一連串EV自燃事件導致生產銷售基本停頓,必須從頭研發、擴產安全性較高的磷酸鐵鋰技術路綫(所以BYD的商業前景十分光明,不過請不要在博客討論炒股的議題);這其實是科技學術管理重要性的又一個例證:選擇三元鋰電池尚且有這麽嚴重的惡劣後果,把大量資源浪費到完全無用的假未來科技上,真正是自殺性的行爲,參見下文日本的例子。
日本點錯的科技樹(亦即氫氣燃料電池)遠遠更加離譜,從一開始就可以簡單預見今日的窘境,我也的確白紙黑字地寫下來(參見前文《永遠的未來技術》)。然而你如果去看本田和尤其是豐田企業主管最近的談話(例如Akio Toyoda去年在日本國會作證時的論點,參見https://www.autobodynews.com/index.php/industry-news/item/24053-toyota-ceo-going-all-ev-could-cost-japan-millions-of-jobs.html),那真是典型的商業恐龍,他們居然預期到2030年才能做到20 \% 的EV額分,届時中國、歐盟和美國市場的EV必然已經達到或接近80 \% 的占有率(例如Mercedes已經明確計劃在2030年之前,停產所有的内燃機),日本的汽車外銷也因此必然會大幅萎縮。Toyoda(他不但是豐田的老闆,也是日本汽車產業協會的主席)自己的估算是上下游生產綫會面臨550萬的減員,這還不包含對整體經濟的間接影響。届時日本的國力若還能維持現在意大利的水平,就算是不錯的了。
新當選的首相岸田文雄是安倍派的人,所以外交政策不會有什麽大變動,頂多是權力不穩,必須多搞些民粹花樣而已。其實中國已經在絞殺日本,只不過手段是非常間接的外貿競爭,需要5-10年才會有明顯的成果。中國政府不必耗費一指之力,就可以靜待日本的全面崩潰。希望以上的討論,回答了你的問題。
\section*{【海軍】有關航母的一些新消息}
\subsection*{2021-11-02 01:23}

J-20是現成的,改艦載版的工作量遠低於從頭完善一架不成熟的概念機,而且海軍可以簡單要求沈飛幫忙加强結構,所以不可能是爲了時間或者研發人力的限制。至於增加產能,純粹是資金的問題,而且除非成都的地價比瀋陽高出幾個數量級,為J-20增建生產綫也不可能會高於F-31。這件事唯一的邏輯解釋,是成飛已經有足夠的訂單,沒有政治積極性去趕盡殺絕,而中航爲了保護沈飛的飯碗,先動用自己的政治權力要求成飛退讓,然後全面扭曲了提供給海軍的技術資料,依托自己的專業權威,壓倒了海軍内部的理性質疑聲浪。這正是我反復討論過的,集專業權威和政治權力於一身後,必然導致的絕對腐敗。如果掌握政治權力的中航管理階層,和擁有專業技術的設計單位沒有人情瓜葛的話,就不會有這種損害國家利益來維護小山頭的醜事。
\subsection*{2021-11-02 00:56}

Strengthening the chassis for catapult/cable-arrest operation is nothing magical. It can be done to virtual any design and incurs a universal cost in weight increase by about 10 \% . The navy could have easily asked the two design bureaus to share expertise in modifying J-20, so this issue simply cannot justify the choice of the inferior plane. What you read was just more smoke and mirror excuses made up by liars in the PR department.
\subsection*{2021-10-31 21:06}

完全不成立。電磁彈射的特點之一就是適應力強、應用性廣,何況還有預警機、加油機和運輸機的彈射需要。就算是50年前的蒸汽彈射都可以簡單放飛F-14,J-20根本不成問題。
我説過許多次,邏輯分析只能可靠地推斷最優解;如果對象(海軍)不在乎最優解,那麽理性預測就不相關irrelevant。沒有相關的理性解釋,那麽自然只能靠撒謊或狡辯了(《觀網》顯然只是復述沈飛公關人員私下的胡扯;這是“内幕消息”的主要缺點,亦即消息來源很可能想做忽悠)。如果有足夠的基本知識來認清謊言、以及嚴謹的邏輯能力來看出狡辯,往往能得到額外的側面佐證,因爲最優解的那一方是不須要撒謊或狡辯的,大對撞機、核聚變發電、量子通信、量子計算和尤其是英美對中俄抹黑的虛假公關也是類似的道理,所以我説中方在新冠起源問題上跟著搞陰謀論,是不智的自我殘害。
\subsection*{2021-09-18 20:13}

脫歐所造成的危害,已經開始明確呈現在日常民生,而且還在日益惡化之中。原本如果有稱職的政府,或許到了第二年或第三年,可以有好轉,但是Boris Johnson是和Trump一個模子印出來的民粹小丑,不知如何解決問題,只會拼命鼓動自己的那33 \% 基本盤,既然後者剛好是選民中最愚蠢的那33 \% ,這些“政策”決定當然是不斷地拿槍打自己的脚。
今年六月蘇格蘭區域國會選舉中,脫英派獲得多數,立刻宣誓要推動脫英公投。之所以一直沒有後續動作,原因很簡單,蘇格蘭國家黨黨魁Nicola Sturgeon自己也公開承認,她們還沒有絕對100 \% 把握會勝選,既然脫歐的後果還在惡化之中,多等一兩年沒什麽大不了的。我估計她在6-12個月後會提出法案,發動公投,然後英國中央政府會全力阻撓,最終必然會鬧到最高法庭,這也需要6-12個月,然後訂下日期開始籌備投票,這又是3-6個月的延遲,所以大約2-3年後蘇格蘭會正式脫英。届時Wales還很難説,但北愛應該會跟進(因爲DUP已經開始從内部崩潰,Sinn Fein可能成爲執政黨)。既然Wales不是一個“kingdom”,而是一個“Princedom”,United Kingdom聯合王國指的是England和Scotland兩個王國的聯邦,那麽自然可以說UK即將不國。
上個月北約聯軍受美國單方面決定,從阿富汗倉促撤軍,期間英國外相Dominic Raab剛好在Crete度假,拒絕回國處理事務,受到多方面的嚴厲指責,於是本周Boris Johnson調整内閣,把他調為法相。一般人以爲是他受到懲罰,但其實Johnson一向不在乎執政優劣,只管是否忠貞,這次的調動,也只是表面上安撫輿情,實際上特別多給了Raab“副首相”的頭銜,可以簡單看出他根本就不是受罰。我認爲這裏Johnson選擇法相職務來安插自己最信任的内閣成員,還有另一個用意,亦即前任法相Buckland不但是Wales來的,而且不在Johnson的核心圈子内,既然蘇格蘭脫英的大官司即將到來,有個可以放心的忠實追隨者當法相是很重要的。
\subsection*{2021-09-18 09:46}

這個AUKUS同盟,純粹是慌不擇路、狗急跳墻的騷操作:澳洲政府固然腦殘歷史悠久,Boris Johnson也是不世出的小丑,真正讓人唏噓的,在於Biden手下那一批年輕的NeoLiberals戰略智商竟然等同共和黨的NeoCon。上周在白宮向中國遞橄欖枝的同時,國務院卻在打算與台灣做外交升級,就已經算是很明顯的内部精神錯亂跡象,再考慮美國上月從阿富汗撤軍,跳出來反對的,居然包括建制派重鎮Council on Foreign Relations,可以斷言Biden完全脫綫,各個利益集團自説自話,美國已經無所謂外交戰略可言。
爲什麽成立AUKUS做核潛艇交易是個餿主意?在軍事上,澳洲那8艘核潛艇最早也要到2040年才能開始服役第一艘,等形成系統戰力已經是2050年之後的事了;中美博弈早已塵埃落定,緩不濟急。與此同時,搞同盟中的同盟,立刻就讓既存的所有其他同盟都大幅貶值,尤其法國本來就是北約中的異類,這下子得罪了他,北約解散的可能性從可忽略一變成爲不可忽略,已經是得不償失。然後還有違反核不擴散條約的問題,以及中方的反擊辦法(例如提升軍售水平給幾個美方不待見的國家;當然俄國對中方出售Yasen-M級核潛艇也更加名正言順)等等。英美能得到的,最多只是澳洲軍費的油水,這對英國是海市蜃楼(因爲它有大機率會在兩三年内分崩離析),對美國是杯水車薪,完全不足以彌補自我分化親美陣營的效應。
美國當前制定外交政策的團隊(據稱有Biden的高級助手與國務院沆瀣一氣,主張升級對臺關係),顯然是極度的不入流,水準還不如Navarro,依舊活在單極世界,完全沒有考慮中方有簡單的反擊手段:像是胡錫進所提的派戰機到台灣上空碰瓷,應該並不是他自己的點子,而是中方準備的預案之一,原本可能受美國軍艦在南海諸島碰瓷的啓發,在美國違反一中原則的前提下作爲回應,倒是極爲合適的。届時吃不了兜著走的,不只是蔡英文,美國也一樣沒轍。
\subsection*{2021-09-09 18:22}

航母是專職欺負弱小國家的利器,在超强對抗中,其實非常脆弱。光看航母數目和質量,是很外行的事。純粹從軍事層面來看,中方也應該先把隱身轟炸機和核動力攻擊潛艇這兩個大短板在技術水平和部署數量上都先補齊了,再談航母全球部署。
事實上,即使軍事裝備全面領先,也還有其他戰略工具必須考慮,例如海外基地和盟友等等;既然這些必要的支持因素違反了中國的基本外交原則,那麽對複製美軍的全球控制誇誇而談,我覺得是毫無意義的。再説,2035年,中美對抗早已塵埃落定,航母緩不濟急,難道要用來全世界狂轟濫炸、當世界警察嗎?
中國在完成霸權交替後,不該也不會成爲新的美國,在全球戰略上永遠都是防禦性,那麽10艘核動力航母的價值何在?要說國際宣傳,其他有益於反饋經濟的手段多的是,何必往無用的軍備這個大坑裏扔錢?這還沒有考慮國際輿論對中國最正面的評價,正是專注於經濟、不輕易動武這個特點。小粉紅想求爽,劉慈欣和何新的幻想著作多的很,軍事評論員不應該去搶他們的飯碗。
\subsection*{2019-10-20 03:50}

這類問題,隨著科技和社會結構的進化演變,是必然會出現的;一個朝氣蓬勃的國家,必須與時俱進,時時刻刻檢討反省新的現象是否對整體公共利益有助益。你如果去看美國40-60年代的輿論,他們大體上可以做到這一點;從70、80年代開始,就反過來了。這個反轉最早的始作俑者,是芝加哥大學的Milton Friedman;他開始宣傳“Greed is good;greed is the source of all human progress.”我看過他在電視上受采訪,他舉的頭一個例子居然是Einstein!但是不論他的歪論如何離譜,芝加哥大學硬是能出雙倍的薪水給年輕經濟學教授,所以在1980年之後,成爲美國經濟學的正宗。
擔心國家社會整體利益就是“善”,堅持事實與邏輯是“真”,排除商業性低級娛樂是“美”。能强調追求“真善美”的國家民族,理當興起;反其道而行者,該當沒落。
\section*{【美國】【工業】熱帶風暴之後}
\subsection*{2021-10-29 02:29}

過去40年,美國將20世紀中期的社會主義政策扭曲反轉,中底層勞動力被徹底踢出經濟利益的分配桌,到現在已經基本回歸歐洲16世紀的Indentured Servitude,這是當前工資上漲而勞工仍然在不斷主動離職的背景因素。
我在一年前就把新冠的長期影響總結為增大貧富不均,而且是全球性的、長期性的影響。專注在美國,這反映為經濟供給鏈斷裂,社會勞資對立,然後政治進一步分裂,金融危機隨之而來,都是我已經反復預言幾百次的事,不再贅述。當然,Cassandras Curse在歐美比中國更爲嚴重,但我原本就不指望對英語世界做建言,反正無視歷史巨輪走向的族群,終究會被碾壓。這裏的問題在於美國的軍力和財富纍積極爲豐厚,人類世界如何管理它的衰敗過程、避免瘋狂的同歸於盡,才是我們必須關心的事項,這也是博客建言的重要方向之一。
\subsection*{2021-02-19 16:13}

這些論點基本和我自己的觀察相符合。我對中國歷史雖有所知,但主要依靠近代和現代國際社會作爲事實根據。中國歷史悠久,在國家治理上的經驗極其豐富,即使只專注在一國,也有足夠的案例給出同樣的教訓。所以在政治上是否有大局觀,與中國還是外國沒有關係;真理只有一個,堅持事實和邏輯的理性思維就能達到正確結論,這是我一再强調的研究態度。
在另一方面來看,我認爲做這類比較抽象的大角度總結,其實是非常有實用性的手段。正如我要批評大對撞機的無用,高能所能丟出種種專業流行詞來搪塞和混肴是非;一旦把話題提升到整個學術界的腐敗,反而容易讓主事者明白其中的道理。要指出某個利益集團的花招,他們可以簡單拿許多表面有聯係、實則不相干的細節來扭曲議題,但如果先拿這些歷史教訓來做鋪墊,就更容易揭穿真相。
\subsection*{2021-02-18 14:04}

其實又是美國宣傳洗腦系統,為一個低劣體系塗脂抹粉的結果。
廠商欺負消費者,是一個典型的以強欺弱問題,那麽解決方案就必須有另外的强者下場,為弱者代言;這是邏輯上的必然結論、無可避免,差別只在於這個新的强者是誰。美國從兩個羅斯福總統時代,原本采納的聯邦改革是由新設立的各種專業行政單位來監督企業、保護消費大衆。這些官僚當然有可能淪爲低效、傲慢、腐敗,但可以藉著健全文化、嚴格監督和高薪養廉來預防,事實上也證明是極爲成功的。
到了1960年代,美國財閥被民主黨的一系列帶有社會主義性質的政策(尤其是Johnson的Great Society,居然敢把扶貧明文列爲政府的重要任務)徹底激怒,開始醖釀全面反撲奪權。我以前已經反復談過他們在文化思想上,亦即通過控制學術、媒體、智庫所作的工作;與其同時,他們也必須腐化和弱化監管他們的聯邦官僚。在這方面,我談過游説行業的爆發性發展和現代政商旋轉門的出現,但是法律系統從行政體系奪取監管仲裁權其實也是其中的一步。
剛好當時出現一個叫做Ralph Nader的消費者權利Advocate(怎麽翻譯?推行者?鼓吹者?),針對的議題就是你所説的汽車安全裝置,鬥爭的平臺則是媒體和法庭。財閥很快地就意識到,不但必須避免新的行政監管,而且可以反過來借力打力,推翻既有的聯邦監管體系,於是和媒體財團以及律師行業形成聯盟,全力扭曲國家社會處理這個問題的反應。手段是由新聞媒體大力報導,侮辱抹黑行政單位,突出美化律師和法庭,然後通過打官司獲得從個人觀點是巨額的賠償,實際上從廠商的角度是相對的小錢,最後再由好萊塢塗脂抹粉,進一步創造出新的萬靈丹形象。這後來成爲白左教的基本教旨之一;是的,我剛剛描述的這個過程,正是白左思想在法政方面的濫觴。
實際上光用簡單邏輯,就可以判斷這對弱勢的消費者來説,是完全行不通的賠本生意。首先,律師集團的利益和社會公益,根本南轅北轍。他們的最終目的在於費用最大化,所以過程越是拖延杳渺、規則越是繁瑣晦澀,對他們越有利;發掘真相、做出公平合理的判決,則是他們完全不在乎的事。如果你射箭根本就瞄錯方向,那麽中靶當然是機率極小的隨機事件。其次,既然律師界的原則是有奶就是娘,那麽在行業榨取最大利益之後,剩下的麵包屑要在强弱懸殊的官司雙方做分配,依舊是强者才僱得起“好”律師,能在他們有意創立的繁瑣晦澀規則之下,更好地鑚漏洞。所以充斥媒體的“良心律師”,其實是Harry Potter小説中的神奇動物,並不存在於我們的宇宙中;現實裏,企業財團樂於接受律師界每年幾千億美元的剝削,因爲這依舊是遠勝於被有效監管的選項。
我談私有化的危害的時候,總是要提起法律行業,指的就是這個效應。並不是說要禁掉私家律師,而是絕不能容許他們主導維護社會公義、仲裁監管企業的功能,更加不可以讓他們制定自己的規則。法律界的正確定位,在於對行政監管做額外的監督彌補,但是行政爲主、法律爲輔的原則必須堅持,這是因爲法律系統過於低效,而且先天就不講理,只在乎死條文,一旦獲得太大的權力、不再受節制,所謂的解答就會演化成比原本的問題更可怕得多的吃人怪獸。
\subsection*{2020-09-26 22:36}

我不是這方面的專家,不過近年閲讀他們論文的印象是,全球暖化對中國的降雨量問題反而會有幫助。尤其是華北人口高密度區缺水,原本是一個很嚴重的長期環境威脅(所以才有南水北調工程),但是最新的預估是當地降雨量會在本世紀逐步提升,華中和華南反而會下降,這是不幸中的大幸。
當然平均降雨量不是Whole Story:全球暖化依舊會使極端旱澇天候更加頻繁,這還沒有算入海平面上升對上海這些沿海低窪地帶的危害,所以防治這類天災仍舊需要更大的投入。
我在《歐系文明的起源》一文中曾解釋過,人類之所以進入農業時代正是受益於末次冰期結束後,氣候相對穩定,年復一年的耕作規律成爲可行,聚集村落可以歷經數個世代來開墾建設,然後才有發展出文明的機會。而全球暖化的危害,最主要的就在於它所蘊涵的高度不穩定性。人類的科技今非昔比,但是人口密度也已經成長到土地負擔能力的極限。尤其亞非貧困國家還有許多Subsistence Farmers,他們首當其衝。展望21世紀,美國人只想要繼續搜刮,歐洲人搞空口說白話的形式主義,來自中國的幫助是這些經濟底層人口的唯一希望。
\subsection*{2020-09-08 22:52}

在關閉領事館和限制記者群這類外交事務上,已經做到立即對等反應。外交官員反唇相譏也成爲日常。
至於貿易戰,我早已説過,一旦錯過立即對等反應的短暫機會,就陷入被動挨打的狀態,再想反擊,對方已經打到第四、第五回合,那麽前幾輪受害的帳算不算?兩年來做的妥協犧牲是否前功盡棄?Trump只剩短短幾個月的任期反而不能忍了?事實上,立即對等反應是唯一能保持戰略主動、有隨機應變餘地的選擇;兩年多前全中國所有智庫學人專家一致決定避免升級、消極地息事寧人,就已經實質放棄此後一直到美國下次大選之前的所有反擊選項,等同於給Trump一張空白支票,可以不斷另行開闢戰場,對特定企業(如華爲)一擊不中,也可以反復嘗試升級。我人微言輕,三年前講的話,中國政界學界不當一回事,我也沒辦法;但是我講的都是事實真理,日後的事態發展是有現實後果的,肉食者鄙、未能遠謀,那就承受這些後果吧。
\subsection*{2020-08-17 09:07}

歐美的重大分歧點之一,就在於法規條例的鬆緊程度,不論是消費者保護、網絡規範、稅務、碳排放、其他環保議題,都有著難以協調的差異。TTIP談不成、網絡稅成爭議、波音/空客在WTO的官司、甚至英國脫歐,都是這種差異的後果。而追究其原因,則是歐盟沒有經過Murdoch/Reagan/Thatcher的自由主義化,而且政治體系分散,沒有洲級的資本力量,所以他們的財閥遠遠不能像美國那樣左右每個法規的細節。
我認爲這是好事。中方應該尊重歐盟作爲法律創新的典範地位,容許、鼓勵和支持他們引領國際規則的制定,間接排擠美國在這方面的話語權。畢竟歐盟頂多只是有點自私,中歐合作可以談出建設性結果,不像美國逢中必反,怎麽談都只是不同型式的停火協定。
\subsection*{2020-08-16 23:13}

因爲迷信私有制,所以電力公司PG\&E必須自負盈虧,也無法和消防單位協作。
像這樣一方面强制自負盈虧,另一方面又要求繼續負擔公益責任,很自然地只能在以下的脚本裏二選一:1)基本放棄公益責任,社會整體利益大幅受損,例如PG\&E;2)因爲公益責任而入不敷出,反而成爲新自由主義用來洗腦宣傳、鼓吹私有制優越性的例子,像是鐵路公司Amtrak,常年作爲媒體和國會的出氣筒(Punching Bag)。
郵局USPS更加可憐,不但背著公益責任的包袱,還必須面臨私有企業的逐利競爭。近年有客觀的學術研究指出,如果USPS不是必須單獨擔負對偏遠住民服務的責任,它的經營盈利效率已經高過UPS和FedEx。但是財閥作爲私有制的既得利益方,手下的學術買辦根本不跟你講理,多年來消滅USPS一直是他們的目標之一,背後支持他們的農民剛好是最需要郵政服務的族群。這又是一個典型的火鷄投票過聖誕節的例子。
在新自由主義的指導下,不論公益事業選擇哪一條因應對策,也不論它們是私有還是公用,最終的結果都是服務水準下降、收費水平上升,這兩者都壓低經濟整體效率,而且是由底層民衆買單。像是停電頻仍,能負擔得起的住戶只好裝發電機,這是非常低效的設置,遠不如在電力公司的層級就做出足夠的投資來保證不會經常斷電;更不用提低收入民衆根本沒有這個選項,被熱死、冷死也只成爲一個被大衆媒體有意忽視的統計數字。我曾提過光在英國每年就有上萬人被凍死,美國的數字藏得更深,但絕對是更大更驚人的。中國民衆驚訝於美國人對新冠死亡以十萬計而無動於衷,其實是他們幾十年來對這類窮人無辜喪命的案例早已習以爲常。
\subsection*{2020-08-16 04:12}

我已經反復强調過很多次,英美玩弄這些宣傳洗腦的伎倆,只能佔一時的便宜,長久下來思想毒素危害自身,一旦非理性的謊話和成見深入文化之中,那麽除非亡國重來(例如二戰後德國;但這沒有絕對的成功保證,日本就沒有悔改),否則不可能根治。
如果不是美國的自我欺騙已經無可挽救,我怎麽會從一開始就否決和美國和解的可能,一直堅持要對等反擊?正因爲謊言和僞科學的蔓延太快、毒害太大,所以我才會急著建議整治中醫教、端正學術風氣;畢竟中國仍在國運上升時段,財富的纍積自然能暫時遮掩許多問題,但長期下來卻會感覺積重難返,務須及早預防解決。例如明朝對皇室子弟津貼過於優厚(世襲罔替),幾代之後就成爲尾大不掉的財政負擔;清朝汲取教訓,一早對世襲罔替做了嚴格的限制,避開了明末沉厄的來源之一。
\subsection*{2020-08-15 18:26}

剛好相反,要認清事實比盲目接受政治正確難出太多了。在我出來寫博客之前,你能想象有人能夠對所有世界上的重要議題都做出深刻的分析,並且有90 \% 的預估準確性嗎?但是六年下來,多少讀者有足夠的求知欲望和能力來學習我所試圖傳授的真相?還不到一個普通港臺名嘴的零頭!就算是《GrayZone》這樣只專注於政治宣傳洗腦的單一議題的網站,能有多大的影響?
更加可怕的是,實話能有3.5 \% 的聽衆躲在角落裏分享,就謝天謝地了,但是“民主自由”那套謊話只要有3.5 \% 的人迷信、肯上街,就足以完成顔色革命,這正是宣傳洗腦在英美霸權架構裏的强力效果。我們要喚醒蠢人是不可能的,人類唯一的救贖路徑,在於釜底抽薪,推翻美元對外無限吸血的能力,然後靜待英美内部原本被財富遮掩的矛盾和問題進一步發酵,終於無力禍害世界爲止。
\subsection*{2020-08-14 10:45}

我對中國商界的現況不熟,不過美國會全面MBA化,的確是壟斷優勢的後果。
這裏的壟斷有兩個層面:一方面美國獨占世界科技產業鏈、儲備貨幣和經貿規則的一哥地位,所以能夠在國際上壟斷遠高於GDP和貿易額佔比的利潤額分;另一方面,雷根的新自由主義政策,使得在美國國内也是由財閥階級壟斷經濟生產價值。其結果,是美國的企業不須要擔心幹實事(亦即認真開發新技術),依舊可以賺得盆滿缽盈(例如高通靠收專利費賺大錢),那麽真正需要投入精力和資源的,就是内部分贓的競爭。
這些競爭也有幾個不同的層面:首先在國際上,靠著霸權開道,可以用金融手段控制他國的高科技企業(例如三星),無須真正參與生產;其次是國内的階級和專業之間的利益分配,打壓員工、榨取利潤是重要的熱點,所以脫穎而出的是Neutron Jack或Chainsaw Al,他們都以大規模裁員而得名;最後,既然沒有實事可幹,各級企業幹部的選拔自然成爲相互忽悠的人緣競賽。你如果去仔細想想這三層競爭,就會發現它們都極度偏好自吹自擂、不事生產的MBA;40年下來,連總統都換成這種人(Trump只從Wharton商學院拿到本科學位,但他的言行舉止依然是典型的MBA),那麽全面腐化當然是勢不可擋。
\subsection*{2020-08-14 10:30}

我同意,教育和學術是中國内政未來20年最大的隱憂。馬英九坐視臺獨課綱而不改,因而徹底腐化台灣社會,也是我早已一再用來强調思想教育重要性的例子(當然,還有美國、土耳其和印度等等)。
不過在這個議題上,我人在國外,對國内腐敗諸般細節的掌握不像大對撞機、Alstom和737 Max那樣深入,雖然大局上的判斷絕對是精確的,但一般人不懂邏輯主軸和細節旁支的差別,如果文章對後者沒有顯示出壓倒性的熟悉,他們往往就假設前者也是可疑的,所以我不是適合站出來大聲疾呼的人。
此外,我的博文素來曲高和寡,一般只當思想上的推手,用來點醒其他意見領袖,然後由他們進一步推廣正確的認知。像是《GrayZone》揭發美國人冒充香港民權鬥士的事,我知道《觀察者網》的編輯讀我的微博,隔幾天自然會轉載出來。你看,昨天他們就一連發了兩篇後續的文章,討論了許多細節;去年初的《美國陷阱》也是如此。這個整頓教育/學術界的話題,我會不斷提醒大家,但最終還是要有業内人士出面主導,就像楊先生在大對撞機一事上的作用。
\subsection*{2020-08-13 17:06}

那當然了。這是人類自從形成超越家族血緣的大型社群之後(可能起自七萬年前,參見《人類的起源》;最近有新理論認爲智人獲得壓倒其他Homo的跨家族結群能力,來自自我馴化Self domestication;這是因爲智人的腦容量減小、腦腔變圓、面部骨骼縮小、眉骨降低等等特徵,和其他家畜在馴化過程中的改變很類似,例如狗和狼的差別;換句話說,智人之於Neanderthal就像狗之於狼,在某種程度上,甚至可以説,智人是被自己族群裏的權貴階級所馴化的家畜,當然這個馴化的過程並不完整,所以才會到了21世紀還有我這樣喜歡做獨立思考的社會主義者;忍不住繞彎子駡人,我對敏感的讀者致歉),領導性部族先後此起彼伏的基本因素,我們在幾年前的留言欄也討論過中國歷史上朝代更替的規律。
\section*{【歷史】冷戰、學運和五四運動}
\subsection*{2021-10-06 13:15}

你的理解是正確的:毛思想的根本問題,正在於反智民粹。如果我有時空穿越能力,可以回到1977、78年當鄧小平的幕僚,那麽頭號建議,必然是要把反智民粹凸顯出來,作爲反思批判的核心對象。這裏的問題在於,當時全世界沒人能看出事情最深刻的本質(包括我在内,不過我有一個很好的藉口,亦即那年還在念初中;等到我上了大一,已經可以簡單看出台灣所謂的黨外,就是反智民粹,只是還沒有足夠的邏輯表達能力,所以清華的同班同學們,大概覺得我的政治意見很極端),所以十一大的諸多檢討和決議,都是治標而不治本,難怪時間一久,自然出現美化文革的心理動力。
換句話説,40多年下來,要做深刻檢討的機會早已錯過,兩代人在毛式群衆運動的思潮下受教育長大:要讓無知者理解自己的無知,永遠都是世界級的難題,何況是從小喊著“群衆至上”口號長大的那一群。我最近談中方對五四運動的定位錯誤,那還只是問題中很小、較爲容易解決的一部分。要對多數民衆講清楚民治和民享之間的根本矛盾,我並不樂觀。
\subsection*{2021-10-05 16:34}

孔夫子沒有把邏輯説清楚,這裏其實有好幾個層面:“成事不説,遂事不諫”的原因在於諫官的政治資本應該用在關鍵議題上,亦即懸而未決的民生大事;例如我批評學術管理,是因爲有改正的可能,至少十五五可以不再浪費資源在假未來科技上。“既往不咎”的前提則是必須洗心革面,不能繼續文過飾非,參見日本對軍國主義的反思(or lack thereof),所以美化文革在道德和實用層面都是極度錯誤的。
在“使民戰慄”這件事上,真正不該提的理由,是怕讓魯哀公得到錯誤結論,畢竟他不是理性知識分子,這個細節也與原本話題不相干,宰我又沒有深入解釋探討的餘裕,那麽純粹爲了炫耀自己知識廣博而冒引發誤解的危險,顯然是不對的。
\subsection*{2021-10-04 07:15}

研究解説歷史,當然不可能有100 \% 客觀絕對的精確性,但是在某些能夠達到99.9999999 \% 精確的議題上,無限上綱、非黑即白,把所有歷史認知都一竿子打成0 \% 可靠的主觀印象,卻是既愚蠢又有害的態度。對毛時代錯誤政策的批判也屬於這一類:我們可以確定它們是歷史性的人爲災難,但也完全沒有必要因爲政府的避諱態度去Invoke歷史虛無論,這裏唯一的難題在於國家和黨組織的歷史延續傳承,如果誠實、完整、精確地做全社會廣汎討論,雖然在狹義的事實和邏輯上完全正確,但只要在思維的Metalevel上提升一級,就可以簡單看出是不可行的。
道德規範的意義在於輔助政治,來達成公益的最大化。既然現代社會和經濟體系是如此複雜,而任何固定的條文都是極度簡化的結果,那麽自然不可能普遍絕對地適用。這並不代表它們沒有用;恰恰相反,不但絕大多數的人口沒有從第一原則來對每一個公共事務來做理性分析的能力、欲望和時間,Rule of Thumb是他們唯一的選項,而且即使是百萬中選一、能對一個特定議題的所有相關事實和邏輯做出完全掌握的人,也必須尊重道德原則作爲對内對外宣傳内容核心的重要作用,不能遇到Rule of Thumb不是最優解的時候,就把它全盤掀翻。這裏相干的Rule of Thumb道德原則是“誠實”。
毛領導的中國政府,曾經犯下極爲嚴重的錯誤,但是放任普羅大衆對這段歷史簡單下定論,卻必然會引發非理性的聯想,危及現今政權的正當性和合法性。既然習近平治下的中國政府,是當前國際上最先進、優秀、高效的政治體系,爲了保護人類社會整體的福祉(這也是一切政治和社會活動的最終、最高目標),對不方便的歷史事實做出委婉的遮掩和回避,反而具有道德正確性。換句話說,我的博客是理性高級知識分子的論壇,可以而且必須把事實翻開來說清楚,但這並不代表一般輿論也能夠或應該做同樣的事。
説得再簡單一點,我在《美國的開國神話》一文中討論的美國對自身歷史做出嚴重扭曲和美化,本身並不自動構成邪惡,因爲這是保護政權穩定性的必要舉措。它之所以值得特別挑出來撻伐,是因爲現在的美國已經成爲國際財閥的政治載體、人類社會的最大害蟲,對内壓榨勞動群衆、提倡反智心態,對外挑起動亂、遏制後進,而其開國神話是愚弄國内國外群衆、動員反動力量的重要軟件成分。這和中國形成强烈的對比,所以雖然表面上同樣是對歷史塗脂抹粉,實質上有剛好相反的作用,我們對兩者做出差別待遇,並沒有真正的邏輯矛盾。
\subsection*{2021-09-16 18:52}

這是個全球都有的現象,其實是女性在學界和職場開始逼近男性權威的結果,實際上可以算是社會進步的徵兆。你聽説過“Uncanny Valley”嗎?擬人的形象如果與真人的相似度中等的話,一般人心理會有好感,但是如果很像又不完全一樣,那麽反而會引發驚悚的感覺。男女在社會上的地位也類似:一開始只有少數女性就職,或者只擔任某些職位的時候,男人並不受威脅,一旦在數目和質量上接近平等,就會有反感了。然後女性自然會不滿那些有意無意的不友善眼光或不公平待遇,終於選擇用非理性仇恨來回擊非理性蔑視。
在歐美,因爲女性還要面對很高的性侵危險,仇恨更深,所以才會讓女權成爲最近“Woke”文化的主要成分。現在好萊塢的新電影,有大半反而是女主角把一幫男反派拿來修理,正是心理補償作用。我早在二十幾年前就很少看電視了;今年稍早偶然在Amazon撞上《A Discovery of Witches》,一開始也是很不適應,但稍作思量,理解到它的劇情基本就是《倚天屠龍記》裏的張無忌換成一個現代英國女子,心裏便即釋然,居然能把它看到完。
既然中央已經開始嚴格審查網絡和娛樂事業,掃除這方面的極端言論只是舉手之勞。我覺得指望普羅大衆普遍具有理性智慧,是緣木求魚,所以任何社會永遠都是充滿愚昧的偏見和仇恨。中國政府有能力遏制最極端的非理性言論和行爲,雖然只是治標,但既然沒有治本的可能,我們也就不必浪費精力,另求解決方案。
\subsection*{2021-08-05 22:18}

《讀者須知》第二條,要求留言不能采用俳句格式;這其實有兩層考慮:一方面空白篇幅浪費而不雅觀,另一方面從内涵來看,要做邏輯推演,就必須遵循大前提->小前提->結果的循環;幾個循環達成一個中間結論則自然形成段落;幾個段落再組合成全篇的邏輯架構,完成對主題的論證。所以會喜歡用俳句的人,基本都沒有邏輯思維能力,每一句話都是不相干的論述,純粹靠意識流聯想跳來跳去。我已經反復説過,在公共議題上應用這樣的純感性表述,除了公示你的智商和人品之外,沒有任何意義。
從管理部落格的角度來看,即使是綠衛兵來講歪理,只要還試圖遵循邏輯,那麽指出他們的謬誤是有價值的。相對的,完全沒有邏輯的口號,或者拿聯想來冒充因果的“論證”,則不但要求我重做整個客觀分析,而且還必須連帶評論一大堆主觀敘述,這對我是極大的折磨,成果卻毫無價值。我建議你反復重讀這裏的博文和討論,有空做做數學證明題,建立自己的邏輯能力,在有成之前,請不要再來發言;你的賬戶我代爲拉黑,以幫助你抵抗誘惑。
至於公辦基礎教育對階級機會平等和社會整體公益的重大影響,是我在過去幾年反復討論的重點。中共的新政策,扭轉以往私有化的邪路,正是改革必需的第一步。能從這樣的事實背景,説出“改成菁英才能進大學”,不但是反邏輯的奇異論述,也證明你沒有讀過部落格的留言對話,又違反了《讀者須知》的第一條,所以我在拉黑這個決定上實在沒有選擇。
\subsection*{2021-07-21 20:42}

我說“學運”,指的是中產階級年輕學生發動的反政府或至少是反政治人物的街頭運動。中產階級和有組織規模的中高級教育體系,都是近代才發展出來的;古代的“清流”議政雖然同樣也是在野知識分子干政的運動,但依舊有明顯的差別,例如不會上街游行示威,當時也遠遠還沒有達到後來行業分工的細分程度。此外這些清流所用的手段更適合對應其他現代常見的非官方政治力量,亦即現代英美民粹思想和大衆媒體中的“Fake News”現象:一方面假裝聖母,另一方面盡情造謠誇大抹黑。他們所批判的貪腐現象,當然或許存在,但既然這些人並不堅持理性或尊重事實,他們的解決方案、治理能力和道德人品並不能保證什麽先天優越性。例如Nero並沒有在大火中彈琴,Marie Antoinette也沒有真正説過“Let them eat cake.”這些都是反對派有意創造的謠言;奪權成功之後,國家在職業騙子兼野心家的主導下,也都很快陷入一段更糟糕的亂局。
既然談到Marie Antoinette,我暫時離題,講講法國大革命。一般人以爲現代民主體制源自美國獨立,其實18世紀建立的美利堅合衆國,很明顯地是一個土豪階級主導的Republic,而不是Democracy;換句話説,它原本就被設計成一個Plutocracy(當然,這個設計過程至少是相當理性的,亦即掌握主導權的大地主們優先照顧自己的利益;後來他們自我誇耀的種種“制衡”,其實都是爲了削弱政府干涉精英階級利益的能力(“暴政”Tyranny),自由媒體、市場經濟、獨立司法和民代立法,樣樣都方便社會強豪不受規範、或乾脆重訂規範),因此後來不論羅斯福等人如何努力改革,財閥要捲土重來簡單得很。相對的,法國大革命才是真正的群衆運動,但正因爲如此,自然必須完全依賴民粹口號和假新聞來做動員,非理性思想做主的後果,就是後來的“Reign of Terror”。可以說現代直選制Democracy從一開始,就充分展示了其内建的民粹、虛僞和愚昧等等因素;這距離我反復解釋過的民主基本悖論(民治和民享的互不相容)只有邏輯上的一步之遙。Marx在許多類似的歷史範例之後對直選體制仍存些許幻想,我認爲是他所犯的最嚴重錯誤,沒有之一。
至於你說“海晏河清...學潮就極難被煽動起來”,這並不是事實;人類社會從來就不是完美的,要吹毛求疵、做求全之毀容易的很,公平長遠的分析評價反而是極爲困難的,學生尤其沒有足夠的知識和能力來做判斷。例如美國社會在60年代要比20年代健全富裕得多了,那麽爲什麽學運集中在前者,只有工運發生在後者?學潮是否發生的標準是學生對國家社會的主觀認知評價,不是治國的客觀實際成就,你的論述必須假設兩者完全重合,這只有先進一步假設學生和群衆具有絕對正確的治國眼光和能力才能成立;你想活在哪個星球上是你的自由,但地球這裏的現實是剛好相反的,而且我在上周已經明確解釋過了。
言歸正傳。基層革命奪權須要動員群衆,當然只能走民粹路綫,過程中勝出的領導人也自然是搞非理性煽動的專家。然而一旦革命成功建立新政權,治國的正確方針恰恰必須反其道而行,以科學理性為基礎,堅持事實和邏輯。那些革命領袖往往無力做出相應的轉變,反而成爲包袱和纍贅;如果不甘寂寞、不願意退入第二綫,則更會成爲嚴重的禍害。這正是我上周所談,革命奪權上臺遠易於建設發展成功那個現象幕後的基本原因。這裏的重點在於它的適用性很廣,不止於學運,而是任何把體制推翻重來的嘗試都會面臨的危險;所以中共在鄧小平之後,能持續選拔理性人才、堅持公益爲上,其實已經是人類歷史上罕見的成就。那些無知、無良、為反對而反對的人,不但是在危害中國的整體利益,更威脅了現今人類社會的希望所繫,與他們作鬥爭是全世界每個有良心、有見識的知識分子的天然責任。
\subsection*{2021-07-12 04:35}

我認爲任何龐大的官僚體系,都會有訊息被基層執行人員遮掩過濾而無法有效上達的現象,合理的街頭運動有針對切身利益(沒有切身直接經驗的人,無法證實訴求的可靠性,就不應該跟著起鬨)被底層官吏或單位侵犯(請注意,體制和政策問題是無法經由街頭抗議而“改善”的,“改惡”的例子倒是很多;例如2006年倒扁反貪腐運動,參與人數和熱情都遠高於2014年的太陽花,結果是哪一個成功了?參見正文裏的詳細論證)而做公開訴求的社會功能,有其無可替代(或許在e-government完善之後,可以被取代)的價值。反過來看,如果濫用這個管道,假造、誇大訴求,其對社會公益的危害,甚於所指控的罪狀,應予嚴懲。大衆傳媒做地方報導的社會效益也在於此,所以同樣應該給予適當的保護和重視,對製造假新聞者則須調查追訴到底。
\subsection*{2021-07-10 03:07}

對有商業主管經驗的人,這件事是極爲淺白、不言自明的。即使在過去幾年,習近平高調進行反腐,《經濟學人》最喜歡用的批評,也是會影響經濟成長率。我個人並不同意《經濟學人》的邏輯,因爲不同的經濟發展層次,需要不同的地方招商自由程度:90年代建設基本的消費導向輕工業時,地方政府給予私營企業優惠補助,是最高效迅速的路綫;到了2010年代想發展半導體和高端機床這類長周期、大投資的先進工業,還任由地方官員胡搞,純粹騙補就是必然的後果了。
回到89學運這個原話題,當時那些學運學生能看到的,並不是有憑有據的貪腐現象,而只是自由市場經濟裏必然的資本集中效應(亦即有些人先富起來)。如果鄧小平搞反腐,首當其衝的就是運用公權力來方便私營企業建廠的官員,即使他們的動機不是爲了回扣,而是真心想要發展工業。在整個國家的政治文化依舊籠罩在毛式均貧思想的大環境下,這絕對會將改革開放扼殺於襁褓之中,所以根本不是可行的選項。

部落格的老讀者都知道,任何我不想回答的留言,不論因何原因,都會被直接刪除。你來質問爲什麽被刪,本身就露餡,證明你沒有讀完所有的討論。先禁言一個月,如果還不夠你讀遍整個部落格,就請你繼續安靜研讀,直到全部吸收爲止。
新讀者須要有自知之明:不但我對你的人品、性格、知識、能力不熟悉,無法針對性地做答復,所以特別容易出現不想回答的情況。而且新讀者(尤其文科生)往往還帶著很重的壞習慣,語焉不詳、無病呻吟,為了喜歡聼自己説話的聲音而説話;如果我對你的留言做邏輯的濃縮沉澱,發現有一整個段落完全沒有任何實際意義,基本就可以決定你沒有資格發言。再加上過去這一年有近20人被拉黑,其中必然有重新注冊、捲土重來的,所以我對新冒出來的賬號更加不會有寬容的餘地。
\section*{【基礎科研】再談Google的量子霸權}
\subsection*{2021-10-03 09:55}

運氣好的話,或許能做出一個原型吧。運氣極佳的話,或許能在某個特殊目的優於古典電腦的AI。運氣達到天文數字的時候,或許這個特殊目的還真能有一點經濟或社會效益。
但是這麽一來,量子計算的整體實用價值的Expectation Value,還不如把某些特定的基因片段在人體内如何作用搞清楚,亦即基因工程的幾十萬個發展前途的其中一個,憑什麽來作爲國家的頭號科研方向?光是有可能用途是廢話,隨便哪個科幻夢想可能性都不是數學上的零;既然敢要求國家投入最優先的資源,那麽論證它價值的責任不在質疑者,而在推動者身上,而且論證的標準不能只是或許、最終、如果、可能會有一點結果,美國學術界認爲它是個發論文或給獎金的熱門題材更加是毫不相干的論述,這個論證必須證明它的實用期望值要比所有其他的科研方向都更高。原本Shor's Algorithm有超越經濟價值的軍事戰略意義,所以可以多給幾個數量級的Utility Weight加權,現在改成搞Deep Learning,這純粹是商業應用,量子計算連入門都不夠格。
\subsection*{2019-11-06 14:34}

我對兩三百年之後的事,無法置喙,但是在可見的未來幾代人生命中,除非發生如全面核戰之類毀滅人類經濟和文明的大災難,整體科技的不斷進步是必然的。
一般常見的誤解,是把整體的進步,誤認為個別專業的永恆發展。其實每一個領域都有極限,一旦低垂的果子被摘光之後,就會停滯不前,例如民用飛機的氣動造型,頭60年進步飛速,但是在噴射客機出現之後,逼近了音速障礙,在其後的60年,就基本沒有改進,這也是爲什麽Boeing會想要繼續沿用737的原始機體設計,以避免重新設計新機的費用與時間。
另一個例子,當然是高能物理,自從1974年確立QCD之後,理論上的進步不大於零,這也是我解釋過很多次的事實。
量子計算是一個新領域,倒不像傳統的Si基半導體那樣已經接近黃昏,但是全新的科技有更大的危險,就是絕大多數都沒有實用性,這一點我也寫過幾篇文章來討論了。例如現在的所謂第四代核裂變反應器有十幾種,30年後頂多有兩種能存活下來;核聚變反應堆的設計種類更多,但是50年後仍然可能是無一成功。
我覺得這類成功機率很小的新科技,如果有特殊背景考慮,例如量子計算對大國之間的鬥爭可能有巨大影響,可以適度投資,至少保持不落後太遠;但是絕對應該量力而爲,避免蘇聯被忽悠進科幻武器軍備競賽的窘境。
\section*{【海军】即将出现的新装备(三)}
\subsection*{2021-10-01 22:06}

拜智能手機普及之賜,感光元件飛速發展,過去十幾年民用衛星成像的解析度輕鬆進步,從5m、3m、1m到現在,0.5m並不稀奇。
不過H.I.Sutton(知名的高級軍迷;不過他的專業興趣主要在於潛艇,雷達反射截面和隱身技術非其所長)的這篇文章談的是SAR(合成孔徑雷達)衛星的觀察結果,那麽0.5m倒是有點驚人的;這應該是藉助於計算機算力的增進。至於“清晰圖像”,不必過度緊張;外形隱身只對頻段較高的雷達有效,所以22級(據説不是022,而只是22,就像093其實應該是09III才對;當然這種内部定名的細節,並不是太重要)導彈艇原本就不可能完全看不見。Capella Space的衛星雖然用的是X band,但因爲是從正上方拍攝,並非隱身外形的設計重點,結果還能讓影像變得如此模糊(對照周邊的樹木),已經很讓人滿意了。
\subsection*{2015-11-23 00:00}
不是先入为主,衹是这两个礼拜精神不好,老是犯一些低级小错(昨晚辅导小孩做高中数学竞试,简单的那半居然错了几题,反而是难的全对),所以连下一篇文章写了一半都不想继续。

有时事实证据太强了,我对其中的推理就不太细心;这是因为三四个考量中即使错了一半,仍然不影响结论。一个好物理学家和好的数学家毕竟还是不同的。我以前也解释过几次:我的分析细节或许有错,但是若不影响结论,我往往就不太在乎;读者若要吹毛求疵,可以,但不要无限上纲。

的确如你所説是7mm,我看错了,已更正;其他的论点大致成立。那个厚度虽然衹被説成"蜂窝材料",其实必须是非导体,不衹有支撑作用,不能取消。

潜艇的隐身是声学上的,和雷达波没有什么关系。要上水面舰只在工程上有可能,但是在经济上这种结构比iPhone的荧幕还脆弱,性费比很低,我还是觉得不会发生。
\section*{【美国】【金融】美式民主的真正主人}
\subsection*{2021-09-28 22:20}

Sachs始終是有良心、有理想的,只不過年少得志,成名時還沒有足夠的智慧,迷信課堂所學的那套美式經濟學,被幕後的美國財閥利用來矇騙蘇東政府。我覺得他其實也是受害者,事後他也明顯悔悟,30年下來試圖彌補過往的錯誤一直不遺餘力。
Stiglitz沒有這樣的污點,在過去30年被排擠得更加厲害。像是這類有良心、有見識的學術大佬,正是中國外宣的先天盟友;這裏的意思不是說必須私底下塞錢,那反而會弄巧成拙,而是政策細節不應該打這些人的臉。例如今年稍早有一個諾貝獎得主的聚會,達賴也獲得邀請,結果中國外交部居然公開譴責。這又不是官方訪問,難道達賴在紐約買麵包,你也要譴責那個麵包店嗎?結果Stiglitz被迫寫公開信反駁,這是典型的拿槍打自己的脚,真讓人懷疑幕後決策的外交部主管是否打著紅旗反紅旗、故意扯國家的後腿。
\section*{【台灣】返臺隨筆}
\subsection*{2021-08-24 00:42}

我在博客反復提過,貧富不均是21世紀人類面臨的三大問題之首,其背後的原因是市場經濟自然使資本不斷集中和纍積,所以要靠政治手段逆水行舟,先天就極爲困難。更糟糕的是,資本完全掌控了既有的國際霸權,所有現行的國際規則、制度、機構和理論,都是英美在過去200年精心設計來保護自家財閥的,所以只有中國的興起,才能挽救人類社會。
習近平在過去幾個月出手打擊國内的壟斷性財閥,然後進一步提出“共同富裕”的政策方針,已經是比我預期的更積極、强力的措施,值得全世界的良心人稱贊和支持。然而我們必須認識到,這些依舊只是非常初級的步驟,而且必然會面臨國内國外財閥的反撲。下一階段的鬥爭,將會是全方面的,而其中最讓我擔心的有兩項:
1)資本無國界,國内財閥早已有準備做資產轉移,國外的巨鰐則始終想要吃下中國的金融體系,以往的壁壘或多或少減低了金融方面的危險;現在中國正式向資本宣戰,卻同時大開國門,在金融上“與國際接軌”,這顯然是戰略冒進,承擔了不必要的風險。
2)一旦試圖把以往資方獨霸的權力和利潤,拿來與國内的勞方和消費者分享,那麽無可避免地會影響財團在國際上的競爭力。如果中國有獨步全球的研發環境和團隊,倒也罷了,偏偏中國的科研學術界沉厄甚深,造假、抓權、騙錢的風氣極盛,甚至形成多種畸形產業(例如代爲假造論文、假期刊等等),幾乎每個我深入研究過的領域,都由自私自利、竊位自肥的學閥主導。這在追趕階段,因爲只需複製歐美的成功前例,還不成問題,一旦中國進入第一梯隊,那麽美國在過去6、70年研發路綫選擇的低效現象,在中國會遠遠更加惡劣,很難想象不靠996和泡沫投資來維持競爭力,技術水平要如何進一步提升,整體經濟要如何進一步發展。
總結來説,我一直認爲解決貧富不均非常困難,即使不看整個人類世界,先專注在中國本身,而且只考慮初級步驟,也有三大難關:政治意願、金融策略、和研發效率。目前習近平只解決了第一項,後面的金融管理和學術風氣,是我多年來反復討論的議題,但是中方並沒有全面采納那些建議的跡象,所以我對這個“共同富裕”政策能否持久深入,並不樂觀。
\section*{【基礎科研】談量子力學(三)}
\subsection*{2021-08-22 01:29}

化學和生物一樣,是近乎100 \% 實驗性的科學。20世紀的化學理論,基本都是量子力學在多體系統下的各種近似公式。我個人覺得,在不可能有確解的系統裏求好的近似,其實往往比有確解的題目更難;物理學内的氣動力學也是例子。其實即使是高能物理,也早已明白所有的模型都只是等效理論(Effective theory,包括目前最尖端的相對論和標準模型);真正的確解,只存在於數學裏,而數學原本就只是對宇宙中某些特性的抽象化概念,並不直接對應現實。
Google AI號稱能對蛋白質摺叠做到90 \% 以上的精確度,這裏當然有你所討論的誇張成分,但即使是在若干局限下,能做到80 \% 的可靠性,依舊比除了冷凍電鏡之外的所有手段要優越得多,而冷凍電鏡在金錢和時間上都是極爲昂貴的,而且也有它自己的嚴重局限性。
我對AI在過去幾年的成就,印象很好,也鼓勵過年輕人入行;但是即使大方向是正確的,細節上依舊可能出錯,所以建設性的批評還是很重要的。
\subsection*{2021-04-30 05:15}

你斷章取義了。文小剛的全文是,“張量網絡理論就是量子比特理論。量子比特理論也許可以取代任何一個有良好定義的理論。 ”
這裏我的看法是剛好相反:張量網絡理論有沒有價值我不知道,但是量子比特理論的基本假設(“It from bit”,信息就是物理)是錯誤的、或者說是致命性的不完整。這是因爲信息有無限多種邏輯自洽的體系,而宇宙的物理法則卻只有一個;不先解釋這個矛盾,徑行在無限多個解之中找接近已知現實的模型特例,不是科學,而是Tautology。超弦起碼原本有個精簡版,可以用來預測時空維度;是在預測錯誤之後,才不得不放鬆假設、增加自由度以得無限多解來避免被證僞。而量子比特理論則是跳過做出任何預測的階段,直接開始水論文。自欺欺人的效率大幅提高了,但我不確定這能算是進步。
回到文小剛的原話,字面上可以解釋成他認爲量子比特理論並沒有良好的定義(正確!),“取代”也的確在論文出版數量上實質發生了,然而這並不像是他的原意。
\subsection*{2021-04-23 06:32}

宇宙中最猛烈的天體作用是黑洞,可以產生比LHC能量更高10\^9倍上下的粒子,然而這依舊距離Planck能階太遠,不能提供直接的綫索(即使人類科技足以發射探測器到黑洞附近做觀察)。
做非常精確的測量,間接地從小偏差上來反推高能級的物理法則,是唯一的希望,但這只是從“不可能”改善到“極度困難”罷了。本月初,CERN宣佈Muon g-2實驗的新結果就屬於這一類;公關稿號稱實驗結果與理論計算相差4.2個標準差,已經很接近5個標準差的傳統要求。其實4.2已經足以證明實驗做得很好,統計誤差很小,這個結果應該不是統計噪音。
然而他們忘了向公衆解釋的是,這裏的理論計算非常複雜,用到Feynman Diagram的第五層,而且包括强子Hadron,所以根本不可能精確算出,必須用許多近似手段來估計,連理論誤差都包含很大的猜想成分。更糟糕的是,目前用兩個不同的近似估算工具所得的理論結果,相差3個標準差;這不但立刻證明這些理論計算並不可靠,而且CERN只挑距離實驗結果遠的那個理論預測來大作文章,暗示新物理就在轉角後面,這是不太誠實的。實驗除了統計誤差,還有系統誤差,只能經由不同的實驗設計來排除;而理論值的估算誤差,在這個案例上,更是最可疑的嫌疑犯。在理論學家解決那兩個數值之間的矛盾之前,任何慶祝都爲時過早。
\subsection*{2017-07-21 00:00}
这些专业性极强的研究结果,一般大众媒体是不可能做出深入、持平、全面的评论,而圈内人又不愿得罪作者,不会出来澄清事实。所以即使是理工出身的读者也往往有如雾里看花,无法理解它的意义和重要性。

Fermion就是半自旋的粒子,所有物质的构成成分都是Fermion,例如质子、中子和电子。但是在写出它的相对性(Schrodinger方程式是非相对性的;半自旋是四维时空的特殊解,只有在完全尊重Lorentz Transformation的理论下,也就是相对性理论下,才能写得出来)量子方程式的时候,理论学家发现,除了一般"正常"的质量项(叫做Dirac Mass,这些Fermion就是Dirac Fermion,对应着有既有Fermion也有Anti-Fermion的现实,例如电子和正子)之外,还有另外一种可能的形式,对应着Fermion做为自己的Anti-Fermion。这类Fermion就依发明这个理论项的数学家Majorana而命名。

但是高能物理发现,所有宇宙中的基本Fermion都有反粒子,目前只有Neutrino因为实验很难做,所以还有丝微可能是Majorana Fermion。

这次的实验是Solid-State(凝态或固态)物理学家做的。固态物理一贯忽略固体是由许许多多个别基本粒子组成,而把它简化为一个背景的模型。如此一来,就好像一个人造的新宇宙,所以能產生真实宇宙中没有的粒子。这些粒子不但只是模型里的近似解,而且通常不对应真实的粒子。换句话説,它们是Virtual(虚拟)粒子,例如Phonon(声子)是晶格振动被量子化后的最小能量单元,但是在固态物理的方程式里,一样被假装是真实粒子。

这个Majorana Fermion的实验,所观测到的是否真正是Majorana Fermion,还不能被确定。他们所测量到的是一个半整数现象,Majorana Fermion是一个可能的原因,但是也可能是其他未知的机制,只不过Majorana Fermion的名头大,用来当标题更为惊人。高能物理的牛屎文化,显然已经传染到固态物理。

上周有另一篇关于"时间带宽极限"的报导(参见《观察者》的《中外科学家联手 能否打破"时间带宽极限"百年物理魔咒》),更是被作者的公关文章完全忽悠了。"时间带宽极限"其实只是电机工程里的一个Rule of Thumb,而不是一个物理定律。非綫性晶体能打破Lorentz reciprocity,也是已知的事实,凭藉的同样是固态物理把晶格当作背景宇宙的近似简化过程。我在第一时间就写了更正信给《观察者》的编辑,后来他依我的建议做了修改,包括引用了两段我写的文字来澄清事实,所以现在的版本至少没有明显的谬误。

现代科学家越来越向Donald Trump和Elon Musk这样的生意人学习,无中生有的吹牛已经成了惯例。每次有所谓的"突破",我就得考虑是否写稿或写信给媒体编辑来更正。正是因为不胜其烦,所以这次就没写,直到你问了,我才发发牢骚。\section*{【美國】言論自由的假相}
\subsection*{2021-08-13 10:51}

好萊塢的猶太主管不願意雇用他了,只能自己當製作人/導演,另找體系外的投資來源。《Hacksaw Ridge》的成本很低,全部加起來只有四千萬美元,基本和文藝片相當(差不多就是Scarlett Johansson原本預期從Disney的《Black Widow》拿到的數目,結果變成兩千萬,還因此對簿公堂),是Gibson的私交好友Bill Mechanic聯合劇中主角人物所屬的教會,加上澳洲團隊做的,澳洲政府也出了錢,然後由一家完全在主流之外的小公司(IM Global,聽説已經破產了)發行。這有什麽不對嗎?我說他的演藝生涯完蛋,指的正是在常規好萊塢體系内,輕輕鬆鬆當演員,一次賺個幾千萬的生涯。《Hacksaw Ridge》事後大賣,在籌備階段絕對沒有人預期。
這種無關宏旨的細節,對一般讀者沒有意義;如果有人好奇,應該自己先研究透徹,確定有需要更正,再來討論,不要隨手丟出來,等我收拾垃圾。
\subsection*{2020-07-27 09:14}

他們除了普遍是既得利益者之外,也有許多是浪得虛名、其實愚不可及的蠢人,例如馬英九和龍應臺;我一直覺得他們的智商應該和穩定的天才Trump相似。
昨天我在復習Nuremberg Trial的時候,注意到被告群除了兩三個納粹黨的老創始人(後來並沒有獲得重用,只當到省級的職位)以外,其他那些國家級的幹部平均智商在130左右,有幾個達到140級別,例如我最近介紹過的經濟部長Schacht是143,Goring和Donitz都是138。德國能夠以寡敵眾,硬是撐了六年才戰敗,領導階級在組織、戰術和作業上的優秀能力是主因。民選體制自然是蠢人當道,這是英美一直強力對外推銷的真正原因,尤其專注在競爭對手上;其實如果是好東西,禁運還來不及,怎麽會花大工夫和對手分享?英國的秘密在於政務官可以極蠢,實際負責執行的事務官(即Home Civil Service,例如《Yes,Minister》裏面的Permanent Secretary Sir Humphrey Appleby)卻都是名校出身的職業官僚,這保證了作業上的效率、理性和連續性。但過去30年的民主化宣傳,把自己也忽悠了,這些事務官的權力被不斷削弱;本月Civil Service的總長Cabinet Secretary Sir Mark Sedwill被迫辭職,確保了Johnson政權可以自由追求無協議脫歐的自我毀滅政策,是明顯的例子。
\section*{【美國】【工業】波音衰敗之源}
\subsection*{2021-08-05 17:24}

這裏我的建議,還是兩年前就説過的,公事公辦,以最嚴格的客觀標準,來檢討737 Max氣動不穩的安全性和飛控軟件的可靠性,不要試圖政治化。當然,這並不保證Biden當局會正面回饋,對C919的監管也維持專業態度。但中國買什麽、用什麽飛機另有國營公司來把關,反之國外的航空公司也不會因爲民航局一家釘子戶而永遠拒絕737 Max,C919的市場前景最終還是取決於商飛自己。如果美方真想要打壓,依照華爲的往例,直接從供應商角度下手遠遠更爲高效,拿適飛認證來做文章徒然進一步摧毀FAA僅存的一點公信力。從大戰略來看,中方對應美國的原則方針,也應該以科學、專業、理性的態度來和政治、權謀、私心做對比,這是最簡單、有效的手段。
\subsection*{2019-12-08 14:55}

是股市,但不只是股市。 
我在《八方論壇》的節目中已經評論過,這次又是一個典型的金融資產泡沫,但是比較類似於2000年那次,而不是2008年的大災難,主要的差別在於銀行業(至少是大銀行)學乖了,不但沒有加入狂歡盛宴,而且一見泡沫吹得太大,就提早收回現金入手,準備過冬。大銀行不願意繼續借錢給影子銀行,正是最近Repo Rate不斷上升的近因,遠因則是政府發債和美聯儲削減QE,使美國金融業的現金流量捉襟見肘。 
美國經濟周期内的短期榮衰,有三個決定性的支柱:金融業負責放貸,企業界負責投資,消費者則必須願意花錢;這次的問題主要出在企業上。因爲美聯儲放水太多太久,利率長期低迷,美國的企業普遍發債借貸(美國的債市發達,借錢並不一定要向銀行借),或者用來購并競爭對手,或者直接回購自己的股票,這兩個行爲同樣都會人爲地把股價推高。至於企業主管爲什麽會想要人爲地推高股價,剛好就是這篇正文要論證的。 
觀察現在的美國企業界,最重要的兩個指標是負債對收入的比例(Debt to Income Ratio,DTI Ratio),以及股價對收益的比例(Price to Earnings Ratio,P/E Ratio;Earnings就是Net Income,如果只説Income,通常指的是Gross Income,也就是稅前的總收入),它們都在2016年之後,突破合理的區間,直綫上升到泡沫級別。這就好像是山坡上一直下雪,雪崩是必然的,但美聯儲拼命在避免任何震動,我們也就無法預期什麽時候雪崩會真正發生。 
本周的失業率數字被Trump吹噓成50年來最佳,其實美國中產階級的工作早已停滯甚至萎縮,現在增加的就業都是因應消費的低級職位。但是這仍然代表著消費堅挺,而在美國經濟中,消費向來是主導;雖然長期來看,消費者的負債率也會向泡沫級別邁進,在短期卻給了美聯儲一個喘息的空間,可以針對金融界的流動性(Liquidity)問題直接對影子銀行放水,利率反而不一定要調降了。 
順便提一個題外話,“泡沫”/“Bubble”這個字是1720年南海泡沫事件(South Sea Bubble)期間被發明的,但是原本不是指整體經濟,而是針對個別公司;換句話說,它的用法原本是“Bubble Company”(“泡沫公司”)而不是“Bubble Economy”(“泡沫經濟”)。
\section*{【基础科研】回答丘成桐教授}
\subsection*{2021-07-25 15:30}

Weinberg是有真正成就的科學家,和那些純靠水超弦論文來發跡的騙子相比,有本質上的差別。他和詐騙集團搞到一塊兒,與其説是同謀,更像是受害者。可惜他始終沒有足夠決斷來做徹底的切割。
如果SSC沒有被砍掉,只不過是把LHC的(負面;亦即除了1960年代就已被預測的Higgs粒子之外,什麽東西都沒發現)結果早個十幾年發表,完全不影響高能物理已經被困在死巷子尾端拿頭撞墻的事實,既然近半個世紀理論進展總和等於零不可能改變,所有論文都是垃圾也一樣是必然的。我一輩子不願意水垃圾文章,所以最終大概還是會離開。至於Weinberg團隊,Postdoc原本就是短期職務,和後來是否轉行沒有因果關係。
\subsection*{2016-09-04 00:00}
LHC的结果基本就是TeV尺度上什么都没有,再提升7倍的能阶也没有用,因为Naturalness已知被打破了,被打破到1/100或是1/700无关紧要,背后必然有一个未知的机制在作用。既然已经确定有这么一个机制,那么它很可能可以作用到1/10000000000。连这个机制是什么都不知道,就要拿1000亿美元来赌它在1/700的精度上现形,是很不划算的,所以才会找上中国这个冤大头。

《观察者网》邀我写一篇稿,对中国高能物理的未来发展做建设性的建议。我已经想到的一点是,以往高能物理理论界有建议能打破naturalness的机制的论文,一直都被主流嗤之以鼻,往往期刊连审都不审。这是因为他们正在要钱建Tevatron、SSC和LHC,如果被决策者发现有可能除了标准模型之外什么都找不到,那么资金来源就会有危险。现在LHC的实验结果果然如此,中国高能物理理论界应该有系统地去挖坟,然后择优深入研究。还好高能物理理论自30年前开始,就是第一个在互联网上有了Preprint Depository的学科,所以这应该不难做才对。\subsection*{2016-09-01 00:00}
是的。LHC造价90亿美元,每年再加10几亿运作费,总价至今约200亿,比原估价高三倍多。而且它沿用原本现成的洞和基础设施。如果要从头挖,SSC挖了不到1/3,就花了20亿,那么岂不是要再加上60亿=260亿。这还不包括上面提到的基础设施,如水、电、路、房子等等。对撞机用的电路等设施,比军事标准还高,看来相同的东西,如电綫,都必须专门制造,比一般民用贵得太多了。

新对撞机比LHC大了四倍,能阶提升了7倍多,所有土木工程必须从头做起,怎么可能用50亿美元就解决?1000亿美元才是合理的估价。

丘和王贻芳故意低估价码好几倍,并且分段建造,这都是以往忽悠欧美政府的手段。钱一旦投下去,他们自然会要求追加几倍,要半途而废就难得多了。到时谁还记得他们原本的承诺?

正因为他们一开始就这样撒谎,存心不良,这个计划更不可能有好结果。\section*{【战略】【美国】当代美国战略局势与策略}
\subsection*{2021-07-09 00:42}

我之所以會想要寫文討論美國70年代的財經金融歷史,正是因爲與當代有很好的類比。
我在《美元的金融霸權》中討論的美元先放再收的熱機循環,就是從Nixon放棄黃金錨定開始的:70年代大放了一陣,逼著歐日跟著拼命印錢,引發全球性通脹,到了80年代轉爲收割,這才給了Reagan赤字花銷的底氣,可以同時刺激經濟並且進行軍備競賽。到了1985年,外來資金吸乾了,循環歸零重啓,反過來通過Plaza Accord强迫其他貨幣停發並升值,不過這波通脹靠產業外包降低生產成本而大部抵消,只有日本放任資產價格冲上天,然後到了90年代又收割了一次,幫助Clinton輕鬆享受經濟繁榮。
現在美聯儲的策略,基本是照抄;不過70、80年代,其他工業國家在冷戰的軍事戰略考慮下,不得不接受搜刮,現在政略環境變了,只有Anglo-Saxon集團和日韓必須任由美國玩吸星大法,中俄都不必當冤大頭,只要能爭取到歐盟,更有可能讓美國走火入魔。
\subsection*{2021-07-06 19:46}

如果真要在21世紀保衛自由、防備暴政,步槍遠遠不夠,至少必須有防空和反坦導彈。美國早在80年代支援阿富汗游擊隊的時候,就已身體力行,實踐了這個道理。即使在機關槍發明之前,步馬炮兵種協同也已經有400多年的老歷史。在美國獨立戰爭期間,所謂Minuteman民兵對英軍的殺傷,很可能還不如蚊子,那些故事純粹是募兵宣傳罷了。
有興趣的讀者可以在這本書《The Second: Race and Guns in a Fatally Unequal America》中找到更多細節:
https://www.amazon.com/Second-Race-Fatally-Unequal-America/dp/1635574250
\subsection*{2021-07-05 10:11}
簡單的答案是80年代冷戰升溫當然有利財閥;複雜的答案則要從二戰後談起。美國經過50年代全面主導西方工業化世界,生活水準大幅上升,生活形態也開始急劇轉變,城市化(嚴格來説,是城郊化,因爲鄉下人口固然向都會轉移,城區本身卻因爲高速公路直通市中心,方便長途通勤上班,於是中產階級向郊區遷移,尋求更舒適寬敞的居住環境)加速,大學教育開始普及。到了60年代,溫室中長成的新一代知識青年,觀察到美國社會種種不合理、不公平之處,形成叛逆風潮;然而上一代的理性傳統已經弱化。有異於2、30年代的工運有效幫助了羅斯福推動社會政策改革(所以西方媒體把Biden和羅斯福相比,可笑至極;姑且不論個人修養和理念上的差異,光是時代背景就完全不同,在當前被徹底愚化過的公共環境下,羅斯福復生也不可能搞出什麽名堂來),這些學潮的訴求空洞幼稚、不切實際(這是邏輯上的必然,所以從原則上就可以反對沒有工作經驗的年輕學生插足政治和社會議題;中共因爲歷史關聯和文革需要,大幅美化了五四運動,這直接導致了後來“再加十”的事件),很簡單就被既得利益階級搞斗轉星移,反而成爲後來社會急速腐化的種子(包括同一時期的其他改革訴求,例如消費者運動被扭曲成爲司法權無限擴張的起點;參見兩個月前在留言欄的討論)。70年代的石油危機,經濟上帶來長期滯漲;越戰失利,更是完全摧毀了財政秩序和社會共識。美國沒有選擇,必須做出戰略收縮;而財閥也同時出手,組織出新一代的代言傀儡,準備從思想層面徹底推翻羅斯福的政治和社會遺產。由於當時的學術界受學運一代影響,對財閥非常敵視,所以除了以芝加哥大學和耶魯大學為嵌入點、逐步污染蠶食之外,還必須通過主流媒體鼓勵越來越像例行公事的街頭運動、推動形式主義至上的白左思潮,將敵人引向瘋狂。到了1980年,Reagan成爲第一個成功上臺的現代財閥代理人,還未上任就先得到兩個天上掉下來的大禮物:除了蘇聯出兵阿富汗,重蹈美國越戰的覆轍,近十年前Nixon將美元與黃金脫鈎,强迫出口通脹,也已經被其他工業國家在無奈下接受(亦即選擇增發本國貨幣,避免大幅升值,而不是聯合起來推翻美元的地位),反而進一步加强了美元的國際影響力,賦予美國無限增長赤字的底氣。既然可以憑藉赤字和借貸來暫時維持、甚至加速提升全民生活水準,那麽財閥就自然獲得產業外包、以追逐更高利潤的自由。所以50年前Nixon打破Bretton Woods System,其實是後來這一波貿易全球化的隱性起因之一(英語和華語學術界似乎都還沒有認識到這一點;最近我在考慮寫一篇財政金融方面的文章,因此對這些事件做了進一步的思考)。至於Reagan治下冷戰升溫,對財閥有兩層意義:首先,軍工企業在越戰後的節衣縮食終於可以反轉;其次,這原本就是Reagan能上臺(Reagan本人當然早有成見,但他能脫穎而出,是財閥精心選擇的結果,所以有個堅決反共的總統,不是隨機偶然)的主要政見之一(另一項是絕對自由主義經濟)。這裏的時代背景,在於財閥主導的現代美國社會左右撕裂,還處於初始階段:一方面保守派選民(亦即鄉下/低教育程度/基督教徒/白人)仍佔絕對多數,另一方面《Fox News》還不存在,這些白人的政見還未激化,在北方工業州甚至因爲工運的歷史傳承,而普遍支持社會政策上略為偏左的民主黨。日後右翼宣傳機構又努力了40年,才有現在近40 \% 選民質疑大選結果、支持衝擊國會的現象;這40年的説辭,主要包括1)外交强硬仇外,2)經濟上解除政府監管,3)挑撥種族對立,4)宗教原教旨主義(例如墮胎和反同),5)文化上抵制白左(例如反氣候科學,也包括反控槍;不過美國憲法第二修正案的擁槍權,其實從200多年前至今,始終沒有明説的真正最重要目的,是維持“種族秩序”,所以這一點也隱含前面所提的種族對立)。在1980年,還只有前兩項有效,所以冷戰升溫在當時是財閥短期奪權、長期洗腦的必要手段。\subsection*{2021-04-10 22:02}

我的博文極度著重於事實與邏輯的分析,早年還在為複雜的現實世界體系做介紹、為讀者打根基的時候,談的診斷最多、預後比較少、處方更是常常“left as exercise"。你的這個分析,大致沒有錯;但是美國的衰頹,正在於他們已經無心選擇最優解,即使明白事情的道理,也無力好好執行。
現在美國外交政策,基本由國内的政治考慮來決定。過去十幾年的仇中洗腦,副作用就是自我設限,不再有理性反應的自由。美國面對臺海戰爭打也不是、不打也不是的窘態,内部的建制派智囊説來説去,還是想要靠洗腦“盟友”來忽悠他們去做反中鬥爭的炮灰。中國的反制,必須是在徹底理解這個態勢的前提下,制定宣傳戰的合理戰略和戰術。外交部在過去一年,終於挺直腰桿、不再當縮頭烏龜,固然是很大的進步,但是不分輕重、一味强硬,而且自説自話,反而落入美國人的圈套之中。這也就是我一連寫了好幾篇博文來下處方的動機。
\subsection*{2015-06-25 00:00}
我说"好玩"是指这本身并不是战略任务,只是为达成战略任务的手段中的一个选项。在这里留言的多是熟人,所以用词轻松些。

在核子时代要对核大国争取领土、领海是不可能的(除非是从内部颠覆),所以中共不能指望拿下关岛。打日本只有在师出有名、能压低规模的前提下才做得到;而要压低规模、避免美国全面介入,就必须在国力上全面超越,所以不可能是对外用兵的第一步。

我知道要先打日本来折服台湾,这是很多非军事专业的深蓝人士一厢情愿的梦想,但是它经不起逻辑的考验。台湾的军力比日本弱得多;定义上是内战而不是集团对抗,不会引起外敌联合多面作战;最重要的是台湾海峡只有150公里宽,全岛都在便宜的火箭弹射程之内,可以轻易进行饱和打击;而陆基航空兵不用空中加油就可以长期留滞于整个战区,更是得以形成无法抗拒的数量优势;以上种种都使台海战争完全一面倒,对日本就不同了。而且一旦占领台湾,潜艇一出基地就直接进入太平洋深水,不受琉球诸海道的限制;其次是远程警戒雷达向东前进400公里,不再有陆地遮拦,所有关岛的美军军机一起飞就会被锁定;空军基地倒不是首要的,但是也比你想像的重要些,因为共军的真正对手不是日军而是美军,所以距离是针对关岛算的,不是日本本岛。

\subsection*{2015-05-28 00:00}
2. 要打全面核战,没人打得过美国,不过美国人也不愿冒损失一两个城市的危险。中国和英、法一样,一直奉行Minimal Deterrence,也就是200-300个弹头,能确定回击几发核弹就好了。现在中国开始站到世界的前列,Minimal Deterrence明显不够,必须慢慢向美俄的水平靠拢,可是又不想减缓美俄裁减核武的步骤,所以很难办。改进战略潜艇是必要的,也是最佳的手段,但是中方在这方面技术很落后,只怕没有七八年做不出096级。094级目前勉强够用。

3. 日本人愚不可及。明明德国人已经示范了一条明路,还当上了欧洲盟主。日本人原本可以当中国的副手,世界的老二,现在若没有被中共拿来祭刀就算好的了。台湾人也是这样;原本可以成为新霸主的科技核心,世界的新硅谷,现在没等人家来打,就先把全岛搞烂了。或许蠢气相投才是台湾人崇日的原因。

4. 新技术用在发电上比较容易,用在动车上很难有汽油和柴油的能量密度。我们现在谈的是15-20年后的事,这个时段不可能有突破性的新能源技术来取代石油。目前全球70多亿人中,还有40多亿人的经济开发程度在中国之后,这些人未来都会想买车,所以石油的重要性长期来说只有越来越高。\section*{【金融】三谈股市}
\subsection*{2021-06-28 15:48}

1.因爲他們故意對外宣佈“軟銀做了投資”,導致市場的韭菜們把風險評估值下調,於是所有的金融資產價值自然因而上升,尤其内含杠桿越高的衍生品,價值升得越多,包括Convertible Bonds。當然,Convertible不是杠桿最高的金融工具,但因爲他們計劃立刻脫手兌現,必須考慮流通性,所以Convertible是最佳平衡。
2.這種轉讓,不像股市那樣一切都標準化,買家賣家必須簽訂契約,契約一般有標準版,但雙方(尤其是買家)可以要求添加條文。《Economist》的意思是這次的買家很謹慎,訂下了若干時間内不能破產的前提;這不一定準確,如果是真的,代表著軟銀的這批人也沒有意識到Wirecard的實際財務有多糟糕。換句話説,螳螂捕蟬、黃雀在後,他們被Wirecard的老闆作假帳騙了。
20年前,我和Deutsche Bank打交道的時候,就注意到那些美籍高管代表公司做交易,如果看到特別划算的生意,會先拖時間,然後再設法以個人身份和資金參與。很多人都有類似的經驗,所以我以前反復説過,DB的紀律差,名揚華爾街。
\subsection*{2015-09-20 00:00}
这种资本的贪婪是普世问题,美国在2007-2008年出现的危机中被偷走的国民财富比这次大陆股灾大至少2倍(5兆人民币相比后者的上限2.3兆),事后没有一个人被关,而中共已经很认真地抓了几十个。制度孰优孰劣不是很明显吗?

处身事中的傻子、呆子、疯子会比隔岸冷眼理性的分析可靠?你从出生前就处在重力场中,能写得出广义相对论的重力方程式吗?

我吃过台湾米,所以想为台湾尽心说些实话。说瞎话害死贫苦民眾的人才是国贼,很不幸的,台湾的国贼已经占了多数。到了这个地步,你们还在担心说实话是"唱衰"?你们花了20年把台湾"做衰"了才是真正的问题吧?

不延续自己开始的讨论,只能顾左右而扯逻辑上无关的东西,而且还只是网上的意见,连事实都称不上。这个部落格仅限讨论事实与逻辑,两者皆无是浪费大家的时间,依规矩直接删除。\subsection*{2015-09-20 00:00}
你若是有兴趣,可以找专门书籍来研究,这里我只简单举个例子。

目前最重要的美国国债是10年期的Notes(原本是30年期的Bonds,后来停发了三年,市场就把注意力转移到Notes去了)。假设你买面额 \$ 100,000的Notes,它有固定的2 \% 的年利率(这些利息叫做Coupon),那么你每年可以收 \$ 2000的现金利息,十年到期后再收一次"本金" \$ 100,000。但是你买的实际价钱并不是 \$ 100,000,而是由一个公开大拍卖决定的。最近的一次拍卖得到的价钱是 \$ 98,812.50,所以实际的报酬利率比2 \% 高一些,相当于2.13 \% ,这就是所谓的Yield。美国新闻对国债的报价都是直接採用Yield。

很明显的,市面上的利息涨了之后,国债的买家也会要求较高的利息,但是Coupon是不变的,所以结果就是拍卖价降低了,使Yield上升。如果你是上一季花高价买的,那对不起,你在一季之间已经损失了那个价差,这可以是很大的数目。例如你在市场利率是2 \% 时买,那么公平价就是面值 \$ 100,000;等Fed加息后,Yield跳到2.13 \% ,那么同样的国债只值 \$ 98,812.50了,你转眼就损失了 \$ 1,187.50,亦即半年多的利息。\section*{【歷史】【戰略】希特勒的戰略選擇}
\subsection*{2021-06-20 15:04}

空中支援力量,從Guderian在1930年代中晚期的實驗開始,就是德軍戰術的重要成分之一。所謂裝甲集中使用,其實是外行媒體以訛傳訛的結果,完全只相對法軍分散支援步兵的裝甲兵準則(Armor Doctrine)來談,無視英軍更極端的全裝甲旅級編制。德軍總結西班牙内戰經驗後,在二戰早期的裝甲師(Panzer Division)組成,摩步、坦、炮、支的團級單位比例是1:2:1:1,已經是介於英法之間的折中;到了征服法國之後,重新檢討,發現仍然不是最佳平衡,所以在1941年開始征蘇之前,比例已經改爲2:1:1:1。有人以爲是Hitler做形象工程,把舊裝甲師的坦克一份為二,用以增加名義上的裝甲師數目,其實專業的軍事歷史研究人員,一般認爲德軍後期的編制效率更高。美軍(4th Armored)的比例是1:1:1:2,也算是合理的。
至於空中支援,德軍的準則叫做Close Air Support,CAS;美軍在80年代計劃的空地一體戰,其實正是二戰德軍準則的延續。偏重CAS的問題在於飛機航程過短、載重過小;所以德國空軍打英倫空戰力不從心,最近10年美軍在西太平洋的窘態只是歷史的重演。至於Space Force,目前的科技水準還只相當於一戰早期,亦即剛要從偵察/反偵察向爭奪空優邁進,真正的打擊力量,重點依舊在於不真正進入軌道的傳統彈道導彈和新興的高超音速技術。
\subsection*{2020-08-07 06:57}

大哉問。長期來看,新通訊科技、新產業鏈模式和新增財富的確是推進國際間政經整合的强大動力,不過...
首先這並不保證會是全球化,基於主權意識的地緣政略考慮,使得大型區域板塊更加可行。其次,整合不必是完全或絕對的。以汽車產業爲例,20世紀末有幾個并購的嘗試,並不太成功;本世紀比較流行的是中等程度的聯盟,例如Renault-Nissan-Mitsubishi。航空公司之間也自然形成了幾個大聯盟,聯係更鬆散些。
目前歐盟把握英國脫歐的時機,把重點放在内部整合,這會形成最緊密但也是最弱小的主要集團;美國企圖組建反中聯盟,這天然必須是中等緊密的主僕關係;中國則是尋求相對單純的經貿合作,所以可以在第三世界廣汎搜羅對象,然後以鄉村包圍城市。
未來20年國際關係的主軸是霸權交替,也就是親中集團逐步分解、吸收、壓倒親美集團的過程,歐盟基本保持中立。中國確立領導地位之後,我希望它能以理性、互助、公平的原則,持續推動全球的進一步整合,但中國必須先解決自身内部的許多問題(例如我多次討論的學術腐敗和教育公平問題),所以目前言之過早。
\subsection*{2020-07-19 14:16}

是的,所以演化利基先占先得,後來者即使有功能優勢通常也必須等到原本的霸主自行消亡才能取而代之。最明顯的,是大型陸上草食動物之爭,一開始是兩栖類(Amphibian)為霸主,然後爬蟲類(Reptile,包括Anapsid和Pelycosaur)興起,接著先後是Dinocephalian和Dicynodont(哺乳類的兩種遠房親戚),其後是恐龍(Dinosaur),最終才是哺乳類(Mammal)。這裏每一次的霸權交替,都是Mass Extinction Event(集群滅絕事件)的後果,例如爬蟲類取代兩栖類,靠的是演化出蛋殼,所以不必囘水裏下蛋;這乍看之下是壓倒性的優勢,但實際上還是必須等到Carboniferous Rainforest Collapse(CRC,石炭紀雨林崩潰事件,3.05億年前)才完成。結束Pelycosaur的是Olson's Extinction(2.73億年前),終結Dinocephalian的是Capitanian Mass Extinction(2.6億年前),Anapsid一直拖到Permian-Triassic Extinction Event(二叠紀-三叠紀滅絕事件,2.514億年前)才完結,恐龍則到Carnian Pluvial Event(CPE,2.3億年前,我曾撰文介紹)才取代了Dicynodont,至於6600萬年前的Cretaceous-Paleogene Extinction Event(白堊紀-古近紀滅絕事件)更是大家耳熟能詳的事。
所以霸權交替,有新興能者固然重要,最要緊的前提還是既有霸主的衰亡。
\subsection*{2020-07-16 17:07}

美軍在1945年春,歐戰勝利在望之際,已經軍心浮動,上至總統、下至列兵,沒有人願意為完全無疑義的結果發生更多的傷亡(這個現象在《Band of Brothers》裏有觸及)。一旦復員開始,基本不可能再重新全面徵兵。後來韓戰無預期地打成大規模戰爭,也只能勉强重建10個師,就這樣還促成杜魯門提早退休。
另外一個很多人不明白的事實,是當時歐美對中國的鄙視極深,尤其在1944年日本駐華占領軍出乎所有人意料之外地傾巢而出(Operation Ichi-Go),孤注一擲想要優先打垮國軍,結果得到重大進展,讓蔣中正臉上非常難看。當然這裏有國民黨自己腐化的因素,但當年即使在事後,美國人依舊沒有看出日軍絕望到放棄華北鄉下來打通平漢/粵漢綫(這背後的戰略考慮是臺海周圍的海運路綫岌岌可危,所以日本陸軍希望建立從廣州到朝鮮的陸路交通),才是國軍戰事不利的真正主因。
再加上中國根本不算一個工業國,所以同盟國沒人把中國當一回事,唯一的例外是小羅斯福,在他的堅持之下,“美英蘇”變成“中美英蘇”。他死後馬歇爾繼承遺志,中國才能成爲常任理事國。這並不是他們有什麽非理性的親華心態,也不是中國有什麽戰略價值,而是國民黨被視爲一隻忠實的狗,等同於讓美國的票數和影響力加倍。當然英國人完全心知肚明,作爲交換,也把法國拉進安理會,這才有了現在的五常。
所以指望美國在1947-1949年直接出兵干涉國共内戰,是只有事後諸葛亮才想得出來的事。
\subsection*{2020-07-16 11:16}

我以前也喜歡去想像許多“What If”的脚本,看看有沒有什麽偶然事件可以造成不同的結果,但在二戰這件事上,越是詳細地瞭解了歷史進程,越明白除非希特勒或裕仁能有100 \% 的後見之明,能做出嚴重違反常理的決定,否則他們的失敗是一開戰就已注定的。這是因爲兩邊的資源和實力太過懸殊,美蘇可以一連做出100個低效的決定,而只是延緩勝利一年半載;相對的,德日在100個決定裏只要有10個不是最優,就會面臨極爲惡劣的後果,然後後世讀者自然想要挑出這10個事件來反復討論。它們的確是真實版本的歷史裏,德日失敗的過程和方式,但如果他們沒有犯這些錯,一樣贏不了戰爭,只不過是多拖一段時間,期間必須做更多的決定,而這些決定不可能全是最優解。
以色列對戰略危險的認知是世界第一的;中國在一年前還在談中美夫妻論,完全不在同一個層次。
\subsection*{2020-07-14 23:59}

德國財政混亂的主要後果,還是在於欠缺外匯;對内納粹早就全面管制經濟,工資和商品價格都是公定,就像中共在改革開放前一樣,所以赤字不是緊急的問題。
“一國的戰略綱領豈能只靠大規模的搶劫來解決?”這是中國文化的看法,西方剛好相反,從大航海到殖民時代到20世紀三次全球戰爭,强權的興起基本就是靠大規模的搶劫,内部的改革建設只不過是用來支持對外搶劫罷了。我在正文解釋了,希特勒其實是英國式殖民主義的忠實信徒,所以他準備依靠搶劫來解決經濟財政問題是很自然的。現在的美國不也是在對日本和蘇東吸血之後食髓知味,想要對中國和歐盟搞同樣的掠奪肢解嗎?
你如果回去看二戰期間的第一手資料,就會發現在德軍決策階層做討論的過程中,希特勒居然還是唯一在乎經濟、資源和後勤問題的人。那些名將只會打仗,在經濟上純屬文盲。例如在戰後,Halder和Guderian還寫書說Operation Barbarossa沒有集中所有力量在中路進攻莫斯科,是整個計劃失敗的主因,這真的是顛倒是非。希特勒發動對蘇戰爭的戰略目標就是南路的烏克蘭和高加索地區,打下莫斯科一點用都沒有,拿破侖的前車之鑒還歷歷在目。
實際上,Halder也做了手脚,把希特勒原本計劃集中在Army Group South的兵力偷偷轉移了一個Panzer Army給中路,所以戰事前期南路在名將Rundstedt的指揮下依舊進展最慢(另一個原因是他面對的Southwestern Front西南方面軍司令Kirponos是所有蘇軍前綫指揮官之中最能幹的)。還好史達林很理性地預期德軍會以南路爲主攻重點,所以一開始在中路準備不周,必須緊急把戰略預備隊全部投入,然後Zhukov跑到Kiev去監督Kirponos,强迫後者從機動防禦轉爲全面反攻(有可能是揣摩史達林的旨意),結果正如Kirponos事先警告的,德軍得以中央突破,把西南方面軍撕成兩半,這時希特勒終於注意到Halder的花樣,緊盯著他(希特勒在此前對手下授權很放任)把裝甲兵團送囘南路,剛好形成鉗形包圍,70萬蘇軍中60多萬被俘虜,Kirponos戰死,德軍歪打正着,反而比原本的計劃還要有效,這才把戰事拖到1942年決戰Stalingrad。
但是蘇聯在1939年到1941年兩年期間是德國的原油主供應商,他們當然知道後者的窘境。Timoshenko元帥在一開戰(當時是Stavka長官,也就是參謀總長)就强調,蘇軍的總戰略在於掐死德國的石油供應。後來Stalingrad之所以成爲兩軍會戰的焦點,也在於石油;這是因爲蘇聯的高加索原油必須經由Volga River運輸往後方,而Stalingrad就位於Volga River最靠德軍方向的突出部,因此德國要切斷蘇聯的原油供應,必須拿下Stalingrad。有英美的“歷史學家”宣稱希特勒爲了那個城市的名字而落入陷阱,這是胡扯;希特勒的戰略定力有問題,但是整體層次還是遠超這些“專家”所能想象的。
\subsection*{2020-07-14 21:45}

幾年前有歷史學家翻出一些沙俄的老文件,宣稱史達林原本受雇於沙俄秘密警察,奉命滲透進入共產黨臥底的。這或許是他掌權後必須大開殺戒的真正原因,不過俄國人當然沒有興趣配合進一步證實這個説法,所以目前還是只能存疑。這有著已證實的類似歷史實例:Nixon之所以會鼓勵手下去監視民主黨競選總部,最終引發水門案,也是擔心後者挖出他的黑歷史,所以睡不安穩、陷入偏執和受害幻想狀態。這裏他擔心曝光的是,1968年他選第一任期間,曾經和南越政府有密約,故意製造一批越戰的壞消息,以便打擊當時民主黨政府的民意支持。
不論如何,史達林只知權謀、而沒有理想,是很明顯的事實。不過二戰前他的大戰略是絕對正確的:他早早就判斷不論誰掌權,英法德的真正目標是消滅蘇聯,所以必須犧牲一切(包括對烏克蘭做極限壓榨)發展工業,尤其是軍工。1939年希特勒撕毀慕尼黑協定,他看出有機會讓德國優先打擊英法,所以和希特勒做出暫時妥協,也是絕對理性的最優選擇。至於1941年他忽略許多德國即將入侵的情報,也情有可原,因爲他是高估了希特勒,沒有想到後者會在資源實力如此欠缺的前提下,還敢孤注一擲;這是因爲史達林不像希特勒是個賭徒,所以心理上很難設身處地精確估算後者的意圖。此外,西方歷史記載往往忽視一些細節,尤其是當時也有一大堆情報指向相反的方向,例如德軍在四月還出兵南斯拉夫、最合適對蘇開戰的五月平安度過、而且德國根本就沒有動員生產冬衣。史達林這種喜歡微觀管理的操控狂,怎麽能想象希特勒樂觀到以爲兩三個月就必然會勝利的程度?事後德軍也的確無力在冬天到來之前完成戰略任務,並且因爲缺乏耐寒衣物而付出慘痛的代價;所以這件事其實是聰明人也很難預料瘋狂的傻子會做出什麽決定,有點像我們無法以理性來推斷Trump、民進黨的發言和行動細節一樣。
\subsection*{2020-07-14 14:24}

必須對大戰略形勢有精準正確的判斷,才能打破慣例常規,把握時機,贏得最優結果。現在中國還有半年Trump任期,可以自由爭取歐盟,而德、法、歐盟都有明顯的正面回饋跡象;中方必須放棄矜持保守,以大幅開發金融服務業為誘餌,換取中歐經貿的全面深入挂鈎。
半導體設計的進步很可觀,但是光刻機被卡住,生產方面是一大隱憂。這只有Biden當選後選擇中美緩和,或者中歐建立極爲緊密的外交經貿關係,才有可能真正解決。中國要100 \% 自建商用極紫外光光刻機,是10年這個數量級的工程,一般大陸網民自嗨本土光刻機的進步,那都是無知之輩輕信公關忽悠的後果,對現實狀況毫無助益,希望這種希特勒式的盲目樂觀態度,不影響官方決策單位。
\section*{【科研】流行病的起源(上)}
\subsection*{2021-06-13 07:08}

你來問我對某篇文章的意見,卻不提供鏈接,連標題或作者都不給,更別提遵守《讀者須知》的第4條規定了。如果連這樣都不拉黑,規則不必定了。
剛好我前天看到一篇《Vanity Fair》有關新冠的文章(參見The Lab-Leak Theory: Inside the Fight to Uncover COVID-19’s Origins | Vanity Fair)充滿民主黨文字打手的風格,有興趣的讀者可以去看看。那裏所提的“被警告不准調查”,是很典型的Fake News斷章取義伎倆,連完整的句子都不給,更別提來源鏈接;通常這樣的語法,標識的不是被攻擊的對象有問題,而是作者本身就在試圖誤導讀者。
這篇文章的宗旨是什麽呢?基本就是新一波民主黨版的武漢實驗室起源論,所以核心指控是無意間泄露,但這個作者又不願放棄美國媒體界既有的各式各樣自相矛盾的陰謀論,只好另外多加一層陰謀,亦即病毒是被武漢研究人員修改過的,所有已經出面指責人工合成陰謀論的的科學家(尤其《Lancet》被指名批判)都在撒謊,而撒謊的動機被暗示為專業内部自相保護。其實我這麽歸納,已經太擡舉那個作者,因爲她的文章並沒有自洽的邏輯,東說一點可疑、西説一點反對,但實際上根本不敢明説任何可以確實檢驗的論述主旨;也就是前面我所說的“文字打手的風格”。

有關那個“被警告不准調查”,文章作者故意語焉不詳,所以無法確認幕後運作的真相,不過我的猜測是那名國務院官員可能咨詢過美國國内專家,被告知所謂的“Gain of Function”是國際上的新研究潮流,最熱衷的剛好是美國的研究人員。如果去捅這個馬蜂窩,中方可以簡單反擊(他或許高估了中國外宣能力,至今外交部依舊只引用了網絡上的謠言,並沒有真正提起專業性的質疑)。
“Gain of Function”指的是用基因工程技術,人工合成新病毒變種。Fake News很高興地挖出2015年登在《Nature Medicine》的一篇論文(參見A SARS-like cluster of circulating bat coronaviruses shows potential for human emergence | Nature Medicine),討論如何將一種蝙蝠冠狀病毒的Spike Protein(SHC014-COV)轉移到老鼠版的SARS病毒之上;因爲石正麗也是作者之一,所以這篇論文成爲美國版陰謀論的明星證據。
然而Fake News不告訴讀者的是,這個研究其實完全由北卡大主導,拉了少數哈佛和瑞士的研究人員做的,武漢實驗室只提供SARS的樣本和咨詢,所以石正麗在15個作者中倒數第二,這正是生醫界論文實際上最末的排名(除了倒數一號作者是總指導之外,頭號作者是總執行,然後貢獻依排名遞減)。這樣的扭曲事實,實際上是把專業的科學從業人員趕到爭論的另一方,免費贈送盟友給微弱的中國外宣,是我做建議的背景參考之一。
\subsection*{2021-06-11 19:10}

這是自由媒體和互聯網的特性:越是離譜驚人的論述,越有市場,參見《大衆媒體的内建矛盾》。
Crackpots到處都有,包括不少哈佛教授;你應該自問,如果一個新消息很驚人,它必然根據全新的事實證據,那麽其他人對這個新證據的反應合不合理?一個簡單的例子:有人跟你說他能證明鬼魂真實存在,那麽你應該自問,這種推翻人類全部已知物理知識的新突破,拿10個諾獎也不爲過,爲什麽他還沒有名利雙收、退休養老?就算是遺珠,他爲什麽不去找物理學教授,而會是只和外行人談?這位鞠女士明明可以簡單得到在《自然》發頭條論文的榮耀,她和《歐洲時報》談個什麽勁兒?這種無關學術界自身利益的事,指望通常一盤散沙的學術人聯合封殺真相,是明顯違反人性的陰謀論;如果事關學術界的重大私利,而導致全體走偏,那個過程也必須是以數十年計,所以必然會不斷有人出面揭發,外人即使沒有專業知識,只要檢驗這些吹哨人所說的是否有邏輯一致性,也能對真相猜個八九不離十。
\subsection*{2020-04-22 04:58}

人類的自動化技術正在高速發展,而中國施政的核心目標正是產業升級,在這樣的背景下,中國的勞動力不會有嚴重的欠缺,畢竟純勞動力密集的產業可以外移到東南亞和非洲,而不是把工人帶進國内。
以中國人口基數之大,滿足中高級以及無法外包產業(如建築)的所需,並沒有不可跨越的困難。既然大批引進外勞有已知的嚴重負面後果,那麽這類決策必須在嚴謹深入完整的成本效益分析完全確定最佳方案之後才能采行,光憑半吊子經濟學和人口學理論來拍腦袋,是自殺性的行爲。
我對人口問題沒有太大的興趣,也沒有特別的專長。幾年前就曾經截斷討論,現在還是一樣,與其一知半解的做推測,不如請大家另尋專才。尤其和正文離題已遠,到此爲止,後續留言一律刪除。
\subsection*{2020-04-20 21:45}

Absence of evidence is not evidence of absence.沒有直接證據支持正反兩方,並不代表我們不能做理性的評論。堅持事實和邏輯,與在迷霧下估計真相,並沒有矛盾。我們只要堅持客觀態度,以邏輯為準繩,廣汎考慮相關事實,然後在做結論時,提醒讀者論證中沒有直接證據,而只是以間接證據和人情事理來做推測,那麽這個討論就是在客觀背景下,人力所能及的最佳結果,當然有其價值。
非法移民是普世問題。中國沒有確實可信的統計資料,那麽除非你提供了中方有能排除非法移民的特殊條件,我們就必須假設它和其它經濟發達國家類似。這裏的重點是,基於來自世界其他國家的間接證據,有責任提出反證的只是你這一方;你在己方完全沒有間接證據支持下,一味要求對方繼續提供更強的證據,是狡辯術的體現。
黑人犯罪率,同上。然而我很不喜歡一竿子打翻一船人;不同的非洲國家,有不同的民情文化,把幾十個國家、幾千個族群堆在一起討論,確實是一個僞命題,所以正確的説法,是“部分非洲國家”,有高犯罪率趨勢。至少Nigerian在國際詐騙集團中的强勢地位,是有目共睹的。反過來看,中國官方也把非洲視爲一個整體來制定政策,這顯然是脫離現實的不智行為。
\subsection*{2020-04-20 02:45}

是政治操作,但是美國掌握了話語權和規則權,耍無賴是傳統,光和他講理沒有用。
目前他們只是在發動民意、為事後找碴做鋪墊,届時醫療設備的供應早已成爲上古歷史。若想要獲取有利中方的結局,歐盟的態度是關鍵,尤其是德法。德國被英美滲透很深,已經有些勢力開始聲援美國,但是主流應該有足夠的理智不去站隊。如果由我來決策,就會立刻以WTO仲裁法庭被美國癱瘓為著力點,全力推動建設能限制駁回長臂管轄的國際管理機制,讓美國偷鷄不着蝕把米。Macron在Alstom案之後,已經有意向這方向努力,意大利、西班牙對中國的支援記憶猶新,德國對强化歐盟國際地位的建議沒有强力反對的理由,所以這是可以做得到的;可惜外交部連日常公關上的表現都太過稚嫩,這種大戰略上的未雨綢繆、連消帶打,只怕根本不在他們内部討論的範疇内。
\subsection*{2020-04-19 18:34}

印度那是一個網紅在胡鬧,美國卻是很認真的:法庭的門對所有的律師大開著,律師除了收錢之外就只想打知名度做免費廣告,法官則是普選出來的,選民中最熱情的正是現在游行示威要取消隔離的那群絕對利己主義者,那你想會不會有官司?
英美70年代的財閥奪權,始於思想改造,其中包含了一個新的社會契約:公民不再有任何責任,政府和社會反過來必須包容並滿足他們的所有私利、情緒和妄想。這種絕對自由主義應用在自己國内,就是右翼民粹的理論基礎(“Government is the problem!”);面向外國,則成爲白左支持美國對外顛覆的思想路綫(例如台灣、香港)。當然,新自由主義的實際目的,是消滅選民的責任感,將他們自私、自大的態度合理化和合法化,這樣財閥控制下的媒體才能有效驅趕牛群來衝擊政府和社會裏既有的公共利益結構(如福利和稅收)、壓倒理性知識精英的聲浪、製造新的政治正確和代罪羔羊,最終讓富人的特權和資本無限擴張而且不可扭轉。
英美成爲真正的資本天堂之後,對新興工業國的資本家也就有了極大的磁吸作用,這時新自由主義能鏟除他們賣國叛國的最後心理和環境障礙,進一步增進國際財閥集團的勢力。
\subsection*{2020-04-18 23:17}

理論上,股市應該反應6-9個月後的經濟狀態,但是實際上當然可以有人爲的操弄,尤其美聯儲有近乎無限的鑄幣權力,目前大家預期今年會有大約5萬億美元(相當於25 \% GDP)的注入,這都必須流入金融市場,而股市是其中流動性最高、相對於債券資產更實在的去處,所以吸收了大部分的新錢很自然。
未來三季,關於實體經濟的壞消息會不斷確認,財政和貨幣刺激的效應卻會遞減。請你稍安勿躁,等到年底再來復盤。而且這還只是第一階段;1990年日經指數從歷史高峰開始崩潰,經歷了三個斷層式下降才真正觸底:第一階段花了一年,下一個谷底在1992年,最低點則到2003年才完成。我認爲這一波美股崩盤,至少也要三年才會利空出盡。
\subsection*{2020-04-18 09:19}

有關那些假道歉的句法,倒不是英文有特別多的拐彎抹角,而是西歐語言(包括拉丁系和日耳曼系)先天時態(Tense)和變位(Conjugation)就複雜纍贅,在啓蒙時代(Age of Enlightenment)和工業革命之後,又纍積了幾百年的科學、工業和法律傳統,所以語法(Grammar)進一步邏輯化、規範化和複雜化。這容許叠床架屋的句子結構可以只包含正面字眼,但整個句子的意思卻是否定的。你舉的例子還算是簡單的,受過教育的人稍微一想就知道不但不是道歉,事實上是一個侮辱(亦即“你的過度敏感十分可悲”)。真正Obfuscatory的例子,可以在《Yes,minister》劇集裏經常看到;那是連以英文為母語的族群都覺得很好笑的。 
至於外交部這段時間的一連串戰術失誤,當然無須深責,畢竟全世界的外交人員原本職責只限於說空話,中國外交部被迫越俎代庖來護衛國家聲譽和利益,是在宣傳單位和媒體太不爭氣的獨特背景下不得已而爲。至於他們那些失誤,的確如你所說,來自沒有事先正確定位:國内憤青和國外民粹都不是國際宣傳的對象,要代表國家官方來發聲,目標聽衆應該是歐美的知識精英,所以在原則上要尊重理性和事實,堅持占據道德高地;在執行細節上,則必須從對對方的深刻瞭解出發,未雨綢繆、慎密籌劃、料敵機先、爭取先手。中國在國際輿論戰場上是客隊,光是抱怨裁判不公是沒有用的,跟著主隊玩骯髒手段更是落入他們的圈套,要在風度和技巧上都遠超對方才有些許勝算。
\subsection*{2020-04-17 10:28}

昨天中方把疑似新冠死亡的人數,加入了以往確診死亡數目,所以官方數字一下暴漲了一半。這原本是科學上的例行決定,英方自己只報醫院内的確診死亡,比中國的舊標準更脫離現實,但是《BBC》(參見https://www.youtube.com/watch?v=wfyNqt1OqWQ;《BBC》的編輯原本是些白左,不過過去十年保守黨連續執政,不斷用預算和職位來“整治”國有媒體,結果這群人被打怕了,從2015年左右開始,在政治議題上已經成了右翼的應聲蟲)報導下來,强調的只是上修的突然和幅度,到最後才對數字背後的意義草草敷衍了兩句,然後還是以懷疑中方誠信來做總結。我一面看一面搖頭;倒不是驚訝於《BBC》狗改不了吃屎,畢竟英國人在對應疫情上被中國這樣比下去,是真正的信仰危機,而我多次説過,只有非常聰明理性的人,才能在危機下接受事實而放棄信仰。 
這裏的問題,在於外交部原本可以輕鬆地用簡化精練的文法來發佈消息(亦即我在上文第一句所用的“疑似”和“確診”;最好是兩個數字一起給上幾天,然後再融合爲一),而不是浪費篇幅去詳列新增數字背後的諸般細節,如此自然能限制有敵意的外國媒體選擇性扭曲報導的空間。這是因爲中國外交部所面臨的國際輿論態勢,很類似美國嫌犯被警檢團隊質詢拷問的過程,任何隻字片語都會被拿來斷章取義,所以能小心就應該小心。在這個例子裏,英美媒體明顯正等著找中方數據不可靠的證據,一下子“重新定義”,可以預見會被拿來做文章,所以發佈的方式必須特別顧慮到舊有數字的權威性和有效性,不能授人以柄,助長舊數據被推翻取代的誤導解讀。我知道外交部沒有這方面的經驗和專長,但避免敵方有意誤讀扭曲是公關領域危機處理(Crisis Management)的日常考慮,中國有很多企業界的公關人才,爲什麽不借重一下呢? 
至於《經濟學人》,我以前仔細討論過,負責前半(一般新聞)的那套編輯,很可能是美國Deep State國際宣傳體系的一個分支,所有的文章都建基在“美國=好人、中俄=壞人”的基本假設上,完全不在乎客觀行爲標準。換句話説,他們仇中是日常操作,你所提的這篇文章只是在美國也陷入疫情泥淖之後,回歸常態,代表的是美國建制派對中國威脅自己思想理論、政經體制和國際霸權的深刻疑慮。這既含信仰問題,也是利益相關;和右翼民粹相比,他們的仇中心態其實層次更深、範圍更廣,所以兩年前貿易戰爆發,中方原本還指望財閥建制派出手遏制Trump,我就明確解釋那是妄想。現在Trump一路搞下來,至少幫助海内外有見識、有理性、有良知的中國人看清局勢、放棄幻想,所以不論國外輿論界如何抹黑,我對中國的前途反而更樂觀了。只希望他們在戰術細節上能再優化一些,一個堂堂大國連最基本的公關技術都不懂,實在讓人看得心急。
\subsection*{2020-04-15 01:02}

一方面是這些第三世界國家有對内宣傳“國家尊嚴”的需要(Nigeria之流則有額外的惡意和傲慢),另一方面西方傳媒向來喜歡拿這類事件搞雙標。英美那不用説了,北歐有很强的聖母心態,德法則討厭一帶一路對東南歐的吸引力,尤其在新冠疫情的國際援助上,中方的慷慨讓他們十分難看,反而更結了仇。
廣州的事,我只有一點簡單的消息,不能深入詳細分析,不過似乎是一開始要堅定執法,撞上Nigeria外交官就軟了,隨後國民對防疫執法無力做出自發的升級反應。如果事實主軸的確如此,那麽廣州政府是被對外籍人士跪得習慣的對外單位拖累了,所以我才猜測一帶一路的宗旨被執行單位扭曲,是這次事件的病根。有時我的邏輯跳得很快很遠,沒有在第一時間解釋清楚,或許昨天有些讀者奇怪爲什麽談著廣州,我一下就扯到一帶一路去了。
事後回應,外交部和宣傳單位的態度應該是强調有外國人拒絕檢測,一律立刻驅逐出境。這是因爲事件已經複雜化,如果國際輿論專注在房東和旅館拒收非人的角度,中方怎麽解釋都不好看,這時就應該全力把話題拉囘防疫,自然沒人能反駁。遇到英美記者,還可以順便揶揄他們,說中方是要提供免費立即的病毒檢測,據瞭解,貴國内部還有很多公民求之不得。
\subsection*{2020-04-13 09:20}

我說可以考慮把Nigeria排除出一帶一路,並不是因爲它是蕞爾小國,而是一帶一路已經搞成跪求外國參與的姿態,在實踐推行和國際輿論上產生了顯著的負面影響。如果能根據自己常説的“互利共贏”,對一味得寸進尺、只想占便宜的少數國家給予懲戒,反而能提升整體的效果和形象。
Nigeria的問題,絕不只是他們在中國的那少數僑民,而是他們的文化素質太過惡劣,根本沒有合作、互利、感恩的概念,不論是官方互動,還是經貿接觸,如果沒有讓另一方吃大虧,他們就不會滿意。像這樣的國家民族,和他們交往越深損失越大,非常適合拿來殺鷄儆猴,讓其他第三世界國家瞭解,一帶一路是給他們的優惠,不是中國欠他們什麽。
國臺辦一味綏靖和親的策略在台灣民意上適得其反。台灣至少是一家人,Nigeria憑什麽得到同等的待遇?
\subsection*{2020-04-11 23:20}

你對歐盟國家對立的理解,基本是正確的;這也是歐盟中短期内最大的内部問題。其實我在去年已經説過,2020年是七年一次的歐盟預算年,我預期他們會吵成一團,但應該可以矇混過關;下一次2027年才是内部矛盾的張力真正威脅到歐盟結構的節點。
新冠疫情惡化了這個結構性問題,提早凸顯南北的矛盾。德國、荷蘭和芬蘭不願意發新冠債券,是怕這樣明目張膽的財政轉移(Fiscal Transfer)開了先例,以後會是個無底洞,所以他們堅持只用上次希臘危機後建立的歐盟應急基金來一次性安撫意大利和西班牙兩國。然而南歐國家的經濟在單一貨幣之下,有你所描述的長期性弱點,這次的疫情危機應該足夠把他們推過臨界綫。德國是歐盟最大的受益者,非常希望維持南歐和東歐為自身經貿禁臠的現狀,這也是爲什麽德國對中國與其他歐盟國家之間的外交經濟行動始終極爲敵視的原因(法國考慮的則是政略)。然而德國和南歐國家之間這個針鋒相對的矛盾,已經被新冠尖銳化了,2020年的後半必然是一連串的交涉折衝;其後果目前還難以預料。可以確定的是,中國作爲南歐國家的另類資金來源,必然會是談判中的重要籌碼和議題。
\subsection*{2020-04-11 13:12}

一月裏說新冠是針對黃種人的生物武器,是大陸的一些網民;其實懂生物的人都知道,RNA病毒突變太快,非常不適合當武器。
歐美政客和一般人民(行内的專家自然有明白事理的)在一月的態度,是流行病屬於落後國家才有的問題。歷史上,他們先是經由對美洲和澳洲的土著做種族滅絕,賺了第一桶金,獲得進行工業革命的資本;一旦工業革命使資本持續纍積,就有能力和需要來繼續把亞非也瓜分為殖民地。二戰之後,美蘇聯手鼓動殖民地獨立,以削弱傳統西歐强權,但是遏制和剝削第三世界仍然是維持歐美生活水準和國際地位的必要。由於核子武器限制了大集團之間的直接軍事對抗,美國轉向金融和宣傳手段。宣傳不論對内還是對外,都必須吹噓本身的神聖性;而創造神話最簡單的辦法,就是譏笑窮國實際上是因爲資本纍積不足而導致的生活水準低、災害對應不到位,然後把自己花大錢解決問題,解釋成體制優勢。時間久了,不但是老百姓,連許多政商精英也相信了己方的忽悠;這種流行病動不了“先進國家”的迷思,就是美國神話的一部分。這時忽然發現中國決策和執行的效率高出一個數量級,那麽美方不但要擔心失去霸權,更有許多人因爲一輩子的迷信被事實抵觸了,面臨嚴重的認知失調(Cognitive Dissonance)。一般庸人沒有能力跨越信仰危機(Crisis of Faith),只能尋找藉口來自我安慰,不論這些藉口如何低級可笑;大部分美國人是庸人,所以就有最近的一連串鬧劇。
\subsection*{2020-04-11 05:17}

從上海傳到德國第一例的確是有紀錄的,也就是你所説的那位女士;但是那發生在一月後半,而C類新冠病毒是一月8日左右演化出來的,所以那個突變的分支點很可能發生在上海或者是那位女士的武漢家人。有一個比利時的病例,病毒基因在那位女士的上游就分叉出去,應該是獨立傳染過去的,但是除此之外,歐洲的第一批疫情爆發,可以都視爲那位女士的病毒版本的後代。
南華早報的文章,我覺得可信度很低,不但沒有證據,而且和許多已知事實抵觸。
至於人傳人先發生在廣東,並不是不可能,但是機率真不大(亦即在1 \% 以下),我們不應該妄做揣測。事實上,鍾南山院士和一些其他專家猜測新冠在武漢大突變之前就有人傳人歷史,其基礎在於新冠病毒在過去三個月的突變速率低於預期,所以可能是因爲已經對人體優化過,導致演化速率變慢。但是這裏的“低於預期”,是拿新冠來和其它RNA病毒做比較(因爲學術界一直沒有好好研究冠狀病毒,SARS平息之後,剛到的資金就斷了,所以很多冠狀病毒的特性還在迷霧之中),沒有考慮到冠狀病毒的基因比大部分其它RNA病毒(尤其是流感)大了一個數量級,它所能生產的蛋白質也就更複雜許多,那麽是否有特別的基因糾錯機制也還屬未知。所以我個人認爲,在武漢之前就有人傳人現象的機率是顯著小於50 \% 的。
\subsection*{2020-04-11 04:51}

我剛看到一個統計報告,發現美國白人欠缺維生素D的佔47 \% ,但是黑人則是86 \% 。這很合理:維生素D不是靠飲食攝取,而是由皮膚細胞在日曬下合成,黑人的膚色是為赤道非洲演化出來的,在北美自然會有日光照射不足的問題,而維生素D是免疫系統必要的養分之一。
事實上,流感的季節性可能和日曬有部分關係。現在處於夏天的澳洲和紐西蘭疫情遠比冰島輕微,維生素D説不定也有貢獻。我建議大家每天補充4000IU的D3。

有關黑人受疫情打擊更嚴重的現象,我個人認爲是可信的:其背後的原因是他們窮,沒有好的養生習慣,沒有醫療保險,所以往往帶著慢性病。這其中最嚴重的,就是他們的飲食明顯偏向廉價的工業加工食品,富含高脂肪和高果糖,前者引發心血管疾病,後者則是糖尿病的主因。請參見前文《一個嚴重的公共健康問題》。
當然這會是民主黨選舉的宣傳點之一,不過美國社會的不平等(至少在美國)不是新聞,效果可能有限。
這次爲了追蹤疫情,我也定期看來自加拿大的報導;最感嘆的是他們基本心甘情願地做美國的小弟,談起紐約來,感覺像是自己的首都似的。
\subsection*{2020-04-11 01:04}

那是2018年的評估建議,後來沒有被采納,所以必須針對新發展有所修正。
首先是中方沒有速戰速決,浪費了一年才摸清美國不是要搶錢、而是試圖謀殺的本質,拖延的結果使情勢惡化,不得不在Trump任期只剩一年的背景下,接受一個不對等的停火協議。中方的打算,應該是要慢慢執行己方不公平的義務,如果Trump落選,就可以翻盤重來。要是Trump當選連任,那麽這些龐大而有待履行的讓利,也應該至少防止事態失控。
新冠疫情和後來Trump拿中國來背鍋的努力,推翻了上述的計算。美國的仇中民意被尖銳化之後,不論大選結果如何,要翻盤重來或穩定事態都更困難許多。這裏的公關論戰和其所產生的民意新趨向,都還在進行演化之中(參見《ABC》本周搞出的陰謀論),所以無法確實估計。不過可以確定的是,疫情後的歐洲不會再心甘情願地對美國亦步亦趨,即使中美貿易戰大幅升級到必須完全切割的地步,中方至少不必擔心被圍攻。至於日本,雖然安倍仍然夢想著美日聯手封堵中國,但今年的經濟衰退對日本的打擊會是極爲嚴重,他沒有大規模興風作浪的本錢。
\subsection*{2020-04-10 01:13}

瑞士鄉下吃狗,是偷偷摸摸地,盡量避免國外的批評,絕對不敢公開頂撞。西方媒體對瑞士沒有先天的敵意,也就不會老在傷口上摸鹽;中國可沒有這樣的Luxury(餘裕?)。西方媒體喜歡拿中國人吃狗來做文章,往往有很深的惡意,大家有反感很自然,但是這和禁不禁完全是兩回事。
我一再强調,一個國家制定政策,不在於百姓以爲自己喜歡什麽,更不應該依據意識形態,唯一的標準是全民的實利。目前國際的現實是人類的政經精英已經普遍接受不吃狗為現代文明規範(Norm;這本來就不必有道理,只需要約定俗成,例如在公共場合不能裸體,還有一夫一妻制等等)之一;中國如果不禁,就會在國際輿論戰場上不斷失血,憑空賦予反中勢力殺傷力極强的額外彈藥。所以網民的反對意見,全是基於錯誤的假設(“人類文明規範必須是有理由根據的”)和標準(“政策不必考慮國際現實”)。當前是中國正在跨向世界舞臺中心的關鍵時刻,這種國内代價極低、國際影響卻大的選擇,其實該問的是爲什麽等到現在才做。
\subsection*{2020-04-08 15:25}

我個人認爲最可能的脚本(Scenario),是習近平要求外交部嚴正反擊(這點我認可;任由美方造謠而不反擊,絕對不是最優方案),然後外交部在北京的人員Brainstorm出了這個點子,沒有事先咨詢最資深的駐外使節。
我在32樓的回復裏,已經討論了這個失誤的後果並不只是讓僑民難過,而是真正提升了美方做出兩敗俱傷的進一步愚蠢行動的可能性。你所列出的理由都很薄弱,在理性權衡考慮下遠遠不足以Justify(我至今想不出完全合適的中文翻譯)冒這個險。至於外交部内部是怎麽想的,我猜不到,但是客觀條件絕對顯示他們的戰術選擇是錯誤的。
加拿大和紐西蘭的治理水準和理性程度,都遠高於美英澳三國;我已經説過,這很可能是拜Murdoch沒有把他們當作生意的重心所賜。
\subsection*{2020-04-08 12:09}

John Yoo是韓裔美籍的律師,屬於超極右派,比Navarro和Bolton還要極端。在小布希任内,他負責撰寫法律意見,認可所有的行動,包括狂轟濫炸、殺俘虐俘,尤其是為Water Boarding正名的一篇備忘錄讓他聲名大噪。
照理説,這種人的意見沒有主流精英會當真,但是美國的體制和中國不同:政經精英固然可以忽悠愚民,一旦愚民認同了“普世公理”,執政者卻無法在短期内扭轉過度瘋狂的民意反應。換句話説,在鼓吹强硬外交政策之後,會作繭自縛,失去戰略節制和戰術運作的空間。以往美國憑藉著地緣位置和整體國力的優勢,發動不公不義的侵略戰爭(例如印第安戰爭、1848年美墨戰爭、1898年美西戰爭、并吞夏威夷、以及二戰後80多次對外顛覆和侵略)大都沒有對自身不良的後果,反而開疆擴土、鞏固了單極霸權。現在Trump團體爲了甩鍋,製造了“中國責任”這個普世公理,即便執政者理解中國不是印第安人、墨西哥、西班牙和夏威夷那樣的弱鷄,繼續升級對抗會兩敗俱傷,也有可能被自己早先鼓動出來的民意挾制,騎虎難下而不得不硬上。所以正文中討論的外交部公關失誤,雖然短期後果似乎只是僑民的處境惡化,中長期卻代表著美國在外交政略上會采行更爲瘋狂極端的打擊手段,届時中方如何穩定局面、避免無限升級,困難度會比目前的公關論戰更高一級。
\subsection*{2020-04-08 08:48}

目前疫情發展的不確定因素還有很多,很難討論經濟細節,我盡可能談談可以預見的事態發展。
西方社會的組織能力不夠,沒有可能像中國這樣基本完全杜絕病例,現有的隔離措施可以減緩病毒的傳播,甚至把疫情壓制到較低的程度,但是不能全面復工,否則前功盡棄。所以到五月,意大利和西班牙這些疫情進程較快的地區,必須開始依賴抗體檢測(中方應該現在就開始大量生產,這會是下一個熱賣的抗疫產品)來辨認已經得過新冠、有了抵抗力的部分人口,容許他們自由離家活動。不過這必須有連帶的法律和科技(類似中國所用的手機App)手段;以歐洲國家執行力之差勁,又全無先見之明、沒有預做準備,只怕會再一次亂成一團,中方可以考慮現在就先爲他們開發出意大利文和西班牙文的軟件,並通過大使館給出建議。
支援第三世界,不宜過强過猛。他們的醫療基礎設施水平很低,基本沒有大規模檢測治療的能力,部分國家還必須面臨其他的流行病和蝗災等等,中國的財力物力人力都不足以全面支撐那麽大的地域和人口,更別提工作環境對援外人員有很大的危險,所以只能提供最基本的人道援助,例如通過聯合國來運送食物和藥品。
新冠要在全世界受控制,工商業全面恢復活動,應該要等疫苗大批量生產才會發生。大部分的專家依往例推測這至少要12-18個月,我覺得目前事態危急,各國投入的研發資源史無前例,所以略有提早並非不可能,但最早仍然要到今年第四季。那麽在未來半年多,有全面復工本錢的只有中國。雖然進出口會大幅衰減,如果能針對各國隔離抗疫期間的所需,高速高效地預見配合,内需則有系統地鼓勵復蘇,中方在美國喘過氣來的時候,早已是以逸待勞。再加上年底的美國大選,我並不認爲美國人能使出什麽殺手鐧,所以整個美方甩鍋爭議的後果,長期來看,若是德法不跟進,反而是中方與美國做切割的契機,一方面杜絕國内若干官員和企業主管的買辦心態,另一方面則在國際上開創中美歐三集團鼎立的新局面。
\subsection*{2020-04-08 00:24}

我並不排除在幾萬種正統的中醫配方中,還有幾款或甚至幾十款是實際有點醫學效益的可能,但這必然是極少數(因爲人類心理的Placebo Effect太强,在沒有大規模的雙盲實驗之前,很難準確地凴試錯找出正確有效的方法),效應也一定相對微弱,所以在幾十年的科學研究後,仍然沒有許多類似青藁素的發現。
中醫説不定對傳染病有一點助益的處方,絕大部分可能不是像青藁素這樣針對特定病原體的特效藥,而是健全刺激先天免疫系統的間接幫助。但是這種加强免疫系統的手段,最強的一級顯然是充足睡眠、適當運動和維生素D的補充吸收,其他都是高階的修正項。這些修正項中,目前最有希望的,是冷熱療法,例如桑拿(芬蘭是否會在新冠疫情下,得到比其他北歐國家更低的感染率和致死率,還有待觀察);它利用模擬人體發燒的免疫回路,刺激Monocyte(白血球的一種)的生產。這在抗生素發明之前已經通過大規模人體實驗,但是顯然很難做到雙盲,效果當然也比不上特效藥,所以就被擱置至今。
總之我並不敵視中醫,類似屠呦呦這樣的科學研究人員絕對應該受到支持鼓勵,但是不做科學研究(亦即大規模雙盲實驗),直接從古書上找字面聯想,或甚至憑空捏造來賣藥的,基本上都是故意做錯實驗、誤解統計分析來欺騙外行,完全是挂羊頭賣狗肉的傳銷集團,藉著侵占消耗中醫科學家的信用和名譽來謀私,真正愛好中國傳統醫學的人應該對他們深惡痛絕才對。換句話説,中醫教其實不是中醫,而是中醫的寄生蟲;就像法輪功和佛教法輪一點關係也沒有,純粹是侵占詞匯來欺騙無知百姓。
如果容許中醫教繼續滲透到醫療產業,乍看之下似乎類似皮膚保健品業(只有最基本的保濕Moisturizer和防曬有用,但這非常便宜,所有的名牌都是騙人的):沒有實用,但是能憑空創造GDP和許多消費者自行幻想出的快樂。然而這裏有兩個很大的差別:1.皮膚老化與否,沒有人命關係;2.保養皮膚,完全沒有特效藥。在醫療界,中醫教卻會擠壓真正有效的治療方法,然後造成許多不必要的人命損失和疾病痛苦,這還沒有算入騙子傳銷集團對社會文化風氣的長期腐蝕作用。但是正因爲中國學術界的普遍腐敗,執政者無法得到專家的理性建言;這又一次是不整治學術貪腐的間接惡果。
\subsection*{2020-04-07 00:29}

這正是因爲白左的理性派置身事外,所以疫情由中國背鍋已經成爲美國人的普世公理,不要倒因爲果。我從沒說要“爭取”左派,我說的是不給右翼民粹明顯的把柄,白左理性派自然會抨擊他們的非理性甩鍋;事實上我才剛批評了中國對外只知道搞直接接觸和爭取的壞處(參見第19樓)。
客觀來看,美國受新冠毒害,當然是Trump政權準備不周的責任。我做一個不太優雅的比喻:這像甲村的村長性侵了多名村裏女孩,接著把問題怪到乙村賣短裙子的店上,然後那個店的外交部出面說甲村的短裙子是受害者自己在家裏做的,乙村的憤青鼓噪稱是,但是甲村人卻群情激昂。旅美華人就是住甲村的乙村人,楊安澤站出來說乙方賣短裙是人類的羞恥。這件事就是如此之荒唐可笑又可悲。
\subsection*{2020-04-06 18:46}

我對你很失望,一個老讀者(我假設你是五年前就參與討論的“K”)居然也會堅持自己主觀迷思,胡扯硬拗。我的邏輯論述已經很嚴謹完整,針對反面意見的解答都已經白紙黑字地寫在博客上,你自己也知道重複被反駁過的歪論一般會被刪除,還明知故犯,真是不應該。
你還是把《常見的狡辯術》印出來,再對照著自己的言論仔細想想,你的論據攻擊的真是我的論點嗎?我什麽時候說過要依賴歐美的友善人士?至於理性的人,怎麽可能在工業化社會裏完全不存在?他們雖然是少數,但當然是知識精英,影響力遠大於簡單的人口比率。至於“提供彈藥”,你的辯證手法更是低劣,直接把我原文裏的“額外”兩字給藏起來,然後針對美國極右派原本就有很多彈藥來大做文章。官方搞陰謀論不但是額外的彈藥,而且是唯一能立刻讓白左理性派噤聲的宣傳核彈。
以後不會再寬容。
\subsection*{2020-04-06 15:43}

我並不認爲會有像二戰期間把日裔關進集中營的那樣的官方政策,但是由“民間”發起的類似二戰前在德國納粹黨對猶太人的打砸搶在極端情形下是有可能的。
三月初Trump剛開始甩鍋中方,我就到網絡上搜索了一陣,發現一周内在紐約都會區就有十幾起類似的事件,當時還主要只是吐口水、叫駡,真正出拳的是少數。這些事件起初只有地方小報報導,後來全國性的白左媒體剛開始利用它們來指責Trump種族歧視,外交部就搞出這個婁子,美國的輿情急轉直下。現在反而很少看到這類事件的報導,但實際上一定是更爲普遍嚴重。至於在職場和入學上歧視華裔,那也絕對會更加普遍、公開而離譜的。
我自己已經計劃回臺,所以挂念的只是小孩。但是高中年紀的人,再聰明也無法理解大時代的轉折點,結果是他對我的嘮叨很不耐煩,最後我只能提醒他不要站到火車/地鐵站月臺邊緣,治安不好的地方不要亂跑。
\subsection*{2020-04-06 01:32}

你説的部分沒錯,但是“不指望能夠說服西方理性派精英”是沒有做足夠研究的結論。這些精英幾百年來的原則就是堅持真正的理性和道德高地,Trump和Johnson在他們眼中是美奸和英奸。我一個人就看到幾十次這種左派意見領袖批評把疫情甩鍋給中國的做法,但是在趙立堅發言之後,他們反而不説話了。中國的國際形象在疫情過程中,在很多無既設立場的國家得以改觀,是自然的結果;但是各式各樣爲了私利和意識形態必須抹黑中國的勢力因爲外交部的失誤而獲得了額外的宣傳彈藥,也是事實,兩者並無矛盾。
追根究底,中國在鄧小平之後,内部失去了互相公開攻訐的傳統,所以執政官員再怎麽聰明,還是不懂英美政界、商界和媒體界不斷既聯合又鬥爭的現實,所以對外處置上總是基於簡單幼稚的一維度認知,這必須由智庫引進真正在國外長期生活工作的知識精英來解決,我已經一再説過,退休的外交官是很合適的現成資源。
\subsection*{2020-04-05 02:22}

我同意你的描述。中共的行政決策,似乎主要是由主官身邊的極少數幕僚來決定,如果牽涉到專業話題,那麽會另外引進專家來做咨詢,例如這次武漢封城就是專家組的建議。
這裏的毛病在於它不足以很好地回應治理上的很多議題,尤其是長期性卻又需要戰術上不斷做修正對應的,例如外交、宣傳和科技引進;或者是新技術變革和新知識產生導致組織、政策和資源分配必須重新構思,例如果糖的危害和稅法的改革;又或者是專家集團本身的山頭利益防止他們提供誠實的建議,例如衆所周知的一些機關。這裏因爲中共的體制,把公衆言論隔離出決策過程,在平時能避免愚民或資本扭曲政策,但是如果事務的考慮超出核心幕僚和體制内少數專家所能理解想象的範疇,就會有讓山野閑人們看著乾着急的情況出現。
照理說,智庫是彌補這些盲點的最佳手段,但中國的智庫基本是學術界的延申,所以就有你所說的空汎脫節問題,在短兵相接的行動點建議上,基本空白,決策核心自然不把他們當一回事,這又反過來減低智庫界的重要性和急迫性,結果説空話和吃空餉的現象更爲普遍。解決的方法,至少要引進有實際執行經驗的人,例如退休官員,進入智庫;然後必須要求智庫沒事找事,在體制本身沒有注意到的話題之中,去尋找改革的可能,列出確切實在的方案細節,而不只是等著上面下公文來咨詢,或者老是發同樣的清談八股。
\subsection*{2020-04-05 02:20}

我以前已經説過了,講“或許”、“可能”卻不説明是0.0001 \% 還是99.9999 \% ,就是耍流氓、欺騙讀者。我自己如果用這些詞匯而沒有進一步解釋,指的是10 \% -90 \% 之間;其他人可不一定有這種修養。
病毒是十二月初演化出來的,有相當一部分患者是無症或輕症,在那之後如果傳播出早先未知的國外病例也完全合理,但抗體不算實證,畢竟他可能是二月底、三月初得病而沒有症狀。
要不要買槍,視個人環境而定;我自己住在以白左聞名的鎮裏,不覺得有必要,而且槍支對自己和家人的威脅往往更大。手槍和霰彈槍都是在室内抵抗强盜用的,如果是要在戶外交戰,就可能需要軍用步槍的半自動版本。在美國,AR-15是最便宜實惠的,但是你若沒有經驗會是個問題。我個人非常不認可在自衛行動中使用這類大威力、強殺傷的遠程步槍,除非是團體與團體之間的對抗,例如LA韓國城抵抗黑人打砸搶的前例。
\subsection*{2020-04-03 23:32}

很高興有人去把我五年前的談話找出來。的確,你説的正是我的思路原則。
這件事的後果,主要有四:1)住英美的亞裔會很難過,但這並不影響(狹義的)中國本身的利益;2)英美的仇中態度更加牢不可破,但其影響是間接、隱性而且無從彌補的;3)中方民衆對英美的觀感也更加負面,但是因爲中國的政治體制原本就能超脫群衆的非理性喜惡,反而沒有鉗制外交戰略選項的立即影響;以上三點,都已經無法(或無須)挽回,然而4)你所說的對内政自我檢討的干擾,就是一個未來衝突的關鍵。我以前解釋過,禁絕野味買賣必須有民衆的認同(Buy-in),現在雖然政府依舊明白應該做,但執行上會有額外的困難。我曾經希望藉著新冠疫情,讓中方有機會檢討以往容忍鼓勵中醫教腐化中國社會理性思考能力的謬誤,這也成爲現階段的不可能了,但是這個問題最終總必須解決,否則就是埋下未來衰退的種子。
\subsection*{2020-04-03 05:57}

沒有信心是沒有足夠知識的後果,必須提升智庫和幕僚水準來解決。
至於中宣部,那是不同的案例了。美國陸軍的兵器部(Ordnance Department,後來改稱Ordnance Corps),從南北戰爭開始,就是惡名昭彰的山頭主義,扯後腿的專家。在1953年,北約決定開發新一代步槍和彈藥來對抗AK-47,當時遠遠最佳的方案分別是比利時的FN FAL和英國的7毫米高速彈,但是美軍的兵器部爲了硬推自己的M14和7.62x51,先是哄騙比利時,說如果他們支持7.62子彈,美軍會購買FN FAL,等到美國子彈被定爲標準,兵器部立刻食言而肥,大批采購M14。到了1960年代,M16的早期版本AR-15出現,兵器部百般阻撓(因爲它是Colt的產品,不在兵器部自己的Springfield Armory生產),後來靠著空軍的訂單,也小批量進入陸軍部隊試用,在1965年的德浪河谷戰役中,M16表現傑出,兵器部被迫正式接納它成爲建制裝備;隨即出現一連串的嚴重品質問題,使得M16的口碑崩盤。到了1968年國會舉行聽證會,發現這些問題都是兵器部故意製造的(例如槍管不鍍鉻,所以鏽蝕嚴重,還換了便宜的發射藥,卻不讓廠商修改槍支來適應),結果是近兩百年歷史的Springfield Armory被撤銷,兵器部被縮編改組。每次我讀這段往事,就聯想到中宣部;不知何時中共高層才會舉辦類似的聽證會來做檢討。
英美政客哄騙民衆,不但是傳統,而且是正常治理手段,幾個月前才被英國最高法院再次確認完全合法;反正他們有宣傳霸權,幾百年下來,靠著搜刮搶劫來的財富解決問題,還順便譏諷打擊貧弱的對手,結果民衆洗腦徹底,實際治理水平再怎麽低下,也沒有後果。
\subsection*{2020-04-03 04:35}

唉,正是我已經解釋過的,RNA病毒發生突變有一定的速率,所以從各分支之間的基因差異可以倒推估算分歧發生的時間,這些時間沒有一個早於2019年十二月中,包括S和L的分歧(最佳估計:十二月13日),甚至L/S的共同祖先在何時獲得那個能利用ACE2的S-Protein都可以估算(十二月6日)。
我沒看到所謂“五毒病人”的報導,也不認爲這個説法合理可信。至於我爲什麽做那個推測,是因爲病毒本身不是活體,只有在人體細胞内才能進行生物作用,所以你在基因譜系圖看到的每一個分叉點,都對應著一個未知的病人。從華盛頓州的病例族群的最早樣本(一月19日,也是美國的第一個病例)往上溯源,所遇到的分叉點後來成爲在武漢、福建和重慶的發病族群,這完全吻合病毒從武漢經由病患國内和國外旅行而傳播開來所應有的結果。如果你硬要說這些病毒族群的共同祖先(2019年十二月24日)來自美國,那你必須解釋爲什麽四周之後,這一支病毒譜系在中國百花齊放,而美國只有一個病例,後來的病毒都是它的直系子孫。尤其是這一個武漢+福建+重慶+美國宗族的第一個分叉發生在一月9日,不含美國病例的另外那一支後來全都只出現在武漢,所以它們的共祖當然應該也在武漢;既然這比美國頭號病例早了十天,誰是頭誰是尾非常明顯。
\subsection*{2020-04-03 00:29}

趙立堅剛開始發聲的時候,我還指望那是他個人的躁動,但是沒多久就很明顯是受命而爲,所以我也就忍到現在,一直到大陸的輿論越來越把戰術權宜誤解為事實真相,再不説話就會讓華語圈的知識分子也被誤導進去,才不得不寫文澄清。 
一般人看英美的白左,往往只注意到像是好萊塢明星那群純粹閑得發慌、爲了自我感覺良好而盲目追求聖母姿態的名流(Celebrity),其實真正的意見領袖是一些學術界和媒體界的知識精英,他們雖然還是愛國,但是有足夠的理性在很多議題上就事論事。尤其這次搞事的是Trump和Johnson這兩個小丑和他們的團隊,白左的宣傳力量原本蠢蠢欲動,可以為中方所用。我看到他們欲言又止,拼命躲開陰謀論的尷尬態度,才會在正文裏那樣寫。
\subsection*{2020-04-02 21:14}

你説錯了,事實剛好相反,基因分析明顯表示所有被檢測到的新冠病毒(至今有2600+個被分析出來),都源自同一個在去年底突變而獲得强大人傳人能力的共同祖先。 
那些基因譜系圖(那個網站我很早就介紹給華語界,非專業讀者會知道它,可能直接或間接源自我的評論),並不在乎各分支之間有多大的差異,只要有任何差異就畫成獨立的一支。鍾南山院士自己說過(參見正文),他懷疑新冠在華南市場之前就可以人傳人,是因爲它已經非常適應人類細胞,在疫情爆發之後,新冠的突變速率因此而低於RNA病毒應有的表現(基因對環境的優化程度越高,繼續突變的動力就越小)。換句話説,目前檢測到的新冠基因的變異分佈,甚至少於全部同宗應有的結果,那麽還要硬拗說有不同的族群分支,就是完全無知的外行説法。 
我在正文裏特別强調“群衆的愚蠢”,正是因爲這種判斷應該由有相關知識能力的專業人員來做;完全外行的憤青展現出謎之自信,在不斷被打臉之後,還一個接一個的陰謀論層出不窮,實在只能用Herd Mentality來解釋。
\section*{【宣佈】讀者須知}
\subsection*{2021-06-11 20:32}

我是天生就高度自律,年紀大了,受健康問題所困,反而沒有年輕時的堅毅,只能設法用成熟智慧來彌補。
我兒子不一樣,天性開朗愛玩,但無法承受精神壓力;他小學期間曾拿過三次國際象棋的州冠軍,進了中學之後反而因爲受不了競爭壓力而放棄了。從中式升學體系出身的父親,一開始當然不太諒解;但是我後來終於體會到人類心理的脆弱,改爲一方面寬容、支持(但並不是忽略客觀標準的一味稱贊),另一方面反復鼓勵他建立自律能力。過去這年,他在家上網課,展示出中學時期做不到的主動刻苦用功,讓我十分欣慰;我接著建議的是,他開始改善Time Management的效率。
你既然明白自己的心理弱點,可以把改進列入長期Self Improvement計劃之中。我每年都會回顧過去一年是否有自我進步,並為下一年做出展望;這裏我所指的,不是外在的任務成績,而是内發的能力修養。一般人離開學校之後,就忘了必須有系統地自求進取;其實人生7、80年,如果持續學習成長,往往可以突破很高的上限。
\section*{【邏輯】常見的狡辯術}
\subsection*{2021-06-09 12:39}

寫作的最基本技巧,是造句;這裏當然需要廣汎的詞匯,和多種語法的靈活換用。這方面主要是自己下苦工,市面上也有很多教學資源。
我個人更重視全篇文章的結構,事先想好該説的内容,然後分別安置在合適的段落,邏輯因果環環相扣,則文字自然順如流水。
學術論文並不是一般的“寫作”;它們其實是現代的八股文,有著非常嚴格的特有格式,幾乎像是填表一樣。此外這些特定文章格式,再加上文字句法,都隨行業而異。我小孩去年在家上大學一年級的網課,同時修了生物和化學,作業都包含寫模擬的論文報告;可是兩門課的要求不盡相同,有些地方甚至完全相反,例如生物報告裏普遍使用被動句法,但用在化學報告就會被扣分。至於經濟系把主變數放在縱軸、函數值放在橫軸,更是讓做自然科學的人百思莫解的奇怪習慣。
\subsection*{2019-11-27 12:29}

歐盟是很鬆散的聯盟,有些事務像是公共投資或者電信管理,主要由各別國家決定。中國去和東南歐的窮國合作,是很自然也有益的,但並沒有任何國際宣傳上的效應。
我所討論的,不是這方面的議題,而是美歐聯手在國際規則上給中國下絆子,例如WTO修改規章的事,很顯然是德法兩國説了算。他們或許在内部協調上,會有不同的意見,但是絕對都把中國視爲主要競爭對手,知道要一致對外,所以不能指望要分化他們。
但是他們和英美不同,是可以講理的。這兩天德國經濟部長和美國駐德大使爲了華爲的事而互懟,德國精英在《FT》上的留言評論就是一面倒地支持自己的部長。中國要是有像樣的宣傳部門,可以和德國企業家協調起來,尋找適當的議題,一次一個逐步扭轉民意。當然,現實裏這大概是不可能的。
\subsection*{2019-11-26 18:10}

英國不管是否真的脫歐,在歐洲的話語權必然會大幅降低;經過這麽一閙,歐盟的邊緣國家,暫時也不敢再作怪太甚,所以未來幾年,德法核心反而會更爲鞏固。我說對歐統戰,其實主要還是對德統戰;德國的政界和媒體受美國控制很深,工業界的政治影響力遠遠不如美國,光是指望企業家出面為中德關係説話,是不夠的。 
歐盟的CAP(農業補助)已經開始縮減,現在正在爭吵中的七年預算大概還可以沿襲舊制、矇混過關,但是到了2027年下一個七年預算就很可能會有真正的分歧,隨即產生新的離心動力。然而中國崛起在外交上的最關鍵時刻,正是未來的這幾年,尤其如果民主黨建制派在2020年當選總統,必然會重拾聯歐制中的戰略,所以我才會强調要改進對歐洲(特別是德國)的統戰宣傳。
\section*{【美國】【戰略】三談中美貿易戰}
\subsection*{2021-06-06 21:01}

美國農產品的批發價在過去這年漲了不止一倍,中國卻因爲和Trump的那個“第一階段協定”而必須繼續采購,而且這個國際條約義務必然也影響了中澳貿易戰的計劃和決策。
現在Biden的主要意圖,應該是維持中國對美農的采購額,以便爭取選票。我從3、4年前中美貿易戰一開打就反復强調,以退讓來換取妥協是徒勞的,美方必然得寸進尺;可惜當時全華語世界,只有我一個理性聲音,孤掌難鳴。我當時也已經解釋過,如果錯過第一反應時間來做對等反擊,就會陷入慣性的泥淖,很難擺脫單方面的讓利。中國決策單位或許已經理解當年的錯誤,但與美國的後續談判必然還是會有原本不必要的困難和損失。
更糟糕的是,目前還沒有任何跡象顯示決策階層有足夠的戰略認知和決心來擺脫純粹被動的防禦心態。攻擊是最好的防禦;美元則是美國霸權最重要的根基,卻也是最關鍵的弱點。疫情的餘波蕩漾,很可能會把美國的通貨膨脹(至少短暫地)推高到3 \% 以上。這將是出手打擊的極佳時機,如果又錯過了,那麽外來的麻煩就會比最優解延長10年左右,造成十萬億級別的機會損失(Opportunity Cost)。
\subsection*{2020-04-29 16:49}

我以前沒有注意到翟東昇教授,不知道他對中美關係和全球化趨勢有和我類似的判斷。我至少身處美國,可以就近觀察體會他們的權力運作;翟教授能從中國看出美方態度的轉變,實在很了不起。
不過我必須附加一個説明,美國固然急著要逆轉全球化,歐洲很可能也會有所反思,但是歐美獨占世界工業經濟的時代已經一去不返了。中國有足夠的體量和技術,不必純被動地接受歐美的決定;可以在失去美國市場之後逆流而上,與世界其他國家繼續增强經濟上的分工互補。這絕對不像1979-2008年全球化風潮下的順風環境那樣簡單容易,但是也不會是一戰和二戰之間那30年的重演。所以與其稱之爲“去全球化”,我想“中美切割”是比較精確的説法。
\subsection*{2019-06-01 02:02}

很多讀者都想當然爾,認爲我應該到大陸去任職。其實光因爲我的國籍,這就是不可能的事;連我的文章發表在《觀察者網》上,都困難重重。中共内部的審查,已經明定我的這個博客“不適合”在大陸傳播。上次有關《美國陷阱》的那篇《域外管轄權》,差一點就被政治類編輯無視,還好和我熟的那位科技編輯力保,拖了三天才出現在首頁。如果當時真被否決了,現在就不會有中文版的《美國陷阱》,歷史也就走上一條不同的道路。
中共是世界上最大的政治組織;組織的規模越大,官僚趨勢越强,這是自然的規律。我因爲局外人的身份受到排擠,固然有點泄氣,但是如果這些官僚原本就都識貨、能看出最優解,那也用不着我來置喙了。所以我以旁觀者的姿態來做評論,很可能要持續很久。
\subsection*{2019-05-16 12:13}

丘成桐我不擔心,他的數學很強,但是忽悠人的伎倆也就只是幼稚園級別的,大家一看便知。
習近平最大的軟肋,還是在蕭墻之内,地方和專業幹部各種賣國自利的行爲充斥。去年我已經提過,外國芯片公司(如高通)搞地方合資來扼殺真正的本土產業,在貴州浪費了幾十億,那個合資公司上月剛剛倒閉,這個貴州又和高通另開一家。至於智庫和學術界裏的帶路黨,更是猖狂。
如果美國對中國禁運高科技零件例如芯片,一些次要的公司會倒閉,但是最主要的企業如華爲會有足夠的體量和能力存活下去,幾年過後反而會促進本土替代工業,成爲中國產業升級的最大助力,所以我個人並不擔心Trump能怎麽樣升級貿易戰,他的Trump Cards只在短期有效,長期來説反倒是幫中國的大忙。
\section*{【國際】新年的回顧與展望(三)}
\subsection*{2021-06-01 01:55}

不是。我的考慮在於歐洲老工業國家普遍出現中產階級不生或只生一孩,占人口比相對低的無業文盲或宗教狂熱份子卻一生十幾個,所以光是靠補助或鼓勵來試圖解決總量問題,反而會促生更糟糕的人口平均素質大幅下降。例如以色列的Fertility Rate表面上在工業化國家中高達獨一無二的3.0,但實際上只占半數的工業文化人口生育率和歐洲差不多,總平均的浮腫來自佔人口少數的Muslim和Ultra Orthodox Jews(目前分別占21 \% 和13 \% ,Fertility Rate分別是4.5和7.7),後者的教義甚至無分男女禁止服兵役和找工作,全靠國家福利在家拼命生小孩,眼看再過50年必然會成爲人口多數,届時以色列還能不能依靠美國補貼來發福利,是一個極大的隱憂。
此外,這次中國人口普查的數據,表面上說Fertility Rate是1.3,但大家要理解,這不能直接拿來和國外數字對比。這是因爲如同GDP一樣,各國的統計方法不同,所以一般只能在時間縱向來作比較,要跨國橫向對比的話,就必須依賴專門為這個目的而特別做的分析。在Fertility Rate方面,最新的國際橫向研究,是2019年World Bank做的,所以我做比較始終只引用那裏的1.69數字(並不代表World Bank的統計數字就一定更可靠,但如果他們依舊沒有貫徹統計方法的橫向一致性,責任不在我)。中國的網紅大V拿著紙面數字來大做文章,正是我反復警告過的笨蛋峰式行爲:既然連基本的統計原理都不懂,就沒有針對公共事務做批評論斷的權利。
\subsection*{2021-05-13 20:29}

完全符合過去幾年的主流估計,沒有什麽新的信息啊。
至於生育率下降的問題,我以前已經反復評論過:社會工業化、現代化,女性普遍獲得教育和就職的權利和機會,自然生育率就會低於Replenishment Rate(2.1),政治和經濟環境只有微弱的影響。庸人本末倒置,把自然規律的主導性效應轉嫁給次要的人爲因素,不但設立完全不切實際的期望,對解決問題無濟於事,反而有干擾妨礙的作用,而且危害社會風氣與和諧,扭曲群衆的三觀,方方是前車之鑒。
這裏,一胎政策沒有及早取消,當然是個錯誤,但並不是主因;現在要彌補,可以從加大對基礎教育和公共醫療的投資來著手(因爲它們原本就是解決貧富不均問題的必要手段),但也不能指望會有奇跡,能在未來20年維持1.7就很好了。你看北歐國家中位收入和社會福利那麽高,到處都有免費托兒所,又沒有升學文化或996工作的壓力,但丹麥和瑞典也只有1.7,挪威1.5,芬蘭1.4,比起保守黨執政十年,大砍福利又内部撕裂的英國(1.6)並沒有實質上的差距。中國政府的行政效率世界第一,但人力有時而窮,不可能事事逆天而爲。
\subsection*{2021-04-07 13:14}

人性。
人類心理先天就不是拿客觀標準來決定自己是否幸福,而是和周邊的人評比;這裏的重點是“周邊”。例如扶貧時欠缺動力的村落,往往是因爲客觀的封閉導致心理的狹隘,覺得全村人都一樣窮,沒什麽不好;即使外出打工、見過市面,也因爲城市是“另一個世界”而不在乎。一旦村裏有人致富了,大家才會積極地想模仿。日本和台灣的反中、仇中文化,有重要的一部分來自這個非理性心理因素:美國人是白人,屬於另一個世界,向來都是主子,不適用正常道德標準,對次等人種可以愛打就打、愛殺就殺;只有臨近的中國泥腿子們一夕之間反超自己,才讓他們心裏不舒服。清朝的知識分子也是出於這個心理效應,在鴉片戰爭之後50多年,依舊盲目自信,直到甲午戰爭才全面覺醒。
當然,人性中的非理性因素不止一種,而且互相矛盾;即使是同樣對鄰居羡慕評比的心態,也有正面和反面兩種結果。日本和台灣都是由有心人多年經營,有意引導啓發,才慢慢把仇中心態特別挑出來建立成爲全民共識。所以李登輝操弄民意、製造仇恨的本領固然令人驚嘆,國臺辦坐視自己的任務不斷倒退而無動於衷,這種定力也遠非常人所能及。
\subsection*{2021-04-02 06:31}

金融巨鰐掌控的美國學術界和傳媒,花了半個多世紀吹噓自由市場,細節之一是故意忽略匯率大幅浮動所隱含的代價(其實大部分的資產價格波動也有同樣效應),原因是漲跌之間有的賺的是金融資本,賠錢損失的卻是實業。中國的經濟以實體製造業爲主,如果匯率自由追逐每個月、甚至每天的平衡點,就會與生產運輸周期大約是3-6個月的進出口貿易脫節,所以做決定必須前瞻幾個月。在當前這個時節,特別不能放任人民幣大幅上漲,這是因爲美國通脹已經冒出苗頭,長期債券的利率開始上升,依往例這是美元大水回收的前奏;雖然這次美聯儲受聯邦赤字拘束,不能自由提高利率、收緊銀根,但匯率出現多空交戰、上下波動是必然的,中國雖然不像印度、土耳其和巴西那類高債務國家那麽敏感,但仍然應該小心謹慎、避開風頭,2014、2015年資金外逃的風波是前車之鑒。
\subsection*{2021-03-29 16:05}

你的討論很好,非常有助於出身中華民族理性士大夫傳統的人,認清西方思維的脈絡。不過越是深刻的原理,距離應用層次就越遠;深陷非理性宗教深坑的歐美民衆,光是和他們說“你們受宗教思想荼毒”,一點正面效果都不會有。那條被我刪掉的評論就是例證。
說“宗教是反智的”,其實還不夠精確,因爲它暗示著還存在著非宗教的反智;正確的説法是“有系統的反智就叫做宗教”:這是因爲宗教是有系統的傳播分享信仰(Faith),而信仰的定義正是無視事實和邏輯的懸空結論。與此同時,科學是基於事實與邏輯來發掘真相的系統,所以從定義上宗教和科學就是相反並對立的。你可以脫離事實和邏輯,拿科學和數學的詞匯堆砌出新的系統,例如超弦,但這就是宗教而不是科學了。
\subsection*{2021-03-29 07:54}

我想提醒讀者,耶穌是否曾經真實存在,目前沒有足夠的證據做定論。能夠確定的是,現存的新約聖經和Matthew、Mark、Luke的原版已經相差甚遠,是第四世紀基督教成爲羅馬國教之後,召開大會修訂整理出來的,主要遵循Book of John的説法去改寫其他福音書,而John(作者的真實身份無法確定,這裏的論斷來自對文字的分析)比前三者晚了一代,明顯偏愛對神話和寓言做Literal Interpretation,所以才會有許多怪力亂神的東西。
另一個能夠確定的,是耶穌的畫像原本都沒有鬍子,到了公元三世紀中葉,羅馬帝國内亂頻仍,基督教做爲慈善組織,開始獲得廣汎的支持,包括一些高級知識份子,於是耶穌才長出鬍子來。這是因爲羅馬以軍事立國,向來都把鬍子刮得很乾净,但它也繼承了希臘文化,而在希臘的畫像和雕像中,奧林匹克山的衆神以及雅典的哲學家都是蓄鬚的。當然那時的耶穌還是南歐臉,要到中世紀他的臉才日耳曼化。

你說基督教内含二元論,真是切中要點。現代西方一神教的起源,一般人以爲就是猶太人在1st Millennium BCE早期興起(1177 BCE前後,東歐/西亞青銅文明大崩潰Bronze Age Collapse,五大帝國有三個消失,剩下的埃及和Assyria也大幅收縮,幾百年後地中海東岸的廢墟中出現了許多城邦,包含以色列)後所創的猶太教,其實舊猶太教到1st Millennium BCE的後半就嚴重腐化衰弱,根本沒有擴張成長的動力;那時每個城邦小國幾乎都有類似的本土宗教,猶太教一點也不特別。
到了第四世紀BCE,Alexander The Great征服了整個西方世界,使得西起希臘、東至印度、南到埃及、北及波斯的文化能夠自由交流,其中最重要的是當時波斯的國教,也就是祆教(Zoroastrianism,又稱拜火教;大家比較熟悉的明教,又叫摩尼教Manichaeism,是公元三世紀的改版),傳遍整個Macedonia帝國。因爲波斯是那時代那地區裏最先進的城市文明,祆教和各城邦的本土宗教相比,處處顯得高大上,所以基本上所有的舊宗教都或多或少吸收引進了祆教的教義,尤其是善惡二元論。
到了公元前第二世紀,埃及和以色列的宗教團體在認真吸收了祆教思想之後,都出現百花齊放的現象,分裂出許多新教派,二元論的影響極爲重大深遠。在公元前一世紀,猶太人基於舊有的Messiah觀念,改爲希臘文Christ(意思是“The One”),引進埃及人發明的救世主死而復生故事,與自己新創的天堂/地獄對立論(其中天堂的大天使剛好叫做耶穌Jesus)相結合,又醖釀了100年左右,在公元70年代,由三個Rabbi(亦即Matthew、Mark和Luke)整理出正式的課本,這也就是基督教的起源。
所以與其說西方宗教的特點是一神教,更精確地看,二元論才是他們教義的最終核心。
\subsection*{2021-03-22 16:03}

你沒有看出這次Anchorage會談是中國外交戰略的歷史性轉變嗎?中國首次采納了Putin式的高調對等反擊,而且不像我三年前所提倡的那樣專注在實利上,現在連場面話都不退讓。對亞非第三世界國家來説,這裏的信息是,我們至少和歐美平起平坐,你們好自爲之。
其實如果只是明白美國想要置你於死命的真面目,照我的老建議做(亦即實利上不退讓,但臺面上的空話就隨便了)依舊是最佳策略;所以習近平決定更進一步,錙銖必較,我覺得有其深意。最可能的考慮,是為出兵臺海做準備、提早打預防針,讓歐盟瞭解中國有自由行動權,如果妄加干涉,會有很大的代價。預先改變外人的評估,在對Trump步步退讓了三年之後,尤其重要。Putin在2014年之後,任何妥協都不肯做,就同樣因爲收復Crimea是歷史性任務,不能拿出來討價還價。
既然已經撕破臉,我想提醒中共,應該儘快把金無怠的烈士身份明定出來。當年敵强我弱,讓一位愛國志士死後還得繼續犧牲,是無奈之舉;現在已無必要否認他的貢獻,爲他正名是國家最起碼的回報。
\subsection*{2021-03-21 13:50}

我從2009年看到美國精英啓動仇中宣傳,知道這即將成爲全民共識,而且美國人之蠻橫下流向來沒有下限,所以從Trump一開始對中國施壓,就反復强調最佳策略只能是立刻對等反擊,否則美方必然得寸進尺、永無寧日。
經過三年多的教訓,中國終於做出正確的決策。這一方面是客觀上,Trump把所有的牌都打完了,中方無可畏懼(但反過來看,就是不必要的虧也都吃了;我一直說國家放任學界水準低、不爭氣,必然會由全民買單,這又是一個例子),但更重要的是,主觀上中國執政階級認清了美國的現實真相,尤其在美方政權更迭、卻變本加厲之後,才放棄最後的一絲幻想,明白和解共存是不可能的。
從美方的角度來看,Trump是公然敵對的始作俑者,但中方認清事實、采納正確對應出現時間差,倒霉的卻是Biden:他未來幾年被共和黨貼上對中“軟弱”、“失敗”的標簽,是跑不掉的事。這對2022年和2024年的選舉,可能會有影響。
我知道很多讀者在三年多前認爲我太激進了,但是我可沒有激進到說要公然撕破臉、在外交場合强硬囘懟的地步。換句話説,現在中國官方的政策比我原先的建議還要激進。這正是我當時解釋過的:一開始不敢做、不願做,日後反而必須做得更多。當年不同意我結論的人,應該自我反省,爲什麽會拒絕接受客觀、完整、詳細的邏輯分析,反而堅持直覺反射去和稀泥;這不是科學研究的正確態度。
既然我事先努力解釋的事實已經被接受,再放馬後炮寫文章分析就沒有實質意義,純粹是非理性的娛樂消遣;那是媒體上其他評論者的工作。最近有《八方論壇》的觀衆,通過史東向我推薦一篇網文,結果很明顯是我早年博文被抄襲轉述百代之後的子孫,讓我啼笑皆非。如果有陌生人向Darwin推薦演化論,我想他也會有相同的感受。博客這裏嚴格要求讀完所有以往的評論,才能參與留言討論,這是原因之一。
\subsection*{2021-03-17 22:13}

烏克蘭全國腦死,無可挽回,貿易可以,投資合作就算了。緬甸不一樣,它是中國的近鄰、備用的印度洋出海口,已經建了油氣管道,國民被西方宣傳洗腦固然是一大問題,卻不能輕易放棄。
不過這裏中方把投資經商看得太輕鬆。其實英美的經驗早已明示,經商賺錢必然會引發土著反感,所以必須先殖民(英國用軍事,美國以思想爲主、軍事爲輔)來確保安全。美國人老是說中國白嫖,其實有其道理:現代全球自由貿易體系的支柱並不只是海軍,美國經濟學界宣傳洗腦的重要性往往更高。如果中美撕破臉之後,還指望靠著美國打下的思想基礎通行無阻,實在太天真了。緬甸的特點就在於它的文化水準太低、戰亂歷史太久,所受的美式宣傳洗腦全屬政治性的,完全沒有學到針對一般穩定國家精英的自由經濟理論。面對這樣的挑戰,光是依靠簡單的通商互惠,必然行不通,非得有頭腦靈活的主管,全方位考慮應變,因時因地制宜,才可能慢慢扭轉局勢。然而中方若有這樣的人才,當然是優先去處理中歐關係,所以短時間内是做不到的,只能審慎接觸,預先防範當地民意的反彈。尤其緬甸地小民貧,貿易和投資的利益都不大,重要性來自地緣位置,那麽爲了賺一點小錢去挑起民憤,更加是本末倒置。
\subsection*{2021-03-05 23:38}

文攻武嚇的前提是對方有可能回心轉意,民進黨和支持他們的愚民可能嗎?現在高層已經默認文統無望,還有什麽好談的?你以爲中國領導階級是英美説一套做一套的套路嗎?正是因爲什麽都不談,反而代表決心已經下了,確實的方案已經批准了,只等著適當的時機。
我從2015年就估計關頭會在2025年前後,現在有了新冠這個變故,全世界對中國刮目相看、敬畏有加,你想是方便還是阻礙動手?其實過去六年每次我評論相關議題,都有點擔心,不是對自己的預估沒有信心,而是怕目標聽衆之一的民進黨聼不進去,反而是國民黨的大佬們拿來參考,覺得是自己鹹魚翻身的機會,會想要及早佈局來當三姓家奴。所以從一開始就反復强調,真正的考慮重點不在一般人所關注的轉折點,而在於其後的治理善後問題,連正確的方案都一再解釋過了。中方是否能汲取香港的教訓,以長遠的眼光來選擇最優的策略,是目前僅存的不確定性。
\subsection*{2021-03-04 10:22}

首先,你可能有誤解:在歐洲和東亞都面臨生育率驟降的問題後很久,美國一直保持一枝獨秀,到2008年還維持2.055的Fertility Rate(生育率,亦即成年女人一生中所生產嬰兒數目的期望值),在金融危機之後才慢慢減低,過去四年穩定在1.78上下。相對的,中國的生育率在1995年就跌到1.75,過去20年穩定在1.6幾的範圍,開放二胎的效應要等到2022年才會重新衝上1.70的關卡;所以中國的老齡化問題始終是比美國嚴重的。
不過客觀來看,1.7其實不是特別糟糕,例如德國是1.60,瑞士1.55,日本1.37,新加坡1.23,台灣1.20,南韓1.09。女性能普遍接受高等教育、進入職場的社會,自然會有低於2.1(考慮到小孩夭折和男嬰多於女嬰等等因素,一般接受2.1是長期穩定的級別)的生育率,但是這樣的現代工業國家,如果能確保健康的經濟成長、合理的社會福利以及公平的教育機會,並不難把生育率維持在接近2.0的程度,例如美國在1970年代遭遇能源危機,也曾經有過一波生育率萎縮到1.77的經驗,一旦經濟復蘇,生育率也跟著回升了。
中國處理老齡化問題的關鍵,在於一方面必須推進產業升級以保障經濟成長,另一方面必須立刻扭轉醫療和教育私有化過程,反過來大幅增加在這些領域的公共投資,改善全民的基本生活品質。這兩者最終都依靠科技研發效率必須高於國際上的競爭對手,所以我才一再强調學術界的貪腐是當前中國内政上最大的威脅。如果有了誠實健康高效的科技研發,自然就會有足夠的新創財富來解決其他的問題。
\subsection*{2021-02-26 12:31}

你沒注意到外交部的反應在過去這個月已經完全脫胎換骨,轉爲我倡議了多年的主動鬥爭態勢嗎?幾天前公開懟上歐洲白左,說英國和德國才是應該自我反省的國際惡霸;對海地政權的批評更是史無前例。可惜這些論點不如去年瑞典的案例那麽驚人,不過只要策略對了,機會自然會浮現。
我討論政策的重點,是大方向的修正,談到執行細節只是爲了做示範例證。而且我的目標是對世界有實際影響,不是要製造最大批量的文章,或吸引最大數目的眼球,所以不論國家采納正確政策有沒有直接或間接參考我以往的論證,目的已經達成,應該把精力和時間專注在還沒有改正的議題上。世界上多的是事後諸葛亮,用不着我們去和他們搶飯碗;新疆這事已經絕對明顯化,我事先預警的任務已經完成,在位者自有分寸,公共論壇上的炒作討論純屬滿足群衆的非理性心理需求,不會有一絲一毫的實際意義。
至於明年奧運,中國在新冠疫情之後的國際地位,早已不可同日而語,白左教的宣傳抹黑只在四眼國家(New Zealand有足夠的智慧,知道自己是小國,必須依小國的邏輯來行事)有廣汎受衆,連在歐陸都會遭遇强大的質疑和制約力量,所以除非發生非常重大的意外事件,否則沒有什麽好擔心的。
\subsection*{2021-02-17 22:50}

雖然我對這條規則已經反復强調過,但還是有人前仆後繼地往同一個坑裏跳,顯示出這的確是一個普遍的心理盲區,所以我正在考慮是否要再解釋得更詳細一些,你就剛好也做了評論。
你説得並沒有錯,但只適用於已經有足夠背景知識和邏輯辯證能力的讀者。這個博客是理性分析的平臺,然而讀者來自各行各業、水準參差不齊,而我並不想要完全關閉初級班(畢竟有求知欲和正確學習態度的人,多讀我的博文應該會有很快的進步),所以不得不接受做不到你所描述的分析檢驗的讀者發文。結果自然是留言欄充斥著水平不怎麽樣的問題,這時我決定哪些值得回答、哪些直接刪除、哪些必須禁言所用的標準如下:如果提問者本身連相關的事實和邏輯都懶得打理,直接談結論的,就必須準備挨駡被禁了;引用野鷄評論是這類行爲的子集。如果讀者至少試圖事先做一點邏輯分析,然後用不是太含糊或囉嗦的語言表達出來,那麽我就會考慮回答;如果我能在議題上談出有價值的回應,那麽留言就會被保留下來,否則就刪除。所以留言被刪除並不一定是壞事,可能只是沒有足夠的正面價值罷了。
\subsection*{2021-02-06 15:03}

我才剛說不能引用野鷄評論,你就要以身試法,還特意挑這麽蠢的,哎。禁言一個月。
這個博客是理性討論的平臺。讀者引用外人評論自然是爲了跳過邏輯辯證的過程,直接拿一個不成立的論述來讓我討論。我反復地說,要尊重Russell‘s Teapot原則,也就是提出結論的人有責任事先完成嚴謹的事實檢驗和邏輯推演,不能自己偷懶,然後强迫別人花時間來試圖證僞。
雖然權威和言論品質完全沒有因果關係,但我體諒部分讀者是初級班,邏輯辯證的能力還在發展培養之中,所以禁掉野鷄評論是無奈之下的折衷,至少有名有姓的人會比較有動力避免特別低級的談話(這句話不適用於台港這類已經徹底愚化的社會),如果真沒有水準,我指出過一次,下次讀者就知道要躲開這人,並不是我們崇拜權威,否則這些權威的身份名氣能和教宗相比嗎?難道我有時間精力對聖經的每字每句都做出長篇大論?
\subsection*{2021-02-05 04:29}

英美式資本主義體系先天就是很鬆散的啊,因爲財閥巨頭人數不少嘛,誰也不服誰。
在紀律管理上,只須要抓董事會,確定他們是自己人,重要的事接了電話會照辦就行了。這裏最直接的辦法,是財團并購,例如過去40年美國近千家媒體公司被整合為6家聯合企業Conglomerate。控制6個董事會,簡單得很。
董事會負責大事上監督,日常執行細節由總經理和總編輯貫徹,每家媒體兩個人,這也不難找。偶爾必須從CFR/ECFR空降已經算是很著痕跡了。然後他們再確定只雇用“志同道合”(亦即完全接受白左或右翼民粹洗腦)的記者,很難嗎?有時這些記者良心發現,行業就嚴格封殺,幾年一次就足以殺鷄儆猴,你真以爲多數人會願意爲了誠實描述中國而選擇職業自殺嗎?
還有,確定段落之間沒有太大的空白是你的責任,再犯我要開始格殺勿論了。
\section*{【台灣】【工業】台灣能源供應的未來}
\subsection*{2021-05-18 13:24}

這是民粹社會的常態,我已經反復解釋過成千上百次。缺電缺水缺疫苗之類的問題早早就可以預期、也的確一再被預言警告,但是選民連最基本的因果邏輯都不懂,所以始作俑者有恃無恐。
2014、2015年台灣公共論壇上曾有一波針對未來電力供應的討論,凡是稍有專業能力和態度的官員學者,都出面列舉數據分析,證明的確有很大的隱憂,但民進黨拿個“政治性危言聳聽”的標簽就一語抹殺那些質疑。這種無視論據的真實性和有效性、轉換話題去談動機的狡辯術,只對完全沒有科學素養和邏輯思維能力的聽衆有效,偏偏在美式民主體制下,做最終決策的正是極端弱智的選民群衆(如果他們原本不夠笨,就必須由媒體經由多年努力去降低平均智商,並且專爲笨蛋峰上的山寨寨主們提供發聲的平臺,例如《蘋果日報》在台灣和《Fox News》在美國),所以轉換話題去談動機也就成爲百試不爽的伎倆,在不同國家反復出現,像是英國脫歐論戰了五年,所有的潛在經濟損失都一再被專家精確預測過,脫歐派也是一句“Project Fear”就搞定,事後也沒人問責。針對這個現象,還有人發明了一句話,並僞托給Mark Twain:“It is easier to fool people than to convince them that they have been fooled."
選民愚昧,倒不一定馬上自己倒霉。但是最易受傷害的永遠都是底層的貧苦民衆;正因爲窮人沒有選舉能量,每年無辜死亡千百人也就必然毫無後果。我在《八方論壇》上提過,2010年英國保守黨上臺之後,大幅削減福利,立刻導致每年平均多凍死7000人,2018年冬天居然凍死了16500人(參見https://www.dailymail.co.uk/news/article-7535747/Britain-facing-silent-crisis-emerges-16-500-people-died-freezing-homes-year.html),但是媒體報了也沒用,因爲不影響選舉結果;真正影響選票的,是像上個月Boris Johnson(顯然是故意)先違反脫歐協定、不發捕魚証給法國漁民,激發後者到英國島嶼示威抗議,然後再派軍艦去繞一圈,自然就在選舉大獲全勝。民進黨一樣也是對著大陸叫駡幾句就能勝選,那當然不必去關心水電供應的問題。
以上所談直選體制+媒體愚化的國家衰敗機制,是無解的,如果沒有外力打破循環,就只能等到既有的財富資源慢慢消耗殆盡,社會徹底崩潰爲止,這可能要百年以上,無數窮苦民衆的死亡和受難也只能越演越烈。雖然台灣的資本纍積不如英美,但好在它可以有外來的拯救力量,希望底層民衆的苦難早早結束。
\subsection*{2020-01-15 18:04}

定性來説,我同意(不過你是否把冬夏說反了?),但是大部分的政策,魔鬼在細節裏,定量的分析我必須等年底囘台灣,找到本地氣候變化的專家做討論之後,才能斷言。 
季節的問題是比較難解決的;晝夜的差別在十年後應該不是大問題,這是因爲聯網的電能儲存是現在全世界投資的重點,用鋰電池的計劃幾乎天天有新的宣佈,雖然這麽一來液流電池暫時拿不到錢,但是長期和整體來看,應該還是會有迅速的進步。 
我上個月上《八方論壇》的時候,提到當前全球油氣產能過剩,尤其是天然氣,果然新年才十幾天,已經有幾家美國的頁岩氣公司在發SOS。這代表著台灣未來幾年改用天然氣發電阻力很小,但是十年之後就很難說了。 
我寫這篇文章的用意,並不是要越俎代庖、搶電力專家的飯碗,而是從政治分析的角度,描述這種超級規模的投資,風險大、貪腐的機會多,執行的品質和受益人,可能是台灣剛結束的這次大選最重要的實際後果之一。
\subsection*{2019-11-26 12:56}

這個問題的癥結,其實在於GDP計算方法,你可以參考我以前寫的相關文章。 
中國的經濟,仍然主要是實體工業,所以GDP和工業用電有近乎正比的關係。美國的製造業在總量上看已經回天乏術,少數幾個還有優勢的區域,如石化工業並不是用電大戶,高科技則獨木難撐。舉個具體的例子,工業上用電最凶的,可能是煉鋁;在1988年,世界產量第一是美國,中國只有它的1/5,到了2018年,第一是中國,達到美國的35倍。 
至於美國GDP的持續增長,我以前也反復討論過了,虛胖的現象很嚴重;不過這和印度不同,不是統計單位刻意作假,而是金融、醫療、法律等等與生產沒有關係的服務業掠奪太大的利潤。例如美國的醫療佔GDP的18 \% ,而西歐有全民健保的國家,平均也只有9 \% ;這其中藥品公司雖然以天價著稱,其實只拿了醫療開支的14 \% ,真正刮錢的是保險公司和醫院,光是後者就佔30 \% (不包括醫生和小診所,這兩者另佔了20 \% ),也就是藥品的兩倍多,隨便一個盲腸炎手術就要十多萬美元,GDP不膨脹才怪。
\section*{【海军】未来十年的中美武器对比(三)}
\subsection*{2021-05-05 02:54}

航母和潛艇不一樣,核動力不是剛需、而只是錦上添花,所以200MWe的艦用反應堆並非高度優先。
過去一年多的各種蛛絲馬跡顯示,海四代花落沈飛,將是FC-31的改版。
“嫉惡如仇”指的是以嚴格的道德標準加諸社會,並全力批判不達標的人。我的立場表面上是如此,但實際上是不同的思路。我以前解釋過,道德和政治都以公益最大化為目的,但前者是用來自律、後者才是律人的,所以博文選擇批評對象的標準,並不是他們道德有虧,而是亟需出臺政策處理的重大危害公益事件;換句話說,我寫的不是道德批判,而是政策建議。因爲任何社會都有極多不公不義的現象,從理性的政策檢討而不是道德評判出發,更方便挑選其中最隱蔽、最嚴重的損人利己做法,集中火力進行打擊。例如我討論過,高能所向來假稱貢獻、幾億幾億地騙公款,其實那些實驗徹徹底底沒有任何實用前景,但因爲金額還不離譜,我也就不做負面批評,一直到他們升級出千億級別的大對撞機,才挺身而出、揭發真相。反之,如果只看道德標準,很容易輕重不分、抓小失大,把時間和公信力浪費在鷄毛蒜皮的小事上;這正是白左聖母心態被英美霸權主義操弄利用的機制。
你把“嫉惡如仇”這個標簽誤用在我頭上,除了我不懼強豪學閥之外,可能也因爲我往往是全社會頭一個出面做控訴的人。不過這也不是因爲我的道德水平特別高,而是我懂得多、看得清楚,所以才會屢次出現衆人皆醉我獨醒的局面。我的用意在於提升華語知識分子的理性思維能力,指出他們的既有盲點,不是爲了誇耀自己的覺醒,而是爲了教育有志之士,希望大家一起參與全民公益最大化的奮鬥,從而影響實際政策決定。
我的努力是以實際改革成果為最終目標的;既然台灣已經沒救了,自然也就沒有必要做無謂的犧牲。
\subsection*{2021-05-03 13:36}

不是。要造一個功率夠大的核反應堆很容易,但噪音達標卻極難。美軍到了1970年代的洛杉磯級,蘇聯到了1990年代的Akula級,才獲得足夠的靜音航速,能夠遂行現代核潛艇戰術。
其實到了洛杉磯級和Akula級的靜音水平,全艇的各種渦輪、齒輪箱、閥門等等都可能成爲噪音短板,例如Akula有一個水泵特別容易老化產生噪音,原製造商建議每年一換,但俄軍在冷戰結束後當然沒那個閑錢,所以實用上就遠遠不能達到設計的靜音指標。
中國的09III級落後很遠,09V級可能達到洛杉磯/Akula的水平,但和美俄最新的Virginia和Yasen級仍然會有差距。中俄現在越走越近,我建議中方用055、075、076和054B等等先進水面艦艇和俄方交換4艘以上的Yasen級,這一舉三得:
1)中國提早10年(開發09VII所需時間)獲得世界一流的攻擊核潛艇,在2020年代就能彌補水下戰局的劣勢;
2)Yasen級所含的先進技術(尤其是核反應堆)可以加速09VII的研發;
3)如果開戰,俄方很可能有中立國地位,敵方即使偵測到共軍的Yasen級潛艇,也不能貿然加以攻擊,那麽這批潛艇就特別適合穿插到對手後方作爲偵察節點,伺機打擊第二綫補給船隊。
\subsection*{2015-09-03 00:00}
是的,DF26有反舰型是今天透露的消息中,对我来说最重要的。美国航母被拒止的范围又被后推了2000公里,共军用航母打击关岛就成了更大更实际的威胁。我想前两年习近平硬是决定多买两艘航母和这事有联繫。

航母的事我以前零零星星讲了一些,如果有读者不熟,我在这里很快复述一下:原本中共海军的计划是016辽寧号(即001级)就足够做研究和训练,到2020年左右再开建一艘002级有弹射器的航母来做弹射训练,如此一来要到2025后中共海军才有两艘航母。航母的出勤率是低于50 \% 的,那么同一时间只能出动一个航母战斗群。习近平一上任就加了一艘001A级(即017号,目前在大连船台总装,明年初下水,2019年服役)航母,002级也提前并增加为两艘,所以在2025年中共海军就会有四艘航母,可以同时出动两个战斗群,这就有实战意义了。\section*{【台湾】一件小新闻}
\subsection*{2021-04-20 20:05}

我囘你上一個留言的時候,還沒有看這篇,結果談了基本同樣的事情,連“唯意志論”都殊途同歸地用上了。
這個博客討論過幾千個議題,但只有一個最重要的核心論述,那就是一切必須以事實和邏輯為根據,遵守科學原則和方法,才能求真,然後才可能做到公益最大化。所有的“政治”和“道德”,都是爲了追求公益最大化而發明的,只不過前者代表著組織性的努力,而後者是個人對自我的要求。
雖然理工科必須與自然規律打交道,所以更常感覺到事實與邏輯的重要性,但正因爲社科容易誤入歧途,其實更須要强調對科學和理性的堅持、以及邏輯思辨上的嚴格訓練。我對文科人的批評,並不是在貶低這些學術,剛好相反,正是由於我對社科極度看重,所以絕對不能任由它腐爛。
\subsection*{2015-10-26 00:00}
别忘了美国也在过去40年逐步法西斯化了。所有的新移民都吃亏,不过风头最健的是反拉丁移民的风潮。

臺湾还衹走出了几步,要悬崖勒马还有可能,但是必须面对自己丑恶的真相,才能深刻反省。如果出了问题,衹想找各种藉口来把它淡化,反而是最糟糕的路。

这件事的真正症结,最重要的在于臺湾有没有普遍的歧视现象(从校方和市府的反应,答案很不乐观),其次是韩小妹一生有没有感到排斥、歧视和霸凌(这不应该由她的母亲来回答吗?可是臺湾社会急急忙忙地躲开了查询的责任),最次要的才是她自杀当天的触发因素是否也是有心的排斥、歧视和霸凌(我认为是有的,但是可以接受反面的可能性)。至于这个最不重要的一点有没有法律上可以起诉的实证,完全是无关紧要的烟幕。\subsection*{2015-10-23 00:00}
这个事件不论细节如何,重点在于一条人命有可能是族群歧视的受害者,那么就必须正视并深刻反省。越是忽略学校和政府的反应,而反过来钻研细节、咬文嚼字的,就越是在转变话题,并藉此躲避基本的社会责任(亦即在客观真相大白之前,假设最糟的情形;我们不是在审判那个老师,不但无须而且不能做无罪的假设),反而更印证了臺湾社会在本事件中的完全失败。例如校长居然在第一天就能对外宣称:"以人格担保康姓老师并未歧视或霸凌女学生",既然校长自己不是目击者,这在逻辑上是不可能成立的;处理学生自杀事件,用这样的搪塞之词,从下到上,都在敷衍,怎么可能有可信的调查?这种对弱势群体的明显藐视,本身就证明了歧视的存在。那个老师的态度还有可能是个案,政府机关和社会大众的反应如此,那就无可卸责了。\subsection*{2015-10-23 00:00}
特别把这个留言满血復活,好让大家看看是不是如我所说的把大家针对社会反应的讨论,偷偷摸摸地改成对个人法律审判所需的水准,藉此把事件细节的不确定性喧宾夺主用来达成刚好相反的结论。

我以前对玩狡辩的留言有时还有些耐心,这次特别激愤,是因为做臺湾人还是中国人都没有做人重要,而做人最要紧的不是智商或"知识分子"与否,而是要有良知。这里一个弱势群体的女孩子莫名其妙地死了,她的家长受到的衹是漠视和推诿。你就算对我们的讨论有异议,提起时也应该对受害者家庭有基本的尊重,不该耍轻佻的嘴皮子。我自我反省了一番,正是你的态度才让我对你的狡辩论起特别大的反感。我年纪不小了,自己受批评也不会激起强大的情绪反应,但是看到对弱势群体无感的冷血动物还是无法忍受的。\subsection*{2015-10-23 00:00}
你犯的都是我已经解释了几十遍的伪科学狡辩术,我今天忙着写稿,实在懒得再解释一次,删你的留言对我是省事,对你也少了羞辱。现在我稿发了,你既然非要自取其辱,我就简单说几句。

虽然没有口供,但是从事情的背景环境,最直接最简单的解释就是霸凌,而且可能有族群歧视成分。依Occam's Razor,这就是科学的结论。

全世界每天都有被霸凌而自杀的年轻人,每个都是心理不稳定的,这和霸凌的社会背景完全无关(更别提太阳花事件,那扯得更是远了),而后者才是这里讨论的重点。你硬要扯上前者,就是Straw Man Fallacy,用中文讲就是放烟幕、转话题。来我的部落格还敢搞这种不入流的把戏,不管你是不是有自觉,我都不会客气的。\subsection*{2015-10-23 00:00}
我知道霸凌的细节还没有确定,但是正因为它还没有确定,政府和社会都有责任把它当一回事认真处理,这也刚好是这件事的重点所在。我看过的欧美类似事件太多了,不论实际上谁对谁错,在刚开始有指控时,一个先进社会的正常反应是假设它是真的,而认真追责。可是臺湾没有这样做,却是可以确定的。

你的论点若是我们不能百分之百确定那个老师用了族群歧视的字眼,这我可以理解,但是正如我已经解释过的:1)可能性很小(那个老师事后所给的版本基本不可能是事实,否则一个如此易怒的小孩活不到13岁),不是科学上应做的假设;2)我不是在讨论是否应该把她立刻枪毙,而是臺湾社会事后的反应,那么她和受害者的互动细节就反而完全无关。衹要族群歧视有可能发生(更别提目前的证据显示可能性很大),那么臺湾社会就必须满足一些行为规范。我所讨论的正是臺湾在这方面的失败。

我再说一次,对霸凌细节的不确定性,不能当成不作为的藉口。这是一个针对社会责任的讨论,他扭曲成必须采用个人法律责任的认定水准,若不是有心的狡辩,就是愚蠢的错误认知。狡辩者和蠢蛋一样,在这里都没有受欢迎的权利;当这个人也明显地没有良知的时候,我极强地后悔与他有任何交集。\subsection*{2015-05-10 00:00}
我看过,基本上是在极少数几个细节上吹毛求疵,不影响整体估计的可信性。

真抱歉,我知道大陆读者从小崇拜毛主席长大,可是事实上他是个杀人不眨眼的自大狂。共产党若是没有邓小平,就会成了中国的大灾难。事实如此,我也不能避开不说,只希望你们慢慢吸收习惯。

几年前有一个德国经济学家回忆他80年代上大学的时候,他的好朋友是左派学生,一天到晚跟他说美国是国际金主的工具,世界万恶之源。他那时笑那朋友太偏激、想象力太丰富了。20几年后,他见过世面、能做独立思考了,有一天遇到那个朋友,他特别正式道歉,说你一直都是对的,我后知后觉。我自己对美国的印象也经过一个类似的转变;我想很多大陆学生也必须慢慢学着对毛泽东时期的历史做独立理性的分析。\subsection*{2015-05-10 00:00}
我写这个博客不是为了交流,而是为了正台湾的视听,所以一切以事实为本。我当然欢迎你们来,但是我连自己母亲的绿色政治立场都可以诚实批评,对你们的信仰自然也不会有禁忌。

科学家的学习态度是基于Bayesian Statistics,也就是对自己不确定的事,会内定两个变数,第一个是可信性,第二个是对前述的可信性估计值的不确定性。我对大跃进不是专家,主要的根据是杨继绳和欧美的历史学家。杨先生的写作态度相对地严谨,我觉得比官方记录(例如亩产万斤)要可靠得多了;换句话说,我认为他的可信性是高于50 \% 的,但是因为可靠的官方记录是完全空白(这可不是杨先生的错,而是共产党的责任,所以他们要是不高兴,大可以做一个严谨可观的研究啊;我指的是一个独立的研究,不是光对杨的批评),而杨先生的研究只是一个个人的业余调查,所以第二个变数(也就是不确定性)是相当高的。既然你有从父辈的第二手资料抵触了杨的说法,那么我对他的可信性估计值就要往下调,而不确定性则更往上调。正如我在前面所说,我对大跃进不是专家,不能做确定的结论,这个话题纯粹是一句话带过而扯出来的,我并不想争辩,因为我没有足够的知识和你争辩。

至于文革和毛泽东晚年的领导风格,我却可以确定那是灾难性的糟糕的。没有人能否认邓小平的领导风格和文革时期的习气南辕北辙,那么除非你要否定邓小平的实事求是精神,否则他的前任就必须为留下一个大烂摊子而负责,不是吗?\section*{【空军】未来十年的中美武器对比(二)}
\subsection*{2021-04-20 01:57}

美國權力精英公然無恥的自私自利,始於Milton Friedman的歪論,經由Reagan發揚光大,現在早已司空見慣、習以爲常。所謂的大家為私利奮鬥,自由市場自然會把公益最大化,當然只在霸權無敵的前提下,才可能維持私利、公益兼顧的假象;實際上是消耗祖產(小羅斯福遺留下來的全面霸權),再豐厚也有耗盡的一天。
我在1990年代剛轉金融,首席交易員就叫我去讀《Atlas Shrugged》,把我嚇了一跳,原來這麽明顯的胡扯,只要和自我利益重叠,聰明的人也會自願受洗腦。從那時起,我就常常引用Upton Sinclair的那句話:“It is difficult for a man to understand something, when his salary depends on his not understanding it.”
\subsection*{2016-07-04 00:00}
F22在东亚不管用,但是在欧洲和中东对付俄国,却是它原本设计的目标。现在美俄的关系又不好了,所以勉强可以说有正当的需要。

但是实际上180架对付俄国已经绰绰有余,对付中国却没有足够的航程。美国原本私下打的算盘就是让"盟友"出钱为F35开发新电子系统,然后反过来用在升级F22上。这当然还是可行的,但是要解决航程不够的问题必须重新设计整个机身,那可就是超大工程了。

所以这个所谓重启生產綫的建议,主要是几个国会议员为选区争取联邦经费的套路,空军没有太大的热忱,连Lockheed都怕影响F35的采购计划而衹是客气地点点头,在未来五年会真正启动的机率不到20 \% ,而且拖得越久,越没有价值,大家不用当真。\section*{【美國】【國際】新年的回顧與展望(二)}
\subsection*{2021-04-16 14:08}

我對這種文科角度所作的理論闡述,有點過敏,因爲他們説來説去,就是收集一些零碎的片段事實,然後做出聯想式的論斷,每一個邏輯步驟都沒有嚴謹的因果關係,更別提預測能力了。以我的標準,這不是在做研究,而是在編故事。我一再强調社科教育應該是基於等同理工科的邏輯標準;現有的這些理論就是反面的例證。
把國家神聖化,不是每個建國神話的目的嗎?美國的建國神話還不夠離譜?
國家的“靈肉分離”,那麽它的“靈魂”就是宗教嗎?靈肉結合又是怎麽定義的?西方的興起,與“Age of Enlightenment”重叠,Enlightenment的重點就在於以科學取代宗教,那麽他們自我否定靈魂的過程,也是最成功的階段,怎麽解釋?
世界性的文明,是在近200年科技發展後的新現象,別説中國作爲新的世界霸主是否適任還在未定之天,近代國家體系彼此交往、衝突、融合,原本就是一個全新的動態進程,用古文化的遺留來做大部分的解釋,合理嗎?你不必假設這200年都沒有重要改革和發明嗎?
你不是理科出身的?怎麽還這麽容易受文科論述的騙?
\subsection*{2021-03-08 08:36}

有讀者把我想要傳授的核心價值觀,用自己的話重述出來,總是讓我很欣慰的。這裏我只補充一點:“普世價值”、“民主”、“自由”等等詞匯,一樣都被英美宣傳體系侵占、扭曲、污染到親媽都不認識的程度。其實只要基於邏輯重新出發仔細思考,就會明白要是真正的“普世”,只能是演化的結果,而人類演化的過程是在游獵經濟和小規模部族的環境下發生的,所以一方面任何現代政治體制都不可能適用,另一方面普世人性也絕對不會在工業社會中有完全正面的價值。中國古代人性本善/本惡的爭議,尚且是毫無意義的空談,英美把自己吹捧成普世真理,更是無恥、無知、無良的騙人把戲。
科學是比普世人性還要深刻廣汎的真理,它的“内在一致性”是絕對的、獨立的,而且不但在空間維度有普世通用性,在時間上也是永恆不變的。當然,這並不代表每個人在每件事上都必須只做純粹科學的思維考慮,但在有關國家運作、國際競爭的事務上,科學是對内公益最大化、對外保證勝利的唯一手段。
\subsection*{2021-02-16 23:36}

其實説明白了,並不難理解:996是資方犧牲勞工的生活品質,來換取較高的研發生產效率。這種勞資對立的議題,應該由國家(而不是選票)來做仲裁,這時整體利益是決定因素。因爲人類社會目前處在一個高度分裂、相互競爭的國際環境下,中國的首要考慮必須是維持國際競爭力,也就是高人一等的研發生產效率。只有在學術界和工商業界廉潔高效的前提下,才不需要勞工加工加時來彌補,也才有餘裕强力出手立法遏止强迫加班。所以縱容學閥腐敗,不只是他們侵占的那些公款損失了,而且是整個產業升級的長期努力都受其掣肘,連帶地全民都必須在收入和生活品質上做出原本不必要的犧牲。這裏每一步次優後果都把損失放大了千百倍。科技部和稀泥危害之大,遠超它整個預算好幾個數量級,既然公款吃喝只占預算的一小部分,放任前者而嚴查後者是典型的Penny wise, pound foolish。
至於忍受美元搜刮,正是因爲美國有全方位的霸權,中國必須在整體國力上超越美國,才可能以人民幣取代美元。國力競爭的關鍵,正又是產業升級,而產業升級的必要條件,是高效的研發,追根究底,又回歸到健全的學術研究環境。我在2014、2015年就提過,習近平的反腐,是創造内部條件,一帶一路則是創造外部條件,真正的長期戰略目標是產業升級。既然國家的最高戰略就在於此,怎麽能容忍一個貪腐低效的科技部和學術界?這麽簡單基本的邏輯矛盾,被置之不理,真令人百思不解。
你沒看懂這些我幾天前的解釋,是不是沒有細心去讀舊文?我的文章和評論不是寫來給讀者消遣用的,内涵比多數教科書還要廣汎深刻,文字卻極度簡潔。反復閲讀直到熟練爲止,是讀者的責任。我鼓勵大家有空回頭復習舊文,尤其是理論性較高、較爲抽象的文章,對照這幾年的新時事做為例證,應該會有更深刻全面的理解。
\subsection*{2021-02-09 19:34}

資本的特性是必須有意願犧牲一切(尤其是良心和公益)來追求利潤報酬,才有大機率的可能繼續纍積財富,所以任正非是偶然的例外,在國際資本主導的絕對自由市場背景下,他的財富纍積速度必然落後於聯想之類公司背後的大亨。這是極爲嚴格的逆淘汰,長期下來,適者生存的最終勝利者只能是Trump之流無恥、無知、無良的吸血鬼。
我以前已經解釋過,文革的確是世紀級的大災難,光從當時各種虛報誇大造假成爲每個人生存下去的必要手段,就可以看出是絕對的錯誤。我説過,我一生追求的只是一個“真”字,其原因在於“真”是“善”和“美”的前提,要作惡必須先造假,虛假的“美”則内含絕對的醜陋。
然而我也同意,中國在改革開放40多年之後,近年的問題早已不是太左,而是遠遠太右了。一般民衆雖然不完全明白其幕後的複雜效應,卻可以感覺到資本和權力處處結合起來,扭曲經濟和社會上的公平競爭。但正因如此,我們必須堅持以事實真相和理性邏輯來引導他們,不能容許社會思潮走上虛僞的捷徑(例如美化文革的歷史),否則就徒然給別有用心者挾民意自重的機會,薄熙來、MAGA和白左都是前車之鑒。
\subsection*{2021-02-09 12:52}

這其實也是我們討論多年的議題:英美的資本必須愚化整個社會,才方便自己獨占國家財富;白左和右翼民粹雖然表面上對立,實際上同樣是非理性的宗教式迷信,只在膜拜的偶像和教義有所不同。一旦社會裏的理性力量被消滅,就必然越來越愚蠢、越來越反智,台灣早就示範了這個過程,而英美在2016年之後才明顯化罷了。
不過我想提醒你,英美的媒體財團和法輪功有一個根本性的差異,就是前者抹黑中國是損人利己,而後者卻是損人不利己。這是因爲一旦激起英美民間對華裔的深刻仇恨,首當其衝的受害者當然是當地的亞裔居民。一年前紐約剛開始有新冠時,立刻就有好幾起隨機攻擊亞裔的案件。這類事件其實遠比一般人所知更爲普遍而且持久,就在本周加州還有警局在一連幾起重案之後成立特別小組。所以美國人是壞,而法輪功卻是極端愚蠢的瘋狂。
\subsection*{2021-02-08 16:05}

拜登當選後,新疆會成爲英美污衊抹黑的焦點,我已經預先討論了一年了。幾天前博客留言欄進一步探討,英美的洗腦宣傳在近年(主要在2008年金融危機後)開始被現實事件明顯打臉,在脫歐和Trump之後進一步加速加劇,至今已經無力對外(指非白左國家)攻擊,只能轉爲防禦,專心忽悠自家愚民。
到這個地步,誠實靠譜的體制與體制比較對照,在白左國家境外逐步占了上風,我多年寫作的重要目標之一已經達成,是時候讓其他評論員接手,自己可以專注在新的重要話題上。我素來不喜歡攪和熱門題材,放馬後炮的專家到處都是,有獨見的人應該引領輿論,點出被大衆忽略的重點事實。
上月我在《八方論壇》說,7年博客只有大對撞機和《美國陷阱》兩次有明顯直接效果,指的是爆炸式地成爲全國性輿論焦點,其實在切實揭穿美國體制騙術上,我曾經有幾年是華語界孤單的領路人,幾十篇文章雖然沒有馬上引起大波瀾,但文中的事實和論點慢慢傳播出去,被一再轉載復述,幾百代之後我到現在還能在公共討論中看到遺傳基因。例如《自由撒謊的美國政府和媒體》絕對是華語界第一篇談美國警察每年殺人數目被造假的文章(因爲英語界剛開始有小新聞,我就注意到了);《美元的金融霸權》則被借鏡最多最久、影響最深遠,因爲在那之前,其他作者不明白美元一升一降其實是一個完整的循環,他們只談其中片面的一個方向,邏輯上不自洽,很容易被批評反駁。
現在英美宣傳體系關起門來騙自己人,我們已經沒有插嘴的餘地。我自己連用英語發聲都覺得太危險、太徒勞,不值得冒險;外交和宣傳單位固然必須繼續見招拆招,民間社會把白左那套當笑話看就行了。
\subsection*{2021-02-04 06:37}

還記得2017-2018年,中美貿易戰剛開始,我就說必須認真考慮對等報復,在名正言順的背景下,可以解除長期的負面影響,同時嚇阻美方繼續升級?結果中方選擇不做,這是完全放棄短期利益,只專心確保長期的對外企吸引力。我也反復解釋過,一旦在開戰時這樣選擇,就不能反悔,不再有對等報復的選項。後來中方熬過了短期的痛苦(雖然高科技禁運和承諾購買美國產品仍然餘波蕩漾),在新冠衝擊全球的加成下,達成吸引外企的戰略目標。
現在對Trump政權幾個政客和很少數的軍工企業做制裁,是在前述的大環境下,爲了保持尊嚴所做出的象徵性作爲,並沒有立即的實質影響。長期來看,也只是立下前例,在中美力量對比改變之後,才會真正長出牙齒。
\subsection*{2021-02-01 23:55}

要能這樣睜著眼睛說瞎話,還被(至少許多己方民衆)當一回事,是要有本錢的,否則徒然惹人譏嘲。台灣的本錢,在於孫運璿建立的經濟軟硬件和兩代(亦即我這代和上一代)相對高素質的理工人,體現出來就是臺積電和類似企業。英美的本錢則在於幾百年纍積的財富、技術、資源,比台灣更加豐厚得多。例如這次新冠疫情,英國政府的處置可謂教科書式的錯誤,該做的不做、不該做的(例如挪用大筆NHS的資金給政客的親朋好友來購買昂貴、遲到、不及格的防護裝備)層出不窮,以致單位人口死亡率甚至高於東歐的落後國家;但是一旦疫苗開始生產,他們的注射率也很快讓歐陸國家望塵莫及,其原因就在於英國因爲歷史因素,控制了歐洲最大的幾家藥廠。
以往的區域性或全球性危機,英美可以簡單憑藉超越其他國家幾個數量級的雄厚財產和資源來應對,事後再吹噓體制的優越性。七年來,我一直嘗試著揭穿這個騙術,新冠疫情讓這個工作容易了許多。
\section*{【基础科研】孪生质数假设}
\subsection*{2021-03-28 11:46}

你死纏爛打,到底想要説什麽?堆砌一連串似相干不相干的論述,浪費大家時間,虧你還是主修數學的。純學術的應用價值很難預測,這就是你的結論?那不是我剛剛説過的嗎?我最近的那篇正文才剛提醒大家體諒我時間精力有限;你要吹噓現代數學論文的成就和貢獻,請發在你自己的博客,這裏不是讀者自打廣告、滿足虛榮心的場合。
丘成桐去偷Perelman的成果,數學界沒有一個大佬站出來批評,光從這一點,可以看出整個行業那時就已經爛透了;2003年美國科學界可還沒有現在這麽腐敗啊。因爲數學系不像高能所那樣成天想著騙公款,就算徹底腐敗、永遠搞不出新的實用成果,對國家社會也沒什麽危害,偶爾有新解還可以當做消遣軼事的來源;這也是我評價基礎科研的底綫:最起碼不要危害國家社會。然而滿足這個底綫,並不代表有什麽了不起。數學要是能重現17、18、19世紀推動科學整體進展、提升人類生活水平的舊榮光,我會第一個寫文章致敬。在那發生之前,請不要浪費口舌談未來可能。
\subsection*{2021-03-27 22:21}

那我也再解釋清楚一些:我反復定義過,基礎科研(數學不是科學,但在這個話題上,性質與基礎科研完全相同,可以類比)是沒有直接立即的應用,純粹為滿足主流理論完整自洽的努力;然而這並不代表它的價值評估,就可以不顧對人類社會的最終貢獻,剛好相反,那依舊是一切公共事務的準繩,基礎科研也不例外,只不過這裏的應用價值是非常間接的,所以很容易被行内有心人扭曲以自肥(參見丘成桐、王貽芳的文章)罷了。
Riemann Hypothesis之所以那麽重要,就在於它處於許多數學子學科(而且是Proven Useful的子學科)的交匯點,不但可以同時解決多個完整自洽問題,而且證明所采用的方法,必然會為既有理論開啓新的、更深的理解。孿生質數問題,的確要比它低一級,但這只是特級和一級的差別,並不是把21世紀新數學自動擡舉到同等重要性的藉口。解決老領域的老難題,其意義在於這些領域的應用性,在過去一兩百年都已經浮現出來;即使新的理解沒有直接的幫助,至少把理論從片面破碎提升到完整自洽,這絕對是有間接好處的。
相對的,現在數學界發論文的主題,往往和理論物理一樣,是在沒事找事幹。例如丘成桐深度參與的一系列微分幾何新方向(包括上個月有個神童陳杲所發表的有關Supercritical Deformed Hermitian-Yang-Mills Equation),啓發自超弦所給出的計算議題,其“應用”如果存在,也只能是在超弦,既然超弦給人類社會帶來的貢獻是負值,那麽這些新數學的價值也必然是負的。
我在前一個評論裏,說人類社會的大觀點和行内人員發論文的考慮,往往背道而馳,其核心邏輯就在於純數學(和基礎科研)可以研究的議題有無限多,其中對人類有價值的卻是有限的,而人類所能投入的人力財力資源也是有限的,那麽光是拿過往的經驗(也就是不必事先深思是否未來會有實用價值)强迫用在研究飽和的新世紀裏,必然會產生很大的浪費。這就好比以往是沿河而下(陸地代表應用),所以不必考慮航行方向和淡水供應等等問題,一旦出海了,還以爲能隨便找個方向繼續無限航行,就是很快要渴死全船人的方案。(地球的海洋並不真是無限,所以這裏的類比假設航行器很原始,在獨木舟級別。)
我並不是說獨木舟就不能出海,而是管理單位必須有正確的理解,不能讓所有的船隊都跟著某大佬只走一條路,必須統籌規劃,把有限的資源分配到盡可能多的探索方向上,而且事先就為所有船隻準備好淡水補給,不能聼大佬的意見只分配給他的跟班。正因爲基礎科研的先天不確定性,大佬的偏愛,往往沒有實際意義,如果有私心,則更加只有負面的參考價值。這也包括Langlands Program在内;我對它沒有足夠的專業知識來做論斷,不過並沒有看到任何客觀睿智的分析,如果未來真有實用意義,那麽只能是機率極小的隨機事件(實際上很可能是有限除以無限,等於零)。
\subsection*{2021-03-27 21:09}

中國的基礎科研管理,是典型的“Penny Wise, Pound Foolish”;用中文成語來説,就是“幹大事而惜身,見小利而忘命”。
基礎科研的尖端,先天就沒有已知的正確方向,只能靠人員自身的執著、多方嘗試;所以歐美的終身教授制,是在這個前提下,不得已的妥協:年輕時的成就換來入門的門票,然後是極度的研究自由。這和應用科學剛好相反,後者的方向和進度都可以事先預估,過程中也可以嚴格監管。
中國傳統上處於追趕狀態,研究方向可以簡單參考先進國家的經驗,所以資源摳門、考核嚴格還沒有大礙。現在開始與歐美平起平坐,不從改善研究大環境著手,只想著照抄外國剛在考慮的忽悠項目(亦即歐美還沒有驗證為正確方向),那麽自然就是鼓勵“假”“大”研究,提拔不做真科研、只會玩政治的高手。出現一批從學術生涯一開始就以造假來評上的院士,是必然的後果。
\subsection*{2021-03-27 10:01}

我們的出發點不同:我是以人類整體知識包綫為標準,所以解答一個完整重要子學科最後最難的釘子戶問題,才是第一級的大突破;你用的是行内生涯的觀點,哪一個新興題目容易出論文,就是最快最Exciting的“進步”。Perelman的解答是完整的,丘成桐想要寄生後續的論文,結果無法超越,只好作弊;張益唐倒是留下一點後續工作,但也不夠大家分。這樣的題材,對只想著多出論文的人,當然不是很“有趣”的。
事實上只要行業還有研究員額,總會有人找到熱點題材來發論文。我在博客提過,物理界的Frank Wilczek、Lisa Randal、Sean Carrol、Nima Arkani-Hamed都是發掘容易發論文的新題材,然後炒作成爲熱點的專家;但除了Wilczek50年前的成名論文之外,發在這些新題材上的幾萬篇論文,對科學的實際貢獻是負值。所以一樣數目的論文並不代表同等幅度的進步;我說數學是成熟的學科,正因爲它出論文的速率或許還在增加,但實際進步的脚步卻明顯地越來越慢。行内的評等和資源分配,是相對的比較,庸人(只有專業技巧而沒有智慧的匠人,如Witten或丘成桐,在專業之外的議題上都是庸人)超越不出小圈圈的視野;但只有絕對的產出對人類社會才有真正的貢獻和意義,這也是我做評論的標準。
\section*{【宣佈】幾則博客事務和感言}
\subsection*{2021-03-07 18:54}

高能所把自己和其他新興學科放在同一個高度來爭資源、搶經費,是典型的狡辯,利用虛僞前提來顛倒是非。
承受曼哈頓計劃的餘蔭,世界各主要工業國家在冷戰期間都對高能物理過度投資,在美國尤其突出,人員比合理程度高出一個數量級,但正因爲最頂尖的人才高度集中,在60年代之後就已經只剩下純理論的產出,與現實中的經濟、產業、軍事、社會完全脫節,到了70年代之後,連理論都做完了。但是美國一直到1990年代初,冷戰結束,才開始緊縮其規模,SSC超導超級對撞機被裁只是一個標杆事件,幕後則是高能研究員額縮減了大約半個數量級的慘劇。但客觀來説,還有半個數量級應該裁撤,只不過教授的職業生涯是50年,而美國至今才只有20幾年時間等他們退休罷了。你看他們除了寫科幻來騙億萬富翁的錢之外,就是全世界東跳西跳想要忽悠各國政府建對撞機,不正是在自己國家快活不下去的結果嗎?
中國對高能物理的過度投資,並沒有像美國那麽誇張,但是80年代的主政者依舊受核彈的聯想,不聽楊振寧先生的勸告,在裁撤大飛機、發動機、和芯片投資的同時,把當時極爲寶貴的資金投入基本沒用(同步輻射的價值,遠遠比不上被裁撤的產業,而且因爲實際上用不着北京對撞機的能階,甚至連投入的那些錢都賠掉大部分,更別提專業人才的誤置)的高能物理。我説過,鄧小平一生的政策,就以這件事爲最大錯誤。所以目前中國高能界的員額浮濫,事實上和美國的程度是相似的,只遜於歐洲。那麽正確的處置,不但不是扶植成長,連維持現有規模都屬浪費,應該采行類似美國90年代的緊縮手段才對。畢竟連像數學這種極度成熟的學科,非常冷門,員額、資金十分短缺,每隔十年左右都還能有一個現象級的大突破,而高能物理理論卻是從1974年起就基本死透了,現在這些動輒上億的實驗不是在驗證上古理論的細節(例如中微子),就是想要瞎貓去碰死老鼠,你憑什麽和有應用價值的學科搶人搶經費?
現在中國的高能所,已經是吃香喝辣,我所知道的中大型計劃,沒有十個也有半打。他們不但沒有裁員,而且一直在擴張招生。這雖然是純粹浪費國家珍貴的基礎科研人才和經費,但還沒有到傷筋動骨的地步,我原本也不會想要花時間揭露他們的把戲(我從不會空口白話,你去看看2014年我談錦屏山實驗,不全是正面的評論?後來講大亞灣實驗,也是一句批評都沒有的;換句話説,我對中國高能界,不但不嚴苛,其實是很寬待了);大對撞機這個級別的坑,實在是太過貪心、離譜,我說王貽芳和他的嘍囉們是賣國賊,絕對不是危言聳聽,而如同我博客幾千個論斷的傳統,是精確的客觀事實,只不過是既得利益者非常想要遮掩的客觀事實罷了。
至於什麽“圈子”、“生態”之類的説法,我保證在清末和民國被用過至少幾千萬次來為當時的烏暗政治文化正名;學者不想著撥亂反正,卻去引用亡國的論調,對得起自己的良心嗎?犬儒主義的質疑態度,可以用來揭穿利益集團的花樣,但不是放棄堅持真理、事實和公益的藉口。
\subsection*{2021-02-26 08:09}

“偏創造性的洞見而不是體系理論化”是準確的評論。畢竟我對政治學連半路出家都説不上,學術界既有的理論和文獻我多半既不熟悉也沒有興趣。然而這並不代表我不能有所貢獻;事實上這個領域非常需要行業外的思想家來注入新的觀點,尤其是從真正科學角度所作的分析。而且即使是拿Toynbee為例子,你也必須注意他的“完整體系”固然風靡一時,但不到一代人時間,就因爲過於淺薄和陳腔濫調,評價一落千丈。
我的經驗是,中國大陸各個專業都人才濟濟,如果我去談任何一個只牽扯單一專業的問題,不太可能超出行内專家已知的範疇。既然我的目標是有實際影響,而不在於追求個人名利,那麽重複那些論點純屬浪費。所以專注在思考多領域的交集議題,是客觀上對自己時間精力應用的自然優化結果。這裏的例外,是當整個行業聯合起來對外做忽悠,那麽我從道德誠信的角度出發來揭露真相,就是十分有意義的事。另外就是一些孤立的盲點,預先看清事實的人佔極少數,那麽我來提供額外的詳細論述就很重要,例如美國在戰略上對中國的深刻敵意,以及自由主義經濟學的邪惡與謬誤;不過既然有其他内部人士(像是翟東升和陳平)有類似的見解,很快地學術界和媒體界的共識就扭轉過來,然後既然我並不局限於特定專業,自然可以功成身退,另找新的話題。
我堅持不收錢,也是對自己志向最優化的自然決定。在一個偏僻的台灣博客上敲邊鼓,依舊有五個管道可以影響華語世界的輿論:1)是直接面對讀者;2)是有其他意見領袖會想要直接引用我的論點,這裏我當然希望他們注明見解的來源(不是爲了我個人的名利,而是希望博客有更高的知名度和公信力,以方便未來有更大更快的影響力),不過現代媒體本來就習慣互相抄襲,即使是學術專家在不是論文的場合裏也沒有嚴格的法律或倫理責任注明出處,所以從大局著眼,也只能順其自然,再加上有些理論性的話題,我反過來希望有熟悉行内知識的人對我的新見解做補充、演繹,以便建立你所提的“完整體系”、產生更長久的影響,所以自然更不方便過度抱怨;3)在資訊相對不完整的議題上,我的意見和偏向會是許多意見領袖的重要參考,因爲我的論斷有很高的機率會是正確的,那麽一些乍看之下比較激進的選項有了我的背書,會得到新的合理性,其他作者也就敢於高調倡議;4)我的研究態度和方法,尤其是運用邏輯思辨的技巧和優先次序,以及理性討論的基本規則,對其他的意見領袖來説,是很重要的示範和教育;5)高水準的整體敘事,自然對公共論壇的其他評論者造成間接壓力,逼迫他們放棄嘩衆取寵的捷徑,提升尊重事實和邏輯的標準。
前面提到,我希望從我的文章轉述見解的人,既然是免費拿到很有價值的知識,至少應該注明出處;而其用意,不在於追求我個人的名利,只是爲了建立博客的品牌,進一步增大未來的影響力。我在博文和留言也常常自吹自擂,指出以往預測準確兌現的案例,同樣也是基於這個考慮:亦即不是爲了滿足自己的虛榮,而是方便將來傳播正確的意見思想,增大對人類福祉的貢獻。當然這裏的前提,在於這些準確的預言,必須有强烈統計意義而且是真正獨一無二的,否則就淪爲常見的虛僞宣傳了。
其實這個博客最重要的讀者群體,基本從未參與留言討論:他們的身份資歷太高,不會願意冒著被我批評指責的危險來發表意見。我知道他們存在,是通過一些間接跡象得到的結論,尤其是我對一些細節做出意見後被轉述的案例。另外還有一些軼事Anecdote:例如我大姑在紐約皇后區住了50多年,有一位來自同一棟公寓的華裔鄰居後來成爲Stanford的資深教授,幾年前重聚進餐,聊天中這位教授談起《王孟源的博客》,我姑姑很得意地說,那是我侄子寫的。我的二姑自己是台灣的退休教授,和老同事聊天,也有一模一樣的經驗。
正因爲我在乎的只是公益,所以絕對敵視黨同伐異的圈子和門派;希望大家從這個博客帶走的,純粹是正確的思維方法和分析結論,而不是有我無他的山頭心態。我們是山野閑人,自然形成鬆散的好人聯盟是合理的;至於只論親疏不管是非的態度,那屬於政治鬥爭的範疇,只有實際在位的人須要擔心。
\subsection*{2021-02-26 08:02}

你不必羡慕我的兒子:任何人如果從出生開始就日常接觸一件事物,不論它是如何重要或珍貴,都自然習以爲常,不當一回事;我自己也沒有每天感謝空氣、陽光和水。他在除了文學藝術之外的所有議題上有任何疑問,都可以立刻得到正確而深刻的解答,往往遠超教科書的水準,但對他來説,這是爸爸理所當然的職責,而且還會嫌我太囉嗦,牽扯出寫作業不需要的多餘考慮。他真正天天關心的是我的厨藝很差勁。
這並不代表我沒有用心栽培他,只不過技能、知識和智慧受天分和興趣局限,不能强求。我認爲家教的重點,在於儒家“修身齊家治國平天下”的理念和情懷,用英語來表達就是“Be a decent man.”這裏的Decency是從家人做起,外推到全人類的。去年他上大一,主動選修了《社會主義》的課,教授是美國學界少數極左的老學究之一,眼中的蘇聯和中共都遠遠不夠正統。一個學期下來,教授特別關注了他,說他思想純正、見識深刻;這可是在我沒有幫助寫作業的前提下達成的,基本靠多年閑談聊天的耳濡目染,所以我還是很高興的。
\subsection*{2021-02-26 07:53}

(今天很忙,先囘第一段)
雖然從未謀面,大家其實已經有多年的交情,讀者不吝贊美,當然是讓我高興的事,然而這個博客的理性客觀原則和引領思潮的目的不太適合這類天然是主觀和感性的表述。
先談一個我親身經歷的小故事:我聼歌劇30幾年了,一向偏愛意大利作曲家,尤其是Puccini。他的作品中,《蝴蝶夫人》是很知名的,但我最喜愛的Arias卻不是一般人熟知的《Un bel di vedremo》,而是第二幕Cho-Cho San抱出兒子所唱的《Che tua madre Dovra》(參見黃英的電影版,例如https://www.youtube.com/watch?v=3stgof-xyN0,1:21:41處,導演在這段選景特別感人)。這裏她談一個人撫養兒子含辛茹苦,我第一次聼到就爲之淚流滿面。我父親曾經生病,母親照顧全家老小;她雖然出身農家,沒有受過高等教育,但任勞任怨、毫不猶豫為夫為子犧牲奉獻,我小時候只看在眼裏,到長大離家才真正體會到那份恩情,而這首歌總能觸動這些回憶和感情。
兩年前夏天我囘台灣,弟弟特別請假出國旅游,以便把車子留給我用。現在汽車的音響都有Bluetooth,可以簡單地播放手機裏的音樂。那時我父親還未過世,住在醫院長期病房,我們每天早午晚看望三次,經常用車。有一天我忽然想起那首歌,媽媽正坐在我身邊,就放給她聼,心中充滿了孺慕之思,覺得一個可回憶的人生亮點又即將到臨。媽媽轉過頭來,看著我欲言又止,我滿懷期望地問“您覺得怎麽樣?”她又遲疑了一下,終於說“這是什麽鬼叫?”

後來我把這個故事和小孩提了,用意是提醒他主觀的感覺基本無法在人與人之間精確傳遞,而且越是强烈、深刻、真誠、精微的,越是如此。大批群衆之間的共鳴,往往只能建基在粗鄙、謬誤、邪惡的黨同伐異情緒之上(很不幸的,這是演化遺傳的自然結果),成爲族群仇恨和非理性惡行的源頭。
我感謝你對我的尊崇,但這個態度來自你心中把我當作導師的定位。實際上我的目標在於從思想和策略上改善世界、國家和社會,手段則是把環境現實用正確的邏輯理論整理出來,讓其他在乎公益的人理解之後能把力量投注到有效的方向上;過程中同時有端正三觀的效應,只是一個附帶的偶然。所以其他讀者看完同樣的三百篇文章,主觀態度仍然會因人而異,有些人把我當成諍友(我私以爲自己符合“友直,友諒,友多聞”的標準),有些人把我當做怪物,有些人把我當傻子(因爲不求名利),更有人會把我視爲仇敵(例如高能所;他們不見得不會想要偷偷來看關於其他話題的文章)。
我不可能針對這些不同的態度一一做出反應,事實上還必須用心避免任何主觀情緒化的心態。這不但是因爲改善世界的目標,要求絕對的客觀理性,而且對任何一類讀者的過度關照,都是對其他人的忽略。再加上我自己時間精力有限,還是讓我們繼續專注在實際議題之上;王孟源個人的價值和定位這個話題,屬於八卦,剛好和本博客的宗旨相反,請大家不要追述討論。
\subsection*{2021-02-26 07:50}

我已經多次討論過這個現象。針對你的論點,讓我再明確回復一次:大社會中整體的交流管道越是豐富、多元、自由、放任,主流内容的品質就越是低級、虛假、邪惡、愚蠢。這是因爲參與媒體討論的成員,不是理性的Homo economicus,而是自然演化出來的Homo sapiens,繼承了許多演化過程中有益於當時游獵部族基因生存繁衍的本能,這些本能在現代工業社會中卻具有完全負面的價值。例如理性思考先天就是很花時間精力的事,如果不是有絕對必要(像是策劃如何安全地追獵一群長毛象),否則還是節省那些時間和卡路里爲佳;剛好人類是群居動物,連間歇性的動動腦子,都只有領導人物才有需要,一般人出勞力執行分派的任務就行了,這是爲什麽至今大部分群衆還是懶得深刻思考的原因。
那麽出勞力(Grunt work)的一般群衆,在演化過程中被優先選擇的心理趨勢是什麽呢?是對群體的認同感和歸屬感。在舊有家族關係被打破、個人必須與極大數目的陌生人共處交往的城市文明之中,就只能人爲地去追求替代。這反映在經濟和社會結構,是共享信仰和粉絲心態,於是自然會有利用這個心態來榨取利潤的新產業,例如宗教(廣義的,含中醫教和白左教)、偶像藝人和職業球賽;反映在政治,就是黨派互鬥、排斥移民和對外戰爭;反映在公共論壇上,則是對邪惡和愚蠢論點的選擇性共鳴和放大。
但是自從人類社會出現了國家民族之後,人類歷史的演化就又多出了一個層次,亦即國與國、民族與民族、文化與文化、體制與體制之間的競爭。理性的公益最大化成爲這個層級勝出的正確方向,反射性的愚昧本能和自私利己的短視態度,只會導致結黨營私這樣宏觀上的自殺行爲,尤其在工業革命之後,公益和工業效率以及科技研發近乎完全重叠,科學方法和理性態度越加重要。自由主義經濟學說,就是反其道而行,徹底放縱人類(財主)自私(群衆)愚蠢的天性,中國的對手自願沉迷其中,是可遇不可求的機運。
既然人類有邏輯思考的潛能,在國家層級又有堅持理性思維的必要,那麽從公共政策、文化建設和教育系統上多管齊下,遏制群衆的非理性趨勢,貫徹科學態度和方法,就成爲現代國家崛起的先決條件。當然這裏的前提是一個有紀律、有理想、有執行力的公權力組織,而環視全球,還有哪一個政府比中國更符合這個條件呢?所以不必悲觀,與人性相抵觸當然不容易,但國家利益和人類福祉有需要,所以有見識、有理想的人必須加倍努力,即使只有部分成功,也是對全世界的一大貢獻。
\section*{【经济】谈通货膨胀}
\subsection*{2021-03-04 13:21}

有關美式通貨膨脹計算錯誤地專注在一般消費品之上,多年我已經討論過了。周小川在這裏指出的是,即使依照美國經濟界常用的邏輯,房產也與其他金融資產不同,有很高的實用價值,是幾乎所有消費者的購買目標,應該列入通貨膨脹的考慮之中。
其實美國本身的GDP就包含大約9 \% 的住房價值,不但算進CPI(不過一直是有意低估,只算部分房租,不算房價;例如與Case-Shiller指數相比,少了一半多),也包含在GDP Deflater裏面,只不過會被衝淡稀釋罷了。中國的GDP數字故意嚴重低估住房,所以連Deflater都不怎麽包含它的效應。
當然退一步說,房屋市場極爲重要,是生活必需品中最大的一項,應該受到社會主義政策的特別關注。它的漲跌不但可以視爲通貨膨脹的重要成分,而且應該在稅務、貨幣、開發商管理上以强力手段確保價格平穩合理。
中國的問題在於過去20年放任自由開發炒作,不但養肥了許多有官方後臺的地產商,而且衆多中產階級在被搜刮之後,反而成爲市場的人質。這一來,要回歸正確的政策,就會扯動幾億屋主的切身利益,困難程度一下衝上天了。我批評前任主政者的不作爲、不改革、不獨立思考、盲目引進美式經濟治理,指的就是這類的案例。
但是房產的牛市不可能無限持續下去,如果漲得太高必然崩盤,届時全國經濟的損失將比擬美國2008年次貸危機。一般老百姓不懂經濟,眼光只看到自己財產的名義價格,那麽政府只能把房市先穩定下來,然後慢慢等待消費品的通貨膨脹趕上來,這將需要幾十年不説,而且對無屋民衆買不起的問題沒有立即的幫助,但是上兩届留下一個大爛攤子,早點開始收拾比繼續拖下去要好些。
\section*{【金融】【國際】印度的影子銀行}
\subsection*{2021-02-26 22:44}

其實美聯儲自己早已把下一階段的決策標準說明白,而且還怕金融市場沒聽懂,公開高調地反復了幾十次:就是要等美國國内的通貨膨脹率漲到2 \% 以上,而且明顯地即將維持在2 \% 以上,才會開始緊縮銀根。
拿2021年的美國來和1990年的日本相比,有三個重大的差異:1)美國社會有極端的不平等,底層民衆佔多數,其消費基本是剛需;2)美國經濟以消費爲主,日本則偏重製造和外銷;3)美國有美元金融霸權。
所以我認爲美國的零利率政策不會陷入日本式的陷阱,一段時間之後,消費還是會復蘇,而且不見得要很久。這並不是說美國經濟不會像日本那樣摔下懸崖,只不過方式會有所不同,而且必須等美元的國際地位動搖之後才可能發生。這正是爲什麽我已經强調許多年,要對美國軍事外交霸權釜底抽薪、一勞永逸,必須認識到美元既是那個霸權的基礎,也是它的軟肋。
\subsection*{2021-02-17 10:19}

美國聯邦政府今年的三萬億赤字,若沒有美聯儲兜底,利息要冲上天了。會蠢到這個地步的,比一般的野鷄評論還低級,難不成是香港或台灣媒體?Yuck!
違反《讀者須知》規則,禁言一個月。

我原以爲美元是博客的重點話題之一,應該已經解釋得很清楚了,但一連串有人徹底誤解,所以這裏點出誤區何在,對宏觀經濟學不熟的讀者,請自行找閲讀資料補充基本知識。
美元在Nixon任期與黃金脫鈎之後,美聯儲得以針對國内的經濟熱度和就業情況來自由調整銀根,一開始是控制短期收放利率,到本世紀開始量化寬鬆。這個循環的回收階段,一般始於美國GDP加速成長,通貨膨脹壓力浮現,長期國債利率上升,然後美聯儲做出政策因應,美元匯率隨即升值,再然後金融抄手才有機會搜刮世界。但2008年的金融危機,不是普通的周期性經濟衰退,而是美國經濟長期不可逆衰頹的重要步驟,美聯儲一連放水6年,到2014年才敢收手,其後又努力了幾年,也只收回1/3,到了2019年又必須重回放水階段,而且變本加厲,一年就放出接近上次6年的總量。所以2015年中國外匯流失的壓力,其實並不大,後來才能在有内賊的情況下還簡單渡過。
這次的新冠危機,與Trump的倒行逆施互相叠加,應該是足以威脅美元霸權的歷史性關鍵事件。雖然雪崩的確實時間點無法預先確定,但美國金融界自己已經開始擔心,所以在經濟依舊走低的前提下,長期國債利率卻面臨上行壓力,然而美聯儲不可能不為今年的赤字買單,所以利率和匯率的聯係被打斷,不能套用1997年的脚本。
\subsection*{2020-05-10 11:19}

首先,PMI是一個既非客觀也非綫性的指數,它是企業界對經營環境持樂觀態度的百分比,除了環境好壞之外,接受調查的企業願不願意說實話,其實是更重要的變數。這次新冠疫情明顯是無可究責的外加因素,所以印度商界可以很高興地承認生意受到負面影響;其他國家的企業或許仍然有必須說場面話的壓力。此外,5.4雖低,但是說跌幅多少並無實際意義,因為PMI與經濟環境沒有綫性的關係。 
當然,印度經濟的確面臨很大的問題:不但浮腫很像美國,而且金融界還纍積了大幅不良貸款,再加上他們高度依賴來自美國的資金和業務,現在後者已經面臨2-3個季度10 \% 的GDP衰減,印度像是狗尾巴,搖動的幅度會比狗屁股再增加幾倍。 
至於Modi爲了轉移民衆注意力而發動戰爭的機率,反而因爲新冠而極度減小,這是因爲上面已經提過的,疫情本身就是絕佳的藉口,雖然人謀不臧才是主因,但他已經可以簡單卸責,也就無須在國際上冒險。
\subsection*{2018-10-19 16:33}

你是做國際關係的嗎?我注意到中國的國際關係專業,都有特別樂觀的趨勢。
比利時的那個管道,我也曾在公開媒體上看到過。美聯儲或許自己不會注意(美國經濟學者對統計數據有字面上的迷信),但是CIA必然是能追根究底的。
我對中國國内的政策不熟,只看到幾個整頓P2P的例子。如果真如你所説,是系統性的處理,那麽是件極大的好事。
收緊地產是有點晚了,但是當然比不做要好。
美國在金融和貿易上的好牌,說來説去,根本還是美元的國際儲備貨幣地位。但是Trump越是欺凌弱小,中俄要把美元拉下神壇就越容易。
歐盟和英國的問題都是慢性病,即使有了全球性的經濟衰退,要再拖下去並不難。Merkel和May固然有大幾率會在2019年底前下臺,但是她們原本就是和稀泥的專家,換個人還是一樣的。Kosovo這樣的小打小鬧很常見,要搞成真正戰爭的機會並不大。
\section*{【台灣】台灣大選後可行的内政改革}
\subsection*{2021-02-07 00:26}

我覺得中共對資本有足夠的、健康的疑慮,人代會原本就只是給他們發聲的正規平臺(我不是在評論LGBTQ的時候説過,這類正規溝通管道有其價值?),並不代表那些意見會被照單全收。
至於996,是個很大的議題,原因是我在討論假大空那篇文章的結論裏提到的,中國的崛起(主要)是和平的,那麽超人實力的基礎必須是研發生產的效率。所有工作的效率,都取決於科技的層次,所以我們可以簡化議題,專注在高科技產業上。企業用996來榨取科研人員的效率,當然是次優的解,但如果學術貪腐不解決,整體研發效率不提高,那麽996就成爲國際競爭的必要手段。例如大對撞機,如果幾千個博士被浪費到那個無關國運的方向,高能所固然吃香喝辣,剩下還在幹實事的人才當然只能加班加點來彌補。所以解決996問題的前提,是整頓科研學術界。
很多人以爲社會公益不干自己的事,其實剛好相反,社會公益正是對全民生活水平有重要影響的事項,只不過因爲是間接的,所以一般人看不出來罷了。好人不出面,不只是坐看壞人得志,而且會有切身的損失,中間的普羅大衆,更加是被剝削的對象。讀者或許因爲我不怎麽提996,就以爲我不關心這件事,其實它不但和中國的前途緊密相關,對全世界如何縮減貧富差距、容許社會主義進一步發展,也是必須考慮的重要成分。這些都是我日思夜想的議題;只不過因爲高層級的理論探討,對一般人太過抽象,所以在我自己融會貫通、能夠用淺顯的語言來解釋來龍去脈之前,不方便深入討論。
\subsection*{2020-08-01 07:14}

我覺得是很危險的。
100年前美國學術界試圖引進歐洲學者的時候,本身的風氣已經是正派的,沒有假大空的問題。其後除了美方出大錢之外,歐洲自己亂成一團,然後大打出手,才是把金頭腦趕到美國的真正動力。
現在歐美並沒有陷入全面戰亂的可能,中國要吸引國外人才,只能依靠遠為優越的環境。但是我以前已經解釋過了,假大空充斥本身就是一個極端惡劣的環境,錢出得再多也只會招來一丘之貉。
中國到目前還可以承受一個高度腐化的學術界,靠著兩個前提:1)中國還在追趕先進國家,研究的方向是事先就已知而且確定的;2)真正要緊的技術(例如軍工),另有一套國營工業體系來開發,不需要學術界參與。但是中國已經開始超趕歐美,未來的研究不會有前例可循,那麽搞假大空的專家就有了極大的忽悠餘地,大對撞機是一個明顯的案例。與此同時,一個全球領先的文明,必須全方位地引領世界,不能有一大堆短板,否則不只國家無法長期興盛(尤其政治學、經濟學、金融管理、教育理論、社會文化等等,對國家治理水平有決定性的影響),全世界都會被連帶拖累;即使小如中醫教,它在10-20年後中國國際地位上升,也必然會出口到第三世界,殘害更多的人命。
\subsection*{2020-06-25 00:35}

反腐是一件非常非常困難的事,畢竟你所依賴的體系結構内部到處都是問題,不但會有各種積極和消極的抵制和反抗,整個組織都有崩塌的可能。習近平所做的,已經遠遠超過我自己能及的地步,但是他也有極限:過去兩代官員都是在紀律寬鬆的環境下提拔起來的,如果一律追查到底,人人自危,這個政體就不用運作下去了。
所以他不得不退守幾個重點:十八大之後不收手的、公開公款吃喝的、有明顯證據被舉報的。但是這麽一來被抓的有很大的壞運氣成分,那麽處分太嚴就不合理,所以只能判徒刑。當然,目前有許多案例的處罰實在太過輕微,這倒並不是上層有意的,而是在只抓重點的背景下,沒有建立專職的監督復查機制,留給中下級官官相護的空間。這其實在建立E-Government之前,就可以通過擴大監察單位的職權來解決,有了高科技工具之後,可以更加普及和詳細。
至於財產申報,在新一代人員和文化接手之前,牽連過廣,難以貫徹,最終必然流於形式,不如等等。
\subsection*{2020-02-27 16:50}

這是我所擔心的,把E-Government純粹當做監管大衆的新工具;其實正確的重點,是用它來監管政府官員,尤其是中低級的組織。
例如俄國,表面上是用來防止大衆逃稅,實際上的重點是監管國稅局自己人。這裏的差別,在於E-Government的操作細節,是否下放給既有的官僚系統。如果只是把新的資訊和權力交給原有的基層組織,那麽反而更方便他們上下其手,維持政治紀律和效率只會更加困難。所以每一個E-Government系統,它的隱性特徵都是必須架空一兩級官僚(事實上,削減官僚層級是E-Government的終極目標之一),讓高層能直接監督下級的行政。如果只是花錢買科技裝備和軟件,那麽不只是浪費在花俏的浮面上,而且很可能會有反效果。
\subsection*{2020-01-24 10:44}

是的,才志太遠大,被埋沒的機會就更高。三國時代,20年間出了多少英雄豪傑,難道是一代中國人基因忽然都有了突變嗎?只不過是舊有的資源管控體系崩潰,大環境容許能人出頭罷了。諸葛亮若是早生或晚生20年,後世根本就不會知道有這個人。換個角度來看,代代都有諸葛亮,只不過大多數沒有機會一展長才;你只要看看台灣這個體制,就知道爲什麽了。
其他隨機的條件也很多:張益唐如果不是遇上一個容許他自由過活的老婆,而是天天被逼著想方設法升等賺錢,你想他現在能有什麽成就嗎?
我若是早生30年,在物理界的成就大概達不到楊振寧,但是Weinberg級別的研究,並非望塵莫及的。反過來說,楊先生和Weinberg如果也是60後,你覺得他們就能衝破過去40多年高能物理的大沙漠嗎?我認爲兩人的差別只在於楊先生有智慧、有品德,能看出並承認這一點,所以會設法勸阻國家和年輕人陷入這個無底洞,而Weinberg反其道而行,想要浪費國家資源來賭一把罷了。所以現在高能所出來懟楊先生的,不是王貽芳這樣存心賣國求榮的混蛋,就是像當衆質問楊先生的研究生那樣自以爲遠遠强過楊的SB。
\subsection*{2020-01-22 15:55}

所謂的“機會”,局部性的就是“資源”(亦即“道天地將法”中的“地”,全局性的機會是“天”),而資源和能力(“將”)的對立,正是許多重要社會議題的根本。
我們討論貧富差距、階級固化、濫用特權等等問題,其實基本上就是要排除獨霸資源以尋租的現象,容許有能力的人辦實事以推進整體福利。資本主義市場經濟有天然的擴大貧富差距的趨勢,正來自資源的稀缺性和重要性。BNP的那位首席交易員,並不是不知道我所貢獻的算法是公司盈利的關鍵,但是他認爲把我開除之後,他就獨占所有既有的資源,那麽我再怎麽有能力,要複製同樣的系統所需的大量資源也很難從陌生人獲得,他勝利的機會是99.9 \% 。雖然在這個例子裏,我撞上了0.1 \% 的好運,但是人生不是電影,運氣在現實裏是可一不可再的事,下一次我的境遇就回歸機率分佈的平均了。
資源的關鍵性,是客觀的事實。控制資源本身並不邪惡,反而是創造社會新福利的必要元素;邪惡的是獨霸資源來尋租,因爲它打擊新生的、更高效的競爭對手,以損害社會整體為代價來謀求小集團利益的最大化。我們討論不同的政治體制,正是要找尋能鼓勵把資源分配盡量做到有利於全社會的制度。
美國商學院的全套學説,其實濃縮到核心,就是教人怎麽爭奪然後獨占資源來尋租。這是他們對美國文明腐化的貢獻。
美國在雷根之後的政治、外交、經濟、宣傳體系由富豪掌控,基本上就是把商學院那一套應用到全球治理。商業文化不應該被推廣到文明社會的所有層次和方向,是這裏的基本教訓。
中共的人事體制雖然絕非完美,但是相對於美國和台灣來説,有一個很大的優越性,就是它並不主要依據資源獨占的程度來提拔人才,因爲每個地方官都有同一級別的資源,成就上的差別就能反映出個人的能力。
中國商界的成功人士,如同美國的一樣(如Musk),絕大部分也是撞上或騙到關鍵性的資源;其實真正辦事的,是大衆所不知的幹部(如Musk手下的火箭專家)。企業家對公司和經濟的真正貢獻,是識別並扶持有能力的幹才,把稀缺的資源做最高效的再分佈,這是市場經濟的正常運作,沒有什麽不好。學術界也有類似的現象;在找到更好的資源分配體系(例如E-Government)之前,沒有必要過度抨擊。
\subsection*{2020-01-22 07:57}

這個問題很多人問過了,我再回答一次:
我自己很喜歡在美國閑雲野鶴的退休生活,既沒有責任時限,也不必交際應酬。囘台灣是爲了照顧雙親,其他的事都是次要的。如果有別的地區需要我的服務,第一就必須先解決我父母的安頓問題。
其次以我的美國公民身份,就是光棍一個,大概也不會有人要。
1999年我被BNP開除、到處找工作的時候,一開口就要2000萬美元和20個人頭,基本上沒有介紹人認爲我能找到識貨的東家。結果遇到UBS願意出錢,其實它在當時有非常特殊的内情和考慮,我以後有空再解釋。後來就沒有這樣的好運了;沒有了足夠的支援和授權,巧婦也難爲無米之炊。你要理解,大家都是從自己的人生經驗裏學習,面對新的人或事,我們自然會以過往經驗的平均值來做基本假設;一個人或一件事太不尋常,就不能指望別人曾經有過多次類似的遭遇,也就不可能得到針對性的待遇。很多年輕讀者奇怪爲什麽我做貌似悲觀的人生規劃,其實我一輩子的經歷都是如此。現代社會先天就是優待騙子而不是能人。我小時候還天真,到博士班時代,整個行業裏都是自欺欺人的超弦騙子,還個個自大得要死,我很快就學會逆來順受。如果BNP那個首席交易員不是欺人太甚,已經拿了我的Algorithm去賺50倍於我的收入,還不給我公平發展的機會,我或許就會留下去,因爲五年的共事,讓周邊的聰明人都承認你遠遠更強,值得組織投入資源,這不是可以輕易取代或重建的環境條件。
\section*{【反导】中共再次成功进行陆基反导试验}
\subsection*{2021-02-06 03:21}

1.這次的公關稿,暗示是全球領先或至少先進,而且是完成幾十年來的大目標,席亞洲認爲指的是成功擊落更遠程的彈道導彈,可能是正解。
2.從微觀來看,反導的效費比並不高,即使等到未來技術完全成熟,也必須至少兩發打一發才能有把握成功。然而反導系統所用的導彈(尤其是比較理想的中段反導)和目標導彈基本是同一級別的東西,這還沒有算入額外所需的衛星和地面監控系統。所以反導系統並不能取代擴充核武庫的嚇阻力量。反導的意義,在於重點保護核心資產,使其能生存過第一擊,從而提升戰略反擊的整體效率。一個有效的戰略嚇阻,必須兼有先進的反導和足量的核武庫。
3.核爆是很一個很挑剔、很脆弱的過程,如果不是照著設計意圖完美起爆,即使勉强發生連鎖反應,也會比正常爆炸低好幾個數量級。因爲一般核彈爆炸後的重要放射性污染源是被中子照射引發原子核嬗變的產物,所以污染性隨實際爆炸當量而增減(選用材料和結構設計的影響可以更大,但它們是獨立變數,這裏忽略)。被反導系統成功攔截的戰略導彈,放射性污染主要來自那幾公斤的Plutonium裂變燃料(現代一級大國的原子彈基本上不用Uranium了),一對一來比較,遠低於1950年代核試爆的後果。如果是中段反導,還發生在大氣層之外,污染由全球均分,影響更低。
\section*{【科研】統計與謊言}
\subsection*{2020-09-21 18:58}

中方對外的戰略思路沒有問題,但是戰術上老是慢半拍,以致極爲被動。
例如最近有報導,說劉鶴在處理中美貿易戰之時,指導原則是無視短期經貿損失,專注在限制反擊以避免衝突升級之上。假設這是準確的描述,那麽就是(包含劉鶴在内的)領導核心的正確戰略優先順序被智庫和學術界的錯誤情勢判斷所誤導,終而弄巧成拙。美方的仇中,來自霸權利益的基本矛盾,根本無法調解;一旦美國Deep State的宣傳機器動員起來(起自2009年),更加無可逆轉。所以在戰略態勢上,衝突敵對在十多年前就注定了,不是中方近年的任何作爲(除非自我解體)能夠“避免”的。在戰術上,Trump並不在乎美國的長期戰略利益和計劃,他只不過是利用既有的仇中民意,試圖取得自我宣傳的亮點;所以他不會因爲霸權衝突的考慮而堅持與中國做生死搏鬥,而只是在全世界挑軟柿子吃。中方兩年多前畏首畏尾、自我設限,反而鼓勵他食髓知味、變本加厲,不斷回頭另找新藉口、升級貿易戰。這是我早就反復預言警告過的,可惜14億人的中國,養了那許多智庫,居然個個都本末倒置,以爲可以由國家忍辱、花人民的錢來消災,給予決策單位完全錯誤的建議,真是讓人嘆爲觀止。
這次對歐外交的機會窗口,來自Trump政權四年來全力推倒既有國際結構的成果,並不是中方的對外宣傳有了什麽長進。像是德國這種白左文化根深蒂固、美國代理人主導媒體、智庫和情報界的社會,指望他們對中國產生主觀好感是緣木求魚。我一再强調,中方只能寄望他們頂尖的政經精英,從客觀形勢和利害關係出發,選擇在中美之間中立。這種事顯然是只能做不能説的;那麽高調的友好親善訪問自然是事與願違、適得其反。正確的手段,是私下獲得不成文的共識、談判簽署專業協定,例如我反復建議的對歐系銀行開放金融投資,並且共同扶持歐元取代美元。像是打倒美元的意圖,很可能已經超出德國人的胃口,所以必須先説服法國,讓Macron來主導。這些都是很簡單的道理,但是中國的智庫又一次辜負了國家民族,真是讓人扼腕嘆息。
\subsection*{2020-01-11 07:55}

我最近讀了一篇方舟子的訪談記錄(http://www.xys.org/xys/netters/Fang-Zhouzi/interview/uwashington.txt ),覺得他誤入歧途得非常嚴重。他一開始打假用意是好的,但是不分輕重緩急,時間久了,成爲網紅,更加信口開河,隨性褒貶。我以前就説過,我不想成爲網紅名嘴,正是因爲如果要討好大衆讀者,必然會使討論的品質下降;更何況方舟子的專業知識層面,並不是特別廣汎,偏偏他又沒有自知之明,不懂的事也硬要插嘴;對客觀事實和理性邏輯又沒有絕對的堅持,往往選個立場然後走極端。例如轉基因這事,雖然目前所有被批准的轉基因作物都沒有發生安全問題,但這並不代表未來不會有問題;尤其美國財閥公開以人命換利潤,再碰上假大空盛行的學術界,轉基因作物絕對是必須由政府嚴格監管,然後謹慎、緩慢推進的。上個月剛剛公佈,一個實驗性的轉基因乳牛出了問題,就是一個警示訊號(https://hoards.com/article-26865-setback-slows-pathway-for-gene-edited-dairy-cows.html)。
其實假大空,中外皆然;正因爲這問題在中國學術界比較嚴重,更應該只挑最離譜的來專注打擊。如果方舟子真的在乎打假,就應該專職來做,而且絕對小心嚴謹,寧缺勿濫。結果他反其道而行,隨便哪個教授被舉報給他,他就出面斥駡;我在四年前開始談大對撞機,這明明是遠超他專業能力的話題,他卻也插嘴,稱我為“科妄”。他這句話不只是可笑,而且是在最關鍵的假大空問題上幫倒忙。
假大空對國家社會的損害,是可以估計的:一個靠假造論文成名的院士,所造成的損害,大致是幾百萬美元;一個沒有意義的研究計劃,一般是浪費幾千萬美元;但是大對撞機是千億美元級別的坑。就算方舟子能成功打下幾十個假院士,他的正面貢獻也只是一億美元級別;但是他隨口在大對撞機這事上攪和,潛在的負面危害是千倍。很明顯的,除了虛名之外,他的實際貢獻是負值。
大對撞機成爲一個話題之後,我就一直堅持以其為批評的核心重點;討論其他假大空,都是媒體先過度炒作,我才會置喙,其目的一方面是嚇阻公關吹噓,另一方面是爲了讓讀者明白物理界的風氣敗壞,方便大家接受整個高能所都在賣國的事實。
醋醋對國家已經有了很大的貢獻,我希望他以方舟子的例子為警惕,繼續專注在學術界最大的危害之上,保持嚴謹的態度,對事實細節追求100 \% 的正確。正因爲端正中國的學術風氣,還有很長的路要走,我們這些良心人,必須脚踏實地,一步一個脚印,不要被知名度衝昏了頭。
\subsection*{2020-01-11 05:52}

是的,統計造假有4類常見的手段:直接假造數據、排除不方便的資料、利用既有噪音、呈獻無實際意義的結果。正文中其實都涵蓋了,只是沒有詳細解釋。 
美國的企業開始以人命換取利潤,正是我做文中那個研究的初衷。這種現象在開發中國家很普遍,在先進國家卻是罕見的;又一次印證了美國的衰落。 
王貽芳在過去四年,已經拿了兩次超過百萬美元的大獎:頭一次是美國的基礎物理突破獎,Witten是評審,第二次可能是跟風吧。基礎物理突破獎是Witten的富豪粉絲設立的,除了偶爾給天文物理(還有一次給了凝態物理)之外,高能物理的部分一直是Witten的熟人輪流拿,唯一的例外就是2016年特別給中微子物理,這才包括了王所長。但是Witten有不到十個知名的學生(我認識其中兩個),大部分還沒有拿到獎/錢。你想他對王貽芳如此關愛,後者難道不是感激涕零嗎?我以前解釋過,高能理論要活得滋潤,必須有像大對撞機這樣超級費錢的項目,才能巧立名目,雨露均沾。王所長和丘成桐已經反復說過,大對撞機會吸引成千上萬的國際物理專家到中國;你想想,這些“物理專家”除了來賣儀器的實驗學家之外,有多少會是做超弦的?所以高能所或許憑著中微子實驗和新對撞機原理探討就能在錢堆裏游泳,但是超弦界卻是非要有大對撞機不可的。
\section*{【國際】【戰略】回顧洞朗事件}
\subsection*{2020-09-12 16:33}

這幾年下來,你的邏輯思辨反而有所倒退。有空應該自我反省一下,是否有效地運用日常閲讀和思索時間。尤其是博客這裏,你真的用心仔細讀懂了嗎?
討論中印關係,和“干涉別國内政”八竿子打不到一塊兒。這裏中美的真正差別,在於中方願意互利共贏,而美國一直是或暗搞、或明説的“America First”,所以對實力上升的大國必然出手打擊。我們面對的議題,不是中國也要學美國玩有我無人的霸權思維,而是對像印度這樣仇中態度深入骨髓的非理性國家,中國是否有必要主動犧牲國家利益、奉獻公共資產,損己利人。
你説亞投行大幅資助印度,是千金買馬骨;但馬骨只是欠缺正面價值,資助印度卻對應著負面價值,而且不只對中國是負面,對其他第三世界國家,亦即買馬骨的觀衆,也是負面的,因爲錢多給了印度,其他國家自然就少拿。那麽這個類比還恰當嗎?即使你硬拗它不是比喻不倫,那麽這個昂貴宣傳給予其他國家的教訓是什麽?中國是鄉愿,越是敵對、越有糖吃。這是適合“爭奪世界領導地位”的廣告嗎?“中印都開戰了,亞投行照樣給印度貸款”,那麽這些國家和中國親善有什麽意義?
亞投行必須獨立,原本就純粹是歐美防範中國增進國際影響力所下的絆子,沒有任何真正的邏輯性和合理性。資本主義一貫强調私有產權至高無上,怎麽中國人民的血汗錢就應該拿出來白送給仇敵?這和40多年前忽悠蘇聯自我閹割的那套宣傳如出一轍,結果Gorbachev把國家賣了,只拿到幾個獎牌;殷鑒不遠,居然還有中國人願意上當,真是愚不可及。
\subsection*{2020-09-12 00:27}

印度的發展模式,基於英美財閥資本主義的教科書,在光鮮的芝加哥學派理論遮掩之下,是極端嚴重的官商勾結和貧富分化。在對實體工業發展真正重要的國民教育水準和物流基礎設施上,並沒有實際上的比較優勢。
近年來印度欣欣向榮的表象,其實始於美國遏制中國的戰略需要,因此發動在國際上的宣傳優勢(這是我反復提過,美式霸權的三大支柱之一),推銷全球資本進入印度市場,一旦有足夠的企業界信衆,被推高的成長率自然吸引更多的投資,成爲一個self-fulfilling prophecy。
在這樣的背景下,中國的私有企業要下場和歐美資本競逐印度消費市場,完全沒有問題,反正他們自負盈虧。但是國資企業,尤其是基建方面,内含國家多年來的資助支持,原本的目的在於增進中國自身的社會公益,那麽就絕對沒有道理去摻和一個深度仇中國家的社會建設;否則即使有點賬面上的盈利,算入隱性成分和代價,依舊是大大的賠本生意。中國沒有必要去遏制印度,但那和用國家補助去資敵是兩回事;說“中國無力左右印度經濟發展”的人,玩弄的是典型的狡辯術,我的讀者如果看不出來,只能是不用心。
至於要捆綁雙方利益來改進關係,這必須假設對方是完全理性,並且在乎的純粹是本身發展,而不是相對優勢;這兩個假設都根本不成立。只有完全不懂邏輯的人才會犯這樣的錯誤;爲什麽完全不懂邏輯的人能在中國學術界和思想界有影響力,是一個值得深究的問題。
\subsection*{2020-06-18 14:15}

1. NDEA的用處在於研究小國;大國不用政府補助,也會有學生去學他們的語言文化。此外,這些研究外國的專家頂多也就是一般大學教授的水準,在當地常住過,可以為國家提供知識細節、避免低級錯誤,並不代表他們能有突創性的大局觀。中國的崛起,是現代人類社會的奇跡之一,不但組織力强大,而且最高層堅持以實踐來檢驗真理,基本對美式宣傳免疫。美國人自己先忽悠了自己,自然無法想象有個開發中國家能在思想做出如此大幅度的超越。
2. 紅脖子沒有人類應有的基本思維能力,但是歐美還是有理性的意見領袖,中方可以用堂堂正正的態度來和他們講理,同時也避免被歐盟的外交精英抓住把柄、坐實抹黑中國的指控。印度全國根本就沒有做理性討論的傳統,也沒有人在乎實據,反正有需要就編造出印版的“現實”,所以我說可以戳穿對手的小九九,是純粹針對Modi個人的心理戰,和對歐美的宣傳戰完全不是一回事。
\subsection*{2018-12-15 23:28}

兩者是有本質上的差異的。 
印度和中國抗衡,純屬做夢,所指望的也只是面子;而從中國的戰略考慮來看,它的優先順序很低,沒有什麽大搞的理由,就像街對面住了一個瘋子,對著你吼叫,哄一哄、笑一笑就算了,沒什麽好在乎的。所以一開始中方的姿態太高太硬,是失策的。 
孟晚舟事件剛好相反,這是美國内部達成遏制中國的共識(大家必須注意,這個事件的挑動者,不是Trump和他的團隊,而是正在調查Trump通俄門的司法部建制派勢力)之後的一個打擊步驟,是未來20年中國繼續興起的最大障礙,如果有任何退讓,美國必然得寸進尺,美國周邊的附庸如澳洲、日本也會更加熱情地與之合作,所以中國絕對不能容許他們立下前例。美國的最終目的,是消滅和肢解挑戰它全球霸主地位的任何國家;這樣的威脅,不可能有真正的妥協,中方只有一個選項,就是持續鬥爭,堅持到底。
\section*{【美國】美國大選中的危險人物}
\subsection*{2020-09-07 23:10}

華裔有什麽優勢?法律、慣例、文化、成見,樣樣都是公開或半公開歧視華人,所謂“Diversity”的藉口可笑之極,佔世界人口一半的亞洲,只夠格反映出3億人口美國文化的1/6?楊安澤這種明顯陪跑的候選人,又哪有什麽大局可以損害?出來競選只有兩種考慮:第一是打自己的知名度,以便將來升官發財,第二是為了理想,用這個機會教育民衆、倡議改革。他要是對華裔的利益有一絲的在乎,至少出面説一句:華裔的成功在於比別人努力,用名額限制公平競爭、剝奪權利和機會,可能反而是種族歧視。光是跟在Sanders後面炒同一個類別的福利開支,除了為自己在白左媒體財團找工作,又有什麽作用?將來他頂多入閣;華裔幹到部長級又不是沒有前例,低著頭當象徵性的卒子,反而讓白種人更加名正言順、心安理得地欺壓華裔百姓,這不是踩在同胞的頭上求富貴是什麽?
限制大額政治獻金,楊是排名大概第一億的鼓吹者,這除了討好白左、圖利自己之外,有什麽實際貢獻?白左媒體願意談這個議題,第一是因爲空談擺姿態不費一文錢,第二是如果太陽從西邊出來,居然能改,那麽財閥要左右選舉得靠什麽?不就是要重新依靠這些主流媒體嗎?這樣就能從金錢政治解脫出來?
唉,我的回答或許很簡短,但都是深思熟慮過的結果,請你也先把正反兩方論據都考慮之後再來爭辯。
\subsection*{2020-04-12 17:11}

我同意Trump對連任有以往總統都沒有過的强烈危機感,然而這並不代表民主黨和其他政敵會真的去追究。南韓和台灣學習的是1970年代之後,美國白左紙面上的民主政治理念,其中的總統犯法與庶民同罪的論點,其實是純想象的產物,從來沒有在美國實行過。美國從立國開始,總統就是選出來的國王,而且是17、18世紀朕即國家的那一種。憲法裏講三權分立,總統所領導的行政權既然獨立於立法和司法之外,理所當然不必理會法律;我本周才提過的Trump用Signing Statement來禁止督察長遵照法條要求對國會作報告,就是很明顯的例子。
南韓、台灣和其他國家(如菲律賓)有誤解,來自白左有意無意扭曲對水門事件的解讀:他們說Nixon被罷免是法治體制的必然後果,後來被特赦是Ford代總統個人的特別決定。其實剛好相反,Nixon成爲美國歷史上唯一辭職的總統,是因爲當時老百姓還普遍相信主流媒體,他在被攻訐了五年之後,根本沒有像Trump那樣鞏固的基本盤,所以在政治上壓不住國會議員的罷免運動。至於特赦,反而才是體制下必然的結果,不管誰接任都必須給他特赦;事實上衆所周知,Ford是他在黨内的主要政敵之一。
至於共和黨想方設法來遏制窮人投票,這也是歷史悠久的事,只不過一百年前KKK和紅脖子支持的是民主黨,所以出手的不是同一批人。你覺得最近幾年美國壓制投票變本加厲,倒不是Trump本人直接所爲,而是因爲他指派了新的法官,使共和黨在最高法院和多數聯邦上訴法庭取得了壓倒性的優勢,所以各州選舉法可以隨便搞,不必擔心被糾正。
\subsection*{2020-04-09 09:43}

正是因爲Trump自我矛盾是常態,所以聽衆可以選擇相信只有自己喜歡聼的話才是他真心的,反面的話是用來騙敵對方的。當然這樣的聽衆必須很蠢才行,所以Trump參選之後,很快成了右翼民粹的領袖;十年前他的政見還是溫和/建制派的。
不管現在的民調怎麽說,大選日還有七個月,這是非常足夠的操弄空間。Sanders棄選還不到24小時,ABC就搞出新聞,說去年11月美國軍方的情報單位就經由竊聽中方的通訊,得知武漢有疫情,而且上報了可能威脅全球的嚴重警告。這很明顯地與許多已知事實相抵觸,包括中方會在11月就有針對疫情的内部官方討論,而且還已經對病毒做出足夠的研究來瞭解它可能會威脅全球。事實上到一月中,中方才確定新冠會人傳人;到二月中,全世界還沒人預料到疫情會在三月爆發到那個程度;別忘了,歐美政府過於托大,是疫情慘重的主要原因,而這不是能理性預測的。
但是這個謠言的妙處,在於它被栽贓在美國情報單位;不論個人政見立場,連美國紅脖子都不會相信情報單位的公開否認。出面造謠的不是民主黨官方,而是一個主流媒體:ABC,它是白左陣營中的溫和派,對中間甚至若干右翼選民有最強的公信力。而且這個謠言的内容,並不為中國開脫,而是直接把疫情的責任重新丟回給Trump團隊。昨天曝光了Navarro一月的備忘錄,内容據説也是預測疫情會全球化;那應該是真實的,只是被誇大扭曲:公共衛生和防疫,不是Navarro的專業和興趣所在,他很可能只是想恐嚇威逼Trump來對中國禁飛,事實上也成功了。今天一下又出現更嚴重的(但顯然是假的)指控,當然目前不能排除純屬巧合的可能性,但是它也剛好完全符合建制派Deep State的Modus Operandi。
我覺得最可能的脚本,是建制派看到Trump甩鍋中國的動作很成功,這不但威脅到他們自己人的選情和利益,而且即使Biden最終選上了,也可能因爲民意沸騰而被迫做出和中國兩敗俱傷的蠢事,所以就為外交部示範了一下搞陰謀論的正確方式(重點是不能由正角自己出馬),一舉扭轉了疫情責任論戰的局面。這並不代表美國人會忘記對中方追責,只不過是轉移了中國在此事上的部分美國民意壓力,把它重新放到Trump的頭上。
\subsection*{2020-03-04 13:10}

不是“遲投票選民“,而是“遲決定選民”,亦即到最後幾天才決定投給誰的。這群人正是對政壇形勢沒有什麽研究理解的所謂溫和派,是英美幕後財閥能左右選舉結果的中堅力量。
美國的社會主義青年,連在左翼的45 \% 人口中,都達不到半數,民主黨自然會繼續留在建制派手中。這也示範了爲什麽西方民主制要做政策修正,都是不可能的事,更別提體制革新了。這是因爲在大衆傳媒完全滲透社會每一個角落的今日,不論事態多麽明顯嚴重,要讓多數選民對提倡改革的“激進派”產生疑慮依舊極爲容易。還是那句老話:群衆是愚蠢的;現代英美的政治體制、社會規範、教育内容、意識形態、媒體宣傳,無一不是爲了確保集體智商向最低的成員看齊而設計的。
\subsection*{2020-03-02 11:56}

Biden直奔南卡是因爲他的民調崩盤,唯一剩餘的號召力來自他是Obama的副總統,而只有黑人才吃這一套。南卡是民主黨初選時間表上的第一個黑人州。
建制派當然也希望Bloomberg早早退選,但是他有本錢任性到底,所以什麽時候知難而退很難説,不過至少應該會先等著看超級星期二的結果有多慘吧。
Trump是The Devil We Know,而且花招已經出盡了,所以大家都會有他連任更有利的想法。其實中國政府在貿易戰停火協定做了很多讓步,給我的印象是他們預期Trump有可能落選。這倒不是完全沒有道理的。
我個人覺得Sanders以工會的利益優先,不一定會想要加入TPP。他對台灣這樣的美式民主政權有原則上的支持,但中方原本就沒計劃在2024年前動手。Obama是認同建制派的人(不過是半路加入的僕從軍,不像Buttigieg是正規軍親信),他的政策最終是為精英大佬們服務,所以必須消耗國力來維持霸權。我不認為Sanders會浪費資源在霸權上,他應該會大幅削減軍費,並裁撤海外基地,這對世界和平是好事。
\subsection*{2020-02-27 20:33}

比你說的,還要複雜一些:必須先考慮另一個維度,也就是草根相對於(政商)精英,然後前者再分爲“溫和派”和“激進派”。所謂的“溫和派”或者“中間派”,其實就是願意聽從精英掌控的媒體宣傳的一般群衆,而“激進派”則看出大衆傳媒是假新聞,所以依賴網絡上的新信息通道。當然網絡上撒謊更加無節制,所以才會產生茶黨。左派現在還有些理性,但是像是仇中這種無分左右、全面洗腦已經大功告成的議題,也不能指望他們能跳出那個大坑。
2016年的大選,是現代美國政治的分水嶺,正在於這一點:Hilary對決Trump,38家全國性大衆媒體有36家支持Hilary;在以往這樣一面倒的宣傳優勢,必然會產生精英所希望的選舉結果,但是實際後果大家都知道的。
至於這次Sanders如果被做掉了,那麽Trump連任的機會自然大增。民主黨草根再怎麽不滿,也沒有用,因爲他們佔人口總數不到一半,而美國警察動手維穩的實力,其實是世界第一的。
\subsection*{2020-02-25 16:51}

美國右翼民粹(例如茶黨)興起至今已經有幾十年了,其中有很强的紅脖子反智成分,於是郊區長大的新一代年輕人慢慢培養出一股反動力量。在2011占領華爾街的時候,還是老左派和城裏的激進白左的合作;到了2016年,分佈在全國各地城市圈的年輕社會主義者已經開始集中力量支持Sanders(我兒子也當了志願義工),甚至Hilary都受到威脅;到今年已經成爲民主黨最大的草根力量,所以我才會預期Sanders在初選中獲得最高票(不是獲勝,因爲建制派很可能會耍賴)。
至於Sanders對陣Trump,很難說哪一個更偏離美國社會的中間;事實上美國的政壇,已經不再有什麽中間了(這也是爲什麽彈劾案根本就是浪費時間),不論誰出面,佔45 \% 的共和黨人絕不會投給民主黨,佔另外45 \% 的民主黨人絕不會投給共和黨,與其去爭取那10 \% 原本腦袋就飄在雲端的“Others”,其實勝選的關鍵在於把自家選民的出動率最大化,所以所謂Bloomberg才選得上,完全是外行人的想法。
\subsection*{2020-02-25 04:59}

他能否推動政策,要看國會選舉的結果:如果民主黨在他領導下大勝,連參議院都重新掌控,他基本有兩年時間可以爲所欲爲。
至於政策是否切實際,那是另一個話題了。我覺得全民醫保反倒不是問題。Medicare的體制、組織、規則都是現成的,砍掉私營健保公司和“Benefit Manager”(近年新增的又一層中間商),可以節省每年萬億美元級別的浪費。這裏越是激進、越不容易出錯,ObamaCare正是因爲求妥協,結果增加開支而沒有減少中間剝削,在經濟上本來就説不過去;所以Sanders反而是改革醫保的最佳人選。
他的其他政策,問題比較大。例如豁免學生貸款,其實大宗是職業訓練的研究所(即法、商、醫學院)以及營利性大學,前者主要是上中產階級的學生,拿稅金來補助他們並不合理;後者的根源是傳銷式的野鷄大學,原本就是爲了詐騙學生貸款而設立的生意,豁免貸款最終反而是補助騙子。我認爲這裏就不應該激進,合理的做法是對一般學生以還清貸款爲條件豁免利息,對營利性大學則必須至少取消它們借貸的資格,然後考慮是否追繳它們的利潤,最後才部分減免受騙學生所欠的本金。
外交更是他的弱項:不止是過度相信白左的那一套,以致很難采納務實的政策,而且也可能成爲幕後大佬們阻撓他連任的手段。上一次美國有一個這麽天真的理想主義總統,是Jimmy Carter。他在競選連任期間,不但伊朗政變沒有被CIA預防,解救人質的軍事行動都莫名其妙地被自己人搞砸了,結果剛好方便財閥支持的Reagan上臺。我一直有疑心財閥的運氣爲什麽這麽好,不過沒有找到正面證據(然而也不像新冠病毒是人造那樣有很强的負面證據),所以不能做論斷。但是如果Sanders總統也在2024年遇到一連串的外交、軍事挫折,或許就可以說不一定純屬巧合。
\section*{【能源】【經濟】2030年左右}
\subsection*{2020-08-02 15:41}

這的確是人類在解決氣候變化這一大問題的當下,要提高電能來源中光伏和風能所占比率的主要難關;所以我一直在不斷更新這方面的消息。
不過先糾正你論述中的小技術錯誤:首先如果忽略傳輸和儲能上的費用,光伏和風能在很多地區已經比煤電便宜,尤其是前者;其次,煤電和核電不只是“能夠”24小時運轉,事實上是須要24小時運轉,否則每次停機都會有很大的效率損失。這也是爲什麽當前適合Load Following的天然氣電廠往往和光伏配對建設,以滿足傍晚的用電尖峰時段。
歐美因爲被他們白左政治正確的扭曲思維所誤導,連這種純粹是工業技術上的議題都走上歧途,去追求低效、困難、昂貴的氫經濟;目前只有中國一家對全釩電池的大型化、產業化在做持續投資。這當然又是西方主動送上門來的機遇,不過中國仍舊只把液流電池視爲衆多備份選項之一,沒有當作重點來攻關,在研發時間上不夠急迫,在明眼人看來有點可惜。
\subsection*{2020-08-01 15:16}

你是想要逼我做個六年後回顧嗎?其實你自己就可以做,但只引用一行數據是不夠的,而且零碳能源佔年度裝機比例不是一個好的總結參考。
正文裏主要對核電和火電做取捨,這兩者都是基本負載(Base Load),裝機量比和發電量比基本一致。但是光伏和風電不一樣,它們必須長途傳輸,建設極高壓輸電網路消耗很大的財力和時間,再加上不能24小時運作,所以利用率遠低於其他能源。換句話説,不能看裝機量,必須看實際發電量。
所以2019年,火電佔總裝機量59 \% ,佔總發電量卻是69 \% ;而在2013年,火電佔發電量比是78 \% 。反過來看,零碳能源佔發電量比,從2013年的22 \% 增長到2019年的31 \% ,正文裏估算2030年的目標是60 \% ,那麽很明顯地,仍然有待進一步努力。不過正如我在六年前預言的,美國人大選過後,自然食言而肥,這些承諾和目標,完全只是中方對自我的要求。
\section*{【二炮】印度的烈火系列弹道飞弹}
\subsection*{2020-07-24 07:44}

印度的航天部門一直是DRDO唯一有點真正實力的;不過即使在這方面,他們的民用液態燃料火箭仍然强過軍用彈道飛彈。
印度的Mars Orbiter是他們航天部門最偉大的成功項目;然而衛星本身非常弱小簡陋,沒有什麽實際科學意義,所用的火箭推力嚴重不足,必須犧牲時間換取最高的能量轉換效率。整個計劃是一個形象工程,爲的純粹是那一句“成功探測火星”,雖然這次賭贏了,但並不代表整體水平達到第一梯隊(美、俄、歐、中、日)。
至於發射Agni導彈到澳洲附近,應該指的是Agni V;它的射程是5000公里,還不如朝鮮。印度在2018年年底完成第七次試射,然後宣佈正式部署。這並不代表它具有真正的實用性:雖然DRDO為導彈加了一個儲存筒,發射車依舊是印度獨有的長板拖車。
\section*{【经济】世界经济未来走向}
\subsection*{2020-07-23 16:00}

有關未來幾年的GDP成長率,的確是會受全球經濟環境的拖累,不過從長期大戰略的角度來看,重要的是中國相對於競爭對手的表現。對方承受100 \% 的下行壓力,如果中方只感覺到1/3,那麽固然在成長率的絕對值上有負面影響,但是相對來説,中國的崛起反而加速了2/3。
我認爲本届政府,對2009年那一波不定向内需刺激的負面作用已經有深刻的認識和反省,不會重新大幅舉債來吹大泡沫,而是改爲針對有長期效益的工業和基建方面做投資。中國經濟管理的重要考慮之一是維持穩定合理的就業率;剛好生育率降低所帶來的人口曲綫變化已經開始浮現,短期内反而對全球經濟減緩背景下的就業問題有幫助。
至於彎道超車,因爲既有的行業領袖在技術、人員、收支、銷售網絡等方面有很大的優勢,一般是在工業技術或商業模式有換代變革的背景下,後來者才有超趕的機會。例如目前汽車業正經歷從内燃機改爲電動的轉換,這是百年一次的技術過渡,中國必須抓穩這個難得良機。芯片則相反,5-10年内既有的技術路綫還不會撞墻,那麽直接挑戰臺積電的勝算就頗爲渺茫。大飛機比較特別一點:這裏是行業霸主波音的自我毀滅創造了新興廠家的生存空間;而且後新冠世界的長途旅行需求大減,使單通道機型的重要性進一步增加,剛好適合商飛的產品陣容,就連時機也恰到好處,C-919交機拖延反而避開了航空公司停飛的時段。
美國的國力衰退和歐洲的發展停滯,仍然會不斷提供中國升級、替代和超越的機會。如果未來五年歐美經濟是零成長,那麽中國接受5 \% 的年成長率並無困難。事實上中國發展模式的一大毛病,在於努力有餘、思考不足,沒有選好正確的方面,空耗了許多資源,有些地方政策甚至是花大錢砸國家的脚,例如貴州和高通的幾次合資。如果能放下對維持高速成長率的執拗,事先多做一些詳細的評估和整體的協作,未嘗不是件好事。
\subsection*{2020-07-23 10:30}

中美歐三大集團近年的發展局勢,我一直不斷在追蹤討論;你所説的是合理的總結。
印度的起點太低,即使只以很低的效率搞基建和工業化,也還有很大的向上發展空間。越南則對中共政策亦步亦趨,在現階段沒有走錯岔路的危險。既然這兩個國家不可能接受和平親善的橄欖枝,中方應該立刻停止資助他們產業升級,反過來扶持臨近的競爭對手,例如巴基斯坦、柬埔寨和泰國。不過不要對他們寄望太高;這些國家並不處於東亞儒家文化圈内,沒有適合現代工業社會的文化傳統和政治體制,發展的上限低於越南。
日本的安倍政權對中國同樣有深刻的非理性敵視,中日合作必須等到思路不同的新首相上臺才有可能。在等待期間,只要穩住日本,不讓它把自己全盤賤賣給美國即可,同時必須小心防範日本在區域整合過程中攪局的作用,對排除日方擋路的企圖做好預案。中長期來看,日本的衰退無可挽回,在世界舞臺上,只剩下一點工業技術方面的殘餘價值。
\subsection*{2020-04-13 10:47}

個別幹部和政策的問題我不熟,無法進一步評論。
吸引外國的精英子弟來受教育,自希臘、羅馬開始到近代英、美、日等帝國,向來都是安撫收攏蠻族的重要手段,從清朝到民國、再到現代,許多中國知識精英依據自己留學地而成為親英、親日和親美派,是那些國家有意而爲的結果。現在中國準備重回既有的國際一流地位,難免也得隨俗,所以這裏有很高層次的戰略考慮,官僚系統也必然受到囑托要好好照顧這些未來的蠻族首腦。整體來説,稍許優待外國學生(商人就沒有理由享有任何特權)無可厚非,中國只是做得有點生硬過分。
我覺得中國教育的主要毛病,出在兩個方面:1)學術界的風氣惡劣,假大空充斥,已經到了逆淘汰的地步,這不但影響產業升級,甚至間接拖累智庫的水準,使政策和制度的制定和改革都額外困難;2)義務教育在保障社會垂直流動性以及消弭貧富不均的重要作用,沒有被足夠地重視理解,結果是鄉下學校所得的資源長期匱乏,主管單位甚至盲目引進英美的失敗教育政策,造成不斷的倒退。這兩點,我在博客上都已經反復討論過了。
\subsection*{2020-04-12 16:24}

中國的產業升級,任重道遠,畢竟要在幾十年的時間内,趕上歐美幾百年纍積下來的技術根底,絕非容易的事;像是AI這種全新的技術領域,反而不是問題。
在必須從後超趕的工業範疇,原本因爲許多權力圈子内外的官員和“經濟學家”受美國自由主義理論的蠱惑,部分產業執迷於追求利潤的短綫操作,若干所謂民族企業純粹靠攔截竊取國家和人民的投入以自肥。在《中國製造2025》施行之後,整體趨勢有所改善。2018年開始的中美貿易戰,更是當頭棒喝,對中國國内產業發展路綫的議題上,有著極强的統一口徑、撥亂反正的效應。我認爲Trump的經貿打擊手段,將會適得其反,在短期的陣痛之後,反而加速中國的上升態勢。
至於中國學術界的腐敗,始終是成員找規則和執行上的漏洞來Game the System(玩弄體制?)的問題,修改規則只是促使他們去尋找新的漏洞。教育部仍然不願意直接出手打擊假大空,就好像有個輪胎漏氣,不去把洞找出來補上,而只是輪胎換位,那當然是一點正面效果都不會有的。
\subsection*{2017-09-26 00:00}
我和陈平在经济学的原则和理论上同意,在金融业的实践和执行上则有不同的看法。这可能是因为他是做理论经济学出身,而我是靠在金融界实战所获得的经验。

我觉得他太过乐观,远远高估了中国体制下打金融战的灵活性。灵活是西方自由经济体制的强处,中国不应该以弱击强。保守,并不只是修长城,而是在长城内也有纵深防御,层层布防。宋朝并不是亡在金兵入关,而在于没有第二綫防御能力。换句话说,陈平认为美国的金融界像是塞外的骑兵,灵活机动,所以我们也应该建立自己的骑兵部队,出塞远征。但是别忘了,汉武帝固然大破匈奴,也把国家搞破產了。而且同样的宋朝步兵,在岳飞这样的将军领导下,对抗金国的骑兵并无劣势。我的偏好是你打你的、我打我的,先蹲在家里开发火枪,一旦技术成熟,擅长骑射的游牧民族自然被歷史洪流淹没,从征服者变成被征服者,就像俄国拿下中亚和西伯利亚那样,不是靠针锋相对,而是依赖整体国力和体系的绝对优势。

中国的经济金融界,不但上层有迷信自由经济主义的带路党,中层的金融管理人员的品德和专业水准,也十分可疑。金融原本就是资產虚拟化的体系,自由、灵活的代价是复杂性和不透明度。整个行业先天就是靠信息不对称来占实业界的便宜,而越是复杂和不透明的金融体系,信息不对称性就越高。拿陈平所建议的自主的对衝基金来説,既然占自己人的便宜,必然更快更容易,那为什么要卖命和美国人打硬仗?如果有足够的监管和透明度来确定枪口对外,那么绝对就不会有什么灵活度可言。这是金融业的内建矛盾,就像热力学第二定律一样,注定了不可能有永动机。陈平提到明朝的长城,可是清兵入关并不是明朝败亡的原因,而是一个后果;明朝实际上是亡于内乱。如果采行陈平的建议,就有如唐朝一样,武力强大,但是藩镇割据,反而自食恶果;金融藩镇正是美国现今的写照,中国千万不可仿效。

我是讲究绝对理性的人,所以不会有不切实际的幻想。光衝着我没有中国籍这事,就注定我不可能在中国做主官。而且主官的特质在于能干、肯干,诚实和深思反而可能是负面的资產。

我所学甚杂,但是最强的几个方面,如金融、物理、战略,中国都有自已的专家,没有我的发声,也无关痛痒。反而是一些次要的能力,刚好是中国最弱的地方,如果我的意见能获得采纳,可能会有立竿见影的效果,例如科学行政和对外宣传。上次有关对撞机的讨论,就是我在科学行政方面间接做贡献的例子。\subsection*{2015-11-25 00:00}
你所说的大趋势并非没有道理,不过我觉得细节上比你说的复杂些。机器人是工业革命后,几百年来机械化、自动化的下一步。目前由非技术劳工所做的工作,有一部分会被取代,不过不是全部。这在长期来看,正如你説的,对工业化程度低于中国的地区有额外的压力;然而他们面对的最大问题还是中国本身。中国的体量太大、效率太高,挡在他们前面,他们永远衹能捡中国淘汰的產业。

日本原本可以藉此恢復一些竞争力,但因为受中国的挤压更严重,再加上安倍这种军国主义復辟份子的人谋不臧,长期的前景反而更为黯淡。美国和臺湾也是一样的。

中国的策略,是建立以自己为中心,但择优包容国外环节的全球生產链;那么10-20年后,能随中国一起同步升级的国家,就衹有全力合作融入其產业链而且能奋发自强、保卫自己专业的几个国家了,如韩国。

当然世界经济并不是零和的游戏,所以中国的长期战略主轴也包括提携低度开发国家进入至少是初步工业化的程度,这样才能扩大世界的总需求,否则所有高端產业的总產值连让中、美、欧分食都不够,会造成割喉竞争的残烈环境,对经济、外交和社会的稳定都很不利。所以像越南这样的国家虽然绝对生活水准还可能持续提升,相对于中国大概永远都处于低端的层次。反而是目前最原始的地区如非洲,如果有足够的政治智慧和机遇,可以有赶上相对高端国家(如越南)的空间。\subsection*{2015-09-02 00:00}
1. 一个国家可以完全控制自己的货币,如果需要钱,多印就行了,后果只是通货膨胀。外匯就是政府手中的外币,那当然不能在国内流通,它是用来平衡匯率的。

2. 引进外商投资,要的不只是资金,更重要的是技术和管理。外商拿外币来设厂,必须先换成当地货币,才能购买材料和发工资,这其实是外匯的三大来源之一。国内储蓄若是愿意投资当然很好。我说过,促进生產是对国家最有利的资本应用方式,所以不论那个资本来自国外或是国内一般都是好事。

3. 外匯指的是(广义的)外币现金,它的好处是随时需要(例如被美国对冲基金攻击匯率时)随时可用。若是有多余的,可以投资在流通性低但是回报率高的固定资產,例如外国的土地和企业。所以外匯基本上就是在金融战场上防御美元霸权打击的国防力量,必须有足够的吓阻力,但是太多就影响国计民生。

4. 中国的外匯总额看来很大,其实占GDP的比率并不高,而且中共已经努力将部分转化为前面提到的高报酬率资產,所以最近开始下降(就像中共今天也宣布要裁军),未来应该也不会再显着增长。

顺便提一下,美国的名义GDP还比中国高一些,但是他基本没有外匯也没有必要,因为美元是国际储备货币,多印钞票造成的通货膨胀主要由其他国家承受,如果不拼命用白纸来换外国资產,那才真是蠢呢,所以美国自然就成了消费国。\subsection*{2015-08-24 00:00}
印钞票是中央银行的职权,所以只有当中央银行买国债时才能说"印钞票来维持经济"。美国人的婉辞是量化宽松。

日本是由国内银行和邮政储蓄系统来买国债,所以只能说是"印债卷来维持经济"。这些资金的来源的确是人民的储蓄和保险金,一旦还不出来后果极为严重。若要避免金融机构连环倒闭,只有把所有储蓄和保险统一"理髮"一条路。理髮的比率应该会在50 \% 以上,再加上日币匯率会完全崩溃,日本人留在国内的资產一夕之间挥发近净。因为只有大企业和财阀才能到海外避险,结果是中產阶级会被一笔抹消,国家也从先进行列跌入"开发中"层次。二战后的歷史里,只有苏联解体才发生过类似的惨剧。俄国人素以能吃苦耐劳着称,不知现代的日本人还能不能忍受同样的经验。\subsection*{2015-08-24 00:00}
国债不是人民欠的债,而是政府欠的债;政府若是被革命推翻了,新政府就可以考虑注销这些债(当然要看与债权人谈判的结果,国际之间没有破產法,我已经在《美元的金融霸权》里解释过了),自然人的死亡和它无关。

日本的特色在于它的国债不是由外国人持有,而是邮政储蓄系统和银行奉政府之命拼命买的,所以利息支出到了税收50 \% 以上他们还是得乖乖地接受低利率继续买。到了2025年以后,这个比率必然会超过100 \% ,届时就算邮政储蓄系统和银行仍然奉命继续买,国债增加的速度也会大幅提升为指数函数,要硬撑也撑不了几年。

请务必读完所有的正文再发问,如果能读留言栏更好。絶大多数的问题已经被讨论过了。

还有,留言请勿使用俳句格式。\subsection*{2015-08-23 00:00}
Good to hear from you again. How is the demonstration against TPP going?

Regarding your questions:

1. Without the QE, the US economy would have dived to great depth. Even if it eventually recovers, the GDP level will still be much lower than pre-2007.

2. GDP is not perfect, but it is convenient, so I use it for the sake of economy (pun intended).

3. Euro has been great for Germany and other export countries, not so good for the PIGS. Overall, though, Europe is better off with it than without.

4. Yes, tribal bias ends up hurting oneself. Kind of a poetic justice.\subsection*{2015-08-23 00:00}
放任股市上涨的理由是吸收游资(避免外流)并刺激消费,但是这先天上就不是最好的手段,因为真正赚钱的是内线的大户,而这些大资本恰是在涨高了后第一个外逃的,所以不但不能很好地刺激消费,最终反而会增进贫富不均,更造成大笔资金外流。

我以前就说过几次,正确的做法是为低级公务员加薪,不但立刻刺激消费,而且与反腐相辅相成。股市可以微涨,但不能大涨。结果放任大涨之后,果然成为大户杀小户的屠宰场,这时就应该亡羊补牢,马上出台我在前文《谈中国股市和其他问题》中所提到的资本利得税(当然必须有预案,最好一个周末就办完)。结果金融主管只是笨手笨脚地去一个一个查,这是事倍功零的蠢主意,果然到现在一个都抓不到。你只要先把所有股市账户里的资本利得冻结适当的比率来准备交税,然后慢慢地查,这些大资本就跑不掉。大部分的暴利还可以归公,刚好可以为公务员加薪。还有,资本利得税的比率可以视政策方向调整:根据买入的日期,政府想扶持股市时下降,想打压股市时提高。

总之,中共当局处理股市的过程不只是笨手笨脚,根本就是缚手缚脚,和美国政府如出一辙,显然是受既得利益者的控制。习近平应该不会如此腐败,所以大概是李克强的手下有问题。\section*{【金融】【经济】谈中国股市和其他问题}
\subsection*{2020-07-20 14:49}

我寫稿已經六年,信譽建立起來了,有不少專家學者直接間接采納我的看法,幾乎每周都有讀者私下聯絡我,說又在中文媒體或新出版的書籍裏看到其他作者復述我的意見,像是昨天發現《波音衰敗之源》被寫進暢銷書,或者上個月發表《談美國的反種族歧視運動》之後兩天就有人把同樣的内容重寫了一遍,不過他還增補了5 \% 的額外資料,也算是有新貢獻。
但是除了正確的事實之外,我真正希望傳授並示範的,是科學的態度和邏輯的方法,因爲這是扭轉不良輿論風氣的關鍵所在。態度/方法和知識不一樣,不能靠其他作者的借用來廣爲傳播,必須由讀者群到其他論壇去糾正普羅大衆;所以你看到有人陰陽怪氣,那其實是你的機會和責任去指出造謠、傳謠的錯誤。大家共勉之。
\subsection*{2015-07-07 00:00}
基础科研发论文的标准要是很松的。这是一个"有趣"(但是绝对是没用)的性质。我批评的是他对记者所吹嘘的实用价值,你要不要打赌,30年内都不会有光推的动力在太空中实用化。

这类的实验每年有几百个。在媒体上吹一吹后就永远销声匿迹的还算好的。当局要是肯追踪问责,教授们就不敢胡吹乱盖,那么经费就不会被浪费掉了。像是我在《永远的未来技术》里谈的核聚变,已经浪费多少个百亿美元,就因为那些教授们要骗钱吃饭。你知道他们发几千篇论文了吗?超弦的论文以十万计,有哪一篇是说实话的?

我去年写那篇《丁肇中和高能物理的牛屎文化》时,丁肇中说"很快"、"马上"就会发现暗物质。我在超大也遇到一些容易受骗的,当时我就说到年底绝对没有下文,到30年后也不会有下文。现在已经一年了,你去Google看看,丁肇中找到暗物质没有。\subsection*{2015-07-06 00:00}
Insider trading is another big issue, but I don't have a simple solution, so I did not mention it in the main article.

On the other hand, the few suggestions I did make are all easy to implement and hard to weasel out. They should make big improvements on the capital markets. I hope they don't fall on deaf ears.\section*{【工業】開發太陽系的經濟效益}
\subsection*{2020-07-08 12:45}

美國政府從來沒有找到UFO的確實證據,這已經反復被幾百個卸任官員(包括所有前總統)證實,他們口徑一致來撒同一個謊的可能是零,因爲沒有什麽共同利益。從美國軍事工業和其他尖端科技的發展經驗來看,也沒有任何來自外星的突破。所以你只要看到有文章談Area 51,就可以完全忽略。
但是美國軍方的確搜集了一批很奇怪的疑似記錄,尤其是戰鬥機錄下的影像,至今沒有簡單的解釋。
從科學的觀點來看,即使是銀河系中心黑洞這類能量級別高於大對撞機百億倍以上的極端系統都無法產生超光速粒子,即使相對論有漏洞,超光速有理論上的可能,要把它工業化成爲曲速引擎也不是任何生物能力所及。如果地球上真有UFO,也必然只能是由AI操作、經由幾百甚至千年次光速旅行而來的訪客,而且所費不菲,那麽爲什麽外星文明要這樣浪費錢就很難解釋。但這種經濟人文考慮不像物理是絕對的,所以我不願意說Impossible,只是非常非常Improbable。
\subsection*{2018-11-06 08:43}

1.宇宙綫的能量極高,可以遠超出憑空創造新粒子對的所需,所以被阻擋之後,反而會產生所謂的Secondary Radiation(二次輻射) Jet(噴流),對人體的殺傷性反而更高。 
2.電磁彈射適合在真空作業,所以月球或小行星基地都可以用。2009年有一部科幻電影叫做“The Moon”,講的就是在月球開采氦三,然後用電磁彈射到地球軌道。 
3.你如果相信美俄這些新星際發動機的宣傳,我在火星有大片上佳土地可以便宜賣給你。 
玩笑歸玩笑,載人航天不能超越月球的真正障礙,還是經濟效益問題。就算有了比現有的火箭效率高一百倍的新發動機,在工程上能推得動兩米鉛壁的飛船,它也一定比用AI+Robot貴許多倍,所以就不會發生。 
美國每年死於車禍的,有三萬多人。如果全國都開防彈車,就可以大幅減少死亡人數,但是卻顯然不可能,對不對?而防彈車只不過增加費用一倍而已;載人航天可是會比AI+Robot貴將近百倍的。
\subsection*{2018-11-02 14:20}

有關高鐵,做決定很簡單,因爲它是老技術,根本就沒有物理和工程上的疑慮;至於經濟效益,鐵路在中短程是效率最高的交通方式,也是一百多年的老常識了。尤其新的、便宜的、方便的交通渠道是一切工商業的基礎,我覺得這是No Brainer。
天眼是純科學,對經濟沒有任何實際貢獻,所以必須靠科學家的熱情和政府或富豪的支持。不過它是真科學;如果我來主政,把高能物理那些做僞科學的錢省下來給天眼,是很合理的。輿論也可以推動馬雲這些人捐錢。
藏水入疆這事,我不是專家;不過感覺它不一定合理,必須仔細客觀論證,看看經濟效益到底如何。我最不喜歡它的一點,是所有藏區的主要河流裏,最北、也就是最靠近新疆的,正是黃河的上游,可是黃河的下游,才是中國最缺水的精華地帶。把其他河流的水送進黃河還有道理,把黃河的水送到新疆,就顯然違反經濟學常理了。
\subsection*{2018-11-01 14:47}

能源是工業的基礎,所以未來科技以解決能源問題爲先。
太陽能和風能在近年產量上去了,價錢也就下來了,已經到達和煤電同價的地步,這還沒算進燒煤的社會成本。現在太陽能和風電的主要問題,在於它們不穩定,時有時無,所以仍然嚴重依賴所謂的Base Load,也就是能24小時持續發電,而且在有需要的時候,能調高出力的電力來源。這也是核(裂變)電還在發展建造的主因。
所以我個人認爲,與其浪費資源在核聚變這種極不靠譜、又絕對緩不濟急的科幻項目上,真正應該砸錢的,是大型的能量儲存技術,可以把白天的太陽能留到晚上使用,例如全釩氧化還原液流電池就是一個我很欣賞的未來技術。另一個可能,是用多餘的能量來裂解水,得到氫氣,需要的時候再用燃料電池來發電。如此一來,危險的氫氣可以集中處理,就沒有我以前提過的安全問題;這也才是所謂的“氫經濟”的正道。
可惜我人微言輕,執政者仍然被學術界的利益團體矇蔽。
\section*{【歷史】土星環與富士康}
\subsection*{2020-06-28 13:00}
唉,你沒有仔細去讀歐洲近代史。16、17世紀的人口增長,常常被戰爭、氣候、瘟疫打斷,它們除了直接作用,間接造成穀物欠收也一直是人口批量死亡的主因之一,即使在相對不嚴重的年份,也會大幅提高幼兒死亡率。例如當時的傭兵部隊通過農村,除了把穀倉搜刮一空之外,往往還故意把田裏的作物放一把火燒了,以免日後另一方軍隊有糧食可用。農民即使事先得到消息,先在山裏或林地躲起來,等軍隊過境回村,一樣活不過冬天。所以後來神聖羅馬帝國境内就有人開始推廣馬鈴薯,原因除了產量大之外,就算地表的枝葉被燒光了,地下的塊莖還在,可以慢慢收成。到了Frederick The Great,這甚至成爲普魯士最重要的農業政策,爲的就是戰爭潛力。愛爾蘭對馬鈴薯的依賴,也是衆所周知。即使是清朝人口的暴漲,也部分得益於這些高產作物。\section*{【科研】【媒體】新冠病毒和媒體亂象}
\subsection*{2020-06-26 00:07}

這類官方統計數字,國際媒體的記者求之不得,因爲很容易把數字玩弄“解讀”一下,水出新的文章來。而且絕大多數人對數字有非理性的崇拜,經常無條件地接受為事實真理。就算是行内頂尖的專家面對可疑的統計結果,如印度的GDP,頂多也只能簡單加上一句“可能有偏差”的警告,在進一步的比較和討論時,還是只能用上“權威”數據。
中方對無關國安的統計資料,應該展現出一點自信,沒有明顯理由要保密的就公開出來。要注意的是,中外的統計規則可能不一樣,例如中國的GDP和美國相比,就嚴重低估了住宅的貢獻。這裏美國檢測新冠報告的是“人次”,亦即同一個人(例如醫護人員)可以接受幾十人次的測試;中國發佈數字的時候,應該把編纂數字過程中所做的類似選擇也列出來,尤其是可能必須和國外做比較的東西,要先想想會不會引起誤解。
我注意到博客這裏有關教育的評論,引起了特別大的共鳴,應該是因爲這事對民衆切身相關,政策的優劣大家多年來在潛意識裏都有體會。一旦問題的核心被點明了,正確的結論很容易傳播開來。希望大家繼續努力;中共對來自人民的反饋其實是很在意的,只要能言之成理,在中國推動這種改革要比美國少了很多既得利益者的阻撓。
\subsection*{2020-03-30 18:06}

這些都是事前就可以預期的,他們的政治正確容許任何人拿中國當替罪羔羊,無能的政客和面對Cognitive Dissonance的種族主義者自然會發明各種歪論。
反制的辦法也有,希望外交部和媒體界趕快想通。例如誣賴中國瞞報病患數據這種事,早應該每天公佈測試次數(很奇怪中方宣傳管道根本不提,好像當作國家機密一樣,真是愚不可及),自然可以和英國他們直接比較;其次是强調WHO對各國防疫工作的評價,順便提一提WHO官員的國籍(加拿大、北歐等等)。至於Trump團隊所謂的中國遮掩疫情,以至於美國措手不及,回應也很簡單:南韓和新加坡是怎麽比美國提早兩個月做準備的?難道他們有Psychic Power?
像是援外的醫療器材會被挑剔,也是一個月前歐洲疫情爆發就應該想到的,但是中國自己對私營企業販賣假貨、劣貨的確監管鬆弛,被別人拿來説事,不論是否精確,也不能算是無辜。外交部沒有想到要求嚴格品管、統一出口,當然也是我所説的“應對失當、協調不足”的又一個例子,但是連最基本的印上統一的國徽國旗標識都做不到,就好像是九九乘法表都背不齊,討論微積分是沒有意義的。
\subsection*{2020-03-25 22:36}

因爲新冠往往症狀很輕,而且檢測器材到處都匱乏,全世界所有國家的統計資料都低估實際患病者人數好幾倍,差別只在於是像中國和德國這樣2-3倍,還是英國、日本那樣1、20倍的低估。
印度的檢測普及度並不廣,但正因如此,所以Modi不可能確定實際疫情嚴重到什麽程度,他之所以決定要封國,必然是被歐美的慘狀促成的,算是很有魄力的及早行動。
溫暖潮濕的天候對病毒傳播的可能影響,不是二元的有無選擇,而是連續的Spectrum。綜合目前的傳播速度來判斷,我認爲新冠對夏天的敏感度低於流感是大機率的事,所以歐美不能只靠季節的幫助。他們沒有中國的行政效率,要徹底根除内部社區感染,即使在認真面對疫情的國家,也會需時更久。與此同時,中國因爲只需防範Back Flow回流病例,可以更快、更有效地復工,在經濟上會有極大的優勢。
像是Trump、Johnson和安倍,顯然還沒有“Get It”(懂事?),這些國家的災情就會因而延長。不過除了天候之外,還有另一個可能的助力,是中國沒有經驗到的,也就是那80 \% 的輕症病患。目前已經開始有抗體檢測器上市,這可以檢驗出誰是已康復的免疫者,那麽這些人就能安全地承擔起在公共場所勞動的社會責任。我想這在意大利和西班牙這樣在疫情發展上先行的國家,會有明顯的正面影響。
\subsection*{2020-03-15 03:06}

西班牙流感的確非常非常地厲害。1970和80年代的研究報告中說它的“Population Mortality Rate”(死亡佔總人口比率)超過2.5 \% ,後來被誤引成爲“Case Lethality Rate”(患病後的致死率)是2.5 \% ;其實致死率是在10 \% 以上,這比新冠要高一個數量級。不過這些致死率是在百年前醫療技術的背景下(亦即沒有呼吸機Ventilator)發生的,而且從一個地區到另一個地區變化很大,所以和新冠的對照沒有太大的意義。真正重要的差異,在於西班牙流感專殺年輕人,而新冠剛好相反,這一點讓我很安慰,至少不必太擔心小孩的安全。
若干國家放棄Containment不是問題,真正的問題在於英國的Delay策略口惠而不實。換句話説,不上溯病例的感染來源情有可原,但是對新病患和他們的周邊人員也不要求隔離,那所謂的延遲流行要從何著力?我覺得德國人或許還有足夠的自律,能主動決定宅在家裏;英國則連政府都說除了正在發病的病人本人必須自行禁足之外,家人朋友可以繼續自由活動,這顯然是存心放手,草菅人命。
\subsection*{2020-03-14 23:13}

Herd Immunity當然是騙人的字眼,實際上就是放棄防治,坐等病毒傳播。Boris Johnson所謂的從“Containment”跳到“Delay”,純屬忽悠,因爲照理兩者的所需手段其實是一樣的,差別只在於要不要回溯追蹤每個病例。英國可並不只放棄回溯,它連隔離、檢測都不做了,那所謂的“減低高峰”要怎麽達成?
2007年,Boris Johnson在被電視台訪問的時候,談起《Jaws》這部電影,他居然說最佩服那個死再多人也不肯關閉海灘的市長。其實從他在牛津時代,就有很多跡象顯示他是一個對他人禍福死活完全無感的Sociopath。英國這些火鷄投票要過聖誕節,現在Johnson把他們放進烤爐,説真的很難讓人同情。
\subsection*{2020-03-14 15:27}

首先,從有關Space Elevator的問題可以看出你並沒有讀完舊文下的留言。請你先花一個月時間閲讀整個博客,尤其是《讀者須知》,期間如果再急著發言,我會把你列入黑名單。
1.美國是已經失控了,但是中方既沒有必要也沒有管道來做宣傳,歐美的民衆和媒體自然會抨擊美國政府。
2.Biden有民主黨建制派的全力支持,應該不止會是總統候選人,而且會是下任總統了(除非他犯下極爲嚴重的Unforced Error)。
3.當然有宣傳價值,但是可以做得更好,例如在醫療用品上打上明顯的中國援助或中國製造的標簽,或者甚至派軍方運輸機載一隊軍醫高調去協助友好的國家。
4.這我不是早已説過了嗎?
5.同上,Space Elevator是大忽悠,我早已説過了。
\subsection*{2020-03-12 18:09}

1.我自己也只能多睡覺,吃維生素D,並且盡量不出門,出門則戴口罩。剛好小孩從女朋友那裏得了B型流感(就是今年CDC沒猜對的那一型,這是測試確認的),也只能宅在家裏,兩人一起看《Witcher》。要說準備世界末日(我並不預期本地的社會秩序崩潰),我缺的只有槍械了;在美國30幾年,我一直不願意玩槍,因爲統計結果是平民槍支會“使用”(亦即不是打著玩的)起來,有3/4是用在自己和家人上。
2.新冠疫情的確滿足了《再談統一》一文中的條件4,不過我覺得並沒有滿足條件1。中國自身的經濟也因貿易戰和新冠而受傷頗重,歐美在疫情打擊下,有可能會爲了轉移民眾的怨氣而試圖把問題外部化,所以軍事干預的機率反而增大了。反正長期的後果是中方能較好地復蘇,暫時再等下去是明顯的最優選擇。當然這假設沒有台灣社會秩序完全崩潰的Scenario,但現在討論那一類的極端可能爲時尚早。
\subsection*{2020-03-11 21:31}

我已經討論過了,疫苗最早也是年底才會有,而新冠不一定和流感一樣會在夏天消退,那麽在揭開歐美治理水平低下的真面目之後(英美宣傳抹黑的一個主要伎倆,是把對手所面臨的現實困難和資源匱乏歸咎在體制缺陷上,這次是難得的相對公平的比較),還會造成嚴重的人命和經濟損失,可能成爲二戰之後,人類社會最大的全球性災害;在最極端的情形下,甚至可能觸發個別西方政體的深刻改革。
反觀中國,我相信改革的方向已經確立了,現在應該是在斟酌方案的細節。醫療健康系統會是重點,例如我所建議的廣設縣級以下的先進醫院、提升醫護人員薪資、整頓人事規則、扭轉私有化進程、授權給獨立的流行病監控管理系統,都可能會在未來兩年成爲正式的政策。
不過這次疫情最重要的教訓,其實是不能坐等事後處理,必須尊重有遠見的專家意見,事先未雨綢繆。中國政治和外國一個很大的不同,是把最聰明能幹的幹才安排並任命到主管職位上。世界所有其他國家,不論文化背景、體制設計、地理位置、經濟層次,領導都不是選拔最聰明的人才得出的,所以他們必須有強勢稱職的智庫和幕僚(典型的如英國,參見《Yes, Minister》),提供專業和深層的策略建議。中共剛好相反,主官比幕僚聰明;這有它的好處,但是選拔主管必然是以辦事效率為標準,偏偏只有專業幕僚和智庫才會有餘暇去製作預案、未雨綢繆。所以重主官而輕智庫的結果,自然是專注於事後彌補而忽略事先預防。這是中共體制的先天特性,要改不是不可能,但是必須有最高層的重視和背書。在當前國內國外多事之秋,只怕無法獲得優先考慮。
\subsection*{2020-03-09 17:03}

其實如果連症狀太輕而沒有被檢測的也估算進去,歐盟在今天的病例應該已經超過三萬了(名義上是一萬出頭),美國則至少是5000(沒人相信只有600)。我也不看好他們靠隔離來遏制疫情的能力(美國是連嘗試都懶得去做的),但是仍然必須減緩傳播,減低疫情的高峰,避免醫療資源被洪水式地冲倒。
這一波新冠流行,在經濟上對歐洲的負面作用會比中國還大。美國想靠無爲而治來避免直接影響,但是信心崩潰一樣會刺破泡沫,間接地打擊經濟,受損可能超過歐洲;不過這裏有很大的自作孽成分,所以不能完全算到新冠的頭上。Trump是典型的民粹,特點是不講理、不尊重事實邏輯,這和外界更大的客觀力量(例如大自然)一接觸自然會崩潰;這一點我以前在討論台灣政局的時候,已經反復提過了。
\subsection*{2020-03-09 03:25}

至少德國的測試劑生產足夠,所以歐盟沒有日本和美國的埋頭鴕鳥問題,他們的困難在於被白左的“民主自由”宣傳忽悠過度,已經忘記政治的最終目的是全民公共利益的最大化。
今天又有新文章揭露湖北處理新冠的錯誤,在一月中旬有十天左右,武漢的衛健委不但對專家組遮掩人傳人案例,而且也把使用直報系統的標準提高到不可能的地步,嚴重干擾了中央反應的速度和精確性。當然這不是中國防疫過程中所面臨的唯一難處,當時情況不明、工具不夠也有很大的影響。像是伊朗在上月也一副慢半拍的樣子,但是一旦有外援的檢測劑大批供應,他們的統計數字就開始合理化,可見客觀資源上的限制是主因。美國和日本才是典型的人謀不臧。
我明天要再趕辦一次雜貨,然後準備自我隔離,避免人際交流。目前一般老百姓,除了保持睡眠充足來加强免疫系統之外,唯一應該試試的是維生素D。所有的維生素都不是藥,而是所謂的微量營養,只有在吸收的分量嚴重不足的時候,才會有健康問題。而現代人隨便吃,營養都比古代豐富得多,所以一般不會有維生素欠缺的問題,唯一的例外是維生素D,因爲它主要不是靠飲食,而是由皮膚在日曬後合成,現代的生活以室内爲主,尤其像我這樣成天在閲讀的宅男,維生素D不足是很常見的,偏偏它又是免疫系統的所需養分,所以特別補充一下很合理。
\subsection*{2020-03-08 04:44}

Trump治下的美國,反應極爲混亂乏力,連測試劑都是本周才剛開始批量生產。2009年的H1N1流感和2015的Zika病毒,CDC統籌的大規模測試從一開始就很認真,後來雖然對H1N1放棄隔離,但也減緩了散佈,使醫院沒有人滿爲患得太離譜。現在美國比起日本來,一樣是不檢測、不公佈的掩耳盜鈴,但日本至少還能讓學校停課;像是意大利把米蘭封城,在美國根本無法想象。我覺得Trump的反智、反科學態度,在過去三年對聯邦政府裏的專業官僚體系,打擊非常嚴重,下一任民主黨總統就算有八年時間,只怕也難以完全恢復。
新冠對20嵗以下的年輕人,殺傷力只和流感相當,所以我不用擔心小孩。至於我自己,該有的預防手段當然還是會做,但是50幾歲不抽烟的人,致死率也就是2 \% 左右,而且我在這個世上該做的事也差不多了,實在沒有緊張的必要。
\subsection*{2020-03-06 22:21}

這篇論文似乎和那個網站用的是同一個國際基因資料庫,不過是三周前寫的,所以只有93個樣本。在美國采樣發現新的突變,絕對很正常,因爲正如你所説的,取樣不夠,所以目前樣本和突變的數目相差不到一個數量級,本來每一個新樣本就有可觀的機率會包含新的突變,美國如此,尼泊爾如此(C24034T,亦即在基因第24034字母上,從C突變為T是在尼泊爾最先看到的),南韓如此(T4402C,G5062T),澳洲如此(T18603C,C24990T,C25587T),英國如此(A2480G,C2558T,T18488C,T23605G,A29596G),法國如此(G22661T),新加坡如此(C10138T),瑞士如此(C24378T,C26894T),巴西如此(C2388T),意大利也是如此(G11083T)。正確的分析方法是去看基因家譜,那麽很簡單地可以讀出,目前已知所有新冠病毒都來自同一個源於武漢的祖宗。一般人沒有專業能力去做這個分析,不要緊,但是明明不懂還要硬做結論,製造並散佈謠言,導致惡劣的社會影響,這不是知識分子所應為。
新讀者搞不清楚狀況,連《讀者須知》都沒有看就開始發言,這雖然是Annoying,我還可以理解。那個談055的,卻是典型的無腦噴子,網絡上多的很,博客這裏以往有過,將來也一定還會有,所以我先把話説清楚,以後直接刪除也有依據。
\subsection*{2020-03-05 16:25}

表面上的理由是沒有足夠的證據證明有效,實際上的原因當然是庫存不夠。事實和邏輯雖然沒有强到可以說口罩絕對能隔絕每個傳播,但是有足夠的力量來說能幫助減低流行。他們搞的這是律師玩的文字游戲,參見《常見的狡辯術》一文。
先抱歉,借這個留言談一個不相關的話題:一個月多前我開始解釋有關新冠病毒的特性,包括它本身就異常厲害、早期又有嚴重的戰爭迷霧、所以中方的反應能力其實獨步天下、為世界爭取時間做出極大的貢獻等等,在當時是很孤獨的聲音,現在疫情擴散到英美,人人都能看清的確是如此了。我並不是要誇耀先見之明;其實每次有重大事件,我都是要有獨排眾議的精確預測才會想要討論,否則根本懶得去重複大家都知道的事實。不過最近有些新讀者回頭看舊文,居然會在極爲準確的預測裏,硬是鷄蛋裏挑骨頭,找一些還沒有被明顯驗證的細節來説事。這樣的事後諸葛亮很是煩人,像是年初有人對《055級的設計概念》大放厥詞,上周訪客簿又有人要挖中美貿易戰的老墳,在新冠病毒這事上,只怕在大家的記憶模糊之後,也會有人拿無數專家幾年努力所得的詳解來評論我的事先預測,所以在這裏先立一個竿,有認爲我分析得不好的,請趁讀者群記憶尤新之時趕快發言批評。
\subsection*{2020-02-19 15:18}

其實這次疫情所暴露的真正問題,大衆輿論上沒有專注;不是問題的,反而被反復攻訐。這又一次示範了我一再强調的事實,也就是群衆是愚蠢的,媒體是自私的,兩者合在一起,只會讓非理性的情緒發泄產生共鳴而不斷放大。
不是問題的,是事後的反應。中方試圖對一個傳染力近似流感的新病毒進行隔離防堵,這是世界獨一無二的壯舉,就像50年前載人登月一樣地困難。爲什麽Apollo計劃的工程細節毛病百出,Apollo 13直接出事,全球都同情關注,而中方在防疫過程中,做了一些只有事後才能確定是次優的選擇,就被無限上綱?2009年北美出現H1N1的新流感,美國連全面防堵都不做,任由它傳染到世界各地,至今仍然是每年流感流行的主要變種之一,11年來殺死了近600萬人。中國的努力,是爲了避免重蹈那個覆轍,拯救的是全世界未來的幾百萬人命;即使沒有完全成功,至少也把疫情的擴散拖延幾個月,要出現全球大流行必須是下一個冬天了,届時疫苗已經可以大量生產配發,疫情不至於失控。這樣造福全人類的壯舉,美國人基於保護霸權的需要來詆毀倒也罷了,中國人自己也反過來當作抹黑中共政府的藉口,實在是無良、無知、無恥。
中方真正的錯誤,是沒有在SARS之後,汲取教訓,嚴禁野生動物的活體運輸、集中和販賣;另一方面,也沒有建立完整有效的防疫體系,對醫療系統不但投資不足,而且迷信美國的放任經濟理論,居然去搞私有化。私有企業的唯一目的是賺錢,什麽引進技術、提升服務、回饋社會、犧牲奉獻,都沒有偷搶拐騙來得容易方便。要是有嚴格的政府監管,嚴禁偷搶拐騙,還可以不太離譜,但是辦醫院原本就純粹是要照顧病患、服務社會,政府要是有能力、有閑空去嚴格監管,直接管公立醫院就好了,搞私立來搜刮醫藥費有什麽好處?即使是西方經濟學,也早有誠實的聲音,承認私營體系在醫療、法律、教育這三大方面完全不合適,這是因爲人命、公理和年輕人的心智是不能交由市場來自由買賣的。
\subsection*{2020-02-18 23:21}

你説的,中醫在明清(尤其是清朝中期)被玩票者和空想家借殼上市,網絡上討論的文章不少,資料很詳盡了,但是自欺欺人的假知識分子仍然不在乎。
另外還有一個歷史因素,就是清末民初,民族自信被外國炮艦打趴了,很多士人一夕之間對自己所學的中國文化和價值完全失去信心。比較理性的辦洋務或者搞革命,比較蠢的就開始發明神話了。因爲生醫議題特別複雜,不確定性極高,精確結論很難從幾個實驗觀察得出,所以除了國術(功夫和中醫在近代中國歷史上有很多相似之處;例如我曾介紹過,少林寺也是在明朝被賣把式的借殼上市)之外,造神運動就集中在中醫之上。尤其是這些士人往往自已也玩票,所以順便也是往自己臉上貼金。我以前説過幾次,必須有民族自信,但是自信必須建築在邏輯思辨能力之上,其實就是暗指這件事。
世界上的病毒多得很,中醫要“發明”抗病毒藥有幾十年的時間來搞,結果什麽都沒有;然而一個新型病毒上了頭條,他們在幾周時間就有了解藥。這分明是利用民衆恐慌和事件初期的信息迷霧來騙人自肥;新冠肺炎是一個全球性的問題、國家級的災難,他們利用國難爲己謀利,可惡至極。
\subsection*{2020-02-18 03:36}

其實我拿中醫和宗教相比,是有很多重深意的。其中之一是宗教先天築基於非理性的信仰之上,所以原則上就無法依據理性來做討論對話,再加上我和讀者的時間精力都有限,應該專注在沒有其他公共人物能替代的話題上。我不是提過,像方舟子那樣什麽都評論,是誤人誤己嗎?這裏的差別在於我雖然也是在自媒體上發聲,但我不是提供新聞性娛樂或娛樂性新聞的網紅,實際上是在辦教育。教育的課題,取捨自然必須嚴格;孔夫子“不語怪力亂神”,就是出於這個考慮。所以寫博客快六年了,我也沒有主動去談中醫;倒不是因爲我不熟它的細節:錯誤原則下搞出再多的細節也不影響錯誤的程度。正因爲我的專業能力在於科學方法和邏輯思辨,批判非理性信仰違反了科學與邏輯,反而是對方沒有能力和資格來反駁的。
所以這並不代表我應該一直回避。我沒事也不會去批評宗教,但是偶爾談到一神教對西方文化的惡劣影響,也會直面事實做簡要的評論。現代中醫的主體,其實不再是屠呦呦用科學方法去搜檢少數可用的偏方那樣的研究工作,而成爲源自古代陰陽家的神秘主義迷信,陰陽五行、氣和食補很好地對應著上帝、耶穌和天堂地獄的迷思。這個神秘主義的中醫教對中國社會的腐蝕作用在新冠病毒疫情之下,多方面地浮現,不但妨礙事先預防,而且扭曲事後的治療(嚴重到WHO想要整頓中國測試藥物的標準,參見這篇新論文https://www.nature.com/articles/d41586-020-00444-3)和檢討,這還沒有考慮多年來在學術界、商界和民智上對理智誠信的長期消耗。所以事實上中醫教已經成爲中華文化圈裏負面效應遠遠最大的頭號邪教。你看法輪功一閙就被鐵腕處理,但是2003年SARS那麽嚴重的災難之後,中共都無法禁絕野味市場,結果直接導致了新一輪的病毒,不但傷害數以千計的人命,而且憑空消滅了萬億人民幣級別的產值。這都是因爲中醫教徒以文科知識分子居多,在國家社會的地位和分量遠非法輪功能比。既然它對民族社會有這麽大的危害,我如果噤聲避談,如何面對自己的良心?
至於正確的處置,第一步就是正名。現在的所謂中醫其實有三個不同的意義:首先是類似屠呦呦所做的科學研究,這應該只叫做“傳統藥物”,因爲中醫的理論基礎是完全錯誤的,唯一有一點點實用價值的來自經驗法則得到的許多偏方,而其中又以止痛最容易觀察實驗,所以垃圾成分最低;其次,是由歷史系和博物館保存的歷史記錄和文化遺產,這應該叫做“古醫學”;然後那些既不是搞科學研究、也不是做歷史保存的,即無良商人、迷信群衆和業餘玩票者(包括中醫學院裏不懂雙盲對照原理的那批人),這些才是“中醫教”。當然這些邪教成員佔絕大多數,要好好教育、嚴肅處理並不容易,得看執政者的智慧和理性民衆的擇善固執了。
\subsection*{2020-02-03 00:40}

我自己做過10多年主管,手下最多也就是二十幾個人,但是對情況細節的掌握,仍然是一件很困難、很吃力的事。這是因爲沒人願意向自己的老闆掏心挖肺地說實話,即使難看的會是同僚,一般人也不願意做壞人。如果是有不確定性的事件,自然會保留一些餘地,描述已經有99 \% 或90 \% 信心的估計,而不是更精確的50 \% 信心的估算,所以每上報一級,報告裏的敘述就緩和一級,這在迅速惡化的實際形勢下,就會產生極大的誤差。
如果把目前已有的知識,時空穿越到12月中,我相信沒有任何一個腦袋正常的人會拒絕立刻動員防疫。問題在於當時情況不明,有些肺炎病例太集中,不一定是流感;電子顯微鏡可以看出一個可能是新病毒,但是還沒有證據證明它真是新的,或者這個新病毒不是並發症。如果換成是我,可能也會只説加緊上報、急送樣本到北京的實驗室確認病原體,畢竟封城兹事體大,不是靠疑心就可以做決定的。這個問題叫做“Fog of War""戰爭迷霧",事後諸葛亮用上帝視野可以簡單批評Napoleon的大部分戰術、戰役和戰略決策,但這並不代表Napoleon不稱職。
我並沒有否定武漢政府的責任,但是要決定它的責任,必須有在幾個關鍵節點上所有專家提供的全部細節,才能精確回溯官員所面臨的抉擇,然後判斷反應決策是否合理。這些環境細節顯然是還沒有完整地被整理出來,任何下斷言的人都是言之過早。但是群衆不想等,於是自然有大V和媒體提供結論,即使這些結論是無中生有、憑空創造的。
\subsection*{2020-02-02 20:23}

這種群衆的非理性,是我在博客一再强調的現象,這裏的讀者應該都能接受是現實。
至於傳播事實真相和邏輯理性是否徒勞,我想一刀切完全放棄是不負責任的做法。少數但是不可忽略的一部分人,已經具有邏輯能力和習慣,稍作提醒他們就會能得到正確的觀點和態度。另外還有一些年輕人,涉世未深,更需要有好的示範。更重要的,是打斷非理性言論的正向反饋回路,不能容許他們成爲社會的正統和規範。
至於對社會整體的影響,容許並鼓勵自然但是非理性的思想習慣,年久日深,會有嚴重的腐蝕效應。我反復討論的美國社會在1970年代之後被富豪腐化的過程,正是前車之鑒。在1960年代之前,並不是美國人民的平均智商高或情緒化趨勢低,而是社會價值取向尊重智慧和理性,無知和自私的非理性言論會受到排斥和嘲笑,所以呈現出來的現象是個人的智商沒變,但是社會整體的智商高了很多,國家執行的效率也同樣受惠。台灣在過去30年,也有類似美國的衰退。
中共體制的優點之一,就在於公民的品質和文化對社會整體沒有直接影響;但是工業化、信息化的趨勢,是如果要在經濟上保持先進地位,必然也得在政治上提升公民的參與空間,所以我們事先做正確的教育,是必要的準備工作。
\subsection*{2020-02-02 12:39}

美國、英國和其他仇中國家,對中國事務動用陰謀論是日常,但是這不代表中方應該和他們比蠢。正因爲有敵對的態勢,更必須保持清晰的頭腦和眼光,才能理性選擇正確高效的對應策略。美國搞中國陰謀論,是因爲他們的體制要求先對愚民洗腦,這是他們動員全國做對外鬥爭的第一步,但是缺點也很明顯:百姓被弄蠢了之後,對中決策就完全失去彈性、無法折中,只能一味蠻幹到底。
其實對生命科學有點常識的人都能簡單看出,光是你説的“選取病毒”,挑“合適”的投放,就已經是極度危險而困難的工作。如果病毒還沒有跨越物種障礙,那麽傳播給人和其後突變獲得人傳人的能力都是極低機率的事,投放一千萬次都不一定有用。如果要在實驗室内先解決這些必要的突變,那麽就算對病毒非常瞭解,而且能隨心所欲地修改成千上百個基因,仍然要拿活人來多次實驗以確定效果。畢竟生物現象不像物理系統,不是設計算得仔細,成品就能八九不離十。這還沒有考慮到在投放現場必然會留下的人工痕跡;而武漢海鮮市場的幾百個病毒樣本都符合自然突變的特徵,光是僞造佈置這些樣本,就遠超人力之所能(假設每個佈置的假樣本有99 \% 的成功機率,500個假樣本都成功的機率是0.6 \% ;然後你必須考慮一大群美國人在海鮮市場翻箱倒櫃的能見度問題)。
很多一般群衆會誤以爲我以前也提倡過陰謀論。其實這是把“沒有證據”和“創新獨特”搞混了;前者是陰謀論,後者是正當深刻的分析。我介紹過的美國過往的醜事,即使是全球獨家的看法,也都是建築在環環相扣、可簡單驗證的事實和邏輯之上。例如我在《訪意大利有感(二)》中説,美軍在二戰中故意在意大利戰綫拖英軍的後腿,就給出許多强力的正面證據,基本否定了傳統歷史上的説法。像是登陸地點的選擇,大家上Google Earth就可以馬上檢驗我的理論;Clark得到最高勛章的奇特時間和背景,也是上網一查就有的歷史記錄。庸人把我的獨到分析和陰謀論相提并論,是因爲對他來説,兩者都挑戰正統敘事,他卻沒有能力看出我論述中所含的嚴謹事實和邏輯。試圖教育這樣的群衆,就像是對天生沒有聽力的人解釋Mozart一樣,必然是徒勞;所以我一向都説我的目標讀者,只限於有邏輯能力的人。
\section*{【美國】有關Trump的一些新觀察}
\subsection*{2020-06-25 17:24}

這種事背後的真正動力有好幾個可能,很難確定,但是我個人的猜測是,加拿大國内的務實派集團真的認爲不值得為美國賣命,與中國決裂,所以由已經退休的人員出面,發表專業意見。這不是試圖直接改變Trudeau的心意,也不是行政部門主動放的烟幕,而是要預先給他一點選擇的空間,如果他決定出手,至少可以說是尊重專家名流的建言。
本周一些英國保守派議員在受媒體訪問的時候,說Johnson在得了新冠之後判若兩人。其實這是胡扯,Johnson向來都是以無恥撒謊著稱,根本沒有改變。那麽爲什麽他自己黨内的同志要這麽說呢?我覺得是因爲他的民調降得太低,10家支持脫歐的主要媒體(全英國另有2家反對)中倒有8個開始批評他,代表著幕後的財閥土豪認爲他的治理能力太差,可能成爲一個纍贅(亦即危害到保守黨的掌權和脫歐大業,畢竟脫歐只是第一步,下一步是自貿協定也不能接受歐盟的反避稅指令),所以已經要為換首相未雨綢繆。這麽快半途換馬必定會遭遇質疑,反對派會說去年不是你們興高采烈地把他捧上臺的嗎?說他受新冠病情影響,就是預埋伏筆,為未來的動作留下餘地。這並不代表叛變已經要發動了,只不過是幕後的財閥需要一個選項,所以預先先創造行動的空間,如果日後決定要出手,不必面臨尷尬的質問。
\subsection*{2020-06-20 14:16}

我不可能瞭解你的思維方式確實如何,所以只能很籠統地討論一下。
爲了通過Turing Test(亦即創造出能夠像人類一樣交談的AI),有很多不同的技術途徑被嘗試過,其中之一是“Commonsense Knowledge”“常識庫”,也就是把一般人在現代社會中生活纍積的基本知識列舉出來,例如“糖是甜的”,然後用邏輯程序來做進一步的推論。
這個辦法其實是直接模仿人類心理成長的進程;不管你自己是否意識到,但是所謂學習就是從原有的常識庫出發,然後加入更多、更高階新常識的過程。理工科理想的學習方式,是每一層新的常識都通過邏輯築基在既有的前一層常識上,而且其間的因果關係本身也算是新常識;如此一來,可以最大限度避免錯誤的論述被存入常識庫裏。
我個人既追求常識庫的量,也非常在乎它的品質。想要讓自己心中的幾百萬個常識條款都正確無誤,首先必須要求它們彼此邏輯相容,所以邏輯思維能力的高效、靈敏和嚴謹是極端重要的技能。下一步你會發現世界上很多有趣的事實並沒有被絕對確認,所以你得要有很好的機率概念和估算能力;這裏一般是有意或無意地采用Bayesian統計來做心算,而且是多維度交叉,難度不小,天分和努力都不能欠缺。然後是做類比;聯想式的浮面類比當然是庸人的囈語,但是一個抽象的因果關係可以反復在表面上無關的不同現象後面有同樣的作用,要正確地排查並判斷,難度又再高了一級。
我的博文是為後來者指路的燈,不但介紹了許多重要的常識,以及它們之間的邏輯關係,我還示範了正確的思維、分析和學習步驟,包括很多有用的科學原則。至於你能領會多少,除了你原本的基礎之外,就要看你肯下多大的功夫了。
\subsection*{2020-06-20 02:19}

我覺得列寧會被如此醜化,其實是因爲維持紀律非常困難,再加上資本家最怕的就是有能力抗衡他們的政治組織,所以大多數國家無法模仿,英美霸主更視之為死敵。
美國政客對外掠奪,像是Bob Dole這樣把錢放入自己口袋的,在建制派媒體眼中是高大上;圖利本國企業金主,更是不計其數,那麽Trump這樣想要把錢直接搞給選民的,怎麽就成了十惡不赦的罪狀呢?
美國政壇哪裏會去分辨爲民服務和爭取選票的差別?要爲民服務,中國的體制效率最高,哪用得着搞直選?美國體制的核心思想正是選票=服務;這當然是狡辯術裏轉移話題的伎倆,方便財閥主宰政治、剝削底層,然後可以名正言順地說這些火鷄主動投票要過聖誕節的。所以你提的那個國内自由派公知的説法,是雙重的錯,錯得都已經脫離這個宇宙了。
\subsection*{2020-06-19 19:13}

沒事,我不是真要批評你,只是藉勢發揮。有時我的回復會Go off the tangent(沿切綫飛出?),並不是完全針對留言而寫的,就像你所説的有可以聯想到的事,利用機會寫出來,畢竟正文的内容是很嚴謹的,不容許這類談話。
昨天我是想起博客這裏固然是片净土,在別的媒體上轉發文章,卻常常會招來噴子,剛好有你的留言,就那樣回答,讓老讀者如果在別處遇上了想反駁,可以有依據。
繼續Off the tangent,我想提醒大家,看到理性的分析引導出自己不同意的論點,不要反射性地排斥,如果有時間精力,應該沿著文中的邏輯想清楚,試圖客觀地解決結論上的衝突。要是找不出文章的毛病,就至少必須存疑,心裏在這件事上加個標簽,說舊有的觀點可能有些問題,然後等待事實發展驗證對錯。否則若是上網就只找符合既有認知的説法來讀,那麽永遠不會有進步,而且幫助形成又一個非理性的共鳴腔;民粹就是這樣被不斷加强壯大起來的。
\subsection*{2020-06-18 23:47}

在我的300篇文章和8000條留言回復裏,滿滿的是幾千個確切的預測和判斷,事前預估和事後復盤都是90 \% 的精確;在中文和英文論壇上,並沒有能接近這個勝率的其他評論者。這靠的不只是既有的思辨能力,也有著在時間和精力上的高度投入;然而能夠記得舊有的結論、注意到事件發展結果、並且在乎推論是否正確的讀者,先天就是極少數的。所以遠超一般人想象範圍的深度分析,在網絡上所受的注意,反而不如一個鍵盤俠30秒内隨便打出來的情緒化謠言。
我原本就明白這個道理,所以始終專注在極少數識貨的讀者上。《讀者須知》裏的頭一條要求,就是先讀完整個博客再參與討論,除了那篇博文裏列出的兩個原因之外,還有一個好處,就是看完全部既有討論歷史的讀者,自然知道我的分析的準確度,遠非其他評論者能及。所以請你不要對又一個判斷/預言被印證的家常便飯而表示吃驚,否則可能會對新讀者產生誤導,以爲這是偶爾發生的事。
\subsection*{2019-09-04 17:16}

真正重要的新聞是林鄭放棄了引渡條款。 
我在視頻中解釋過了,因爲香港的整個經濟都是建基於法制差異,這個條款威脅了無數中產階級和上層階級的生計,因而中間派也支持了暴亂。下一步的邏輯很簡單,就是釜底抽薪已經成爲不得已的必要,但是做得實在慢了些。這可能是内地的官員原本沒有瞭解到它的嚴重性,現在終於有人提醒。我的視頻不一定每個人都看了,但是正確的論點被點明之後,自然會有很多其他評論者模仿采納,然後代爲傳播;例如暴亂幾個月之後,這兩周總算出現了一些討論香港經濟基礎的文章。 
我以前説過,中國的聰明人很多,有時我不須要把預後和處方都拿出來大肆宣傳,只要把正確的診斷說出來,接下來的邏輯推演自然有其他人能做。 
其實這次的經驗告訴我,不要把全部的分析說完,留下幾步推理給圈内人,他們沒有了NIHS(Not Invented Here Syndrome,非我發明的不用症候群),反而更有動力去傳播正確的思想路綫。這和《如何創造研究熱點和一些其他物理話題》裏楊先生因爲把可以做的研究一次做完,所以自然沒有人給他Citation的現象,是同一個道理。
\section*{【美國】【政治】談美國的反種族歧視運動}
\subsection*{2020-06-21 19:48}

1.歐洲向來不像美國那樣無條件支持以色列,不過以往在全球地緣戰略上是美國的絕對附庸,所以只能偶然動動嘴皮子。現在外交和經濟上開始有點兒自主的苗頭,但是北約仍然在那兒,軍事上的獨立會是最後的一步。當前對以色列動嘴皮子更頻繁一些是可以的,要有根本性的政策改變還早得很。
2.我已經反復强調過了,美國如果只看經濟的基本面,早已萬劫不復,現在全靠美元挺住;只要美元霸權還在,就可以無限撐下去。
3.中國幫助印度和越南的經濟發展和工業化,是極度腦殘的行爲;如同資助Nigeria和Zambia這類流氓民族來國内念大學,同樣是外交智庫不入流的明證。
4.Trump從來沒有公平交易的概念。買美國的農產品純粹是因爲1)自身有需要,2)實現承諾,3)方便繼續教訓澳洲。中方去年底急著做成協定,原本就是爲了要避免選舉年和Trump打交道。你回頭讀讀我過去兩年的分析,自然不會問出這樣的問題。
\subsection*{2020-06-20 11:40}

沒有錯,他們是會面臨危險的,但是並不大,如果有紀律,嫌犯的死亡可以大幅減少,而警察自己的傷亡不會顯著增加。
其實警察的死亡率已經被壓得很低了。我們來看看一些簡單統計數據:
警察總人數約70萬,每年被罪犯殺死約50人,死亡率為每10萬有7個;車禍和意外死亡110人,總死亡率是每10萬22。
藍領和白領工作總人數約1億5000萬,每年死亡5000+人,死亡率為每10萬有3.5。
只算藍領,死亡率為每10萬有14個(參見《統計與謊言》)。
建築工人數約720萬,每年死亡約1000人,死亡率為每10萬有14個,是典型的藍領職業。
職業駕駛總人數420萬,每年死亡1000人,死亡率每10萬24個。
可以看出警察被殺的機率並不突出,只有藍領平均的一半,即使算入車禍和意外死亡,也還低於職業駕駛;而這些都遠遠不及伐木工人,他們的死亡率是74/53600=每10萬有138,這是警察被殺機率的20倍。所以美國的警察,絕對不算是高危職業。
\subsection*{2020-06-16 14:52}

美國體系的重要機制,是在多數欺壓少數時說“民主”,強豪掠奪弱勢群體,則强調“自由”。這不但為自己正名,甚至能安撫被洗腦的受害者,尤其是華裔特別多。所以任何政治社會的變革,真正的決定性關鍵,不在於全民的利益,而在於白人内部富豪集團之間的折衝,部分參考上中產階級的意見。正文裏解釋了,這一波反歧視運動風起雲湧,獲得國家中堅力量的廣汎支持,就在於過去三年民主黨建制派媒體的宣傳奏效,爭取到了白人上中產階級壓倒性多數(別忘了,白人有近一半是紅脖子,所以全部白人中過半,代表著幾乎所有教育程度高的一致支持)的人心。
不過重寫開國神話這事,從來不是《CNN》、《MSNBC》或《紐約時報》報導的重點。歷史事實在公立學校也基本是不提的,必須是Progressive的私立學校才有可能。固然上中產階級進私校的較多,但也有在好學區上公立的;私校還要在左右政治頻寬裏散開分佈,真正能學到真相的小孩,是小部分的小部分,只能在白左大學裏互相影響構成多數。所以鬧事歸鬧事,不會有長期、廣汎的效應。
\subsection*{2020-06-15 11:00}

我覺得階級之間的生育率不同,本身不是問題,但卻是一個很嚴重病態的症狀:中產階級必須花費大量金錢和精神來保障小孩的升學,自然無力多生,這代表著教育系統被金錢扭曲而成爲階級固化的助力,而且未能進入頂尖學校的勞動人民生活水準與上中產階級有太大的落差。我常説貧富不均是21世紀人類面臨的頭號問題,這只是它的部分體現。
這讓我聯想到上周有人批評一位資深的女舞蹈家,說女人最大的失敗是沒一個兒女;有些人包括我自己在内,認爲養育下一代是人生最重要的意義之一,既然女人也是人,這個説法在狹義的邏輯上可以算是正確的,但是一般人的語言不是數學式的純邏輯敘述,那句話的隱藏語義是它對男人不適應,那才是它的錯誤所在。
\subsection*{2020-06-12 00:00}

我寫作時,精神越好、文章就越簡短。寫這篇博文時,一直頭疼,完稿之後總覺得似乎不是最最精練的版本,不過要談的東西太多,也沒辦法了。你如果看出有可以修正補充之處,歡迎討論。
如果你想提的是Sheriff和Police的差別,我的確是故意忽略不談。原因是他們的職權在實質上是同樣的東西,只不過依母單位的層級和地域不同,有不同的名稱和組織細節。而且即使只在名義上,也絕對不是簡單的Sheriff和Police的二分,各州都有自己的獨特花樣(例如有些州警叫做Highway Patrol,有的Highway Patrol又根本不算Police);更別提還有Ranger之類的特警,也是不受州警管轄的獨立分支。此外,聯邦或州的部門機構和甚至非政府組織,也可以有自己的執法力量,實質上就是某專業内的警察。所以出了大事,到場的往往有好幾個互不統屬的單位,光是協調就很花時間;例如《Matrix》一開場,那個Police Lieutenant見到Agent Smith就先急著討論“Juris-my-diction”。那部電影是在澳洲拍的,但劇本卻是針對美國寫的。
美國人去當警察的,的確很多是想當兵又不敢或者不夠格,還有高中時代就是Bully,所以心理原本就喜歡欺負人的。不過這方面我找不到確實的客觀資料,所以在正文裏就省略不提。
在美華人的處境,原本就不好,過去20多年很多基本權利被逐步蠶食(例如名校的入學),現在又有中美脫鈎和BLM運動,就更加不樂觀了。最可笑的是,台灣綠營、香港黃絲和法輪功還在拼命落井下石,鼓動仇中心態,其實美國人對中國的打壓早有他們自己的組織和動力,這些額外的抹黑,最後買單的倒霉鬼只會是住在美國的亞裔,包括法輪功自己在内。
至於波音的外包,網絡上消息很多,你可以自行搜尋。波音被麥道並購(雖然用的是波音的錢)之後,一直對資本市場誇耀外包,所以光是自己的公關新聞稿就以百計。這其中最重要的是Spirit AeroSystems,它原本是波音造機身的分部,在2005/2006年才割裂出去;在那前後,美國商業媒體界有一連串吹捧空客分佈式生產的文章,其實空客是四國合資、不得不依政治需要而做分配,當時我就懷疑是波音公關爲了交待拆分部門(員工福利受影響,有政治壓力)而資助的文字打手作品(“資助”不一定是用金錢,很多記者胸無點墨,全靠復述行内人的意見來寫稿,公關可以免費左右他們的題材和結論,最後反而是記者欠他們人情)。
\subsection*{2020-06-11 09:53}

有些話題,只關事實,不須要太多微妙的邏輯分析,那麽方向點出來,内容細節大家大可以自行搜尋整理。我個性疏懶,遇到這種事往往就一句話丟出去,靜等勤勞的年輕人接手。Liberia是個典型的例子;《觀察者網》動作有點慢,但是成果還算讓人滿意。
五六年前,華語世界就我一個人能揭露這些真相,現在已經好多了。上周《觀察者網》談Koch兄弟的節目就很詳盡;他們事先聯絡我確認Koch真是萬惡不赦之輩,我給了肯定,算是有一點點參與。我從博客剛開始就講明,光我一個人説實話,一點用都沒有,必須影響其他的意見領袖,幾百幾千地反復挖掘事實細節,才有扭轉社會輿論的可能。
我一再强調,不論是歷史、理論、個人或者先進國家,都不能盲目崇拜模仿,必須從客觀環境出發,理性地分析權衡,尋找最優解。這是鄧小平的治理哲學,也是中國能經歷蘇美兩個帝國的大起大落而不斷進步的基本原因。
\section*{【美國】【工業】幾則新聞}
\subsection*{2020-06-21 19:27}

這件事我一直連續關注。中方沒有出手,應該是一方面市場規模不大,另一方面德國去年剛剛有一連串動作要刁難來自中國的並購,這時硬上,徒然碰一鼻子灰。
要改變這個局勢,外交部必須把握當前美國内部的動亂和歐盟外交戰略的轉向,積極爭取更緊密的中歐關係。不過這種事中方一向喜歡私下搞,所以只有在有成果時,外界才能知曉。很不幸,最新的消息是中國與歐盟27國的高峰會被取消了,改爲部長級的會議;這背後的阻力似乎主要來自白左勢力的反對,連趙立堅的陰謀論都再次被拖出來當衆鞭笞,所以中方不用瑞典對患病老人强迫安樂死一事來消消他們的氣焰,是自縛手脚。如果不想自己出面,可以在外交交給亞非小國、在宣傳上通過《RT》來炒作。畢竟歐洲媒體沒有任何實據,一樣還在對中國疫情防治做惡意揣測和抹黑,瑞典的罪行是公諸於世的,他們卻一字不提,歐盟更沒有做任何指責,光是這個雙標就可以大做文章。
\section*{【香港】對八月12日八方論壇訪問視頻的一些補充}
\subsection*{2020-06-19 10:39}

這類事,雙方必然都會想要扭曲事實來爭取同情,所以根本不應該在只有一面之詞的前提下做任何評論和反應,否則就會鼓勵撒謊、誇大和鬧事。
“E-Government的重點不在防民,而在於防吏”,是大家必須在整個體系架構的設計被確定之前,廣為傳播的原則。我在行政上的建言,除了給出正確方案,並且詳細解釋背後的考慮之外,還會盡力想要創造出一句簡練的總結,方便大家把它用爲推廣時的口號。例如“生命不能交由市場來自由定價”,就遠比博文的長篇大論更容易應用在日常討論上。這種“Catchphrase”或“Slogan”是廣告學的重要技巧,像是Nike的“Just do it”、或者BMW的“The ultimate driving machine”都是一句話就有心理效果;我只不過是加入了額外的理性成分罷了。
過去六年來,許多我所做的政策建議,後來都實現了。這其中必然有部分是殊途同歸,中方執行單位在我發表意見之後,獨立地發現最優解。但發生次數太多,所以也可能有中下級官僚能看到我的意見,然後進行内部討論。另一個可能的機制,是其他的意見領袖從我的文章獲得啓發或加强,間接地傳達了正確的觀點。後面這兩種途徑,都會受益於一個簡單易懂、又能觸動人心的口號。
\subsection*{2020-06-19 00:52}

那個長征五號第一級墜落在大西洋的事,是美國一些仇中的網絡小媒體編造出來的。因爲稍有常識的人都知道美國自己年年幹同樣的事,所以英文主流媒體還不好意思提。會在中文論壇炒作這事的,原本就是恨國黨(否則一般不會去關注像是《大紀元》這類的謠言工廠);這些人水準太低,不值得我們花時間詳細討論,不過他們對普羅大衆的殺傷力,卻可以是很大的,長久下來,就是台灣化/香港化。
AI實用化之後,應該會有能力批量處理這些天量的假信息,輕者禁言一段時間,嚴重者罰款或扣分。我想社會信用體系就很適合用來打分。這裏的問題其實在於低層官僚必然會想要利用這種管制來壓制暴露自己無能、貪腐的言論,所以我一再强調,E-Government的重點不在於防民,而在於防吏。
\subsection*{2020-06-18 14:59}

民主黨建制派的宣傳體系一向看不起紅脖子,他們針對的是有高等教育程度的上中產階級白人;這些目標聽衆有若干邏輯辯證能力,所以玩狡辯術有點兒危險,不如直接做選擇性報導,只提供支持己方論點的片面證據。但是在一些美國國内的社會議題上,被忽略的反面證據有時正是紅脖子親眼看到的事件,長久下來,右翼民粹就看出主流媒體是假新聞。這時Trump出現,他根本不在乎事實(所以無所謂選擇性報導),直接玩起各式各樣的狡辯術,這對沒有一絲一毫邏輯能力的紅脖子來説,是難得來自精英階級的尊重,自然會引起很大的共鳴。
勾結有政治關係的當地豪强,是英美企業在全世界都會玩的把戲,只不過會依照他們心裏認知的“文明程度”來決定做多少遮掩。滙豐在中國的作法其實反應了他們對中國法治的認知,亦即不需要避嫌,和印尼這類國家相似。這包含著很深的侮辱和鄙視,中共實在應該再加緊擴大反腐。
\subsection*{2020-05-25 05:14}

“一國”是“兩制”的前提。與外國情報單位合作,試圖顛覆政權,這是公開反叛,早已違反了前提,那麽談“兩制”就毫無意義。
歷史上達賴集團也曾自願當CIA的馬前卒;那麽西藏是如何維穩的,自然就是處理香港的參考。
其實這次的收緊,仍然是很客氣的;香港依舊有自治權,自由自主程度還是遠高於英治時期。你試試建議恢復1990年的制度,看這些吹捧英國的香港“民主派”願不願意。
我前天說,新的香港政策是顯著進步,但依然不是最優,是因爲它只收緊國安,連政治體系都沒有大動,更別提經濟和社會結構;而香港的體制其實到處都是問題,最嚴重的就在經濟和社會:不但貧富極度不均,而且富豪靠尋租、中產靠讓利。這些不勞而獲、少勞多獲的既得利益者不先處理好,必然會反復聯絡外國敵對勢力來試圖顛覆政權,阻撓改革那更不在話下。換句話説,沒有解決問題的根源,一味防堵,必將事倍功半。偏偏中國國内明顯有政治勢力想要繼續利用香港做爲引進英美“服務業”,尤其是金融業,的後門,這是極爲短視的做法:不但對香港和大陸都有强烈的腐化效應,而且將金融開放政策建立在歪斜的基礎上,即使暫時似乎成功,日後也必然會有更大的麻煩。正確的策略是以上海為金融開放的著手點,嚴厲監視、輕鬆管制。
至於香港,先阻止他們對内地實體經濟的吸血,然後拿它來做全民福利和後工業社會改革的試驗田,這才是“一國兩制”的真義。
\subsection*{2019-11-25 13:32}

過去20多年,統戰部門尸位素餐,留下來的債終於爆發,這當然不可能有快速、簡單的解決辦法。我在兩年多前《八方論壇》上曾經説過,統一的正確方案,是由内地派出成千上萬名優秀的年輕幹部,從中央到村里全系列改造社會;這個建議不但適用於台灣,事實上就是在英美宣傳已經根深蒂固的前提,完成精神統一的唯一方法。 
當然,因爲有一國兩制的承諾,這在香港是不可能的。退而求其次,只能盡量維持穩定,然後慢慢改造香港社會最不合理的部分;這將會是一個極爲緩慢、艱難、痛苦的過程。2009年對經濟的過度刺激,是在國際戰略和經濟結構上都極爲愚蠢的決定,結果習近平盡力彌補到現在,房地產市場和地方政府的債務仍然沒有回歸正常;香港統戰單位失職的後果也是一樣的長期難題。
\subsection*{2019-08-29 10:04}

在英美台港的事實證明,媒體不但有影響力,而且對社會規範和國家共識有決定性的影響力。 
在社會規範上,例子到處都是,像是火車上霸座,這固然是違法,而且也需要公權力來執法,但是社會仍然必須建立規範,說大家都認爲這是錯的,才能長治久安;而要建立規範,就是靠媒體上人人喊打。 
至於國家共識,美國是最明顯的例子,連要妖魔化中國,對他們的大衆媒體都是很簡單的事。 
美國媒體對中國的妖魔化,太過高效,協同也太完美,所以我覺得背後是有組織來督促主流媒體口徑一致的。有能力、而且有慣例幹這件事的,是The Council on Foreign Relations,不過我沒有證據,所以不能多説。你自己去找找資料,瞭解一下這是什麽樣的機構吧。
\subsection*{2019-08-27 15:23}

我在這裏談的,不是要消滅普羅大衆看的Tabloid,而是要保護專爲知識份子保留的一個清净角落。 
在Murdoch之前,Tabloid是上不了臺面的,對國家大事也不會做評論,談的是外星人綁架鄉下人或者貓王復生之類的議題。 
《Fox News》的創新,在於它在表面上完全是一級主流新聞媒體的樣子,對一般新聞也如同舊媒體那樣報導,但是在社論性節目裏嚴重夾帶私貨,把謠言當事實來討論,如果沒有謠言,就自行編造一些。 
歷史上,在二戰之前,美國並沒有建立主流媒體的道德標準,在19世紀末、20世紀初,像《Fox News》這樣的宣傳傳聲筒僞裝成中立專業的新聞,是很普遍的。但是在羅斯福任内前後,如同其他行政的組織,美國的新聞業也被道德化和規範化了。 
《Fox News》打破了半個多世紀來的優良傳統,而且用的Motto居然還是“Fair and Balanced”。當然他們並不是真正的中立,只不過是拿口號當藉口,在真與假、對和錯之間找平衡。 
我一直都拿小羅斯福的美國來和習近平治下的中國相比,既然前者可以撥亂反正,建立新的新聞規範,爲什麽後者就必須放任主流媒體自暴自棄,降格去和Tabloid搶生意呢?一個洲級國家,經濟發展上來,有了上億的中產階級,那麽其中自然會有足夠的知識份子來支持一個堅持求真求實的一級主流媒體。 
這裏真正的問題在於錢,而且不是很大一筆錢。《觀察者網》現在墮落的原因,頂多不過是幾千萬人民幣的預算缺口。Murdoch建立《Fox News》,也就花了幾億美元。中國政府要支持前者、打壓後者,根本就沒有資源上的困難,純粹是願不願意做的問題。
\section*{【空軍】【海軍】根深蒂固的誤解(二)}
\subsection*{2020-06-15 01:46}

DCS正是我談“業餘模擬器”時,心中所想的例子。它的氣動和電子戰都比以往的消費者產品高出一個等級,但是低高度和低速度的仿真仍然比較弱。我還沒有研究過微軟的Flight Simulator 2020,不過它的重點在於通用和商業航空,所以應該會更注重起落階段的真實性。
DCS裏,如果選擇只用機炮,那麽對飛行包綫的徹底熟悉就很重要了;不過我看到它模擬的Powered Stall還是遠遠太溫和、太Predictable,尤其是60年代以前的飛機,有不少實際上是很容易進入Spin尾旋,而且很難改出的。
DCS那個JF-17的套件很受歡迎,也是很好的廣告,畢竟真正的飛行員也會想玩高階的業餘飛行模擬器,尤其中東的玩家可能就有不少是空軍的。一開始游戲裏的SD-10性能遠高於AIM-120,結果出版商還必須對兩者的參數都做調整,我覺得蠻有趣的。
\subsection*{2020-06-14 17:05}

不是的。拉漂(Flare)不屬於進場(Approach),後者是從Pattern直綫斜坡降到跑道起端的過程,前者則是飛機到達跑道上,開始感覺到翼地效應(Ground Effect)的額外升力,這時如果你急著觸地(Touch Down),反彈的力量加上升力仍然足夠把飛機重新送入空中,必須拉起改爲平飛一段時間,繼續損失空速,到很接近失速時飛機自然會因爲升力不足而落到跑道上,做得好就是很平穩成功的著陸。
航母著艦當然是沒有Flare這個步驟的,額外的空速沒有餘裕來消耗流失,所以進場的標準動作絕對必須嚴格遵循。在陸地跑道上,不但通用航空的飛行員常常爲了省時間而高速進場,軍用飛機為戰術原因也經常使用更高的進場速度,但是民航客機就不行,不但乘客的性命重要,飛機本身重量大、所需跑道長,而且在進場過程中往往是排隊的,所以和艦載機一樣必須遵守準則。
最常見的民航機出事,是在下雨天滑出跑道,這通常就是因爲飛行員沒有嚴格遵守進場速度,在跑道上空漂浮太久,著陸後刹車又受積水影響,最後只能衝出跑道末端了。
上個月的PK8303在Karachi失事,是更極端的事件:那個機長在轉入進場過程的時候,高度比規定高出三倍多,依舊不聽塔臺建議硬要著陸,於是在油門放空的前提下,空速仍然達到260節,不但比規定的130-140節高出一倍,甚至超過了A320施放襟翼和起落架的氣動安全極限,一路警告鈴聲長鳴,在跑道上空Flare的長度也遠超規定,到了跑道後端才觸地,這時機長才發現起落架根本就被機載電腦拒絕放下,著陸的是引擎,再試圖復飛已經太晚了,受損的引擎在幾十秒後就失去動力,飛機墜落在機場外的市區。
我懷疑這位機長是空軍退役的,所以視標準進場流程為無物。和施洋聊天的艦載機飛行員應該也是在描述航空兵一般的戰術運作習慣而不是標準的技術規範。不論如何,降落過程中的反區操作是很普通的事,絕非艦載機的獨有。
至於飛行模擬器,業餘的很少專注在起落過程上,對翼地效應和地表亂流的模擬都很粗糙。相反的,真正的飛行學員時間主要是花在起落過程和為它做準備上,例如近失速的深反區操作。我在受訓的時候,最討厭的就是Powered Stall,操控界面不但反應反常,而且還出現了明顯的不穩定性,再加上Stall Warning和渦流的風聲在耳邊轟鳴,對一個知道失敗可能無法重來的駕駛員來説,壓力是很大的。
\section*{【科研】如何解讀nCov的傳染病學參數}
\subsection*{2020-06-15 00:28}

是的,世界上多數國家的實際感染人數普遍高於官方確診數字數倍之多,不過也遠低於抗體檢測的反推;這是因爲目前的抗體檢測器材針對Sensitivity來優化,要求它必須顯著高於99 \% ,所以Specificity只有95 \% 以下。換句話說,你即使到New Zealand這種真正沒有多少病例的國家,也會得到至少5 \% 的陽性抗體比率;它的總人口是500萬人,那麽無腦的簡單推論會說是25萬人以上,雖然實際上就是一兩千。我覺得這是醫學院一年級學生就應該懂的道理,有學術論文忽略這個效應只能是故意的。至於用返港的旅客來做統計,只能反推這些人所住的小社區,不能說代表武漢,更不能拿湖北的人口數字來算;否則用觀察到的比率乘上全中國或甚至全世界的總人口,要有多誇張的結論都可以製造出來。
我在一月開始就不斷正確預估了半年後的事實,包括死亡率、重症率、全球總感染人數、會引起社會動蕩等等。請你回去復習博文以及《八方論壇》的訪問,自然能理解真相。我因爲在事先的預估就是準確的,所以不必像其他評論員一樣,有什麽新消息就趕快出來換一套説法,如果我有一段時間不做新評論,往往是因爲事實還在繼續追趕我的預測,所以沒有必要一再重複。
世界上名頭大的人很多,能做正確判斷的卻極少。我花了十幾年做高能物理,只學了兩件事:第一是量子力學的真正原理,但這沒有什麽實際用處;第二是諾貝爾獎得主的公開發言有99 \% 以上是錯誤或虛假的,這後來證明非常有用,尤其是我轉行做金融,而美國學術界如果有比高能物理更加喜歡胡扯的,就只有金融和經濟學了。在許多領域和國家,名教授之所以有名,靠的是開創新的論文發表方向,或者就是頭幾個隨後鑽進這些新方向的人,這和正確描述世界客觀真相沒有什麽關係。
\subsection*{2020-02-07 09:56}

英美上層媒體已經出現了好幾個這樣的討論,態度其實驚人地中肯,基本沒有以往仇中的偏見扭曲。這是因爲這次的疫情兹事體大,本質上是人類與自然界的鬥爭,而且必然會每隔幾年反復出現,有點頭腦的人都知道不適合政治化。至於中外雙方拿它來搞陰謀論抹黑對手的人,更加讓人無語。
正文裏已經探討過nCov徹底流感化的可能性,並且解釋過它比SARS和MERS都要棘手得多的原因。一個傳染力接近流感的全新病毒,防治上的難度原本就至少和流感相當(因爲醫療體系已經針對流感而優化過了),而現代歐美對流感基本是只能打疫苗然後聽任自然,事後再統計死亡人數。我以前説過,這裏再强調一次:nCov如果出現在任何一個其他國家(包括美國在内),當地政府都不可能有所需的組織力和行動力來試圖隔離控制它,所以必然早就傳遍全球了(當然,中國沒有美國那樣對活體野味的管制,所以新病毒出現的機率更高,nCov在中國發生,不能歸咎到厄運上;但是可以部分怪到陰謀論者頭上,因爲他們消耗了做出正確政策對應的動力,世界上蠢人蠢事都很多,我們不必事事計較,但是妨礙未來防疫的蠢人蠢事必須大力撻伐)。中共固然是世界上獨一無二有直面nCov能力的政權,但是也只算勢均力敵,絕對不是穩操勝券。(最新的疫情報告顯示2月6日的新增確診曲綫終於向下突破了1月22日以來的直綫上升趨勢,如果能持續下去,這可能成爲防疫勝利的轉折點;不過不確定性仍然很高。)
因此你擔心疫情失控之後對政府信用有打擊是合理的,但是我想中共政府會顧忌實話實説可能進一步促進民衆恐慌。中國自古以來的政治理想,是把民衆當小孩子來關照的;這在原則上是對的,事實上群衆的確既沒有能力也沒有理性來好好處理公衆事務。但是正因爲他們的集體政治智商只有小孩子的程度,情緒上也會期待哄騙安撫。以往通訊不發達,人民的生活期望又低,所以“民可使由之,不可使知之”行得通;現在互聯網使謠言起鬨每隔幾分鐘就進化一代,中產階級衣食無憂、閑得發慌、成天想方設法花錢找樂子。英美在哄騙民衆這方面真正是先進國家,經驗老到豐富,連整個政治體制都是設計來創造群衆直接主導的假象。當然這樣做的最終後果是富豪可以收買媒體、挾持群衆來控制政府,所以並非適合亦步亦趨地模仿;但是中共的宣傳教育體系這幾十年來實在爛得不像話,所以連在這種情況下實話實説的選項可能都不存在了。
\subsection*{2020-02-06 22:42}

正如你所説的,目前這是羅生門。其實在重大突發意外事件的過程中,幾千個組織參與下,有一些不協調、慢半拍、或各自有非外人所能理會的特別專業細節考慮的,不但是難免,而且是正常。
我不認識武漢病毒所的所長,也不知道她做決定之前考慮了些什麽(可能包括醫藥、公關、法律、現實可行性等等),但是Remdesivir我懂,可以在這裏解釋一下,應該會幫助解答你的疑惑。
新病毒藥物的標準研發過程如下:第一步先在培養皿裏試試能否抑制目標病毒,如果有效用,那麽第二步是人體安全實驗,看看是否有不可接受的副作用。如果安全性過關,那麽第三步是對病人做有效性實驗,觀察在人體内是否仍然能抑制病毒。Remdesivir是一種理論上可能適用於所有RNA病毒的實驗性藥物,最早是針對Ebola來做研發測試,通過了第一和第二步,但是在第三步失敗了。其實這並不奇怪;病毒在培養皿裏無可躲藏,在人體卻有很多角落不是藥物能輕易碰觸的,所以大多數的實驗藥品會死在這一關。
這次nCov樣本被廣爲分配給各個機構之後,很多既有的抗病毒藥物都被拿來對其做實驗。目前已經通過第一和第二步的有兩種,也就是Remdesivir和Chloroquine(氯喹)。後者是很老的藥,專利早已過期了;只有前者才有專利的問題。但是我要特別再强調一次,第三步才是真正的難關,而目前根本就沒有足夠的時間來完成人體有效性實驗,所以Remdesivir和Chloroquine會有效的機率實際上並不大。但是目前的客觀背景是沒有任何已知有效的藥物,而Remdesivir和Chloroquin至少已經通過了前兩步實驗,副作用可以接受,所以依照“死馬當活馬醫”的邏輯,可以姑且試一試,這在美國叫做“Compassionate Use”。
至於像是泰國或中醫宣稱的用某某藥物“成功治愈”的病例,考慮nCov的致死率在2 \% 左右,也就是有98 \% 的人不用任何藥物也能痊愈,那麽根據一兩個例子來做結論,根本沒有任何意義;反過來看,會做出沒有實際科學意義的論斷的人,他的誠信和專業能力都非常可疑,所以反而是應該避免的警示信號。
\subsection*{2020-02-04 10:01}

我覺得還沒有到可以準備慶祝勝利的時候。
我每天去看中國疾病預防控制中心的報告:http://2019ncov.chinacdc.cn/2019-nCoV/
請注意網頁下方,全國病例統計的圖表,然後只看“新增確診”。當然各地進行檢測、統計、上報不一定是連續的,所以天與天之間會有一些人爲的波動,我們應該專注在大趨勢上。可以看出在1月22日有一個向上的轉折(春運!?),這代表著病毒突破包圍圈,開始成指數擴散。但是隨即這條綫變成直的了:上彎是病毒占優勢,下彎則是人類開始壓縮疫情,綫性增長就是兩方暫時勢均力敵,隨時有可能向上或向下突破。
目前nCov基本還被局限在中國境内,外國只須要處理個位數的案例,這是相對來説非常容易的事。而中國有20個省級單位感染人數超過100,其中河南超過1000,但是主戰場還是在湖北。同時春節假期結束,全國防疫的壓力又升高了;未來這兩周會是關鍵。
\section*{【戰略】談中共修憲}
\subsection*{2020-06-14 09:09}

我想中美夫妻論和想要照搬外國體制的崇美派還是有根本性不同的;後者是真正接受洗腦,前者卻只是認爲美國不會對中國下定你死我活的決心。
雖然崇美派對中國的政策有些不良的影響,他們在過去20年安撫美國的貢獻卻是實實在在的。美國從1990年代的Engagement戰略,到2010年的重寫國際規則,其基本假設都在於能忽悠或逼迫中國精英主動放棄自主思想路綫,采納英美制度,以方便國際資本的掠奪。中美能撐到Trump上臺才正式撕破臉,這些人在很長的時間裏,讓美國軟實力陣營一直有希望的寄托,功不可沒。
赤字貨幣化是國際儲備貨幣的特權,我想人民幣還沒有那個基礎。
中共的智庫,一般只能討論政策的執行細節,像是如何消弭階級鴻溝和城鄉差距這類深刻的政治哲學議題,似乎沒人敢談。就連在學術界和企業界建立誠信規範、嚴懲詐騙造假,都沒有推動的聲浪,這的確是一大隱憂。
\subsection*{2019-10-31 10:02}

我一直不想對中國政府的經濟政策做過於詳細的討論,這是因爲我人在美國,閲讀的文章以英文爲主,而且大陸媒體對政策的報導非常不可靠,就算花時間去讀,也會是在垃圾堆裏尋寶。
所以我只能從很高、很遠的觀點來考慮最宏觀的現象。中國當前的困難,的確是源自胡溫任内的錯誤,尤其是沒有做結構改革,反而在金融危機後過度發債,進一步促成金融資產泡沫。習近平的團隊從上任以來,始終在盡全力彌補這些錯誤,一方面推行結構變革,另一方面要為泡沐消氣,但爲了同時維持穩定,步驟很緩慢謹慎,GDP增長率被很小心地逐步壓低就是一個體現。
我認爲七年下來,他們的工作還只完成了一半;換句話說,你所列舉的段落,並沒有原則上的錯誤,不過中國經濟的升級轉型是否成功,還有待觀察。
\subsection*{2018-03-26 05:58}

你既然熟悉Piketty的研究,那麽問題就出在你沒有仔細看我的評論,那裏的重點是民主制度和貧富均衡沒有直接的因果關係,所以即使你簡單得到的表面上(Nominal)Correlation是正值,仍然必須視因果樹的形狀來決定民主制是幫助還是阻礙貧富均衡。 
我已經在好幾篇文章裏解釋過了,實際上的因果樹是一戰和二戰創造了戰後貧富均衡的西方社會(這也是Piketty的結論之一),也同時創造了普選制的潮流。所以Nominal Correlation必然是正的。然而70多年來,普選制越走越極端,貧富差距卻越來越大。你的簡單Regression給出與事實相反的信號,就在於1)你只測量了Correlation;2)你只考慮不同國家這一個維度;3)你的主動變數只用了民主相對於所有其他制度,那麽很明顯地犯了我一再討論的籠統歸類的毛病。事實上,國家這個維度本身就是很糟糕的選擇,因爲大小、文化、歷史、資源和制度的差異極大。如果你考慮了至少時間這個維度,應該就會得到異號的正確答案。 
學術界只管發論文,所以可以做錯Regression,反正正負號搞顛倒了也沒人在乎。做金融的,絕對不能把Correlation當作Causality,否則就從賺錢變成賠錢。
\subsection*{2018-03-12 09:19}

羅斯福打壓反對派,主要發生在第二任上(1937-1941),與最高法院的鬥爭是其中的重點。一旦美日開戰,美國民衆基於愛國的理由,任他予取予求,連把日裔美國公民全部關起來這種明顯違憲的事,都不再有任何反對的聲音。
有關羅斯福和最高法院的鬥法,你可以看看Marian C. McKenna的《Franklin Roosevelt and the Great Constitutional War: The Court-packing Crisis of 1937》。
另一本書是John T. Flynn的《The Roosevelt Myth(1953)》,但是這個作者迷信自由主義,所以有很多針對新政的攻擊;你可以忽略政策上的辯論,專注在羅斯福如何不擇手段、玩弄權術的那部分。
\subsection*{2018-03-09 18:02}

其實美國到1968年才開始黨内普選,但是至今仍然不是一人一票:例如Democrats就有所謂的Super Delegate制度,簡單地說,就是黨内大佬一人相當十幾萬票。
最後的大選雖然有規則,人民的選擇卻只能從兩個大黨的候選人中來挑;至於這兩黨的候選人是怎麽挑出來的,法規上基本沒有限制,例如Hillary在2016年掌控了民主黨機器,作弊盡其能事,事後不但沒有法律責任,連要民主黨主席辭職,都是因爲Hillary敗選才可能。
我並不是說如果民主直選制度貫徹到初選,就能解決西方民主制度的缺陷和困境,而是美國實際上自我矛盾、言行不一,民主直選根本就是個糖衣假象,Hillary這樣政客搞的花樣和古代亂世裏世家貴族之間的政治鬥爭手段並無不同。
\subsection*{2018-03-09 17:53}

英國的階級社會,還是沿襲900多年前Norman Invasion的遺產,上層社會人口其實很小,卻控制了政治、經濟和社會的主導地位,所以沒什麽天賦的人,只要欲望和意志堅强,也可能掌權;Churchill就是這樣的一個人。
這些上流社會的子弟,畢業後從政的第一步,常常是當記者(例如Boris Johnson也是);退休之後,則可以當主編(例如George Osbourne)。這是英國政壇的特色之一,Churchill就是這樣出頭的。做新聞的見多識廣,原本並不是一個不好的踏脚石,問題是有野心的年輕人爲了脫穎而出,會嘩衆取寵,鼓吹極端的(通常是右翼)民粹主義,Churchill是這樣、Boris Johnson也是這樣。
Churchill的能力很差、欠缺理性(Boris Johnson也一樣),一戰中當海軍部長,在Gallipoli害死幾萬自己人,本來應該做爲過氣政客,從歷史消失,但是他運氣好,在二戰遇到Hitler這種史上罕見的狂人蠢蛋,讓他的强硬政策看來有道理,得以榮登大位,結果是把霸權拱手讓給羅斯福。這也就難怪美國的宣傳機器到現在還在緬懷(例如電影《The Darkest Hour》)這位“歷史偉人”了。
儒家思想用現代語言來説,就是人本主義(Humanism);我以前也提過,我自認是一個人本主義者(Humanist)。
\subsection*{2018-02-28 16:12}

徒法不足以自行,法治的背後最終還是人治,這個道理我寫了好幾篇專文,在留言欄也已經反復論證;你還是來這裏空喊口號,唉。
你們這些被美國宣傳洗腦的公知,有空去研究一下羅斯福在1937年提出的司法程序改革法案那段歷史。他在第二任上已經完全收服了兩黨的國會議員,就因爲最高法院膽敢阻撓新政的一些條文,他把那群最高法官也鬥臭打服,然後得以無限制地擴展行政權力。換句話說,美國人能搞絕對法治,是因爲他們有全球霸權(所以效率低、圖利財團沒關係,到國外再去掠奪就行了);但是這個全球霸權,卻是靠羅斯福把法治踩在脚下而建立的。羅斯福還不須要反腐,他的任務比習近平容易多了。
至於個別政客的八卦新聞,我沒有興趣。世界很複雜,我的時間是用來抽絲撥繭,找出歷史的主軸;而不是花在那些別有用心的人可以輕鬆炒作的花邊上。
\section*{【歷史】【軍事】現代諸葛弩}
\subsection*{2020-06-02 09:12}

這個穿越想象,我自己並不太好意思多談,因爲全狀態的二衝程滑動弩離不開複合弓的平坦拉力函數曲綫,而複合弓本身對現代的材料科學和精密加工依賴很大,不像Sprave的滑動支撐是簡單木工結構,中古時代的巧匠至少在理論上有可能獨立開發出來。 
當然,如果忽略複合弓的現代技術成分,假想這種性能的連發弩在古代就發明,那麽對歷史走向絕對會有決定性的影響。中國3000年的可靠歷史,一直是北方游牧民族輪流入侵,並且多次建立王朝,而這些游牧民族的人口數目和經濟產能都遠低於中原的農耕族群,他們的戰力優勢基本完全來自馬匹所賦予的機動力、衝擊力和防護力(騎兵更方便披重甲)。正因爲騎兵是冷兵器時代的王牌兵種,歐亞Steppe不斷向外做武力輻射,一直到早期的Musket大規模裝備,俄國人才得以對草原騎兵取得優勢,並將他們徹底征服。這個二衝程滑動弩的性能全面壓倒早期的Musket,在中原漢族王朝的手裏,基本可以對匈奴、鮮卑、柔然、突厥、契丹、女真、蒙古等族的侵略免疫。
\subsection*{2020-06-02 03:03}

重弩一般需要脚蹬固定,然後用轆轤開弓,即使是很熟練的弩手,一分鐘兩發就謝天謝地了。二衝程滑動弩瞬間射速應該可以達到兩秒一發,真正的限制反而來自肩背肌肉的恢復能力,實際上的長時間持續射速可能不比傳統重弩高多少。 
520米只可能是最大射程,亦即朝天45°左右發射的結果,既無準頭、也無存速,在實戰上只對輕甲兵集群有點騷擾作用,一般情況下是不值得這樣浪費箭支的。電影/電視導演喜歡它的視覺效果,可是中世紀的畫作裏,弓箭手永遠都是平射,有效射程基本在100米以内,而且對手的甲胄越高級,有效射程就越短,如果是一流的板甲,那麽只有最強的弩在極近距離才能射穿正面。 
弩的確是宋兵依賴的主要利器,不過它只對遼和西夏作戰時比較有效。金朝早期的馬匹載重力極高,容許鐵浮屠穿戴雙層甚至三層甲,大幅壓低了宋兵强弩的殺傷力,遼軍的反曲弓就更加沒轍,這是遼和北宋迅速滅亡的主因,也是武器的性能細節決定歷史大方向的一個例子。
\section*{【政治】政府的第一要务}
\subsection*{2020-05-15 20:20}

後知後覺的,不只是老百姓,其實絕大部分的所謂“學者”也是一樣的,因爲他們也是官僚體系下的產物。
官僚系統最嚴重的副作用,就是制度和規律僵硬化,結果得利的必然只是表面上願意無條件接受所有不合理規則細節、實際上為私利而鑽營的小人。真正尊重事實和邏輯的人,則必定無法完全接受不合理、低效率的現象,也就會被系統自動排斥。然而官僚體制是維持穩定和傳承的基礎,不能一舉推翻。
所以解決官僚惰性的方法,是任命實幹的主官,並賦予做出實際重大貢獻的確切任務。在任務導向的前提下,不斷修正既有制度的缺點,推動人事變動和法制改革。
Trump的四年任期和最近的新冠疫情,大幅加速了美國的衰退和中國的崛起進程,所以改革的迫切性也被凸顯出來。習近平八年來做了不少事,但是組織、制度和社會文化上的問題還多著很,改革的步調不但不應該減緩,反而必須更加大刀闊斧才對。
\section*{【軍事】從美國看閲兵}
\subsection*{2020-05-10 01:13}

我看不懂你的問題。 
現有的反彈道導彈系統有很大的局限性,只對中短程、自由彈道和數量有限的非飽和性攻擊有效。 
武器效能決定戰術優劣,戰術優劣決定戰略形勢,戰略形勢決定歷史進程。而武器中的矛與盾,往往會有此消彼長的長期波動現象。我們現在正處在導彈的矛相對於反導系統的盾逐步占優的一個歷史階段。這是因爲一方面手機部件的技術發展和批量生產,再加上中俄的新導航衛星系統,使得導彈的精確制導極端普遍化和廉價化,可以簡單地用飽和攻擊壓倒防禦系統;另一方面高超音速滑翔彈頭的發明,不但延申了射程,也使既有的預警和反導系統形同虛設。所以我們又囘到冷戰後期,戰略武器的投送基本無法阻擋的局勢;這其實是有利於互相嚇阻,避免單方無限升級的和平基礎。
\subsection*{2020-01-04 10:52}

這次Trump的決定,是極爲高明的戰略和戰術運作,不但比以往的一般操作要聰明很多,而且打破了他原本對軍事行動的心理障礙;一個70幾歲的人能有這樣的進步,實在很難得。
伊朗國力十分衰弱,内部又不能齊心齊力,別説引發大戰,就是要對等反擊都很勉强。尤其是Trump這個打擊手段,拿捏得恰到好處,不但對伊朗的虛名和實力都有足夠的傷害,而且並沒有動到對方的國土和正規軍,所以伊朗很難找出不繼續升級的有效反擊。最可能的是零星地、小規模攻擊美國的軍民據點,勉强挽回一點顔面。
如果真的如此草草收尾,Trump的聲望自然大增;要是真打起仗來,美國的愛國教育更會讓民衆團結在總統身邊。總之,Trump的支持率,少則增加2 \% ,多則上升10 \% ,他穩居勝算。當然,這種好事是以國家長期利益為代價的,但是我的讀者群應該對英美政客犧牲國家來爭取選票的操作十分熟悉才對。
我個人對這件事的主要反應,其實是納悶現在Trump會不會繼續對德國打貿易戰了。照理說已經不再有政治需要,但是他也可能習慣成自然,依舊和德國翻臉。換句話說,我不能確定他這次的智商暴漲是特例還是新趨勢,我們繼續觀察吧。
\subsection*{2019-10-03 14:02}

這是謠言。我已經論證了飛行高度在50公里左右,不是150公里。馬赫3-5並不是高超音速,在50公里高空,除了機頭,發熱並不嚴重,機腹下有光學和紅外照相窗口是必然的,在閲兵過程裏也看不到。空空導彈的速度一般在馬赫4以上,而且飛行高度很低,空氣密度高出百倍,紅外窗口一樣能裝,還裝在彈頭正前方(這是因爲用的是石英,隔熱能力絕佳),你就應該可以推論出這個作者純屬胡扯。其實整段話裏,連一句實話都沒有。
我一輩子堅持事實與邏輯,所以對那些以撒謊騙人爲樂的人,他們的心態我實在無法設身處地地感受瞭解。不過客觀來説,我知道他們不是少數,中國尤其多(或許是因爲一步就從農業社會跨入網絡時代,社會規範還來不及建立),你們閲讀大陸網民的文章時,應該像是對待香港報紙一樣,先嚴重存疑。
至於海軍還有好東西不能展示,這是當然的。一輛超重型卡車也就能拖個幾十噸的負重,海軍的裝備動輒以千噸計,自然不可能上街。這一點我在正文裏也提過了。
\section*{【美國】【工業】波音和載人登月}
\subsection*{2020-05-07 14:33}

這裏是因爲中國航天目前只立項了空間站,長征5、6、7都是在這個大目標下發展的,順便改用安全環保的燃料/氧化劑。既然以往的長征2F要淘汰掉,那麽乾脆開發長征5的載人型,反正它的進度還比長征7要快些,也方便一次送更多人(或者人+貨)來往空間站。
我在正文中談的是把空間站的優先度降低、同時搞載人登月。當然現在有了新冠疫情,已經沒有必要再走那條路。
這次發射有兩個載荷,主載荷是新型載人艙的原型,成功了;次要載荷是另一個實驗性的載貨返回艙,這個失敗了。後者用廉價的柔性吹氣隔熱障壁;最早是俄國人的腦洞,但是沒搞成,後來NASA成功了兩次,不過都只是次軌道實驗,亦即是用小火箭送到200公里高度,然後重返大氣層。200公里高已經算是衛星軌道高度,但是NASA的實驗起點沒有什麽初速,實際上要從空間站載貨返航的初速會是第一宇宙速度(大約7.8km/s),難度要高得多。所以中方這次是人類史上第一個做全難度的柔性吹氣隔熱障壁實驗,不幸失敗了;我相信他們在兩三年内會再嘗試。
\subsection*{2019-08-04 12:40}

首先,高功率軍用鐳射是非常非常困難的技術,即使是現代,美國軍方已經至少從2010年左右就大幅投資,至今第一階段目標,也就是穩定的100KW鐳射還是沒有實用化(大概還要兩年吧)。在幾十年前則更糟糕,像我們視爲理所當然的聚焦、維穩等電子光學技術,通通不存在。所以一直到Reagan任内搞Star Wars這個大忽悠,才提了出來。 
蘇聯在1960年代,絕對是要和美國爭著先載人登月的。後來失敗,有兩個原因:1)航天團隊内部鬥爭太激烈,一旦總師Sergei Korolev在1966年因病死亡,技術和人事問題就一起爆發;2)原始設計采用現在SpaceX的路綫,用了二十幾個中推,無法達到夠低的容錯率。 
至於James Webb望遠鏡,問題不在載重,而在於體積。有了100+噸級的火箭,整流罩就能做到30米寬(整個望遠鏡展開後,最大長寬高20+米)嗎?我覺得是不行的。如果可以,那麽用20+噸級的火箭先送上LEO,然後再補充燃料,飛離地球軌道,一樣會是節省一個數量級的費用。 
SpaceX有一群受夠了NASA官僚習性,而只想安安靜靜做火箭的傑出工程師,所以有若干技術突破,是很自然的。但是他們仍然不可能超越物理和工程規律;Elon Musk總歸還是圈錢大師。
\subsection*{2019-08-04 08:19}

NASA現有的載人登月計劃,就包括一個繞月空間站,叫做Lunar Orbital Platform-Gateway(LOP-G)。它走一個高度橢圓形的軌道,最高點就在地月轉換點附近,所以從地球來的飛船,可以相對高效地在那裏對接,然後人員在最低點轉搭著陸器。
不過這種設計,只有在多次反復使用的前提下,才有效率。如果只是偶爾用一次,反而不如由飛船自己直接轉換軌道。所以它假設:1)月球上有一個人員常駐的站;2)空間站和著陸器都很結實耐用。前者我討論過了,沒有經濟理由;後者依太空梭的往例,基本不可能做到。太空梭還可以在地面整修,這個Gateway可沒有這個餘裕。
現在國際空間站上的人員,雖然沒有大氣層屏障,但還在地球磁場保護之下。即使如此,所受的輻射綫是一般人在海平面的400倍。載人登月就只是做廣告,一兩次就夠了;有實際的工作需要,還是交給機器人吧。
\subsection*{2019-08-04 01:26}

NASA最大的貢獻,對美國是載人登月,對科學是一系列的太空望遠鏡。但是後者的所謂科學,是純基礎科研,完全沒有任何經濟上的效益,所以你在同一個段落裏,先鼓吹純科學研究,然後以解決房市問題收尾,是有些自相矛盾的。 
載人登月是廣告,廣告的對象是一般群衆,而一般群衆先天就無法理喻,否則絕大多數的廣告既無必要,也無效果。但是這不影響廣告可以產生實際經濟效益的事實。 
我的人生態度,是堅持理想原則,但仍要追尋對現實世界最大的實際影響和貢獻。前者是理想主義,後者是實用主義。請注意這兩者是絕對相容的,因爲理想主義的相反,是現實主義,而不是實用主義。你先走理想主義,再換現實主義,從兩方面來批評我的實用主義立場,在邏輯上既不自洽,也不相干。 
你有死纏爛打的壞習慣,我先警告你,這事我已經盡力解釋清楚了,除非你有合乎邏輯的新論點,否則到此爲止。
\section*{【基础科研】丁肇中与高能物理界的牛屎文化}
\subsection*{2020-04-18 23:09}

饒毅先生的文章,昨天有人私下問過我了,我說我很贊同。正如你所説,這似乎是楊先生阻斷國際高能詐騙集團財路的又一個翻版。短期内,只能靠有操守的大佬願意爲國家得罪同行;中期則必須讓騙錢失敗的人付出代價;而這只是整頓學術界文化的長期努力的第一步。
很不幸的,肯公開說實話的行内人,包括楊先生、饒先生、趙午教授、還有我自己全都是國外學術界出身,這似乎指出中國學術界完全沒有自清的能力。我想並不是國内沒有良心人,而是逆淘汰的效率太高,以致這些良心人早已被體系排斥出核心。
在這個問題上,專業知識的障壁只是次要因素,真正的關鍵在於詐騙企圖被揭穿之後毫無負面後果:漢芯的發明人不但沒有坐牢,連職位都還在,詐騙所得也得以保留。所以說不能解決只是推諉責任,真正的問題在於主管官僚出於無知或無恥不願意解決。好消息是這種不作爲的問題,高層可以直接下令改變現狀,壞消息是他們似乎還不理解此事的嚴重性和重要性。
\subsection*{2017-09-20 00:00}
I have no clue about the proper channel of self-introduction. They contacted me.

On the other hand, I did self-introduce with 《China Times》. I simply wrote a letter (yes, snail mail) addressed to the chief editor and sent another copy to the president. I got a positive reply back from the vice-president after 3 months.\section*{【美国】【戦略】美国的欧洲代理人板块重整}
\subsection*{2020-04-15 23:23}

招收留學生以增强外交影響力雖然源遠流長,現代的版本最早是英國人在20世紀初定型示範的,大家或許聽説過,叫做Rhodes Scholarships。後來美國取代英國的霸主地位,反過來吸收西歐人才,這在二戰後歐洲一片凋零的環境下,效果特別好。除了你提的這個計劃,其實我以前討論過的哈佛甘迺迪學院也是同樣的道理。
請注意,這些成功的方案都是針對研究生以上的頂尖人才,可以確定是未來國家政經中堅力量。甘迺迪學院更進一步,專收已經當過政務官和事務官的在職進修,所以效益更直接、强大。然而這些手段是以世界一流的學術研究為後盾;中國學術界的腐敗,在這裏又一次限制了國家的戰略選項,隱性的損失代價實在驚人。
中國的學術界只夠引進第三世界較年輕的精英,並不代表就可以濫收。留學生後來反咬一口,是真正的危險。近代有山本五十六留學美國的例子,羅馬帝國則更慘,公元九年被自己軍隊裏服役的原小留學生Arminius引誘到Teutoburg Gap(拉丁原文“Saltus”有很多不同的意思,以往被翻譯成“森林”,這在近年Teutoburg古戰場地點被德國考古學家確定之後,已經證實是誤譯,它其實是在山丘和沼澤之間的一條小徑,考古學家建議翻譯成“Gap”或“Path”,類似華容道裏的“道”),埋伏好的日耳曼蠻族一次就消滅了三個精銳的軍團,屋大維被迫撤守萊茵河一綫,羅馬不但從此無力再進軍現代德國的地界,而且西羅馬就是亡在日耳曼蠻族(Goth)手裏。
所以英美學乖之後,真正引進的蠻族精英專注在政治和經濟兩方面,而且有意無意教的是有毒的自由主義歪理,這對遏制第三世界國家的工業發展有很大的貢獻,例如中南美洲的經濟金融政策就是以芝加哥學派爲主。
至於對外國人的超國民優惠,這不但是史無前例,也毫無道理,對外籍學生和本國國民都有極度惡劣的心理影響,我實在想不出有什麽論點可以把它和國家民族的利益連到一塊,只能說和中宣部和教育部對國家的自我傷害是同一類的詭異現象。
\subsection*{2020-04-14 19:07}

你的分析過於樂觀,尤其是假設各國政策都基於理性考慮,來做國家利益的最大化,這和實際情況完全脫節。現實裏,英美在全世界推行自己的普選制,在外交層面上的用意,就是藉由這個制度下欠缺紀律、獎勵自私的特性,方便維持各國對英美霸權以及他們背後的國際財閥的非理性依賴。德法内部的有識之士,當然明白和俄國和解是正確的方向,但是英美對他們政經社會(尤其是德國)幾十年的經營,有著從媒體到金融到智庫到學術界到地方政客一系列的帶路黨,即使總理不是英美的人,他/她的幕僚也必然充斥著間諜/買辦,這是爲什麽他們始終無法擺脫美國霸權桎梏的原因。
日本則更進一步,從文化上就認同白種人的“優越性”。你沒注意到他們的漫畫裏,日本人/好人都是高鼻、深目,甚至金髮、藍眼的嗎?
\subsection*{2015-09-10 00:00}
I don't think it is "incredible kind". It is actually quite a pragmatic solution.

Most of the refugees want to go to Germany anyway. There is no way to stop them without producing a lot more cases of front-page news on tragedies. By taking the lead and urging the whole Europe to act together, not only does Germany get other countries to share the burden but also reaffirms its leadership position within EU.\section*{【经济】美式经济学是骗人把戏的又一表徵}
\subsection*{2020-04-06 01:56}

《UDN》”新版“的bug很多,其中之一是如果你留言裏的網絡鏈接太長,就會搞壞網頁的版式。讀者不常用,不知不罪,但是這對我是個困擾,請大家盡可能避免。 
美國經濟系的“學人”轉化為財閥土豪的文字打手,其實源自Rockefeller設立芝加哥大學,1929年搞出大蕭條之後暫時噤聲,在二戰後好了瘡疤忘了疼,在Milton Friedman領導下學術賣淫業重新開張,比Murdoch進軍英美媒體界、游説業的興起、右翼智庫的建立和擴張都還早了20多年。早期(1960、1970年代)還有爲了名利而同流合污的,最近30年已經完全洗腦成功,對整個行業完成了逆淘汰,比超弦席捲高能物理還要徹底。 
台灣和香港是英美宣傳洗腦的重災區,社會的理性傳統一旦消失,就必須等所有人口自然死亡才能改變,這至少要兩代人的時間。我一再强調台灣最需要的是教改,就是希望挽救還在學校裏的那一代。
\section*{【宣佈】我計劃搬囘臺灣謀求教職}
\subsection*{2020-01-16 11:29}

台灣人民的集體迷思和迷失,主要有兩個心理動力。第一個是你提的這種藉著改寫歷史和扭曲自尊來挑起獨立自決的欲望,是李登輝的絕技,這在輕家國而重鄉土的南方食米區(參見《訪意大利有感(一)》)特別容易生根。第二個則是以馬英九為代表的腐儒民主論,這是白左的理想國,原本就是美國資本因爲完全掌控媒體可以自由左右民意,所以要强調政客的奴性,用來弱化理性的社會和政治力量。在台灣,這更賦予民衆一個針對大陸的優越感和隔離感,加强了獨立的意識,所以我說過,馬英九這一套其實是教育台灣年輕人當臺獨,那麽國民黨的基本盤自然越來越小。換句話說,國民黨的衰亡,在蔣經國容許黨内精英一味精神崇美的時候,就已經注定了。而現在大陸最大的發展優勢,正是鄧小平當年同樣搞現代化,卻沒有被美國宣傳衝昏了頭,照著白左的政治正確標準或財閥的放任經濟理論來執政。 
世界上政治學美國最虔誠的,是Liberia,這是當年一些美國黑奴重返非洲建立的國家,從憲法、法律到行政體系都是完全照抄。如果美國制度真的完美,Liberia現在應該是非洲第一强國了,但是現實是Liberia在西非都算是落後國家。這麽簡單的事實證據,馬英九和龍應臺都看不到,說他們蠢還侮辱了那兩條蟲。 
台灣在過去20年,綠營縣市的中小學老師早就不敢批評臺獨;蔡英文執政之後,更是有系統地清洗高等教育機構,反臺獨的教授先後賦閑在家,這次大選之後必然會變本加厲,所以我是否還能教書很難説。智庫反而比較安全,我不是已經開始討論糖尿病這類中性題材了嗎?我做人的原則不容許違心之論,但是非政治性的公益議題還是有值得努力的。
\subsection*{2019-05-24 21:49}

我最近忙,沒有時間去細想這類的大題目,不過只從你的留言來看,這個論述的視野還是稍微狹隘了一點,畢竟客觀背景環境還是很重要的。 
這裏背景環境又包括對外的國際關係和對内的社會結構兩個方面,前者主要是必須知己知彼,戰略戰術都合宜,後者則是要不斷前瞻,避免社會僵化腐朽。兩者都需要持續、深入而踏實的研究,既不能照抄外來的理論,也不能死抱舊有的經驗。中共在過去40年做得不錯,但是現在所面臨的挑戰又提升了一個層次,所以一些弱點和不足就暴露出來了。其實這正是我寫作博文的主軸題材之一,像是智庫研究不夠精確踏實,教育制度不合理、鼓勵階級固化,地方官員賣國求利、以致有些高科技工業的產業升級事倍功半,等等,我們都討論過了。
\section*{【海军】055级的设计概念}
\subsection*{2020-01-14 18:19}

這是一個不入流的噴子在玩弄語義上的吹毛求疵,不用理他。我說100MW,是指到2030年左右,055改爲全電後的總電力供應,用詞適當精確,他故意扭曲解釋,自説自話,再加上下面幾條無關緊要、不知所云,獨立來看是事實,但和話題全無邏輯相干(中國的渦輪對高溫適應性不好,哪個軍迷不知道?但是我文章哪一句話沒有考慮這一點了?),都徒然表明他的用心和智力都十分低下。 
我的文章是五年前寫的,當時就引發軒然大波,因爲沒有一個大陸軍迷願意相信第一批的055就會有20MW的供電;現在已經成爲主流估計了。2015年時,每個軍事論壇的大神都反復猜測055的設計,如果現在有人覺得我預測得不好,請找出任何一個預測得更准確的例子,尤其是這位事後諸葛亮自己在五年前公開發表的評論。 
我一向强調,評論水平是由事先預測的準確性而不是事後的硬拗來決定的。這個博客的文章基本都是在做事先預測,而六年下來錯誤率在10 \% 以下;什麽噴子都可以當網紅,但是要做科學性討論,請先證明自己也有至少50 \% 的預測正確率,否則就是浪費大家的時間。這個傻蛋事後來挑刺,還只是這個水平,可見他就是習慣在網上滿足想要浪費時間的那群聽衆的欲望,這和咱們可沒有交集。
\section*{【美國】簡評民主黨彈劾Trump}
\subsection*{2019-11-15 08:16}

我並不迷信任何一個個人的寫作,因爲我有自己的雙眼和大腦,可以自行依事實和邏輯得到正確的結論。 
馬克思對資本主義欠缺公平合理性的批評,是到位的,但是像“資本主義的趨勢是長期利潤率下降”這樣的論述就是明顯錯誤的。我看到的21世紀被動資本投資報酬率(也就是不參與主動經營),依舊是5-8 \% ,這和18、19世紀並沒有本質上的差別。實際上是,任何特定產業遲早都會飽和,然後利潤率下降,但是整體工業社會有不斷的汰舊換新,新的產業出現的頻率甚至還是隨時間而增高的。做真空管的,利潤早就消失了,但是半導體工業起而代之,所產生的總利潤遠超既往。 
階級鬥爭是一個更大的誤區:不論出身如何,一旦有了政權,就自然成爲新的最上層階級,無產階級也不例外。不管經濟路綫走的是市場還是計劃,不管掌權者來自投票、指定、還是父死子繼,他都會面對無數誘惑要侵害國家的利益以自肥,這是不可能從制度上完全防堵的,最終還是要依靠個人的理想、能力和道德。政治經濟體制的設計和改革,不在於追求一勞永逸,而是盡可能提拔適任的人來居高位,然後鼓勵他們謀求公益的最大化,所以也必須時刻檢討組織自我的缺失,因應世界環境的改變而與時俱進。 
我已經在《讀者須知》裏面説過,拿既有的迂腐理論來説教,是浪費大家的時間。我這次是法外施仁,讓你知道爲什麽你的那一套説法沒有在這裏討論的價值;以後再犯,我還是格殺勿論。 
你還是執迷不悟,我只好把你拉黑了;我的時間有限,必須兼顧幾萬名讀者,一個人想要獨占大百分比的公共資源,是非常自私惡劣的事。
\subsection*{2019-11-11 05:24}

偏左偏右在社會議題和經濟規則上是完全可以獨立的兩種意見。資本主導媒體,管控專注在後者,對前者反而是鼓勵極端;這正是我反復討論過,白左形成的背後真實原因:上一次美國年輕世代大幅左傾,是60年代,所以70年代資本收緊媒體管制之後,就努力把左傾動力都乾坤大挪移到社會議題去了。
現在又有一波年輕世代對資本勢力的反撲,這總算打破了半個世紀來那一套白左忽悠,但是我還是不樂觀,這是因爲能做理性獨立思考的人不可能達到多數,即使經濟大幅下行,造成民衆普遍對現有財富分配體制的反感,資本手下的宣傳體系仍舊可以很簡單地把問題怪到社會和族群上,反而鼓動出極右勢力;這正是Trump所依賴的群衆,也是20世紀初極左和極右交互出現的原因。極右有利於財閥,還可以依賴和平手段掌權;極左卻必然要靠武力才有可能推翻既有體制。
這個把民衆忽悠到社會議題以便在生計問題上下其手的伎倆,台灣政壇也早已自主發明,只不過他們不叫左右,而是統獨。我也一再解釋過,統獨的主動權(除了文統之外)根本不在台灣手裏,大多數選民專注其上,完全是中了貪腐政客和財閥的套路。
\section*{【國際】有關於玻利維亞政變的一些消息}
\subsection*{2019-11-15 04:56}

玻利維亞的鋰礦的確是很新的發現。
電動汽車是全新的技術革命,美國目前只有Tesla已經規模量產,GM有一個型號。財閥的運作規律,一般是保護自己的獨占性利益,很少會前瞻性地為未來產業做準備,所以我並不認爲這事背後一定有官商勾結。也可能是CIA的中級主管們自己商量一下,覺得局勢發展剛好方便奪取戰略資產,對玻利維亞軍方的帶路黨做些空口承諾也用不着向白宮和國會要錢,就順水推舟鏟除了一個不太聽話的政權。
財閥們其實目光很短淺,一般必須有明確的現成利益才會出手。像是有關911是美國自導自演的陰謀論就非常不可信,因爲利益虛無縹緲,風險卻極大。但是一旦在2003年出兵伊拉克,一批億萬富豪和共和黨高官聯合去掠奪當地博物館的古董,就是本小利大的事,所以也就真的發生了。
\section*{【經濟】放任經濟學的邏輯謬誤}
\subsection*{2019-11-10 11:24}

Milton Friedman根本不是搞錯真相,而是從頭到尾都在故意撒謊,爲企業和財閥的獻金奮鬥,也就是我以前説過的”學術娼妓“。他所宣傳的歪理,還剛好可以遮掩解釋自己的醜行。在這個例子上,純粹是為富人牟利,可以上私立學校,還不用付教育稅。 
我敢這麽說,當然不是胡亂揣測,而是來自一個美國老教授Richard Wolff透露的秘辛。Wolff年輕時在Stanford念研究所,當時是1960年代,他和好朋友們都是加州的典型左派青年。有一天,其中一人在畢業前,忽然宣佈將要到芝加哥大學加入Friedman的團隊。Wolff私下問他怎麽回事,他說Friedman開出比別家學校高出好幾倍的薪水,條件是必須寫違心之論。這人後來成爲知名經濟學家,自由主義的幹將,最近退休了,但是Wolff說他私下的意見和公開的寫作是完全相反的。 
既然Friedman主動去收買聰明的學人來編造自己都不相信的謊話,那麽他本人當然大機率也是幹同樣的勾當。
\subsection*{2018-09-02 17:59}

英文的每個單字通常對應著中文由兩三個字組成的詞匯,所以對同義字的態度是完全不一樣的。
中文自古以來,所有的藝術形態(如詩詞散文等等)都講究一個字不應該一再使用,所以自然有了許多同義字和近義字可以基本混用,甚至接連著反復使用。英文剛好相反,同一個意思只能用一個單字,不能堆叠,所以自然一般就不須要同義字。近義字之間也幾乎必然會有細微的差別,而高等教育就在於教導學生瞭解這些差別,獲得精準用詞的能力。
Freedom和Liberty的意義相近,但是兩者都很常用,光是這一點就已經表明它們不是完全同義的。至於它們的差別,我舉一例:後者只能用在壓迫的來源是政府,所以當美國人在2003年因爲法國人拒絕加入伊拉克戰爭,而把French Fries改名的時候,只能用Freedom Fries,而不是Liberty Fries。
\section*{【臺灣】如果我是新任高雄市長}
\subsection*{2019-11-04 15:04}

我同意人口壓力是帝國長期治理的主要難題之一,氣候變化也常常在學術界被提出來討論,但是難題並不等於無可跨越的屏障,它們只是要求治理水準必須高到能剋服難題的地步。
人口成長絕對不是帝國崩潰的唯一因素,那麽它是否決定性的關鍵要素呢?我覺得政治、社會、經濟結構所決定的治理水平才是最重要的。畢竟清廉高效有其他客觀的標準,不是只從王朝末期的動亂來倒推的;在歷史上每個朝代在中期腐化就很明顯。人口壓力上來之後,往往會有改革的提議,它們都來自對當時社會很熟悉的聰明人,而這些人的計算,都是只要壓制豪强,就能養得活衆多老百姓。他們的失敗,也明顯來自土豪的抵制反擊,不是政策執行到底卻被現實壓垮的。
另一個你必須考慮的事實例證是羅馬帝國。它的人口在帝國早期就達到頂峰,大約1億人。羅馬的耕地面積並不大於中國(因爲不包括德國和東歐),農耕技術更是落後,那麽硬要說中國的王朝被幾千萬人口吃死了,就很沒有説服力。尤其漢朝對長江流域及以南的開發,還很原始,如果有高效的政治執行力,安置多餘的人口完全沒有困難。
\subsection*{2019-01-14 11:43}

臺灣的未來,不取決於大陸和臺灣的力量對比,而在於中美之間霸權交替過程中的角力。在中國實力全面壓倒美國之前,中方必須忍讓,避免衝突無限升級;畢竟美軍是美國三大基本力量中衰落最慢的一支。中國的長處在經貿,雖然近年在軍事建設上頗有建樹,在可見的未來,仍然無法打贏對美的全面戰爭,所以在軍事上以弱擊強甚爲不智。而臺獨卻是唯一一個中共無可退讓的議題,這也是爲什麽他們還在進行笑臉統戰攻勢的基本原因:越多的地方官遵循九二共識,就越代表臺獨沒有法理上的正義性,就算臺灣政府決定冒動,有内部爭議就使得美國全面干預的機率下降。這是韓國瑜所能提供的籌碼,中共在經貿上給予支持,代價甚小,應該是兩利共贏的交易。 
不過“韓國瑜的能力就在於說話直白(趕上流行)和菜市場協調龍蛇流氓的能耐”的確是一個很大的危險,也是我寫本文的主因。不論世界大局如何決定統一的時機,高雄能否有明智高效的治理,還是操在自己手裏的。把地方經濟搞好,原本就是地方政府的任務;只要韓國瑜有能力、有意願,客觀上沒有絕對的困難。 
至於最終的統一,的確也如你所説,還是武統最爲可能;中方現在的和平論調,基本只是為了仁至義盡。我以前也反復論證過,武統不但在客觀上是最可能的結局,在主觀上也是最好的辦法;這是因爲臺灣的政治體制已經爛透了,只有武統才能完全推倒重來。
\subsection*{2019-01-11 00:39}
你說的前半還合理,但是產學合作,在歐美的小城市常由市政府帶頭。畢竟如果你只是一個四綫城市,指望聯邦或歐盟政府來牽綫是不切實際的。 高雄在中國整體來看,就是一個四綫城市;我在這裏建議的,都是歐美和中國類似城市的市長的日常責任。但是許多臺灣讀者還是持“上臺就好”的態度,我很失望。 政治不是體育競賽,它的目標不在於藉著輸贏來滿足群衆心理寄托,而是有極爲嚴重的實際後果。我們評價一個政治人物,不是看他的黨派,而應該取決於他是否履行職責、改善民生。統獨原本就是一個臺灣完全沒有自主權的假議題,它對政治唯一應有的作用,在於是正面還是負面影響貿易和收入,這是非常被動而基本的選擇。如果韓國瑜除了九二共識之外,沒有真正對高雄的產業有促進,那麽高雄何必設市長呢?挂一個“我們承認九二共識”的招牌不是便宜許多?我很不希望看到臺灣所謂的救世主,上臺之後除了作秀和繼續玩弄黨派鬥爭之外,什麽正事都不幹;而想要政客願意幹正事,民衆必須先有要求。 韓國瑜不是神仙,是市長;我不要求他騰雲駕霧,但是產業升級這種幾千個大陸和歐美的市長天天在做的事,他如果連努力都不付出,那麽和蔡英文、賴清德這類政客有什麽差別?\section*{【英國】脫歐鬥爭的細節}
\subsection*{2019-11-01 08:42}

很好的問題,你是學國際關係的專業嗎?
我在正文裏已經解釋過,英國因爲舊日的帝國榮光和Continental Balance of Power的戰略傳統,一向有和歐洲大陸作對的動力,但是在2016年之前,這只是老年昏庸的選民喜歡幻想的事項,出力推動的Nigel Farage被公認是瘋子。
但是英國從70年代加入歐盟開始,就因爲這個扭扭捏捏的態度,可以不斷向歐盟勒索,獲得了不可計數的特權,所以主政者總是容忍並且利用這些要求脫歐的少數,每隔幾年就可以和歐盟重新談判一次,以謀取進一步的利益。
Cameron在2013年的承諾,只不過是這一個傳統的再一次體現,當時根本沒有人當真,不但他有很大機率食言而肥,就算真有公投,也沒有人相信會通過。
總之,脫歐這件事的確萬分複雜,土豪當然不是脫歐的唯一動力,甚至不一定是最大的動力,但卻是過去四年的決定性關鍵所在。早先只有選民中少數的怪老頭在做夢,政壇主流容忍他們,只是爲了定期向歐盟勒索。即便是到了2015年,反移民成爲歐洲右派的主要訴求,英國民意對脫歐的支持率還是顯著低於50 \% 的。換句話說,大家承認反移民是主流力量之一,脫歐卻是那個方向的極端,脫歐公投必敗,但可以安撫反移民勢力。
一直到ATAD通過之後,脫歐勢力被土豪緊急接管,這才有了出乎所有旁觀者意料的一連串結果。否則Cameron承諾的是2017年底之前舉行公投,爲什麽到了2016年二月就急急忙忙地的通過公投法案?這正是因爲Article 50有兩年的緩衝期,在2016年一月ATAD剛出版的時候,預計在2019年一月1日生效,所以不能等到2017年再公投。等到2016年下半,ATAD被發現有漏洞,必須有晚一年的ATAD 2來彌補,土豪才容許May慢吞吞地在2017年三月啓動Article 50。
\section*{【英國】談Brexit}
\subsection*{2019-10-06 00:32}

經濟學界被資本收買,不是“恐怕”,而早就是即成事實。我以前已經一再給出例子了。這個資本控制美國經濟學的系統化、常規化和體制化,才是一般人難以注意到的。 
至於金融在政策分析上的重要性,的確是遠超出一般學者的瞭解,也是爲什麽我能做出他人無法比擬的正確預測的原因之一。其道理,我以前在留言欄也提過了:二戰後70多年長期和平,資本得以成指數纍積,所以已經遠大於世界的GDP,然後在70年代起,又建立了一系列宣傳、收買和控制的組織和機制,使得英美政客成爲他們的傀儡。 
與其同時,金融銀行界和他們結盟,利用不斷演化的金融創新來剝削國民,並且為資本極度加大杠桿,使金融力量達到GDP的上千倍,那麽連專業客觀的財政措施也必然是以資本金融的利益爲先,所謂“Too big to fail”,就是這個原理的體現。 
總之,要瞭解英美的政治社會走向,就必須先分析出資本和金融的利害所在。美國訓練出來的經濟學人,如果是芝加哥系的,反而本能地要掩飾資本和金融的操作和影響,結果連經濟上的道理都會說反;東西兩岸學校出身的,並不懂資本和金融的重要,只能盲人摸象,講一些片面的細節;學金融的,只管在既有法規下鉆漏洞賺錢,沒有大局觀;做媒體的,原本應該暴露資本的惡行,但是媒體企業已經被嚴格掌控了,所以只能去搞政治正確類的白左或右翼議題。最終結果是,像我的博客這樣,把資本、金融、經濟、政治、宣傳之間的互相作用講清楚的,在英文世界並不存在。
\section*{【美國】【工業】波音和載人登月(續)}
\subsection*{2019-08-09 00:19}

你這個想法其實並不太離譜,科幻小説界真正在乎現實的大師,例如Arthur C Clark就研究過,不過一般是設想在月球上,全真空,而且重力只有地球的1/6。 
在青藏高原當然也可以研究,但是要進入軌道,最低的那幾公里,其實無足輕重,真正消耗能量的是1)要剋服300公里的爬升進入軌道高度;2)然後必須保持第一宇宙速度(First Cosmic Velocity),大約7+km/s,相當於海平面的Mach 20+。 
現在你應該可以看出問題的所在:如果主要依靠自由彈道爬升到軌道高度,那麽初速必須在Mach 25左右,這樣的摩擦生熱,連實心彈丸都無法承受,空心的飛行艙基本很快解體。 
用大飛機載到同溫層(~20公里高)發射火箭,比較靠譜些,但是所得不大,在尺寸上和方便上的犧牲卻不小,目前還不能確定是否會有經濟性。
\section*{【戰略】【經濟】再談中美貿易戰}
\subsection*{2019-05-09 10:10}

很簡單。Trump談判一向靠訛詐,現在再度强勢逼迫,如果中方順從,他就可以全拿;如果中方堅持底綫,他也不會有損失,回頭一樣宣稱勝利,反正沒人知道談判細節。 
尤其是現在民主黨有幾十個候選人,其中有些主張對中國要在貿易上更强硬。在野的可以隨口胡扯不切實際的要求,固然是美國大選的慣例,Trump卻不是一般的總統:他對執政一點都不在乎,100 \% 想的只是選舉,所以完全可以忽略現實,做姿態表明自己的“强硬”,討好選民。 
當然習近平是以不變應萬變,不管Trump怎麽虛張聲勢,中方能接受的條件就是那些,到明年美國農民真正活不下去了,Trump自然會又想要簽約;不過他不一定能連任,那麽是否值得簽一個幾個月内就可能作廢的協定,就反過來成爲中方的考慮。
\subsection*{2018-11-07 21:17}

這次期中選舉,雖然聲勢浩大、動員極衆,但是因爲雙方都自以爲是,所以結果仍然是勢均力敵,執政黨小讓衆議院,和歷史上期中選舉的規律幾乎完全一致。
我從幾個月前,上史東的節目時,就說民主黨有七成機率拿下衆議院,但是共和黨也有七成機率會保住參議院。那是因爲我一向努力把自身的主觀偏好放下一邊,只用客觀的態度來做分析預測。過去這一個月,即使是中文媒體也往往是一面倒,或者說共和黨會全贏,或者說民主黨會大勝;這些都是非理性的作者把自己的願望和現實混肴了。讀者如果注意到了,下次就不須要對他們的論點太過關心。
Trump算是成功渡過一劫,并且再次證實他對共和黨草根群衆的掌控。共和黨内部的良心反對者差不多都退光了,此後兩年,應該主要是Trump和衆議院民主黨人的鬥爭,Mueller的調查則是重要的外卡。
在中美貿易戰方面,寄望參議院的財團勢力來解圍,只怕是不太實際了。Trump當然隨時可能心情變了,就決定轉向,但是中方仍然必須做出長期抗戰的準備。如果早先再强硬些,采取了正文中的上策,或許就能節省些時間。
更長遠來看,Trump的繼續掌權,當然對中國和世界都是一件大大的好事。
\subsection*{2018-09-21 12:09}

最近兩周,身體和精神都很不好,勉强回答留言也詞不達意。
美國的確是嚴重精神分裂;這個裂痕也很明顯,就是他們的兩黨制。
詳細説來,兩黨之間的隔閡,其實有兩類:第一類是美國國内可以主觀決定的議題,群衆無須用腦就會有很强而相矛盾的意見和結論,例如移民政策、種族政策、稅率和福利政策等等,這類議題是兩黨黨爭在臺面上動員群衆的手段,是長期持續而且和是否執政無關。
第二種則是必須與客觀現實有立即互動的議題,例如外交和貿易政策。傳統上這方面意見的差別,就不是民主黨對共和黨,而是執政黨對在野黨。在野的時候,總可以迎合群衆的主觀偏見,執政了就必須面對現實,改采理性的策略。在對俄政策上是如此,在對華政策和貿易政策上,以往也是如此,一直到Trump這個民粹總統上臺,才打破成規。
是的,短期上Trump似乎製造了很大的麻煩,但是長期來看,美國絕對是得不償失的。
\section*{【美国】【宣传战】美国的种族偏见和扭曲的媒体报导}
\subsection*{2019-02-27 22:50}

人性的一般趨勢,並不是在每個決定上深思熟慮、做獨立的理性分析,而是建立人生態度和立場之後,凴直覺做判斷。這些律師也是一樣(其實律師的訓練,還特別抹殺理性客觀的分析,强調主觀的硬拗),一旦和官方打對臺成爲習慣,就會忘掉初衷,變本加厲,成爲不擇手段的職業反對者,所以他們一向是美國對外滲透顛覆的主要帶路人。
習近平這次清理外來勢力,除了律師之外,還把很多外籍“志願者”和社運人士的簽證注銷了。倒不是說這些人個個都拿CIA的薪水(當然如果有需要,他們也不會拒絕CIA的徵召);他們大多只是被西方宣傳洗腦,成了美國的自幹五。讓他們開英文學校、搞社運、或甚至到西藏去“扶貧”,等於是幫CIA滲透自己。所以驅逐他們是有必要的;而且未雨綢繆、及早出手,在西方媒體上不會引起爭議,做得很高明。
這些人被趕出大陸之後,有些流落到臺灣,在自媒體上和當地的民運人生互相唱和,但是對中國就無法再做傷害了。
\section*{【基礎科研】人屬的起源}
\subsection*{2019-02-26 06:37}

因爲野牛是高度群居動物,每個群體成千上萬,而且個體的體型夠大,對狩獵小隊不在乎,不會像長毛象或者駱駝(美洲其實是駱駝的起源地)那樣受了驚嚇,就開始逃跑分散,讓獵人可以挑落單的個體追逐到底。反而會一起Stamped,對獵人非常危險。
印第安人的祖先,是古人類中最喜歡趕盡殺絕的獵人。現代馬也是在北美洲演化出來的,和駱駝一樣也是在人類一萬多年前進入美洲不久就被滅絕,結果是北美平原上沒什麽東西吃,印第安人的人口密度就高不上去了。一直到西班牙人征服中南美,有些逃脫的馬匹重新進入北美野化,平原印第安人才學會騎馬,可以在馬匹上安全地狩獵野牛。到19世紀美國白人開始屠殺驅逐他們,只有不到300年的歷史。
中美洲的Maya印第安人,獨立發明了農業,馴服了玉米等等作物。但是他們的耕種方式也是粗暴的火耕,對環境破壞很大。一旦降雨有了變化,就無法支持暴增出來的人口,結果是連綿的戰爭,文明自我滅絕了。
所以北美和中美的印第安人,都是歷史上因爲不知對環境做合理保護而自我傷害的前例。
\section*{【能源】熔鹽堆簡介}
\subsection*{2019-01-16 17:41}

原子彈的工程設計很簡單,學過物理的都可以試試,困難在於提煉高純度的鈾235或鈽239。氫彈剛好相反:提煉氫的同位素氚雖然貴並不難,但是構型設計非常困難。這是因爲氫彈必須用原子彈的爆炸力量來壓縮氚引發核聚變,但是有了原子彈這個級別的爆炸,那麽周邊的一切事物自然是想要飛開而不是被壓縮。所以只有一流的工業國家能做到;例如印度有原子彈40多年了,到現在連氫彈的邊還是摸不上。 
氫彈的設計是絕對機密,蘇聯靠間諜拿到美國的早期構想,不過那和最終設計有很大差別,所以頂多只能說有幫助,不完全是照抄。中國則是自己獨立發展的,當時的主要設計人就是于敏。 
至於現在大陸網絡上廣爲流傳的所謂“于敏造型”,那幾乎可以確定是以訛傳訛的胡扯。氫彈的物理原理是一致的,如果在細節上中美有差別,官方絕對不會泄露,所以也輪不到網絡上那些噴子去討論。尤其是在這些吹捧“于敏造型”的文章中,基本常識錯誤有一大堆,例如他們甚至以爲現代的熱核彈頭不算氫彈了。 
我以前提到李文和案的時候,曾解釋過中共間諜獲得了美國W88熱核彈頭的設計圖,然後又有變節人員把它送囘給美國,所以FBI才開始Witch Hunt。現在共軍ICBM上的核彈頭,很可能是照抄了W88才能夠小型化,這樣東風41才可能裝載10個彈頭,于敏的老設計應該已經退役了。
\section*{【美國】【海軍】蛋糕與美酒}
\subsection*{2018-12-29 09:09}

中世紀海軍或許會餓得你半死不活,但是21世紀的美軍是很人道的,至少規章上說被罰者可以要求任何數量的麵包。 
“Bread \& Water”的確算是比較輕的懲罰,不過連吃三天的硬麵包,只怕是比一般人想像的還要更難受些。而且一般是在港内才會施行;出任務的時候,人手永遠不夠,就只能趕快打一頓了事。所以你出海幾個月,到港了反而被關著吃硬麵包,而同伴們是在客棧裏飲酒招妓狂歡,這不是特別讓人高興的事。 
一旦出海,如果沒有嚴重到Flogging,那麽就只用一條打了結的繩頭來抽人,這叫做“Starting”,“開工”,因爲它一般是用來懲罰偷懶的人。有的船長會有專門助手整天拿著繩頭跟在身邊,隨時可以抽兩下。
\section*{【語言】咬文嚼字}
\subsection*{2018-12-04 11:02}

我可以看出你對那些人理直氣壯的胡扯十分氣憤,但是對抗愚昧、自私、虛假和衝動的最好方法,不是像“民主體制”(Democracy,應該是“民選制”才對)所鼓勵的:也用愚昧、自私、虛假和衝動去比賽誰的聲量大、下限低,而是運用智慧、公益、事實和邏輯來揭穿對方的真面目,爭取其他有理性的盟友。
至於“社會主義”,它在中國是個正面名詞,在美國是負面名詞,在歐洲是中性詞,但是在這三大板塊,它的意義都是“優先照顧下層民衆”。我對臺灣政壇的近況不熟,從你的論點來看,似乎是民進黨和它的Liberal盟友借用了這塊招牌,那麽這是完全違反邏輯的,因爲他們對下層民衆的忽視、侵犯和虐待,有目共睹。
民進黨只知道抄襲西方文化的糟粕,所以以Liberal、Progressive、Avant-Garde爲榮也就是很自然的。實際上這些白左的歪路,都是過去半個世紀,財團主導的媒體有意扭曲左派原本應有的“優先照顧下層民衆”的職志,讓他們消耗自己的公信力,而去侵犯其他公民在文化和倫理上的底綫。我們在討論這些文化和倫理上的議題時,不應該忘記“照顧下層民衆”才是政治的首要任務。
\subsection*{2018-12-04 02:00}

很好的问题。 
Democracy来自古希腊文,Demos是民众,Kratia是治理,所以显然应该翻成“民统”或“民治”(参见我在《谈损人不利己》的讨论)才对;在实际运作上,翻成“民选”可能是最精确的选择。类似的字还有Theocracy(神权统治)并没有翻成“神主”,Kleptocracy(盗贼统治)也没有翻成“贼主”,Autocracy是“专制”等等。 
“民主”是19世纪日本的翻译,他们的语法和字义原本就和中文有出入,强行照抄必然是错的。上次我已经抱怨过,把“Army”翻成“军”会造成严重误解。在Democracy这个例子上,误译为“民主”所产生的恶劣后果,可能比把“Liberal”翻成“自由”更糟糕得多。
\section*{【歷史】爲什麽我不佩服Manstein}
\subsection*{2018-12-04 06:40}

我想你應該會同意,大陸軍事論壇的平均水準很低,謊言、腦補、互噴充斥。如果拿一張動物的照片去問是馬是鹿,只怕答案還是七三分,而且正確的不一定是七的那一邊;例如FC31的事,明顯得不得了,但是願意站隊擡杠的,顯然不是極少數。 
在這樣的環境下,能有壓倒性一致意見的議題,只有與意識形態相關的才可能,而對Manstein的評價,顯然不屬於這一類,所以我說“絕大多數人”,是在這個背景下的評語。我個人的經驗,是每一篇對他的批評,夾雜在100篇吹捧他那三場勝利的文章裏。當然因爲那些軍事論壇的信息密度太低、謊言的密度太高,所以我去掃描這些文章並不是太愉快,結果是我所取的樣本可能不夠多,沒有足夠的代表性。還好我有自知之明,從來不號稱是軍事論壇内容和討論的專家。 
然而魔鬼在細節之中,你也應該比較一下,看看其他批評Manstein的文章是否說到了重點。如果正文所說的,都是重複公共消息,那麽我很抱歉浪費你的時間,但是至少應該加强了這些論點的可信度。 
至於你所介紹的那篇文章,我看了幾頁,覺得他的語氣還算專業,但是沒有任何引用事實的出處,所以很難判斷他到底是專業的軍事研究員還是專業的撒謊家。有爭議性的關鍵事實,例如正文中Manstein下令槍決戰俘的事,作者有絕對的責任要列出出處。這位作者只管自顧自地講故事,我以前遇到太多騙人的軍事論壇文章,所以只能存疑。
\section*{【政治】談損人不利己}
\subsection*{2018-11-23 08:28}

這個比較,是有其意義的,但是它只達到:“西式民主天生低效”這個結論;我們以前已經進一步論證了“西式民主必然會腐化”,這篇正文又討論了西式民主從容並放大人類損人不利己的惡劣趨勢。 
當然這並不代表集權體制必然優越,但是很明顯地,集權容易有高效率,也方便遏止百姓互相傷害;真正的問題在於它的腐化。 
我個人認爲(每次我說“個人認爲”,就代表我無法提供嚴謹的邏輯論證,所以以下的結論只是一個可能性高的假説)能自我修正因而對腐化完全免疫的體制,是不存在的。請參考我以前有關“法治最終還是人治”的文章;換句話説,腐化是普世問題,要扭轉其趨勢而做出自我更新,永遠都要靠有理想、有能力的人來與既得利益者做鬥爭。不同的體制,會給出不同的勝負機率,但是成功與否終究是沒有保證的。 
例如中共,在習近平和王岐山上臺之前,腐化的趨勢已經開始而且蔓延一段時間了。習近平被選爲繼任者,絕對不是體制因爲反腐改革而特別挑選他;相反地,如果他提早露出全面反腐改革的決心,幾乎可以確定既得利益者不會容許他上臺。他在掌權之前,必須非常低調,而且仍然經過了險惡的政治鬥爭,整個過程其實是相當驚險的。不過至少中共的體制容許他上臺並且持之以恆地改革;西式民主下,像Renzi的遭遇卻是體制的必然,所以我不懂臺灣選民一而再、再而三地找新的救世主來自嗨能有什麽實際結果。
\subsection*{2018-11-12 14:25}

陳平老師的女兒,到底被美國内部宣傳洗腦到什麽程度,我無法置評,但是我兒子絕對已經是免疫了。我以前曾經在留言欄提過,他在學校常被老師和同學圍攻,就因爲他知道1)美國的國際政策是極端自私的;2)美國的歷史有很多扭曲、自我美化的結論;3)美國的現行體制主要為財閥服務;4)改革是不可能的,因為美國公民的平均素質低下。
他之所以會忽然提起中國,我想有兩個原因吧。首先國際關係是我的研究興趣之一,以往我多有為中國辯解澄清的言論;既然我們父子之間在擡杠,他自然會想要戳我的痛處。你如果有撫養16嵗小孩的經驗,就會知道他願意和我擡杠幾句,還算是給我面子的。其次,我們在上個月有另一場討論,話題是前面提到的公民素質;當時我就說中國剛開始工業化,人民還沒有完全適應工業化、城市化社會的足夠時間,所以在現代公德的標準下,可能還比不上臺灣,更加比不上歐美了。
不過我那時所說的“公德”,其實範圍比這篇正文討論的“損人不利己”要大得多,絕大部分是有關於“大損人、小利己”的自私行徑。本文受篇幅所限,專注於最離譜、最不理性的人類心理趨勢。如果我們擴大範圍,去考慮一切以私害公的行爲,自然會是一場全新的討論。
只看“損人不利己”的極端心理,當然在中國因爲政策的管制,並不明顯,但是這不代表人民沒有這個潛在趨勢。臺灣其實是中國文化和中國民衆的小縮影,90年代政策一變,結果如何,大家有目共睹。
\section*{【工業】談未來的能源技術}
\subsection*{2018-11-21 14:41}

因爲私人資本在完全無風險的情況下,每年還指望5-8 \% 的報酬率。有風險的時候,自然要再加上溢價。儲能技術,從國家觀點來看,十幾年内做出來是十拿九穩,又有極大的好處,投資是應該的。私人資本,對十幾年後才可能有回報的計劃,卻根本就不可能有興趣,因爲兩三年就可以翻牌的投資項目太多了,每兩年賺30 \% ,17年是130 \% 的8.5次方=930 \% ,這種長期工業技術開發根本不可能有這麽高的回報(因爲大部分的利益,被整個國家社會分享了),而且8.5個不同的投資項目,風險自然抵消一部分(中心極限定理),多餘的風險溢價讓這個短期投資策略更加有利。
臺灣的問題,根本在於體制,偏偏大家還是每隔兩年就指望一個新的救世主。就算韓國瑜真是個聖人,又能怎麽樣?Renzi當到總理,還不是鎩羽而歸,何況區區一個市長?
\subsection*{2018-11-15 02:16}

很好的問題。 
在一個技術被實用之前,的確不可能預見所有的攔路虎,我們只能估計他的機率。就連全釩氧化還原液流電池這樣非常簡單、有前景的技術,實際上成功的機會也不過是略大於50 \% 。它之所以應該被優先投資、嘗試,是相對性的,因為別的科技成功的機率只是略大於10 \% ,那麼它自然應該被另眼相看。 
氫儲能就是成功機率在10 \% 左右的一種技術,但是至少比起核聚變那種10\^(-n)的不靠譜忽悠,有質的不同。 
我並不是說到2035年之前,只有這兩種儲能技術應該被嘗試。其他不知名、還在實驗室被慢慢摸索的技術,至少以百計。而且科技進步的道路,往往是在失敗中學習。如果全釩氧化還原液流電池因為莫種未預見的因素而不能實用,了解這個因素就會幫助研究團隊選擇一個代替的方案,或許把釩換成其他的配方。 
所以我說在2035年之前,儲能的問題會在原則上被解決,並不是基於對正文中所提的兩個技術特例的信念,而是從整個挑戰的困難度(並不是很高)和人類社會的努力程度(正在逐漸正視這個問題)來做的估計。
\section*{【陸軍】【海軍】軍隊單位的誤譯}
\subsection*{2018-10-28 13:55}

謝謝你的補充。看來“軍”被翻成“Army”這個毛病,真要怪到日本人頭上。 
我也同意你的結論,因爲歷史傳承的原因,共軍現在已經無法回頭改變錯誤的翻譯。不過純從原則來討論,共軍實在沒有理由要强制決定自己部隊的英文名稱是什麽。翻譯要達到信達雅的目的,一般的做法是由翻進的那方來負責。反正事後翻得是否適切,後果也是由同一方的後世讀者來承擔。 
這個原則當然也有例外:中共的國家主席被自己翻成“President”,“總統”。不過這是爲了和美國拉近乎,避免提醒美國人中國是共產國家。我覺得是個很聰明的公關手段。 
但是軍隊建制的翻譯,沒有這樣的效應,共軍又不是想要忽悠美軍,實在不必去管英文是怎麽翻的。 
至於我這篇正文的論點,是英文讀者應該把“軍”和“集團軍”都翻成“Corps”,而中文讀者應該把英軍的“Regiment”翻成“營”。目前常見的錯誤是怎麽來的,以及官方是否能改變,只是額外附加的資訊,並不影響我們理解和更正這些誤譯。
\subsection*{2018-10-27 18:26}

啊,我對日本的軍制不熟,大概Corps和Army被搞混,還真有來自日本的根源。 
但是不論如何,在1950年代,所有世界主要陸軍,Division以上一級都是Corps,再上一級才是Army。中翻英,必須優先尊重這個事實;而且是應該由講英文的國家來行使如何翻譯的選擇權。就像英國在Regiment的定義與衆不同,所以中文應該對此做出適應,不要把它翻譯成“團”,而應該是“營”(戰術單位)或者“營本部”(行政單位),美軍的歷史人員也應該把“軍”和“集團軍”的英文,更正為“Corps”才對。 
例如正文中提到印軍買了5個Regiment的導彈,中文應該翻成“5個團”還是“5個營”,孰優孰劣,一目瞭然。
\section*{【美國】【戰略】從Manafort案談起}
\subsection*{2018-10-04 04:06}

我想你也是愛之深、責之切。不過客觀來説,中國政府對正確政策的選擇和執行能力,還是遠超過地球上任何其他國家的。我們在這裏發牢騷,似乎是毫無實際作用,但是如果分析得對,總有那麽一點機會慢慢傳播到掌權者的耳朵裏;這個可能性不是很大,但是至少我們知道中共能聽得進理性的建議。像是臺灣或美國,就算上達天聽了,被采納的機率也必然是零,反而會有麻煩,所以我從來不用英文發表文章。
至於人口素質,我們以前已經討論過;我覺得在貧窮農業社會長大的一般群衆,自然不會有適合富裕工業社會的習慣。中國人民的平均素質,現在低於先進國家是正常合理的;等到老一輩人口被時間自然汰換之後,新一代人是否有尊重社會公益的態度,才是真正的議題。所以教育的確是中國内政的重中之重。
\section*{【美國】美國政壇的系統性腐化}
\subsection*{2018-10-03 23:59}

你聽起來像是政策研究方面的專業人士;我很高興能得到這樣的反饋。
兩三年前,這個博客上有一系列關於房地產稅制的討論;不過最近我已經注意到中共在這個方向的準備動作,所以這次就沒有把它列爲一個問題。
教育改革,真正的關鍵不在細節,而在於它對階級固化的影響,這只有長期和間接的效果,中共傳統的地方試點不是合適的解決方案。這個議題的高度,到達了共產黨和社會主義存在的根本意義,同時又是歷史上大帝國維持社會、經濟和政治活力的樞紐,只有最高領導階層從原則上定調,才有做出正確選擇的可能。
至於在貿易戰中打擊美商的利益,這不能只是低調、獨立地做,必須是公開、系統性的對等反應;這是因爲中方的最終目的不在於“打贏”一場必然兩敗俱傷的戰爭,而在於以戰止戰,對歐盟和其他國家做出示範和嚇阻。
\subsection*{2018-09-30 01:14}

你沒有看懂我這系列文章的重點:英國之所以比起美國成熟,不在於它的“國民素質”更優秀,而在於它的體系和傳統,原本就不讓素質低的國民有太大的話語權。換句話説,“Plebeians”(簡稱“Plebs”,賤民,這是牛津貴族學生在社團裏譏嘲的對象,包括Cameron和Johnson都參加過,參見我以前的文章)即使通過政治和社會運動而强迫國家采行什麽政策,制定規章細節的仍然是統治階級。
這個體系的重要環節是間接選舉議會制:Plebs選出的議員即使佔了多數,在議會層級仍然只能當Back Bencher(無職務、無實權的普通議員),頂多拿下一兩個象徵性的内閣位置。所以即使出了“脫歐”這樣的大紕漏,Farage還是閑人一個,Johnson做個挂名首長也不長久,最後(從金融財團的觀點)該做“軟脫歐”(Soft Brexit)還是做了軟脫歐。
\subsection*{2018-09-28 14:42}

是的,美國現在的亂象,如你所説的“為一己之私打死不退”,臺灣十幾年前就有了,自己叫做“死豬不怕開水燙”。在這方面,臺灣倒是領先美國。
我也同意你提的,美國太年輕,是根基不穩的原因之一。英國人想一套、說一套、做一套、對他人的要求又是另一套;美國人就沒有足夠的經驗和智慧來把不同的套路分清楚。例如英國表面上把民主這塊招牌抱得緊緊的,自己卻從來不搞直選民主,一直到這次脫歐公投才違反了其傳統智慧,像是Osborne這樣有見識的人,事先就極力反對;後來被一票民粹政客如Johnson硬推上架,現在除了後悔之外,也懸崖勒馬,說什麽也不再搞第二次公投。這不但是我原本就預測的,也是英國值得佩服、比美國高明的地方。
我年輕的時候,喜歡讀科幻小説,看過Larry Niven的《Footfall》。這是一本典型的外星人入侵美國,最後被擊退的故事。書本身不是很好,遠遠不及Niven早先的《Ring world》(我給自己的英文名字是Louis,就是取自《Ring world》的主人翁Louis Wu),但是裏面的外星人很有意思:他們的生理、心理、文化和社會結構都還相對原始,之所以能做出星際旅行、試圖征服其他星系,靠的是繼承一個已經滅亡的先進文明,直接拿現成的科技成果來應用。Larry Niven大概沒有想到,但是這不就活脫脫地是美國自己嗎?
\section*{【美國】【經濟】再談Trump的政策}
\subsection*{2018-09-20 17:14}

你的邏輯不通。我建議的是對特定美國企業在中國零售的營收,開徵銷售稅;如果關稅可以隔天取消,這種銷售稅也可以隨時取消。
Trump加關稅的目的,不是爲了討價還價,而是要根本性和永久性地改變跨國企業產業鏈的成本方程式,將生產基地移出中國。中國高層不想開徵美商銷售稅,是因爲這也同樣會促使跨國企業把生產基地移出中國。
但是這個思路,似是而非。Trump要的是最終在美國銷售的產能;我討論的,卻是在中國銷售的產能。Trump一竿子打遍所有產品,中國的回應卻可以是有針對性的,亦即只對有國内替代品的才出手。美國企業一退出中國市場,中國企業自然接手,對國内就業總數並無不利影響,利潤和產能的增加卻會加强中國企業在全球的競爭力。
至於所謂國際信用,美國出手傷人,中方做合理對等的反應,怎麽會算是背信呢?你的這個邏輯,和九一八事變之後,蔣介石因爲只想到國際聯盟去哭訴,而決定不予反擊、坐以待斃,是一模一樣的。其實有點腦子的人都知道,東北軍若是對日軍做出相當的殺傷,反而在國際上容易得到支持。現在的貿易戰也是一樣的;中國要是能對美國做出報復性的傷害,歐洲才會有所忌憚,否則Trump一下臺,下一任民主黨總統約歐洲聯手來欺負中國,歐洲何樂而不爲?
\subsection*{2018-04-18 06:09}

是的,長遠來看,這會消滅投降主義的蠢話,幫助中國内部集中力量來攻關,但是短期可能會很痛苦。 
我在正文裏解釋了,中美進出口的不平衡,使得美國似乎有關稅戰爭的絕對優勢,但是看得更深一層,美商的利潤卻是遠高於中方,所以貿易戰爭超越了關稅的範圍,美方的得失比就會改變。但是我也簡單提到,美國掌握了某些產業鏈的關鍵環節,如果沒有替代品,中方就不能去動它。換句話説,如果貿易戰不只是持久,而且打到全面戰爭的階段,美方又會有不可替代性上的優勢。 
目前中方只能避免戰爭的無限升級,每次出手都必須是有理有節,然後希望不可替代的美商(如Intel和Qualcomm)爲了利潤而對政府施壓,讓中興成爲個案。換句話説,這場仗小打是美方贏,有限升級是中方損失小,但是核子武器卻又是美方的優勢。非常微妙的形勢,而且還有期中選舉這個變數;中方如果不願意無條件投降,以拖待變是目前唯一的選項。 
臺灣,包括臺積電在内,只是中級的技術,對整體貿易戰影響很小,而且動武反而有讓貿易戰無限升級的必然結果,必須絕對避免。
\subsection*{2018-04-08 07:43}

這篇文章的核心主旨有兩個:
1)必須小心處置歐洲的反應;
2)不能只用關稅,必須對P\&G和GM這樣已經投資的美商動手。
歐洲媒體現在積極轉述美國的宣傳稿,說中國要求技術轉讓是偷竊(Theft)。我很奇怪爲什麽中方沒有在這一點上嚴正反擊(例如在英國和德國告報紙毀謗):這樣的宣傳一般是會胡作非爲先打底子,而且偷竊的罪名實在離譜,哪一國的法條有把技術轉讓定義為偷竊的?哪一國的法庭曾經因爲要求技術轉讓而判了偷竊罪?中車賣給美國地鐵車輛,不也被要求轉讓技術嗎?印度更是每個產品都要求技術轉讓,爲什麽歐美不提?
至於不能局限於關稅,其實我在正文裏不好意思提(因爲我不想被誤會對樓繼偉有不敬之意),樓部長上周說他認爲報復手段應該是依大豆、汽車和飛機的順序來打關稅。我在正文裏解釋了,先打大豆是對的,但是汽車不應該是進口關稅,而是對合資企業打特別稅(基本上是把GM趕出去);至於飛機,還是不動的好,否則歐洲有恃無恐。
\section*{【歷史】讀中古英國歷史}
\subsection*{2018-09-16 20:22}

Tocqueville痛心自己國家制度僵化,對英國和美國的評價頗有溢美。他在這方面所做的比較,當然有參考價值,但是我們不應該照單全收。尤其英法歷史地位分道揚鑣的關鍵在16、17世紀,Tocqueville的觀察發生在19世紀中期,英國已經全盛並開始腐化之後;他所説的這個現象當然是存在的,但是是否有他認爲的決定性因果關係,我覺得很難説。
現代美國社會僵化,但是下中產階級仍然崇拜像Trump這樣全憑祖蔭和詐騙起家的大富豪,也是因爲他們夢想自己也可以同樣不靠任何技能就發跡。但是回顧19世紀末美國興起的過程中,做這種夢的美國百姓當然是有的,但顯然不是主要的心理穩定因素;真正讓社會中下層滿意的機會,在於努力工作之後就有很大的可能能躋身中產階級。我想400多年前剛跨出中世紀的英國人民,也很有可能更關心如何晉級為中產,而不是上層社會。在這方面,英國的航海傳統、地理優勢和開明放任(Liberal)政策,顯然提供了比法國更多的機會。
\subsection*{2018-09-15 00:37}

12世紀,英國中古時代開始的時候,英法兩國的制度基本是一致的。事實上,威廉征服者自己就是法國公爵,他所帶來的貴族,也是他在法國時的原班人馬。這是英國政治社會體制的最後一次大變動,此前英國的文化和趨勢,基本被抹净重啓。
此後英法的政治結構上分道揚鑣,其實是16世紀才明顯化;在此之前官僚組織只有少數貴族能出任、規模很小,換一個國王就可以通盤變革。16世紀大航海,加上文藝復興,商業化和城市化產生了第一批中產階級,這時英法的王權發展才有了根本性的差別:英王聯合中產階級打擊貴族和教會,法王卻是聯合後者來打擊前者,所以體現在官僚體系上的差異,不是像你所説的只有法國發展職業官僚,而是英國的官僚系統是中產階級的掌權管道,而法國的官僚系統卻是貴族壓迫平民的手段。
法國的官僚系統一發絕塵,極度龐大化,那是共和革命後,拿破侖的所爲了。
\subsection*{2018-09-13 10:57}

你把因果搞反了。Luther的Reformation運動始自1517年,Henry VIII到1534才和教廷決裂。
Henry VIII的English Reformation和其他的新教不一樣,教義和教會組織都沒有變動,只是不再效忠Pope。這對王權很有幫助,所以我說英國的王權在16、17世紀到達巔峰,但是要説英國國教和其他新教一樣,有助於商業化和工業化,那就是腦補過度了。
清教徒其實是一個很小的Cult(Brownists),遠遠不算是英國文化的主流,對商業化和工業化的影響更加有限。在美國也是極少數,只因爲他們建立了美國北方New England的第一個殖民地,所以後世出名。我們做研究,不能只看教科書的章節標題,然後自行腦補。
這個部落格的留言,應該和正文一致,言簡意賅,所以一句一段落的格式非常辣眼。下不爲例。
\subsection*{2018-09-12 18:31}

英國人在過去400年運氣亨通,當然不代表他們自己一無是處。
例如他們在中古之前就有很強的法治觀念(Common Laws),也就是契約法;事實上大憲章就是一個契約。國王是否願意遵守這些契約是另一回事,但是至少地方貴族肯冒著秋後算賬的危險,嘗試讓契約來約束國王。這個法治的傳統,在16世紀的商業化和城市化過程中很有助益。
在16到19世紀英國興起到全盛的300年裏,英國人展現了顯著高於歐洲其他國家的理性思考和改革意願,把原本處於歷史頂點的王權連同地方貴族勢力一路消減退讓到現代中產階級能接受的地步,成就不可謂不高。
至於熱飲,我不像你這麽挑剔:茶是綠的還是紅的都可以。美國式咖啡(用泡的)當然是泥漿水,但是意大利式現磨現蒸還是蠻好喝的。不過Espresso Machine必須是意大利或德國製的,美國的粗製濫造(很奇怪,意大利製造的厨房電器,絕對是世界一流,比起德國毫不遜色),勉强用也用不久。
\section*{【基礎科研】【戰略】從貝爾實驗談起(一)}
\subsection*{2018-08-28 01:59}

是筆誤,多謝更正。不過BB84仍然依賴單光子跑完全程;目前單光子能可靠在光纖裏跑的距離,還是在100公里左右,中繼站絕對是必要的。一旦有了中繼站,那麽它當然成爲Single point of failure。
AI我不熟;邏輯位元的細節如果錯了,請指出參考資料,我會去研究。邏輯位元需要多個物理位元的原因是錯誤糾正,畢竟保持qbit在糾纏態是很不容易的。
我看過很多理論論文,討論如何做出邏輯位元,但是幾個月前有一篇文章說這些都是騙人的,實際上還沒有一個團隊能實際做出一個可靠穩定的邏輯位元。不過我沒有保存那篇文章,所以現在找不到了。當然,那個作者可能說錯了,或者我記錯了。
我一向覺得只要先做了足夠的研究、盡力不犯錯,事後被更正沒有什麽丟臉的;畢竟討論一個議題,經常比確定自己沒説錯話更重要。
\section*{【政治】大衆媒體的内建矛盾}
\subsection*{2018-03-29 11:05}

這個學校是17年前一個對衝基金的老闆娘,為自己的兩個女兒辦的,順便對外招幾個學生,讓她們有玩伴;聽説她前後花了超過2000萬美元。她這樣花錢雖不少,比起別的豪門消費的門道,要高尚得多了。 
剛好這個學校距離我家不到1公里,我又不放心公立學校裏龍蛇混雜(即使是我們這個所謂的好學區,一樣也有吸毒的問題),就送兒子去了。因爲那位老闆娘自己在大學主修的是古典文學,所以它就成爲學校教學的重點,第一批雇的老師就是拉丁和古希臘文的專家。我也知道這一點實用性都沒有,並不把它當作長處。 
後來我兒子果然說他想成爲作家和詩人。但是到了10嵗那年,去同學的生日宴會,同學的媽媽花錢請了名作家Brian Selznick(電影《Hugo》的原作者)來和小朋友們聊天。Selznick提起自己在成名之前,當了多年的窮作家,有一次口袋只剩兩塊錢,如果買了McDonald的漢堡,就沒錢坐地鐵回家。兒子回來之後,志願就改了。 
我在這個部落格,曾經勸年輕學生不要進高能物理;其實我自己的兒子也被別人挽救過。
\subsection*{2018-03-29 01:08}

我對你的論證有以下幾個評語: 
1)高斯函數只有在簡單的物理現象上才準確,任何生物(包括人文、經濟、社會、金融)現象的機率分佈都有“胖尾巴”(“Fat Tail”),也就是在大約兩個標準差的地方,從指數衰減變成次方衰減(一般是2-4次方),所以用高斯函數來推演三、四、五個標準差外的機率,是不正確的。這也剛好是量化經濟學(Quantitative Economics)最常犯的錯誤之一。 
2)我對自己的期許,在於知識廣博、思考深入、堅持理性和道德標準、有正確的三觀;智商是爸媽給的,不辜負自己的天賦就行了,不值得吹噓。 
3)我兒子並不算笨;他12嵗的那一年(倒不是我要揠苗助長,美國有一個獎學金項目叫SET,要求資優生在12嵗去考SAT)就在SAT Subject Math Level II(這是美國大學入學考試的三級數學測驗裏的最高級,包括了三角函數;其實SET只要求考最低的那一級,也就是標準SAT,但是那對他來說,太容易了)考了滿分。他現在考SAT拿滿分有困難,是因爲他上的一直是特別學校,學的是拉丁和古希臘文的古典文學(例如《Iliad》,10嵗讀英文版,12嵗讀拉丁版,13嵗讀古希臘原文),反而沒有像公立學校那樣專注在英文的基本文法上。 
4)我說和他溝通有困難,那也只是在數學、物理、經濟和國際事務這些話題上。在其他的討論,例如文學、藝術、生物、歷史和社會時事,他大致可以和我平等交談,至於音樂和西洋棋(他曾是幾年的州冠軍)這些他的專長,反而是我插不上嘴。 
5)我對我兒子的期許,從他三歲開始反復教導,只有一句話:“Be A Decent Man.” 用中文說就是“做一個堂堂正正的男子漢”。至於是否成功或快樂,那都是次要的。
\subsection*{2018-03-23 07:24}

我素來不喜歡囉嗦。一個道理不論如何深刻,如果一句話能解釋得完,我就不會像社會學者(例如福山)那樣去寫一本書;這樣我才可能有時間去想新的話題。 
我也知道“得道”(這是道家的詞匯,佛家說波若波羅蜜;原本都是指深刻的道理和智慧,不過現代人往往把這兩個詞神化了)的喜悅,但是獨立思考出新的智慧,比起上網閲讀要難得多,所以大家不應該期待我每個星期都能想出一個來。 
很多時候,我的一些想法已經醖釀了許多年了,但是如果沒有形成完整的體系,我就不會寫下一整篇文章來。例如這篇文章,開頭的那個實驗,是我轉金融的第一年就知道的,其他有關大衆媒體的負面效應也是觀察過好幾百次,但是一直等到本月讀到MIT的那篇新論文,我才覺得罪證確鑿,邏輯上有完美的嚴謹性,不怕任何挑戰,可以寫一篇正文。 
至於我零星提到的道理,不見得就不深刻,例如政治、經濟和社會學普遍犯了把Correlation當作Causality的錯誤,若是福山想到的,他大概可以寫兩本書了。你應該仔細重讀所有的正文和留言欄評論;我並不會把重點加黑,反而通常只是點到爲止,所以你自己必須小心慢慢地去思考;有能力、肯用心的讀者,就會有遠比他人更深刻的體會。例如【後註一】裏面,我說我的政治評論其實是把孔孟思想應用在現代社會,那麽用心的人就應該想想論語裏面,那一個段落是預言了西方民主制的謬誤。我說過我的目標讀者是1 \% 的1 \% ,所以做引申是讀者的責任,我不會像大學教授一樣,為20嵗的毛頭小子們划重點、給作業。
\subsection*{2018-03-20 16:48}

先用邏輯瞭解現實的規律,然後下一步才是決定如何反應。 
這篇文章的論述,簡單來説如下: 
1)即使沒有任何偏見(Bias),大衆媒體的普及只會增加誤差(Error); 
2)在大衆已經有偏見的前提下,私有媒體只能反饋這些偏見; 
3)互聯網不但反饋偏見,而且選擇性地加强謊言的傳播。 
這裏最重要的一點結論,是新聞從業人員的素質、涵養和品格,並不能扭轉這三個系統内建的矛盾;反過來是劣幣必然驅逐良幣,有見識、有良心的新聞工作者,必然會被排斥和淘汰。 
要跳出這三個矛盾的局限,我們必須仔細檢驗論證過程中的顯性和隱性前提。我的整個推論的重要邏輯前提,只有兩個: 
1)大衆沒有絕對理性和專業知識,也不知道要服從有絕對理性和專業知識的少數精英; 
2)大衆媒體是自由市場的一部分。 
所以唯一的解決辦法,就是由政治權力取代自由市場來主控媒體,然後只容許理性和專業的論述出現在媒體上。這聽起來不就像中共的體系嗎?只不過他們搞宣傳的人員層次、眼界和手段都不夠高就是了。 
至於那些面對事實和邏輯,卻仍然拒絕接受的人,他們正是以身示範爲什麽愚民沒有資格行使任何政治權利,就像嬰兒沒有資格行使任何政治權利一樣。這裏的重點是“行使”兩個字,他們的基本權利仍然存在,只是既然他們太蠢,任何政治上的主動行爲和意見都會危害自己和社會,那麽這些權利只能由第三者來執行和保障。
\section*{【基础科研】如何创造研究热点和一些其他物理话题}
\subsection*{2018-02-14 11:55}

量子場論(已被實驗證實)+超對稱(強加參數來硬拗的模型)+弦論(沒有任何根據的假設粒子其實是弦) = 超弦
超弦(預測與實際空間維度不符)+特定幾何(純爲計算方便)+半吊子數學(無法證明,只有“暗示”,但這在超弦是家常便飯) = AdS/CFT Holography
AdS/CFT Holography(說宇宙是超弦的全息攝影)+量子糾纏+熱力學 = Holographic Entanglement Entropy
AdS/CFT Holography+更多簡化的假設 = SYK Model
這票人做個錯誤、沒有意義的假設,就可以靠成千上萬的論文,吃上十年。容易發的論文發完了,再堆上另一層錯誤、沒有意義的假設,又可以再吃十年。你問的這兩個話題,就是用更多錯誤假設堆出來的最新論文熱點。
\subsection*{2018-01-29 00:06}

前兩天《觀網》來約稿,我整理了一下我的意見,然後決定不要公開評論。
雖然他的研究明顯地比超弦還不靠譜(19維時空比10維時空的自由度還要多很多),但是1)我在中國物理界已經得罪人太多了;2)這不像大對撞機或悟空衛星一樣浪費公家的錢;3)吳院士至少沒有和超弦同流合污,而是自己埋頭找冷門的題目來苦幹,正是我建議的正確態度。
他的理論是Kaluza-Klein的引申,這正是1985年超弦興起之前,Witten那票人發論文的重點熱門題目,但是内含的問題很大很深,所以超弦一出來,問題稍小一點而且更容易發論文,KK就被放棄了。
吳院士可能是只凴30多年前的那些論文來做研究;可惜那些作者就是後來領頭做超弦的人,很不誠實,深埋的毛病都不公開提(必須在圈内私下才聽得到),論文裏只一味吹噓,所以他也算是被誤導的受害者。
\subsection*{2017-09-26 00:00}
我也认为理论物理的Low-hanging Fruit已经被摘采殆尽,未来的进展只有越来越慢。偏偏世界的潮流是跻身工业化经济行列的国家越来越多,他们一旦解决了温饱问题,自然会开始投资在基础科研上,所以全球整体来看,进入这些行业的人才也只能是越来越多。

在正文里,我说我不知学术营销和进展停滞,哪个是因,哪个是果。但其实最可能的是互为因果:因为进展停滞,行业人口却增加,粥少僧多,竞争过于激烈,所以只有懂营销的才能成功出名;但是出名、主导的都是营销专家,结果是大家跟着一窝蜂追求无实际意义的流行题目,进展就更慢了。

暗物质从重力的观点是必须存在的,但是若要能够对它做任何进一步的研究,只能假设它参与弱作用力,这个假设我估计成立的机率在1/4以下,换句话说,那几万篇论文有75 \% 以上的机率是集体做梦。若是果然如此,那么暗物质就是你説的"不可知"项目的最佳范例。\section*{【戰略】【海軍】再談共軍的電磁炮}
\subsection*{2018-02-12 10:13}

私資公司都是短綫操作,頭幾年節衣縮食,想盡辦法搞好財務報表,一旦上市大賺一票,真正的毛病才紛紛出籠。這樣的態度,非常不適合大型火箭的長期發展。
SpaceX選擇專注的技術,都是短平快的,例如回收,如你所説,沒有真正的意義,所以別的國家從來沒有認真去搞,SpaceX為的就是可以拿來做“世界領先”的公關。現在Falcon Heavy的載荷能力也號稱是“世界領先”,但是我們還是等到它真的搭載了那樣的負荷,再相信吧。
最危險的是,低估出事的概率。這在金融界很容易,2008年的次貸危機,就是銀行選用特定時段(即前十幾年,沒有房貸問題的資料;1990年是上一次美國的房貸危機,絕對不能回顧那麽遠)的歷史資料來做統計分析,以得到低得離譜的風險估計,這樣一來,次貸的衍生品的價值就可以提高幾十倍。我怕SpaceX現在也是這樣搞的。
\section*{【海軍】談超音速彈丸}
\subsection*{2018-02-07 08:25}

沒有錯啊,那篇文章說初速Mach 5,射高42公里,最大射程130公里。我的Excel電子表格如果用k = 0.17,初速Mach 5,得到射高43公里,最大射程127公里。這個電子表格是我花了幾分鐘隨便搞出來的,只保證誤差在10 \% 以下,像是地球自轉、表面曲率、重力變化、大氣溫度和音速改變這些高次項全部忽略,所以我自己覺得實際誤差大概在5 \% 左右。
這裏的真正重點是爲什麽巴黎炮的速度衰減係數只有0.17。巴黎炮原本口徑是211毫米,彈丸質量是106千克。後來口徑擴增到238毫米,對應的彈丸質量沒有資料,但是我們可以很簡單地等比例放大,求得估計值為152千克。這樣的彈丸比我討論的120+毫米高爆彈大7倍。速度衰減係數隨尺寸增加而減低,所以從0.3降到0.17很正常。我在前一篇文章也暗示過,要用Mach 5把彈丸送出100+公里,做到8寸就行了。若是能做到16寸,那麽説不定k會降到0.13(這必須靠實驗才能確定),我的Excel說射程會是160公里。
但是電磁炮能隨意增加彈丸質量嗎?22千克Mach 5的彈丸,動能是32MJ;152千克Mach 5的彈丸,動能卻是220MJ。美國人連做夢都只敢做到64MJ。而且152千克的巴黎炮彈,裝藥只有6千克,和41千克的155炮彈一樣,你覺得這划算嗎?
\subsection*{2018-02-06 13:34}

如果你實在看不懂一篇文章,至少試著去看懂第一段開宗明義和最後一段總結。
我寫了三篇文章,洋洋灑灑,在這裏最後那一段還做了很淺白的總結,核心論點就是你給的這些圖片和宣傳,都是謊言,結果你還是來照本宣科一番。
32MJ就是Mach 5初速(64MJ就是Mach 7初速)的電磁炮,他們說可以打到120海里,也就是220公里,但是我的論證發現實際上只有1/4,也就是55公里的射程,其餘的165公里,是用火箭助推和滑翔增程搞出來的,代價就是1)整個彈頭的質量和空間都被用到火箭、滑翔翼和引導頭上了,根本沒地方放高爆藥,2)因爲要讓引導頭和滑翔翼能承受極大的加速g力,比同射程的導彈和火箭彈還貴好幾倍。
做白日夢很容易,但是憑空胡扯些數據被打臉之後,至少要有尊重邏輯的態度。600-1000公里的射程的導彈,需要用電磁彈射來增加不到10公里(我的論證發現Mach 5初速只增加13公里的實際射程,大型導彈不可能以電磁彈射加速到Mach 5)的射程嗎?如果價錢是0,或許可以的,但是絕對沒有實戰上的大影響(實際上發射射速比起垂直發射大爲減低,不適合用在必須對抗飽和攻擊的對空導彈)。
射程是300公里的火箭彈,有必要換成貴100倍而且彈頭載荷小10倍,“射程”只有220公里的電磁炮嗎?絕對沒有。謊言重複1000000遍,還是謊言,只不過成了1000000個蠢蛋相信的謊言。
\section*{【空軍】【海軍】【陸軍】共軍小道消息刷新(2017年第四季)}
\subsection*{2018-02-02 04:12}

這個“電磁炮”不是美國人做的或者我以前談過的電磁炮(Railgun),原理顯然不一樣,炮彈似乎不是綫圈環路的一部分;它的綫圈是圍繞著炮管的圓柱形,那麽“炮彈”本身就必須有強磁場。這是很奇怪的,因爲永磁磁鐵很貴。或許它有我目前沒有想象到的妙招,我們必須等更多的細節。
我對電磁炮的反對意見來自兩個考慮:1)炮管磨損比火炮嚴重得多,這是工程上的問題,共軍的新設計可以解決它(但是會有其他的毛病,例如上述的炮彈磁場);2)高速動能彈在現代海戰沒有什麽用,這是物理+軍事的問題,很難想象這個新電磁炮如何脫困。
我在這裏再詳細解釋一下最後這一點:高速動能彈如果平射,因爲海平面大氣密度高,阻力大,不管初速多高,射程還是不可能超過40公里(剛好是地球曲率所造成的海面視界),而現代海戰不可能讓敵我接近到這種距離。如果曲射,那麽精度必然一塌糊塗,必須有制導,炮彈的價錢一下子提高兩個數量級,接近導彈了,可是射程和航路的靈活度卻遠遠不如,定位十分尷尬。
至於用來反導,射速必須極高(~每分鐘10000發),口徑反而不重要。照片裏的炮顯然不是為這個目的設計的。
從照片來看,這個設計最可能的,是電磁助推增程的對地攻擊火炮,口徑在200毫米左右。因爲是對固定的地面目標攻擊,制導相對簡單,只要口徑夠大,能裝載足夠的炸藥,價錢還可以接受。炮彈受發射藥和電磁綫圈雙重加速,炮管的長度和厚度都可以減半,以往要10000噸的重巡洋艦才能搭載的8寸炮,現在2000噸的小船就可以搞定,曲射又有超過100公里的射程,那麽我想用來對臺灣這樣的目標做先期轟炸,還是可能有經濟效益的。
\subsection*{2017-12-22 03:07}

三角翼是超音速阻力最低的翼型,缺點是低速時升力低,不利於狗鬥。加上Canard之後,在次音速的靈活性和超音速配平都有良好的效果,所以公認是目前已知最佳的殲擊機氣動構型,自從瑞典人發明它之後,采用的有Gripen,Rafale,Typhoon,J-10和J-20。其中J-20是第一個解決Canard和隱身之間矛盾的戰機。美國人因爲發動機特別強,所以一向不屑在氣動設計上冒險,也就沒有對Canard造型做足夠的技術纍積。
不過光是三角翼並不足以使航速明顯超過馬赫2,發動機和機身也必須爲之高度優化,其他性能大受影響,再加上飛行高度也有嚴格限制,所以馬赫2以上的航速被認爲是沒有實戰意義的指標。第五代戰機的所謂超音速巡航,指的是馬赫1.5-1.8之間的速度,重點在於不須開Afterburner,所以可以長時間使用。F-22和J-20在開Afterburner之後,可以短暫衝刺到馬赫2.2-2.4左右,不過這是逃命用的緊急手段。
\section*{【基礎科研】評悟空衛星}
\subsection*{2017-12-06 03:45}

Glashow只是反對超弦和超對稱,他到底還是做了一輩子高能物理的人,不論是情懷還是眼界,都放不下;尤其是中國出錢,他連對同胞的關懷這個因素都無須考慮。 
 
我從來沒有說他會反對建對撞機,連提起他的名字,都是因爲丘成桐一味胡扯出身這個問題才簡單陳述事實。 
 
其實他的立場如何,是由他個人的職業決定,在邏輯上和問題本身的是非完全無關,唯一的關聯就是我做爲他門徒的人際關係;這是舊封建社會的思想,在美國是很難想象的。 
 
此外,像是我的父母都是激進的臺獨,那麽高能所的意思,就是我不應該批評臺獨了。其實我已經52嵗了,如果還沒有自己的氣節和定見來擇善固執,才是真正的有負父母養育之恩和師門教導之義。
\section*{【经济】漫谈近来的经济态势}
\subsection*{2017-10-15 00:00}
你确定他用的数据是对的吗?在美国,对应M1的V是6左右,对应M2的V都有2以上。臺湾的V如果真的掉到1以下,那是非常不正常的。

我想他用GDP来简单估计P*T(也就是他説的P*Q)是不对的。例如组装一个手机,必须先买零件,零件商又有自己的上游。这些层层的交易,在P*T是简单的叠加,但是GDP必须消除重复计算,只考虑增加的附加价值。

臺湾的经济有虚胖的现象,但那主要是受目前全世界热钱充斥的影响,应该不是统计局有意作假。

"前瞻计划"的主要目的是为蔡英文和她背后的土豪们捞钱。当然,附带的作用是短期继续吹大泡沫,长期会使崩盘更加严重。

臺湾若要避免经济自杀,除了统一让中共来管理经营之外,就只能希望多数选民在很短时间内就长出他们一辈子都没有发展完全的大脑。这当然是不切实际的。\subsection*{2016-06-22 00:00}
伏击区是战术上的细节,是典型的纯用专业词匯来唬人,实际内容与潜艇是否用得上的战略考虑完全不相干。

我在以前几篇文章已经解释过,臺湾建潜艇,第一在反共军登陆完全用不上;第二,没有AIP的柴油潜艇的静音航程微不足道,攻击一次就必然死于对方反击;第三,没有空优,连在进入伏击区之前就可能被击沉;第四,在反封锁作战中,除非性能至少等同共军潜艇,否则也是自找死路。

油轮是民船,如果国军主动攻击民船,对方就可以也攻击臺湾非军事目标,封锁战也就顺理成章地演化为统一登陆战。当年美国就是因此而加入一次大战的,臺湾真的要效仿德国的无限潜艇战争?

臺湾的造船业有钱可赚,就跃跃欲试。但是现代的柴油潜艇已经发展到二战后的第四代(十年后,共军应该会有第五代柴油潜艇),美国衹有到第二代的建造经验,国军更衹有到第二代的使用经验。苏联当年花了20几年才开发出第三代,臺湾能建得出来吗?如果衹能建出落后对方两三代的潜艇,那就是保证去送死。潜艇一旦被击沉,连个别水兵逃命都不太可能。这些政客和商人,为了自己的钱包,要把潜艇兵以接近100 \% 的机率去送死,可恶至极。

还有,这离题了。后续讨论应该转到我批评国军潜艇采购计划的那几篇文章去。\section*{【经济】谈GDP数字的局限性}
\subsection*{2017-10-10 00:00}
没有错。

 \$ 18.04T是2015年的美元, \$ 18.57T是2016年的美元。直观上总成长是3.0 \% ,其中含通货膨胀1.4 \% ,所以GDP成长率=3.0 \% -1.4 \% =1.6 \% 

中国的GDP还要受匯率影响。2016年,人民币对美元贬值,所以用美元计价的增长看来不大。

以美元匯率计价的中国GDP没有什么意义。如果真的要比较中美经济的体量,至少应该用PPP计价,最好是看工业產值。同样一块美元的GDP,是工业,还是诉讼或金融,差别很大。

目前中国的工业產值约当美国两倍左右,所以人均只有后者的一半。要达到人均工业產值的齐头,大概是2030-2035年间的事了。届时美国约等于华东,欧洲等于华南,日本相当于一个省,臺湾。。。\subsection*{2015-06-04 00:00}
TPP做为一个传统式的自贸协议,其实只是美国的次要目的。中国已经跟过半的TPP成员有了自贸协议(主要的例外是日本、美国和加拿大)。TPP成员中,并没有像中国这样的新兴工业国家(如南韩和台湾)。而且美国此前也已经和过半的成员签了自贸协议(主要的例外是澳洲和日本)。所以TPP在这方面,意义不大。

欧巴马急着签TPP的真正用意,是要藉此建立以往WTO没有的贸易规则。除了在版权和专利上,採用极为严苛和广泛的美式标准,最重要的是保护跨国公司,使未来各签约国不能自行决定会改变商业环境的国内法(例如禁烟)。澳洲和日本签了这个卖身契之后,基本上就对美国金主完全放弃主权,所以TPP有如葵花宝典一样,要先自宫主权才能入门。

中国刚要建立各类管理规章,而过去百年来任意以国内法规蹂躏外国公司的美国现在却不愿受他国法规约束,所以中国绝对不会笨到挥剑自宫。中方顶多藉口可能加入,来询问一下TPP的进度;是欧巴马故意扭曲事实,藉此编出中国要加入TPP的消息,主要还是要忽悠日澳两国乖乖引刀自宫。\subsection*{2015-06-02 00:00}
你在《大停滞的真原因》的那张图上可以看出美国的中位家庭收入大约是五万美元出头,其实2013年的确切数字是 \$ 51939,同年的人均GDP是 \$ 52939,两者基本相同。2014年的美国人口是318900000,而家庭数是115227000,所以平均每个家庭有2.768人,那么人均GDP和中位人均收入的比值就是2.768\&times;52939/51939=2.821。当然在一个完全平等的国家,这个比值应该是1;如果是完全不平等的,比值会是无限大。

一般计算收入平等程度用的是Gini系数,它和上面的比值没有简单的函数关系。在一个完全平等的国家,Gini系数应该是0;如果是完全不平等的,Gini系数会是1。Gini系数又分税前和税后两种,根据Pew Research Center在2013年的统计,美国的两个系数分别是0.499和0.380。这两者中当然是税后的才有真正意义,美国的税后系数在31个已开发国家中排名第二高(亦即极为不平等),仅次于智利。

我找不到中国的税后数据。税前的数字和美国相当,但是开发中国家一般会有较高的Gini系数。\subsection*{2015-06-01 00:00}
我对黄仁宇或他的着作都不熟。

对近年来的大数据热潮,我是存疑的。在真正复杂的系统里,各因素的贡献不是线性的,而是有彼此加成或抵消的现象。但是大数据分析基本就是指纯粹从数据出发,而没有先对现象本身建立理论模型,所以这个分析基本上只能是线性的。线性的现象有是有,但是大部分算是低垂的果实,早已被摘下来了。

如果你是决策者,我会建议你不要迷信大数据,因为它基本上是丢给电脑的黑箱作业,到底线性假设对不对,很难看出来。所以除非是有理论根据,事先认为应该有线性反应,否则大数据分析结果的对错无法判断。

如果你是从业人员,只为混一口饭吃,大数据用来自欺欺人很有效,你可以轻松吃一辈子不怕被抓包。\subsection*{2015-05-31 00:00}
印美钞的极限是很大的。Bretton Woods系统下黄金定价 \$ 35,自1973年美元可以自由发行后,现在黄金是 \$ 1200,增为34倍,亦即美国多印了约34倍。

我的资料是M3在2008年达到13万亿美元(你的数字是不是以人民币算的?),此后联储会不好意思再公布,不过QE总共将近5万亿,所以现在应该是18万亿左右,比美国一年的GDP略高。

18万亿好像没有很多,可是里面有很多是1970、1980年代发行的,那时的美元比现在的购买力强3-5倍。

此外,美国人只有在自己经济低迷的时候印钞,不管别人是否通货膨胀,这是一种期权(Option),它的价值和印的钞票本身是同一个数量级的。\section*{【基础科研】对杨振寧被PRL退稿一事的分析}
\subsection*{2017-10-02 00:00}
我原本以为《辉煌中国》会都是像前两集那样,纯粹列举科技、工业和基建的成果。这些事我基本都已知道,宣传片内容又太浮面,对做深入分析没有帮助,所以我就没有看。

还好你提醒了,第三集是很感人的,尤其是那个西藏的村支书,让人和她一起流泪。

当然《辉煌中国》是为十九大做准备,宣传过去五年政绩的广告片。不过我可以明确地说,光凴第三集前半讲的扶贫政策,就绝对是举世无双。中共和习近平对扶贫之用心,其动用的人力、脑力、物力、财力资源,对比什么公投、普选那些骗人的花样,很明显的,前者才是真正的合法性、正当性的来源。臺港那些愿意被骗的蠢蛋,自作自受,倒也罢了;把对政治无知无感的贫苦民众也拖下水,是绝对的集体谋杀,如果有天理的话(可惜没有),都应该得到谋杀犯的惩罚。

不过这个话题在一篇物理文章底下谈,不合适。我有专门讲扶贫的文章。\section*{【医疗】现代医疗的大倒退}
\subsection*{2017-09-30 00:00}
医疗和法律一样,完全不适用于市场经济;如果像美国这样因为被财阀洗脑而硬上,结果就是人命价值与个人财富成正比。

美国不但没有全民保险,而且2017年的医疗花费已经高达GDP的18 \% ,是西欧国家(都有全民保险)的6-9 \% 的2-3倍。这主要来自三方面的浪费:1)迷信市场经济,硬是多出了私有保险公司这一层中间人的剥削,而且繁文缛节、效率低下;2)中上阶级的医疗保险,居然享受了税制上的大幅补贴,以致极尽豪华;3)美国的制药公司政治能量太大,追求的利润太高。Bernie Sanders的法案可以解决前两个问题,但是因为动了许多既得利益者的蛋糕,过关的可能性极低。

大陆的新健保,听来很好,但是必须注意长期的财政可持续性。臺湾的全民健保,算是过去30年少数几个方向正确的新政策,但是背后的财政管理乱七八糟、寅吃牟粮,只怕也是将来中共必须买单的项目之一。\subsection*{2017-08-14 00:00}
你是在城市里(臺北?)长大的吧?

在乡下,每个村都有几个望族,通常是大地主。要知道一个村里的望族是谁,一个办法是问问庙会的钱哪里来,另一办法是看谁是选举的"桩脚",答案都是同一群人。

我不是说他们个人天生邪恶,而是他们的社会地位,先天就使他们与国家整体利益对立。

因为他们垄断了乡村的经济财富、政治权力和民意宣传,中国歷史上的中央政府,包括国民政府在内,都必须笼络他们。在隋朝以前,是乡举里选、九品中正等等;之后则是科举。清末废了科举,那么就只能搞地方自治。名义上是自治,其实是政治分赃,但是还不如18世纪法国大革命之前直接把收税的职能拍卖给出价最高者那样公开、诚实、透明。

这些世族的权力越大,贪腐的现象就越严重,全国的政治和经济规则也越向他们倾斜,损失的则是中央政府、城市里的中產阶级和乡村里的农民。后汉是直接因世族尾大不掉而灭亡的,其他的王朝也几乎都在中期之后因世族反对而无法改革,因而步向衰亡的命运,例如明朝的东林党,就是代表世族的既得利益集团。

未来的歷史学家,必然会把中华民国也加到这个名单上。

还有,中共的土改会搞到那么血腥,其实不是因为他们有远见,知道要为现代化的工业社会奠定基础。共產党没收土地,固然是他们的理念,但是把已经被剥夺财產的地主们斗争至死则是刘少奇在内战期间推动的。1947年三月,延安被国军占领后,他跑到河北,发现发动贫农开斗争会斗死地主之后,这些贫农都知道手上染了鲜血,如果国民政府获胜,自己会有大麻烦,因此自然成为坚定的解放战士。刘少奇在1947年一份给党中央的报告里面,解释了这个现象,宣称效果惊人。从此这被推广到全国。

一般人以为1947-1948年起,内战的天平开始向共產党倾斜,是因为苏联向林彪交付了大批军火。其实那只解决了枪的问题,人的动员是靠血腥土改解决的,而且这些新的兵员有狂热,战斗力很强。我个人认为影响比武器更大一些。\section*{【陆军】现代坦克装甲原理简介}
\subsection*{2017-09-23 00:00}
共军的装甲技术远优于俄军;与美军路綫不同,但各有千秋,例如美国坦克就不用ERA。所以共军的第四代APFSDS和美军的第五代APFSDS就完全不一样:前者是为击穿贫铀装甲而优化,所以沿径向用了不同的材料,后者则是针对ERA和NERA,所以沿轴向用不同的材料。

99式为了控制重量和价钱,必须做若干取舍,最大的牺牲就是侧面装甲。这是因为以往共军的Doctrine(基础理论)是战略防御,而且是机动防御,所以99式坦克是纯为集群野战而设计的。侧面装甲就可以靠Maneuvering(机动)和队形来弥补。

M1的侧面装甲必须特别厚实,除了美军常打治安战之外,还有另一个原因,就是M1的弹药全部装在炮塔后部,如果炮塔侧面被击穿,那么弹药有殉爆的危险,成员必须马上逃命,这辆坦克只能等战斗结束后再回收了。99式的弹药全都装在底盘上,炮塔侧面被贯穿不一定会强迫坦克推出战斗。

一般的所谓坦克正面,包括左右各30\&deg;之内的扇面里的所有敌军火力,都应该受最强大的装甲保护。99式似乎在这方面也偷斤减两,有最高级保护的角度不到\&plusmn;30\&deg;。当然这并不是说敌军坦克在30\&deg;侧面就可以轻松击毁99式的炮塔,而是会有稍大于0的机会能避开正面装甲。不过以这种角度入射的弹丸,基本不会伤及乘员和要害。

96式的尺寸、悬吊和发动机都承受不起99式级别的装甲。

15式的装甲保护,应该是接近96B的水平,绝对不如99式。不过共军原本就重视步兵,如果在海外打巷战,大概不会像西方军队一样纯靠坦克打前阵。\subsection*{2017-09-22 00:00}
现代战争是体系的对抗。而坦克在体系里的角色,远没有隐身战机重要。

美军的观点是,先占据空优,然后由空中力量压制打击敌方的坦克集群,所以M1眼看着要服役40年了,仍然没有发展下一代坦克的必要。

当然,这是沿袭冷战期间,对抗7万辆华约坦克的策略。而共军则反过来,继承了苏联的战术思想,对坦克集群野战衝锋有较大的依赖和重视。这样的军事理论偏好,对99式和96式的设计有很大的影响。

15式的重量上限的确是为了能上战术级运输机而确定的,所以除了在山地等特殊环境保持高机动能力之外,对海外做快速反应也是未来的主要用途之一。不过这类海外任务,不会是和美军这样的对手硬碰硬,而会有或多或少的治安战成分。下一代的坦克改为双人,并且高度信息化已经是必然的,但是是否也随欧美朝治安战倾斜,倾斜的幅度有多大,还有待观察。\subsection*{2017-09-12 00:00}
那一年,我在金西师炮兵营当少尉观测官,隔壁就是着名的金门军中乐园,有时我会带队到去帮他们扫路。

坦克的履带一直是弱点。现代坦克有一大堆观瞄设备暴露在外,更是禁不起爆炸的衝击波。你如果读过《泥泞中之虎》(Otto Carius的自传),就会知道,虎式坦克虽然不怎么怕对方的直射炮火,对苏方的曲射炮兵(也就是苏联炮兵的少尉观测官向后方呼叫来的火力打击)却必须小心防备,否则履带被打坏了,轻则必须退出战綫,重则会被遗弃在战场上。像Carius这样有责任心的坦克车长必须连夜勘察防御阵綫,把每一个路上的弹坑都记录下来。一旦战斗开始,他的驾驶只管全速飙车,打了就跑。

现在的坦克还是一样的,在前綫敌人视距内停下来不走就是找死。还好现代坦克有运动中射击的能力,完全没有停车的必要。当然,这是野战,巷战又是另一回事。\section*{【台湾】2020年前的台海战役}
\subsection*{2017-09-17 00:00}
不同朝代,不同的体系,也就是不同的国家。

这些建国者的后世评价,其实普遍和自身的作为没有太大的关系,反而是国家后来的国运影响较大。邓小平掌权之后,不愿冒动摇国本的危险,公开批斗毛泽东;但是在当时,绝大多数的中共干部私下都知道毛犯的错误有多大,对国家有多少损害。现在的对毛怀念,其实是把邓小平改革的成果,转嫁在毛身上。邓自己是为国为民干实事的人,不在乎这个,但是我们后人如果要对自己诚实,还是必须搞清楚。

美国的建国者,能有现在的宣传优势,靠的也不是自己的努力,而是一方面美国可以简单地对土着进行屠杀掠夺,不断地有丰富的物產和优质的土地注入经济,另一方面,每次有大危机,都刚好有林肯和罗斯福这样的英明领袖来引导国家。説来説去,也就是运气。

李登辉在20多年前靠权术运作搞臺独的时候,我想也是预期只要建国成功,自己的卑劣手段都不会被后世在意。但是臺湾和美国相比,既没有种族屠杀的机会,也没有对外用兵的能力,更没有让本土出身的人才依靠能力脱颖而出的文化社会和制度基础,所以即使建国能成功(其实臺湾实质上早已是独立的政体),国家前途渺茫,他的歷史评价也会是很低的。\subsection*{2017-09-16 00:00}
你是在说海峡两岸同样都有斯德哥尔摩症候群吗?乍看之下,似乎如此,但是实际上有一个很大的差别:毛泽东是建国的领袖,臺湾则只是日本的殖民属地。

任何一个国家,都会将建国者与国家本体做心理联系,因此自然在群众之中有美化他的趋势。美国建国的"Forefathers"都是土豪大地主,为了阶级自身的经济私利,搞虚假宣传裹胁全国、以牺牲小民的生命和经济利益来叛国自立,事后一样扭曲事实,自我美化,成为国内外普遍崇拜的对象。华盛顿砍樱桃树的故事,就是在这个背景下被后人发明的。

我儿子现在正在准备考美国歷史的SAT,我和他谈起这事,他本能地反弹,说美国独立是为了反抗"无代表的抽税"(Taxation without Representation,这是独立战争的主要口号)。我说英国对美国十三州抽的税率是本土的1/30,英王当然愿意让十三州代表进入国会,顺便把税提高30倍,但是殖民地愿意吗?我儿子不服,花了时间去找资料,最后回来说不是1/30、而是1/26。我説那是平均每人税负的比率;当时殖民地物產丰富、地广人稀,人均GDP已经高于英国本土,那么算税率岂不是低于1/30?

着名的波士顿茶党(Boston Tea Party)的确是针对英国修改茶叶进口税而做的暴乱抗议,不过事先进口税并没有增加,而是反过来被大幅削减了。暴乱者是走私茶叶进口的当地土豪,因为生意受损而鼓动手下去打击合法进口的低价茶叶,这能算是为国为民抵抗暴政?

华盛顿在早先的英法七年战争里做军官,所以歷史书上都说他热爱祖国、为国服务。其实英法当时在全球竞争,重点在亚洲,十三州的土豪们想要扩展自己的地盘,向西越过英法有协议的界限,对英国来説是暴民自行在战略次要方向放火惹事。华盛顿不但不是为国服务,而且是以私害公,把国家拖进一个不利整体、专为当地土豪开拓土地的战争。现在又有多少人了解这一点?

华盛顿上任总统之后,根本懒得管理琐碎的政事,天天开餐宴舞会,结果现在还是被一般"歷史学家"列入"伟大总统"之列,评价甚至在罗斯福之上,这又有什么事实根据了?

我讲到这里,我儿子只能求饶,说他在学校还要做人。以往不小心泄露了口风,转述了我对美国的批评,已经为他招了很多麻烦;这些振动美国人对国家根本认知的史实,必然会激起老师同学们的强烈反应,而且考SAT也可能会因此答错相关的题目。他上的是左翼的私立学校,尚且如此,美国人被洗脑的程度可想而知。

总之,我同意大陆有对毛泽东做严重美化的趋势,但是这是世界各国的普遍现象。至于像臺湾这样对外来侵略者下跪膜拜、对自己祖宗抹杀扭曲的做法,那才是举世罕见的。\subsection*{2015-12-08 00:00}
美国的传统潜艇技术已经荒废了50多年,即使由日本指导,制造起来衹怕仍然旷日费时,价钱成倍增加更不用说了。更大的问题是这是明显的美日同盟针对中国的动作,以往的藉口都不再适用,所以欧巴马绝对不肯。下一任总统则很难说;现在有几个疯狂的候选人,连Hillary都比欧巴马更高调反中。

至于日本,我对安倍的政治智慧已经不再有任何期望,所以什么疯狂决策都可能发生,包括直接卖潜艇给臺湾。Erdogan打下俄国战机也属于这类事前不能以情理推论的蠢事,而安倍和Erdogan两人的背景和行为模式是颇为近似的。

马英九还没有退休,就放任军方炒作这个馊主意,等于是为蔡英文作嫁;他的愚蠢真非常人所能理解。民进党上臺之后,不会敢公然搞独立,但是用重利诱惑美日来公开为隐性臺独站臺,却是极为理想的"外交策略"。民进党的思维模式是一维的,唯一想得到看得到的就是操弄民粹,而这事不但会使中共暴跳如雷、强化美日对臺的"支持",更棒的是臺湾民眾会完全支持"增强国防",所以不论成败,都是他们眼中的好牌。几个月后,还有的闹呢。\subsection*{2015-09-20 00:00}
我两天前仔细地读过了那篇报告,原本是想写一篇文章的,但是发现它没有太多新意,也就作罢。

那篇文章的价值在于它把美军的武器和战术算得十分精准,但是对共军就很外行了。例如KJ-500、056反潜型、Zubr和DF-26在报告里都不存在,许多导弹的射程被严重低估50 \% 左右。我觉得它的所谓2017年的决战,其实比较像是2012年时的对比。它的结论是如果美国不计损失,全力介入台海战争可以跟中国打成平手,这在几年前大概是合理的,到2017年可能就不切实际了。

不过共军也不能太得意。那篇报告简单地提到美军如果对中国做长期的海上封锁,共军将有还手的困难;这点我想大家都同意的。然而这只有在美国决心和中国做生死搏斗时才会发生,从政治层面来看是不可能(除非美国能把欧洲拉进来一起制裁中国)。\subsection*{2015-08-25 00:00}
美国的学校是镇办的,所以水准差次不齐,主要视镇上收的税有多少。像我这个镇平均每个学生(从幼稚园到高中)所收到的预算是每年2.2万美元,算是康州的前列,但是官僚气息很重,对学生要求太松,我最后还是让小孩念了私立学校(但不是寄宿的)。

大陆听说是刚好相反,把小孩逼得太紧了些。我的两个外甥女在台湾念英语学校,一个到大学、另一个到高中才到美国念书,或许这是最好的折中,不过花费还是不少。高中就出来的学生,有受同学影响而成为纨绔子弟的危险,那个外甥女天生爱读书,我们又为她安排了知名很严格的寄宿学校,所以没变坏,不过一般小孩还是跟着爸妈到大学再离家比较保险。

总之,英美的公立学校都不可靠,私立寄宿学校又贵又必须离开父母,还是先在本地念完高中比较合适。\subsection*{2015-08-02 00:00}
I think the answer to your question really depends on the individual circumstances, most importantly, what kind of job you have in Taiwan right now and what kind of job you expect to get in the new place. China has a lot of opportunities but is more cut-throat. I would recommend it to younger people. Taiwan, on the other hand, may be ideal for retirement, as long as your retirement fund is not tied to the government budget.\subsection*{2015-04-24 00:00}
只要中共的实力持续增长,台湾在不远的未来被统是必然的。我当然最希望台湾人觉醒,接受文统,因为这个选项对台湾人民是最有利、痛苦最少的。但是如果他们坚持忽略现实,莫名其妙地创造仇恨、自找麻烦,那么对中共领导人来说,就不得不武统。武统又有两个方案,一个是我以前提过的快打,另一个就是你现在说的慢磨。不论哪一项,其目的都在于制造足够的痛苦记忆,使百姓不再有兴致胡闹。德国和日本在二战后都被打得服了,就在于战争的破坏极为彻底。这也就是我以前说过的,越是血腥的征服,事后的麻烦越少。

你的方案会把台湾彻底打扒在地,50年不能翻身。或许那才是最有效的吧。不过带头闹事的人是元凶,在快打选项下被处理掉也是罪有应得;但是慢磨就会伤害所有台湾人口,包括很多下层百姓的生活和生命都会因此而更苦。我实在不忍心多想。\subsection*{2015-04-23 00:00}
我人不住台湾,也不看台湾的电视,不能保证绝对没有报导。

请注意我的用语,"基本无视"和"整体无视"是两回事。我所读的三个报纸都没有报导,对我来说,这使我有资格说媒体"基本无视"。

我的确注意到大陆年轻人喜欢吹毛求疵。不知是为了带给自已虚荣,还是一般学习环境就是如此。我觉得有批评就说是应该的,但是批评必须是切题而且有根据的,尤其要确定批评的对象写的的确有错,是批评者的责任,不是被批评者的责任。我在大陆网站常被喷子围攻,而这些喷子又分为三类:有完全不入流也不知所云的;有先误解原文再攻击假目标的(这在英文里叫Straw Man Fallacy,也就是自制一个稻草人来打倒以显示自已武功高强);也有在无关紧要的细节上,紧咬不放,而且越扯越远的。最后一类一般有一定的知识,所以急着要找机会炫耀。我在哈佛的时候,一位诺贝尔奖得主解释硕士和博士的差别:硕士是把这行里的重要知识都学全了,而博士则是已融会贯通、能自主创新的人(至少理想的目标是如此)。我拿博士已是20多年前的事了,拿一些琐碎的知识来炫耀,还真不是我现在习惯的事。

我当然并不是针对你,也不是称你为喷子;只是这些感想在脑里有一段时间,又不适合写文章。你刚好提醒了我大陆青年直言敢言的趋向,我就顺便写下来。请不要因此而觉得受侮辱。

至于电视名嘴评论离谱,正是我懒得看的原因。去年我在台湾一个半月,看的电视基本上就只有世界杯足球。\subsection*{2015-04-23 00:00}
Continental Balance of Power其实只是为一己之私考虑下的理性策略,中国早也有了,就是戦国时期的连横。它在西方也不是英美的专利;1853年的克里米亚战争就是法国挑拨英国去打俄国,它的目的是为了打破拿破仑战争后,英俄同盟包围法国的态势。二战后,英法在旧殖民地处处留下未来冲突的导火线,所以才有巴基斯坦,这次中国才能有在印度洋的出海口。如果印度保持了统一,中国的南线就会危险得多了。

台湾眼看着已经无力对大陆沿海做严重打击了。反过来看,台湾对大陆真正有价值的资产,也就只是台积电一家;要在戦争过程中适度地予以保护,并不是很困难的。

至于被利用后,家园破碎,反而更坚定地当走卒,这正是乌克兰的命运;很不幸的,台湾人只怕也会这样,除非有血腥的征服或者及时的觉醒。我当然是寄望于后者,但也知道希望不大。\section*{【陆军】【海军】【空军】共军小道消息刷新(2017年第三季)}
\subsection*{2017-09-09 00:00}
航母集群,视其与假想敌的实力对比,有三个不同的级别:

最低级的是护航航母(Carrier Escort Group),例如苏联的航母主要只是用来保护核潜艇不受北约反潜机骚扰。守势作战只需要制空,还会有岸基兵力支援,所以对航母的打击能力和飞机数量都要求不高。

力量大到足以进行攻势作战时,叫做航母突击群(Carrier Strike Group),歷史上第一个航母突击群是日本海军的"机动部队",由6艘中型航母和300+架飞机组成,珍珠港事变就是航母突击群进行攻势作战的第一个例子。

航母突击群的飞机数量和弹药投射能力,达到了能够奇袭重创敌方大型海空基地的程度,但是在对方有准备的情形下,还不足以进行持久作战、用蛮力把敌人打成稀巴烂。能够正面硬攻,仍然压倒敌方一级海空要塞的,叫做航母战斗群(Carrier Battle Group)。史上的第一个航母战斗群,是美国在1943年用6艘大型航母(Essex Class,后来一共建了24艘)和500+架飞机组成的。这个战斗群一旦出现,日本在太平洋苦心建立的一连串要塞,就只能等着被一个接一个地拔掉。

中共海军的航母集群,在可见的未来,相对美军来説都只能算是护航航母。要跑到美国沿岸去打击美国本土,必须要连升两级,这只有在中国国力已经完全压倒美国的前提下才有可能。但是如果已经到了那个地步,臺湾、南海等等问题必然早已解决,中国将是纯粹在霸凌美国,就如同美国现在霸凌朝鲜一样,因为对方有核反击能力,仍然不能真动手。

003开工的证据不足,必须再等等。逻辑上是蒸弹航母被取消,下一代航母提前,但是这个过程仍然必须有好几年,所以现在就开工不合理。\subsection*{2017-09-08 00:00}
F-35A主要是用来取代F-16和部分旧型F-15的,F-15E反而不会被取代。美国空军的上一代战机中,只有F-15E最新,还可以再用20年以上;其他飞机的结构寿命在一延再延之后,已经没有太大再延寿的余地。

当然老式的F-15是纯空优机,所以它们退役之后,美国空军就只剩下180架F-22专职空优;F-35A对这些即将退役的F-15C/D,并不是同类机的换新,而是不同类的转换。在2025年左右,将形成F-22+F-15E+F-35A的高低搭檔。

这样的汰旧留半新,并不完全是歷史偶然和财力限制的结果,空军的任务环境也变了。F-15C/D和F-22都是针对冷战设计的,如果美苏真打起来,必然会在欧洲上空有持续的超大规模空战,所以美国空军需要1000架左右的专职空优机。一旦冷战结束,假想任务成为海湾战争那样的地区性中小规模衝突,对如伊拉克那样的对手,用多任务飞机效率更高得多,所以在作战任务上,F-15E反而比F-22更新一代,虽然后者的开发周期太长,以致在前者之后才服役部署。

因为有歷史遗留下来的180架F-22,F-35A可以不必考虑空优任务,所以一开始官方的定位是攻击机,但是实际开发出来的F-35A其实是一架典型的后冷战多用途飞机。那为什么不继续买F-15E呢?除了美国空军对高精尖的自然偏好,以及F-35集团的政治压力之外,有若干隐身能力的飞机,在中低烈度的战场环境下,有远远更大的使用弹性,真正符合"多用途"的基本概念。

中共所面对的战略环境,则完全不同:整个现代化建军的首要任务,是为了吓阻美国对中方开战,再加上军工技术仍然在追赶霸主的过程中,所以专职空优的J-20不但多多益善,而且牺牲一些对地打击能力是必要的。既然会有足够数量的隐身空优机,而且J-20虽然载弹量小,仍然有全套对地打击用的电子系统,那么半隐身对用来辅助J-20的多用途飞机来説,就是一个没有必要的奢侈浪费,反而是J-16这种以航程和载重见长的飞机最合适了。

总之,中美双方战略环境不同、技术根底不同、歷史包袱不同、最尖端的隐身战机在空军装备里所占的百分比不同,所以辅助用的多用途飞机也就有不同的侧重。

预警机和电战机无法靠形状隐身,衹能依赖电子干扰。\subsection*{2017-09-07 00:00}
J-20换装发动机这事,我思考了一天,觉得可能性最大的脚本如下:和99M1纸面性能相当的WS-10B/C/D(都是同样的东西,只不过机匣和附件的位置不同)在刚有原型机下綫的时候,原本表现很好,所以在去年共军决策单位决定开始试装。至于平行换装的理由,可能是国產优先,也可能是当时99M1刚好出了品管问题,更可能是WS-15最早要到2020年才设计定型,未来三年,J-20的產量会高到需要AL-31F和WS-10并行装备的地步。很不幸的,后来WS-10B/C被发现有大问题(叶片断裂),生產暂停。所以这张照片出现的时机就有点尷尬。

J-20今年的產量大概在20架以下,还不够一个团。而且中共空军也在团改旅的过程之中:一个团24架,一个旅却是32架。我想后年的生產目标可能是一个旅,明年则可能还达不到。

这次军改的变动之大,远远超出一般人的理解,连基本的坦克连、防空连、自行炮连和其他的机械化连,都由一个连9或10辆车增加到14辆。换句话说,纸面上同样的连,2018年版比2016年又至少强大了40 \% 。其结果是,新的合成营有大约五年前(即习近平上任时)一个团的战力,新的合成旅则强过五年前的一般机步师。那么新的集团军就比旧的强一倍以上,指挥层级反而少了一层。难怪可以从18个军削减为13个。

15式坦克也有液气悬吊系统(为了节约成本、延长寿命、方便维修,只装在前后端一半的负重轮上,不像日本人一样搞华而不实的全液气悬吊),但是在弹药和装甲技术上则有压倒性的代差优势。所以虽然火炮口径小些,重量轻了10吨,火力和防护却与十式基本相当,在价钱和机动性上则远胜。\subsection*{2017-09-07 00:00}
太行发动机也就是WS-10系列。除了基本型之外,WS-10A是为了适应J-10的机身而改动了几何结构的版本(把机匣移到下方),因为没有赶上J-10A的研发和生產节奏,并未量產。WS-10B是增推5 \% 的型号,13.2吨的推力,但是因为有严重的技术问题尚待解决,到2017年仍然只生產了不到10具的原型机。WS-10G(又叫WS-10IPE)是应用WS-15技术的魔改版,预定14.5吨推力,因为进度相当落后,已经不可能在WS-15之前服役,被迫下马(严格来説,是"转技术储备")。

所以目前堪用的太行,只有12.5吨推力的基本型。虽然上个月终于正式"生產定型",但是在实际应用的稳定性、敏捷性和可靠性上,还是连AL-31F的基本型都比不上,何况中方早已进口了一批99M1版的AL-31F改进型,被应用在J-10B/C和J-20上。

那么改装太行,在性能上完全是走回头路,再加上成飞必须花费时间精力来重新测试调整,实在没有道理。

所以这件事颇有蹊跷。逻辑上有四个可能的解释:
1)照片是假的;
2)俄方不卖AL-31F了,中方却急着要儘快部署J-20,被逼无奈,只好暂时牺牲一些性能;
3)换装的是WS-15的原型机,只不过WS-15的尾喷口和WS-10类似,所以被误认;
4)杨伟被调离之后,渖飞系的官员想出这个办法来浪费成飞的资源。

《观察者网》的资深分析师(至于"刀口"或"鱔鱼面"那种连简单事实都拒绝承认的蠢蛋,就请不要在此提起;他们的猴子猴孙居然还以为这张照片印证了J-20以往用的就是WS-15,正在《超大》自我庆祝!)都选择(2),理由是AL-31F的產能有限,俄军自用的订单把它们全部买走了。这很奇怪:俄国的国防总预算还在紧缩的过程之中,怎么战机的订单反而大幅增加了?尤其生產99M1的礼炮厰只负担SU-34,產能一点都不紧张,反而是生產WS-10系列的黎明厰,有一大批J-11和J-16在等发动机。就算礼炮厰有些品管的问题,99M1好歹已经在共军服役了五六年,説明基础设计没有大毛病,比起WS-10B来明显要强多了。

而且照片里的尾喷口有锯齿状的隐身处理;虽然WS-10做这样的修改并不难,但是WS-15的基本型据称原设计就只用锯齿型尾喷口(矩形喷口要等WS-15升级之后才有可能),所以(3)并不是完全离谱的。至于(4),据説J-20计划是共军当前军备发展的重中之重(亦即超过航母和核潜艇),渖飞应该不敢搞这种花样才对。

总之,我现在还不能确定真相何在,再等一段时间吧。\subsection*{2017-09-02 00:00}
MIMO是很容易的技术,连现代消费者级别的WIFI Router都基本有了。过往主要是计算速度不达标;现在的CPU和GPU性能过度,根本不成问题。

隐身机的设计原理是把入射波定向反射到少数几个固定的角度,而不是往所有非入射方向散射,所以MIMO对破解隐身并没有特别的功效。而且MIMO的测量精度比较差,所以只能警戒,不能作火控。发现了而打不着也没有用。

现代防区外打击武器射程越来越远。J-20在臺海战役中主要是用来打击预警机和加油机,共军为此研发的最新超远程空空导弹(体型太大,必须由J-16携带,J-20前出导引),射程可能长达400公里,远超国军雷达和飞弹的范围,何况J-20这一级的隐身机可以压缩火控雷达的探测范围至1/5左右,超音速巡航又压缩飞弹射程1/2。

总之,如果用国军那些70年代水平的电子装备,稍作升级就可以破解隐身,美国也不会前后投入万亿美元来做开发了。\subsection*{2017-08-29 00:00}
1960年代,中国在国际上的孤立,是以全国经济发展和人民生活品质为代价,而得到的什么?什么都没有。现在这一代,吃饱喝足了,能理解当时多少人饿死或饿得半死?这种"光荣"和"尊严",是吃饱没事干的弱者打肿脸充胖子的傻事,现在是俄国在干。正因为中国已经快成为世界第一,所以有比"光荣"和"尊严"更重要的考虑。

在2025年,中共可以安全地排除美国在西太平洋的势力之前,任何区域称霸的动作,都会遭受现任全球霸主的无情打击。

印度不是中国的主要敌人。统战的基本原则是联合什么、打击什么,这一代的中国人都忘光了吗?\section*{【战略】对美军在韩国部署THAAD之我见}
\subsection*{2017-09-06 00:00}
射向中国,要达成什么目的?要嘛金正恩是个疯子、傻瓜,要嘛做这个猜测的人是疯子、傻瓜。

金正恩是个厉害角色,比他爸爸强多了。 我在前面已经解释清楚,他是个聪明、理性、有远见、有决断的人物。他不会不明白,核武是个威慑的手段,如果真的使用,那么他自己也死定了。所以只有在左右是个死的情形下,他才可能会下令发射飞弹。但是他还是得考虑儿女的生存问题。他在核战中死后,他的儿女只可能落入两方的手中:中国和南韩,尤其是前者机率最大。哪一个核打击的目标,是中韩都认为活该而会暗自欣喜的?

晶片都是批量生產的,又如何事先知道朝鲜怎么组装成一个系统?发射轨迹是软件控制的,要怎么用晶片否决?

一个网站或作者如果乱做猜测,就应该避而远之。\subsection*{2017-09-05 00:00}
朝鲜经济不好,人民贫困,一大部分可以归罪于美国的贸易封锁。要是政权被推翻了,也不能指望美国花钱来重建国家。如果美国真的在乎朝鲜人民,只要从南韩撤军,中国自然不必再支持朝鲜,那么它的衰亡就指日可待。所以真正愚蠢、疯狂、并且邪恶的,是美国,而不是金正恩。

以朝鲜的国际地位,类似它而放弃核武的,如伊拉克和利比亚,都被欧美找藉口打成无政府分裂状态。逻辑上能自保的出路,只有两条:要么有核武护身,否则就必须有大国庇护。中方当然希望它选择后者,但是从金正恩的角度来看,中国固然不会容许美国消灭朝鲜,但是却有可能会换别人来当权。唯一能保证他自己身家性命安全的,只有核武,所以必须全力以赴。结果在短短几年内,而且无外来援助的前提下,就达到了与以色列比肩的技术能力,也算是一个人定胜天的奇迹。\subsection*{2016-07-23 00:00}
主要是伊拉克战争浪费了美国十几年的时间,等到可以腾出手来,中国已经坐大了。美国在东亚又没有北约,若是开战,衹有日本联手,不但没有绝对胜算,而且必然大伤自己元气。

"以往的思维包袱"就是冷战思维;如果Hillary当选,中美或许打不起热战,但是在军备、外交、经济和贸易上,必然会继续生死搏斗,例如TPP就是Hillary一定会抛弃竞选诺言而推动的一步棋。这种中美对抗再加上Hillary作为财阀代言人所必须执行的其他政策,代表着她当总统对世界整体和一般美国老百姓都是灾难。Trump虽然明显是为一己利益和虚荣而竞选,他极可能撕毁TPP并自海外撤军,那对世界和美国来説都是好事。\subsection*{2016-07-16 00:00}
They are billionaires and will remain billionaires. I tend to reserve my compassion for people with fewer financial resources.

The standard of free speech applies differently to billionaires exactly because they have the resources to influence a lot more people, regardless of the merits of their arguments.\section*{【基础科研】有关环保和全球暖化的几点想法}
\subsection*{2017-08-20 00:00}
我想他的资料有些问题。两千多万年并没有Mass Extinction Event;人类正在造成新的一个 MEE,但是上一次是六千六百万年前恐龙灭绝的那个事件。

自六亿多年前,地球从Snowball Earth退霜之后,才有了多细胞生物。其后共有6个MEE(包括现代的这次),其中只有2.52亿年前的第三次才有超过90 \% 的物种被灭绝了。灭绝一个物种必须把100 \% 的个体杀死,所以死亡的生物必须是远高于90 \% 的。

我对二氧化碳浓度的变迁并不熟,不确定要如何更正他的论述。我的印象是准确的歷史性二氧化碳浓度测量靠的是南极的冰层,只能上述不到一百万年。超过一百万年的,实际上都是猜测。

最重要的是,本文的核心重点,就在于地质性的气候歷史并不切题。我们考虑全球暖化的后果,是为了人类经济的发展。这个前景是以几十年为尺度的,所用的标准不是多少物种会灭绝,而是多少固定资產和社会结构会被抹消。所以完全无须引用这些不确定性极高的古生物学研究。\section*{【海軍】【空軍】共軍小道消息刷新(2017年第一季)}
\subsection*{2017-08-10 00:00}
这我也想过。

最好的办法是让他经济崩溃,而印度经济的软肋,就是他的IT服务业。这些行业是印度赚取外匯的最重要手段,所以只要把印度最大的几十个Server Farm,包括通讯中心、云计算中心和数据库中心打掉,不但生意马上做不成,而且因为有前车之鉴,欧美的顾客绝对不敢再等他们重建,会把生意转移到其他国家去。

我连攻击的细节都想过了:最好是用燃烧弹头。这些电子器材极为怕火,就算没被烧坏,被烟熏过之后,只怕也要报废。

总之,这是让印度的损失从百亿美元升到千亿美元级,最简单的办法。老实说,我自己都觉得有点恶毒,不过印度人和中国死磕定了,不狠狠教训他们,就是对自己残忍。\subsection*{2017-08-03 00:00}
战略上实在是不打最好,反正再三四个月就准备大雪封山,此事不了自了。

但是我也注意到你说的共军紧锣密鼓的警告,包括"印方不应以拖待变"。这话若不是即将动手,就没有道理。

共军长驻西藏的几个轻装旅,远远不够打入印度纵深。所以要打,只能是空战、炮战加上局部性的杀伤,和55年前一样,顶多打到十一月底为止。问题是一旦动手,所有国际上拉偏架的战略恶果都有了,但是没有占领新德里、肢解印度(最起码也应该收回藏南,并帮助巴基斯坦拿回喀什米尔;进一步,则应该强迫印度割让"鸡脖子"的100公里地,使中国与孟加拉接壤,另外也让锡金独立),胜利的战果是很空虚短暂的。中共在外交上一向避虚务实,所以在这时候选择打这一场战争真是有点儿奇怪。\subsection*{2017-06-30 00:00}
这篇文章,我在《超大》看到了。如果是真的,那么十分惊人。作者言之凿凿,或许可信性超过50 \% 吧。

基本上,文中的新消息是,去年底即将开工的018号被暂停,以便考虑更改设计,转为电磁弹射。这个改变,伤筋动骨,我认为至少会延缓制造时程两年以上,所以原本2020年下水的,现在要到2022-2023年才能发生,也就是要到2025年左右才会服役。

从大战略来看,这代表着中共顶层认为016+017两艘航母,在2020年代的前半已经足够应付外来的威胁,所以有余裕可以跳过次优的蒸汽弹射技术,追求长远的经济性。

电磁炮的不实用性,不只是以往的工程缺陷,更重要的是物理上的阻碍。马伟明或许可以解决工程问题,但是物理阻碍是无法逾越的,所以顶多只能用来做飞弹(和无人机)发射的助推,拿来当主炮仍然是不可能的。\subsection*{2017-06-26 00:00}
超音速军机的机尾,歷史上一直都是气动设计不能触及的地方,这是因为发动机的喷气流速度比飞机本身还要高得多,所以发动机喷气口的气动性能,一直比飞机自己机尾的气动性能还重要。而发动机阻力最小的喷气口形状,就是圆形。一旦用了圆形喷气口,基本上就无法与飞机本身的外形相融合。这其实是很突兀的,只是歷史上每架飞机都如此,所以我们也就见怪不怪了。

最早期的歼8,曾经尝试过渐缩式机尾,后来在风洞实验中证明完全没有用,徒然增重,所以就取消了。

F-22主要是为了隐身,而采用了方形喷嘴设计。这使得推力损失了5-10 \% ,但是F-119是现代第一次推力充沛有余的发动机,所以完全可以接受。既然已经是方形了,融合进飞机尾端的整体形状就容易得多,在飞机机尾的气动改进上,推力损失又被捞回了大半。但是整体来説,阻力还是略有增加的。\subsection*{2017-05-26 00:00}
J-20 itself is pretty big, with significant range and payload. It would not be cost effective to develop another supersonic stealth bomber.

Sub-sonic stealth bombers use the flying-wing design, which is capable of evading even long wavelength advance warning radar in the UHF and VHF bands. This is a major advantage over other stealth designs.

Non-stealth supersonic bombers have no penetrating power with respect to modern air defense systems. Their benefit over existing H-6 and JH-7 is quite limited.\subsection*{2017-04-18 00:00}
我早先以为chenwj所説的臺湾自以为有歼星舰是讽刺,原来他还是客气的。

如果你相信国军以美国40年前的技术,能超越美军最新锐的装备,那么我说什么都没有用。这里只聊一下电浆。

电浆是高温气体的原子的外层电子被剥离了,所以基本上可以看成是有自我电磁作用的气体。电浆束的温度比气割还高,所以也可以用来切割金属。不过受大气阻挡,就如同放屁的初速也许和投球一样,但是屁是不可能和球一样一下飞出几十公尺的,电浆束的长度最多只能达到1公尺左右。

Lucas在拍摄Star Wars之前,为光剑(Light saber)所画的设计概念,就和电浆束一模一样,所以我一直怀疑他早年看见过电浆束。不过后来实际拍片的时候,受当时的特效技术限制,光剑才变成大家熟悉的"日光灯管"。\subsection*{2017-04-13 00:00}
你说的,STOVL比较适合守势作战是有些道理的,但是在纯粹陆基的前提下,像JAS39这样的短距起降战机性价比更高。STOVL成为首选,是海权国家没有真正的航母,又想要进行远洋攻势作战,像是日本和英国(但是日本现在还不敢明目张胆地去买)。

如果有了真航母,STOVL就是鷄肋。战机负责航母战斗群空中防御任务的最外围,这是在1000公里半径执行的,还必须能留空一段时间,守株待兔。牺牲航程对任务影响大,而且STOVL版的购买和维护价钱反而更高。美军是因为陆战队政治能量大,非要有自己版本的航母+舰载机不可。对国家整体利益来説,是很大的浪费。

以上的讨论是比较理论的。现实里隐身战机对传统战机有代差,而臺湾有可能买到的隐身战机,就只有F35。F35既然没有短距起降版,那么退而求其次,STOVL版是最理想的,反正臺湾军购当冤大头习惯了,再多花点钱也没什么大不了。

中国是新兴的海权国家。航母是海军的核心。STOVL不但价钱高、性能差,而且需要更高端的发动机技术,这正是中方的短板。F35是军工技术上的奇迹,实际战力却比F22先天差了半代,就是被STOVL拖累的。中国的技术实力还远不到可以这样浪费的地步。

总之,在远洋作战,STOVL能做的,CATOBAR都能做得更好,而且更便宜(指战机,航母本身当然是越大越贵)。有了真航母+LHD的组合,就没有再画蛇添足、加上LHA的必要。\subsection*{2017-04-12 00:00}
唉,又是一个异想天开的白日梦。

共军的第一波大规模对地打击,会是东风弹道导弹和各式巡航导弹。两岸如此近,前者的飞行时间3分钟;后者10分钟出头,但是贴地飞行,只有E-2T预警机才能提早发现。假设国军的E-2T一年365天、一天24小时警戒,从臺湾东岸的巡逻区,也顶多只能提供两三分钟的预警。

一般发起这类攻击的时间是凌晨,例如这次美军打击叙利亚,就是在当地时间2:30am。假设国军的预警雷达官能每分每秒盯着荧幕,三分钟还是不够确认消息。好吧,假设雷达官有直通蔡英文的专綫,而且习惯性喜欢半夜和她聊天,不须确认,所以一分钟内就拨了电话,请问接电话的助理要花多少时间才能叫醒蔡英文?好吧,再假设蔡英文每晚就等着接他的电话,但是消息还没有被确认,她又能下什么命令呢?好吧,再假设她兴奋得手指发抖,马上就不小心按钮发射,国军的雄二飞弹部队也是整晚不睡觉,就等着射击?

军事界早已研究过预警时间这个问题,结论是即使像核战争这么严重而且优先等级无限高的情况,在资金、人力无限的背景下,至少也还要20几分钟才能针对突袭反应过来,所以1962年苏联在古巴部署了飞行时间在15分钟以下的中程导弹后,甘乃迪不惜封锁古巴、威胁核战也要把导弹逼走。后来美苏签了协约,完全禁止中程导弹,也是为此。至于短程导弹,当然更加难防;在冷战期间,美苏不可能让对方部署到家门口。现在北约东扩,乌克兰政变,Putin这么紧张生气,又是为何呢?

臺湾离大陆如此近,连火箭弹都可以做战略打击,根本不可能预警。而且共军在发射导弹之前,还有隐身战机可以先行消灭E-2T和其他预警雷达。难道蔡英文和雷达站的电话从来不挂,綫路断了就发飙开始反攻大陆?

总体来看,国军的重要装备,质量普遍比共军落后一代以上,数量差距在一个数量级以上,而且体系不完整,又处于被动;偏偏对方对国军兵力部署瞭如指掌,而且国军纪律涣散,根本不能机动隐蔽,就算有几枚导弹残存于第一波攻击之后,又能对共军有多大的打击作用呢?

武统早已不是军事问题,而只是政治问题了。\subsection*{2017-04-06 00:00}
臺湾的未来这个问题,是我写部落格的初衷,所以虽然有些话题被一再重复,我的容忍度还是比较宽的。

简单来説,我觉得这里有两大一小,三个玩家:美国、中国和臺独,来选择四个选项:(1)文统、(2)武统、(3)独立、和(4)不统不独。

美国的偏爱是:4-3-2-1
中国的偏爱是:1-4-2-3
臺独的偏爱是:3-4-2-1

前面说的都是主观的"偏爱",而不是客观的利害得失。我一再想强调的是,真正的标准,应该是臺湾底层民众的实际利益,那么这个顺序会是1-2-3-4,所以它也是我个人的偏爱。

在世界的两大超强面前,臺独没有什么话语权,唯一能做的是否决文统,偏偏这是关键,把现实从1变成4,而臺湾底层民众也就从天堂掉到地狱。这才是臺独的罪孽所在。\subsection*{2017-04-04 00:00}
共军怎么会连枪管都造不好?枪管很容易,连臺湾都造得像模像样的。共军的枪管品质不如欧美,不是技术问题,而是商业问题。这一方面是中共兵工厂品管不好,承袭50年代和苏联学习的流毒,和渖飞的问题一样。现在开始引进民营企业,管理好了,品质自然会上去。另一方面是穷惯了,不舍得花钱,这也在改进之中。

至于共军的小口径武器设计不如美国,也不是技术问题,而是文化问题。美国有狩猎和玩枪的传统,民间的枪支专家人才辈出,军方官僚想扯后腿都不可能。M16不就是民用步枪改军用?中共的步枪研发人员,自己都不会用枪,那设计出来的东西,自然完全不合实战需要。

还好95式的设计人已经退休了,现在领头的那位女士似乎对国际潮流掌握得好的多。\subsection*{2017-04-03 00:00}
那文的作者,连不入流的军迷都称不上,只是个自説自话的傻蛋。

雄三的射程,就是150公里(高-高-高弹道,高-低-低弹道是75公里),哪有什么400公里射程的改进型?航母战斗群的作战半径是800-1000公里,哪是他随口胡诌的400公里?雄三的速度就是马赫2,这对方阵快炮有威胁,所以共军才先后开发了730和1130点防御系统,后者装备在辽寧号上,单位时间的弹丸投射质量超过方阵10倍以上,足以拦阻少数马赫3的导弹。而且还有HQ10。

其实不用讨论这些专业细节,也可以知道他是在胡扯。共军不像美军那样习惯有绝对空优,所以舰队防空防导是首要的投资对象。我已经多次解释过055比Burke 3先进,辽寧号的点防御系统也比美国航母完善。如果雄三是共军航母的克星,那么美军航母岂非不堪一击?这点你已经想到了。航母战斗群的防空防导网从外到内,先是舰载机、然后是护航舰艇的区域防空弹,能突破这两层的,才由近防飞弹(如美军的SeaRAM和共军的HQ10)拦截,等到必须用近防炮打,已经是十中无一。所以共军才会开发反舰弹道导弹,这是舰载机和点防御系统都无法应对的威胁。

400公里射程(高-高-高弹道)、马赫2的反舰导弹是有的,就是印度最新从俄国引进的Brahmos改进型。它比雄三重了2倍多,研发时间晚了50年,你想后者有可能达到同样的射程吗?印度人也是把它吹上了天,但是俄国反而自己不装备,好像他们不须打击敌舰似的。。。其实是Brahmos太大太重,射程却太短,马赫数又太低,所以还远不如其他现成的导弹,尤其是Klub(中共的类似產品叫做YJ18,这是目前共军的标准反舰导弹)。SU30同样出动一次,Brahmos只能带1-2枚,Klub却能带4-6发。射程相近,却用的是高-低-低弹道,更难拦截,终端速度反而比Brahmos还高。你觉得俄国人的选择对吗?\section*{【空軍】【海軍】雷達與隱身技術之間的矛盾關係(上)}
\subsection*{2017-08-10 00:00}
舰载预警机探测飞翼轰炸机有困难,但是探测LRASM这种尺寸(宽度刚好近似VHF/UHF的波长)的目标,不论它外形怎么修,还是可以看得到的。然后负责巡逻的歼击机赶过去,若是能找到轰炸机最好,否则就用空空导弹打下LRASM。这并不难,因为LRASM是亚音速的。

如果没有巡逻的战机在附近,那么就只好通报防空舰,让他们准备用上近防系统了。这些系统都是全自动的,但是有心理准备还是好些。

LRASM的射程极长,远在预警机探测距离之外。美国人寧可选择飞得慢的,也要打得远,就是为了避免轰炸机被拦截。别忘了,这些美军的隐身轰炸机比共军的防空驱逐舰还贵。

这并不代表只靠隐身轰炸机就可以完虐航母舰队,因为他们还是需要实时的侦察定位,才能在600多公里(射程是机密,这是我的估计)外发射LRASM。所以双方都会想办法先打掉对方的侦察卫星、无人机、预警机等等,最终还是整个体系的对抗。\subsection*{2017-08-09 00:00}
无法直接引导。VHF警戒雷达的波长太大、波束太宽,对位置和高度的测量,都有很大的不确定性。

防空驱逐舰(如052D)的主阵列用的是S波段,反隐身能力相对弱得多。不过它的波束窄、功率高、增益大,而且一旦警戒雷达指出可疑的目标,可以集中功率搜索那个方向,所以仍然可以在相当的距离外探测到隐身目标。

在现阶段的武器竞赛,200公里是一个理想的拦截半径。052D的346A雷达应该有能力在这个距离探测到并且攻击F-35。F-22的雷达截面又小了一个数量级,所以探测距离缩短为0.1的四次方根~=0.6倍,亦即120公里左右。如果把AESA的单元数加倍,亦即把雷达直径从4.5米增加到6米,那么就足以抵消F-22的隐身优势,把探测距离又推到200公里以上。

对B-2和B-21这样的大型飞翼,VHF和S波段都无能为力。必须依赖陆基的HF阵列天波雷达做预警,但是这种雷达误差更大,而且杂讯很多。\section*{【美国】【战略】Trump的施政方针}
\subsection*{2017-08-08 00:00}
问题在于李登辉为了搞日系臺独,必须先消灭臺湾社会的理性和自我修正的能力,所以推动了"民主化"(让愚民看门做主,屏蔽一切理性思维)和"多元化"(让真话淹没在假话和蠢话的洪水里)。20多年下来,臺湾愚民乐此不疲,整个社会和国家的消沉,已经无可挽回了。

一个国家社会,像是一架双引擎的飞机。它的引擎,一个叫务实(事实),另一个叫理性(逻辑)。李登辉把两者都关掉了,然后蓝绿互闘,直到引擎完全脱落。现在换上再好的驾驶也没有用,何况蔡英文还在到处放火。我个人则像是一个引擎工程师,用无綫电和这架飞机的乘客联络;但是引擎都没了,要修也无从着手。

我能对大陆有一点贡献,当然很好。但是臺湾无可救赎,终究是遗憾。\subsection*{2017-06-04 00:00}
选举和选择不是同一件事。过去40年的选举,哪有什么真正的选择?Carter是最后一个不在财阀口袋里的总统候选人。

现在财阀的势力,正在用尽一切手段,要阻止Bernie Sanders的人掌控民主党。我觉得Elizabeth Warren是他们深藏的一张王牌(Ace In The Hole),换句话説,她沽名钓誉,伪装为清廉,实际上却不会也不敢做真正的改革。

我是一个脚踏实地的人(Realist),不喜欢自我陶醉,所以我预期下一任总统会是Warren。她在行事风格上,会和Trump南辕北辙,欧洲和白左会因此而兴奋不已,就像8年前对Obama那样,説不定诺贝尔和平奖也一样会马上颁给她,但是实际上仍然不会动到民主党系财阀的蛋糕,她的财政部长也一样还会是高盛的人,对世界人民的剥削会从显性改成隐性,但是反而更为高效。\subsection*{2017-06-03 00:00}
这些假白左,只是自卑感太重,所以跟着真白左犯蠢。

现代的美国白左,并不是真正的社会主义者,他们并不关心扶贫,只是吃饱喝足之后,互相比赛如何作秀来彰显自己的道德优越感。他们对道德的定义,也是完全扭曲的。

当然,白左的腐化并不是偶然,而是过去40年来掌控媒体的财团有心鼓励造成的。越是愚蠢的示威,像是你所提的这个,媒体的报导越积极(左翼媒体甚至是正面报导,难怪右翼媒体有很多材料可以鼓动选民支持财阀);真正切中要害的,例如《占领华尔街》,反而被主流媒体不论左右一律忽视或抹黑,连警方违法(美国警察打人固然是司空见惯,在处理《占领华尔街》的过程中,破坏摄影机和没收一切器材的做法却是非常少见的)驱离,都视而不见。

美国的所谓左右翼媒体,和左右翼政党一样,其实都由财阀控制,只不过服务对象是不同行业而已。至于金融业,则是左右通吃。左右两边的论点,在过去40年都越来越离谱,同样都是让己方观众自鸣得意,但是对方却有很明显的批评着力点。如此一来,两边的收视率都很高,群众的热情也持久不退,但是议题却越来越蠢,对国计民生越来越无所助益。结果是媒体财阀不但自己赚眼球,还帮助相关行业避免政府的有效规范,只不过最后还是老百姓买单罢了。\subsection*{2017-05-20 00:00}
Trump即使真把IS的资料给了俄国,对美国也没有任何损害。

我对Trump的态度也是很复杂的:一方面他所代表的特权、愚昧和信口开河,与我的人生哲学格格不入;另一方面,他当选总统却是人类整体的福音,对中国尤其是特大的机遇。这是因为美国已经成为万恶的根源,是财阀压榨世界人民的中介,而美国的霸权,有三分之一是建筑在虚伪的宣传上。Trump的诸般作为使得这些宣传的虚伪性质暴露无遗,对人类实在是极大的贡献。去年我原本理智上决定要投票给他,到最后临出门,感情上又觉得实在太恶心,就没有去投票。

Trump在政治上,是五嵗小孩的水准,完全没有眼光和自制力。他最大的失策,是选择了一个精明的传统政客Pence为副总统,所以各方势力在搞他下臺的努力上,都可以百无禁忌。如果他选了Bannon当副总统,现在基本没人敢这样无理取闹。\subsection*{2017-05-20 00:00}
又是一个很好的观察。

歷史上,西方霸权的崛起,是建筑在科学,而不是民主上的。我们现在看到的民主的极端形态,是二战后英美以胜利者的姿态,为了抹黑国家社会主义,并且贬低苏联共產系统,而开始自欺欺人的,一直到1960年代末才真正建立完成,可是几年后,美国财阀就开始他们的歷史性反扑,执政水准也很快跟着江河日下。

最可悲的是到现在,西方连科学也被民主污染了,学者不再求真,而专事哗众取宠。

英国的代议制,其实还是设计来尽可能隔绝愚民的直接掌权。这次脱欧公投,让精英们灰头土脸,虽然放不下脸来扭转决议,但是私底下绝对是学了一次乖。SNP的政客没有头脑,还要反其道而行,未来几年必然会遭遇各方面的围剿。\subsection*{2017-04-02 00:00}
像这种消息,是真实的可能性极低,因为没有任何证据,谁都可以伪造。

就像是最近俄国反对派散布Medvedev贪污了多少亿的消息。它是基于实际证据的机率极小(至少没有任何真实的证据被列出来)。当然这不代表Medvedev没有贪污,但是一个这样的谣言它的实际信息内容是零,唯一的作用是把话题专注到某一方面。这类谣言之所以被常用,是因为一般愚民注意力很短暂,又没有逻辑自洽的世界观,只能对眼前的话题做出动物性的本能反应:"噢,Medvedev贪污了,我们当然要示威反对。"换句话説,这些谣言只是骗取愚民注意力的手段。

理性的考虑则是:既然信息内容是零,那就无须反应。就算真有证据,真要应对这个话题做思考,那么实际关键的问题核心,不是Medvedev有没有贪污,也不是他贪污了多少,而是挑起这个话题的反对党若是当了总理,贪污就会少了吗?我想答案大概是否定的,所以智者仍然不会受谣言左右。

回到原本的话题。这个"七不讲"是真是假并不是重点,关键的问题核心是由政府来监管舆论比起由利益团体掌控,哪个比较好些?至少我个人觉得当前臺湾的乱象不是任何正常人应该向往的,而大陆网络上的假消息也已过于汎滥,政府的监管应该加强而不是减弱。\section*{【美国】美国宣传战的新困境}
\subsection*{2017-07-30 00:00}
我同意,中共的宣传部门亟待整顿。

CIA的确从设立之初,就极为注重对外国歷史、文化、语言、政治、宗教和社会的研究,包括自1950年代起,一系列对这些学科的Federal Grants(联邦研究资助)。事实上,这些资助占了美国这些学系研究资金来源的绝大部分。两三年前因为财政危机,国会试图削减它,行内就有文章预言,研究生数目会紧缩一半以上。

不知中共是否已经开始这方面的投资。世界霸主必须对任何一个小国都瞭如指掌,那么即使是再小再弱的国家,也必须有几十名社会学教授专注于对它的研究。几年前,世界上第一部中文/Pashtun对照词典,居然还要靠Afghan学者自费编印,实在再一次説明中共外交部的怠政问题有多严重。\subsection*{2015-12-15 00:00}
第一次海湾战争时,我还是学生,在哈佛校内校外听到看到的评论都说伊拉克部队是久战精锐之师,必然比越南还要难打。当时我就和同学解释,越南是丛林,还可以躲躲美国的优势空军;伊拉克是沙漠,什么都看得到,所以绝对是一面倒的战况。他们当然还是寧可相信大众传媒上"专家"的言论。

其实这还不是我第一次注意到美国宣传完全脱离现实的例子。1989年六四之后,中国流亡学生和美国评论员个个都说中共政权绝对撑不过两年。我心里就想,评论事件的对错好坏还算是主观的意见,预言实际后果却是客观的推论;这些人年纪都比我大,似乎又有些来头,怎么还像小孩子一样专讲一些一厢情愿的话?

后来熟悉了美国歷史,才明白美国立国以来的公共讨论,向来就是幼稚而非理性的。70年代以后大众传媒被财阀控制,自然完全失衡,但是即使在30、40年代的黄金时代,罗斯福一样是自己做一套、对老百姓说另外一套,基本上也是哄小孩的把戏,衹不过罗斯福一心为国造福,而不是为己谋私。\subsection*{2015-06-08 00:00}
你是学人文社会出身的吗?我觉得你对社会文化方面的看法很深刻,逻辑思路却又很清楚,是不常见的组合。我很高兴又有一位这样的读者来这里发言。

我也觉得除了政治和商业需要之外,西方媒体对中共的敌视有很大的种族成分;这其实是我当初会开始对西方有反感的原因之一。日本人自以为不是亚洲人,反而在这方面比美国人还糟糕。例如几年前我觉得时代周刊驻北京的主任写的稿偏见极大,研究了一下以后发现她是美日混血的,在东亚长大,难怪一些中性的文化特质都被写成是噁心的东西。

我在最近几年注意到中共能接近决策阶层的文化精英,其实是世界上最理性最踏实的政治和经济思想家。或许和中国文化没有受西方宗教的污染有关吧。\section*{【美国】【金融】美国式的恐龙法官(三)}
\subsection*{2017-07-27 00:00}
The rule of this blog is that the dissenter must provide his own facts and logic in support of his argument. Outside links do not count, for I have neither time nor interests in reading them all.

You made a clearly false claim without explanation. This is a serious violation. Consider yourself warned. Bringing your facts to the discussion is YOUR responsibility, not mine. Lazy stupid people who do not want to explain their reasoning cannot expect me to do their job for them.

I am no lawyer, but probably have seen more insider trading on Wall Street than any other Chinese alive. If you are going to argue the legal points, that is exactly what this article is about. Namely, something hugely damaging to society/economy and nominally illegal has been whitewashed on technicalities and semantics.

For someone who claims to be familiar with legal matters, you are surprisingly resistant to abiding by the rules. This is my blog, and I run it by a set of clear and reasonable rules, one of which is that only facts and arguments presented here will be considered. Links tend to be a waste of everyone's time, and can only be referenced as proof instead of being used as substitutes for real arguments. Why do you think you can dictate how I spend my time when you yourself are too lazy to summarize the "truth" here? Unfortunate for you, I, unlike the practitioners of the American justice system, am not affected by twisted legal posturing at all. Since you are clearly incapable of logic and disrespectful to order, all your subsequent posts have been and will be automatically removed.\section*{【基礎科研】張首晟和凝態物理界的牛屎文化}
\subsection*{2017-07-25 00:00}
我对他不熟,只听説他曾经打假。不过他这样的言论,网络上很常见。基本上是没有能力从基本事实开始做理性分析,而直接从既有成见或"常识"做反射性动作的喷子。所以他或许可以正确批评低级的骗子,但是遇到专业人员的自我炒作,他反而成为护卫他们的急先锋。

我在这个部落格一再强调的是,有不同的意见,只要把事实基础讲清楚,大家就可以用普世皆准的逻辑一起做分析,依理性共同达到同一个正确的结论。如果不这么做,而是直接做人身攻击,那么基本可以确定他没有事实基础,只好转换话题。另一种话题转换,则是在我文章细节或语法里面找毛病,找不到就曲解一个,然后无限上纲。这两种狡辩术,我想这个部落格的老读者们都看得多了。

如果你要和他讨论,可以请他把他论述的事实逻辑基础讲出来,这可以是正面的新事实,也可以是纠正我所列举、依据的事实(例如超弦有10\^500个自由度,所以不可能被证伪)。否则只是泼妇駡街,毫无实际意义。\section*{【海军】共军的新航母}
\subsection*{2017-07-25 00:00}
大型航母一次出击的飞机数量,大致与长度成正比,与舰载机的翼展成反比(而翼展又和最大起飞重量平方根成正比)。这是因为现代喷气式飞机,暖机时的喷射气流十分危险,所以必须把飞机排列在甲板边缘,尾部向外。但是吨位与长度三次方成正比,对发动机功率的要求也随之迅速增加,所以吨位的增加是有上限的。

以目前的技术和舰载机的特性,10万吨是Sweet Spot。如果要把这个Sweet Spot上移到12-15万吨,最可能的决定因素是舰载机的尺寸和重量,例如如果下一代舰载机最大起飞重量达到45吨,那么我们就会看到有12万吨以上的航母出现。而舰载机的最大起飞重量,同样是性能要求(航程、载弹)与技术限制(结构强度、价钱)的折衷平衡。\subsection*{2017-07-13 00:00}
这人说X波段只能用来火控,是因为他读了有关美军AMDR的文章,其中X波段就只是用在火控上,所以脑补以为全世界都如此。其实X波段专用在火控上,是美军独有的;原因只是必须与大批库存的标准2型的半主动引导头兼容。既然每艘防空舰都有X波段火控雷达,新的标准6型又反过来必须兼有半主动模式。基本上就是老技术的沉没成本太高,无法一次性地乾净淘汰。

欧系的X波段主动阵是你説的搜索兼火控。AMDR-X要用在搜索上,当然没有什么不可逾越的难处,只是美军的雷达体系里,这个功能已经被S波段雷达占据了,重新整合花大钱却没有大效益。

055的系统则是世界领先的综合射频:每个频道的天綫都有,但是每个频道的每个方向都只有一个天綫。利用AESA可以同时发射接收多个波束的特性,不论背后所需的功能是什么(例如通讯、电磁反制、搜索、火控、反导),只要用的频道一样,就都通过同一个主动阵列来发射并接收。欧美俄都还没有这样的技术。\subsection*{2017-05-03 00:00}
The science behind Quantum Computing is well understood, but it is very early in its engineering development cycle. My guess is that no practical application will be realized for 20 more years.

Even when it works in the future, its use will be fairly limited in daily life. The main impact will be in cryptology (thus military) and some isolated research areas such as protein folding.\subsection*{2017-05-01 00:00}
我觉得他们应该改,但是看来不会。

其实美军的战斗群司令也是在航母上。不过他们把战斗群的任务明确分为三种:打击、防空和反潜。最复杂的第二类任务,指挥官在巡洋舰上。战斗群司令不需要自己的庞大空情指挥中心,也不需要全尺寸的对空相控阵雷达。

J-15绝对算是多功能飞机(定义:能发射有制导的对地对海武器),空优只是偏重。016和017号都各只有一个团的J-15,光在数量上就不足以打一场大规模空战。此外,J-15不是隐身机,必须依靠J-20长途奔袭来狙击像是预警机和加油机这些目标。远程的搜索、定位,也最好是有陆基的战略侦查无人机帮忙。至于东风26和东风21D,则是预先削弱美国航母战斗群的防綫的重要手段。而预警机、加油机、指挥机、电战机等等,更是滑跃甲板航母的软肋,有了陆基支援就没问题了。\section*{【台湾】四场演讲}
\subsection*{2017-07-21 00:00}
我说一国两制行不通,是专指"马照跑、舞照跳"的港式地方自治模式,并不是说要照搬大陆的政治体制。事实上,臺湾没有经过真正的土改,也没有足够的共產党员,根本不可能一夕之间成为另一个一般的行省。

香港基本是一个用来实验西式政治制度的试管。固然这个实验已经给出负面的结果,但是还有经济社会层面的实验,是未来的中国须要考虑的。例如现阶段的富商基本在垄断财富上没有上限,但是最终我们必须追求至少是北欧式的均富。即使看的近一点、细一点,也有很多税务上的改革,尤其是房地產税,以及退休制度,必须探索新的形式。在这些方面,臺湾都比大陆还要"成熟",也就特别适合做实验。

要做这些实验,就只能军事管理,才能完全忽略既得利益者的抱怨。\subsection*{2017-07-18 00:00}
我并没有直接唤醒多数臺湾民众的幻想。我的努力一直是为了提供少数清醒者正确的事实和逻辑,以作为他们论战的材料。

30多年前,臺湾远远领先大陆,那么很明显的理性期望是由臺湾统一大陆。后来事实发展改变了,我们的期望也应该随之改变。Keynes的名言是"When my information changes, I alter my conclusions. What do you do, sir?"我完全同意。

林毅夫当年投共,是一件我百思不解的事,因为当时的事实并不能支持他的决定。后来他从芝加哥大学毕业,却能在短短几年明白师门的谬误,才是我欣赏的转变。将来若是有机会,我会想要问问他在这些决定背后的思路。

我对私有市场经济并不排斥,只不过不相信它是能解决一切经济问题的万灵丹。有些行业,还是公营或公私兼具为宜。\subsection*{2015-06-27 00:00}
I have noticed that Taiwanese power elites have no real understanding of the damage TPP can do. This is something that I will need to broadcast harder.

Chinese muscle flexing is a necessary to American encroachment. For ten years, the US was distracted by the mess in the Middle East. Obama, however, thinks it is time to make trouble for China. If China does not push back, Japan, Filipinos and Vietnam will be far more emboldened on the territorial disputes.

As I explained before, there is little consequence for bad behaviors from the US' point of view. Unfortunately, they also control the media, so even reputational problems can be covered up. We who speak the truth are the antidotes.\section*{【宣布】刚做手术,写作暂缓}
\subsection*{2017-07-17 00:00}
自走炮重35吨,高于新轻坦,发动机功率却只有其一半,在4000米高度上,功率要再打个五折左右,连爬个小坡都会很吃力,别説越过喜马拉雅山了。所以山地部队有他们的专用火炮,亦即M777这样的4吨级超轻型火炮,可以由直升机空运。印度刚买了一大批M777来装备新组建的山地军。中共虽然研发出同级的火炮,但是还没有确定列装,而且共军的直升机受涡轴发动机技术能力限制,高原运载力不足。

中印边界,地势崎岖,高低起伏很大,其实不适合迫击炮这样的大弯度曲射武器。这是因为误差十几公尺,炮弹就掉到山沟里去了。步兵直射武器比较合适。除了你説的榴弹枪之外,还有火箭筒和飞弹。印度刚从以色列买了8000枚长钉飞弹,是有针对性的。

共军在西藏的常驻部队很少,在全面战争下,只勉强足够阻滞印军前进,绝对没有越过喜马拉雅山进行攻势作战的能力。如果要大打,共军必须如同1962年那样,事先调派大批部队进藏。这在2017年,根本不可能完全保密。

当然,以印度人之蠢,这可以反过来看成是一个诱敌深入的圈套,亦即让他们以为有机可乘,试图在共军增援抵达之前攻入拉萨,其实是准备用空军和炮兵对其大幅杀伤,等印军的精锐部队和后勤能力都消耗殆尽之后,增援的15军直接空降到他们后方,赶尽杀绝。\subsection*{2017-07-05 00:00}
India lied, as usual.

In order to maintain the fictions, Indian government has to this day banned the Handerson Brooks Report, which is India's own official review of the 1962 war. I don't understand why China does not simply put this report in the open, and refer to it every time India makes a provocation.\section*{【医疗】当前世界的公共卫生危机}
\subsection*{2017-07-17 00:00}
不是Double Blind的大规模实验,没有统计意义。

你的经验,纯粹是人类演化过程的结果,亦即脑自然会从随机事件里,无中生有,幻想出规律来。这对原始人种来説,一般只是浪费心思,但是原始人的心思没有什么价值;一千次里有一次,风吹草动,真的是剑齿虎,那么就会有演化压力了。

宗教的起源来自三个非理性的心理趋势,都是演化的自然结果,这是其中之一。我把中医和宗教相提并论,并非无的放矢,而是有理论根据的。

真正可以做验证的标杆,是像疟疾这样歷史悠久的传染病。一般人以为青藁素算是中医的胜利,仔细想想,就会得到相反的结论。这个中医成就的样板,其实完全不是歷史上中医的主流治法,而是从一本冷僻医书里的几百个偏方筛选出来的唯一一个有效药品,所以中医处方的正确率顶多只有(1/几百)~0.3 \% ,更可能小于0.1 \% 。我说"中医有99 \% 是谬误"是客气的,最精确的估计是99.9 \% 。

近年来,美国生医界也因为研究人员必须出版大量论文的压力而使水准每下愈况。据最坏的估计,有2/3的论文结果是"无可复制的"。但是即使是只有30 \% 的研究结果可靠,仍然比传统中医的0.3 \% 可靠率高出两个数量级。两个数量级在天文学里不算什么,但是对人命就很重要了。\subsection*{2015-09-04 00:00}
The current problem with superbugs has dual causes; abuse of existing antibiotics is just one of them. The other is the lack of investment in finding new ones.

Yes, I agree that the pharmaceutical clauses in TPP will be very harmful.

I don't think it is productive to distinguish western vs Chinese medicines. The correct distinction is whether they have been scientifically vetted.

 The correct lesson from the recent Ebola epidemic is that the rich countries need to contribute more to public health issues for the poor. WHO is a badly run, poorly funded mess. Sorting it out would be a very good first step.\section*{【基础科研】谈量子力学(一)}
\subsection*{2017-06-17 00:00}
我很高兴他终于不再继续在超弦这条邪路上走下去,而开始思考一些真正有意义的物理问题。他所提的话题,正是我在这篇正文里所讨论的。

在高能物理过去40年所面临的无可逾越大沙漠阻拦之下,一个忙着出论文的学者,不论天赋多强、功力多深,必然会被各种细节所迷惑,因而有见树不见林的困境。我自己也是离开物理界之后,才有了足够的高度和距离,能够真正看清大势(Big Picture)。

Weinberg在得Nobel奖之后,就离开哈佛,到德州大学自立门户,因此不再有和Coleman交流的机会,错失了80年代末期,Harvard物理系对量子力学背后哲学问题的一些深刻探讨。然后为了建立并加强自己团队在高能物理界的地位,拼命跟着超弦诈骗集团一起出论文。现在他可能是半退休了,才有余裕回归正道。\section*{【海军】常潜式海洋攻搜作战平臺}
\subsection*{2017-06-11 00:00}
我觉得这个武库舰的谣言,来自中国网友脑补,将几个歷史悠久、有关武库舰的揣测强加到最新的消息上。美国的民间军评水准很低,不但照单全收,而且又加上自己的胡猜。转了一圈回来,已经到了面目全非的地步。

网友们无中生有的本事和习惯,实在太过厉害,我刚在前一篇文章讨论过。为了过滤这些杂讯,我们只能死守已确认的真实材料,亦即董文才的论文和相关报导。

对无还手之力的地面目标做打击,是美军航母战斗群冷战结束后的主要任务。因此他们在过去20几年的建军,主要是针对火力密度和效率进行优化(例如开发DDG1000,取消不能发射巡航导弹的反潜护卫舰,退役专职空优的F14,大量采购便宜、高效但空战性能平平的F18等等),武库舰也是因此才被考虑做为对航母战斗群的便宜替代。

但是中国的大政略,完全不同,没有每年要对几个小国打几百发巡航导弹的必要。共军的战略重心,虽然正从近海推进到远洋,仍然是防御性的。防空和反潜,才是中方舰队的主要任务,而这些任务不是简化了电子系统的武库舰能胜任的。中共海军若有打击任务,也会是针对美军航母战斗群的水面舰只。这些目标本身防御力量极为强大,而空射导弹的总体打击距离(即包括载机的航程)超过舰射飞弹3-5倍,所以发展并改进航母+加油机才是正道。武库舰画蛇添足,被共军采纳的机率很小。\subsection*{2017-05-31 00:00}
It is not a big deal. Just a little sad to realize another youthful dream never to be fulfilled.

BTW, I want to urge all fans of 《Game of Throne》 to check out this theory: https://www.reddit.com/r/asoiaf/comments/6dx676/ 
which I consider very clever and likely true. I already forwarded the link to 《观察者》, but there is no guarantee that they will bother translating it.\subsection*{2017-05-31 00:00}
When my son goes to college in two years, I may find a new gig, probably teaching, but I still won't be able to lead and build a historical project again. 

I built the world's first fully automatic program trading platform 17 years ago, but that had no net contribution to mankind. In fact, it might have had a pretty negative contribution.

My son's girl friend's grandfather invented the floppy disc back at IBM. That was the kind of accomplishment every geek would aspire to achieve in his lifetime.\subsection*{2017-05-30 00:00}
原本想要写的,后来觉得资讯还是不够,再等半年吧,TRRE年底要试射了。

晨枫基本上是先评论了XS-1火箭,然后节录翻译了Aviation Week的文章。大体上还好,我就只挑两个毛病吧。首先超燃发动机的难处,不是他说的激波。其次,中方的这些超燃飞行器,虽然和美国是同级的,而且可能已略有超前,但是从大局来看,仍然只是早期的技术验证机,Aviation Week有点危言耸听,晨枫又加了点油添了点醋,让读者以为很快就会有实用成品。我的理解不是那样的。

不过厦门那个会议透露的消息,确实比起我以前的认知,裤衩更红些。基本上所有尖端的技术,从各类超燃发动机,到爆轰发动机,中国都有团队在研究,而且已经有了初步的结果。

我以前说超燃轰炸机不实用,我现在还是这么觉得,但是用来发射卫星,或作为侦察无人机,却是有可能的。详情如何,我觉得Aviation Week的两篇文章还是不够的,所以再等等吧。\section*{【大陆】真理只在大炮的射程之内}
\subsection*{2017-05-29 00:00}
人类在演化过程中,获得了惩罚规矩破坏者的本能,不过这并不是王道。

人对人的惩罚,必须受国内法和国际法的规范,否则就只是报私仇。Hitler当年也是因为英法真正的压迫而反弹的,但是这个世界不接受报仇就可以无限上纲的论点。

实际执行上,你觉得施行惩罚是教训,承受者却觉得是无故(或因细故)而加害,教育根本无从谈起。所以王道的教化,指的是讲理,不包括损伤身体的"教训",你的主张把语义搞错了,结论也就颠倒黑白。

我以前讨论过的血腥征服的震慑作用,不是指望让承受者从理性上知道自己做错了所以要学乖,而是创造非理性的敬畏。这是在国际法容许范围下的一个不得已的赌博,不是超越法律的肆意惩罚。
\section*{【政治】西方对中国制度的短视}
\subsection*{2017-05-27 00:00}
美国人喜欢说,毛泽东发动文革是为了斗倒刘少奇派,我想这太小看他了。当时他对军队仍然是一呼百应,若只是要打倒刘少奇,简单至极,根本用不着搞文革。

我同意你説的,他是为了"理想"而发动文革的。详细来说,支持社会主义的人,包括我在内,都是讨厌不平等的。但是经济发展,天生就会產生并加剧不平等(参见Piketty的研究),除了全面性的天灾人祸,否则很难扭转。

一般的经济学家,只要经济总量还在增加,就不在乎不平等。比较有远见的少数,则认为只有底层民众的生活有改善,才可以忽略不平等。我个人认为,只有底层民众收入增速超过顶层,才算是好制度;我们应该用尽一切和平的手段来向这个方向努力。毛泽东则显然觉得只要还有不平等,顶层就应该被打击,以便迅速达成一切平头,即使造成歷史性的人祸,也在所不惜。

邓小平的政策,是文革后必需的改革;习近平所推行的,则又是开放30多年之后,必须做的修正。世界的大环境不断演变,内部的组织、文化、社会和资本力量也此起彼伏,政策原本就应该因时制宜、持续改进。反腐是当前的正确方向,从金钱转向理想则是未来的努力目标,这些是当代人的责任,我们不该自满,也无须气馁。\section*{【海军】地效反舰飞弹的原理}
\subsection*{2017-05-18 00:00}
你想得细,很好。

天綫罩的空间是固定的,波长越短,天綫单元越小,但是阵列里的单元数就越多,所以重量随波长变化不大。L波段和S波段的阵列都是极重的,X波段稍微轻一点。

UHF的微波发生器和天綫单元的原理和L波段以下的雷达是不一样的,不是单纯的放大,所以可以节省重量。

AN/APY-9的天綫单元数也是机密,但是我估计为低于100个,所以它的解析度是非常差劲的,只适合做预警。为标准6这样的主动导弹做中段制导还勉勉强强,半主动的标准2就绝对不行了。

至于国军的E-2T,根本就没有阵列,落后了两代,60年代的雷达+90年代的电子系统,那性能就不用提了。\subsection*{2017-05-17 00:00}
这个问题又更复杂一点,所以我在正文里没有提。

反潜机用的不是预警机的对空下视雷达,而是对地雷达的一种,一般是X波段,软件则包含一种对静止目标的特别分析技术,叫做合成孔径,利用数学分析,可以从雷达本身的运动,对静止目标做不同角度的扫描,能够达到信号的大幅增益。

海上的波浪永远在随机起伏,潜艇的通气管是唯一(几乎)静止的目标,所以在合成孔径的分析下,十分醒目;尤其是低海况背景下,甚至会產生所谓的潜望镜波纹,反潜雷达的软件也会特别对其识别。

总之,现代柴油潜艇的通气管和潜望镜都已经尽量做小,在不妨碍功能的前提下,形状也已经为隐身优化,但是一般仍然可以被反潜机在合理的距离外(探测距离视机型和海况而变,确实的数据也是军事机密,我估计是30公里以上)轻松侦测到。\subsection*{2017-05-16 00:00}
Theoretically, yes; practically, probably not. 

Active sonar homing has very limited ranges. If the missile drops the torpedo payload as soon as it pops up above the horizon, the distance is still some 30 km, and yet there is no submarine to provide intermediate guidance. If it waits until the last 10 km, it is already halfway through the close defense zone and flying at max speed. Dropping a torpedo into water will be quite risky and not very rewarding.\section*{【基礎科研】大亞灣實驗的新結果}
\subsection*{2017-05-13 00:00}
我常説美式经济学的基本假设与现实无关,完全只是基于幻想以便为富豪的巧取豪夺来背书。其实这也不是巧合:美式淡水经济学的圣殿,芝加哥大学,当年就是美国歷史上的头号土豪J.D.Rockefeller特意为此而资助设立的,他后来还自夸说那是他最好的投资,没有之一。

我一直觉得中共对任何芝加哥大学经济系出身的人,都应该禁止入境。毕竟光算2008年一年,他们就摧毁了超过10,000,000,000,000美元的人类财富,破坏力之大,仅次于世界大战,远超人类歷史上恐怖分子的所有破坏的总和。这还只算了对财富总量的损害,他们真正的目标是搞贫富不均;我以前说过人类21世纪的最大挑战是贫富不均,就是拜他们之赐。

至于有一个章节在逻辑上是错误的,那指的是美国常用的金融教科书。这算是一个技术上的细节,我不想在此讨论。

这个留言,摆错地方了。\subsection*{2017-05-07 00:00}
2004年他们还信心满满?真是无知者无畏,我在1990年就严重怀疑自己上了贼船了。

高能物理绝对是已经走到尽头了。你看这篇正文里谈的中微子物理,基本上是高能里面还能做一点实验的极少数真科学项目,但也是越来越难,突破的希望越来越小。高能以外的纯科学科目好一点,但是和20世纪相比,新发展也是越来越难、越来越小。例如高温超导,多少聪明人投入,实验和高能比起来也容易多了,但是三十多年,就是弄不出一个完整的理论。

美国早已转向专注于应用性明显的科目,我觉得中国也必须参考这个做法。以往中国在科技上落后,正确的路别人都走过,只要两腿拼命赶就行了。现在开始赶上世界先进水准,没人能指出正确方向,那么其实前面的岔路,十条倒有九条是死巷子,若不用脑袋事先深思熟虑,再能跑的只怕也会累死。\section*{【美國】【歷史】美國的東亜戰略史}
\subsection*{2017-05-12 00:00}
当时亚洲的开发程度远落后于欧美,除了日本和东北之外,没有什么工业可言。在雅尔达会议中,这两地被(隐性)瓜分后,其他地域没有被明确指定是谁的势力范围。

罗斯福拉拢蒋介石,只是延续美国扶持东亚最弱势力的传统;后来马歇尔的调停也只是简单遵循"分而治之"的原则,没有太深刻的考虑。

罗斯福当然不能预见未来的细节,尤其是势力划分不明确的亚洲,基本上留给后世处理。但是他的大政略计划可以说是只要欧美仍然是支配世界的顶级力量,就能永远保证美国的霸权。后来远超其预期的,并不是苏联的发展,而是中国的发展。1950年,共军证明自己可与先进军队比肩,这才使其后20年,美国势力在东亚承受到意外的压力。但是这仍只是区域性的问题,美国的全球霸权真正受到威胁,是近10年来中国国力全面发展的结果。

华盛顿海军条约是英国在挑拨美日矛盾。后来日本不是拼命作弊,很多军舰都比公开数据大一号,也因此自以为可以与美国太平洋舰队一战?\section*{【战略】金灿荣教授的最新分析}
\subsection*{2016-09-14 00:00}
有关南海的资源,那当然是考虑的因素之一,但是我一直觉得它的重要性被夸大了,因为这种深海油藏,开发起来非常地困难昂贵,而且储量的估计也很不靠谱。

至于南海海空基地对贸易綫的影响,我想从美、日异乎寻常的强烈反应就可以看出端倪。

臺湾统一之后,花莲当然会比三亚是更好的核潜艇基地,但是统一的时程还遥不可期,未来十几年,南海还是中国战略核潜艇的唯一安全巡航海域。目前衹能打到夏威夷和关岛,不是很理想,所以下一代的096级会搭载新的JL-3导弹,射程应该会在11000-13000公里之间,至少足以打击美国西海岸。

中美之间的博弈,固然双方都希望不要真的打起来,但是军事力量是外交折衝的安全网、经济谈判的底綫,任何战略和战术上的优势,还是极为重要的。\section*{【基礎科研】回答王貽芳所長}
\subsection*{2016-09-05 00:00}
我的专长是高能理论和金融、经济,所以就从这些方面做原则性的逻辑推演,王所长所编的细节完全无关宏旨。如果你认为王所长的诚信可以压得过歷史和经济逻辑,那是你自己的决定。至少在他最近这篇文章里,就明确是谎话连篇,我的正文已经为你指出来了,你要不要接受事实和逻辑,也是你自己的决定。

我从20年前就说超对称不靠谱,去年LHC统计鼓包一出来我就説机率极小,请问高能所王所长当时说了什么?学物理还迷信权威,而不是依实际结果和逻辑来做论断,那你们本身也是一群自欺欺人的傻蛋。

日本有可能不做ILC,但是几百亿或上千亿的民脂民膏是可以浪费在赌人家放弃上吗?就算日本真放弃了,又怎么样?Higgs的细节一样无关国计民生,衹不过给了几十篇论文而已,你们凭什么叫人民买单?\section*{【基础科研】从暗物质谈起}
\subsection*{2016-07-24 00:00}
我想你指的是这篇文章《Manifold Destiny》http://www.newyorker.com/magazine/2006/08/28/manifold-destiny
后来New Yorker拒绝道歉,丘成桐最终也不敢提告。

为了英文不好的读者的方便,我在此简介一下那篇文章的内容。俄籍天才数学家Perelman解了Poincare问题之后(我在前文《孪生质数假设》曾经简单提过),丘支持两个华裔学生发了一篇新论文,论文的名字和摘要都宣称这才是完整的解。丘接着在北京给了一个演讲,说中国应该为中国人给出完整的解而骄傲。闹了半年多之后,连丘自己也承认以下几点:
1. Perelman的解原本就是完整的;
2.丘的学生的解,被证明有抄袭他人之处。
3.他们的论文名称和摘要都必须从"给出完整的解"改为"给出完整解的一个论述"。
但是丘仍然嘴硬,说他从未说Perelman的解是错的或不完整的。

Perelman后来拒绝与美国数学界打交道,New Yorker的作者说就是受不了丘这种人。Perelman只给过这么一次采访,所以是否真的如此,没有第三方可以印证。\section*{【台湾】【海军】台湾没有航母杀手。。。也没有需要}
\subsection*{2016-07-17 00:00}
这个动图很模糊,我看不出是尾部爆炸。半穿甲弹头的装药,是专门集中爆炸力向侧面展开,所以后面的弹体含发动机可能穿过火球而留下那条烟火痕迹。

图的比例很小,假设靶船是053系列的(亦即2000吨的小船),那么全长还是超过110公尺,火球的直径已经接近这个数值了,应该不衹是残余燃料的爆炸。

动图下面的解説是错误的,因为它讲的是YJ-18。YJ-12是与雄风3同类的全程衝压发动的超音速飞弹,不是火箭引擎。我不确定它用的是雄风一样的液体燃料,还是类似法国流星的固体燃料。如果是液体燃料,就有可能慢烧成火球,但是不会大到如图中所示,因为它基本上就是航空煤油,比汽油稳定得多。

此外,这次演习是为外交而表演的,应该用上真正的实弹才壮观。\subsection*{2016-07-01 00:00}
反舰导弹装的都是半穿甲弹头,但是那和引信的敏感度没有关系,主要是指它有穿甲弹芯,而且会稍微延迟引爆时间,足以让弹头完全进入目标体内再爆炸。现代舰艇用的钢材强度越来越高,所以舰体也就越来越薄,如果引信迟钝到穿过渔船或LST的舰体都没有引爆,那么绝对很有可能穿过驱逐舰的舰体也不引爆,这是个真正的问题。

我觉得那个报导不可信。首先雄风这种战术级的巡航飞弹,卫星是无法实时侦测的。刚发射之后,受地球曲度影响,大陆衹有地波雷达和预警机才能看得到。地波雷达的解析度很差,这枚飞弹又是从港内发射,附近的杂波很多,不可能在一秒之内就分析完成。至于预警机,即使共军刚好在一级战备状态下,其他的雷达站可能也要一两分钟才能获得资讯。

不过雄风三型必须用高弹道才能飞出75公里,如果这次的确用了高弹道,那么在到达巡航高度后,大陆沿岸的一般警戒雷达就可以观察到了。所以一秒钟内反应是不可能的,一分钟内反应要看运气。\section*{【经济】【美国】大停滞的真原因}
\subsection*{2016-07-12 00:00}
It is true that only with wealth accumulation can a society deal with many problems and bring about better lives to its members. Industrialization being the key to wealth accumulation, we cannot possibly turn our backs to it.

That having been said, I don't think there is anything hypocritical about criticizing the rich and powerful, as long as the criticism is based on facts. Wealth is derived from the endeavor of the whole society, both active and passive. Even if some entrepreneurs do exceptional work, a rational observer should still have the ability as well as the right to acknowledge both their contributions and misdeeds, if there are any.\subsection*{2016-07-11 00:00}
说来说去,你就是相信自由化的潮流是天然而无可抗拒的。这是信仰,而不是事实,所有的证据都是相反的。例如到今年,美国已经衰态毕露,还是能推动TPP这种实际上是挖WTO墻脚的反自由化组织;在二三十年前,如果美国要反自由化,那真是弹指即来。别忘了,WTO就是当年靠美国大力推动才创立的。

全球化、自由化,当然可以找到科技的背景,但是最终还是全球霸主美国的权力核心同意了,才能发生。在当时政略上,美国自以为天下无敌,得意忘形,不再忌惮外国的竞争(冷战结束前,美国对苏联的围堵,就轻松地否决了后者参与"全球化"的过程,苏联集团可也是这个"全球"的一部分);在经济上,正是财阀的利益凌驾了中產阶级的影响力,才诞生了自由主义的理论、宣传和政策。

我的"这么重的社会主义思想"是观察了美国实际现象的结论,尤其是政策被决定前的幕后拉锯过程。你的自由主义信仰,在理论上是权力完全分散的极限结果,但是自由市场本身却有强烈的资本集中倾向。换句话说,自由主义必须假设平等的个体,但是却会自然导致极度的不平等,所以自由主义不衹是不符合事实,在逻辑上也是自我矛盾的。

我在前文《民主政治与自由经济》已经仔细讨论过这些问题。你若有新的证据或逻辑论点,可以提出来。光是喊口号,违反了这个部落格的规矩,也浪费大家的时间。\subsection*{2016-07-10 00:00}
如果Cowen的理论如你所説是正确的,那么停滞的应该是全球的整体成长率,毕竟科学技术是普世共通的。可是真正明显停滞的,不但是区域性的,而且衹是中位统计数字,而不是平均成长率。这当然是分配上的问题。

你所说的低端工作的外移,并没有错,但是你的逻辑假设它是Exogenous(外来给定的),这是错的。实际上你应该问问自己,为什么刚好到70、80年代之后,Outsourcing才成为潮流?正因为美国财团在1970年代,成功地打破了以往国家为先的社会共识,用一切以利润为先的思路取而代之,他们才得以自由地解雇美国工人。这样的策略,在1950、60年代和1970年代早期是不能想象的。以臺湾为例,与美国的工资比,在1950年代远低于1/10,到1980年代则超过了这个水平,那为什么是在1980年代才有真正的大规模的Outsourcing进入臺湾呢?

与其同时,美国公共基础教育的投资和水平也急转直下。如果你也把它当做Exogenous Event,那么它当然也帮忙财阀解释了Outsourcing的逻辑,可是实际上这同样是政策改变的结果,而不是原因。在1970年代之后,"自由启发式"教育取代了旧式的"填鸭式"学习;虽然这不是财阀主动发明的,但是你真的以为如果他们不是已经有Outsourcing这条路走,会容许工人教育水平的大幅下降吗?

总之,美国在1980、90年代,达到全球影响力的巅峰,即使有客观的因素,如Cowen和你所举的一些例子,会导致美国中產阶级的损失,如果美国的民主体制没有被财阀绑架,那么它应该以保卫大多数选民为己任,会采用很多政策工具来扭转局势。例如1985年的Plaza Accord,强迫欧日货币升值超过100 \% ,这样的流氓手段为了保护财阀的利益都可以搞得出来,那么你说中位收入停滞不前的问题,美国衹能束手就擒,岂不是自欺欺人?所以你的逻辑,假设了外来因素是给定的,美国政策不能反应(除了保护财阀之外),这才是颠倒因果。\subsection*{2015-11-25 00:00}
我引用的是北欧的过去,在这个议题上和北欧的未来没有关系。

至于贫富差距的合理性,那是一个不同的议题。适当的贫富差距如你所说是必要的,但是纯资本主义已经不再奖励用于生產的资本,赚大钱的多是非生產性的寻租掠夺,经常有很高的政商勾结成分;而真正的生產性事业天生风险高、报酬慢,在绝对自由主义经济制度下不但不能成长,反而会萎缩,所以不能说自由放任对整体国家社会有益。

至于公平性,过犹不及。大公司的CEO薪水曾经只高出小职员一个数量级,或许两个数量级算是合理的,但是现在的美国已经涨到四个数量级,还在向五个数量级迈进。这些CEO主要是靠集团内部政治斗争而掌权的,凭什么每年拿上亿美元?贫富差距奖励勤奋态度的效应衹有在差距是一两个数量级时有效,超过这个范围,你努力一辈子还不如富家子弟一个礼拜的零用钱,那么贫富差距奖励的就衹能是想法娶嫁入豪门,十九世纪的欧洲正是如此,你读读狄更斯的小説就知道了。\subsection*{2015-05-25 00:00}
"It is difficult to get a man to understand something, when his salary depends on his not understanding it."("如果一个人的薪水要求他不懂什么事,那么他就学不会。")是美国作家Upton Sinclair的名言。例子很多,超弦是一个;氢气能源界是一个;美国的军人有几十万,绝大部分相信他们在侵略伊拉克时,是"Good Guys"("好人");我在华尔街看到的几十万工作人员,至少有95 \% 以上自认是自由繁荣的斗士。

杨、李都早已退休了,现在整个高能物理理论界都是超弦的人,他们若是说什么就会被群起而攻之(敢说话而被围攻的,早有几个先例,例如Paul Steinhardt);就像我不想让有争议的话题在这里扯太远,他们也没空投入这个战场。\section*{【基础科研】高能物理的绝唱(二)}
\subsection*{2015-12-19 00:00}
不对。当初计划LHC的人不是傻子,如果不是大沙漠,LHC的能量绝对足以覆盖新物理的出现。这是因为有量子修正,所以要发明什么新理论,一般牵一髮而动全身,在比新理论低好几个数量级的地方就应该出现端倪。例如统一场论的能阶是设在10\^24eV,但是它仍然在几个eV的常温就会引发质子衰变。实验看不到,那么它就是错的。因为有量子修正效应,所有要避免大沙漠的模型,都是叠床架屋、美国人的所谓Rube Goldberg Machine,一看就知道不可能是自然现象。至于你所说的,就在LHC下一个转角的新现象,衹差一两个数量级,那么其量子修正项必然很大,许许多多的精确测量实验却没看到,所以基本上是不可能的。

主流物理界老是说大沙漠才是不自然,其实是自欺欺人的説法。能很自然地解释大沙漠的理论有好几种,衹是高能物理界不诚实,怕金主知道了就不肯花钱建对撞机,所以这些论文都上不了重要期刊。

虽然理论上非大沙漠的现有模型都丑到极点,我并不会因此而说必然有大沙漠,衹是LHC到目前的实验结果显示的就是大沙漠的可能性最大,至少可以说是超过一半,硬要耍赖就是不诚实。"It is difficult to get a man to understand something, when his salary depends upon his not understanding it!"你该问问自己,是否也得了这个病。\subsection*{2015-12-19 00:00}
我的个性是由事实与逻辑主导自己的观念和偏好,而不是反其道而行。

放下这个信号的统计意义很弱不谈,假设它是真的粒子,它的性质除了第二个Higgs之外,不太像是其他的东西,那么它就衹是标准模型的一部分,大沙漠仍然存在,很难真正有什么兴奋的理由。

过去20多年来,很多高能现象学者为了能多发论文,也卖身超弦阵营,例如我在哈佛合作过的Lisa Randall,现在名声极大,但是一辈子写的几百篇论文,没有一篇是对的,而且每一篇在发表的时候,我就知道它不可能是对的。她名义上是现象学者,实际上是超弦的宣传打手,为了名声地位,不择手段,为虎作伥;像这样的超弦现象学(String Phenomenology,典型的Oxymoron)者,我觉得比超弦论者还要可恶(是鬼子可恶呢,还是汉奸可恶?)。\subsection*{2015-12-17 00:00}
我若是写量子力学的哲学涵义,衹怕读者里不到100人真正看得懂。不过我一直觉得物理系不教这个题目,很是遗憾,或许有闲还是写下来吧。

超弦原本是专门设计来容易做计算的(以便多发论文),所以一切计算的基础是在零重力欧式几何背景下的一个Perturbative Approximation。这不但代表着它没有真正的方程式,而且也不能处理某些拓扑上的问题,更讽刺的是超弦号称是量子重力理论,但是它的计算已经先假设重力是零!

到了1995年,已经有近10万篇超弦论文,可以做的计算差不多做光了。于是有一个阿根廷人又创立了AdS/CFT假设,以便可以做新的计算,发新的论文。20年后,基于这个假设的论文又发了近10万篇,但是仍然没人能证实或证伪这个假设,而且根本就没人想去做,因为超弦界已经"接受"它是对的!它被应用在QCD之后,没有几年就被做QCD的人证明不符合实验,但是超弦界同样假装不知道,你现在到超弦会议上,他们还是拿那个QCD计算来当做样板,证明超弦是"有用"的。

D-Wave是一家商业公司,做虚伪宣传是本行,但是整个应该是自然科学的学术界腐败到超弦的地步,我想是人类进入近代文明以来的首例。\subsection*{2015-12-16 00:00}
我想你所谓的微观和宏观之分,应该指的是量子力学和古典力学的适用范围。换句话说,所谓的宏观就是简单的古典力学计算结果可以接受的范围,而微观就是非要使用困难的量子力学才能得到正确答案的问题。古典力学是量子力学的简化版,它所忽略的那一项是很复杂的,不容易用几句话说清楚。至于量子力学的哲学含义,就算是物理科班出身的,也往往似懂非懂。我还在考虑是否写一篇文章来介绍,不过怕太专业了,没人有兴趣。

超弦对科学的影响是负值,对GDP却有很大的贡献,主要就是靠一些宣传书籍。但是你如果要读虚构小説,《Game of Thrones》比它们精彩多了。至于诚实批判超弦的书,有两本:Peter Woit的《Not Even Wrong》和Lee Smolin的《The Trouble with Physics》;前者比较严谨,后者比较易读。如果你衹想读一本,我建议你读《The Trouble with Physics》的前半;它的后半讲Smolin自己的理论梦想,也不太靠谱。\section*{【空军】【海军】共军小道消息刷新(2015年第四季)}
\subsection*{2015-12-15 00:00}
这种事保密层级太高,我没有确实的资料来证实或证伪大陆网络上常见的吹捧余敏的文章。衹能说美俄英法中的现有弹头,主要都是小型化的二级或三级热核弹。原子弹和中子弹都衹是一级或一级半的非热核弹,完全不是同一回事。所谓氢弹被销毁,可能是因为有些专家衹把早期的大型二级热核弹称为氢弹,网民以讹传讹產生出的误解。这些旧弹头太过庞大笨重,当量虽大,杀伤力反而不如多个小弹头,所以就被放弃或淘汰了。

不过正因为它保密层级如此之高,网民又连一些可证伪的简单技术细节都搞错了,所以基本上可以用逻辑论定所谓余敏构型的伟大也是吹捧出来的虚构言论。这就好像一个探险队没有生还者、遗迹也没有被发现,那么后人所写的探险队长的英雄事迹就必然是虚构的。\subsection*{2015-12-07 00:00}
我个人曾经想对"维稳经费超过国防预算"这个説法挖根,但就是找不到确实的数据和根据。中国的国防预算衹有GDP的2 \% ,连美国的一半都不到,所以和"维稳经费"比起来当然吃亏些。不过我仍然不相信中国的政法系统预算超过美国的水准;美国光律师和法庭就消耗4 \% 的GDP,警察更是人工和装备都远贵于中国,而且叠床架屋,大机关(如学校)有自己的警察、镇有镇警、州有州警,我住的镇警察局就占镇预算的10 \% 左右,这还是高级住宅区,警察平常基本就是指挥交通;在低收入城区,警察的密度要高出好几倍。

我的猜测是中国的政法系统(包括武警)预算就算稍高于国防,比起美国来还算低的。

我不觉得维稳经费高是强力政府的结果;真正可以确定提高了维稳费用的因素是美国的宣传颠覆,全世界不肯当美国附庸的国家都必须付出这个代价。

"新孤立主义"被言过其实。美国衹是用兵花钱花得痛了;如果能够便宜地搞宣传颠覆,还是不会放过任何机会。\section*{【金融】中国燕子}
\subsection*{2015-12-12 00:00}
低阶制造业必然还是得让出来的;其实这个过程已经开始了。

国家之间,优胜劣败,在所难免。中国主导的世界至少是公平竞争,不像美国财阀只想着不劳而获。既然中国主导下的国际关系基础是公平诚实的,那么就可以追求更大的整体效率,使世界经济规模成长得更大。这虽然不是人间天堂,至少比美国为一己之私,压榨全球财富,要好得多了。

Triffin Dilemma要求同时保持贸易顺差和国际收支(Balance of Payments)逆差,这其实很容易满足,中共衹要拿贸易赚的钱到海外搜购优质资產就行了。美国在50-60年代就是这么做的;到70年代之后,财阀主政,为了加大贫富差距,自我消灭了制造业,这才导致贸易逆差。\section*{【基础科研】人类的起源}
\subsection*{2015-12-09 00:00}
不必过谦;能来这里讨论的,都是有能力有兴趣思考世界大势的人。我做为博主,必须维持秩序,并不代表我永远是对的;但是偶尔错误的秩序,要比永远淹没真相的混乱杂音好得多。

自由传媒制度的基本毛病是为了保护少数的真相,必须容忍多数的谎言和蠢话。现代电子传媒使得真假的比率降到极低的水准,再加上财阀在背后有心的运作,事实和逻辑就被完全踢出主流媒体了。

我写稿很难针对一个指定的专题,这是因为每篇正文都建筑在事实基础上,逻辑推论必须是直接而明显的,所以一些大题目就不能塞进这种格式里。我的意图是用多篇文章提供足够的事实和逻辑根基,那么读者自然能自己建构完整思想体系来回答大的问题。\subsection*{2015-07-25 00:00}
我所指的西欧是不包括英国的,因为后者在文化和霸权歷史上,与前者有明显的差别。

现代的西欧,除了法国之外,基本已经完全放弃殖民帝国的梦想。这其中固然有现实的因素,但是文化上的偏向也是眾所周知的。

至于酱缸文化,的确是在科技相对原始的背景下,将稳定置于发展之上的一个取舍。不过我不觉得你们的不同论述有绝对的矛盾;看来只是不同角度、不同重点的表述。

我不太确定人口论和我们的话题有什么关系。我欢迎你的参与;不过因为你想的、写的很多,更须要注意简洁、切题、踏实的几个原则。我希望每个部落格的读者都能把留言栏也浏览一遍,那么个人随兴而想到的意识流就不应该在此出现。大家都必须用心提高留言的资讯密度,让大多数的读者觉得投下时间之后有所收穫。

连我自己都有时必须留下超过一半的资料,不写进文章里。在留言栏里要把个人的每条思路都写下来,不是天赋的权利。\section*{【德国】【美国】舍生取义的政治人物}
\subsection*{2015-12-07 00:00}
我同意Sarkozy是个自私的政客,和财阀关系密切,他在2007年上臺接任Chirac加速了法国国力的衰退。他如果回锅,炒做民粹是免不了的。不过我觉得他不敢对欧盟釜底抽薪,所以问题仍然是慢性病,还可以拖过2020年。

Le Pen在2017年上臺的可能性还小;如果真的发生了,她也应该在真正的责任和压力下,避免以前所鼓吹的那套疯狂极端思想,顶多对回教徒做一些象徵性的限制。希腊选了个极左派的总理,不也衹是空耗了一年,最后还是得乖乖就范?

基本上法国不是英国,它是欧洲大陆的核心部分,整个国家的政商菁英都不会容许退出欧盟这样的自杀行为,下任法国政府就算闹一闹,最终也必然是雷声大雨点小。\subsection*{2015-10-21 00:00}
我想你指的是九月初在留言栏的讨论,我说Merkel开放移民是玩小聪明,故做慷慨,实际上是要拖整个欧洲一起来买单。

现在病急乱投医,找上土耳其来当救生圈。这些问题,追根究底,就是Merkel没有远见,没有一贯的战略,一味短綫操作,衹想满足民意,敷衍了事。Syria的难民很难预见吗?那么北非的难民已经闹了好几年了,藉口在哪里?近五年来,大街小巷讨论欧洲前途的文章,没有一篇不提难民问题,结果Merkel衹想坐在欧洲霸主的宝座上,不但不能未雨绸缪,连紧急预案都没有准备。

Renzi在意大利搞真的改革,Osborne在英国做大战略的突破,Merkel却在虚耗,德国危矣。\subsection*{2015-10-08 00:00}
你的另一个留言不但离题,而且是意识流,删了。

讲到智商、文化这些题目,请确定是与政治改革有关的部分。

智商是综合天赋和教育的结果,我自己有时提起,都是用来指出个人做决定和选择的能力,至于它有什么成分,题目太大,与政治没有直接的关系,所以除非有明确的新科学证据,还是不要分心去提的好。

至于文化,我以前讨论过,台湾承袭了旧中国的乡土文化,在民主政治上无法顺利运行。德国的公民水准当然是高得多,所以Merkel以民调治国了十年还没有出明显的大乱子,但这是承袭19世纪末期改革的结果,Bismark对帝国新领土的彻底改造是有明确歷史记录的,所以它是人为的成果,和种族没有关系。\section*{【工业】中共的下一个產业技术攻关:晶片}
\subsection*{2015-11-28 00:00}
工业化和城市化是经济发展过程中相辅相成的步骤。中国的工业化已初有小成,《中国制造2025》是继续提升工业能力的计划;而中国的城市化程度还很低,衹有不到40 \% ,8亿多乡村民众仍然等着踏进现代工业化之后的城市生活,所以中国的中长期经济展望极佳,那些唱衰中国的西方人士都衹是在发泄自己的私欲,而不是做理性的判断。

中国经济在短期(两年)内是有困难的。这主要是前几年以出口为导向的举债投资做得太过头了,现在全世界的需求跟不上来,原本就必须要做一些痛苦的调整。结果今年的股市又遇到严重的人谋不臧,所以眼前的挑战就更严峻了。

今天传出消息,中共常委会特别开了扶贫会议,我想这在全世界都是绝无仅有的。习近平至少大方向搞对了,现在只等着看执行团队是否给力。\subsection*{2015-11-24 00:00}
欧美半导体工业并没有严重衰退,中韩的崛起吃的主要是以往日本的额分。

你提的资料里面有些很小的错误,虽然无关宏旨,我还是顺便更正一下好了。臺积电做的是16nm,不过比三星的14nm技术还稍强一点。

10nm其实衹是一个过度性的阶段,很快(两年左右)就会进展到7nm。我问过我的同学Kurt(半导体的行内人),他说因为FinFET(三维电晶体单元)已经在14/16nm实现了,此后两代制程没有质的改进,纯属微缩,所以可以进步得很快。

5nm才是真正技术上的困难,不过最大的问题还是经济上的。每个电晶体的造价到28nm之后不降反升,所以从20/22nm开始,愿意用下一代制程的顾客就越来越少,到最后必然会不再有足够的报酬率来盖越来越贵的新一代生產綫。\section*{【空军】【自由时报】一架F22戦机完胜中共空军?}
\subsection*{2015-11-27 00:00}
S-400从S-300到现在发展了30多年,它的系统整合已经到炉火纯青的地步,在细节上必然有可借镜的地方。HQ-9毕竟是比较不成熟的產品,光是有较优秀的AESA收发原件,不见得雷达的整体效率样样都胜出。此外S-400有三种飞弹,超远程的还没有研发完成,远程的与HQ-9性能相当,但是中程的却比HQ-16要紧凑得多,同一的发射筒能挤下四枚飞弹,这就有可观摩之处。

J-20除了隐身能力外,光电和雷达系统也达到了F-22和F-35的水准,比SU-35高出一截。J-20应该会被用来攻击美方的预警机、指挥机和加油机;如果还有F-22能到臺海上空来求战,大概会是由J-10、J-11和SU-35在预警机指挥下以多打少。\section*{【空军】共军小道消息增补:Su-35}
\subsection*{2015-11-21 00:00}
你还是中了超大之类的毒,所説的乍听之下很中庸,其实完全经不起逻辑推敲。

24架能补充多少產能?

J-20后年就批量服役了,SU-35交机早它不到一年,说什么"J-20离服役还远"云云?

J-11、J-15和J-16的產量如此有限,正是因为渖飞的技术不过硬,飞机问题多多,共军采购的热情不足,而不是已经到达產能极限。以渖飞这种规模的公司,部队又有极大的需求,却一年衹生產不到两个团的飞机,衹有在印度才有类似的例子。產能要上调也就是几年的事,渖飞的每一个机型都延误了十年以上,如果定型后的飞机真的好用,早已扩大產能几倍了。

这个问题我已经反反復复解释了几十次,如果你没有真知灼见,请不要随便抄袭其他网站的歪论到此乱唱反调,浪费大家时间。

你的新抬杠帖被删了。严重警告一次,再犯则我连你新帖的内容都不用看就直接删除。\section*{【工業】再談中國的核電發展}
\subsection*{2015-11-09 00:00}
Using particle beams to induce fission reaction without going "critical" (i.e. chain-reaction) is an old idea among academics, but has never been used on large-scale application due to its impracticality. Basically, it is a fast-neutron reactor where most of the neutrons have to be supplied by outside sources. The Fast-Breeders will be at least 10 times more economical.\section*{【陆军】即将出现的新装备(二)}
\subsection*{2015-11-07 00:00}
我拿到博士学位后不久就离开学术界,没当过教授。

简氏衹是胡猜。中国的火箭弹技术领先世界,所以厂家自然想赚外匯;而且这类的陆军武器,衹要是共军自用的新装备,就不会拿出来外销,A300、SY300和AR3出现在国外防展反而代表着它们不会被共军采购。

有趣的是WS-1、WS-1B、WS-2、WS-22、WS-32、WS-33、WS-35全家都到杜拜去找国外顾客,WS-43却没有出现。

此外,370mm的直径太小了,和PHL-03的300mm太过接近,衹能用来替换PHL-03,可是这么新的装备需要替换吗?这并非完全不可能,因为PHL-03没有模块化设计,重新装填速度慢,但是共军似乎还没有到如此财大气粗的地步。

若是专为臺海战役而买另一型远程火箭炮,直径必然在400mm以上,超过500mm都不算太大。我仍然认为最可能的是基于WS-43,再加以大型化、远程化。\section*{【政治】自断后路的狷介清官}
\subsection*{2015-11-01 00:00}
有关反贪配套政策的问题,我同意应该是赏罚皆用的;不过在初期一概从严办理有示范性的必要,以免造成一般民众的困惑。此外,你所不满的"不就一点钱的事",真正的限制来自法律本身没有弹性的特质,如果要搞法治就必须接受法治不完美的地方。

至于閲兵,是我自己没有细查,的确是军级的少将走了正步。不过我仍然不同意这个"出洋相"的説法。美军将官即使是到了上将,还是每年要做一次和士兵相同的体能测验;我曾问过一名海军士官,他当然觉得将官的体能不能和他们比,但是大家也都看得到年龄的差距,我觉得这类的政策反而是有益士气的。共军号称是人民解放军,如果当了少将或中将就摆架子,更是不像样。\section*{【媒体】克拉运河的假新闻}
\subsection*{2015-10-25 00:00}
我很快地看了看那篇文章,觉得是典型的"无限上纲"类歪论,亦即藉着挑肥拣瘦,衹给单方面的例子,将普世问题定调为制度问题,然后对制度大幅挞伐。

读这种文章的时候应该时时自问,这些问题在不同的制度下就不会发生吗?衹有"专制"才有权力集中吗?衹有"专制"才有腐败吗?衹有"专制"才会因私害公吗?衹有"专制"才有惨烈的政治斗争吗?其实读读希腊和罗马共和时期的歷史就知道答案何在。这种文章会有市场,往往是读者本身的学识不够,不能自行举出反面的例子,所以多读好书是有益于一个人对宣传的抵抗力的。\section*{【空军】【海军】共军小道消息刷新(2015年九月特刊)}
\subsection*{2015-10-07 00:00}
我的年纪比较大,又在华尔街打过滚,对人性的期待比你低得多,所以"换柱"本身并不奇怪,奇怪的只是这些混到党国顶层的人物,手段居然如此笨拙,基本就是学生社团的把戏。就算达不到《纸牌屋》的水准,找几个白手套来造谣抹黑有那么难吗?我原本是等着看"她是同性恋","她有黑人男宠","她喜欢和狗玩","她在大陆有大笔银行存款"之类的中伤(根据可靠的歷史传记,Lyndon Johnson是"我的竞选对手喜欢和动物玩"的谣言的发明人)。台湾人没有专业精神,连做到总统、主席和院长的职业政客也是如此,真是可怜又可悲。\subsection*{2015-10-06 00:00}
这些视频我在美国看不到,不过没关系,央视的内容必然只是请个军迷来胡猜。共军到2015年都不肯透露J-20的部件细节(Wikipedia英文版的J-20参数还是我几年前给的),不可能在2011年就公布发动机用了哪一型(其实连J-10,J-11,J-15,J-16用的发动机,共军也从未公布,不管是央视、环球网还是其他官办媒体上的资料,都是找军迷来判断的,没有官方的背书)。

至于刀口,他犯错太多太频繁,我自己不会去看他写什么,也建议你不要把他的话当真。像是分辨AL-31和WS-10,是中等军迷的基本功,他连这都做不到。如果他真有自信,可以开始用实名发言,那么我或许会再听听他想说什么。\section*{【基础科研】猿类的起源}
\subsection*{2015-10-03 00:00}
Actually they did, but majority of the mice died within 4 weeks due to hyperuricaemic nephropathy. Apparently, the ancestors of primates already gain significant tolerance towards uric acid prior to the total deactivation of uricase.

In primates, the absence of uricase raised the uric acid level from 1 mg/dL in monkeys to 6 mg/dL in apes. Human is the only species known to suffer from gout spontaneously.

Presumably, the experiment you have in mind can be conducted on monkeys instead, but it will require much more resources. I am not aware of any result from this, but then again, I am not really a professional in this particular field.
\section*{【海军】【空军】忽悠大眾的虚拟武器}
\subsection*{2015-09-16 00:00}
英国发动一战犯了战略大错,给予美国崛起的机会,和小布希侵略伊拉克给了中国十年战略机遇很类似。一战结束后,美国不也就谈出了华盛顿海军公约,公然与英国并列第一,这相当于中美的G2机制,可是中国并没有争取G2,所以很明显地中国比上世纪的美国低调。

叙利亚问题是俄国的战略利益,中俄既然有非正式同盟就必须给面子。

你引用的这篇文,十段话里倒有九段半是空洞的口号,以后请不要拿这样的垃圾来占篇幅。至于"这些移动系统不可能被歼灭光的"论点,考虑共军会有绝对空优和电磁控制,只要国军雷达一开机就马上完蛋,那么就只有近程的红外线制导武器能存活过头几个小时。这些武器连战术都谈不上,只是战斗上的辅助而已,要凭它们挡住两个集团军是典型的痴人说梦。可怜的是,台湾什么都没有了,就是痴人最多。\section*{【美国】从期中选举看美国民主}
\subsection*{2015-09-10 00:00}
美国关心政治的财阀主要通过智库、候选人和游说客在幕后操控,有名的不多,最常被提起的是Koch兄弟。

不是有钱就能加入权力核心。Trump一副小丑模样,真正有权的财阀还不屑与他为伍。而且他原本是民主党人。

打击Trump是为了保护自己的候选人,也就是原本已被内定的小小布希。不过财阀也不是铁板一块;Ted Cruz的背后就是德州的一个石油财团。会议当然是开过,而且开了很多,但不是完全秘密的。1970年代的修法使候选人得以堂而皇之地开募款会,由富豪找附近的朋友来开Private Party。这些为候选人开Party的富豪在他们当选后当然可以直接打电话到白宫来"讨论时事"。\section*{【台湾】愚民主政下的指鹿为马}
\subsection*{2015-08-30 00:00}
Unfortunately, NZ is one of the 5 "eyes", i.e. a surveillance alliance with the US. There has to be heavy infiltration into the NZ government by the CIA.

I don't think those who support Trump are doing so as a joke. They are deadly serious and deadly wrong. Credit it to the American propaganda.

I do not have any activist experience, although I know someone who fits the bill in Massachusetts. If you give me your email, I will introduce you to him.\subsection*{2015-08-29 00:00}
I think this is merely the result of accepting American culture without questioning its merits. As I discussed before in 《印度的烈火系列弹道飞弹》,  under the modern American version of capitalism, everything is free (i.e. does not matter) except for money. So being rich and famous is the only thing that counts. Add the Taiwanese herd mentality and voila, you have porn stars on government issued ID.\subsection*{2015-08-16 00:00}
这个链接打得开了,它也的确说中山院可以生產焦平面阵列型的红外寻标器,可是它没说已经被整合到武器里面了。

一般没有工程背景的人常常以为一个新式部件做出来之后,就可以直接替代旧式的部件,其实重新整合往往比开发那个部件还困难。例如共军的电磁弹射器一两年前就已经试机成功,但是要装上航母只怕是2030年后的事。

台湾有能力做焦平面阵列,我当然知道。上个月我在台湾时还访问了一家公司,他们的民用红外线阵列有500万个像素,应该远远是世界第一的。问题在于中山院有没有能力做出军用标准的產品,并且把它整合到武器上。如果要整合,第一优先的应该就是天剑一型。那么为什么没有听说有天剑1A呢?这是因为换了寻标器就必须重新写软体,要重写飞控软体就必须先吃透飞弹的气动特性,但是天剑只是对AIM-9P4的直接拷贝,中山院当初并没有从头做全部的风洞试验,现在若是要改,工作量与重新开发一个空对空飞弹差不多,那还不如直接照抄AIM-9X容易。

我猜海剑羚也就只是还在卡通阶段,主要的工程步骤如风洞试验都还没做,目前只是想骗经费而已;如果中山院有钱有能力开发海剑羚,那还远不如先抄AIM-9X重要又简单,但是我们没听说有天剑三型,不是吗?\subsection*{2015-08-15 00:00}
本栏只接受科学性的逻辑讨论,歼星舰之类的妄想不受欢迎。你若是再写这种"真假又如何,人家应有发言权吧?"的鬼话就会直接被删,这次例外,因为你刚好为了主文里台湾指鹿为马的乱象做了示范。法治制度下,权利只代表不犯法,并不代表它是对的或者社会大眾应该接受它,就好像拿屎抹脸也不犯法,但我还是可以批评的。所以"发言权"不代表社会不应该嘲笑、批评假话。台湾的问题就在于整个群体只接受自己爱听的谎话;一个正常的社会应该知道真与假是不平等的,对与错是不一样的,黑与白没有平衡可言;指鹿为马成了常态是有代价的,如果这个代价由弱势群体承担,吃饱喝足的胡闹分子就是连续杀人犯,我已在《政府的第一要务》里解释过了。

1. 雄三是ALVRJ的陆射版,只因有国军文字打手出来吹牛,不能代表它有根本性的改进。最重要的证据恰是国军自己的吹牛文化,连沱江级都可以说成航母杀手,大杀052C和SU-30MKK,却不敢出面说雄三射程有150公里以上;那么真相是很明显的。

2. 神盾系统对中山院的难度和歼星舰差不多,都是完全不可能的,那些说辞只是骗经费的。此外连美军都嫌9000吨的Burke级太小,不够装载最新的阵列雷达,中山院准备要造几吨的?万吨以上的现代防空舰也是全世界只有两个国家才有能力,中山院算老几?

其他扯太远了,纯属抬杠。本部落格不搞这套。

\subsection*{2015-08-15 00:00}
1. 搜索了"梅復兴",似乎是一家一人公司的"主任",那家"Taiwan Security Analysis Center (臺海安全研析中心)"没有自己的网站,只有Facebook的网页。不论他实际是谁,我对台湾的名嘴不熟,也没有兴趣。我说过滤,这些名嘴都是第一层就被过滤掉的东西。

2. 你要讲理,我很欢迎。但是请不要占太大的篇幅。留言栏是所有读者分享的资源,必须保持精简,以免浪费大家时间。

3. 台海若开战,所有地面雷达和机场都会马上被火箭弹饱和轰炸5-10遍,如何支援没有自身传感器的海军舰艇?

4. 这篇文章我以前看过,科普方面还好,都是很基本的东西;台湾方面的雄三消息就不太可靠了,因为他自己不是官方,也没有引用官方,贸然断言一些性能指标,完全无法证实,所以我把它归檔于"不可靠-存疑"的类别。

台湾民主化之后,武器发展也染上浮夸风,连明明做不到的东西也公开吹嘘,所以客观的分析只好对没有坚实基础的资料都从严审查,毕竟没有被公开吹嘘不就代表着它比那些被吹的更糟糕?这是逻辑,不是偏见,要怪必须怪那些制造离谱大话的人。\subsection*{2015-08-15 00:00}
共军的文化和民主国家的军队刚好相反,总是拼命隐藏自己的实力,几乎所有武器的性能指标都是被严重低估。

YJ-62的所谓280公里射程是外销型的数据;这主要是因为国际法限制300公里以上射程的飞弹外销。中共的办法一般是用软体限制射程,一旦卖出去以后,买家按几个键把它又"改装"为较高的射程就不关中共的事。

YJ-62的射程有600公里当然不是共军自己吹的,是几年前一个大学教授的论文讨论一个"已实战部署"的巡航反舰导弹的最优编程方法时泄露的,被挖了出来。既然YJ-62是共军射程最高的反舰飞弹,那我们很保守认为那个教授说的是YJ-62。

YJ-62没有空射型(不像你们这些信口编谎的人只有空射型),否则会叫做YJ-62K。

你拿"大陆的百度和铁血网"来说事,哈哈哈哈哈哈哈!它们正是自由导致堕落的示范。

"外界评估"?!台湾的外界就是指鹿为马的集体幻想世界,正是这篇文章要批评的。\subsection*{2015-08-15 00:00}
一. YJ-62的『高-高-低』射程是600+公里,『低-低-低』射程是450+公里。YJ-62是设计来打航母战斗群的,有卫星可以侦察目标。其实航母战斗群的防御半径远大于1000公里,600公里还不太够用呢。雄二、雄三都是基于很旧的美军装备,150公里已经是很慷慨的估计,也就是『高-高-高』。既然海峡只有150公里宽,乾脆只用陆基飞弹岂不是既便宜又安全?
二. 沱江舰的隐身远不如022级,光看发射架之间有多少直角就知道了。即使隐身处理正确,500吨的船还是会比30吨的飞机截面积大,KJ-500这样的L波段AESA也会有300公里左右的探测能力。反之,E-2T的探测距离只有KJ-500的一半左右。还有,"CS/SPG-6N,具有对海对空的搜索能力",你先看看那个天线根本就是2D的,这个所谓的对空能力必须是二战的定义;我们其他人活在21世纪,必须用21世纪的定义。至于数据链,沱江舰没有对空火控,从何谈起。
三. 以国军的军工能力,迅海第二阶段纯属空想。
四. "梅復兴",你觉得是真名吗?
1. 海剑羚既然没有资讯,依逻辑,如果能开发出新阵列式红外线传感器,岂有不大吹大擂的道理。视频里连折迭式弹翼都要吹一下,这么重要的东西怎么会不谈?这么困难的技术,投资发展过程怎么会没有蛛丝马迹?做出来怎么会不先用来改进空对空飞弹?若是没有根据也要做白日梦,乾脆假设国军有歼星舰藏在月球背面算了。讲科学就要用Occam's Razor,最简单直接的合理解释是正解。
2. 最简单直接的合理解释是红外线传感器根本就没换。他不同意就必须提出证据;而且还必须解释为什么中山院没有大吹大擂。
3. 近防飞弹没时间搞发射后锁定。别说沱江舰没有对空雷达,就是Burke级也只有三个引导天线,还是机械式的,哪有时间转过来照射目标?现代的反舰飞弹都是可以编程的,至少会有十几枚同时到达,美国人不是傻子,若是不需独立雷达,他们为什么要多此一举?
4. 原来这是未来式,那么我们就等未来再谈吧。毕竟中山院成功开发这种先进武器的记录是0 \% ,不够赊账。
5. 近防飞弹目前只有两个国家成功研制,连俄、英、法都做不出来,国军还是洗洗睡了吧。要是有那个钱,开发出200-300公里射程的GPS制导火箭弹要容易一个数量级,也更有用一个数量级,让中山院去搞搞,说不定还有1 \% 的机会能在10年内做出来。国军的海军纯属浪费,不如整个裁掉算了。
\section*{【基础科研】基因工程与分子生物学的新发展(二)}
\subsection*{2015-08-28 00:00}
谢谢你的指正。

我知道你说的对,但是我这是一篇科普文章,是给外行人看的,如果细分到核酸化学和基因工程的差别,可读性就没了。我要以一页的篇幅来介绍两个重要的话题(即新一波的医疗突破和CRISPR)必然得要简化一些细节,尤其不能引入太多专业词汇,选择用"基因工程"和"分子生物学"这样笼统的外行人分类是有意的,还请专家们见谅。

我也知道CRISPR技术能对好几个基因同时做修改,这是它相对一些旧技术的另一个优势;不过Gene Editing和Genome Editing之间的差别同样是不适合在这么简短的文章里讨论,所以也就一样蒙混过关了。

我的文章素来注重可读性;这篇已经是开创部落格以来最硬的文章了,我实在必须Cut Cornors,请知情的读者谅解。不过留言栏是个补充细节的好地方,欢迎你们来指教。\section*{【基础科研】基因工程与分子生物学的新发展(一)}
\subsection*{2015-08-26 00:00}
我觉得他的逻辑很乱,但是总结起来基本上就是"生物不像物理那样有简单规律,所以不是科学"。前一句话当然是对的,后一句则当然是错的。科学只是一个坚持事实与逻辑的态度(详细的定义我以前已经讨论过了),用同样的态度去研究不同的现象,自然会有不同的需求和不同的成果。其实生物只是因为研究对象太复杂,所以没有简单规律;在物理方面也有类似的分支,基本上一切都是靠实验观察,例如流体力学。

他列举了很多生物方面的吹嘘夸大,可是却没注意到物理和工程上其实也有很多这样的例子,超弦更是纯粹的偽科学。

你等到下面两篇文章出来,就会了解为什么我对基因工程如此兴奋乐观了。\section*{【美国】谈一个河流污染事件}
\subsection*{2015-08-19 00:00}
我们这里不流行抢沙发。

英美媒体的偏见根深蒂固,主要是歷史上本来对国际关系就没有客观理性的传统,再加上种族歧视的因素,所以像是《经济学人》和《Washington Post》也只是我搜集事实的手段,逻辑必须自备。

是的,电视的频道越多,社交软体越方便,当愚民就越容易越舒服。我在退休之后就强迫老婆把有线电视停了(结果她开始上图书馆借Audio Book来听,很好),自己到现在也还没有开Facebook或Twitter的账户。

我以前已经推荐过几本书了,可以到旧文章的留言栏去找找。例如福山的书就很值得一读。

以后还请你为大家方便着想,设法精简留言。\subsection*{2015-08-19 00:00}
你说的有些道理,不过不太完整。这些事件使国家失去公信力,但是其实都是行政上的决定;如果有法治的制度、机构和传统,就可以完全失去行政上的公信力,而仍然依靠对法治的信心维持向心力和社会秩序。

美国就是一个很好的例子:两个罗斯福建立了强大有为的行政部门,经过尼克森的水门案和雷根的"政府是问题根源"论之后,基本没有人相信联邦政府了。很多民调显示,中国百姓对政府的信任与依赖其实远高于美国。但是这只局限在行政方面;美国的公信力恰是强在法治方面的。

我还觉得中国的公信力缺失和对公共安全的重视不够很有关系。在美国威胁要炸公共建筑是重罪,中国人当作玩笑话处理;在美国如果有潜在的公共危险性的行为,例如偷工减料,社会和法庭都会很认真,在中国要爆炸了才会追查,台湾则连出事了,只要往政治上扯一扯,可能都可以胡混过关。\section*{【陆军】【海军】共军小道消息刷新(2015年第三季)}
\subsection*{2015-08-05 00:00}
See the earlier article 《2030年左右》for China's commitment towards reducing carbon emission.

The US rules the world with an iron fist. If China does not build up a reasonable military, it is sure to be taken advantage of. You are simply falling for American propaganda that blames the victims for America's own aggression.\subsection*{2015-08-05 00:00}
Of course the US military would not stand still, but its new weapons are hugely expensive and have to be budgeted years in advance, so we have a pretty good idea on what is to come in the next decade.

For example, the US navy is moving to the next generation of aircraft carriers and ballistic missile submarines. It simply has no budget left to do a new cruiser or destroyer in the next ten years.

War is the most legitimate reason for territorial changes. If China wants to take back Taiwan or stick to its claims on the islands, it will be very hard to succeed without a war, unfortunately.\subsection*{2015-08-02 00:00}
即使是有这些限制也没有任何实际上的意义,因为替换进口零件不但比重新研制新客机容易得多,而且中共早就着手了。除了引擎之外,其他的部件还必须进口的原因不是中共造不出来,而是中方的版本还没有通过美国FAA的认证。没有FAA的认证,就不能飞出国;但是这是手续问题,不是技术困难。

至于引擎,中方也早已在研制WS-20。这是WS-10的大飞机版本,在2020年前绝对可以量產。其性能应该还比Leap X1差一些,但是军用版的C919稍为费油一点并不要紧。所以一旦C919在2017年有原型机可以移交给军方,改装为预警机、反潜机和加油机的工作就可以马上开始。请注意,不准军用的规则,其重点在那个 "用"字。中共连把这些部件拆开来研究都必然会做,把装有进口引擎的C919的原型机用来做研发自然是不在话下。反正没有部署就不能确定其用途,也就不算违约。等到要量產了再换上WS-20便是。\section*{【台湾】如果我是总统候选人}
\subsection*{2015-07-21 00:00}
他说的并非没有道理,中西亚的乱象的确代表着"一带一路"有其极限,一般媒体也的确是有很多浮夸的地方,例如欧亚高铁我也一直认为是不切实际的胡扯。

不过我不认为"一带一路"整体上是个忽悠的壳子;那些浮夸忽悠的言论并不来自中共,而是媒体自行扯出来的。欧亚高铁不切实际,但是欧亚货运铁路线却是有真正价值的新发展。"一带"是把俄国和中亚拉进中国经济引力圈的招牌;"一路"则包括东南亚、南亚、中东、南欧和东欧,都是中共外交、经济发展的重要方向。那么没有包括进去的东亚国家只有日本、台湾和韩国;前两者本来就是捣蛋鬼,后者则刚好相反,已经签了自贸协定。

中共自己也一直努力否认"一带一路"是马歇尔计划,不过媒体这样渲染,反而鼓励路上的各国热情参与、认真对中企开放,对中方的基建投资和战略路线开辟都有好处。实际上中共当局并不会做赔本生意,那位学者大概是多虑了。\subsection*{2015-07-21 00:00}
台湾在国际上能拿出手的高科技工业只有两个:半导体和机床,但是这里面大部分还是以低价为核心竞争力的中型企业,我预计五年左右就会被全部打翻了。例外的有三家公司,就是台积电、联发科和鸿海。鸿海靠的其实不是技术,而是组织和经营能力,雇员又多不在台湾,所以可以忽略不计。联发科是唯一全面嵌入红色產业链的台湾高科技公司,几年之内应该还有不少成长余地,不过十年之后就要看中国是否要扶持自己的產业了。

台积电不"只是代工",先进制程的智慧產权才是它的核心价值。目前大陆最先进的制程是台积电十年前就有的,但是中方和比利时签约要在2020年引进14奈米制程,这是台积电今年年底才能量產的东西,差距就只剩五年了。依这个速度,在2025年左右,大陆会赶上台积电,那也就是台湾制造业的末日,就算联发科还在,它的规模和產值都比台积电小太多,独木难撑大厦。\subsection*{2015-07-07 00:00}
很可惜的是,20多年的戒急用忍下来,台湾传统的制造產业已经被中国赶上了,却又大多没有插入中方的產业链的核心(其实即使像联发科这样彻底插入的,长期之后也没有用,因为台湾政坛以反中为正统,中共不会放心让台企掐着他產业链的脖子),所以中国做下一波產业升级的时候,台企首当其衝,必然会倒下一大片。

如果下一任台湾政府能力挽狂澜,重建与中国的良性互动,那么联发科或许可以反过来成为世界级企业,如同华为和中兴一样,不过其他的中型电子企业如日月光,和机床业还是死定了。

长期来看,台湾必须做大陆不想也不能全拿的东西,也就是服务业,尤其是生技医疗,大陆还很落伍,而台湾有相同的中医传统,这将是一个极大的市场,但是政治上不先建立条件,最终仍将是画饼一块。\section*{【战略】再谈希腊与欧元}
\subsection*{2015-07-20 00:00}
The democracies in Scandinavia seem to work just fine (as long as new Muslim immigrants are not involved), but these are small homogeneous countries with very high median income and education levels, together with a socialist welfare system. I just don't see how Taiwan can hope to make democracy work.\section*{【金融】【战略】希腊与欧元}
\subsection*{2015-07-12 00:00}
其实如果齐普拉斯真的缴交了欧元区能接受的提议,这是一个相当高招的两面讨好的手段:选民赢了公投,得了面子,欧元区得以强加原先坚持的条件,得了里子,而齐普拉斯更是走了唯一一条能让自己继续当首相的路。若是他公开对欧元区投降,则选民不会饶过他;若是他坚不投降,则希腊被踢出欧元区,选民还是不会饶过他,所以他唯一的出路就是对内譁眾取宠,对外则出卖自己的支持者。希腊选民如此脑残,也只有存心欺骗他们然后把它们包装出卖的政客才能当选并续任。

不过事情还不明朗,欧元区会不会买单很难说;就算这回合买了单,几个月内,齐普拉斯必然会再度反复,到时大家还是回到第一回合。债权人顶多再多损失一些钱,希腊百姓才是最大输家:原本已经稳定的经济再度崩溃,却毫无补偿,这叫自作自受。只有齐普拉斯得以呼风唤雨,算是赢家。

希腊不可能依附美国。美国连乌克兰都没钱支持,收买希腊哪有可能叫价叫得过中俄?

有些大陆读者的帐户被骇过了,来看文的时候会留下广告,我也只能一个一个的删除。我对中时建议过要把广告帐户禁了,但是还没听到回信。\subsection*{2015-06-28 00:00}
自从我写了这篇文章后,四个月来基本上Syriza到处碰壁,而且希腊资金大幅流失,经济更为衰弱。如果Syriza还有点理性,早就投降了;偏偏他们是股市菜鸟的心态,赔得越惨就越要死抱错误政策到底,不肯认赔杀出。所以很可能再拖上一两个月,直到破產为止。

照理说,国债还不出来和退出欧元区是两回事,可是Syriza如此顽固,德国必须杀鸡儆猴,那么被踢出欧元区就成了理所当然的结果。其后Syriza只能重发Drachma,应该在一个月内就贬值70 \% 左右,基本把中產阶级的储蓄扫的干乾净净。这时希腊还是要再发行新国债的,但是除了中俄以外,有谁会拿钱往水里扔呢?中俄政府当然也不是蠢蛋,自然会要求希腊拿有真正价值的资產来交换,到时港口、铁路还是要卖的,而且价钱更低。

一旦希腊对中国完全开放,中共当然寧可希腊留在欧盟,成为中国企业、移民和贸易进入欧盟区的桥头堡。所以只要希腊政府配合,中方的援助可能极为可观。而长期来看,希腊成为中欧之间的快捷通道,对大家都有利,唯一吃大亏的是美国。对德国则有利有弊,有利是可以与"一带一路"进一步整合,有弊是欧洲国家有了另一个大哥可以抱大腿,德国对欧元区以外的欧盟国家影响力会下降。\section*{【歷史】装甲将军}
\subsection*{2015-07-04 00:00}
1. Please be specific. Remember Raus was a low-level commander, with little power to change the strategic situation.
2. Dunkerque was a big mistake; taking on the Soviet was another. But that does not mean defeating the USSR was impossible.
3. The disaster at Stalingrad was entirely Hitler's fault. How does this contradict with German officer corps' superiority?
4. This was not a part of my conclusions.
5. I think the correct statement is: "The USA will never let Western Europe shed its fear of Russia."\subsection*{2015-07-02 00:00}
Even with a very low expectation on the Taiwanese political class, I am still disappointed from this trip that literally no critical thoughts have  been bestowed on the TPP. Everyone, including 洪, just wants to rush in. This is just another ridiculous example of victims of US propaganda walking straight into the oven so that they can be roasted as dinner.\section*{【美国】美国式的贪腐总统}
\subsection*{2015-06-25 00:00}
I think the Chinese government is facing two forces of drawing capital overseas: the first is the US Fed stopping QE, thus reversing the dollar flow; the second is the Chinese property market, whose bubble was recently deflated, releasing a tremendous amount of idle capital. The People's Bank wants to keep the majority of these idle capital in the country, so the stock market is allowed to absorb them.

It makes no sense to grow another bubble, but the market should not be allowed to crash either. The People's Bank will let it run within a very wide band. I do agree with you that the small foreign investors should keep clear.
\subsection*{2015-06-25 00:00}
我的确是和张小姐见了面、交换了名片。原本不知她为什么特别赶来说Hi。

你说的义利之分,其实正是我上了杨世光节目之后所想的。他的观眾大多数是为了炒股而看的,但是他那正确的世界观也就一点一点地感染了群眾,所以是很有意义的。

至于我自己,这次的系列演讲,其实我有一点气馁的事,那就是能真正吸收的听眾,都是教授级的人物;越年轻的体会的越少。我想不管我如何努力把事实真相整理明晰,这些题材先天上就是只适合小眾的。要把它包装起来给大眾消费,这是媒体人的专业,我可能做不到。

我在57金钱爆上或许语出惊人,不过那不是有意而为;我所说的都是针对那些问题的正解,我相信你已经都很熟悉了。如果你觉得我应该说得委婉些,我以后会特别注意,不过我的性格原本就是不说废话,直接讲到事情的中心要点,要学新的技能可能有困难。\subsection*{2015-06-15 00:00}
没什么,美国人被洗脑的程度,应该是世界第一。最近有人做了实验,在美国街头搜集签名,要求美国政府立刻用核子弹头轰炸莫斯科,结果90 \% 以上的美国民眾欣然参与。我实在想不出另外一个如此脑残的国家。

原本那样的留言应该直接删了,但是你们已经回了话,我不想让你们花了时间写的东西也跟着失去意义, 所以就把它留着,自己也做了评论。

真相与虚偽,正义与邪恶,逻辑与愚蠢之间没有妥协的余地,只能斗争到底。一旦真相、正义、逻辑开始犹豫了,虚偽、邪恶、愚蠢就会扩张。大多数的世界民眾是没有话语权的弱势群体,也是强权掠夺压迫的对象。我们这些少数有话语权又了解真相的人,为了保护弱势的多数,有絶对的道德责任不能退让。\subsection*{2015-06-15 00:00}
Oh, I was hired to perform a duty which I did better than anyone else, not to sell my conscience. Do you understand what the word "truth" means? Did you ever have a conscience in your life? Your logic is that taking a job in the US means the person has to cheer all the killing/lying/robbing done by the American government. That is called selling one's soul. It appears that your soul is not only for sale but also pretty cheap. Mine isn't.\subsection*{2015-06-13 00:00}
The forces behind TPP are immensely powerful, and I won't be surprised if they succeed in bulldozing all oppositions.

The US has been exporting corruption to the Western world for over half a century. For example, the French banking industry used to be run like the Chinese one, i.e. focused on serving the real economy. Not any more, it was deregulated in 1960's after intense lobbying. Not even General de Gaulle could stop the deregulation.\subsection*{2015-06-13 00:00}
人类是群居动物,演化过程使"公平"("Fairness")这个概念成为天生的本能。这是货真价实的普世价值,是写在基因上的,所以百姓一旦得到温饱,就会开始在乎贪腐,甚至是公车私用这样的小事。

2008年的金融危机对美国经济造成几万亿美元的损害,不但没有事后算帐,而且始作俑者个个加官进爵,大富大贵,这一方面消灭了社会的公平假象,进一步威胁政权的合法性,但是更重要是连这样公开而且万眾瞩目的事都如此不公,那么真正有关实质利益的施政步骤(往往是一般人很难注意到的法律和行政细节)偏袒财阀的程度就可想而知了。

金融界必须为实体经济服务,所以不该有大利润也不能有大损失,越简单基本越好。要达到这个理想,我只知道国有制有可能。当然反过来说,国有制并不保证金融机构运行合理有效;内行而有力的监管是不可或缺的。\section*{【美国】【战略】与伊朗的核子谈判}
\subsection*{2015-06-08 00:00}
I myself and many other Taiwanese readers here disagree with your assessment that Taiwan will come to its senses. In fact, we think the situation will only get worse. True, few Taiwanese Greens are willing to die for their cause, but they are behaving like 5 years old, crying and fussing without any rational considerations. This is the ultimate dilemma faced by us, the rational people who nonetheless care about Taiwan.\section*{【经济】再谈TPP}
\subsection*{2015-06-07 00:00}
Once in a while, when the persons in charge are not bought by the big money, John Oliver and his viewers can possibly deliver results. But ultimately, this is a healing channel within the system, and the modern American political system has been so corrupted that it is immune to fundamental fixes from the inside. So John Oliver is doomed from the beginning.\section*{【工业】科技发展与美式自由无关}
\subsection*{2015-05-22 00:00}
I have to give you a warning here. Another case of attack or refusal to acknowledge logic conclusion will be automatic delete. 

Making unsubstantiated attack on me or any other commentators without logical or factual support is not going to be allowed. Debate is good, but insults are not.

If you don't like the rule, go live and wallow in your fantasy land. This is my blog, and the rule is: You don't have to be right, but at least have to be open to logical falsification of your opinion. After a long debate, simply coming back and restating your already falsified thesis is a big no-no.

我决定把这篇违规的留言保留做为反例,以后类似的恶性留言会被删除。

这篇留言是恶性的,体现在两方面:1)"Feeding Opium"是纯人身攻击,没有任何论证的支持; 2)"The economic miracle cannot be sustainable..."是前面已讨论过的话题,留言者的论点已被驳斥,不应该忽视这些论证而重复已被证偽的口号。

总之错误的理论可以提出,但是必须能虚心接受逻辑的证偽。\subsection*{2015-05-20 00:00}
I have never set foot to mainland China, not even for transit.

But you are right, most Americans have been brainwashed to believe that Chinese have no freedom to criticize their government. The truth is that, not only do they air criticism all the time, their complaints get acted on, unlike in the US.\subsection*{2015-05-20 00:00}
0) True, you cannot do street demonstrations as freely in China as in the US, but so what? The priority of the people is better lives, not the right to be beaten up or occasionally killed by the white police on the street.

1) Any large scale economic activities have to be organized and enforced with dictatorial power. Under Roosevelt, this was done by a competent Federal bureaucracy, just like Japan from 1950-1980, Taiwan from 1970-1990 and China today. When the bureaucracy is gutted (by "Free Market" fundamentalists in the US, by internal decay in Japan and by 李登辉 in Taiwan), the economy starts its decay also. Furthermore, as I have pointed out several times, when the government does not wield its dictatorial powers, the big companies will. At least the Chinese government cares about the people; big companies by definition have the mentality of a psychopath.

2) Do the best and brightest still want to go to the US? Five years ago, I had exactly that conversation with a Silicon Valley boss originally from Canada, and our consensus was "not as much as before, and still dropping".

3) In the world over, it is the privileged families that control the businesses. The Economists did a study a couple of months ago and found that China has a MUCH HIGHER percentage of rags-to-riches company owners than the US. But even if this percentage is very low like in Korea, it does not mean the economy necessarily sucks. Really good scientists and engineers derive much satisfactions from their work, not from the prospect of becoming multi-billionaires. Besides, which Silicon Valley boss is a technician any way? They are just sales people who enrich themselves on their employees' work. How is this fundamentally different from the horror you describe of China?\subsection*{2015-05-19 00:00}
First of all, I don't bad-mouth; I just tell the truth.

Second, nobody is saying life in the US is no good. Quite the contrary, that is exactly the point: the people in the US have their good lives paid for by the other 95 \%  of the world. Just the fact that they forced their currency on the world, and then took away the promise of not printing money from thin air, is simply and indisputably the largest theft in human history. 

60 \%  of the CO2 in earth atmosphere comes from 5 \%  of its population. Do they bother to pay for it? In fact, they are pointing their fingers at China, which still has only 1/3 of emissions per capita .

I have been writing dozens of articles explaining that the US is strong not because of its "systems" but because of luck and shameless killing/robbing/stealing. If you want to dispute any of those, just lay out your arguments, but please don't pretend that they do not exist and just go on a rant using only baseless statements to the contrary.\section*{【工业】2025年的中国工业}
\subsection*{2015-05-18 00:00}
Contrary to American propaganda, American society has plenty of censorship. Please read my prior article 《言论自由的假相》. Also contrary to American propaganda, even the false sense of American freedom has very little to do with innovation. For example, prior to WWII, Germany was the authoritarian regime but also far more innovative, at least on a per-capita basis. For another example, in the 1980s, the Japanese were beating the Americans on multiple fronts of high-tech researches. Remember, Japan is as conformist as they come.

The reason is simple: the great majority of the population does not do innovation (Otherwise the US would have been in real trouble since its population is particularly stupid among industrialized nations); it is the top 0.01 \%  in IQ who is responsible for driving 99.99 \%  of the progress. It does not matter how much freedom those stupid majority has; only the smart few can really put their freedom to good use anyway. Their freedom is more often defined by being able to buy the right equipment at the right time. This is the effect of wealth. Once China can put enough investments on enough smart people, there is no reason why it cannot outdo the US in the innovation game.

Don't buy the American propaganda without thinking it through first. Use your own brain and follow the logic.\section*{【政治】民主体制下的救世主情结}
\subsection*{2015-05-16 00:00}
我了解这是很多大陆青年的想法,但是无助于局势。台独心理基本上就是自卑感驱动的闹气,以"嗔"字诀回应,只会彼此火上加油。我以前已说过,台独像是一个不学好的妹妹,全家受气之后,终究还是心痛。该骂的就骂,但是气话就让不懂事的小妹妹一个人讲吧。

日本社会的反中和美国人的鼓动一样,只是日本右翼政治精英重建旧帝国光荣时可用的资源,实际上安倍很明白不能和中共真打。美日两国彼此信誓旦旦,要结盟对付中国,其实是互相诓骗、怂恿对方先去送死。反正日本重新军事化,不费美国一毛钱。这样的互相利用,又彼此都心知肚明,其实也就是一场闹剧,大家看看笑话就好了,没什么好气的。中国一但统一之后,日本要是还不改,自然会受教训。\section*{【空军】【海军】共军小道消息刷新(2015年第二季)}
\subsection*{2015-05-13 00:00}
我看过刀口有关FADEC的原文,非常不能同意(现在我对他和老马已经很失望了)。FADEC这东西并不是像他说的那样需要源代码的,就像你买微软的视窗作业系统,然后在上面写应用程式,是不需要视窗的源代码的,只需要把界面(API,Application Program Interface)定义好就行了。

波音这样安全第一的民航公司,用罗罗(Rolls Royce)的发动机时,也是只有API而没有Source Codes的。

俄国人偏爱土星是因为它有基础科研的底子,礼炮原本只是生产商,后来拿了中方的钱,才雇(土星的)人往上游发展。不论如何,99M2还不够稳定,太行的14吨型更只是原型机,现在中俄加起来唯一稳定、可以批量部署的14吨发动机就只有117S。或许明年会不一样,但是这种事,共军一向有谨慎的传统。\section*{【政治】【经济】民主政治与自由经济}
\subsection*{2015-05-13 00:00}
有人在《帝国大反击》也问了这问题。首先这不是封杀,而是暂缓;参议院有45个参议员反对现在对这个法案表决,并没有否决了这个法案。

这基本上是工会控制的民主党议员对财团控制的共和党给下马威,要收买路钱。TPP里面有很多前所未见的条款,赋予财团新的权力,例如各国政府不能自行规范财团,必须由TPP成员国(最终也就是美国)一致同意才行。工会说你甜头那么多,我们也要分一杯羹。

财团现在很急,因为到年底大家就会完全投入准备明年的大选,2016年基本上什么事都不能办(每四年至少有一年空转,这样的制度还好意思大吹大擂?),所以必须把TPA很快搞定,才能留下几个月来做像是与日本谈判和把TPP条约推过参议院+眾议院。\section*{【工業】【能源】高溫氣冷堆}
\subsection*{2015-05-12 00:00}
我对核潜艇反应堆没有专长,所以一直没有写专文。我所知道的是潜艇里寸土寸金,用小堆+AIP不切实际(高温气冷堆尤其如此,因为功率密度太低),大概只是军迷的狂想。如果是用小堆代替AIP,比较有可能。有谣言说共军在开发袖珍型的核潜艇,准备以量取胜;不过中共对核潜艇的保密级别特别高,我们可能要到十年后才知道真相。

共军的新一代大功率核潜艇反应堆是2012年开始研发的(有一个官八股说某某元件从2012年开始,后来可信的分析发现那个元件是核潜艇反应堆环路的一部分),应该在2020年左右服役。

观察网上周有一篇讨论AIP功率增加117 \% 的文章,我想是目前最好的分析。不过我现在找不到了,说不定是因为分析得太好,被拿下来了。它的主旨是共军的AIP技术来自瑞典,而瑞典的AIP有两个功率(75KW和110KW);依前者(日本的苍龙级用了四台)来算,增加117 \% 后是160KW,四台是640KW,这比世界其他先进柴油潜艇例如德国的214级(240KW)高得多了,将来会有新的战术可能。不过这比起核潜艇(洛杉矶级是26MW)还差得远。

日本的右翼人士其实从被轰了之后就想着要钸,但是是1982年中曾根康弘上台才正式偷偷摸摸地干。日本的钸不是压水/沸水堆来的(压水/沸水堆是不能生产钸239的),而是中曾根康弘任内盖的一个快堆。1990年代还因处理MOX(就是钸和铀的混合燃料)错误(居然由手工拿扫把在水桶里搅拌!连口罩都没有带)而有两个工人死亡。

\subsection*{2015-05-09 00:00}
1)中子的来源不是问题。快堆和聚变反应器相比,要便宜、安全、成熟得多。
2)重氢的原子核有多余的中子,提供了额外的吸引力来克服正电之间的排斥力。若是使用普通氢,等离子体的温度要再高两个数量级,磁场就关不住它了。参见《永远的未来技术》。
3)中子的生产,除了裂变和聚变之外,另一个办法是用高速质子去打原子核,有部分会把中子打出来;这也是唯一能产生定向中子束的方法。中子产生时,都是快中子;要慢中子,可以用减速剂和中子做弹性碰撞,吸收它的动能。
4)这絶对是高温气冷堆长期下去的最佳远景,不过压水堆也可以做成袖珍型的和它竞争。中共的型号叫ACP-100,参见《核动力在军用与民用之间的差别》。\subsection*{2015-05-08 00:00}
欢迎你来我的部落格。

我认为氢气太危险了,不能推广到消费者级别,请参见前文《永远的未来技术》。至于在低用电期间生產氢气,直接储存在电厂,在尖峰时间再以燃料电池补助发电,我觉得有可行性,但是技术还不成熟。

中共似乎没有出口的限制,已经准备把高温气冷堆推销给中东。高温气冷堆用水比压水/沸水式少,950度的高温也适合海水淡化,特别适合中东和北非国家。

台湾没有真正的荒地,核废料是个大问题。至于和中方合作,我自己的观察是台湾选民的智商和五岁小孩差不多,遇到不顺意的事只会哭叫打闹;权衡折中是要有十几岁的智商的,台湾近25年还没有做过;至于对外交涉,求同存异,获取双赢的共利,这是完全成熟的成年人才做得到的,台湾除了喊喊口号之外,有人真能做到吗?\subsection*{2015-05-07 00:00}
If everyone in Taiwan is guilty, I would agree to let them just wallow in their own stupidity, but there are people who are not into politics and barely scraping an existence. And they are suffering the most. That is why we have to do something. I am doing my part; it is not much, but if everyone with a conscience does it, maybe together we will make a difference.\subsection*{2015-05-07 00:00}
是的。这事比较复杂一点,所以我没在正文里提。

快堆全名是快中子堆,也就是没有中子减速剂的设计。因为水是中子减速剂之一,所以就不能用;一般快中子堆必须用液态金属来当冷却剂,例如钠或铅。

快中子容易被铀238吸收,成为钸239,这是快中子堆的主用途。但是Trans-Uranium Elements(超铀,即比铀还重的元素,都有高度的放射性,极不稳定)也会吸收快中子而裂变成小原子核。这个好处是这样的:超铀的半衰期一般在几个月到几千年之间,这刚好是最让人头疼的范围;因为半衰期若是很短,放一放就衰退光了,半衰期若是很长,则衰变很慢,放射性就很低,例如天然铀。所以用过的燃料棒里最难处理的元素就是超铀,而理论上快堆可以把它嬗变掉。

当然,在工程上处理高放射性的废料是非常困难的,所以目前快堆基本上还是为了军事用途而存在。\section*{【政治】中共在欧洲的最好朋友}
\subsection*{2015-05-06 00:00}
张教授对历史的分析有独到之处,但是对未来的预测我不赞同。

在英国正式放弃全球霸权的梦想(1956年苏伊士危机)后近60年来,伦敦的金融中心地位带给英国极大的利益,到现在金融业已是英国经济的支柱(没有之一)。但是它正面临法兰克福和巴黎这种欧元区的金融城市的挑战,如果英国退出欧盟,伦敦的金融地位将会遭受严重的打击。英国正在放弃当全球二流强权的梦想,军费在过去五年中被砍了将近1/3;再度崛起在军事和金融上都是不可能的,只要伦敦不被欧洲大陆城市取代就算好的了。

财政部部长楼继伟在清华给的演讲才是最近最重要的文章,不可不细读。其中他提到放松粮食自给自足的要求,加强与阿根廷的农业合作;我觉得这是很有自信的正确方向。在国内,应该同时加强科学教育和政府监管,以求审慎合理地推广转基因作物。
\section*{【战略】【经济】絶地大反攻}
\subsection*{2015-04-21 00:00}
我对CIPS不熟,不过印象中它是一个Back Office的货币结算通道,和SWIFT这种Front Office用的汇兑通讯系统是不一样。

CIPS在美元和欧元似乎没有应对的类似机构,它的存在来自人民币受管制却又必须内外流通的这个矛盾。之前,人民银行已经和一些国家签了SWAP,但是若要让人民币进IMF的SDR,它必须能在所有国家都进行结算才行。所以CIPS是为了SDR而办的。

AIIB对世银是一个严重打击,对IMF则是杀鸡儆猴,现在IMF自己急着要绕过美国,以避免把中国逼上梁山。在年底之前,人民币应该就能进SDR,所以人民银行必须配合它,赶快把CIPS办成。\section*{【工业】访意大利有感(三)}
\subsection*{2015-04-12 00:00}
当今中共对内对外气势正盛,除了经济发展为民造福之外,很大一部分是因为讲王道。王道不只是不害人,也包括不害真理正义。过去十几年,中共在这软实力方面最大的软肋是官员贪腐,所以习近平的反贪是一个极重要的改革。但是现在其他的软肋还在,其中之一是环境污染,所以美国领事馆把PM2.5拿出来说事,就使中共只能被动挨打。另一个则是当年邓小平为维持共產党的正统歷史地位,决定对毛泽东的罪孽加以遮掩,30多年下来,越来越骑虎难下。所谓的言论自由与政治正确之争,根本就是顾左右而言他的非重点所在:言论是否正确,首要标准应该是问它是否真实。毕福剑所说的,完全就是1979年和1980年全共產党都在公开讨论的事实,现在却成了禁忌,这的确是大倒退。再这样掩耳盗铃下去,文革对中共就会像亚美尼亚种族灭绝对土耳其一样,永远是个痛脚。\section*{【台湾】星宿派和F-18C}
\subsection*{2015-04-05 00:00}
我写这个博客,就是希望能在当代愚昧混乱的媒体背景下,起一点拨乱反正的作用。欢迎你们给我回响。

64事件时,我还在哈佛,有大陆同学回去串联。我个人觉得那个事件是文革的最后遗毒,是过度理想化的无知学生任性发挥自身激情,置国家秩序与全民利益于危地的不负责任行为。我在《政府的第一要务》曾讨论过危害全民利益和经济发展就是大规模杀人,这群学生都是大规模杀人的未遂犯(其实1989和1990年,中国的经济发展还是受了很大的打击,所以他们的大规模杀人是部分成功的),因此我对他们后来被镇压的同情是非常有限的。

当然,这样的胡搞和毛泽东比起来都不算什么。邓小平为了未来发展的稳定基础,否决了1979年当时很多干部的建议,而没有诚实检讨毛泽东的过错,是一个时代的不幸,也是64的远因。但是近40年后,连习近平对此也无解,这仍将是中国未来的一个祸根。

李登辉在台湾搞了小文革之后,愚民有了任性危害全民利益的自由和动力,一直到马英九上台,还小心维护推广,看来这个小文革会再延续下去了;这也就是当代台湾的最大不幸。\subsection*{2015-04-04 00:00}
上个月旅游加生病,写的少些。不过我若没有独到的见解,也懒得在这里重复大家都有的观点。你想看每天更新的报导,应该去专业的新闻机构,像是《观察者》。

我在半年前就点出AIIB的重要性,请参读《美元的金融霸权》。四个月前就预言英国会是欧洲对中共最友善的国家。这些大趋势,只要用心,都可以预先看出,只有时间点比较难确定。台湾政坛和媒体,平均智商太低,连基本的逻辑能力都没有,和一群被宠坏的5岁小孩一样(人脑一般在6岁才长出用于逻辑自律的组织)。这样的社会,那有资格搞普选式民主?你看他们的电视节目,纯粹是浪费时间。

马英九政府要加入AIIB,被称为所谓"黑箱作业",根本就是欲加之罪,完全是白痴胡扯蛋的东西。行政权和立法权本来就相互独立,哪能横向透明?总统府又不是Reality Show,难道一天24小时还全程电视直播?总统上厕所,能不能"黑箱作业"?马英九不讲大是大非,不坚持原则,一味讨好愚民,最后里外不是人,其实是咎由自取。\section*{【美国】科学界的卖淫者}
\subsection*{2015-03-07 00:00}
如果你能读英文,看看这篇:http://www.npr.org/blogs/thetwo-way/2014/02/14/277058739/1-in-4-americans-think-the-sun-goes-around-the-earth-survey-says

不相信地球绕着太阳转的美国人四年前是18 \% ,去年是26 \% ,今年是24 \% 。我以前已经提过,80 \% 的美国人不知道DNA是什么。24 \% 不知道独立戦争是跟谁打的。

其实只要在美国上过公立学校(或有小孩上过公立学校)都知道为什么美国的基层教育如此糟糕:一般学生的目的不在读书,而在追求Cool和Popular,所以男孩健身或组摇滚乐团,女孩则学化妆和Cheerleading。美式足球和学年舞会(Prom)才是学校里的大事。\section*{【台湾】【空军】乡愿,洋奴和冤大头}
\subsection*{2015-03-05 00:00}
1.闹独立的不是原住民;原住民也不占人口多数;就算他们要独立,建的也是"中央山脉国"。其实原住民对当今台湾社会理应相当不满,搞独立是应该的,也是好事。
2.为文统,但是台湾有很多白痴拼命想逼成武统。文统这个目的比美日利用台湾让中国人自相残杀的用心好多了。中共又不是慈济,只有白痴才会指望政略的交往上有无私的行为。
3.日本没有侵略台湾?就算你一辈子没读过真的歷史,我的部落格已经在《美国的东亜战略史》里提过日本如何多次"直接"出兵台湾了。此外,台湾当时是清朝的一部分,所以自然也是交战方的一部分。

即使是同文同种,也不是不能搞独立,早些例如冰岛,近些例如Eritrea。我对台独最大的反感在于他们极不诚实,一天到晚只会编假歷史和歪逻辑。明明在法理上都是中国的一部分,要独立只能由母国许可。如果不许可,就得打赢一场独立战争。可是台独拿不到许可,又不敢打仗,最后只敢欺负事实和逻辑。我不管你们的统独理念,可是我对事实和逻辑是很在乎的。\section*{【台湾】无知与短视的后果}
\subsection*{2015-02-15 00:00}
Theories of Chinese collapses have been promulgated more than 10,000 times in the past 30 years. None has come true, so at least empirically, your chance of being correct is 0/10000=zero.

The Great Internet Wall is a response to US propaganda machine. Read my previous articles about it.

Individual merits have nothing to do with a nation's rise; only the collective strength of the organization matters. Have you any ideas what the German people were like before the unification of 1870's? They were lazy drunks! Even today, the worst tourists are not the Chinese but the Americans; The Chinese are only the second worst. Wealth changes cultures; not the other way around. Same with civility and law. You simply got it backward.

KMT cannot even impose the most basic internal discipline. In my book, that is called a losing organization.
 \end{document}